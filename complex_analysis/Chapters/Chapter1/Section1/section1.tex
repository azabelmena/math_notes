\section{The Field of Complex Numbers and the Complex Plane}

\begin{definition}
    We define the set of \textbf{complex numbers} to be the collection of all
    ordered pairs of real numbers $\C=\{(a,b) : a,b \in \R\}$ together with the
    binary operations $+$ and  $\cdot$ of  \textbf{complex addition}, and
    \textbf{complex multiplication}, respectively defined by the rules
    \begin{align*}
        (a,b)+(c,d) &=  (a+c,b+d)   \\
        (a,b)(c,d)  &=  (ac-bd,bc+ad)   \\
    \end{align*}
\end{definition}

\begin{theorem}\label{1.1.1}
    The set of complex numbers $\C$ forms a field together with complex addition
    and complex multiplication.
\end{theorem}
\begin{corollary}
    $\C$ is a field extension of the real numbers $\R$.
\end{corollary}
\begin{proof}
    The map $a \xrightarrow{} (a,0)$ from $\R \xrightarrow{} \C$ defines an
    imbedding of $\R$ into  $\C$.
\end{proof}

\begin{definition}
    We define the element $i=(0,1)$ of $\C$ so that  $i^2=-1$, and the
    polynomial $z^2+1$ has as root $i$. We write $(a,b)=a+ib$. If $z=a+ib$, we
    call  $a$ the  \textbf{real part} of $z$, and  $b$ the  \textbf{imaginary
    part} of $z$ and write  $\re{z}=a$ and $\im{z}=z$.
\end{definition}

\begin{definition}
    Let $z=a+ib \in \C$. We define the \textbf{norm} (or \textbf{modulus}) of
    $z$ to be  $\|z\|=\sqrt[]{a^2+b^2}$. We define the complex
    \textbf{conjugate} of $z$ to be  $\bar{z}=a-ib$.
\end{definition}

\begin{lemma}\label{1.1.2}
    For every $z \in \C$,  $\|z\|^2=z\bar{z}$.
\end{lemma}
\begin{proof}
    Let $z=a+ib$. Then $\bar{z}=a-ib$, and so
    $z\bar{z}=(a+ib)(a-ib)=a^2+b^2=(\sqrt[]{a^2+b^2})^2=\|z\|^2$.
\end{proof}
\begin{corollary}
    If $z \neq 0$, then $\inv{z}=\frac{1}{z}=\frac{\bar{z}}{\|z\|^2}$.
\end{corollary}
\begin{proof}
    The relation follows from above, and it remains to show that it is indeed
    the inverse element. Notice that if $z \in \C$ is nonzero, then
    $z\frac{\bar{z}}{\|z\|^2}=\frac{z\bar{z}}{\|z\|^2}=\frac{\|z\|^2}{\|z\|^2}=1$.
\end{proof}

\begin{lemma}\label{1.1.3}
    The following are true for all $z,w \in \C$.
    \begin{enumerate}
        \item[(1)] $\re{z}=\frac{1}{2}(z+\bar{z})$ and
            $\im{z}=\frac{1}{2i}(z-\bar{z})$.

        \item[(2)] $\bar{(z+w)}=\bar{z}+\bar{w}$ and $\bar{zw}=\bar{z} \
            \bar{w}$.

        \item[(3)] $\|\bar{z}\|=\|z\|$.
    \end{enumerate}
\end{lemma}
\begin{proof}
    Let $z=a+ib$ and  $w=c+id$. Then notice that
    \begin{equation*}
        \frac{(a+ib)+(a-ib)}{2}=\frac{2a+(ib-ib)}{2}=\frac{2a}{2}=a=\re{z}
    \end{equation*}
    and
    \begin{equation*}
        \frac{(a+ib)-(a-ib)}{2i}=\frac{(a-a)+2ib}{2}=\frac{2ib}{2i}=b=\im{z}
    \end{equation*}

    Moreover
    \begin{equation*}
        \bar{(a+ib)+(c+id)}=\bar{(a+c)+i(b+d)}=(a+c)-i(b+d)=(a-ib)+(c-id)
    \end{equation*}
    And
    \begin{equation*}
        \bar{(a+ib)(c+id)}=\bar{(ac-bd)+i(bc+ad)}=(ac-bd)-i(bc+ad)=(a-ib)(c-id)
    \end{equation*}
    so that $\bar{z+w}=\bar{z}+\bar{w}$ and $\bar{zw}=\bar{z} \ \bar{w}$.

    Now, we have that $\|zw\|^2=(zw)\bar{zw}=(zw)(\bar{z} \
    \bar{w})=(z\bar{z})(w\bar{w})=\|z\|^2\|w\|^2$. Taking square roots, we get
    the result
    \begin{equation*}
        \|zw\|=\|z\|\|w\|
    \end{equation*}
    Finally, notice that $\|z\|^2=z\bar{z}=\bar{z} \ \bar{\bar{z}}=\|\bar{z}\|$.
\end{proof}
\begin{corollary}
    The following are also true; provided $w \neq 0$.
    \begin{enumerate}
        \item[(1)] $\bar{(\frac{z}{w})}=\frac{\bar{z}}{\bar{w}}$.

        \item[(2)] $\|\frac{z}{w}\|=\frac{\|z\|}{\|w\|}$
    \end{enumerate}
\end{corollary}
