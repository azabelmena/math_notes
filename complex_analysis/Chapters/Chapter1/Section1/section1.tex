\section{The Field of Complex Numbers}

\begin{definition}
    We define the set of \textbf{complex numbers} to be the collection of all
    ordered pairs of real numbers $\C=\{(a,b) : a,b \in \R\}$ together with the
    binary operations $+$ and  $\cdot$ of  \textbf{complex addition}, and
    \textbf{complex multiplication}, respectively defined by the rules
    \begin{align*}
        (a,b)+(c,d) &=  (a+c,b+d)   \\
        (a,b)(c,d)  &=  (ac-bd,bc+ad)   \\
    \end{align*}
\end{definition}

\begin{theorem}\label{1.1.1}
    The set of complex numbers $\C$ forms a field together with complex addition
    and complex multiplication.
\end{theorem}
\begin{corollary}
    $\C$ is a field extension of the real numbers $\R$.
\end{corollary}
\begin{proof}
    The map $a \xrightarrow{} (a,0)$ from $\R \xrightarrow{} \C$ defines an
    imbedding of $\R$ into  $\C$.
\end{proof}

\begin{definition}
    We define the element $i=(0,1)$ of $\C$ so that  $i^2=-1$, and the
    polynomial $z^2+1$ has as root $i$. We write $(a,b)=a+ib$. If $z=a+ib$, we
    call  $a$ the  \textbf{real part} of $z$, and  $b$ the  \textbf{imaginary
    part} of $z$ and write  $\re{z}=a$ and $\im{z}=z$.
\end{definition}

\begin{definition}
    Let $z=a+ib \in \C$. We define the \textbf{norm} (or \textbf{modulus}) of
    $z$ to be  $\|z\|=\sqrt[]{a^2+b^2}$. We define the complex
    \textbf{conjugate} of $z$ to be  $\bar{z}=a-ib$.
\end{definition}

\begin{lemma}\label{1.1.2}
    For every $z \in \C$,  $\|z\|^2=z\bar{z}$.
\end{lemma}
\begin{proof}
    Let $z=a+ib$. Then $\bar{z}=a-ib$, and so
    $z\bar{z}=(a+ib)(a-ib)=a^2+b^2=(\sqrt[]{a^2+b^2})^2=\|z\|^2$.
\end{proof}
\begin{corollary}
    If $z \neq 0$, then $\inv{z}=\frac{1}{z}=\frac{\bar{z}}{\|z\|^2}$.
\end{corollary}
\begin{proof}
    The relation follows from above, and it remains to show that it is indeed
    the inverse element. Notice that if $z \in \C$ is nonzero, then
    $z\frac{\bar{z}}{\|z\|^2}=\frac{z\bar{z}}{\|z\|^2}=\frac{\|z\|^2}{\|z\|^2}=1$.
\end{proof}

\begin{example}\label{example_1.1}
    \begin{enumerate}
        \item[(1)] Let $z=a+ib$. Then we get that
            $\frac{1}{z}=\frac{\bar{z}}{\|z\|}$ has real part
            $\re{\frac{1}{z}}=\frac{a}{a^2+b^2}$ and imaginary part
            $\im{\frac{1}{z}}=-\frac{b}{a^2+b^2}$.

        \item[(2)] Let $z=a+ib$, and  $c \in \R$. Then
            $\frac{z-c}{z+c}=\re{\frac{z-c}{z+c}}$, so $\im{\frac{z-c}{z+c}}=0$.

        \item[(3)] Let $z=a+ib$, then $z^3=a^3-3ab^2+i(3a^2b-b^3)$ So that
            $\re{z^3}=a^3-3ab^2$ and $\im{z}=3a^2b-b^3$.

        \item[(4)] $\frac{3+i5}{1+i7}=\frac{19}{25}-i\frac{18}{25}$.

        \item[(5)] $(-\frac{1}{2}+i\frac{\sqrt[]{3}}{2})^3=1$, and hence
            $(-\frac{1}{2}+i\frac{\sqrt[]{3}}{2})^6=1$.

        \item[(6)] Notice that $i^n=1,i,-1,-i$ whenever  $n \equiv 0 \mod{4}$,
            $n \equiv 1 \mod{4}$, $n \equiv 2 \mod{4}$, and $n \equiv 3
            \mod{4}$.
            respectively.

        \item[(7)] $\|-2+i\|=\sqrt[]{5}$, and
            $\|(2+i)(4+i3)\|=\|5+i10\|=5\sqrt[]{5}$.
    \end{enumerate}
\end{example}

\begin{lemma}\label{1.1.3}
    The following are true for all $z,w \in \C$.
    \begin{enumerate}
        \item[(1)] $\re{z}=\frac{1}{2}(z+\bar{z})$ and
            $\im{z}=\frac{1}{2i}(z-\bar{z})$.

        \item[(2)] $\bar{(z+w)}=\bar{z}+\bar{w}$ and $\bar{zw}=\bar{z} \
            \bar{w}$.

        \item[(3)] $\|\bar{z}\|=\|z\|$.
    \end{enumerate}
\end{lemma}
\begin{proof}
    Let $z=a+ib$ and  $w=c+id$. Then notice that
    \begin{equation*}
        \frac{(a+ib)+(a-ib)}{2}=\frac{2a+(ib-ib)}{2}=\frac{2a}{2}=a=\re{z}
    \end{equation*}
    and
    \begin{equation*}
        \frac{(a+ib)-(a-ib)}{2i}=\frac{(a-a)+2ib}{2}=\frac{2ib}{2i}=b=\im{z}
    \end{equation*}

    Moreover
    \begin{equation*}
        \bar{(a+ib)+(c+id)}=\bar{(a+c)+i(b+d)}=(a+c)-i(b+d)=(a-ib)+(c-id)
    \end{equation*}
    And
    \begin{equation*}
        \bar{(a+ib)(c+id)}=\bar{(ac-bd)+i(bc+ad)}=(ac-bd)-i(bc+ad)=(a-ib)(c-id)
    \end{equation*}
    so that $\bar{z+w}=\bar{z}+\bar{w}$ and $\bar{zw}=\bar{z} \ \bar{w}$.

    Now, we have that $\|zw\|^2=(zw)\bar{zw}=(zw)(\bar{z} \
    \bar{w})=(z\bar{z})(w\bar{w})=\|z\|^2\|w\|^2$. Taking square roots, we get
    the result
    \begin{equation*}
        \|zw\|=\|z\|\|w\|
    \end{equation*}
    Finally, notice that $\|z\|^2=z\bar{z}=\bar{z} \ \bar{\bar{z}}=\|\bar{z}\|$.
\end{proof}
\begin{corollary}
    The following are also true; provided $w \neq 0$.
    \begin{enumerate}
        \item[(1)] $\bar{(\frac{z}{w})}=\frac{\bar{z}}{\bar{w}}$.

        \item[(2)] $\|\frac{z}{w}\|=\frac{\|z\|}{\|w\|}$
    \end{enumerate}
\end{corollary}
\begin{corollary}
    If $z=z_1+\dots+z_n$, and $w=w_1 \dots w_n$, with $z_i,w_i \in \C$ for all
    $1 \leq i \leq n$, then
    \begin{enumerate}
        \item[(1)] $\bar{z}=\bar{z_1}+\dots+\bar{z_n}$.

        \item[(2)] $\|w\|=\|w_1\| \dots \|w_n\|$.
    \end{enumerate}
\end{corollary}
\begin{proof}
    We prove both results by induction on $n$. For  $n=2$, we have already shown
    that  $\bar{z}=\bar{z_1}+\bar{z_2}$ and $\|w\|=\|w_1\|\|w_2\|$. Now, for
    all $n \geq 2$, suppose that both
    \begin{align*}
        \bar{z} &=  \bar{z_1}+\dots+\bar{z_n}   \\
        \|w\|   &=  \|w_1\| \dots \|w_n\|   \\
    \end{align*}
    Then let $z'=z+z_{n+1}$ and $w'=ww_{n+1}$ for $z_{n+1},w_{n+1} \in \C$. Then
    we have that
    \begin{align*}
        z'  &= z+z_{n+1}    = z_1+\dots+z_n+z_{n+1}   \\
        w'  &=  ww_{n+1}    =   w_1 \dots w_n w_{n+1}   \\
    \end{align*}
    so by the induction hypothesis, we have
    \begin{equation*}
        \bar{z'}=\bar{(z+z_{n+1})}=\bar{z}+\bar{z_{n+1}}=\bar{z_1}+\dots+\bar{z_n}+
        \bar{z_{n+1}}
    \end{equation*}
    and that
    \begin{equation*}
        \|w'\|=\|ww_{n+1}\|=\|w\|\|w_{n+1}\|=\|w_1\| \dots \|w_n\| \|w_{n+1}\|
    \end{equation*}
    which completes the proof.
\end{proof}

\begin{lemma}\label{1.1.4}
    Let $z \in \C$. Then $z$ is a real number if, and only if  $z=\bar{z}$.
\end{lemma}
\begin{proof}
    If $z$ is real, then  $z=a+i0$, for some $a \in \R$, and hence
    $\bar{z}=a-i0=z$. COnversely, suppose that $z=\bar{z}$. Then we have
    \begin{equation*}
        \re{z}=\frac{1}{2}(z+\bar{z})=\frac{1}{2}(z+z)=z
    \end{equation*}
    so $z$ has only a real part, and hence must be a real number.
\end{proof}

\begin{lemma}\label{1.1.5}
    The following are true for all $z,w \in \C$.
    \begin{enumerate}
        \item[(1)] $\|z+w\|^2=\|z\|^2+2\re{z\bar{w}}+\|w\|^2$.

        \item[(2)] $\|z-w\|^2=\|z\|^2-2\re{z\bar{w}}+\|w\|^2$.

        \item[(3)] $\|z+w\|^2+\|z-w\|^2=2(\|z\|^2+\|w\|^2)$.
    \end{enumerate}
\end{lemma}
\begin{proof}
    We first notice that
    $\|z+w\|^2=(z+w)\bar{(z+w)}=(z+w)(\bar{z}+\bar{w})=z\bar{z}+z\bar{w}+w\bar{z}+
    w\bar{w}=\|z\|^2+z\bar{w}+w\bar{z}+\|w\|^2$. Now, let $z=a+ib$ and
    $w=c+id$. Then we have
    \begin{align*}
        (a+ib)(c-id)    &=  (ac+bd)-i(ad-bc)    \\
        (c+id)(a-ib)    &=  (ac+bd)+i(ad-bc)    \\
    \end{align*}
    so that $z\bar{w}+w\bar{z}=2(ac+bd)=2\re{z\bar{w}}$, and we are done. To get
    the identity for $\|z-w\|^2$, we simply replace $w$ by  $-w$, and use the
    above argument.

    Now, we have that  $\|z+w\|^2=\|z^2\|+2\re{z\bar{w}}+\|w\|^2$, and
    $\|z-w\|^2=\|z^2\|-2\re{z\bar{w}}+\|w\|^2$, so that adding them together,
    the terms $2\re{z\bar{w}}$ cancel out and we are left with
    \begin{equation*}
        \|z+w\|^2+\|z-w\|^2=2(\|z\|^2+\|w\|^2)
    \end{equation*}
\end{proof}

\begin{lemma}\label{1.1.6}
    Let $R(z) \in \C(z)$ a rational function in $z$. Then if  $R$ has
    coefficients in $\R$, then $\bar{R(z)}=R(\bar{z})$.
\end{lemma}
\begin{proof}
    We first observe the polynomial $f \in \C[z]$, of finite degree $\deg{f}=n$,
    and of the form
    \begin{equation*}
        f(z)=a_0+a_1z+\dots+a_nz^n
    \end{equation*}
    Then if $f$ has all coefficients in $\R$; i.e. $f \in \R[z]$, where $z \in
    \C$ is treated as indeterminant, then we have that since each  $a_i \in \R$,
    then  $\bar{a_iz^i}=\bar{a_i}\bar{z^i}=a_i\bar{z}^i$. So that
    \begin{equation*}
        \bar{f(z)}=\bar{(a_0+a_1z+\dots+a_nz^n)}=a_0+a_1\bar{z}+\dots+a_n\bar{z}^n
    \end{equation*}
    which makes $\bar{f(z)}=f(\bar{z})$. Now, one can also extend $f$ to a
    polynomial of infinite degree by taking $n \xrightarrow{} \infty$, and the
    same holds.

    Now, let $R(z) \in \C(z)$ a rational function. Recall that $R(z)$ is of the
    form
    \begin{equation*}
        R(z)=\frac{f(z)}{g(z)} \text{ with } g \neq 0
    \end{equation*}
    for some polynomials $f,g \in \C[z]$. Then if $R$ has all real
    coefficients, so do  $f$ and  $g$, and hence we get
    \begin{equation*}
        \bar{R(z)}=\frac{\bar{f(z)}}{\bar{g(z)}}=\frac{f(\bar{z})}{g(\bar{z})}=R(\bar{z})
    \end{equation*}
    which completes the proof.
\end{proof}
