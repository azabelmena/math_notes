\section{Formal Power Series}

\begin{definition}
    Let $F$ be a field, we define the set  $F[[x]]$ of all series of the form
    \begin{equation*}
        f(x)=\sum_{n=0}^\infty{a_0x^n}, \text{ where } a_0, \dots, a_n ,\dots \in F
    \end{equation*}
    the set of \textbf{formal power series} over $F$. We call the elements of
    $F[[x]]$ \textbf{formal power series}.
\end{definition}

\begin{definition}
    Let $f(x)=\sum_{n=0}^\infty{a_0x^n}$ a formal power series over a field $F$.
    We define the  \textbf{order} of $f$ to be the smallest integer  $n$ for
    which $a_n \neq 0$, and write  $\ord{f}=n$. We call the term $a_0$ of $f$
    the  \textbf{constant term} of $f$.
\end{definition}

\begin{lemma}\label{2.1.1}
    Let $F$ be a field, and define the operations $+$ and $\cdot$ on  $F$ by
    \begin{align*}
        f(x)+g(x) &= \sum_{n=0}^\infty{c_nx^n} \text{ where }c_n=a_n+b_n   \\
        f(x)g(x) &= \sum_{n=0}^\infty{d_nx^n}
                    \text{ where }c_n=\sum_{k=0}^n{a_kb_{n-k}}   \\
    \end{align*}
    Where $f(x)=\sum_{n=0}^\infty{a_nx^n}$ and $g(x)=\sum_{n=0}^\infty{b_nx^n}$
    are formal power series over $F$. Then  $F[[x]]$ forms a commutative ring
    under $+$ and  $\cdot$.
\end{lemma}
\begin{corollary}
    Define the action $F \times  F[[x]] \xrightarrow{} F[[x]]$ by
    \begin{equation*}
        \a f(x)=\sum_{n=0}^\infty{(\a a_n)x^n}
    \end{equation*}
    Then $F[[x]]$ is an $F$-module under this action.
\end{corollary}

\begin{lemma}\label{2.1.2}
    Let $f(x)=\sum_{n=0}^\infty{a_nx^n}$ and $g(x)=\sum_{n=0}^\infty{b_nx^n}$ be
    formal power series over a field $F$. Then  $\ord{fg}=\ord{f}+\ord{g}$.
\end{lemma}

\begin{definition}
    Let $f \in F[[x]]$ be a formal power series over a field  $F$. We say that a
    formal power series  $g \in F[[x]]$ is an \textbf{inverse} of $f$ if
    $fg=1$.
\end{definition}

\begin{lemma}\label{2.1.3}
    If $f(x)=\sum_{n=0}^\infty{a_nx^n}$ is a formal power series over a field
    $F$, with nonzero constant term, then there exists an inverse of $f$.
\end{lemma}
\begin{proof}
    Consider the series $\inv{a_0}f(x)$ instead of $f$. Reacall also that the
    geometric series
    \begin{equation*}
        \frac{1}{1-r}=1+r+r^2+\dots
    \end{equation*}
    is a formal power series in $r$ over $F$. Then $(1-r)(1+r+r^2+\dots)=1$.
    Now, let $f(x)=1-h(x)$, where $h(x)=-(a_1x+a_2x^2+\dots)$ and consider
    $\phi(h)=1+h+h^2+\dots$. Observbe that $\ord{h^n} \geq n$ sicne
    $h^n=(-1)a_1^nx^n+\dots$. Thus, if $m>n$, then  $h^m$ has all coefficients
    of order less than  $n$ equal to  $0$, and the $n$-th coefficient of $\phi$
    is the $n$-th coefficient of the sum
    \begin{equation*}
        1+h+h^2+\dots+h^n
    \end{equation*}
    Then, we get by the above geometric series that
    \begin{equation*}
        (1-h(x))\phi(h)=(1-h(x))(1+h+h(x)^2+\dots)=1+\dots=1
    \end{equation*}
\end{proof}

\begin{example}\label{example_2.1}
    Let $\cos{x}=1-\frac{x^2}{2!}+\frac{x^4}{4!}-\dots$. By lemma \ref{2.1.3},
    since $\cos{x}$ has nonzero constant term, it has an invers
    $g(x)=\frac{1}{\cos{x}}$. Notice that
    \begin{align*}
        \frac{1}{1-\frac{x^2}{2!}+\frac{x^4}{4!}-\dots} &= 1+(\frac{x^2}{2!}-
        \frac{x^4}{4!}+\dots)+(\frac{x^2}{2!}-\frac{x^4}{4!}+\dots)^2+\dots \\
            &= 1+\frac{x^2}{2!}-\frac{x^4}{4!}+\dots+\frac{x^4}{(2!)^2} \\
            &= 1+\frac{x^2}{2!}+(-\frac{1}{24}+\frac{1}{4})x^2+\dots
    \end{align*}
    Which gives coefficients of $g(x)$ up to order $4$.
\end{example}

\begin{definition}
    Let $f(x)=\sum_{n=0}^\infty{a_nx^n}$ a power series over a field $F$, and
    let $h(x)=c_1x+\dots$ a power series of order greater than $1$. We define
    the  \textbf{substitute} of $h$ in  $f$ to be the power series
    \begin{equation*}
        f \circ h(x)=a_0+a_1h(x)+a_2h(x)^2+\dots
    \end{equation*}
\end{definition}

\begin{definition}
    Let $f(x)=\sum_{n=0}^\infty{a_nx^n}$ and $g(x)=\sum_{n=0}^\infty{b_nx^n}$ be
    power series over a field $F$. We call  $f$ \textbf{congruent} to $g$
    \textbf{modulo} $x^n$ if  $a_k=b_k$ for all  $k \in \faktor{\Z}{n\Z}$. That
    is, $f$ and $g$ have the same coeficients of terms of order up to $n-1$. We
    write $f \equiv g \mod{x^2}$.
\end{definition}

\begin{lemma}\label{2.1.4}
    Congruence of power series modulo $x^n$ defines an equivalence relation.
\end{lemma}

\begin{lemma}\label{2.1.5}
    If $f_1 \equiv f_2 \mod{x^n}$ and $g_1 \equiv g_2 \mod{x^n}$, then $f_1+g_1
    \equiv f_2g_2 \mod{x^n}$ and $f_1g_1 \equiv f_2g_2 \mod{x^n}$. Moreover, if
    $h_1$ and $h_2$ are formal power series with zero constant term, and $h_1
    \equiv h_2 \mod{x^n}$, then $f_1 \circ h_1 \equiv f_1 \circ h_2 \mod{x^n}$.
\end{lemma}
\begin{proof}
    We prove for substitutions of $h_1$ in $f_1$ only. Let $p_1$ and $p_2$
    polynomials of degree $\deg=n-1$ such thta  $f_1 \equiv p_1(x) \mod{x^n}$
    and $f_2 \equiv p_2(x) \mod{x^n}$. By hypothesis, we get $p_1 \equiv p_2
    \mod{x^n}$, and since $\deg{p_1},\deg{p_2}=n-1$, this makes $p_1=p_2$.
    Then $f_1 \circ h \equiv p_1 \circ h=p_2 \circ h \equiv f_2 \circ h$. Now,
    let $q(x)$ the polynomial of degree $n-1$  such that $h_1 \equiv h_2 \equiv
    q(x) \mod{x^n}$Writing $p(x)=a_0+a_1x+\dots+a_{n-1}x^{n-1}$. Then we get
    $p_1 \circ h_1 \equiv p_2 \circ h_2 \mod{x^n}$ and we are done.
\end{proof}
\begin{corollary}
    Two power series $f$ and $g$ are equal if, and only if $f \equiv g
    \mod{x^n}$ for all $n \in \Z^+$.
\end{corollary}
\begin{corollary}
    $(f_1+f_2) \circ h=(f_1 \circ h)+(f_2 \circ h)$, and $(f_1f_2) \circ h=(f_1
    \circ h)(f_2 \circ h)$. That is, composition of power series distributes
    over the addition and multiplication of power series.
\end{corollary}
\begin{corollary}
    Provided that $\ord{f_2}=0$, then
    \begin{equation*}
        \Big{(}\frac{f_1}{f_2}\Big{)} \circ h=\frac{f_1 \circ h}{f_2 \circ h}
    \end{equation*}
\end{corollary}

\begin{example}\label{example_2.2}
    COnsidre the power series for $\frac{1}{\sin{x}}$. We have by definition
    that
    \begin{equation*}
        \sin{x}=x-\frac{x^3}{3!}+\frac{x^5}{5!}-\dots
        =x(1-\frac{x^2}{3!}+\frac{x^4}{5!})
    \end{equation*}
    so that
    \begin{equation*}
        \frac{1}{\sin{x}}=\frac{1}{x(1-\frac{x^2}{3!}+\frac{x^4}{5!})}=
        \frac{1}{x}(1+\frac{x^2}{3!}-\frac{x^4}{5!}+
        \Big{(}\frac{x^2}{3!}\Big{)}^2+\dots)=
        \frac{1}{x}+\frac{x}{3!}+\Big{(}\frac{1}{(3!)^2}-\frac{1}{5!}\Big{)}x^3+\dots
    \end{equation*}
\end{example}
