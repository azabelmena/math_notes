\section{Metric Spaces}

\begin{definition}
    A \textbf{metric space} is a set $X$ together with a map  $d:X \times X
    \xrightarrow{} \R$ such that for all $x,y,z \in X$
    \begin{enumerate}
        \item[(1)] $d(x,y) \geq 0$ and $d(x,y)=0$ if, and only if $x=y$.

        \item[(2)] $d(x,y)=d(y,x)$.

        \item[(3)] $d(x,y) \leq d(x,z)+d(z,y)$ (The Triangle Inequality).
    \end{enumerate}
    We call $d$ a  \textbf{metric} on $X$. If  $x \in X$, and  $r>0$, we define
    the  \textbf{open ball} centered about $x$ of  \textbf{radius} $r$ to be the
    set  $B(x,r)=\{y \in X : d(x,y)<r\}$. We define the \textbf{closed ball}
    centered about $x$ of radius  $r$ to be the set  $\bar{B}(x,y)=\{y \in X :
    d(x,y) \leq r\}$.
\end{definition}

\begin{example}\label{example_2.1}
    \begin{enumerate}
        \item[(1)] The metric $d(x,y)=\|z-w\|$ defined by
            $\|z-w\|=\sqrt{(x_1-x_2)^2+(y_1-y_2)}$, where $z=x_1+iy_1$ and
            $w=x_2+iy_2$ makes $\R$ and  $\C$ into metric spaces. In fact,
            $d$ defines the norm on  $\C$, i.e. $\|z\|=d(z,0)$. In $\R$,
            $\| \cdot \|$ is the absolute value. We denote howerver
            $\| \cdot \|=| \cdot |$ is $\C$ as well, when we are talking about
            the norm of a complex number.

        \item[(2)] If $X$ is a metric space with metric  $d$, and  $Y \subseteq
            X$, then  $d$ makes  $Y$ into a metric space.

        \item[(3)] Define $d(x+iy,a+ib)=|x-a|+|y-b|$. Then $(\C,d)$ is a
            metric space. We call $d$ the  \textbf{taxicab metric}.

        \item[(4)] Define $d(x+iy,a+ib)=\max{\{|x-a|,|y-b|\}}$. Then
            $(\C,d)$ is a metric space. We call $d$ the  \textbf{square metric}.

        \item[(5)] Let $X$ be any set, and define  $d:X \times X \xrightarrow{}
            \R$ by
            \begin{equation*}
                d(x,y)=\begin{cases}
                    1, \text{ if } x \neq y \\
                    0, \text{ if } x=y
                \end{cases}
            \end{equation*}
        Then $d$ is a metric on  $X$. Notice also that for any  $\e>0$, that
        $B(x,\e)=\{x\}$ if $\e \leq 1$, and  $B(x,\e)=X$ if $\e>1$.

    \item[(6)] Define $d$ on  $\R^n$ by
        \begin{equation*}
            d(x,y)=\sqrt{\sum_{i=1}^n{(x_i-y_i)^2}}
        \end{equation*}
        Then $d$ is a metric on $\R^n$ defining the general norm. That is
        $\|x\|=d(x,0)$.

    \item[(7)] Let $S$ and let  $B(S)$ the set of all complex valued functions
        $f:S \xrightarrow{} \C$ such that $\|f\|_\infty=\sup{\{|f(s|) : s \in
        S\}}$ is finite. That is, $B(S)$ is the set of all complex valued
        functions whose image is contained within a disk of finite radius.
        Define $d$ on  $B(S)$ by $d(f,g)=\|f-g\|_\infty$. Let $f,g,h \in B(S)$.
        Then
        \begin{equation*}
            \|f(s)-g(s)\|=\|(f(s)-h(s))-(h(s)-g(s))\| \leq
            \|f(s)-h(s)\|+\|h(s)-g(s)\|
        \end{equation*}
        taking least upper bounds, we get
        \begin{equation*}
            \|f-g\|_\infty \leq \|f-h\|_\infty+\|h-g\|_\infty
        \end{equation*}
    \end{enumerate}
\end{example}

\begin{definition}
     Let $X$ be a metric space together with metric $d$. We call a subset $U$ of
     $X$  \textbf{open} if there exists an $\e>0$ for which  $B(x,\e) \subseteq
     U$ for every $x \in U$.
\end{definition}

\begin{example}\label{example_2.2}
    $U=\{z \in \C : a< \re{z}<b\}$ is open in $\C$, but  $U=\{z \in \C :
    \re{z}<0\}$ is not, as $B(0,\e) \notin U$ no matter how small we make $\e$.
\end{example}

\begin{theorem}\label{2.1.1}
    Let $X$ be a metric space with metric  $d$. Then $X$ is a topological space
    whose open sets are those subsets of  $X$ containing $\e$-balls for every
    element, and for $\e>0$.
\end{theorem}

\begin{definition}
    We call a subset $V$ of a metrix space $(X,d)$ \textbf{closed} if
    $\com{X}{V}$ is open in $X$.
\end{definition}

\begin{lemma}\label{2.1.2}
    If $(X,d)$ is a metric space, then it is a topology by closed sets.
\end{lemma}

\begin{definition}
    Let $A \subseteq X$ where $X$ is a metric space. We define the
    \textbf{interior} of $A$ to be the union of all open sets contained in $A$,
    and write $\Int{A}$. We define the \textbf{closure} of $A$ to be the
    intersection of all closed sets containing  $A$ and write  $\cl{A}$. We
    define the \textbf{boundry} of $A$ to be  $\partial{A}=\cl{A} \cap
    \cl{(\com{X}{A})}$.
\end{definition}

\begin{example}\label{example_2.3}
    We have $\Int{\Q(i)}=\emptyset$ and $\cl{\Q(i)}=\C$.
\end{example}

\begin{lemma}\label{2.1.3}
    Let $X$ be a metric space and  $A,B \subeteq X$. Then the following are true
    \begin{enumerate}
        \item[(1)] $A$ is open if, and only if  $A=\Int{A}$.

        \item[(2)] $A$ is closed if, and only if  $A=\cl{A}$.

        \item[(3)] $\Int{A}=\com{X}{\cl{(\com{X}{A})}}$,
            $\cl{A}=\com{X}{\Int{(\com{X}{A})}}$, and
            $\partial{A}=\com{\cl{A}}{\Int{A}}$.

        \item[(4)] $\cl{(A \cup B)}=\cl{A} \cup \cl{B}$.

        \item[(5)] $x_0 \in \Int{A}$ if, and only if there is an $\e>0$ for
            which  $B(x_0,\e) \subseteq A$.

        \item[(6)] $x_0 \in \cl{A}$ if, and only if for every $\e>0$, $B(x_0,\e)
            \cap A=\emptyset$.
    \end{enumerate}
\end{lemma}

\begin{definition}
    A subset $A$ of a metric space $X$ is \textbf{dense} in $X$ if  $\cl{A}=X$.
\end{definition}

\begin{example}\label{example_2.4}
    $\Q$ is dense in  $\R$, notice that  $\cl{\Q}=\R$. Moreover, $\Q(i)$ is
    dense in $\C$.
\end{example}
