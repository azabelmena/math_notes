\section{Convergent Power Series}

\begin{definition}
    Let $\{z_n\}_{n \in \Z^+}$ a sequence of complex numbers, and consider the
    series $\sum_{n=0}^\infty{z_n}$. We define the \textbf{$n$-th partial sum}
    to be
    \begin{equation*}
        s_n=\sum_{k=1}^n{z_k}
    \end{equation*}
    and we say that the series \textbf{converges} if there exists a $w \in \C$
    for which  $\lim{\{s_n\}}=w$ as $n \xrightarrow{} \infty$. We call $w$ the
    \textbf{sum} of the series.
\end{definition}

\begin{lemma}\label{lemma_2.2.1}
    Let $A=\sum{\a_n}$ and $B=\sum{\b_n}$ be convergent series with $n$-th
    partial sums  $s_n$ and  $t_n$. Then the sum and product of  $A$ and  $B$
    converge, with
    \begin{equation*}
        A+B=\sum{(\a_n+\b_n)} \text{ and } AB=\lim_{n \xrightarrow{} \infty}{\{s_nt_n\}}
    \end{equation*}
\end{lemma}

\begin{definition}
    Let $\sum{\a_n}$ a series of complex numbers. We say that $\sum{\a_n}$
    \textbf{converges absolutely} if the series of real numbers $\sum{|\a_n|}$
    converges.
\end{definition}

\begin{lemma}\label{2.2.2}
    If $\sum{\a_n}$ is a series of complex numbers which converges absolutely,
    then it converges.
\end{lemma}
\begin{proof}
    Let $s_n=\sum_{k=1}^n{\a_k}$, then for $m \leq n$, notice that
    $s_n-s_m=\a+{m_1}+\dots+\a_n$, hence $|s_n-s_m| \leq
    \sum_{k=m+1}^n{|\a_k|}$. By absolute convergence, let $\e>0$ then there
    exists an  $N>0$ such that  $\sum{|\a_k|}<\e$ whenever $m,n \geq N$. Thus
    $|s_n-s_m|<\e$ which makes  $\sum{\a_n}$ converge.
\end{proof}

\begin{lemma}\label{2.2.3}
    Let $\sum{c_n}$ be a convergent series of real numbers greater than $0$. If
    $\{\a_n\}$ is a sequence of complex numbers such that $|\a_n|<c_n$ for all
    $n \in \Z^+$, then  $\sum{\a_n}$ converges absolutely.
\end{lemma}
\begin{proof}
    Notice that the partial sums $\sum_{k=1}^n{c_n}$ are bounded, hence
    $\sum{|\a_n|} \leq \sum{c_k}$.
\end{proof}

\begin{lemma}\label{2.2.4}
    Let $\{\a_n\}$ a sequence of complex numbers. Then the following are true
    \begin{enumerate}
        \item[(1)] If $\sum{\a_n}$ is absolutely convergent, then the series
            obtained by permuting terms is absolutely convergent, with the same
            limit.

        \item[(2)] If $\sum_{n=1}^\infty{(\sum_{m=1}^n{\a_{mn}})}$ is absolutely
            convergent, then so is the series
            $\sum_{m=1}^n{(\sum_{n=1}^\infty{\a_{mn}})}$, and they converge to
            the same limit.
    \end{enumerate}
\end{lemma}

\begin{definition}
    Let $S \subseteq \C$, and let $f$ be a bounded complex valued function on
    $S$. We define the \textbf{sup norm} of $f$ on  $S$ to be
    \begin{equation*}
        \|f\|_S=\sup_{z \in S}{\{|f(z)|\}}
    \end{equation*}
\end{definition}

\begin{lemma}\label{2.2.5}
    Let $S \subseteq \C$. The sup norm of a complex valued function on $S$
    defines a metric on $\C$.
\end{lemma}

\begin{definition}
    Let $\{f_n\}_{n \in \Z^+}$ a sequence of complex valued functions on a set
    $S \subseteq \C$. We say that the  $\{f_n\}$ \textbf{converges uniformly} on
    $S$ if there exists a bounded complex valued function $f$ on $S$ such that
    for all $\e>0$, there is an $N>0$ for which
    \begin{equation*}
        \|f_n-f\|_S<\e \text{ whenever } n \geq N
    \end{equation*}
    We call $\{f_n\}$ \textbf{Cauchy} if for every $\e>0$ there is an  $N>0$ for
    which
    \begin{equation*}
        \|f_n-f_m\|_S<\e \text{ whenever } n,m \geq N
    \end{equation*}
\end{definition}

\begin{theorem}\label{2.2.6}
    Let $\{f_n\}$ be a sequence of complex valued functions on a set $S
    \subseteq \C$. If  $\{f_n\}$ is Cauchy, then it converges uniformly.
\end{theorem}
\begin{proof}
    We have for all $z \in S$, take  $f(z)=\lim{f_n(z)}$ as $n \xrightarrow{}
    \infty$. Then for $\e>0$ there is an  $N>0$ for which  $|f_n(z)-f_m(z)|<\e$
    for al $z \in S$ and  $m,n \geq N$. Now, for  $n \geq N$, take $m(n) \geq N$
    large enough so that $|f(z)-f_{m(n)}(z)|<\e$. Then we get that
    \begin{equation*}
        |f(z)-f_n(z)| \leq
        |f(z)-f_{m(n)}(z)|+|f_{m(n)}(z)-f_n(z)|<\e+\|f_{m(n)}-f_n\|<2\e
    \end{equation*}
\end{proof}
\begin{corollary}
    If $\{f_n\}$ is bounded for all $n \in \Z^+$, then so is  $f$.
\end{corollary}

\begin{definition}
    We say a series of complex valued functions on a domain $S \subseteq \C$,
    $\sum{f_n}$ \textbf{converges uniformly} if the sequence $\{s_n\}$ of $n$-th
    partial sums converges uniformly. We say that  $\sum{f_n}$ \textbf{converges
    absolutely} if for all $z \in S$,  $\sum{|f_n(z)|}$ converges.
\end{definition}

\begin{theorem}[The Comparison Test]\label{2.2.7}
    Let $\{c_n\}$ be a sequence of real numbers greater than $0$ such that
    $\sum{c_n}$ converges. Let $\{f_n\}$ a sequence of complex valued functions
    on a domain $S \subseteq \C$ such that  $\|f_n\|_S \leq c_n$ for all  $n \in
    \Z^+$. Then the series $\sum{f_n}$ converges uniformly, and converges
    absolutely.
\end{theorem}
\begin{proof}
    Let $m \leq n$. Then  $\|s_n-s_m\| \leq \sum_{k=m+1}^n{\|f_k\|_S} \leq
    \sum{c_k}$. Since $\sum{c_k}$ converges, the uniform and absolute convergnce
    of $\sum{f_n}$ follows.
\end{proof}

\begin{theorem}\label{2.2.8}
    Let $S \subseteq \C$ and  $\{f_n\}$ a sequence of continuous complex valued
    functions on $S$. If  $\{f_n\}$ converges unifromly to a complex valued
    function $f$ on  $S$, then  $f$ is also continuous.
\end{theorem}
\begin{proof}
    let $\a \in S$ and  $n$ be large enough such that  $\|f-f_n\|_S<\e$ for some
     $\e>0$. By the continuity of  $f_n$ at  $\a$, choose  $\d>0$ such that
     $|f_n(z)-f_n(\a)|<\e$ whenever $|z-\a|<\d$. Then observe that
     \begin{equation*}
         |f(z)-f(\a)| \leq
         |f(z)-f_n(z)|+|f_n(z)-f_n(\a)|+|f_n(\a)-f(\a)|<2\|f-f_n\|+\e<3\e
     \end{equation*}
\end{proof}

\begin{theorem}\label{2.2.9}
    Let $\{a_n\}$ a sequence of complex numbers, and let $r>0$ such that
    $\sum{|a_n|r^n}$ converges. Then the power series $\sum{a_nz^n}$ converges
    absolutely and converges uniformly whenever $|z| \leq r$.
\end{theorem}

\begin{example}\label{example_2.3}
    \begin{enumerate}
        \item[(1)] Let $r>0$ and consider the series
            \begin{equation*}
                \exp{z}=\sum_{n=0}^\infty{\frac{z^n}{n!}}
            \end{equation*}
        Then $\exp{z}$ converges absolutely and uniformly whenever $|z| \leq
        r$. Indeed, let $c_n=\frac{r^n}{n!}$, then
        \begin{equation*}
            \frac{c_{n+1}}{c_n}=\frac{r}{n+1}
        \end{equation*}
        Taking $n \geq 2r$, notice that  $\frac{c_{n+1}}{c_n} \leq \frac{1}{n}$
        so that $c_{n+1} \leq \frac{1}{2}c_n$ for $n$ large enough
    \end{enumerate}
\end{example}
