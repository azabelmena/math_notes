\section{Convergent Power Series}

For the remainder of this chapter, we consider only formal power series if $z$
over  $\C$; i.e. all power series in  $\C[[z]]$.

\begin{definition}
    Let $\{z_n\}_{n \in \Z^+}$ a sequence of complex numbers, and consider the
    series $\sum_{n=0}^\infty{z_n}$. We define the \textbf{$n$-th partial sum}
    to be
    \begin{equation*}
        s_n=\sum_{k=1}^n{z_k}
    \end{equation*}
    and we say that the series \textbf{converges} if there exists a $w \in \C$
    for which  $\lim{\{s_n\}}=w$ as $n \xrightarrow{} \infty$. We call $w$ the
    \textbf{sum} of the series.
\end{definition}

\begin{lemma}\label{lemma_2.2.1}
    Let $A=\sum{\a_n}$ and $B=\sum{\b_n}$ be convergent series with $n$-th
    partial sums  $s_n$ and  $t_n$. Then the sum and product of  $A$ and  $B$
    converge, with
    \begin{equation*}
        A+B=\sum{(\a_n+\b_n)} \text{ and } AB=\lim_{n \xrightarrow{} \infty}{\{s_nt_n\}}
    \end{equation*}
\end{lemma}

\begin{definition}
    Let $\sum{\a_n}$ a series of complex numbers. We say that $\sum{\a_n}$
    \textbf{converges absolutely} if the series of real numbers $\sum{|\a_n|}$
    converges.
\end{definition}

\begin{lemma}\label{2.2.2}
    If $\sum{\a_n}$ is a series of complex numbers which converges absolutely,
    then it converges.
\end{lemma}
\begin{proof}
    Let $s_n=\sum_{k=1}^n{\a_k}$, then for $m \leq n$, notice that
    $s_n-s_m=\a+{m_1}+\dots+\a_n$, hence $|s_n-s_m| \leq
    \sum_{k=m+1}^n{|\a_k|}$. By absolute convergence, let $\e>0$ then there
    exists an  $N>0$ such that  $\sum{|\a_k|}<\e$ whenever $m,n \geq N$. Thus
    $|s_n-s_m|<\e$ which makes  $\sum{\a_n}$ converge.
\end{proof}

\begin{lemma}\label{2.2.3}
    Let $\sum{c_n}$ be a convergent series of real numbers greater than $0$. If
    $\{\a_n\}$ is a sequence of complex numbers such that $|\a_n|<c_n$ for all
    $n \in \Z^+$, then  $\sum{\a_n}$ converges absolutely.
\end{lemma}
\begin{proof}
    Notice that the partial sums $\sum_{k=1}^n{c_n}$ are bounded, hence
    $\sum{|\a_n|} \leq \sum{c_k}$.
\end{proof}

\begin{lemma}\label{2.2.4}
    Let $\{\a_n\}$ a sequence of complex numbers. Then the following are true
    \begin{enumerate}
        \item[(1)] If $\sum{\a_n}$ is absolutely convergent, then the series
            obtained by permuting terms is absolutely convergent, with the same
            limit.

        \item[(2)] If $\sum_{n=1}^\infty{(\sum_{m=1}^n{\a_{mn}})}$ is absolutely
            convergent, then so is the series
            $\sum_{m=1}^n{(\sum_{n=1}^\infty{\a_{mn}})}$, and they converge to
            the same limit.
    \end{enumerate}
\end{lemma}

\begin{definition}
    Let $S \subseteq \C$, and let $f$ be a bounded complex valued function on
    $S$. We define the \textbf{sup norm} of $f$ on  $S$ to be
    \begin{equation*}
        \|f\|_S=\sup_{z \in S}{\{|f(z)|\}}
    \end{equation*}
\end{definition}

\begin{lemma}\label{2.2.5}
    Let $S \subseteq \C$. The sup norm of a complex valued function on $S$
    defines a metric on $\C$.
\end{lemma}

\begin{definition}
    Let $\{f_n\}_{n \in \Z^+}$ a sequence of complex valued functions on a set
    $S \subseteq \C$. We say that the  $\{f_n\}$ \textbf{converges uniformly} on
    $S$ if there exists a bounded complex valued function $f$ on $S$ such that
    for all $\e>0$, there is an $N>0$ for which
    \begin{equation*}
        \|f_n-f\|_S<\e \text{ whenever } n \geq N
    \end{equation*}
    We call $\{f_n\}$ \textbf{Cauchy} if for every $\e>0$ there is an  $N>0$ for
    which
    \begin{equation*}
        \|f_n-f_m\|_S<\e \text{ whenever } n,m \geq N
    \end{equation*}
\end{definition}

\begin{theorem}\label{2.2.6}
    Let $\{f_n\}$ be a sequence of complex valued functions on a set $S
    \subseteq \C$. If  $\{f_n\}$ is Cauchy, then it converges uniformly.
\end{theorem}
\begin{proof}
    We have for all $z \in S$, take  $f(z)=\lim{f_n(z)}$ as $n \xrightarrow{}
    \infty$. Then for $\e>0$ there is an  $N>0$ for which  $|f_n(z)-f_m(z)|<\e$
    for al $z \in S$ and  $m,n \geq N$. Now, for  $n \geq N$, take $m(n) \geq N$
    large enough so that $|f(z)-f_{m(n)}(z)|<\e$. Then we get that
    \begin{equation*}
        |f(z)-f_n(z)| \leq
        |f(z)-f_{m(n)}(z)|+|f_{m(n)}(z)-f_n(z)|<\e+\|f_{m(n)}-f_n\|<2\e
    \end{equation*}
\end{proof}
\begin{corollary}
    If $\{f_n\}$ is bounded for all $n \in \Z^+$, then so is  $f$.
\end{corollary}

\begin{definition}
    We say a series of complex valued functions on a domain $S \subseteq \C$,
    $\sum{f_n}$ \textbf{converges uniformly} if the sequence $\{s_n\}$ of $n$-th
    partial sums converges uniformly. We say that  $\sum{f_n}$ \textbf{converges
    absolutely} if for all $z \in S$,  $\sum{|f_n(z)|}$ converges.
\end{definition}

\begin{theorem}[The Comparison Test]\label{2.2.7}
    Let $\{c_n\}$ be a sequence of real numbers greater than $0$ such that
    $\sum{c_n}$ converges. Let $\{f_n\}$ a sequence of complex valued functions
    on a domain $S \subseteq \C$ such that  $\|f_n\|_S \leq c_n$ for all  $n \in
    \Z^+$. Then the series $\sum{f_n}$ converges uniformly, and converges
    absolutely.
\end{theorem}
\begin{proof}
    Let $m \leq n$. Then  $\|s_n-s_m\| \leq \sum_{k=m+1}^n{\|f_k\|_S} \leq
    \sum{c_k}$. Since $\sum{c_k}$ converges, the uniform and absolute convergnce
    of $\sum{f_n}$ follows.
\end{proof}

\begin{theorem}\label{2.2.8}
    Let $S \subseteq \C$ and  $\{f_n\}$ a sequence of continuous complex valued
    functions on $S$. If  $\{f_n\}$ converges unifromly to a complex valued
    function $f$ on  $S$, then  $f$ is also continuous.
\end{theorem}
\begin{proof}
    let $\a \in S$ and  $n$ be large enough such that  $\|f-f_n\|_S<\e$ for some
     $\e>0$. By the continuity of  $f_n$ at  $\a$, choose  $\d>0$ such that
     $|f_n(z)-f_n(\a)|<\e$ whenever $|z-\a|<\d$. Then observe that
     \begin{equation*}
         |f(z)-f(\a)| \leq
         |f(z)-f_n(z)|+|f_n(z)-f_n(\a)|+|f_n(\a)-f(\a)|<2\|f-f_n\|+\e<3\e
     \end{equation*}
\end{proof}

\begin{theorem}\label{2.2.9}
    Let $\{a_n\}$ a sequence of complex numbers, and let $r>0$ such that
    $\sum{|a_n|r^n}$ converges. Then the power series $\sum{a_nz^n}$ converges
    absolutely and converges uniformly whenever $|z| \leq r$.
\end{theorem}

\begin{example}\label{example_2.3}
    \begin{enumerate}
        \item[(1)] Let $r>0$ and consider the series
            \begin{equation*}
                \exp{z}=\sum_{n=0}^\infty{\frac{z^n}{n!}}
            \end{equation*}
        Then $\exp{z}$ converges absolutely and uniformly whenever $|z| \leq
        r$. Indeed, let $c_n=\frac{r^n}{n!}$, then
        \begin{equation*}
            \frac{c_{n+1}}{c_n}=\frac{r}{n+1}
        \end{equation*}
        Taking $n \geq 2r$, notice that  $\frac{c_{n+1}}{c_n} \leq \frac{1}{n}$
        so that $c_{n+1} \leq \frac{1}{2}c_n$ for $n$ large enough. Therefore
        there exists  an $n_0$ such that
        \begin{equation*}
            c_n \leq \frac{C}{2^{n-n_0}} \text{ for some constant } C
        \end{equation*}
        whenever $n \geq n_0$. Comparing this with the geometric series, we get
        absolute and uniform convergence as was required. Moreover, notice that
        the series $\exp{z}$ defines a continuous function on all of $\C$.

    \item[(2)] Take the series $S(z)=\sum{(-1)^n\frac{2^{2n+1}}{(2n+1)!}}$ and
        $C(z)=\sum{(-1)^n\frac{2^{2n}}{(2n)!}}$. Both $S(z)$ and $C(z)$ converge
        absolutely and uniformly for all $|z| \leq r$. Moreover, they define
        continuous functions on al of  $\C$.
    \end{enumerate}
\end{example}

\begin{theorem}\label{2.2.10}
    Let $\sum{a_nz^n}$ a power series. If it does not converge absolutely for
    all $z \in \C$, then there exists a real number $r>0$ such that
    $\summ{a_nz^n}$ converges absolutely whenever $|z| \leq r$.
\end{theorem}
\begin{proof}
    Suppose that $\sum{a_nz^n}$ does not converge absolutely for all $z \in \C$.
    Let  $r=\sup_{s \geq 0}{\{s\}}$ where $\sum{a_ns^n}$ converges. Then notice
    that $\sum{|a_n||z|^n}$ diverges wheneer $|z|>r$ and converges when  $|z|<r$,
    by the comparison test.
\end{proof}

\begin{definition}
    The \textbf{radius of convergence} of a power series $\sum{a_nz^n}$ is a
    number $r>0$ for which the series converges absolutely whenever $|z|<r$, and
    diverges whenever $|z|>r$. $\sum{a_nz^n}$ converges absolutely for all $z
    \in \C$, then we write  $r=\infty$. We call  $\sum{a_nz^n}$ a
    \textbf{convergent power series} if $r \neq 0$, and we say that it
    \textbf{converges} on an open ball $B(0,r)$.
\end{definition}

\begin{theorem}\label{2.2.11}
    Let $\sum{a_nz^n}$ a convergent power series with radius of convergence $r$.
    Then
    \begin{equation*}
        \frac{1}{r}=\limsup{\sqrt[n]{|a_n|}}
    \end{equation*}
    If $r=0$, then the sequence of points $\{\sqrt[n]{a_n}\}$ is not bounded.
\end{theorem}
\begin{proof}
    Let $t=\limsup{\sqrt[n]{|a_n|}}$, suppose first that $t \neq 0$ and that  $t
    \neq \infty$. Given  $\e>0$, there is a finite number of points  $n \in
    \Z^+$ for which $\sqrt[n]{|a_n|} \geq t+\e$. Thus, for all but finitely many
    $n$, we get $|a_n|<(t+\e)^n$, and $\sum{a_nz^n}$ converges if
    $|z|<\frac{1}{t+\e}$. By comparison with the geometric series, we conclude
    that $r \geq \frac{1}{t+\e}$ for all $\e>0$; that is
    \begin{equation*}
        r \geq \frac{1}{t}
    \end{equation*}

    Conversely, given $\e>0$, there exist infinitely many  $n \in \Z^+$ such
    that  $\sqrt[]{|a_n|} \geq t-\e$, and hence $|a_n| \geq (t-\e)^n$. So we get
    that $\sum{a_nz^n}$ does not converge if $r=\frac{1}{t-\e}$, and its radius
    of convergence satisfies $r \leq \frac{1}{t-\e}$ for all $\e>0$. That is
    \begin{equation*}
        r \leq \frac{1}{t}
    \end{equation*}
    and equality is established.
\end{proof}
\begin{corollary}
    If $\lim{\sqrt[n]{|a_n|}}=t$ exists, then $r=\frac{1}{t}$.
\end{corollary}
\begin{corollary}
    If $\sum{a_nz^n}$ has radius of convergence $r>0$, then there exists a
    $C>0$ such that if  $A>\frac{1}{r}$, then $|a_n| \leq CA^n$ for all  $n$.
\end{corollary}

\begin{example}\label{example_2.4}
    \begin{enumerate}
        \item[(1)] The radius of convergence of the series $\sum{n!z^n}$ is
            $r=0$, since $\sqrt[n]{n!}$ is unbounded as $n \xrightarrow{}
            \infty$.

        \item[(2)] The radius of convergence for the series
            $\exp{z}=\sum{\frac{z^n}{n!}}$ is $r=\infty$, as
            $\sqrt[n]{\frac{1}{n!}} \xrightarrow{} 0$ as $n \xrightarrow{}
            \infty$. That is, the series $\exp{z}$ converges on all of $\C$.

        \item[(3)] The radius of convergence of $\sum{\frac{n!}{n^n}}z^n$ is
            $r=e$, where $e$ is Euler's constant. Observe that
            $\lim{\frac{n!}{n^n}}=\frac{1}{e}$.
    \end{enumerate}
\end{example}

\begin{theorem}[The Ratio Test]\label{2.2.12}
    If $\{a_n\}$ is a sequence of positive real numbers, for which
    $\lim{\frac{a_{n+1}}{a_n}}=A$ exists, then $\lim{\sqrt[n]{a_n}}=A$.
\end{theorem}
\begin{proof}
    Suppose that $A>0$, given  $\e>0$, take  $n_0$ such that $A-\e \leq
    \frac{a_{n+1}}{a_n} \leq A+\e$, for all $n \geq n_0$. Without loss of
    generality, suppose that $\e<A$, so that  $A-\e>0$. THen
    \begin{equation*}
        a_n=a_1\prod_{k=1}^{n_0-1}{\frac{a_k+1}{a_k}}\prod_{k=n_0}^{n}{\frac{a_k+1}{a_k}}
    \end{equation*}
    By induction, there exists constants $C_1(\e)$ and $C_2(\e)$ suc that
    \begin{equation*}
        C_1(\e)(A-\e)^{n-n_0} \leq a_n \leq C_2(\e)(A+\e)^{n-n_0}
    \end{equation*}
    Put $C_1'(\e)=C_1(\e)(A-\e)^{-n_0}$ nad $C_2'(\e)=C_2(\e)(A+\e)^{-n_0}$,
    then
    \begin{equation*}
        (A-\e)\sqrt[n]{C_1'(\e)} \leq \sqrt[n]{a_n} \leq (A+\e)\sqrt[n]{C_2'(\e)}
    \end{equation*}
    Then, there exists $N \geq n_0$ such that $\sqrt[n]{C_1'(\e)}=1+\d_1(n)$,
    with $|\d_1(n)| \leq \frac{\e}{A-\e}$ and  $\sqrt[n]{C_2'(\e)}=1+\d_2(n)$
    and $|\d_2(n)| \leq \frac{\e}{A+\e}$ for all  $n \geq N$. Then
    \begin{equation*}
        A-\e+\d_1(n)(A-\e) \leq \sqrt[n]{a_n} \leq A+\e+\d_2(n)(A+\e)
    \end{equation*}
    which shows that
    \begin{equation*}
        |\sqrt[n]{a_n}-A|<2\e
    \end{equation*}
    For the case that $A=0$, it is easy.
\end{proof}

\begin{example}\label{example_2.5}
    Let $a \neq 0$ a complex number. We define the  \textbf{binomial
    coefficient} of $\a$  \textbf{choose} $n$, where $n \in \Z^+$ to be
    \begin{equation*}
        {\a \choose n}=\frac{\a(\a-1)\dots(\a-n+1)}{n!}
    \end{equation*}
    and we define the \textbf{binomial sereies}
    \begin{equation*}
        (1+z)^\a=\sum{{\a \choose n}z^} \text{ where } {\a \choose 0}=1
    \end{equation*}
    By the ratio test, we get that $r=1$ if  $\a$ is not an integer greater than
     $0$.
\end{example}
