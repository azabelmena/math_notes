%----------------------------------------------------------------------------------------
%	SECTION 1.1
%----------------------------------------------------------------------------------------

\section{Mean Value Theorems.}

\begin{definition}
    Let $f:X \rightarrow \R$ be defined on a metric space $X$. We say that  $f$ has a 
    \textbf{local maximum} at a point  $p \in X$, if  there is a $\delta>0$ for which  $f(q) \leq f(p)$ 
    whenever  $d(q,p)<\delta$. Likewise $f$ has a \textbf{local minimum} at a point  $p \in X$, 
    if  there is a $\delta>0$ for which  $f(q) \leq f(p)$ whenever  $d(q,p)<\delta$. We call 
    local maxima and local minumums \textbf{local extrema}.
\end{definition}

\begin{theorem}\label{6.2.1}
    Let $f:[a,b] \rightarrow \R$ be a realvalued function, and suppose that  $f$ has a local 
    extremum at  $x \in (a,b)$. If $f'$ exists, then  $f'(x)=0$.
\end{theorem}
\begin{proof}
    Suppose, without loss of generality that $f$ has a local maximum at  $x$. Choosse  $\delta>0$ 
    such that $a<x-\delta<x<x+\delta<b$. Then if  $x-\delta<t<x$, we have  $|t-x+\delta|<\delta$, so  $f(t) \leq f(x)$, 
    thus  $\frac{f(t)-f(x)}{t-x} \leq 0$. Similarly, for $x<t<x+\delta$, we get $\frac{f(t)-f(x)}{t-x} \geq 0$, 
    hence, as $t \rightarrow x$, we get  $0 \leq f'(0) \leq 0$, thus  $f'(x)=0$.
\end{proof}

\begin{theorem}[The Generalized Mean Value Theorem]\label{6.2.2}
    If $f,g:[a,b] \rightarrow \R$ are continuous on  $[a,b]$, and differentiable on  $(a,b)$, then 
    there is a point  $x \in (a,b)$ such that $(f(b)-f(a))g('x)=(g(b)-g(a))f'(x)$.
\end{theorem}
\begin{proof}
    Let $h(t)=(f(b)-f(a))g('x)-(g(b)-g(a))f'(x)$, for $t \in [a,b]$, then $h$ is continuous on 
    $[a,b]$, and differentiable on  $(a,b)$, moreover, we have  $h(b)=f(b)g(a)-f(a)g(b)=h(b)$. Now 
    if  $h$ is constant, then $h'=0$ for all  $t$ and we are done., Now suppose that  $h(a) \leq h(t)$, 
    and let  $x \in (a,b)$, be a local minimum of $h$, then  $h'(x)=0$, and we are done; the same 
    result follows for local minima of  $h$.
\end{proof}

\begin{corollary}[The Mean Value Theorem]
    LEt $f:[a,b] \rightarrow \R$ be continuous on  $[a,b]$, and differentiable on  $(a,b)$. Then 
    there is an $x \in (a,b)$ such that $f(b)-f(a)=(b-a)f'(x)$.
\end{corollary}
\begin{proof}
    Take $g(t)=t$.
\end{proof}

\begin{theorem}\label{6.2.3}
    Suppose that $f:[a,b] \rightarrow \R$ is differentiable on  $(a,b)$. Then the following hold 
    for all $x \in (a,b)$:
         \begin{enumerate}[label=(\arabic*)]
             \item If $f' \geq 0$, then $f$ is monotonically increasing.

             \item If  $f'=0$, then  $f$ is constant.

             \item If  $f' \leq 0$, then  $f$ is monotonically decreasing.
        \end{enumerate}
\end{theorem}
\begin{proof}
    Let $x_1,x_2 \in (a,b)$, then by the mean value theorem, there is an $x \in (x_1,x_2)$ 
    such that $f(x_2)-f(x_1)=(x_2-x_1)f'(x)$. Then if $f'(x)=0$, we get  $f(x_2)=f(x_1)$, and that 
    $f$ is constant. If $f'(x) \geq 0$, we get  $f(x_2) \geq f(x_1)$, making $f$ monotonically increasing, 
    similarly, if  $f'(x) \leq 0$, we get  $f$ monotonically decreasing.
\end{proof}
