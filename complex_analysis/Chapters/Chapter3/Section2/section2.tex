\section{Analytic Functions}

\begin{definition}
    Let $U$ be an open set in  $\C$, and  $f:U \xrightarrow{} \C$ a complex
    valued function. We call $f$  \textbf{complex differentiable} at a point
    $z_0 \in U$ if
    \begin{equation*}
        f'(z_0)=\lim_{h \xrightarrow{} 0}{\frac{f(z_0+h)-f(z_0)}{h}}=
        \lim_{z \xrightarrow{} z_0}{\frac{f(z)-f(z_0)}{z-z_0}}
    \end{equation*}
    exists, and we call $f'(z_0)$ the \textbf{complex derivative} of $f$ at
    $z_0$. We call $f$  \textbf{complex differentiable} on $U$ if it is complex
    differentiable at every point  $z_0 \in U$, and the value $f'(z_0)$ defines
    a function $f':U \xrightarrow{} \C$ called the \textbf{complex derivative}
    of $f$ on  $U$. If  $f$ is complex differentiable on $U$, and  $f'$ exists
    on $U$ and is continuous, then we call $f$ \textbf{continuously
    differentiable}, and say it is of class $C^1$.
\end{definition}

\begin{definition}
    Let $U$ be an open set of $\C$, and let $f:U \xrightarrow{} \C$ be a
    continuously differentiable complex valued function; i.e. of class $C^1$. We
    define the \textbf{$n$-th derivative} of $f$ recuresively as
    \begin{enumerate}
        \item[(1)] $f^{(0)}(z)=f$ and $f^{(1)}=f'(z)$.

        \item[(2)] $f^{(n+1)}(z)=(f^{(n)})'(z)$, provided $f^{(n)}(z)$ exists.
    \end{enumerate}
    We say that $f$ is \textbf{$n$-th differentiable} if $f^{(n)}$ exists. We
    say that $f$ is \textbf{$q$-smooth} if $f^{(q)}(z)$ exists for some $q \geq
    0$, and  $f^{(q)}$ is continuous, and we call $f$ of class $C^q$. We call
    $f$  \textbf{smooth} if $f^{(n)}(z)$ exists and is continuous for any $n$,
    and we call  $f$ of class  $C^\infty$.
\end{definition}

\begin{lemma}\label{3.2.1}
    If $f:U \xrightarrow{} \C$ is a complex valued function, complex
    differentiable at a point $z_0 \in U$, then $f$ is continuous on  $U$.
\end{lemma}
\begin{proof}
    We have
    \begin{equation*}
        |f(z)-f(z_0)|=\Big{|} \frac{f(z)-f(z_0)}{z-z_0} \Big{|}|z-z_0|
    \end{equation*}
    Taking $z \xrightarrow{} z_0$, we see that $|f(z)-f(z_0)| \xrightarrow{}
    |f'(z_0)| \cdot 0=0$.
\end{proof}
\begin{corollary}
    If $f$ is $n$-th differentiable, for all $n \in \Z^+$, then $f$ is of class
    $C^\infty$.
\end{corollary}

\begin{definition}
    We call a complex valued function $f:U \xrightarrow{} \C$, on an open set
    $U$ of $\C$, \textbf{analytic} on $U$ if it is of class  $C^1$ on  $U$.
\end{definition}

\begin{theorem}\label{3.2.2}
    Let $U$ be open in  $\C$, and let  $f:U \xrightarrow{} \C$ and $g:U
    \xrightarrow{} \C$ analytic on $U$. Then
    \begin{enumerate}
        \item[(1)] $f+g$ is analytic on  $U$ with $(f+g)'(z)=f'(z)+g'(z)$.

        \item[(2)] $fg$ is analytic on  $U$, with
            $(fg)'(z)=f'(z)g(z)+f(z)g'(z)$.

        \item[(3)] For any $\a \in \C$,  $\a f$ is analytic on  $U$ with $(\a
            f)'(z)=\a f'(z)$.
    \end{enumerate}
\end{theorem}
\begin{corollary}
    $\frac{f}{g}$ is analytic on $U$, provided that  $g(z) \neq 0$ for all $z
    \in U$, and
    \begin{equation*}
        \Big{(} \frac{f}{g} \Big{)}'(z)=\frac{f'(z)g(z)-f(z)g'(z)}{g(z)^2}
    \end{equation*}
\end{corollary}

\begin{theorem}[The Chain Rule]\label{3.2.3}
    Let $U$ and  $V$ be open in $\C$. Let $f:U \xrightarrow{} \C$ be analytic on
    $U$ and let  $g:V \xrightarrow{} \C$ be analytic on $V$, with  $f(U)
    \subseteq V$. Then $g \circ f$ is analytic on $U$, with
    \begin{equation*}
        (g \circ f)'(z)=g' \circ f(z)f'(z)
    \end{equation*}
\end{theorem}
\begin{proof}
    Fix $z_0 \in U$, and let $r>0$ such that the open ball $B(z_0,r)$ is
    contained in $U$; i.e. that  $B(z_0,r) \subseteq U$. It suffices to show that
    if $0<|h_n|<r$, where the sequence $\{h_n\} \xrightarrow{} 0$ as $n
    \xrightarrow{} \infty$, then
    \begin{equation*}
        \lim{\frac{g \circ f(z_0+h_n)-g \circ f(z_0)}{h_n}}
    \end{equation*}
    exists and equals $g' \circ f(z_0)f'(z_0)$.

    Suppose first that $f(z_0) \neq f(z_0+h_n)$ for all $n \in \Z^+$. Then we
    have
    \begin{equation*}
        \frac{g \circ f(z_0+h_n)-g \circ f(z_0)}{h_n}=
        \frac{g \circ f(z_0+h_n)-g \circ f(z_0)}{f(z_0+h_n)-f(z_0)} \cdot
        \frac{f(z_0+h_n)-f(z_0)}{h_n}
    \end{equation*}
    Since $\lim{(f(z_0+h)-f(z_0))}=0$, we get
    \begin{equation*}
        \lim{\frac{g \circ f(z_0+h_n)-g \circ f(z_0)}{h_n}}=g' \circ f(z_0)f'(z_0)
    \end{equation*}

    Now, suppose that $f(z_0)=f(z_0+h_n)$ for infinitely many $n$. Write
    $\{h_n\}=\{k_n\} \cup \{l_n\}$, where $f(z_0) \neq f(z_0+k_n)$ and
    $f(z_0)=f(z_0+l_n)$ for all $n$. Since $f$ is analytic, and hence complex
    differentiable
    \begin{equation*}
        f'(z_0)=\lim{\frac{f(z_0+h_n)-f(z_0)}{l_n}}=0
    \end{equation*}
    so that
    \begin{equation*}
        \lim{\frac{g \circ f(z_0+l_n)+g \circ f(z_0)}{l_n}}=0
    \end{equation*}
    Now, by above we also get that
    \begin{equation*}
        \lim{\frac{g \circ f(z_0+k_n)+g \circ f(z_0)}{k_n}}=g' \circ f(z_0)f'(z_0)
    \end{equation*}
    so that $g' \circ f(z_0)f'(z_0)=0$.
\end{proof}

\begin{definition}
    Let $A \subseteq \C$ an arbitrary set of  $\C$. We call a complex valued
    function $f$ \textbf{analytic} on $A$ if it is analytic on some open set of
     $\C$ containing  $A$.
\end{definition}

\begin{theorem}\label{3.2.4}
    Let $f(z)=\sum{a_n(z-z_0)^n}$ a convergent power series with radius of
    convergence $R>0$. Then the following are true.
    \begin{enumerate}
        \item[(1)] The series $\sum{\frac{n!}{(n-k)!}a_n(z-z_0)^{n-1}}$
            converges with radius of convergence $R$.

        \item[(2)] $f$ is smooth on the ball  $B(z_0,R)$ with
            \begin{equation*}
                f^{(n)}(z)=\sum{\frac{n!}{(n-k)!}a_n(z-z_0)^{n-1}}
            \end{equation*}

        \item[(3)] For all $n \geq 0$
            \begin{equation*}
                a_n=\frac{f^{(n)}(z_0)}{n!}
            \end{equation*}
    \end{enumerate}
\end{theorem}
\begin{proof}
    Suppose without loss of generality that $z_0=0$. Then it is sufficient to
    show that $f'$ exists and has the power series  $\sum{na_nz^{n-1}}$. We have
    by definition that
    \begin{equation*}
        \frac{1}{R}=\limsup{\sqrt[n]{|a_n|}}
    \end{equation*}
    Now, by L'\^opital's rule, we have
    \begin{equation*}
        \lim{\frac{\log{n}}{n-1}}=0
    \end{equation*}
    so that $\lim{\sqrt[n-1]{n}}=1$. Let
    \begin{equation*}
        \frac{1}{R'}=\limsup{\sqrt[n-1]{|a_n|}}
    \end{equation*}
    Then $R'$ is the radius of convergence of the series
    \begin{equation*}
        \sum{na_nz^{n-1}}
    \end{equation*}
    Notice that $\sum{a_nz^{n-1}}=\sum{a_{n+1}z^n}$, so that
    \begin{equation*}
        z\sum{a_{n+1}z^n}+a_0=\sum{a_nz^n}
    \end{equation*}
    Now, if $|z|<R$, and  $z \neq 0$, then  $\sum{|a_nz^n|}=\frac{1}{|z|}$.
    Moreover,
    \begin{equation*}
        \sum{|a_nz^n|}+\frac{1}{|z|}a_0<\infty
    \end{equation*}
    which makes $R \leq R'$. Therefore  $R=R'$, and so the series
    $\sum{na_nz^{n-1}}$ converges and has radius of convergence $R$.

    Now, for $|z|<R$, put  $g(z)=\sum{na_nz^{n-1}}$ and
    $s_n(z)=\sum_{k=0}^n{a_kz^n}$ and $R(z)=\sum_{k=n+1}^\infty{a_kz^k}$. Let $w
    \in B$. Let  $w \in B(0,R)$, the open ball of radius $R$ about $0$, and fix
    $r$ such that $|w|<r<R$. Let $\d>0$ such that the closed ball
    $\bar{B}(w,\d)$ of radius $\d$ about  $w$ is contained in  $B(0,R)$; that
    is, $\bar{B}(w,\d) \subseteq B(0,R)$. Let $z \in B(w,\d)$, then we see that
    \begin{equation*}
        \frac{f(z)-f(w)}{z-w}-g(w)=\Big{(} \frac{s_n(z)-s_n(w)}{z-w}-s'(w) \Big{)}
        +s_n'(w)-g(w)+\frac{R_n(z)-R_n(w)}{z-w}
    \end{equation*}
    So that
    \begin{equation*}
        \frac{f(z)-f(w)}{z-w}-g(w)=\frac{1}{z-w}\sum_{k=n+1}^\infty{a_k(z^k-w^k)}
    \end{equation*}
    However, notice that
    \begin{equation*}
        \Big{|} \frac{z^k-w^k}{z-w}
        \Big{|}=|z^{k-1}+z^{k-2}w+\dots+zw^{k-2}+w^{k-1}| \leq kr^{k-1}
    \end{equation*}
    Hence
    \begin{equation*}
    \Big{|} \frac{R_n(z)-R_n(w)}{z-w} \Big{|} \leq
    \sum_{k=n+1}^\infty{|a_k|kr^{k-1}}
    \end{equation*}
    Since $r<R$,  $\sum{|a_k|kr^{k-1}}$ converges so that for any $\e>0$, there
    is an  $N_1 \in \Z^+$ such that
    \begin{equation*}
        \Big{|} \frac{R_n(z)-R_n(w)}{z-w} \Big{|}<\frac{\e}{3}
    \end{equation*}
    whenever $n \geq N_1$ and for all $z \in B(w,\d)$. Also, notice that
    $\lim{s_n'(w)}=g(w)$, and that there exists an $N_2 \in \Z^+$ such that
    \begin{equation*}
        |s_n'(w)-g(w)|<\frac{\e}{3}
    \end{equation*}
    whenever $n \geq N_2$. Now, let $N=\max{\{N_1,N_2\}}$ and choose $\d>0$ such
    that
    \begin{equation*}
        \Big{|} \frac{s_n(z)-s_n(w)}{z-w} \Big{|}<\frac{\e}{3}
    \end{equation*}
    whenever $0<|z-w|<\d$. Then
    \begin{equation*}
        \Big{|} \frac{f(z)-f(w)}{z-w}-g(w) \Big{|}<\e
    \end{equation*}
    whenever $0<|z-w|<\d$. Therefore  $f'(z)=g(z)$.

    Finally, observe that $f(0)=f^{(0)}(0)=a_0$. Using the power series
    \begin{equation*}
        f^{(n)}(z)=\sum{\frac{n!}{(n-k)!}a_nz^{n-k}}
    \end{equation*}
    we find that
    \begin{equation*}
        f^{(n)}(0)=n!a_n
    \end{equation*}
    and we are done.
\end{proof}
\begin{corollary}
    If $f(z)=\sum{a_n(z-z_0)^n}$ is a convergent power series with radius of
    convergence $R>0$, then $f$ is analytice on the open ball $B(z_0,R)$.
\end{corollary}

\begin{example}\label{example_3.3}
    $\exp{z}=\sum{\frac{z^n}{n!}}$ is analytic on all of $\C$.
\end{example}

\begin{lemma}\label{3.2.5}
    If $U$ is open and connected in $\C$, and $f:U \xrightarrow{} \C$ is complex
    differentiable on $U$ with  $f'(z)=0$, then $f$ is constant.
\end{lemma}
\begin{proof}
    Take $z_0 \in U$ and let $\w_0=f(z_0)$. Take $A=\{z \in U : f(z)=\w_0\}$.
    And pick a point $z \in U$. Let  $\{z_n\}$ a sequence of points in $A$
    converging to $z$; i.e. $\{z_n\} \xrightarrow{} z$. Since $f(z_n)=\w_0$ for
    all $n \geq 0$, and if  $f$ is continuous, we have  $f(z)=\w_0$ which makes
    $z \in A$, and  $A$ is closed in $U$.

    Now, let $a \in A$ and take $e>A$ such that $B(a,\e) \subseteq U$. If $z \in
    B(a,\e)$, let
    \begin{equation*}
        g(z)=f(tz+(1-t)a) \text{ where } 0 \leq t \leq 1
    \end{equation*}
    Then
    \begin{equation*}
        \frac{g(t)-g(s)}{t-s}=\frac{g(t)-g(s)}{(t-s)z+(s-t)a}\frac{(t-s)z+(s-t)a}{t-a}
    \end{equation*}
    thus, if $t \xrightarrow{} s$ we get
    \begin{equation*}
         g'(s)=\lim{\frac{g(t)-g(s)}{t-s}}=f'(sz+(1-s)a)(z-a)=0 \text{ for all }
         0 \leq s \leq 1
    \end{equation*}
    This makes $g$ constant, so that $f(z)=g(1)=g(0)=f(a)=\w_0$ and hence
    $B(a,\e) \susbeteq A$ which makes $A$ open. Therefore  $A=U$ and this makes
     $f$ constant on  $U$.
\end{proof}

\begin{example}\label{example_3.4}
    \begin{enumerate}
        \item[(1)] Differentiating $f'(z)=\exp{z}$ we get
            \begin{equation*}
                f'(z)=\sum{\frac{n}{n!}z^{n-1}}=\sum{\frac{z^{n-1}}{(n-1)!}}=
                \sum{\frac{z^n}{n!}}
            \end{equation*}
            which makes $f'(z)=f(z)$. Taking $e^z=\exp{z}$, that is
            \begin{equation*}
                \frac{d}{dz}{e^z}=e^z
            \end{equation*}

            Now, take $g(z)=e^ze^{a-z}$. Then $g'(z)=e^ze^{a-z}-e^ze^{a-z}=0$ so
            that $g$ is constant, and $g(z)=\w$ for all $z \in \C$. Taking
            $e^0=1$, we get $\w=g(0)=e^a$; moreover that $e^ze^{a-z}=a^a$. This
            shows that
            \begin{equation*}
                e^{a+b}=e^ae^b \text{ and } e^ze^{-z}=1 \text{ for all } a,b,z
                \in \C
            \end{equation*}
            and that
            \begin{equation*}
                e^{-z}=\frac{1}{e^z}
            \end{equation*}
            since $e^z \neq 0$ for all $z \in \C$.

            Moreover, notice that by the power series expansion, we have
            \begin{equation*}
                \bar{\sum{\frac{z^n}{n!}}}=\sum{\frac{\bar{z}^n}{n!}}
            \end{equation*}
            so that
            \begin{equation*}
                \bar{\exp{z}}=\exp{\bar{z}}
            \end{equation*}
            Now, if $\th \in \R$, notice that
            $|e^{i\th}|=e^{i\th}e^{-i\th}=e^0=1$ and
            $|e^z|^2=e^ze^{\bar{z}}=e^{z+\bar{z}}=2\re{z}$, and
            $|\exp{z}|=\exp{2\re{z}}$.

        \item[(2)] Define the follwing series $\sin{z}$ and $\cos{z}$ by
            \begin{align*}
                \sin{z} &=
                            z-\frac{z^3}{3!}+\frac{z^5}{5!}+\dots+(-1)^n\frac{z^{2n-1}}{(2n-1)!}    \\
                \cos{z} &=
                            1-\frac{z^2}{2!}+\frac{z^4}{4!}+\dots+(-1)^n\frac{z^{2n}}{(2n)!}    \\
            \end{align*}
            Then it can be shown that $\sin{z}$ and $\cos{z}$ are convergent
            power series with radius of convergence $R=\infty$, so that they
            are also analytic on all of  $\C$. Differentiating the power series,
            we find that
            \begin{equation*}
                \frac{d}{dz}{\sin{z}}=\cos{z} \text{ and }
                \frac{d}{dz}{\cos{z}}=-\sin{z}
            \end{equation*}
            We can also find that
            \begin{equation*}
                \sin{z}=\frac{e^{iz}-e^{-iz}}{2} \text{ and }
                \cos{z}=\frac{e^{iz}+e^{-iz}}{2}
            \end{equation*}
            which shows that
            \begin{equation*}
                \cos^2{z}+\sin^2{z}=1
            \end{equation*}

            In partiuclar if $\th \in \R$, we get
            $e^{i\th}=\cos{\th}+i\sin{\th}$, and hence, for all $z \in \Z$, we
            get $z=|z|e^{i\th}$ where $\th=\arg{z}$. Since
            $\exp{(a+b)}=(\exp{a})(\exp{b})$, we find that
            $\arg{\exp{z}}=\im{z}$.
    \end{enumerate}
\end{example}

\begin{definition}
    We call a complex valued function $f$ on $\C$ \textbf{periodic}, if there
    exists a $c \in \C$ such that  $f(z+c)=f(z)$ for all $z \in \C$. We call $c$
    the \textbf{period} of $f$.
\end{definition}

\begin{example}\label{example_3.5}
    If $c$ is the period of  $\exp{z}$, then $\exp{z}=\exp{(z+c)}=e^ze^c$,
    implying that $\exp{c}=1$. Thus $\re{c}=0$, and so $c=i\th$ for some  $\th
    \in \R$. Now, observe that
    \begin{equation*}
        \exp{c}=\exp{i\th}=\sin{\th}+i\sin{\th}=1
    \end{equation*}
    which makes $\th=2\pi k$, and hence $c=2i\pi k$, where $k \in \Z$.
\end{example}

\begin{example}[The Complex Logarithm]\label{example_3.6}
    Define $\log{w}$ such that $w=\exp{z}$ whenever $z=\log{w}$. Since $\exp{z}
    \neq 0$ for all $z \in \C$,  $\log{0}$ is undefined. Now, let $\exp{z}=w$,
    where $w \neq 0$, if  $z=x+iy$, then  $|w|=e^x$ and  $y=\arg{w}+2\pi k$ for
    some $k \in \Z$. So the set of solutions to  $\exp{z}$ is given by all
    $\log{|w|}+(\arg{w}+2\pi k)$. Notice then that $\log{|w|}$ defines the
    natural logarithm of $|w|$ by definition of  $\exp{z}$.
\end{example}

\begin{definition}
    We define a \textbf{region} of $\C$ to be an open and connected set of
    $\C$.
\end{definition}

\begin{definition}
    Let $U$ be a region of  $\C$, and let  $f:U \xrightarrow{} \C$ a continuous
    complex valued function such that $z=\exp{f(z)}$. Then we call $f$ a
    \textbf{branch of the logarithm}.
\end{definition}

\begin{lemma}\label{3.2.6}
    If $U$ is a region in  $\C$, and  $f$ is a branch of the logarithm in  $U$,
    then then any other branch of logarithm is of the form
    \begin{equation*}
        f(z)+2i\pi k \text{ where } k \in \Z^+
    \end{equation*}
\end{lemma}
\begin{proof}
    Let $f$ be a branch of the logarithm in  $U$ and let  $k \in \Z$. Take
    $g(z)=f(z)+2i\pi k$. Then $\exp{g}=\exp{f}=z$ so that $g$ is also a branch
    of the logarithm.

    Conversely, suppose that $f$ and $g$ are both branches of the logarithm;
    then $g(z)=f(z)+2i\pi k$, for some $k$ (not necessarrily an integer). Now,
    define
    \begin{equation*}
        h(z)=\frac{1}{2i\pi}(g(z)-f(z))
    \end{equation*}
    Then $h$ is continuous on  $U$, and  $h(U) \subseteq \Z$. Now, since $U$ is
    connected, then so is $h(U)$. Noticing that $h(z)=k$, this makes $k \in \Z$,
    and we are done.
\end{proof}

\begin{definition}
    Let $U=\com{\C}{\R_{\leq 0}}$; that is, the complex numbers $\C$ slit about
    the negative real axis. Notice that  $U$ is connected, and that
    $z=re^{i\th}$ where $r=|z|$ and $-\pi < \th < \pi$. Define
    $f(z)=\log{r}+i\th$. Since $f$ is the sum of two continuous functions, i.e;
     $\log{r}$ for all $r \in \com{\R}{\{0\}}$ and the map $\th \xrightarrow{}
     i\th$, we get that $f$ is continuous on  $U$. Therefore  $f$ defines a
     branch of the logarithm
\end{definition}

\begin{lemma}\label{3.2.7}
    Let $U$ and  $V$ be open in  $\C$. Let  $f:U \xrightarrow{} \C$ and $g:V
    \xrightarrow{} \C$ be continuous complex valued functions with $f(U)
    \subseteq V$ and $g \circ f(z)=z$. Then if $g$ is complex differentiable on
    $V$, $f$ is complex differentiable on  $U$, with
    \begin{equation*}
        f'(z)=\frac{1}{g' \circ f(z)}
    \end{equation*}
\end{lemma}
\begin{proof}
    Fix $z_0 \in U$ and let $h \in \C$ such that  $h \neq 0$ and  $h+z_0 \in U$.
    Then notice that $z_0=g \circ f(z_0)$ and $z_0+h=g \circ f(z_0+h)$. This
    makes $f(z_0) \neq f(z_0+h)$. Now, observe that
    \begin{equation*}
        \frac{g \circ f(z+h_0)-g \circ f(z_0)}{h}=
        \frac{g \circ f(z+h_0)-g \circ
        f(z_0)}{f(z_0+h)-f(z_0)}\frac{f(z_0+h)-f(z_0)}{z-z_0}=1
    \end{equation*}
    Now, taking the limit of both sides as $h \xrightarrow{} 0$, we get $g'
    \circ f(z_0)f'(z_0)=1$, and $f'(z_0)$ exists since $g' \circ f(z_0) \neq 0$.
\end{proof}
\begin{corollary}
    If $g$ is analytic on  $V$, then  $f$ is analyitic on  $U$.
\end{corollary}
\begin{corollary}
    If $U$ is a connected; i.e. a region, and $f$ is a branch of the logarithm
    on  $U$, then  $f$ is analytic.
\end{corollary}
\begin{proof}
    Observe by definition that $z=\exp{f(z)}$.
\end{proof}

\begin{definition}
    We define the \textbf{principle branch of the logarithm} on
    $\com{\C}{\R_{\leq 0}}$ to be the complex function
    $\log:\com{\C}{\R_{\leq 0}} \xrightarrow{} \C$ defined by
    \begin{equation*}
        \log{z}=\log{r}+i\th \text{ where }z=r\exp{i\th}
    \end{equation*}
\end{definition}

\begin{definition}
    If $f$ is a branch of the logarithm on some region  $U$, and if  $b \in \Z$.
    The \textbf{branch} of $bf(z)$ to be a complex function $g:U \xrightarrow{}
    \C$ defined by $g(z)=exp{bf(z)}$. We write $g(z)=z^b$, when the branch $f$
    is the principle branch of the logarithm.
\end{definition}

\begin{lemma}\label{3.2.8}
    Let $U$ be a region, and  $f$ a branch of the logarithm and let  $b \in \Z$.
    Then the branch of  $bf(z)$, $g(z)=\exp{bf(z)}$ is analytic.
\end{lemma}
\begin{corollary}
    The branch $z^b$ of  $\log{z}$ is analytic.
\end{corollary}

\begin{definition}
    Let $u:\R \times \R \xrightarrow{} \R$ and $v:\R \times \R \xrightarrow{}
    \R$ be realvalued functions. We define the \textbf{Cauchy Riemann equations}
    to be the systems of equations
    \begin{align*}
        \frac{\partial{u}}{\partial{x}}-\frac{\partial{v}}{\partial{y}} &= 0 \\
        \frac{\partial{u}}{\partial{y}}+\frac{\partial{v}}{\partial{x}} &= 0 \\
    \end{align*}
\end{definition}

\begin{lemma}\label{3.2.9}
    Let $f:U \xrightarrow{} \C$ a complex valued function with
    $f(z)=f(x+iy)=u(x,y)+iv(x,y)$. Then if $f$ is analytic, we have
    \begin{equation*}
        f'(z)=\frac{\partial{u}}{\partial{x}}(x,y)+i\frac{\partial{v}}{\partial{x}}
    \end{equation*}
\end{lemma}
\begin{proof}
    We evaluate $f'(z)$ as $h \xrightarrow{} 0$ along the real axis $\R$ of
    $\C$.
\end{proof}

\begin{lemma}\label{3.2.10}
    Let $f:U \xrightarrow{} \C$ a complex valued function with
    $f(z)=f(x+iy)=u(x,y)+iv(x,y)$. Then if $f$ is analytic, we have
    \begin{equation*}
        f'(z)=\frac{\partial{v}}{\partial{y}}(x,y)-i\frac{\partial{u}}{\partial{y}}
    \end{equation*}
\end{lemma}
\begin{proof}
    We evaluate $f'(z)$ as $h \xrightarrow{} 0$ along the imaginary axis $i\R$
    of $\C$.
\end{proof}

\begin{theorem}\label{3.2.11}
    Let $u(z)=u(x,y)$ and $v(z)=v(x,y)$, where $z=x+iy$ be real valued
    functions on a region $U$ with continuous partial derivarives. If $f:U
    \xrightarrow{} \C$ is a complex valued function such that
    $f(z)=u(x,y)+iv(x,y)$, then $f$ is analytic on $U$ if, and only if $u$ and
    $v$ satisfy the Cauchy Riemann equations.
\end{theorem}
\begin{proof}
    Suppose that $u$ and  $v$ satisfy the Cauchy Riemann equations, and let
    $B(z,r) \subseteq U$. If $h=s+it \oin B(0,r)$, then
    \begin{equation*}
        u(x+s,y+t)-u(x,y)=(u(x+s,y+t)-u(x,y+t))+(u(x,y+t)-u(x,y))
    \end{equation*}
    By the mean value theorem for real valued functions, we have that there
    exist $s_1, t_1 \in B(0,r)$ with $|s_1|<|s|$ and $|t_1|<|t|$ for which
    \begin{equation*}
        u(x+s,y+t)-u(x,y+t)=u_x(x+s_1,y+t)s
    \end{equation*}
    and
    \begin{equation*}
        u(x,y+t)-u(x,y+t)=u_y(x,y+t)t
    \end{equation*}
    Letting
    \begin{equation*}
        \phi(s,t)=(u(x+s,y+y)-u(x,y))-(u_x(x,y)s-u_y(x,y)t)
    \end{equation*}
    Then
    \begin{equation*}
        \frac{\phi(s,t)}{s+it}=\frac{s}{s+it}(u_x(x+s_1,y+t)-u_x(x,y))+
        \frac{t}{s+it}(u_y(x,y+t_1)-u_y(x,y))
    \end{equation*}
    Now, since $|s| \leq |s+it|$ and  $|t| \leq |s+it|$, and  $u_x$ and  $u_y$
    are continuous, we get
    \begin{equation*}
        \lim_{s+it \xrightarrow{} 0}{\frac{\phi(s,t)}{s+it}}=0
    \end{equation*}
    So that
    \begin{equation*}
        u(x+s,y+t)-u(x,y)=u_x(x,y)s+u_y(x,y)t+\phi(s,t)
    \end{equation*}
    Similarly, for $v$ we get
    \begin{equation*}
        v(x+s,y+t)-v(x,y)=v_x(x,y)s+v_y(x,y)t+\psi(s,t)
    \end{equation*}
    where
    \begin{equation*}
        \lim_{s+it \xrightarrow{} 0}{\frac{\psi(s,t)}{s+it}}=0
    \end{equation*}

    Notice then that
    \begin{equation*}
        \frac{f(z+(s+it))-f(z)}{s+it}=u_x(x,y)+iv(x,y)+\frac{\phi(s,t)+i\psi(s,t)}{s+it}
    \end{equation*}
    Taking $s+it \xrightarrow{} 0$, makes $f$ complex differentiable on $U$ with
    $f'(z)=u_x(x,y)+iv(x,t)$. Since $u_x$ and  $v_x$ are continuous, so is  $f$,
    which makes  $f$ analytic.

    Conversely, if we suppose that  $f$ is analytic, then by lemma \ref{3.2.9}
    and lemma \ref {3.2.10}, we get
    \begin{equation*}
        \frac{\partial{u}}{\partial{x}}(x,y)+i\frac{\partial{v}}{\partial{x}}=
        \frac{\partial{v}}{\partial{y}}(x,y)-i\frac{\partial{u}}{\partial{y}}
    \end{equation*}
    which shows that $u$ and  $v$ satisfy the Cauch Riemann equations.
\end{proof}

\begin{definition}
    We call a real valued function $u:\R \times \R \xrightarrow{} \R$
    \textbf{harmonic} if
    \begin{equation*}
        \frac{\partial^2{u}}{\partial{x}^2}+\frac{\partial^2{u}}{\partial{y}^2}=0
    \end{equation*}
    If $v:\R \times \R \xrightarrow{} \R$ is a real valued function, and $f$ is
    a complex valued function for which  $f(z)=u(x,y)+iv(x,y)$, then we call
    $v$ the  \textbf{harmonic conjugate} of $u$.
\end{definition}

\begin{example}\label{example_3.7}
    The function $u(x,y)=\log{\sqrt{x^2+y^2}}$ is harmonic.
\end{example}

\begin{theorem}\label{3.2.12}
    Let $U$ be an open ball, or $\C$. If $u:U \xrightarrow{} \C$ is harmonic,
    then $u$ has a harmonic conjugate.
\end{theorem}
\begin{proof}
    Let $U=B(0,R)$ for $0 \leq R \leq \infty$, and let  $u:U \xrightarrow{} \C$
    be harmonic. Define
    \begin{equation*}
        v(x,y)=\int_0^y{u_x(x,t) \ dt}+\phi(x)
    \end{equation*}
    wheere $\phi$ is determined by taking  $v_x+u_y=0$. Then differentiating
    both sides, and by Leibniz's rule for differentiating under the integral,
    we have
    \begin{align*}
        v_x(x,y) &= \int_0^y{u_{xx}(x,t) \ dt}+\phi'(t) \\
                 &= -\int_0^y{u_{xx}(x,t) \ dt}+\phi'(x)    \\
                 &= -u_y(x,y)+u_y(x,0)+phi'(x)  \\
    \end{align*}
    and $\phi'(x)=u_y(x,0)$. Then $u$ and  $v=\int_0^y{u_x(x,t) \ dt}-
    \int_0^x{u_y(s,0) \ ds}$ satisfy the Cauchy Riemann equations. This makes
    $v$ a harmonic conjugate.
\end{proof}
