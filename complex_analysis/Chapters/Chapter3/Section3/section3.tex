\section{M\"obius Transformations}

\begin{definition}
    We define a \textbf{path} on $\C$ to be a continnuous complex valued
    functiom $\y:[a,b] \xrightarrow{} \C$. We call $\y(a)$ the \textbf{initial
    point} of $\y$ and $\y(b)$ the \textbf{end point} of $\y$. Moreover wecall
    $\y$ \textbf{piecewise $C^1$} if there exists a partition
    $P=\{a=t_0<\dots<t_n=b\}$ of $[a,b]$ such that $\y$ is  $C^1$ on each
    $[t_i,t_{i+1}]$ for all $0 \leq i \leq n-1$.
\end{definition}

\begin{definition}
    We call a complex valued function $S$ on $C_\infty$, defined by
    \begin{equation*}
        S(z)=\frac{az+b}{cx+d} \text{ where } a,b,c,d \in \C
    \end{equation*}
    a \textbf{M\"obius transformation} if $ad-cd \neq 0$.
\end{definition}

We have the following results for M\"obius transformations.

\begin{theorem}\label{3.3.1}
    The set of all M\"obius transformations forms a group under function
    compositon.
\end{theorem}
\begin{proof}
    Since $\circ$ is associative, it suffices to show the identity, closure, and
    inverse laws. Indeed, notice that the function
    \begin{equation*}
        I=\frac{1z+0}{0z+1}=z
    \end{equation*}
    is a M\"obius transformation, and that for any M\"obius transformation, $S$,
    $S=S \circ I=I \circ S$.

    Now, Let $S$ and $T$ be M\"obius transformations. Then
    \begin{equation*}
        S(z)=\frac{az+b}{cz+d} \text{ and } T(z)=\frac{fz+g}{hz+l}
    \end{equation*}
    for $a,b,c,d,f,g,h,l \in \C$. Then we get
    \begin{equation*}
        S \circ T(z)=\frac{aT(z)+b}{cT(z)+b}
    \end{equation*}
    and since $T(z) \in \C_\infty$, for all values of $\z$, then $S \circ T$ is a
    M\"obius transformation.

    Finally, let
    \begin{equation*}
        \inv{S}(z)=\frac{dz-b}{cz-a}
    \end{equation*}
    Then $S \circ \inv{S}=\inv{S} \circ S(z)=I(z)$, and we are done.
\end{proof}

\begin{definition}
    Let $a \in \C$. We call a M\"obius trasnsformation of the form $T(z)=z+a$ a
    \textbf{translation} of $z$ by  $a$. We call a M\"obius transformation
    $D(z)=az$ a \textbf{dilation} of $z$ by  $a$.  Let $0 \leq t \leq 2\pi$.
    Then we call the M\"obius transformation  $R(z)=e^{it}z$ a \textbf{rotation}
    of $z$ about  $t$, and we call the M\"obius transformation
    $S(z)=\frac{1}{z}$ an \textbf{inversion} of $z$.
\end{definition}

\begin{lemma}\label{3.3.2}
    If $S$ is a M\"obius transformation, then  $S$ is the composition of
    translations, dilations, and inversions.
\end{lemma}
\begin{proof}
    Let
    \begin{equation*}
        S(z)=\frac{az+b}{cz+d}
    \end{equation*}
    Suppose that $c=0$, so that  $S(z)=\frac{a}{d}z+b$. Then $S=S_2 \circ S_1$
    where $S_1(z)$ is a translation by $b$ and $S_2(z)$ is a dilation by
    $\frac{a}{d}$.

    Now, if $c \neq 0$, then let  $S_1(z)=z+\frac{d}{c}$,
    $S_2(z)=\frac{1}{z}$, $S_3(z)=\frac{bc-ad}{c^2}z$ and
    $S_4(z)=z+\frac{a}{c}$. Then $S=S_4 \circ S_3 \circ S_2 \circ S_1$.
\end{proof}
\begin{corollary}
    Rotations are compositions of translations, dilations, and inversions.
\end{corollary}

\begin{definition}
    Let $S$ be a M\"obius transformation. We cal a point $z \in \C$ a
    \textbf{fixed point} of $S$ if  $S(z)=z$.
\end{definition}
