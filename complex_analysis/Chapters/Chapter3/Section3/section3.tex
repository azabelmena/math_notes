\section{M\"obius Transformations}

\begin{definition}
    We define a \textbf{path} on $\C$ to be a continnuous complex valued
    functiom $\y:[a,b] \xrightarrow{} \C$. We call $\y(a)$ the \textbf{initial
    point} of $\y$ and $\y(b)$ the \textbf{end point} of $\y$. Moreover wecall
    $\y$ \textbf{piecewise $C^1$} if there exists a partition
    $P=\{a=t_0<\dots<t_n=b\}$ of $[a,b]$ such that $\y$ is  $C^1$ on each
    $[t_i,t_{i+1}]$ for all $0 \leq i \leq n-1$.
\end{definition}

\begin{definition}
    We call a complex valued function $S$ on $C_\infty$, defined by
    \begin{equation*}
        S(z)=\frac{az+b}{cx+d} \text{ where } a,b,c,d \in \C
    \end{equation*}
    a \textbf{M\"obius transformation} if $ad-cd \neq 0$.
\end{definition}

We have the following results for M\"obius transformations.

\begin{theorem}\label{3.3.1}
    The set of all M\"obius transformations forms a group under function
    compositon.
\end{theorem}
\begin{proof}
    Since $\circ$ is associative, it suffices to show the identity, closure, and
    inverse laws. Indeed, notice that the function
    \begin{equation*}
        I=\frac{1z+0}{0z+1}=z
    \end{equation*}
    is a M\"obius transformation, and that for any M\"obius transformation, $S$,
    $S=S \circ I=I \circ S$.

    Now, Let $S$ and $T$ be M\"obius transformations. Then
    \begin{equation*}
        S(z)=\frac{az+b}{cz+d} \text{ and } T(z)=\frac{fz+g}{hz+l}
    \end{equation*}
    for $a,b,c,d,f,g,h,l \in \C$. Then we get
    \begin{equation*}
        S \circ T(z)=\frac{aT(z)+b}{cT(z)+b}
    \end{equation*}
    and since $T(z) \in \C_\infty$, for all values of $\z$, then $S \circ T$ is a
    M\"obius transformation.

    Finally, let
    \begin{equation*}
        \inv{S}(z)=\frac{dz-b}{cz-a}
    \end{equation*}
    Then $S \circ \inv{S}=\inv{S} \circ S(z)=I(z)$, and we are done.
\end{proof}

\begin{definition}
    Let $a \in \C$. We call a M\"obius trasnsformation of the form $T(z)=z+a$ a
    \textbf{translation} of $z$ by  $a$. We call a M\"obius transformation
    $D(z)=az$ a \textbf{dilation} of $z$ by  $a$.  Let $0 \leq t \leq 2\pi$.
    Then we call the M\"obius transformation  $R(z)=e^{it}z$ a \textbf{rotation}
    of $z$ about  $t$, and we call the M\"obius transformation
    $S(z)=\frac{1}{z}$ an \textbf{inversion} of $z$.
\end{definition}

\begin{lemma}\label{3.3.2}
    If $S$ is a M\"obius transformation, then  $S$ is the composition of
    translations, dilations, and inversions.
\end{lemma}
\begin{proof}
    Let
    \begin{equation*}
        S(z)=\frac{az+b}{cz+d}
    \end{equation*}
    Suppose that $c=0$, so that  $S(z)=\frac{a}{d}z+b$. Then $S=S_2 \circ S_1$
    where $S_1(z)$ is a translation by $b$ and $S_2(z)$ is a dilation by
    $\frac{a}{d}$.

    Now, if $c \neq 0$, then let  $S_1(z)=z+\frac{d}{c}$,
    $S_2(z)=\frac{1}{z}$, $S_3(z)=\frac{bc-ad}{c^2}z$ and
    $S_4(z)=z+\frac{a}{c}$. Then $S=S_4 \circ S_3 \circ S_2 \circ S_1$.
\end{proof}
\begin{corollary}
    Rotations are compositions of translations, dilations, and inversions.
\end{corollary}

\begin{definition}
    Let $S$ be a M\"obius transformation. We cal a point $z \in \C$ a
    \textbf{fixed point} of $S$ if  $S(z)=z$.
\end{definition}

\begin{lemma}\label{3.3.3}
    If
    \begin{equation*}
        S(z)=\frac{az+b}{cz+d}
    \end{equation*}
    is a M\"obius transformation, and $z$ is a fixed point, then
    $cz^2+(d-a)z-b=0$ and $S$ has at most two fixed points; unless it is the
    identity transformation.
\end{lemma}
\begin{proof}
    Suppose that $S \neq I$, and consider the equation
    \begin{equation*}
        z=\frac{az+b}{cz+d}
    \end{equation*}
    to obtain a quadratic polynomial over $\C$, which has at most two roots in
    $\C$.
\end{proof}

\begin{lemma}\label{3.3.4}
    Let $S$ be a M\"obius transformation on  $\C_\infty$, and let  $a,b,c,d \in
    \C_\infty$ distinct points with  $\a=S(a)$, $\b=S(b)$ and $\y=S(c)$. If $T$
    is another M\"obius transformation with this property, then  $S=T$.
\end{lemma}
\begin{proof}
    Notice by hypothesis that the transformation $\inv{T} \circ S$ has $a$,
    $b$, and  $c$ as fixed points, which forces  $\inv{T} \circ S=I$.
\end{proof}

\begin{definition}
    Let $z_2,z_3,z_4 \in \C_\infty$ and define the M\"obius transformation
    $S:\C_\infty \xrightarrow{} \C_\infty$ by
    \begin{align*}
        S(z)    &=  \frac{\Big{(} \frac{z-z_3}{z-z_4} \Big{)}}{\Big{(}
                        \frac{z_2-z_3}{z_2-z_4} \Big{)}} \text{ if } z_2,z_3,z_4
                                \in C   \\
        S(z)    &= \frac{z-z_3}{z-z_4} \text{ if } z_2=\infty   \\
        S(z)    &= \frac{z_2-z_4}{z-z_4} \text{ if } z_3=\infty   \\
        S(z)    &= \frac{z_2-z_3}{z_2-z_3} \text{ if } z_4=\infty   \\
    \end{align*}
    and where $S(z_2)=1$, $S(z_3)=0$, and $S(z_4)=\infty$. Then if $z_1 \in
    \C_\infty$, we define the \textbf{cross ratio}, $(z_1,z_2,z_3,z_4)$ of $z_1$
    to be $S(z_1)$.
\end{definition}

\begin{example}\label{example_3.8}
    $(z_2,z_2,z_3,z_4)=1$, $(z_3,z_2,z_3,z_4)=0$, and $(z_4,z_2,z_3,z_4)=0$, by
    definition. Now, if $M$ is aany M\"obius transformation, and  $w_2,w_3,w_4$
    are points on $M$ such that $M(w_1)=1$, $M(w_3)=0$, and $M(w_4)=\infty$,
    then $M(z)=(z,w_2,w_3,w_4)$.
\end{example}

\begin{theorem}\label{3.3.5}
    IF $z_2,z_3,z_4 \in \C_\infty$ are distinct points, and $T$ is a M\"obius
    transformation, then  $(z,z_2,z_3,z_4)=(T(z),T(z_2),T(z_3),T(z_4))$ for all
    $z \in \C_\infty$. That is, the cross ratio is invariant under
    transformations.
\end{theorem}
\begin{proof}
    Let $S=(z,z_2,z_3,z_4)$, then $S$ is a M\"obius transformation. Now, if
    $M=S \circ \inv{T}$, then $M(T(z_2))=1$, $M(T(z_3))=0$, and
    $M(T(z_4))=\infty$, which makes $S \circ \inv{T}=(z,T(z_2),T(z_3),T(z_4))$.
\end{proof}

\begin{lemma}\label{3.3.6}
    If $z_2,z_3,z_4 \in \C_\infty$ and $w_2,w_3,w_4 \in \C_\infty$ are all
    distinct points, then there exists one, and only one M\"obius transformation
    $S$ for which $S(z_2)=w_2$, $S(z_3)=w_3$, and $S(z_4)=w_4$.
\end{lemma}
\begin{proof}
    Let $T(z)=(z,z_2,z,3,z_4)$ and $M=(z,w,2,w,3,w_4)$. Put $S=\inv{M} \circ T$.
    Then $s(Z_2)=w_2$, $S(z_3)=z_3$,and $S(z_4)=w_4$. Now, if $R$ is another
    M\"obius transformation having this property, then  $\inv{R} \circ S$ has
    $3$ fixed points, which makes $\inv{R} \circ S=I$.
\end{proof}

\begin{lemma}\label{3.3.7}
    Let $z_1,z_2,3,z_4 \in \C_\infty$ be distinct points. Then
    $(z,1,z_2,z_3,z_4) \in \R$ if, and only if all the points lie on a circle.
\end{lemma}
