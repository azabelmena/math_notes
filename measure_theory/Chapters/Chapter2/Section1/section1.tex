\section{Lebesgue Outermeasure}

\begin{definition}
    Let $I$ be a nonempty interval of  $\R$. We define the  \textbf{lenght} of
    $I$, denoted  $l(I)$, to be the difference of its endpoints, if $I$ is
    bounded, and  $\infty$ otherwise.
\end{definition}

\begin{definition}
    Let $A$ a subset of  $\R$. We define the  \textbf{Lebesgue outer measure} of
    $A$ to be
    \begin{equation*}
        m^\ast(A)=\inf{\{\sum{l(I_k)}\}}
    \end{equation*}
    Where $\{I_k\}$ is a countable collection of bounded open sets, covering
    $A$.
\end{definition}

\begin{lemma}\label{2.1.1}
    The emptyset has Lebesgue outermeasure $0$. Moreover, the Lebesgue
    outermeasure is monotone; that is, if $A,B \subseteq \R$ such that  $A
    \subseteq B$, then  $m^\ast(A) \leq m^\ast(B)$.
\end{lemma}
\begin{proof}
    Notice that the singleton $\{a\}=[a,a]$ covers the emptyset $\emtpyset$.
    Moreover  $l([a,a])=a-a=0$, so by definition $m^\ast(\emptyset)=0$.

    Now, let $A$,  $B$ subsets of  $\R$ such that  $A \subseteq B$. Then if
    $\{I_k\}$ is a countable collection of bounded open sets covering $B$, they
    also cover  $A$, hence by definition, we get  $m^\ast(A) \leq m^\ast(B)$.
\end{proof}
\begin{corollary}
    Lebesgue outermeasure is nonnegative. That is, $0 \leq m^\ast(E)$ for any
    set $E \subseteq \R$.
\end{corollary}
\begin{proof}
    Notice the length of any interval $I$ is nonnegative.
\end{proof}

\begin{example}\label{example_2.1}
    Countable sets have measure $0$. Let  $C$ be a countable set with
    enumeration  $\{c_k\}$. Let $\e>0$ and define  $I_k=(c_k-\frac{\e}{2^{k+1}},
    c_k+\frac{\e}{2^{k+1}})$. Then $\{I_k\}$ is a countable collection of
    bounded open sets covering $C=\{c_k\}$. Hence we get that
    \begin{equation*}
        0 \leq m^\ast(C) \leq \sum{I_k} \leq \sum{\frac{\e}{2^k}}=0
    \end{equation*}
    So that $m^\ast(C)=0$.
\end{example}

\begin{lemma}\label{2.1.2}
    For any nonempty interval $I$,  $m^\ast(I)=l(I)$.
\end{lemma}
\begin{proof}
    Consider first, the closed bounded interval $[a,b]$, where $a<b$. Let
    $\e>0$. Notice that  $[a,b] \subseteq (a-\e,b+\e)$, so that $m^\ast([a,b])
    \leq l((a-\e,b+\e))=b-a+2\e$. Hence $m^\ast([a,b]) \leq b-a$. It remains to
    show that $b-a \leq m^\ast([a,b])$.

    Let $\{I_k\}$ a countable collection of open bounded intervals covering
    $[a,b]$. By the theorem of Heine-Borel, there is a finite subcover
    $\{I_k\}_{k=1}^n$  of $[a,b]$. Notice that since $a \in \bigcup{I_k}$, at least
    one $I_k$ contains $a$. Hence choose an interval $(a_1,b_1)$ in this cover
    for which $a_1<a<b_1$. Now, if $b<b_1$, we are done as
    \begin{equation*}
        \sum_{k=1}^n{l(I_k)} \geq b_1-a_1>b-a
    \end{equation*}
    Otherwise, $b_1 \in [a,b_1)$. In this case, choose an interval $(a_2,b_2)$,
    distinct from $(a_1,b_1)$ for which $a_2<b_1<b_2$. If $b_2 \geq b$, then we
    are done by similar reasoing as above. Otherwise, continue the process of
    choosing intervals. This process terminates as we eventually exhaust the
    endpoints of each $I_k$ in the open cover. Thus, we get a subcollection
    $\{(a_k,b_k)\}_{k=1}^N$ for which $a_1<a$ and $a_{k+1}<b_k$ for all $1 \leq
    k \leq N-1$. We also have a  $b_N>b$. Then we have
    \begin{equation*}
        \sum{l(I_k)} \geq \sum_{k=1}^N{l((a_k,b_k))}=(b_N-a_N)+\dots+(b_1-a_1)
        \geq b-a
    \end{equation*}
    so that we get $b-a \leq m^\ast([a,b])$.

    Now, let $I$ be any bounded interval. Notice that there exist closed bounded
    intervals  $J_1$ and $J_2$ for which
    \begin{equation*}
        J_1 \subseteq I \subseteq J_2
    \end{equation*}
    and for some $\e>0$,
    \begin{equation*}
        l(I)-\e<l(J_1) \leq l(I) \leq l(J_1)<l(I)+\e
    \end{equation*}
    Then since $J_1$ and $J_2$ are closed and bounded intervals, and by
    monotonicity of $m^\ast$, we have
    \begin{equation*}
        l(I)-\e<m^\ast(J_1) \leq m^\ast(I) \leq m^\ast(J_1)<l(I)+\e
    \end{equation*}
    so that $l(I)-\e<m^\ast(I)<l(I)+\e$ for all $\e>0$. This establishes
    equality.
\end{proof}

\begin{lemma}\label{2.1.3}
    The Lebesgue outermeasure is translation invariant. That is, if $A \subseteq
    \R$, and  $y \in \R$, then  $m^\ast(A)=m^\ast(A+y)$.
\end{lemma}
\begin{proof}
    Notice that a countable collection of open bounded intervals $\{I_k\}$
    covers $A$ if, and only if the collection  $\{I_k+y\}$ of open bounded
    intervals covers $A+y$. Moreover, notice that  $l(I_k)=l(I_k+y)$, so that we
    get
    \begin{equation*}
        \sum{l(I_k)}=\sum{l(I_k+y)}
    \end{equation*}
    the rest follows from definition.
\end{proof}

\begin{lemma}\label{2.1.4}
    The Lebesgue outermeasure is countable subadditive; that is, if $\{E_k\}$ is
    a collection of subsets of $\R$, then
    \begin{equation*}
        m^\ast(\bigcup{E_k}) \leq \sum{m^\ast(E_k)}
    \end{equation*}
\end{lemma}
\begin{proof}
    Let $\{E_k\}$ a countable collection of sets, and let $E=\bigcup{E_k}$.
    Notice that if atleast one $E_k$ has infinite measure, then we are done.
    Suppose then that for all $k$, $m^\ast(E_k)$ is finite. Let $\e>0$. Then for
    all  $k$, there exists a countable collection of open bounded intervals
    $\{I_{k,i}\}$ covering $E_k$, and
    $\sum_i{l(I_{k,i})}<m^\ast(E_k)+\frac{\e}{2^k}$. By definition, we get
    \begin{equation*}
        m^\ast(E) \leq \sum{l(I_{k,i})}=\sum_k{\sum_i{l(I_{k,i})}}
        <\sum_k{(m^\ast(E_k)+\frac{\e}{2^k})}=\sum_k{m^\ast(E_k)}+\e
    \end{equation*}
    for all $\e>0$. This inequality also holds for  $\e=0$.
\end{proof}
\begin{corollary}
    The Lebesgue outermeasure is finitely subadditive.
\end{corollary}
\begin{proof}
    Recall that finite collections are also countable collectuions.
\end{proof}
