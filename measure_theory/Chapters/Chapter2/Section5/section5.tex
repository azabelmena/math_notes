\section{Nonmeasurable Sets, The Cantor Set, and The Cantor-Lebesgue Function}

\begin{definition}
    We call a set $E$ of real numbers  \textbf{nonmeasurable} if it is not
    measurable.
\end{definition}

\begin{lemma}\label{2.5.1}
    If $E$ is a bounded measurable set of real numbers, and
    there is a countably infinite disjoint collection of translates  $\{E+\lm\}$,
    then $m(E)=0$.
\end{lemma}
\begin{proof}
    Since $E$ is measurable, so is  $\E+\lm$ for every  $\lm$. Then by countable
    additivity, we have
    \begin{equation*}
        m(\bigcup{E+\lm})=\sum{m(E+\lm)}=\sum{m(E)}
    \end{equation*}
    Now, since $E$ is bounded, so is each  $E+\lm$, and hence, so is
    $\bigcup{E+\lm}$ so that $m(\bigcup{E+\lm})$ is finite Therefore, $m(E)$ is
    finite. Moreover, since the collection $\{E+\lm\}$ is countably infinite,
    and $m(E)$ is finite, this forces $m(E)=0$.
\end{proof}

\begin{definition}
    We call two real numbers $x,y \in \R$ \textbf{rationally equivalent} if
    $x-y \in \Q$.
\end{definition}

\begin{lemma}\label{2.5.2}
    Rational equivalence is an equivalence relation on $\R$.
\end{lemma}

\begin{theorem}[Vitali's Theorem]\label{2.5.3}
    Any set $E$ of real numbers with positive outer measure contains a
    nonmeasurable set.
\end{theorem}
\begin{proof}
    Consider rational equivalence on $E$, which partitions $E$ into equivalence
    classes. Define  $\Cc_E$ a choice set of the equivalence classes on  $E$
    consisting of exactly one member from each class, such that
    \begin{enumerate}
        \item[(1)] For all $x,y \in \Cc_E$,  $x-y \notin \Q$.

        \item[(2)] For all $x \in E$, there exists a  $c \in \Cc_E$ for which
            $x=c+q$ for some  $q \in \Q$.
    \end{enumerate}

    Now, by countable subadditivity, suppose that $E$ is bounded, and consider
    the choice set  $\Cc_E$  (defined above) of $E$. Then  $\Cc_E$ is
    nonmeasurable.

    Suppose otherwis e. Let  $\Lambda_0$ a bounded countably infinite set of
    rational numbers. Then each $\{\Cc_E+\lm\}$ is measurable for each $\lm \in
    \Lambda_0$. Then we have a countably infinite disjoint collection of bounded
    translates, hence by lemma \ref{2.5.1}, $m(\Cc_E)=0$. That is,
    \begin{equation*}
        m(\bigcup{\Cc_E+\lm})=\sum{m(\Cc_E+\lm)}=0
    \end{equation*}
    Since $E$ is bounded, choose  $\Lambda_0=[-2b,2b] \cap \Q$, for some $b \in
    \R$. If  $x \in E$, there exists a  $c \in \Cc_E$ and a  $q \in \Q$ such
    that  $x=c+q$. That is $x,c \in [-b,b]$ and $q \in [-2b,2b]$ so that $E
    \subseteq \bigcup{\Cc_E+\lm}$. But $m(E)$ is positive, which yields a
    contradiction as $m(\Cc_E)=0$. Therefore $\Cc_E$ can't possibly be
    measurable.
\end{proof}

\begin{theorem}\label{2.5.4}
    There exist disjoint sets $A$ and  $B$ of real numbers such that
    \begin{equation*}
        m^\ast(A \cup B)<m^\ast(A)+m^\ast(B)
    \end{equation*}
\end{theorem}

\begin{definition}
    We define the \textbf{Cantor set} to be the intersection
    \begin{equation*}
        \Cc=\bigcap{C_k}
    \end{equation*}
    where $\{C_k\}$ is a decreasing sequence of closed sets such that for every
    $k$,  $C_k$ is the disjoint union of  $2^k$ closed intervals of length
    $\frac{1}{3^k}$
\end{definition}

\begin{theorem}\label{2.5.5}
    The Cantor set is a closed uncountable set of measure $0$.
\end{theorem}
\begin{proof}
    Since $\Cc$ is an arbitrary intersection of closed sets, it is closed in
    $\R$. Moreover, since each $C_k$ is the disjoint union of closed intervals,
    which are measurable, and since measurable sets form a  $\s$-algebra, then
    each  $C_k$ is measurable, which makes  $\Cc$ measurable.

    Now, by definition of $C_k$, by finite additivity, we have
    \begin{equation*}
        m(C_k)=(\frac{2}{3})^k
    \end{equation*}
    so that by monotonicity of measure,
    \begin{equation*}
        m(\Cc) \leq m(C_k)=(\frac{2}{3})^k
    \end{equation*}
    now, as $k \xrightarrow{} \infty$, $m(C_k) \xrightarrow{} 0$ so that
    $m(\Cc)=0$. It remains to show that $\Cc$ is uncountable.

    Suppose  $\Cc$ is countable, and let $\{c_k\}$ be an enumeration of $\Cc$.
    Now, there is a disjoint interval $F_1$ in $C_1$ which fails to contain the
    point $c_1$; similarly, there is a disjoint interval $F_2$ in $C_2$, whose
    union is $F_1$, that fails to contain $c_2$. Proceeding inductively, we
    obtain a countable collection $\{F_k\}$ such that
    \begin{enumerate}
        \item[(1)] Each $F_k$ is closed.

        \item[(2)] $F_k \subseteq C_k$.

        \item[(3)] $c_k \notin F_k$.
    \end{enumerate}
    by the nested set theorem, the intersection $F=\bigcap{F_k}$ is nonempty.
    Now, let $ x \in F$, then we get that  $x \in C_k$ for some  $k$. But since
     $\Cc_k$ is countable, and enumerated by $\{c_k\}$, then $x=c_n$ for some
     $n$. That is,  $c_n \in F$ which contradicts that  $c_n \notin F_n$.
     Therefore  $\Cc$ is uncountable.
\end{proof}

\begin{definition}
    Define $U_k=\com{[0,1]}{C_k}$ and $\Uc=\bigcup{U_k}$, so that
    $\Cc=\com{[0,1]}{\Uc}$. Define the functin $\phi:U_k \xrightarrow{} \R$ to be
    the increasing function, which is constant on each of the $2^k-1$ open
    intervals, and which takes the values of the form $\frac{2^k-1}{2^k}$ in
    each of the intervals. We define the \textbf{Cantor-Lebesgue function} to be
    the extension of $\phi$ to $[0,1]$ by defining it on $\Cc$ as follows
    \begin{equation*}
        \phi(0)=0 \text{ for all } x \in \Uc
        \text{ and } \phi(x)=\sup{\{\phi(t) : t \in U \cap [0,x) \text{ if } x
        \in \com{\Cc}{0}\}}
    \end{equation*}
\end{definition}

\begin{lemma}\label{2.5.6}
    The Cantor-Lebesgue function is increasing continuous whos image is the
    interval $[0,1]$. Moreover, $\phi$ is differentiable on $\Uc$, with
    $\phi'=0$ on  $\Uc$, where  $m(\Uc)=1$.
\end{lemma}
\begin{proof}
    By definition, $\phi|_{U_k}$ is increasing so the extension $\phi$ is
    increasing as well. Likewise,  $\phi|_{U_k}$ is continuous, hence so is the
    extension $\phi$.

    Now, consider  $x_0 \in \Cc$ such that  $x_0 \neq 0,1$. Then $x \notin U_k$,
    and for  $k$ large enough,  $x_0$ is between two consecutive intervals of
    $U_k$. Let  $a_k$ be in the lower of these two intervals, and  $b_k$ in the
    upper. Since  $\phi$ is increasing, inparticular, by $\frac{1}{2^k}$, we get
    $a_k<x_b_k$ and $\phi(b_k)-\phi(a_k)=\frac{1}{2^k}$.. Then as $k
    \xrightarrow{} \infty$ $\phi(b_k)-\phi(a_k) \xrightarrow{} 0$ so that $\phi$
    has no jump discontinuities at $x_0$. This makes $\phi$ continuous at
    $x_0$. Now, if $x_0=0$ or $x_0=1$, a similar argument follows. Now, since
    $\phi$ is constant on  $\Uc$, and continuous on  $\Uc$, it is differentiable
    on $\Uc$, whith derivative  $\phi'(x)=0$ for all $x \in U$. Moreover, since
    $\Cc$ is measurable with $m(\Cc=0)$, and $\Uc=\com{[0,1]}{\Cc}$, by excision,
    we get $m(\Uc)=1$. Finally, notice that since $\phi(0)=0$, and $\phi(1)=1$, and by
    continuity, by the intermediate value theorem,  $\phi([0,1])=[0,1]$.
\end{proof}

\begin{lemma}\label{2.5.7}
    Let $\phi$ be the Cantor-Lebesgue function and define  $\psi:[0,1]
    \xrightarrow{} \R$ by $\psi(x)=\phi(x)+x$ for all $x \in [0,1]$. Then $\psi$
    is strictly increasing, and takes  $[0,1]$ onto $[0,2]$. Moreover
    \begin{enumerate}
        \item[(1)] $\psi$ maps  $\Cc$ onto a measurable set of positive measure.

        \item[(2)] $\psi$ maps a measurable subset of  $\Cc$ onto a nonmeasurable
            set.
    \end{enumerate}
\end{lemma}
\begin{proof}
    $\psi$ is continuous since it is the sum of two continuous functions.
    Moreover, since $\phi$ is increasing and the function  $f(x)=x$ is strictly
    increasing then so is $\psi$. Notice, also, that $\psi(0)=0$ and $\psi(1)=2$
    so by the intermediate value theorem, $\phi([0,1])=[0,2]$.

    Now, since $[0,1]=\Uc \cup \Cc$ (where $\Uc$ is defined in the definiton of the
    Cantor-Lebesgue function), we have $[0,2]=\psi(\Uc) \cup \psi(\Cc)$. Since
    $[0,2]$ is measurable, and measurable sets are closed under unions, then
    $\psi(\Cc)$ is measurable; moreover, since $\psi$ is continuous and
    increasing, it has continuous inverse, and hence maps $\Cc$ to a measurable
    set  $\psi(\Cc)$. Moreover, $\psi(\Cc)$ is closed, and $\psi(\Uc)$ is open.

    Now, let $\{I_k\}$ a collection of intervals of $\Uc$, i.e.
    $\Uc=\bigcup{I_k}$. Since $\phi$ is continuous on each  $I_k$, $\psi$ takes
    $I_k$ onto translates of  $I_k$, and since  $\psi$ is  1--1, the collection
     $\{\psi(I_k)\}$ is disjoint. Therefore, by countable additivity
     \begin{equation*}
         m*(\psi(\Uc))=\sum{l(\psi(I_k))}=\sum{l(I_k+\lm)}=\sum{l(I_k)}=m(\Uc)
     \end{equation*}
     since $m(\Cc)=0$ and $m(\Uc)=1$, $m(\psi(\Uc))=1$ and $m(\psi(\Cc))=1$ as
     well.

     Finally, by Vitali's theorem, there exists a nonmeasurable set $W \subseteq
     \psi(\Cc)$, with $\inv{\psi}(W)$ measurable with $m(\inv{\psi}(W))=0$.
\end{proof}

\begin{theorem}\label{2.5.8}
    There exists a measurable subset of $\Cc$ which is not Borel.
\end{theorem}
