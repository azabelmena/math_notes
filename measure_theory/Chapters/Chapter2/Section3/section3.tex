\section{Inner and Outer Approximations}

\begin{lemma}[Excision]\label{2.3.1}
    If $A$ and  $B$ are sets, with  $A$ Lebesgue measurable of finite outer
    measure, and  $A \subseteq B$, then
    \begin{equation*}
        m^\ast(\com{B}{A})=m^\ast(B)-m^\ast(A)
    \end{equation*}
\end{lemma}

\begin{theorem}[The Outer Approximation Theorem]\label{2.3.2}
    Let $E \subseteq \R$. The following are equivalent.
    \begin{enumerate}
        \item[(1)] $E$ is Lebesgue measurable.

        \item[(2)] For all $\e>0$ there is an opne set  $U$ of  $\R$ containing
             $E$ such that  $m^\ast(\com{U}{E})<\e$.

         \item[(3)] There exists a $G_\dl$ set  $G$ containing  $E$ for which
             $m^\ast(\com{G}{E})=0$.
    \end{enumerate}
\end{theorem}
\begin{proof}
    Suppose first that $E$ is measurable and let  $\e>0$. Now, if  $m^\ast(E)$
    is finite, then there is a countable collection $\{I_k\}$ of open intervals
    covering $E$, for which, by definition of  $m^\ast$ as a greatest lower
    bound,
    \begin{equation*}
        \sum{l(I_k)}<m^\ast(E)+\e
    \end{equation*}
    Let $U=\bigcup{I_k}$, then $E \subseteq U$, and  $U$ is open in  $\R$. Thus
    by definition of  $m^\ast$ again, we have
    \begin{equation*}
        m^\ast(U) \leq \sum{l(I_k)}<m^\ast(E)+\e
    \end{equation*}
    so that $m^\ast(U)-m^\ast(E)<\e$. Now, since $E$ is measurable of finite
    outer measure, by excision, we get
    $m^\ast(\com{U}{E})=m^\ast(U)-m^\ast(E)<\e$.

    Now, if $m^\ast(E)$ is infinite, then let $\{E_k\}$ be a countable disjoint
    collection of measurable sets each of finite outer measure, and let
    $E=\bigcup{E_k}$. Then by above there exist open sets $U_k$ containing
    $E_k$, for each  $k$ such that  $m^\ast(\com{U_k}{E_k})<\frac{\e}{2^k}$. Let
    $U=\bigcup{U_k}$, then $U$ is open in  $\R$, and  $E \subseteq U$. Moreover
    observe that
    \begin{equation*}
        \com{U}{E}=\bigcup{\com{U_k}{E_k}}
    \end{equation*}
    Then we get by subadditivity
    \begin{equation*}
        m^\ast(\com{U}{E}) \leq
        \sum{m^\ast(\com{U_k}{E_k})}<\sum{\frac{\e}{2^k}}=\e
    \end{equation*}

    Now, suppose that assertion (2) holds, and choose an open set $U_k$ containing
     $E$ for which  $m^\ast(\com{U_k}{E})<\frac{1}{k}$. Define $G=\bigcup{U_k}$.
     Then $G$ is a  $G_\dl$ set, and  $E \subseteq G$. Moreovoer we have that
     \begin{equation*}
         \com{G}{E} \subseteq \com{U_k}{E} \text{ for all } k
     \end{equation*}
     so by monotonicity
     \begin{equation*}
         m^\ast(\com{G}{E}) \leq m^\ast(\com{U_k}{E})<\frac{1}{k}
     \end{equation*}
     Then as $k \xrightarrow{} \infty$, this outer measure approaches $0$.

     Now if (3) holds, since $m^\ast(\com{G}{E})=0$, the set $ \com{G}{E}$ is
     measurable. Since the space of all measurable sets is a $\s$-algebra, then
     the set $E=G \cap \com{\R}{(\com{G}{E})}$ is measurable.
\end{proof}
\begin{corollary}[The Inner Approximation Theorem]
    The following are equivalent.
    \begin{enumerate}
        \item[(1)] $E$ is Lebesgue measurable.

        \item[(2)] For all $\e>0$ there is a closed set $V$ of $\R$ contained in
             $E$ such that $m^\ast(\com{E}{V})<\e$.

         \item[(3)] There exists an $F_\s$ set $F$ contained in $E$ for which
             $m^\ast(\com{E}{F})=0$.
    \end{enumerate}
\end{corollary}
\begin{proof}
    One can apply DeMorgan's laws.
\end{proof}

\begin{theorem}\label{2.3.3}
    Let $E$ a Lebesgue measurable set of finite outer measure. then for every
    $\e>0$ there is a finite disjoint collection  $\{I_k\}$ of open intervals
    for which if $U=\bigcup{I_k}$, then
    \begin{equation*}
        m^\ast(\com{E}{U})+m^\ast(\com{U}{E})<\e
    \end{equation*}
\end{theorem}
\begin{proof}
    By the outer approximation theorem, there is an open set $V$ containing  $E$
    for which  $m^\ast(\com{V}{E})<\frac{\e}{2}$. Now, since $E$ is measurable
    of finite outer measure, by excision we have
    \begin{equation*}
        m^\ast(V)-m^\ast(E)<\frac{\e}{2}
    \end{equation*}
    so that $m^\ast(V)$ is also finite. Now, recall that every open set of real
    numbers is the disjoint collection of open intervals, hence let
    $V=\bigcup{I_k}$. Each $I_k$ is measurable with  $m^\ast(I_k)=l(I_k)$.
    Thereofre, by lemma \ref{2.2.4} and monotonicity,
    \begin{equation*}
        \sum_{k=1}^n{l(I_k)} \leq m*(V) \text{ is finite}
    \end{equation*}
    So $\sum{I_k}$ is finite. Now, choose an $n \in \Z^+$ for which
    $\sum_{k=n+1}{I_k}<\frac{\e}{2}$ and define $U=\bigcup_{k=1}^n{I_k}$. Then
    $\com{U}{E} \subseteq \com{V}{E}$ so by monotonicicty,
    $m^\ast(\com{U}{E})<\frac{\e}{2}$. Moreover, we have $\com{E}{U} \subseteq
    \com{V}{U}=\bigcup_{k=n+1}{I_k}$ so that $m^\ast(\com{E}{U})<\frac{\e}{2}$.
    Therefore, we see that
    \begin{equation*}
        m^\ast(\com{U}{E})+m^\ast(\com{E}{U})<\frac{\e}{2}+\frac{\e}{2}=\e
    \end{equation*}
\end{proof}
