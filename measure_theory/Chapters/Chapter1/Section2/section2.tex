\section{Sequences of Real Numbers}

\begin{definition}
    A sequence $\{a_n\}$ of real numbers is said to \textbf{converge} to a point
    $a$, if, for every  $\e>0$, there is an  $N>0$ such that
    \begin{equation*}
        |a-a_n|<\e \text{ whenever } n \geq N
    \end{equation*}
    We call $a$ the  \textbf{limit} of $\{a_n\}$ and write $\{a_n\} \xrightarrow{}
    a$, or
    \begin{equation*}
        \lim_{n \xrightarrow{} \infty}{\{a_n\}}=a
    \end{equation*}
\end{definition}

\begin{lemma}\label{1.2.1}
    Let $\{a_n\} \xrightarrow{} a$ a sequence of real numbers converging to $a
    \in \R$. Then the limit of  $\{a_n\}$ is unique, $\{a_n\}$ is bounded, and
    for any $c \in \R$, if  $a_n \leq c$ for all  $n$, then  $a \leq c$.
\end{lemma}

\begin{theorem}[The Monoton CVonvergence Theorem]\label{1.2.2}
    A monotone sequence of real numbers converges to a point if, and only if it
    is bounded.
\end{theorem}
\begin{proof}
    Without loss of generality, suppose that the sequence $\{a_n\}$ is
    increasing. If $\{a_n\} \xrightarrow{} a$, by lemma \ref{1.2.1}, $\{a_n\}$
    is bounded. On the otherhand, suppose that $\{a_n\}$ is bounded. Let
    $S=\{a_n : n \in \Z^+\}$, then by the completeness of $\R$, let
    $a=\sup{S}$. Let $\e>0$. Notice that  $a_n \leq a$ for all  $n$. Now, since
     $a-\e$ is not an upperbound, there exists an  $N>0$ for which  $a_N>a-\e$,
     then since  $\{a_n\}$ is increasing, $a_n>a-\e$ whenever  $n \geq N$. So we get
     \begin{equation*}
         |a-a_n|<\e \text{ whenever } n \geq N
     \end{equation*}
     Which makes $\{a_n\} \xrightarrow{} a$.
\end{proof}

\begin{theorem}[Bolzano-Weierstrass]\label{1.2.3}
    Every bounded sequence has a convergent subsequence.
\end{theorem}
\begin{proof}
    Let $\{a_n\}$ be a bounded sequence, and let $M>0$ such that  $|a_n| \geq
    M$ for all  $n \in \Z^+$. Define  $E_n=\cl{\{a_j : j \geq n\}}$. Then $EE
    \subseteq [-M,M]$. Thus $\{E_n\}$ is a decreasing sequence of closed,
    bounded, and nonempty sets of $\R$. By the nested set theorem, the
    intersection $E=\bigcap{E_n}$ is nonempty. Choose an $a \in E$. THen for
    every  $k \in \Z^+$,  $a$ is a point of closure of the set  $\{a_j : j \geq
    k\}$. SO that $a_j \in (a-\frac{1}{k}, a+\frac{1}{k})$ whenever $j \geq k$.
    By induction, construct a strictly increasing sequence  $\{n_k\}$ of natural
    numbers for which $|a-a_{n_k}|<\e$. Thenb by the principle of Archimedes,
    $\{a_{n_k}\} \xrightarrow{} a$, and we have a convergent subsequence.
\end{proof}

\begin{definition}
    We call a sequence $\{a_n\}$ \textbf{Cauchy} if for every $\e>0$, there is
    an  $N>0$ for which
    \begin{equation*}
        |a_m-a_n|<\e \text{ whenever } m,n \geq N
    \end{equation*}
\end{definition}

\begin{theorem}[The Cauchy Convergence Criterion]\label{1.2.4}
    A sequence of real numbers converges if, and only if it is Cauchy.
\end{theorem}
\begin{proof}
    Suppose that the sequence $\{a_n\} \xrightarrow{} a$ converges to $a \in
    \R$. Then for any  $m,n \in \Z^+$, notice that  $|a_m-a_n| \leq
    |a_m-a|+|a-a_n|$. Let  $\e>0$ and choose  $N>0$ such that
    $|a-a_n|<\frac{\e}{2}$, and $|a_m-a|<\frac{\e}{2}$. Then if $n,m \geq N$, we
    get  $|a_m-a_n|<\e$, which makes  $\{a_n\}$ Cauchy.

    Conversely, suppose that $\{a_n\}$ is Cauchy. Let $\e=1$ and choose  $N>0$
    such that if $m,n \geq N$, then $|a_m-a_n|<1$. Then we get $|a_n| \leq
    1+|a_N|$ for all  $n \geq N$. Define  $M=1+\max{\{|a_1|, \dots, |a_N|\}}$.
    Then $|a_n| \leq M$ for all  $n$. This makes  $\{a_n\}$ bounded. By the
    theorem of Bolzano-Weierstrass, $\{a_n\}$ has a convergent subsequence
    $\{a_{n_k}\} \xrightarrow{} a$. Let $\e>0$, since  $\{a_n\}$ is Cauchy,
    choose an $N>0$ such taht $|a_m-a_n|<\frac{\e}{2}$ whenever $n,m \geq N$.
    Likewise, we get  $|a-a_{n_k}|<\frac{\e}{2}$ and $n_k \geq N$. Thus we
    observe that  $|a_n-a| \leq |a_n-a_{n_k}|+|a-a_{n_k}|<\e$ and so $\{a_n\}
    \xrightarrow{} a$.
\end{proof}

\begin{theorem}\label{1.2.5}
    Let $\{a_n\} \xrightarrow{} a$ and $\{b_n\} \xrightarrow{} b$ be convergent
    sequences. Then for any $\al,\bt \in \R$, we have that the sequence $\{\al
        a_n+\bt b_n\}$ converges and that
    \begin{equation*}
        \lim_{n \xrightarrow{} \infty}{\{\al a_n+\bt b_n\}}=\al a+\bt b
    \end{equation*}
\end{theorem}

\begin{definition}
    We say a sequence $\{a_n\}$ of real numbers \textbf{converges to infinity}
    $\infty \in \R_\infty$ if for every  $c \in \R$, there is an  $N>0$ such
    that  $a_n \geq c$ whenver  $n \geq N$. We write  $\{a_n\} \xrightarrow{}
    \infty$, or
    \begin{equation*}
        \lim_{n \xrightarrow{} \infty}{\{a_n\}}=\infty
    \end{equation*}
\end{definition}

\begin{definition}
    Let $\{a_n\}$ be a sequence of real numbers. We define the \textbf{limit
    superior} of $\{a_n\}$ to be
    \begin{equation*}
        \limsup{{\{a_n\}}=\lim_{n \xrightarrow{} \infty}{(\sup{\{a_k : k \geq
        n\}})}
    \end{equation*}
    Similarly, we define the \textbf{limit inferiro} of $\{a_n\}$ to be
    \begin{equation*}
        \liminf{{\{a_n\}}=\lim_{n \xrightarrow{} \infty}{(\inf{\{a_k : k \geq
        n\}})}
    \end{equation*}
\end{definition}

\begin{theorem}\label{1.2.6}
    For any sequences $\{a_n\}$ and $\{b_n\}$ of real numbers, the following are
    true:
    \begin{enumerate}
        \item[(1)] $\limsup{\{a_n\}}=l \in \R_\infty$ if, and only if for every
            $\e>0$, there exists infintiely many  $n \in \Z^+$ such that
            $a_n>l-\e$ and finitely many  $n \in \Z^+$ for which  $a_n>l+\e$.

        \item[(2)] $\limsup{\{a_n\}}=\infty$ if, and only if $\{a_n\}$ is not
            bounded above.

        \item[(3)] $\limsup{\{a_n\}}=-\liminf{\{-a_n\}}$

        \item[(4)] $\{a_n\} \xrightarrow{} a \in \R_\infty$ if, and only if
            $\limsup{\{a_n\}}=\liminf{\{a_n\}}$.

        \item[(5)] If $a_n \leq b_n$ for all  $n$, then  $\limsup{\{a_n\}} \leq
            \limsup{\{b_n\}}$.
    \end{enumerate}
\end{theorem}

\begin{definition}
    Let $\{a_n\}$ a sequence of real numbers. We call the series
    $\sum_{k=1}^\infty{a_k}$ \textbf{summable} if the sequence of partial sums
    $\{s_n=\sum_{k=1}^n{a_k}\} \xrightarrow{} s$ converges to a point $s \in
    \R$.
\end{definition}

\begin{lemma}\label{1.2.7}
    Let $\{a_n\}$ a sequence of real numbers. Then the following are true.
    \begin{enumerate}
        \item[(1)] The series $\sum{a_k}$ is summable if, and only if for every
            $\e>0$, there is an  $N>0$ such that
            \begin{equation*}
                |\sum_{k=n}^{n+m}{a_k}|<\epsilon \text{ for all } m \in \Z^+
                \text{ whenever } n \geq N
            \end{equation*}

        \item[(2)] If $\sum{|a_k|}$ is summable, then so is $\sum{a_k}$.

        \item[(3)] If $a_k \geq 0$, then  $\sum{a_k}$ is summable if, and only
            if the sequence of partial sums $\{s_n\}$ is bounded.
    \end{enumerate}
\end{lemma}
