%----------------------------------------------------------------------------------------
%	SECTION 1.4
%----------------------------------------------------------------------------------------

\section{The Complex Field}

\begin{definition}
    We define a \textbf{complex number} to be a pair of real numbers  $(a,b)$. We denote the 
    set of all comlex numbers by $\C$. We define the \textbf{addition} and $\textbf{multiplication}$ of 
    complex numbers to be the binary operations $+:\C \rightarrow \C$ and $\cdot: \C \rightarrow \C$ such that 
        \begin{align*}
            (a,b)+(c,d) &= (a+c,b+d) \\
            (a,b)(c,d) &= (ac-bd,ad+bc) \\
        \end{align*}
        Lastly, we define $i$ to be the complex number such that  $i=(0,1)$.
\end{definition}

\begin{theorem}\label{1.4.1}
    $\C$ forms a field together with  $+$ and  $\cdot$.
\end{theorem}

\begin{theorem}\label{1.4.2}
    For $(a,0),(b,0) \in C$,  $(a,0)+(b,0)=(a+b,0)$, and $(a,0)(b,0)=(ab,o)$.
\end{theorem}
\begin{proof}
    This is a straightforward application of the addition and multiplication of 
    complex numbers.
\end{proof}

\begin{theorem}\label{1.4.3}
    $i^2=-1$.
\end{theorem}
\begin{proof}
    $i^2=(0,1)(0,1)=(0-1,1-1)=(-1,0)=-1$.		
\end{proof}

\begin{theorem}\label{1.4.4}
    Let $(a,b) \in \C$, then  $(a+b)=a+ib$.
\end{theorem}
\begin{proof}
    $(a,b)=(a,0)+(0,b)=(a,0)+(0,1)(b,0)=a+ib$.	
\end{proof}

\begin{definition}
    Let $a,b \in \R$, and let  $z \in \C$ such that  $z=a+ib$. We define the  \textbf{complex 
    conjugate} of $z$ to be the complex number  $\bar{z}=a-ib$. Moreover, we define the 
    \textbf{real part} of  $z$ to be $a$, and the \textbf{imaginary part} of  $z$ to be $b$, 
    and we denote them $a=\re{z}$,  $b=\im{z}$
\end{definition}

\begin{theorem}\label{1.4.5}
    Let $z, w \in \C$. Then 
        \begin{enumerate}[label=(\arabic*)]
            \item $\overline{z+w}=\bar{z}+\bar{w}$.

            \item $\overline{zw}=\bar{z}\bar{w}$.

            \item $z+\bar{z}=2\re{z}$ and  $z-\bar{z}=2i\im{z}$.

            \item $z\bar{z}$ is a nonegative real number.
        \end{enumerate}
\end{theorem}
\begin{proof}
    Let $z=a+ib$, and let  $w=c+id$. Then $z+w=(a+c)+i(b+d$, so  $\overline{z+w}=(a+b)
    -i(b+d)=(a-ib)+(c-id)=\bar{z}+\bar{w}$; similarly, we get  $\overline{zw}=\bar{z}\bar{w}$. 
    Moreover, we have  $(a+ib)+(a-ib)=2a$, and  $(a+ib)-(a-ib)=2ib$, we also have that 
    $z\bar{z}=(a+ib)(a-ib)=a^2+b^2 \geq 0$, and $z\bar{z}=0$ if and only if $a=b=0$.
\end{proof}

\begin{definition}
    Let $z \in \C$. We define the  \textbf{modulus} of  $z$ to be  $|z|=\sqrt{z\bar{z}}$.
\end{definition}
\begin{remark}
    $|z|$ exists and is unique.
\end{remark}

\begin{theorem}\label{1.4.6}
    Let $z,w \in \C$, then:
        \begin{enumerate}[label=(\arabic*)]
            \item $|z| \geq 0$ and  $|z|=0$ if and only if $z=0$.

            \item  $|\bar{z}|=|z|$.

            \item $|zw|=|z||w|$.

            \item  $\re{z} \leq |z|$.

            \item  $|z+w+ \leq |z|+|w|$.
        \end{enumerate}
\end{theorem}
\begin{proof}
    Let $z=a+ib$, and  $w=c+id$. Then  $|z|=\sqrt{a^2+b^2} \geq 0$, and  $|z|=0$ if 
    and only if  $a,b=0$. Moreover,  $|\bar{z}|=|a+i(-b)|=\sqrt{a^2+(-b)^2}=\sqrt{a^2+b^2}=|z|$. 
    We also habe $|zw|^2=(a^2+b^2)(c^2+d^)=|z|^2|w|^2$, likewise,  $\|re{z}|=|a+i0|=
    \sqrt{a^2} \leq \sqrt{a^2+b^2}$. Finally we prove $(5)$.

    We have  $|z+w|^2=(x+w)(\bar{z}+\bar{w})=z\bar{z}+\bar{z}w+\bar{w}z+w\bar{w}= 
    |z|^2+w\re{z\bar{w}}+|w|^2 \leq |z|^2+2|s\bar{w}|+|w|^2=(|z|+|w|)^2$.
\end{proof}

\begin{theorem}[The Cauchy Schwarz Inequality]\label{1.4.7}
    Let $a_i,b_i \in \C$, for  $1 \leq i \leq n$. Then:
         \begin{equation}
             |\sum_{i=1}^{n}{a_i\bar{b_i}}| \leq \sum_{i = 1}^{n}{|a_i|^2}\sum_{i=1}^{n}{|b_j|^2}		
        \end{equation}
\end{theorem}
\begin{proof}
    Let $A=\sum{a_j|^2}$,  $B=\sum{|b_i|^2}$, and $C=\sum{a_i\bar{b_i}}$. If  $B=0$, then 
     $b_i=0$ for  $1 \leq i \leq n$, and we are done; so suppose that  $B>0$. Then
        \begin{align*}
            \sum{|Ba_j-Cb_j|^2} &= \sum{(Ba_j-Cb_j)(B\bar{a_j}-\bar{Cb_j})} \\
                             &= B\sum{|a_j|^2}-B\bar{C}\sum{a_j\bar{b_j}-BC\sum{\bar{a_j}b_j}}+|C^2|\sum{|b_j|^2} \\
                             &= (B^2A-B|C|^2)=B(AB-|C|^2) \geq 0 \\
        \end{align*}
    Since $B>0$, we get  $|C|^2 \leq AB$ as required.
\end{proof}
