\section{Measures}

\begin{definition}
    Let $X$ be a set together with a $\s$-algebra $\Mc$. We define a
    \textbf{measure} on $\Mc$ to be a function  $\m:\Mc \xrightarrow{} [0,
    \infty)$ for which the following hold:
    \begin{enumerate}
        \item[(1)] $m(\emptyset)=0$.

        \item[(2)] If $\{E_n\}$ is a countable disjoint collection of members of
            $\Mc$, then
            \begin{equation}\label{equation_1.2}
                m\Big{(} \bigcup{E_n} \Big{)}=\sum{m(E_n)}
            \end{equation}
    \end{enumerate}
    We call $m$ a \textbf{finitely additive measure} if instead of (2), $m$
    satisfies:
    \begin{enumerate}
        \item[(2')] If $\{E_i\}_{i=1}^n$ is a finite disjoint collection of
            members of $\Mc$, then
            \begin{equation}\label{equation_1.2}
                m\Big{(} \bigcup_{i=1}^n{E_i} \Big{)}=\sum_{i=1}^n{m(E_i)}
            \end{equation}
    \end{enumerate}
\end{definition}

\begin{definition}
    We call a set $X$ together with a $\s$-algebra $\Mc$ a \textbf{measurable
    space}, and we call the members of $\Mc$ \textbf{measurable sets}. If $m:\Mc
    \xrightarrow{} [0, \infty)$ is a measure on $\Mc$, then we call  $X$
    together with $\Mc$ a \textbf{measure space}.
\end{definition}

\begin{definition}
    Let $X$ together with a $\s$-algebra be a measure space with measure $m$. If
     $m(X)<\infty$, then we call $m$ a \textbf{finite measure}, and if $\{E_n\}$
     is a covering of $X$ by measurable sets, each with  $m(E_n)<\infty$ for all
     $n$, then we call $m$  \textbf{$\s$-finite}. We also call the set
     $E=\bigcup{E_n}$ \textbf{$\s$-finite}. We call $m$ \textbf{semi-finite} if
     for any measurable set $E$, of $m(E)=\infty$, there is a measurable set $F$
     contained in  $E$ such that  $0<m(F)<\infty$.
\end{definition}

\begin{lemma}\label{lemma_1.2.1}
    $\s$-finite measures are semi-finite.
\end{lemma}

\begin{example}\label{example_1.2}
    \begin{enumerate}
        \item[(1)] LEt $X$ be a non-empty set, and let $f:X \xrightarrow{}
            [0,\infty)$ be any function on $X$. Then $f$ defines a measure $m$
            on  $2^X$ by the rule
            \begin{equation*}
                m(E)=\sum_{x \in E}{f(x)}
            \end{equation*}
        Now, $m$ is semi-finite if, and only if $f(x)<\infty$ for all $x \in  X$,
        and $m$ is  $\s$-finite if, and only if $m$ is semi-finite, and the
        pre-image $\inv{f}((0,\infty))$ is countable.

        \item[(2)] Consider the measure $m$ of example (1) above, where $f(x)=1$
            for all $x \in X$. Then we call $m$ the  \textbf{counting measure}
            on $2^X$. Indeed, observe that
            \begin{equation*}
                m(E)=\sum_{x \in E}{1}=|E|
            \end{equation*}
            which counts the elements of $E$.

        \item[(3)] Consider the measure $m$ of example (1) above, where $f$ is
            defined for any  $x_0 \in X$ to be:
            \begin{equation*}
                f(x)=\begin{cases}
                        1, \text{ if } x=x_0 \\
                        0, \text{ if } x \neq x_0 \\
                     \end{cases}
            \end{equation*}
            We call this measure the \textbf{Dirichlet measure}.

        \item[(4)] Let $X$ be an uncountable set, and let $\Mc$ the $\s$-algebra
            of all countable or co-countable sets. Define  $m$ on  $\Mc$ by
            $m(E)=0$ if $E$ is countable, and $m(E)=1$ if $E$ is co-countable.
            Then $m$ defines a measure on $\Mc$.

        \item[(5)] Let $X$ be an infinite set, and define $m$ on $2^X$ by
            $m(E)=0$ if $E$ is finite, and $m(E)=\infty$ if $E$ is infinite.
            Then $m$ is a finitely subadditive measure on  $2^X$, but not a
            measure on  $2^X$.
    \end{enumerate}
\end{example}

\begin{theorem}\label{theorem_1.2.2}
    Let $X$ be a measure space with measure $m$. The following are true.
    \begin{enumerate}
        \item[(1)] If $E$ and $F$ are measurable with $E \subseteq F$, then
            \begin{equation*}
                m(E) \leq m(F)
            \end{equation*}

        \item[(2)] If $\{E_n\}$ is a countable collection of measurable sets,
            then
            \begin{equation*}
                m\Big{(} \bigcup{E_n} \Big{)} \leq \sum{m(E_n)}
            \end{equation*}

        \item[(3)] If $\{E_n\}$ is a countable collection of measurable sets, in
            which $E_1 \subseteq E_2 \subseteq \dots$, then
            \begin{equation*}
                m\Big{(} \bigcup{E_n} \Big{)}=\lim_{n \xrightarrow{}
                \infty}{m(E_n)}
            \end{equation*}

        \item[(4)] If $\{E_n\}$ is a countable collection of measurable sets, in
            which $\dots, \subseteq E_2 \subseteq E_1$ and $m(E_1)<\infty$, then
            \begin{equation*}
                m\Big{(} \bigcap{E_n} \Big{)}=\lim_{n \xrightarrow{}
                \infty}{m(E_n)}
            \end{equation*}
    \end{enumerate}
\end{theorem}
\begin{proof}
    For the first statement, let $E \subseteq F$ be measurable sets, then
    observe that
    \begin{equation*}
        m(E) \leq m(E)+m(\com{F}{E})=m(E \cup \com{F}{E})=m(F)
    \end{equation*}
    For the second statement, define $F_1=E_1$, and
    $F_i=\com{E_i}{\bigcup_{i=1}^{i-1}{E_i}}$ for all $i>1$. Then $\{F_n\}$ is a
    finite disjoint collection of measurable sets, with $\bigcup_{i=1}^n{F_i}
    =\bigcup_{i=1}^n{E_i}$. By the above argument, we get
    \begin{equation*}
        m\Big{(} \bigcup_{i=1}^n{E_i} \Big{)}=m\Big{(} \bigcup_{i=1}^n{F_i} \Big{)}
        =\sum_{i=1}^{n}{m(F_i)} \leq \sum_{i=1}^n{m(E_i)}
    \end{equation*}

    Now, for (3), let $E_0=\emptyset$, then
    \begin{equation*}
            m(\bigcup{E_n})=\sum{m(\com{E_i}{E_{i-1}})}=
            \lim_{n \xrightarrow{} \infty}{\sum_{i=1}^n{m(\com{E_i}{E_{i-1}})}}=
            \lim_{n \xrightarrow{} \infty}{m(E_n)}
    \end{equation*}
    Additionally, consider when the collection $\{E_n\}$ is decreasing with
    $m(E_1)<\infty$. Take $F_i=\com{E_1}{E_i}$, then $\{F_n\}$ is an increasing
    collection of measurable sets, and hence we apply the above argument. We get
    that $m(E_1)=m(F_n)+m(E_n)$, and
    \begin{equation*}
        \bigcup{F_n}=\com{E_1}{\bigcap{E_n}}
    \end{equation*}
    therefore, we get
    \begin{equation*}
        m(E_1)=m\Big{(} \bigcap{E_n} \Big{)}+\lim_{n \xrightarrow{}
        \infty}{m(F_i)}=m\Big{(} \bigcap{E_n} \Big{)}+\lim_{n \xrightarrow{}
        \infty}{(m(E_1)-m(E_n))}
    \end{equation*}
    Subtracting $m(E_1)$ from both sides of the equation yields the result.
\end{proof}

\begin{definition}
    Let $X$ be a measure space with measure $m$. We say that a statement about
    points in $X$ holds  \textbf{almost everywhere} (with respect to $m$) if it
    holds for all $x \in \com{X}{E}$, where $m(E)=0$. We call the measure $m$
     \textbf{complete} if its domain contains all subsets of sets with measure
     $0$.
\end{definition}

\begin{theorem}
    Let $X$ be a measure space with $s$-algebra $\Mc$, and measure $m$. Let
    $\Nc=\{N \in \Mc : m(N)=0\}$, and define
    \begin{equation*}
        \bar{\Mc}=\{E \cup F : E \in \Mc \text{ and } F \subseteq N \text{ for
        some } N \in \Nc\}
    \end{equation*}
    Then $\bar{\Mc}$ is a $\s$-algebra, and there exists a unique complete
    measure $\bar{m}$ on $\bar{\Mc}$.
\end{theorem}
\begin{proof}
    Since $\Mc$ is a  $\s$-algebra, then so is $\Nc$, moreover, since both are
    closed under countable unions, so is $\bar{\Mc}$. Additionally, let $E \cup
    F \in \bar{\Mc}$, then we get $E \cup F=(E \cup N) \cap ((\com{X}{N}) \cup
    F)$, so that $\com{X}{(E \cup F)}=\com{X}{(E \cup N)} \cup \com{N}{F}$.
    Since $\com{X}{(E \cup N)} \in \bar{\Mc}$, and $\com{N}{F} \subseteq F$,
    then we get $\com{X}{(E \cup F)} \subseteq \bar{\Mc}$. This makes $\Mc$ a
    $\s$-algebra.

    Now, for  $E \cup F \in \bar{\Mc}$, define $\bar{m}$ on $\bar{\Mc}$ by
    $\bar{m}(E \cup F)=m(E)$. Then $\bar{m}$ is well defined. Let $E_1 \cup
    F_1=E_2 \cup F_2$, where $F_i \subseteq N_i$, with $N_i \in \Nc$, for
    $i=1,2$. Then  $E_1 \subseteq E_2 \cup N_2$, so that $m(E_1) \leq
    m(E_2)+m(N_1)=m(E_2)$. Similarly, we also get $m(E_2) \leq m(E_1)$.

    Now, let $E \in \bar{\Mc}$, such that $\bar{m}(E)=0$. Now, we have $E=A \cup
    B$, where  $A \in \Mc$ and $B \subseteq N$, for some $N \in \Nc$. Moreover,
    $\bar{m}(E)=m(A)=0$. Now, we get $E \subseteq A \cup N \in \Nc$, since
    $m(A)=0$. Now, let $F \subseteq E$. Then observe that $F \subseteq A \cup
    N$, so that $F \in \Nc$. Then $F=\emptyset \cup F$, so that  $F \in
    \bar{\Mc}$. Moreover, $\bar{m}(F)=m(\emptyset)=0$.

    Lastly, suppose there is another complete meaure $\bar{n}$ on $\bar{\Mc}$
    for which $\bar{n}(E \cup F)=m(E)$. Let $E \in \bar{\Mc}$. Then $E=A \cup B$
    where  $A \in \Mc$, and $B \subseteq N$, $N \inn \Nc$. Then
    $\bar{n}(E)=\bar{n}(A \cup B)=m(A) \leq m(A)+m(B)=m(A \cup B)=\bar{m}(E)$.
    By similar reasoning, we get $\bar{m}(E) \leq \bar{n}(E)$, which establishes
    uniqueness.
\end{proof}

\begin{definition}
    Let $X$ be a measure space with $s$-algebra $\Mc$, and measure $m$. Let
    $\Nc=\{N \in \Mc : m(N)=0\}$, and define
    \begin{equation*}
        \bar{\Mc}=\{E \cup F : E \in \Mc \text{ and } F \subseteq N \text{ for
        some } N \in \Nc\}
    \end{equation*}
    We call \bar{\Mc} the \textbf{completion} of $\Mc$ with respect to $m$, and
    we call the unique complete measure, $\bar{m}$ on $\bar{\Mc}$ the
    \textbf{completion} of $m$.
\end{definition}
