\section{Outer Measures}

\begin{definition}
    Let $X$ be a set. An \textbf{outer-measure} on $X$ is a function $m^\ast:2^X
    \xrightarrow{} [0, \infty)$ for which the following are true:
    \begin{enumerate}
        \item[(1)] $m^\ast(\emptyset)=0$.

        \item[(2)] If $A \subseteq B$, then $m^\ast(A) \leq m^\ast(B)$.

        \item[(3)] If $\{A_n\}$ is a countable collection of subsets of $X$,
            then
            \begin{equation*}
                m^\ast\Big{(} \bigcup{A_n} \Big{)} \leq \sum{m^\ast(A_n)}
            \end{equation*}
    \end{enumerate}
\end{definition}

\begin{lemma}\label{lemma_1.3.1}
    Let $X$ be a set, and  $\Ec$ a collection of subsets of $X$ for which
    $\emtyset \in \Ec$ and $X \in \Ec$, and let $l:\Ec \xrightarrow{} [0, \infty]$
    a function for which $l(\emptyset)=0$. For any $A \subseteq X$, define
    \begin{equation}\label{equation_1.4}
        m^\ast(A)=\inf{\Big{\{} \sum{l(E_n)} : E_n \in \Ec, \text{ and }
       A \subseteq \bigcup{E_n} \Big{\}}}
    \end{equation}
    Then $m^\ast$ defines an outer-measure.
\end{lemma}
\begin{proof}
    For all $A \subseteq X$, there is a collection $\{E_n\}$ of sets of $\Ec$
    for which $A \subseteq \bigcup{E_n}$. Observe first, that since $l(E_n) \geq
    0$ for all $n$, that  $\sum{l(E_n)} \geq 0$. This makes $m^\ast(A) \geq 0$.
    Now, choose $E_n=\emptyset$ for all  $n$, and we get  $m^\ast(\emptyset)=0$.

    Now, let $A \subseteq B$ subsets of $X$, and let  $\{E_n\}$ a countable
    cover of $B$. Then  $\{E_n\}$ is also a countable cover of $A$. Define then
     $E=\{\sum{l(E_n)} : A \subseteq \bigcup{E_n}\}$ and $F=\{\sum{l(E_n)} :
     B \subseteq \bigcup{E_n}\}$. Since $A \subseteq B$, $F \subseteq E$.
     Therefore, by least upper bounds, we have $\inf{F} \leq \inf{B}$, that is
     $m^\ast(A) \leq m^\ast(B)$.

     Lastly, let $\{A_n\}$ be a countable collection of sets of $X$, and let
     $A=\bigcup{A_n}$. Now, if at least one of the $m(A_n)=\infty$, then we are
     done. Suppose then that $m(A_n)<\infty$ for all $n$. Now, there exists a
     cover of  $A_n$,  $\{E_{n,k}\}_k$ for which
     \begin{equation*}
         \sum_{k}{l(E_{n,k})}<m^\ast(A)+\frac{1}{2^k}
     \end{equation*}
     consider now the countable collection
     $\{E_{n,k}\}_{n,k}=\bigcup_{n}{\{E_{n,k}\}_k}$. Then $\{E_{n,k}\}_{n,k}$ is
     a countable cover for $A$, and we get
     \begin{equation*}
         m^\ast(A) \leq \sum_n{\sum_k{l(E_{n,k})}}<
         \sum_n{m^\ast(A_n)+\frac{1}{2^k}}=\sum_n{m^\ast(A_n)}+\e
     \end{equation*}
     Take then $\e>0$ small, and we get the result.
\end{proof}
\begin{corollary}
    If $E$ is a set of $\Ec$, then $m^\ast(E)=l(E)$.
\end{corollary}
\begin{proof}
    Observe that $E$ covers itself, so that
    $m^\ast(E)=\inf{\{\sum_{i=1}^1{E}\}}=\inf{l(E)}=l(E)$.
\end{proof}

\begin{definition}
    Let $X$ be a set. We call a subset $A$ of  $X$  \textbf{$m^\ast$-measurable}
    if for any subset $E$ of  $X$,
    \begin{equation}\label{equation_1.5}
        m^\ast(E)=m^\ast(E \cap A)+m^\ast(E \cap \com{X}{A})
    \end{equation}
\end{definition}

\begin{lemma}\label{lemma_1.3.2}
    Let $X$ be a set. A subset $A$ of $X$ is  $m^\ast$-measurable if, and only
    if
    \begin{equation*}
        m^\ast(E) \geq m^\ast(E \cap A)+m^\ast(E \cap \com{X}{A}) \text{ for all }
        E \subseteq X
    \end{equation*}
\end{lemma}

\begin{theorem}[Carath\'eodory's Theorem]\label{theorem_1.3.3}
    Let $X$ be a set, and $m^\ast$ an outer-measure on $X$. Then the collection
    of all $m^\ast$-measurable sets forms a  $\s$-algebra. Moreover,  $m^\ast$
    is a complete measure on this $\s$-algebra.
\end{theorem}
\begin{proof}
    Let $\Mc$ be the collection of all  $m^\ast$-measurable sets. Observe first
    that if  $A \in \Mc$, then so is $\com{X}{A}$, by symetry of equation
    \ref{equation_1.5}. So $\Mc$ is closed under complements. Now, let  $A,B \in
    \Mc$. Then we have
    \begin{align*}
        m^\ast(E) &= m^\ast(E \cap A)+m^\ast(E \cap \com{X}{A}) \\
               &= m^\ast(E \cap A \cap B)+m^\ast(E \cap A \cap \com{X}{B})
                        +m^\ast(E \cap B \cap \com{X}{A})+m^\ast(E \cap
                        \com{X}{A} \cap \com{X}{B}) \\
    \end{align*}
    Now, since $A \cup B=(A \cap B) \cup (A \cap \com{X}{B}) \cup (B \cap
    \com{X}{A})$, so by subadditivity, we get
    \begin{equation*}
        m^\ast(E \cap A \cap B)+m^\ast(E \cap A \cap \com{X}{B})+m^\ast(E \cap
        \com{X}{A} \cap B) \geq m^\ast(E \cap (A \cup B))
    \end{equation*}
    i.e. $m^\ast(E) \geq m^\ast(E \cap (A \cup B))+m^\ast(E \cap \com{X}{(A \cup
    B)})$. That is, $A \cup B \in \Mc$, making $\Mc$ an algebra.

    Now, let  $\{A_n\}$ be a countable disjoint collection of
    $m^\ast$-measurable sets, and take  $B_n=\bigcup_{i=1}^n{A_i}$, and take
    $B=\bigcup{B_n}$. Then for all $E \subseteq X$
    \begin{align*}
        m^\ast(E \cap B_n) &= m^\ast(E \cap B_n \cap A_n)+m^\ast(E \cap B_n \cap
                        \com{X}{A_n}) \\
                     &= m^\ast(E \ca A_n)+m^\ast(E \cap B_{n-1})
    \end{align*}
    an induction argument on the collection $\{B_n\}$ gives us
    \begin{equation*}
        m^\ast(E \cap B_n)=\sum_{i=1}^n{m^\ast(E \cap A_i)}
    \end{equation*}
    therefore
    \begin{equation*}
        m^\ast(E)=m^\ast(E \cap B_n)+m^\ast(E \cap \com{X}{B_n}) \geq
        \sum_{i=1}^n{m^\at(E \cap A_i)}+m^\ast(E \cap \com{X}{B_n})
    \end{equation*}
    letting $n \xrightarrow{} \infty$,
    \begin{equation*}
        m^\ast(E) \geq \sum{m^\at(E \cap A_n)}+m^\ast(E \cap \com{X}{B_n})
    \end{equation*}
    so that $B \in \Mc$. Taking $E=B$, we get  $m^\ast(B)=\sum{m^\ast(A_n)}$ so
    that $m^\ast$ is countably additive, and $\Mc$ is a $\s$-algebra.

    Finally, let $m^\ast(A)=0$, then for any $E \subseteq X$, we have
    \begin{equation*}
        m^\ast(E) \leq m^\ast(E \cap A)+m^\ast(E \cap \com{X}{A})=
        m^\ast(E \cap \com{X}{A}) \leq m^\ast(E)
    \end{equation*}
    so that $A \in \Mc$, which makes $m^\ast$ complete on $\Mc$.
\end{proof}

\begin{definition}
    Let $X$ be a set, and  $\Ac$ an algebra on  $X$. We define a
    \textbf{pre-measure} on $\Ac$ to be a function  $m_0:\Ac \xrightarrow{}
    [0,\infty]$ for which
    \begin{enumerate}
        \item[(1)] $m_0(\emptyset)=0$.

        \item[(2)] If $\{A_n\}$ is a countably disjoint collection of sets in
            $\Ac$, for which  $\bigcup{A_n} \in \Ac$, then
            \begin{equation}\label{equation_1.6}
                m_0\Big{(} \bigcup{A_n} \Big{)}=\sum{m_0(A_n)}
            \end{equation}
    \end{enumerate}
\end{definition}

\begin{lemma}\label{lemma_1.3.4}
    Pre-measures on algebras define outer-measures on the overlying sets.
\end{lemma}
\begin{proof}
    Consider the definition of the outer measure $m^\ast$ from equation
    \ref{equation_1.4}, simply take $l=m_0$, and $\Ec=\Ac$.
\end{proof}

\begin{lemma}\label{lemma_1.3.5}
    Let $X$ be a set, and  $\Ac$ an algebra on  $X$. If  $m_0$ is pre-measure on
    $\Ac$, and the measure $m^\ast$ is define by
    \begin{equation*}
        m^\ast(A)=\inf{\Big{\{} \sum{m_0(E_n)} : E_n \in \Ac, \text{ and }
       A \subseteq \bigcup{E_n} \Big{\}}}
    \end{equation*}
    then the following are true.
    \begin{enumerate}
        \item[(1)] $m_0=m^\ast$ on $\Ac$.

        \item[(2)] Every set in $\Ac$ is  $m^\ast$-measurable.
    \end{enumerate}
\end{lemma}
\begin{proof}
    For (1), suppose that $A \in \Ac$, and that $A \subseteq \bigcup{E_n}$ for
    $E_n \in \Ac$. Take
    \begin{equation*}
        F_n=A \cap \com{A_n}{\Big{(} \bigcup_{i=1}^{n-1}{A_i} \Big{)}}
    \end{equation*}
    then $\{F_n\}$ is a disjoint countable collection of sets of $\Ac$ for which
     $A=\bigcup{F_n}$. Hence
     \begin{equation*}
         mo(A)=\sum{m_0(F_n)} \leq \sum{m_0(E_n)}
     \end{equation*}
     it follows from hypothesis that $m_0(A) \leq m^\ast(E)$. For the reverse
     inclusion, simply take $A \subseteq \bigcup{E_n}$ with $A=E_1$ and
     $E_n=\emptyset$ for all  $n>1$.

     For (2), if $A \in \Ac$, and $E \subseteq X$, and $\e>0$, there is a
     collection  $\{B_n\}$ of sets of $\Ac$ with  $A \subseteq \bigcup{B_n}$,
     and
     \begin{equation*}
         \sum{m_0(B_n)}<m^\ast(A)+\e
     \end{equation*}
     by additivity of $m_0$ on $\Ac$, we get
     \begin{equation*}
         m^\ast(E)+\e \geq \sum{m_0(B_n \cap A)}+\sum{m_0(B_n \cap \com{X}{A})}
                    \geq m^\ast(E \cap A)+m^\ast(E \cap \com{X}{A})
     \end{equation*}
\end{proof}

\begin{theorem}\label{theorem_1.3.6}
    Let $X$ be a set, and  $\Ac$ an algerba on  $X$. Let  $m_0$ be a pre-measure
    on $\Ac$, and let  $\Mc$ the  $\s$-algebra generated by  $\Ac$. Then there
    exists a measure  $m$ on  $\Mc$ whose restriction to  $\AC$ is  $m_0$.
    Moreover, if $n$ is another measure extending from $m_0$, then
    \begin{equation*}
        n(E) \leq m(E) \text{ for all } E \in \Mc
    \end{equation*}
    where equality holds when $m(E)<\infty$. Lastly, if $m_0$ is $\s$-finite,
    then  $m$ is the unique extension of  $m_0$ to $\Mc$.
\end{theorem}
\begin{proof}
    Define again,
    \begin{equation*}
        m^\ast(A)=\inf{\Big{\{} \sum{m_0(E_k)} : E_k \in \Ac, \text{ and }
       A \subseteq \bigcup{E_k} \Big{\}}}
    \end{equation*}
    then by Carath\'eodory's theorem, lemma \ref{lemma_1.3.5}, the first result
    follows, since the $\s$-algebra of all  $m^\ast$-measurable sets contains
    $\Ac$, and as consequence, also contains $\Mc$.

    Now, let $E \in \Mc$ with $E \subseteq \bigcup{A_k}$, where $A_k \in \Ac$.
    Then
    \begin{equation*}
        n(E) \leq \sum{n(A_n)}=\sum{m_0(A_n)}
    \end{equation*}
    which gives us $n(E) \leq m(E)$. Now, set $A=\bigcup{A_n}$, and observe that
    \begin{equation*}
        n(A)=\lim_{k \xrightarrow{} \infty}{n\Big{(} \bigcup_{i=1}^k{A_i} \Big{)}}
        =\lim_{k \xrightarrow{} \infty}{m\Big{(} \bigcup_{i=1}^k{A_i}
        \Big{)}}=m(A)
    \end{equation*}
    if $m(E)<\infty$, choose $A_k$ such that  $m(A)<m(E)+\e$ for $\e>0$. Then
    $m(\com{A}{E})<\e$, and
    \begin{equation*}
        m(E) \leq m(A)=n(A)=n(E)+n(\com{A}{E}) \leq n(E)+m(\com{A}{E}) \leq
        n(E)+\e
    \end{equation*}
    taking $\e$ small, we get $n(E)=m(E)$.

    Finally, suppose that $m_0$ is $\s$-finite, and let $X=\bigcup{A_k}$ for
    s me0disjoint collection $\ A_n\}$, then $ m_0 m_0)<\infty$. Then for every
    $E \in \Mc$,
    \begin{equation*}
        m(E)=\sum{m(E \cap A_k)}=\sum{n(E \cap A_k)}=n(E)
    \end{equation*}
    so that $m=n$, making $m$ unique.
\end{proof}
