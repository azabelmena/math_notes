\section{$\s$-Algebras}

\begin{definition}
    Let $X$ be a nonempty set. An  \textbf{algebra} of sets on $X$ is a nonempty
    collection  $\Ac$ of subsets of $X$ which are closed under finite unions and
    complements in $X$. We call  $\Ac$ a \textbf{$\s$-algebra} if it is closed
    under countable unions.
\end{definition}

\begin{lemma}\label{lemma_1.1.1}
    Let $X$ be a set and $\Ac$ an algbra on $X$. Then $\Ac$ is closed under
    finite intersections.
\end{lemma}
\begin{proof}
    Let $\{E_\l\}$ be a collection of sets of $\Ac$. Then by finite union
    $E=\bigcup{E_\l} \in \Ac$. Then by complements,
    $\com{X}{E}=\bigcap{\com{X}{E_\l}} \in \Ac$.
\end{proof}
\begin{corollary}
    $\s$-algebras are closed under countable disjoint unions.
\end{corollary}
\begin{proof}
    Let $\Ac$ a $\s$-algebra, and let  $\{E_n\}$ a collection of (not
    necessarily disjoint) sets in $\Ac$. Then take
    \begin{equation}
        F_n=\com{E_n}{\Big{(} \bigcup_{k=1}^{n-1}{E_k} \Big{)}}
    \end{equation}
    Then each $F_n$ is a set in $\Ac$, and are pairwise disjoint. Moreover,
    $\bigcup{E_n}=\bigcup{F_n}$.
\end{proof}

\begin{lemma}\label{lemma_1.1.2}
    Let $X$ be a set, and $\Ac$ an algebra on $X$. Then $\emptyset \in \Ac$ and
    $X \in \Ac$.
\end{lemma}
\begin{proof}
    By closure of finite unions, notice that if $E \in \Ac$, then
    $E \cup \com{X}{E} = X \in \Ac$ lemma \ref{8.1.1} gives us that $E \cap
    \com{X}{E}=\emptyset \in \Ac$.
\end{proof}

\begin{example}\label{example_1.1}
    \begin{enumerate}
        \item[(1)] The collections $\{\emptyset, X\}$ and $2^X$ are
            $\s$-algebras on any set $X$.

        \item[(2)] Let $X$ be an uncountable set. Then the collection
            \begin{equation*}
                \Cc=\{E \subseteq X : E \text{ is countable } \text{ or }
                \com{X}{E} \text{ is countable}\}
            \end{equation*}
            defines a $\s$-algebra of sets on $X$, since countable unions of
            countable sets are countable, and $\Cc$ is closed under complements.
            We call $\Cc$ the \textbf{$\s$-algebra of countable or co-countable
            sets}.
    \end{enumerate}
\end{example}

\begin{lemma}\label{lemma_1.1.3}
    Let $\{A_\l\}$ be a collection of $\s$-algebras on a set  $X$. Then the
    intersection
    \begin{equation*}
        \Ac=\bigcap{\Ac_\l}
    \end{equation*}
    is a $\s$-algebra on $X$. Moreover, if $F \subseteq X$, then there exists a
    unique smallest $\s$-algebra containing $F$; in particular, it is the
    intersection of all  $\s$-algebras containing  $F$.
\end{lemma}
\begin{proof}
    Notice that since each $\Ac_\l$ is a  $\s$-algebra, they are closed under
    countable unions and complements. Hence by definition,  $\Ac$ must also be
    closed under countable unions and complements.

    Now, let $F \subseteq X$ and let $\{\Ac_\l\}$ be the collection of all
    $\s$-algebras containing $F$. Then the intersection $\Ac=\bigcap{\Ac_\l}$ is
    also a $\s$-algebra containing  $F$; by above. Now, suppose that there is a
    smallest $\s$-algebra $\Bc$ containing $F$. Then we have that $\Bc \subseteq
    \Ac$. Now, by definition of  $\Ac$ as the intersection of all  $\s$-algberas
    containing  $F$, we get that  $\Ac \subseteq \Bc$; so that $\Bc=\Ac$.
\end{proof}

\begin{definition}
    Let $X$ be a nonempty set and  $F \subseteq X$. We define the $\s$-algebra
    \textbf{generated} by $F$ to be the smallest such $\s$-algebra $\Mc(F)$
    containing $F$.
\end{definition}

\begin{lemma}\label{lemma_1.1.4}
    Let $X$ be a set and let $E,F \subseteq X$. Then if $E \subseteq \Mc(F)$,
    then $\Mc(E) \subseteq \Mc(F)$.
\end{lemma}
\begin{proof}
    We have that since $E \subseteq \Mc(F)$, and $\Mc(E)$ is the intersection of
    all $\s$-algebras containing  $E$, then  $\Mc(E) \subseteq \Mc(F)$.
\end{proof}

\begin{definition}
    Let $X$ be a topological space. We define the  \textbf{Borel $\s$-algebra}
    on $X$ to be the  $\s$-algebra  $\Bc(X)$ generated by all open sets of $X$;
    that is
    \begin{equation*}
        \Bc(X)=\Mc(\Tc)
    \end{equation*}
    where $\Tc$ is the topology on $X$. We call the elements of $\Bc(X)$
    \textbf{Borel-sets}
\end{definition}

\begin{definition}
    Let $X$ be a topological space. We call a countable intersection of open
    sets of  $X$ a \textbf{$G_\d$-set} of $X$. We call a countable union of
    closed sets of $X$ an  \textbf{$F_\s$-set} of $X$.
\end{definition}

\begin{theorem}\label{lemma_1.1.5}
    The Borel $\s$-algebra on $\R$,  $\Bc(\R)$, is generated by the following.
    \begin{enumerate}
        \item[(1)] All open intervals of $\R$.

        \item[(2)] All closed intervals of $\R$.

        \item[(3)] All half-open intervals of $\R$.

        \item[(4)] All open rays of $\R$.

        \item[(5)] All closed rays of $\R$.
    \end{enumerate}
\end{theorem}

\begin{definition}
    Let $X_\a$ be a collection of non-empty sets, and let  $X=\prod{X_\a}$. If
    $\Mc_\a$ is a $\s$-algebra on $X_\a$, then we define the \textbf{product
    $\s$-algebra} on $X$ to be the smallest $\s$-algebra generated by all
    $\inv{\pi_\a}(E_\a)$, where $E_\a \in \Mc_\a$, and $\pi_\a:X \xrightarrow{}
    X_\a$ is the projection map onto the $\a$-th coordinate. We denote the
    product  $\s$-algebra by $\bigotimes{\Mc_\a}$.
\end{definition}

\begin{lemma}\label{lemma_1.1.6}
    Let $\{X_n\}$ be a countable collection of sets, each with a $\s$-algebra
    $\Mc_n$, and let $X=\prod{X_n}$. Then the product $\s$-algebra
    $\bigotimes{\Mc_n}$ on $X$ is generated by all $\prod{E_n}$, where $E_n \in
    \Mc_n$.
\end{lemma}
\begin{proof}
    Let $E_n \in \Mc_n$, then by definition of the projection map,
    $\inv{\pi_n}(E_n)=\prod{E_k}$ where $E_k=X_k$ for all  $k \neq n$. On the
    otherhand, we can see that $\prod{E_n}=\bigcap{\inv{\pi_n}(E_n)}$.
\end{proof}

\begin{lemma}
    Let $\{X_\a\}$ be a collection of sets, each together with a $\s$-algebra
    $\Mc_\a$. If each $\Mc_\a$ is generated by some $\Ec_\a$,
    then$\bigotimes{\Mc_\a}$ is generated by all $\inv{\pi_\a}(E_\a)$, where
    $E_\a \in \Ec_\a$.
\end{lemma}
\begin{proof}
    Let $\Fc=\{\inv{\pi_\a}(E_\a) : E_\a \in \Ec_\a\}$. Then by lemma
    \ref{lemma_1.1.4}, $\Mc(\Fc) \subseteq \bigotimes{\Mc_\a}$. On the
    otherhand, for any $\a$, the collection of all $\inv{\pi_\a}(E) \in
    \Mc(\Fc)$ is a $\s$-algebra on $X_\a$, containing $\Ec_\a$; and hence,
    $\Mc_\a$. That is, $\inv{\pi_\a}(E) \in \Mc(\Fc)$ for all $E \in \Mc_\a$,
    which gives us the reverse inclusion.
\end{proof}
\begin{corollary}
    If $\{X_\a\}$ is a countable collection, theb $\bigotimes{\Mc_\a}$ is
    generated by all $\prod{E_\a}$, where $E_\a \in \Ec_\a$.
\end{corollary}

\begin{lemma}\label{lemma_1.1.8}
    Let $X_1, \dots, X_n$ be metric spaces, and $X=\prod_{i=1}^n{X_i}$ on the
    product topology. Then
    \begin{equation*}
        \bigotimes(\Bc(X_i)) \subseteq \Bc(X)
    \end{equation*}
    Moreover, if each $X_i$ is seperable, then equality is established.
\end{lemma}
\begin{proof}
    We have that $\bigotimes{\Bc(X_i)}$ is generated by each $\inv{\pi_i}(U_i)$,
    where $U_i$ is an open set in  $X_i$. Since these sets are open, again by
    lemma \ref{lemma_1.1.4}, $\bigotimes{\Bc(X_i)} \subseteq \Bc(X)$.

    Now, suppose that each $X_i$ is seperable, and let  $C_i$ a countable dense
    set in  $X_i$, and let  $\Ec_i$ be the collection of all open balls in
    $X_i$ with rational radius $r$, and center in $C_i$. Then every open set in
     $X_i$ is a countable union of members of  $\Ec_i$. Moreover, the set of
     points in $X$ whose $i$-th coordinate is in $C_i$, for all $i$, is
     countable dense in $X_i$. Hence, $\Bc(X_i)$ is generated by $\Ec_i$, and
     since  $\bc(X)$ is generated by all $\prod_{i=1}^n{E_i}$, where $E_i \in
     \Ec_i$, we get $\Bc(X) \subseteq \bitotimes{\Bc(X_i)}$, and equality is
     established.
\end{proof}
\begin{corollary}
    $\Bc(\R^n)=\bigotimes_{i=1}^n{\Bc(\R)}$.
\end{corollary}

\begin{definition}
    We define an \textbf{elementary family} on a set $X$ to be a collection
    $\Ec$ of subsets of $X$ such that:
    \begin{enumerate}
        \item[(1)] $\emptyset \in \Ec$.

        \item[(2)] If $E, F \in \Ec$, then $E \cap F \in \Ec$.

        \item[(3)] If $E \in \Ec$, then $\com{X}{E}$ is a finite disjoint union
            of members of $\Ec$.
    \end{enumerate}
\end{definition}

\begin{lemma}\label{lemma_1.1.9}
    Let $X$ be a set and $\Ec$ an elementary family on  $X$. Let $\Ac$ be the
    collection of all finite disjoint unions of members of $\Ec$. Then $\Ac$ is
    an algebra on $X$.
\end{lemma}
\begin{proof}
    Let $A,B \in \Ec$, and let $\com{X}{B}=\bigcup_{i=1}^n{C_i}$, where each
    $C_i \in \Ec$ for all $1 \leq i \leq n$, and are disjoint. Then we have
    \begin{equation*}
        A \cup B=(\com{A}{V}) \cup B \text{ and }
        \com{A}{B}=\bigcup_{i=1}^n{(A \cap C_i)}
    \end{equation*}
    so that $A \cup B \in \Ac$, and $\com{A}{B} \in \Ac$. Now, by induction on
    $n$, suppose that $A_1, \dots, A_n \in \Ac$ are disjoint, then
    \begin{equation*}
        \bigcup_{i=1}^{n+1}{A_i}=A_{n+1} \cup \bigcup_{i=1}^n{\com{A_i}{A_{n+1}}}
    \end{equation*}
    is also a disjoint union. Moreover, we have that if
    $\com{X}{A_n}=\bigcup_{i=1}^{N_m}{B_m^i}$, where the union is disjoint, then
    \begin{equation*}
        \com{X}{\Big{(} \bigcup_{m=1}^n{A_m} \Big{)}}=
        \bigcap_{m=1}^n{\Big{(} \bigcup_{i=1}^{N_m}{B_m^i} \Big{)}}
    \end{equation*}
    is also a disjoint union. This makes $\Ac$ an algebra on $X$.
\end{proof}
