\section{Borel Measures on $\R$}

\begin{definition}
    We call measures defined on the $\s$-algebra  $\Bc(\R)$ of all Borel sets of
    $\R$  \textbf{borel measures}. If $m$ is a finite Borel meausre on $\R$, we
    define the  \textbf{distribution function} of $m$ to be the function $F:\R
    \xrightarrow{} \R$ by the rule
    \begin{equation*}
        F(x)=m((-\infty, x])
    \end{equation*}
\end{definition}

\begin{lemma}\label{lemma_1.4.1}
    Let $m$ be a finite Borel measure on  $\R$. Then the distribution function o
    of  $m$ is an increasing, right-continuous function. Moreover, if  $b>a$ are
    extended real numbers, then  $m((a,b])=F(b)-F(a)$.
\end{lemma}
\begin{proof}
    By the monotonicity of $m$,  $F$ is an increasing function. Now, let
    $\{x_n\}$ a sequence of points for which $\{x_n\} \xrightarrow{} x$ from the
    right. For $x \geq 0$, the collection $\{(-\infty, x_n]\}$, then by
    continuity from below, we haev
    \begin{equation*}
        m\Big{(} \bigcup{(-\infty, \times_n]} \Big{)}=\lim_{n \xrightarrow{}
        \infty}{m((-\infty, x_n])}=
        \lim_{x_n \xrightarrow{} x+}{m((-\infty, x_n])}=m((-\infty, x])
    \end{equation*}
    That is $\lim{F(x_n)}=F(x)$ as $x_n \xrightarrow{} x+$. A similar argument
    holds for $x<0$, using continuity from above. Lastly, observe that
    \begin{equation}
        (-\infty, b]=(-\infty, a] \cup (a, b]
    \end{equation}
\end{proof}

\begin{lemma}\label{lemma_1.4.2}
    Let $F:\R \xrightarrow{} \R$ an increasing right-continuous function. If
    $\{(a_i,b_i]\}_{i=1}^n$ is a finite collection of disjoint half-open
    intervals, and $m_0$ is defined by
    \begin{equation}\label{equation_1.8}
        m_0\Big{(} \bigcup_{i=1}^n{(a_i,b_i]}
        \Big{)}=\sum_{i=1}^n{F(b_i)-F(a_i)} \text{ and } m_0(\emptyset)=0
    \end{equation}
    then $m_0$ is a pre-measure on the algebra of all finite unions of half-open
    intervals.
\end{lemma}
\begin{proof}
    Denote the algebra of all finite unions of half-open intervals by $\Ac$.
    Notice then by theorem \ref{theorem_1.1.5}, that $\Bc(\R)$ is the
    smallest $\s$-algebra generated by $\Ac$. Now, let $\{(a_i,b_i]\}_{i=1}^n$ a
    finite disjoint collection of half-open intervals, and take
    $(a,b]=\bigcup_{i=1}^n{(a_i,b_i]}$ Then $(a,b]$ is partitioned by the points
    $P=\{a=a_1<b_1=a_1<b_2= \dots <b_n=b \dots \}$ Therefore
    \begin{equation*}
        \sum_{P}{F(b_i)-F(a_i)}=\sum_{i=1}^n{F(b_i)-F(a_i)}=F(b)-F(a)
    \end{equation*}
    That is, if $\{I_j\}_{j=1}^n$ and $\{J_i\}_{i=1}^n$ are finite collections
    of disjoint half-open intervals, where
    $\bigcup_{j=1}^n{I_j}=\bigcup_{i=1}^n{J_i}$, then
    \begin{equation*}
        \sum_{j=1}^n{m_0(I_j)}=\sum_{i=1,j=1}^{n,n}{m_0(I_j \cap J_i)}=
        \sum_{i=1}^n{m_0(J_i)}
    \end{equation*}
    this makes $m_0$ well defined, and finitely additive, by construction.

    Now, consider $\{I_n\}$ a countable collection of disjoint half-open
    intervals. And let $I=\bigcup{I_n}$. Then $I \in \Ac$, and
    $m_0(I)=\sum{m_0(I_n)}$. Now, since $I$ is a finite union of disjoint
    half-open intervals, partition  $\{I_n\}$ into finitely many subcollections
    $\{I_{n_k}\}$ for which $\bigcup{I_{n_k}}$ is a single half-open interval.
    Then $I=(a,b]$ a single half-open interval. Thus, by the finite additivity
    of $m_0$, we get
    \begin{equation*}
        m_0(I)=m_0\Big{(} \bigcup_{j=1}^n{I_j} \Big{)}+
        m_0(\com{I}{\Big{(} \bigcup_{j=1}^{n-1}{I_j} \Big{)}}) \geq
        m_0(I)=m_0\Big{(} \bigcup_{j=1}^n{I_j} \Big{)}=\sum_{i=1}^n{m_0(I_j)}
    \end{equation*}
    Taking $n \xrightarrow{} \infty$, we get $m_0(I) \geq \sum{m_0(I_n)}$

    Now, suppose that $a,b \in \R^\infty$ are finite, and take $\e>0$. By
    hypothesis, we get  $F$ is right-continuous, so there is a  $\d>0$ for which
     $F(a+\d)-F(a)<\e$. If $I_n=(a_n,b_n]$, there is a $\d_n>0$ for which
     $F(b_n+\d_n)-F(b_n)<\frac{\e}{2^{n+1}}$. Now, $[a+\d,b]$ is compact, by the
     finite collection $\{(a_i,b_i+\d_i)\}_{i=1}^N$. Now, refine this subcover by
     discaring any $(a_i,b_i+\d_i)$ contained in another of that cover, and
     reindex $i$ to $j$ by letting $b_j \in (a_{j+1},b_{j+1}+\d_{j+1})$ for all
     $1 \leq j \leq N-1$. Then
     \begin{align*}
         m_0(I)     &=      F(b)-F(a+\d)+\e \\
                    &\leq   F(b_N+\d_N)-F(a_1)+\e \\
                    &=      F(b_N+\d_N)-F(a_N)+\sum_{j=1}^{N-1}{F(a_{j+1})-F(a_j)}+\e \\
                    &\leq   F(b_N+\d_N)-F(a_N)+\sum_{j=1}^{N-1}{F(b_j+\d_j)-F(a_j)}+\e \\
                    &<   \sum_{i=1}^N{F(b_i)+\frac{\e}{2^{i+1}}-F(a_i)}+\e
                    &<   \sum{m_0(I_n)}+\e
     \end{align*}
     Since $a$ and  $b$ are finite, taking  $\e$ small enough gives us the
     $m_0$. as a pre-measure on $\Ac$. Now, if $a=-\infty$, for all  $M<\infty$,
     there is a cover  $\{(a_i,b_i+\d_i)\}$ of $[-M,b]$, so that by the pervious
     argument, $F(b)-F(-M) \leq \sum{m_0(I_n)}+\e$. If $b=\infty$, then for all
      $M<\infty$, by a similar argument,  $F(M)-F(a) \leq\sum{m_0(I_n)}+\e$.
      Taking $\e$ small then gives us the same result.
\end{proof}

\begin{theorem}\label{theorem_1.4.3}
    If $F:\R \xrightarrow{} \R$ is any increasing right-continuous function,
    then there exists a unique Borel measure on $\R$, $m_F$ such that
    $m((a,b])=F(b)-F(a)$ for all $a,b \in \R^\infty$. Moreover if $G$ is another
    increasing right-continuous function, then  $m_F=m_G$ if, and only if  $F-G$
    is a constant function. Lastly, if  $m$ is a Borel measure on  $\R$, finite
    on all bounded Borel sets of $\R$, and if $F$ is defined by
    \begin{equation}\label{equation_1.9}
        F(x)=\begin{cases}
                m((0,x]), \text{ if } x>0   \\
                0, \text{ if } x=0   \\
                -m((0,x]), \text{ if } x<0   \\
        \end{cases}
    \end{equation}
    Then $F$ is the distribution function associated with  $m$.
\end{theorem}
\begin{proof}
    We have by lemma \ref{lemma_1.4.2}, that $F$ defines pre-measures on the
    $\s$-algebra  $\Ac$ of finite unions of open half-intervals, and moreover
    that $m_F=m_G$ if, and only if $F-G$ is constant. Moreover, each $m_F$ is
    $\s$-finite since $\R=\bigcup_{-n}^n{(n,n+1]}$.

     Now, let $m$ be a Borel measure on  $\R$, and define  $F$ by equation
     \ref{equation_1.9}. By lemma \ref{lemma_1.4.1}, $F$ is increasing and right
     continuous. Lastly, since $m=m_F$ on  $\Ac$, and $\Bc(\R)$ is generated by
     $\Ac$, then $m=m_F$ on all $\Bc(\R)$.
\end{proof}

\begin{definition}
    We call Borel measures on $\R$, with dsitribution functions defined by
    \begin{equation*}
        F(x)=\begin{cases}
                m((0,x]), \text{ if } x>0   \\
                0, \text{ if } x=0   \\
                -m((0,x]), \text{ if } x<0   \\
        \end{cases}
    \end{equation*}
    \textbf{Lebesgue-Stieltjes measures}.
\end{definition}

\begin{lemma}\label{lemma_1.4.4}
    Let $m$ be a Lebesgue-Stieljes measure on $\R$. Then for any $m$-measurable
    set $E$,
    \begin{equation}\label{equation_1.10}
        m(E)=\inf{\Big{\{} \sum{m((a_n,b_n))} : E \subseteq \bigcup{(a_n,b_n)} \Big{\}}}
    \end{equation}
\end{lemma}
\begin{proof}
    Observe that for all $m$-measurable sets  $E$, that
    \begin{equation*}
        m(E)=\inf{\Big{\{} \sum{F(b_k)-F(a_k)} : E \subseteq \bigcup{(a_n,b_n)} \Big{\}}}
    \end{equation*}
    Let
    \begin{equation*}
        n(E)=\inf{\Big{\{} \sum{m((a_k,b_k))} : E \subseteq \bigcup{(a_k,b_k)} \Big{\}}}
    \end{equation*}
    and let $E \subseteq \bigcup{(a_k,b_k)}$. Then $\{(a_k,b_k)\}$ is a
    countable collection of disjoint half-open intervals $I_j^l$. Specifically,
    $I_k^l=(c_j^l,c_j^{l+1}]$, where $\{c_j^l\}$ is an increasing sequence, with
    $c_j^1=a_j$, and $\{c_j^l\} \xrightarrow{} b_j$ as $l \xrightarrow{}
    \infty$. Then
    \begin{equation*}
        E \subseteq \bigcup_{j,l}{I_j^l}
    \end{equation*}
    We get
    \begin{equation*}
        \sum{m((a_k,b_k))}=\sum_{j,l}{m(I_j^l)} \geq m(E)
    \end{equation*}
    so that $n(E) \geq m(E)$. On the other hand, letting $\e>0$, there exists a
    countable collection $\{(a_k,b_k]\}$ of disjoint half-open intervals where
    $E \subseteq \bigcup{(a_k,b_k]}$, and
    \begin{equation*}
        \sum{m((a_k,b_k))} \leq m(E)+\e
    \end{equation*}
    Thus, for every $j$, there is a  $\d_j$ for which
    $F(b_j+\d_j)-F(b_j)<\frac{\e}{2^{j+1}}$. Therefore $E \subseteq
    \bigcup{((a_j,b_j+\d_j))}$ and
    \begin{equation*}
        \sum{m((a_j,b_j+\d_j))} \leq \sum{m((a_k,b_k])}+\e \leq m(E)+\e
    \end{equation*}
    so that $n(E) \leq m(E)$.
\end{proof}

\begin{theorem}\label{theorem_1.4.5}
    Let $m$ be a Lebesgue-Stieltjes measure on $\R$. If $E$ is  $m$-measurable,
    then
    \begin{equation}\label{equation_1.11}
        m(E)=\inf{\{m(U) : U \text{ is open and } E \subseteq U\}}  \\
        =\sup{\{m(K) : K \text{ is compact and } K \subseteq\}}
    \end{equation}
\end{theorem}
\begin{proof}
    By lemma \ref{lemma_1.4.4}, for every $\e>0$, there is a countable
    collection $\{(a_n,b_n)\}$ of disjoint intervalscovering $E$, for which
    \begin{equation*}
        \sum{m((a_n,b_n))} \leq m(E)+\e
    \end{equation*}
    Let $U=\bigcup{(a_n,b_n)}$, then $U$ is open, and $E  \subseteq U$, with
    $m(U) \leq m(E)+\e$. On the other hand, by monotonicity of $m$,  $m(E) \leq
    m(U)$, so we get the first equality.

    Now, suppose that $E$ is bounded. If  $E$ is closed, then $E$ is compact,
    and there is nothing else to prove. Otherwise, let  $\e>0$ and choose an
    open set $U$ contained in  $\com{(\cl{E})}{E}$ (where $\cl{E}$ is the
    topological closure of $E$) such that $m(U) \leq m(\com{(\cl{E})}{E})+\e$.
    Now, let $K=\com{(\cl{E})}{U}$. Then $K$ is compact, and  $K \subseteq E$,
    and $m(K)=m(E)-m(E \cap U)=m(E)-m(U)-m(\com{(U)}{E}) \geq
    m(E)-m(U)+m(\com{(\cl{E})}{E}) \geq m(E)-\e$

    Now, if $E$ is unbounded, let  $E_n=E \cap (n,n+1]$. Then by the preceding
    argument, for every $\e>0$, there is a  $K_n$ compact, contained in  $E_n$
    for which  $m(K_n) \geq m(E_n)+\frac{\e}{2^{|n|}3}$. Let
    $H_n=\bigcup_{-n}^n{K_n}$. Then $H_n$ is compact, and contained in $E$, and
    \begin{equation*}
        m(H_n)=m\Big{(} \bigcup_{i=-n}^n{E_i} \Big{)}-\e
    \end{equation*}
    By continuity from above, we are done.
\end{proof}

\begin{theorem}[Inner and Outer Approximation]\label{theorem_1.4.6}
    Let $m$ be a Lebesgue-Stieltjes measure on $\R$, and let $E$ any subset of
    $\R$ of finite $m$-measure. Then the following are equivalent.
    \begin{enumerate}
        \item[(1)] $E$ is $m$-measurable.

        \item[(2)] There exists a $G_\d$-set $V$ for which $E=\com{V}{N_1}$, and
            $m(N_1)=0$.

        \item[(3)] There exists an $F_\s$-set  $H$ for which  $E=H \cup N_2$,
            and $m(N_2)=0$.
    \end{enumerate}
\end{theorem}
\begin{proof}
    Since $m$ is complete on all $m$-measurable sets, statements (2) and (3)
    imply (1). Now, let $E$ be $m$-measurable, with $m(E)<\infty$. Choose $U_n$
    open, containing $E$ for all  $n$, and choose  $K_n$ compact, conatined in
    $E$ for all  $n$, such that,
    \begin{equation*}
        m(U_n)-\frac{1}{2^n} \leq m(E) \leq m(K_n)-\frac{1}{2_n}
    \end{equation*}
    Let $V=\bigcap{U_n}$, and $H=\bigcup{K_n}$. Then $V$ is a $G_\d$-set, $H$ is
    an $F_\s$-set, and $H \subseteq E \subseteq V$. Moreover,
    $m(V)=m(H)=m(E)<\infty$, so that $m(\com{V}{E})=m(\com{E}{H})=0$.
\end{proof}

\begin{lemma}\label{lemma_1.4.7}
    Let $m$ be a Lebesgue-Stieltjes measure. If $E$ is $m$-measurable, of finite
     $m$-measure, then for every  $\e>0$, there exists a finite collection
     $\{I_j\}_{j=1}^n$, such that
     \begin{equation}\label{equation_1.12}
         m(\com{E}{A} \cup \com{A}{E})<\e \text{ where } A=\bigcup_{j=1}^n{I_j}
     \end{equation}
\end{lemma}

\begin{definition}
    We define the \textbf{Lebesgue measure} on $\R$ to be the complete
    Lebesgue-Stieljes measure $m$ associated to the distribution funtion
    $l(x)=x$. We call all $m$-measurable sets \textbf{Lebesgue measurable}, and
    denote the domain of $m$ by $\Lc$.
\end{definition}

\begin{theorem}\label{theorem_1.4.8}
    If $E$ is Lebesgue measurable, then so is  $E+s$ and $rE$, for all $s,r \in
    \R^\infty$, and where $E+s=\{x+s : x \in E\}$, and $rE=\{rx : x \in E\}$.
    Moreover
    \begin{equation*}
        m(E+s)=m(E) \text{ and } m(rE)=|r|m(E)
    \end{equation*}
\end{theorem}

\begin{theorem}\label{theorem_1.4.9}
    Countable sets of $\R$ have Lebesgue measure  $0$.
\end{theorem}

\begin{example}\label{example_1.3}
    Let $m$ be the Lebesgue measure on $\R$.
    \begin{enumerate}
        \item[(1)] $m(\Q)=0$, since $\Q$ is countable.

        \item[(2)] Here is a pathological example where topologically ``large''
            sets can be measured to be as small as one likes, and where
            topologically ``small'' sets can be measured as large as one likes.
            Let $\{r_n\}$ be an enumeration of $\Q$ in the interval $[0,1]$.
            Take $\e>0$, and let  $I_r=(r-\frac{1}{2^n}, r+\frac{1}{2^n})$ the
            open interval centered around $r$ of length $\frac{1}{2^n}$. Then
            take $U=(0,1) \cap \bigcup{I_r}$. Then $U$ is open and dense in
            $[0,1]$, but $m(U) \leq \sum{\frac{1}{2^n}}<\e$. Now, let
            $K=\com{[0,1]}{U}$. Then $K$ is closed and nowhere dense in $[0,1]$,
            however $m(K) \geq 1-\e$. Notice however that non-empty open sets of
            $\R$ cannot have Lebesgue measure $0$.
    \end{enumerate}
\end{example}
