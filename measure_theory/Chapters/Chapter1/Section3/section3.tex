\section{Continuous Functions of a Real Variable.}

\begin{definition}
    A realvalued function $f$ on a domain $E$ is said to be \textbf{continuous}
    at a point $x \in E$ provided for any $\e>0$ there is a  $\dl>0$ for which
    \begin{equation*}
        |f(x)-f(y)|<\e \text{ whenever } |x-y|<\delta \text{ for any } y \in E
    \end{equation*}
    We call $f$  \textbf{continuous} on $E$ if it is continuous at every point
    in  $E$. We call  $f$  \textbf{Lipschitz continuous} if there is a $c \geq
    0$ for which
    \begin{equation*}
        |f(x)-f(y)| \leq c|x-y| \text{ for all } x,y \in E
    \end{equation*}
\end{definition}

\begin{lemma}\label{1.3.1}
    A Lipschitz continuous function on a domain is continuous on that domain.
\end{lemma}

\begin{lemma}[The Sequential Criterion]\label{1.3.2}
    A realvalued function $f$ defined on a domain  $E$ is continous at a point
    $x \in E$ if, and only if for any ssequence  $\{x_n\} \xrightarrow{} x$ of
    points in $E$, converging to  $x$, that the sequence $\{f(x_n)\} \xrightarrow{}
    f(x)$ converges to $f(x)$.
\end{lemma}

\begin{theorem}[The Extreme Value Theorem]\label{1.3.3}
    A continuous realvalued function defined on a nonempty, closed and bounded
    domain takes on a maximum value, and a minimum value on that domain.
\end{theorem}
\begin{proof}
    Let $f$ be a continuous realvalued function defined on the domain  $E$,
    where $E$ is nonempty, closed, and bounded. Let $x \in E$ and  $\dl>0$ and
    $\e=1$. Define the open interval  $I_x=(x-\dl,x+\dl)$. Then if $y \in E \cap
    I_x$, then  $|f(x)-f(y)|<1$. So that $|f(y)| \leq |f(x)|+1$. Notice also
    that the collection $\{I_x\}$ is an open cover of $E$. By the theorem of
    Heine-Borel, there is a finite subcover of  $E$,  $\{I_{x_k}\}_{k=1}^n$.
    Define, then, $M=1+\max{\{|f(x_1)|, \dots, |f(x_n)|\}}$. Then we get that
    $|f(x)| \leq M$ and $f$ is bounded.

    Now, let  $m=\sup{f(E)}$. If $f$ does not take the value  $m$ for any points
    in  $E$, then the function  $x \xrightarrow{} \frac{1}{f(x)-m}$ is a
    contoinuous unbounded function on $E$; which is impossible. So there is an
    $x \in E$ with  $f(x)=m$ and $m$ is a maximum value. Now, since $f$ is
    continuous, then so is  $-f$, and hence  $-m$ defines a minimum value on $f$.
\end{proof}

\begin{theorem}[The Intermediate Value Theorem]\label{1.3.4}
    If $f$ is a continuous realvalued function on a closed bounded interval
    $[a,b]$, for which $f(a)<c<f(b)$, then there exists an $x_0 \in (a,b)$ for
    which $f(x_0)=c$.
\end{theorem}
\begin{proof}
    Define $a_1=a$ and $b_1=b$ and let $m_1$ be the midpoint of the interval
    $[a_1,b_1]$. If $c<f(m_1)$, define $a_2=a_1$ and $b_2=m_1$, otherwise define
    $a_2=m_1$ and $b_2=m_1$, so that in either case we get $f(a_2) \leq c \leq
    f(b_2)$ and $b_2-a_2=\frac{b-a}{2}$. By induction, construct the collection
    of closde bounded intervals $\{[a_n,b_n]\}$ such that $f(a_n) \leq c \leq
    f(b_n)$ and $b_n-a_n=\frac{b-a}{2^{n-1}}$. This collection is a descending
    collection, so by the nested set theorem, the intersection
    $I=\bigcap{[a_n,b_n]}$ is nonempty. Choose an $x_0 \in I$, and observe that
    \begin{equation*}
        |a_n-x_0| \leq b_n-a_n=\frac{b-a}{2^{n-1}}
    \end{equation*}
    So the sequence $\{a_n\} \xrightarrow{} x_0$. By the sequential criterion,
    since $f$ is continuous at  $x_0$, we get the sequence $\{f(a_n)\}
    \xrightarrow{} f(x_0)$. Since $f(a_n) \leq c$, and $(-\infty,c]$ is closed,
    we also get $f(x_0) \leq c$.

    By similar reasoning to the argument provided above, we also get that
    $f(x_0) \geq c$ so that equality is established.
\end{proof}

\begin{definition}
    A realvalued function $f$ on a domain $E$ is said to be \textbf{uniformly
    continuous} if for everyu $\e>0$ there is a  $\dl>0$ such that
    \begin{equation*}
        |f(x)-f(y)|<\e \text{ whenever } |x-y|<\dl \text{ for all } x,y \in E
    \end{equation*}
\end{definition}

\begin{lemma}\label{1.3.5}
    If $f$ is a uniformly continuous function on a domain  $E$, then it is
    continuous on  $E$.
\end{lemma}

\begin{theorem}\label{1.3.6}
    A continuous realvalued function on a closed and bounded domain is unifomrly
    continuous.
\end{theorem}
\begin{proof}
    Let $f$ be continuous on  $E$, and  $E$ a closed and bounded domain. Let
    $\e>0$. For every  $x \in E$, there is a  $\dl_x>0$ for which
    $|f(x)-f(y)|<\e$ whenever $|x-y|<\dl_x$ for some $y \in E$. Define
    $I_x=(x-\frac{\dl_x}{2}, x+\frac{\dl_x}{2})$. Then $\{I_x\}$ is an open
    cover for $E$, so that by the theorem of Heine-Borel, there is a finite
    subcover  $\{I_{x_k}\}_{k=1}^n$ of $E$. Define  $\dl=
    \frac{1}{2}\min{\{\frac{\dl_{x_1}}{2}, \dots, \frac{\dl_{x_n}}{2}\}}$. Then
    $\dl>0$ moreover, if $x,y \in E$, with $|x-y|<\dl$, then asince
    $\{I_{x_k}\}$ covers $E$, there is a  $k>0$ such that
    \begin{equation*}
        |x-x_k|<\frac{\dl_{x_k}}{2} \text{ and } |x_{x_k}-y|<\frac{\dl_{x_k}}{2}
    \end{equation*}
    Then we have $|f(x)-f(x_k)|<\frac{\e}{2}$ and $|f(x_k)-f(y)|<\frac{\e}{2}$
    so that $|f(x)-f(y)|<\e$, which makes $f$ unifomrly continuous.
\end{proof}
