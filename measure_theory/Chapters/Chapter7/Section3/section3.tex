%----------------------------------------------------------------------------------------
%	SECTION 1.1
%----------------------------------------------------------------------------------------

\section{Integration and Differentiation.}

\begin{theorem}\label{7.3.1}
    Let $f$ be Riemann integrable on an interval  $[a,b]$, and for  $a \leq x
    \leq b$, let  $F(x)=\int_{a}^{x}{f(x)}dx$, then  $F$ is continuous on
    $[a,b]$. Firtheremore,, if  $f$ is continuous at $x_0=[a,b]$, then $F$ is
    differentiable at  $ x_0$ and $F'(x_0)=f(x_0)$.
\end{theorem}
\begin{proof}
    We have that $f$ is bounded, so for some  $a \leq t \leq b$, we have  $|f|
    \leq M$. Now if  $a \leq x<y \leq b$, then  $|F(y)-F(x)|=|\int_{x}^{y}{f(t)}
    dt| \leq M(y-x)$.

    Now given $\epsilon>0$, we see that  $|F(y)-f(x)|<\epsilon$, whenever
    $|y-x|< \frac{\epsilon}{M}$. Thus $f$ is continuous at  $ x_0$. Let
    $\epsilon>0$ and  $\delta>0$ such that  $|f(t)-f(x_0)|<\epsilon$ whenever
    $|t-x_0|<\delta$ whenever $a \leq t \leq b$. Thus if  $ x_0-\detla<s \leq
    x_0 \leq t<x_0+\delta$, and $a \leq s<t \leq b$, then:
        \begin{equation*}
            |\frac{F(s)-F(t)}{s-t}|=|\frac{\int_{x}^{y}{f(u)-f(x_0)} du}{s-t}|<\epsilon		
        \end{equation*}
        thus, we get $F'(x_0)=f(x_0)$.
\end{proof}

\begin{theorem}[The Fundamental Theorem of Calculus]\label{7.3.2}
    If $f$ is Riemann integrable on  $[a,b]$, and if there is a differentiable
    function  $F$ on  $[a,b]$ such that  $F'=f$, then:
        \begin{equation}
            \int_{a}^{b}{f(x)} dx=F(b)-F(a)		
        \end{equation}
\end{theorem}
\begin{proof}
    Let $\epsilon>0$, and construct a partition  $P$ of  $[a,b]$ such that
    $U(f,P)-L(f,P)<\epsilon$. Then by the mean value theorem, there is a  $t_i
    \in [x_{i-1},x_i]$ such that $F(x_i)-F(x_{i-1})=f(t_i) \Delta{x_i}$, for  $1
    \leq i \leq n$. Taking sums, we have
        \begin{equation}
            \sum{f(t_i) \Delta{x_i}}=F(b)-F(a)
        \end{equation}
    thus, by theorem \ref{7.1.6}, we have:
        \begin{equation*}
            |\int_{a}^{b}{f(x) dx}-(F(b)-F(a))|<\epsilon.		
        \end{equation*}
\end{proof}

\begin{theorem}[Integration by Parts]\label{7.3.3}
    Let $F$ and  $G$ be differentiable functions on  $[a,b]$, and let  $F'=f$
    and  $G'=g$  be Riemann integrable on $ [a,b]$. Then:
        \begin{equation}
            \int_{a}^{b}{Fg(x) dx}=FG(b)-FG(a)-\int_{a}^{b}{fG(x) dx}		
        \end{equation}
\end{theorem}
\begin{proof}
    By theorem \ref{7.2.2},  $FG$ is Riemann integrable, and also notice that
    $(FG)'=Fg+fG$, thus by the fundamental theorem of calculus:
        \begin{equation*}
            \int_{a}^{b}{FG dx}=\int_{a}^{b}{Fg dx}+\int_{a}^{b}{fG dx}=FG(b)-FG(a)
        \end{equation*} 
\end{proof}
\begin{remark}
    Integration by parts is what will allow us to find integrals of products of
    functions.
\end{remark}
