\section{Properties of Lebesgue Measurable Functions}

\begin{lemma}\label{3.1.1}
    Let $f$ be an extended realvalued function on a measurable domain  $E$. Then
    the following are equivalent.
    \begin{enumerate}
        \item[(1)] for some $c \in \R$, the set $\{x \in E : f(x)>c\}$ is
            measurable.

         \item[(2)] for some $c \in \R$, the set $\{x \in E : f(x) \geq c\}$ is
            measurable.
    \end{enumerate}
\end{lemma}
\begin{proof}
    Let $E_1=\{x \in E : f(x)>0\}$ and $E_2=\{x \in E : f(x) \geq c\}$. Suppose
    that $S$ is measurable, then notice that
    \begin{equation*}
        T=\bigcap{\{x \in E : f(x)>c-\frac{1}{k}\}}
    \end{equation*}
    Now, each of the sets in this intersection is measurable, and since
    measurable sets form a $\s$-algebra, $T$ must also be measurable. Likewise,
    if $T$ is measurable, notice that
    \begin{equation*}
        S=\bigcup{\{x \in E : f(x)>c+\frac{1}{k}\}}
    \end{equation*}
    is measurable by the same argument.
\end{proof}
\begin{corollary}
    The followingh are equivalent
    \begin{enumerate}
    \item[(1)] for some $c \in \R$, the set $\{x \in E : f(x)<c\}$ is
        measurable.

         \item[(2)] for some $c \in \R$, the set $\{x \in E : f(x) \leq c\}$ is
            measurable.
    \end{enumerate}
\end{corollary}
\begin{proof}
    Notice that these statements are the contrapostives of the statements above.
\end{proof}
\begin{corollary}
    For some $c \in \R$, the set  $\{x \in E : f(x)=x\}$ is measurable.
\end{corollary}
\begin{proof}
    Let $E_3=\{x \in E : f(x)=c\}$. If $c$ is finite, notice that  $E_3=\{x \in
    E : f(x) \geq c\} \cap \{x \in E : f(x) \leq x\}$, which makes $E_3$
    measurable. Now, if $c=\infty$, then  $\{x \in E : f(x)=\infty\}=\bigacp{\{x
    \in E : f(x)>k\}}$ for some $k$, which is again, measurable.
\end{proof}

\begin{definition}
    Let $f$ be an extended realvalued function on a measurable domain. We say
    $f$ is  \textbf{Lebesgue measurable} if it statisfies one of the conditions
    of lemma \ref{3.1.1} (or its corollories).
\end{definition}

\begin{lemma}\label{3.1.2}
    Let $f$ be an extended realvalued functoon on a measurable domain $E$. Then
    $f$ is measurable if, and only if, there exists an open set $U$, such that
    $\inv{f}(U)$ is measurable.
\end{lemma}
\begin{proof}
    Suppose that $U$ is open in  $\R$ such that  $\inv{f}(U)$ is measurable.
    Then the interval $(c,\infty)$ is open, which makes $\inv{f}((c,\infty))$
    measurable. Notice that $\inv{f}((c,\infty))=\{x \in E : f(x)>c\}$. This
    makes $f$ measurable.

    Conversely, suppose that  $f$ is measurable, and let  $U$ be open in $\R$.
    Then $U=\bigcup{I_k}$ for some countable collection of bounded open
    intervals $\{I_k\}$. Let $I_+=B_k \cap A_k$ where
    \begin{equation*}
        B_k=(-\infty, b_k) \text{ and } A_k=(a_k, \infty) \text{ for some
        } a_k,b_k \in \R
    \end{equation*}

    Since $f$ is measurable, then the preimages $\inv{f}(A_k)$ and
    $\inv{f}(B_k)$ are measurable. Hence, so is the union
    \begin{equation*}
        \bigcup{(\inv{f}(B_k) \cap \inv{f}(A_k))}=\bugcup{\inv{f}(I_l)}
        =\inv{f}(\bigcup{I_k})=\inv{f}(U)
    \end{equation*}
\end{proof}
\begin{corollary}
    A realvalued function continuous on a measurable domain is measurable.
\end{corollary}

\begin{lemma}\label{3.1.3}

    Monotone functions defined on an interval are measurable.
\end{lemma}

\begin{lemma}\label{3.1.4}
    Let $f$ be an extended realvalued function on a measurable domain  $E$. The
    following are true
    \begin{enumerate}
        \item[(1)] If $f$ is measurable on $E$, and  $f=g$ almost everywhere on
             $E$, for some extended realvalued function $g$ on $E$, then  $g$ is
             measurable on  $E$.

         \item[(2)] If $D \subseteq E$ is measurable, then  $f$ is measurable
             if, and only if the restrictions  $f|_D$ and  $f|_{\com{E}{D}}$ are
             measurable.
    \end{enumerate}
\end{lemma}
\begin{proof}
    Suppose that $f$ is measurable and that $g$ is an extended realvalued
    function on $E$ for which  $f=g$ a.e. on  $E$. Let  $A=\{x \in E : f \neq
    g\}$. Observe that
    \begin{equation*}
        E_1=\{x \in E : g(x)>c\}=\{x \in A : g>c\}z \cup \{x \in E : f>c\}
        \cap \com{E}{A}
    \end{equation*}
    Since $f=g$ a.e. on  $E$, then  $m(A)=0$, so that $\{x \in A : g>c\}$ is
    measurable. Then since measurable sets are a $\s$-algebra, $E_1$ is
    measurable. This makes $g$ measurable.

    Now, observe, also, that for every  $c \in \R$, and  $D \subseteq E$
    measurable, that
    \begin{equation*}
        \{x \in E : f>c\}=\{x \in D : f>c\} \cup \{x \in \com{E}{D} : f>c\}
    \end{equation*}
    So that if $f$ is measurable, so are its restirctions  $f|_D$ and
    $f|_{\com{E}{D}}$, and vice versa.
\end{proof}

\begin{theorem}\label{3.1.5}
    Let $f$ and  $g$ be measurable functions on a measurable domain, for which
    $f$ and  $g$ are finite almost everywhere on  $E$. Then
    \begin{enumerate}
        \item[(1)] For all $\al,\bt \in \R$, the function $\al f+\bt g$ is
            measurable.

        \item[(2)] $fg$ is measurable.
    \end{enumerate}
\end{theorem}
\begin{proof}
    Suppose, without loss of generality, that $f$ and  $g$ are finite on all
    $E$. If  $\al=0$ and  $\bt=0$, the  $\al f=0$ and we are done. Now, take
    $\al \neq 0$ and  $\bt=0$. Then observe that if  $\al>0$ then  $\{x \in E :
    \al f>c\}=\{x \in E : f>\frac{c}{\al}\}$, where as if $\al<0$ then  $\{x \in
    E : \al f>c\}=\{x \in E : f<\frac{c}{\al}\}$. Since $f$ is measurable, both
    these sets are measurable, which makes  $\al f$ measurable.

    Now, take  $\al=\bt=1$ and observe the function  $f+g$. If  $f+g<c$ for all
     $x \in E$, then  $f<c-g$, and by the density of  $\Q$ in  $\R$,  there is a
     rational number $q$ for which  $f<q<c-g$. Then notice that
     \begin{equation*}
         \{x \in E : f+g<c\}=\bigcup_{q \in \Q}{\{x \in E : g<c-q\} \cap \{x \in
         E : f<q\}}.
     \end{equation*}
     then since $f$ and  $g$ are both measurable, this countable union is
     measurable.

     Lastly, notice that  $fg=\frac{1}{2}((f+g)^2-f^2-g^2)$ so that it suffices
     to show that $f^2$ is measurable. Indeed, for  $c \geq 0$  $\{x \in E :
     f^2>c\}=\{x \in E : f>\sqrt{c}\}$ and for $c<0$,  $\{x \in E : f^2>c\}=\{x
     \in E : f<-\sqrt{c}\}$. In either case, $f^2$ is measurable. Hence, by
     linearity, so is  $fg$.
\end{proof}

\begin{definition}
    We define the \textbf{Characteristic function} for a set $A$ of real numbers
    to be the function $\chi_A:A \xrightarrow{} \{0,1\}$ defined by
    \begin{equation*}
        \chi_A=\begin{cases}
                1, \text{ if } x \in A  \\
                0, \text{ if } x \notin A   \\
            \end{cases}
    \end{equation*}
\end{definition}

\begin{example}\label{example_3.2}
    Consider the function $\psi:[0,1] \xrightarrow{} \R$ given by
    $\psi(x)=\phi(x)+x$, where $\phi$ is the Cantor-Lebesgue function. Then
    $\psi$ is strictly increasing and maps a measurable subset  $A \subseteq
    [0,1]$ to a nonmeasurable set $\psi(A)$. Extending $\psi$ to the function
    $\Psi:\R \xrightarrow{} \R$, $\inv{\Psi}$ is continuous, and hence,
    measurable. Now, since $A$ is also measurable, so is the characteristic
    function for  $A$,  $\chi_A$. However, let $I$ be an open interval with  $1
    \in I$ but  $0 \notin I$. Then $\inv{(\chi_A \circ
    \inv{\Phi})}(A)=\Phi(\inv{\chi_A}(I))=\Psi(A)$. Since $\Psi$ is an extension
    of  $\psi$,  $\Psi(A)$ is nonmeasurable, so that the function $\chi_A \circ
    \inv{\Psi}$ is nonmeasurable; despite being the composition of two
    measurable functions.
\end{example}

\begin{lemma}\label{3.1.6}
    Let $g$ a measurable function on a measurable $E$ and  $f$ a continuous
    function on $\R$. Then  $f \circ g$ is measurable in  $E$.
\end{lemma}
\begin{proof}
    Let $U$ be open in  $\R$, by continuity,  $\inv{f}(U)=V$ is open, and since
    $g$ is measurable,  $\inv{g}(V)=\inv{g}(\inv{f}(U))=\inv{(f \circ g)}(U)$ is
    measurable, which makes $f \circ g$ measurable.
\end{proof}
\begin{corollary}
    If $f$ is measurable, then so is the function  $|f|^p$ on $E$, for all
    $p>0$.
\end{corollary}

\begin{lemma}\label{3.1.7}
    For a finite collection $\{f_k\}_{k=1}^n$ of measurable functions with
    common measurable domain $E$, the functions  $\bar{f}=\max{\{f_1, \dots,
    f_n\}}$ and $\bbar{f}=\min{\{f_1, \dots, f_n\}}$ are measurable.
\end{lemma}
\begin{proof}
    For all $c \in E$, notice that  $\{x \in E : \bar{f}>c\}=\bigcup_{k=1}^n{\{x
    \in E : f_k>c\}}$ and $\{x \in E : \bbar{f}>c\}=\bigcup_{k=1}^n{\{x
    \in E : f_k<c\}}$.
\end{proof}
