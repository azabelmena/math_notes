\section{Sequential Pointwise Limits, and Simple Approximation}

\begin{definition}
    Let $\{f_n\}$ a sequence of functions on a common domain $E$, and  $f$ a
    function on $E$. Let $A \subseteq E$. We say that  $\{f_n\}$
    \textbf{converges pointwise} to $f$ on  $A$ provided that
    $\lim{f_n(x)}=f(x)$ on $A$ for all $x \in A$ as $n \xrightarrow{} \infty$.
    We write $\{f_n\} \xrightarrow{pointwise} f$, or simply $\{f_n\}
    \xrightarrow{} f$. We sya $\{f_n\}$ converges \textbf{uniformly} to $f$ if
    for every  $\e>0$, there is an  $N>0$ for which
    \begin{equation*}
        |f-f_n|<\e \text{ for all } n \geq N
    \end{equation*}
\end{definition}

\begin{lemma}\label{3.2.1}
    If a sequence $\{f_n\}$ of measurable functions with common measurable
    domain $E$ converge pointwise almost everywhere to  $f$ on  $E$, then  $f$
    is measurable.
\end{lemma}
\begin{proof}
    Let $E_0 \subseteq E$ with $m(E_0)=0$, and suppose that $\{f_n\}
    \xrightarrow{pointwise} f$ on $\com{E}{E_0}$. Then $f$ is measurable if, and
    only if $f_|{\com{E}{E_0}}$ is measurable. Hence, suppose that $\{f_n\}
     \xrightarrow{} f$ on all $E$.

     Let $c \in \R$ and observe for all  $x \in E$, since  $\lim{f_n}=f$, then
     $f(x)<c$ if, and only if there exists $n,k \in \Z^+$ such that
     $f_j(x)<c-\frac{1}{n}$ for all $j \geq k$. Thensince  $f_j$ is measurable,
     we get  $\{x \in E : F_j<c-\frac{1}{n}\}$ is measurable, and for all $k$,
     \begin{equation*}
         \bigcap_{j=k}{\{x \in E : f_j<c-\frac{1}{n}\}}
     \end{equation*}
     is measurable. Then notice that
     \begin{equation*}
     \{x \in E : f<c\}=\bigcup{(\bigcap_{j=k}{\{x \in E : f_j<c-\frac{1}{n}}\})}
     \end{equation*}
\end{proof}

\begin{definition}
    A realvalued function $\phi$ on a measurable domain  $E$ is said to be
    \textbf{simple} if it is measurable, and takes only finitely many values. If
    $\phi$ takes the values  $c_1 ,\dots, c_n$, we define the \textbf{canonical
    representation} iof $\phi$ to be the representation of the form
    \begin{equation*}
        \phi=\sum{c_k\chi_{E_k}}
    \end{equation*}
    where $E_k=\inv{\phi}(c_k)$.
\end{definition}

\begin{lemma}[The Simple Approximation Lemma]\label{3.2.2}
    Let $f$ be a measurable function bounded on its domain  $E$. The for every
    $\e>0$, there exists simple functions  $\phi_\e$ and  $\psi_e$ on  $E$ for
    which
    \begin{equation*}
        \phi_\e \leq f < \psi_\e \text{ and } 0 \leq \psi_\e-\phi_e<\e
    \end{equation*}
\end{lemma}
\begin{proof}
    Let $(c,d)$ be the open bounded interval containing $f(E)$,a nd let
    \begin{equation*}
        c=y_0<y_1<\dots<y_{n-1}<y_n=d
    \end{equation*}
    be a partition of $[c,d]$ such that $y_k-y_{k-1}<\e$ for all  $1 \leq k \leq
    n$. Define $I_k=(y_{k-1},y_k)$, and $E_k=\inv{f}(I_k)$. Since $f$ is
    measurable, so is each  $E_k$. Noiw, define  $\phi_\e$ and  $\psi_e$ by
    \begin{align*}
        \phi_\e     &=  \sum_{n=1}^n{y_{k-1}\chi_{E_k}} \\
        \psi_\e     &=  \sum_{n=1}^n{y_k\chi_{E_k}} \\
    \end{align*}
    Then $\phi_\e$ and  $\psi_\e$ are simple functions. Then for  $x \in E$,
    there exist a unique  $1 \leq k \leq n$ such that  $y_{k-1} \leq f(x) \leq
    y_k$. So that $\phi_e=y_{k-1} \leq f(x)<y_k=\psi_\e$. Moreover, since 3each
    $y_k-y_{k-1}<\e$, we get $0 \leq \psi_\e-\phi_\e<\e$.
\end{proof}

\begin{theorem}[The Simple Approximation Theorem]\label{3.2.3}
    An extended realvalued function $f$ on a measurable domain  $E$ is
    measurable if, and only if there exists an sequece  $\{\phi_n\}$ on $E$, of
    simple functions such that  $\{\phi\} \xrightarrow{pointwise} f$ and
    $|\phi_n| \leq |f|$ on  $E$ for all  $n$.
\end{theorem}
\begin{proof}
    Since simple functions are measurable, by definition, $\{\phi_n\}
    \xrightarrow{} f$ implies that $f$ is also measurable.

    Conversly suppose that  $f$ is measurable, and that  $f \geq 0$ on  $E$. Let
     $n \in \Z^+$ and define  $E_n=\{x \in E : f \leq n\}$. Then $E_n$ is
     measurable, and  $f|_{E_n}$ is measurable, nonnegative, and bounded. By the
     simple approximation lemma, choose $\e=\frac{1}{n}$ and take $\phi_n$,
     $\psi_n$ simple functions on  $E$ such that
     \begin{equation*}
         \phi_n \leq f \leq \psi_n \text{ and } 0 \leq \psi_n-\phi_n<\frac{1}{n}
     \end{equation*}
     Then observe that $0 \leq \phi_n \leq f$ and  $0 \leq f-\phi_n \leq
     \psi_n-\phi_n<\frac{1}{n}$ on $E_n$.a So that  $0 \leq
     f-\phi_n<\frac{1}{n}$. Now, extend $\phi_n$ to a function  $\Phi_n$ on
     $E$, defined by
     \begin{equation*}
         \Phi_n(x)=0 \text{ if } f(x)>n \text{ and } \Phi_n=\phi_n \text{
         otherwise}
     \end{equation*}
     Then $\Phi_n$ is a simple function on  $E$ with  $0 \leq \Phi \leq f$ on
     $E$. Now, let  $x \in E$, if  $f(x)$ is finite, choose an  $N>0$ such that
     $f<N$. Then  $0 \lwq f-\Phi_n<\frac{1}{n}$ for all $n \geq N$, making
     $\lim{\Phi_n}=f$. On the otherhand, if $f(x)$ is infinite then $\Phi(x)=n$
     for all $n$ so that  $\lim{\Phi_n}=f$.
\end{proof}
