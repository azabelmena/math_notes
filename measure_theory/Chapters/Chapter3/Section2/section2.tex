%----------------------------------------------------------------------------------------
%	SECTION 1.1
%----------------------------------------------------------------------------------------

\section{Subsequences}

\begin{definition}
    Let $\{x_n\}$ be a sequence, and let  $\{n_k\} \susbseteq \Z^+$ such that 
    $n_k<n_{k+1}$. We call the sequence $\{x_{n_k}\}$ a \textbf{Subsequence} of $\{x_n\}$. If 
    $\{x_{n_k}\}$ converges, we call its limit the \textbf{subsequential limit} of $\{x_n\}$.
\end{definition}

\begin{theorem}\label{3.2.1}\
    A sequence $\{x_n\}$ converges to a point  $x$ if  and only if every subsequence  
    $\{x_{n_k}\}$ converges to $x$.
\end{theorem}
\begin{proof}
    Clearly if $x_n \rightarrow x$, then every subsequence  $x_{n_k} \rightarrow x$, (since subsequences 
    can be thought of as subsets of thier parent sequences). On the other hand, let 
    $x_{n_k} \rightarrow x$ for $\{\n_k\} \subseteq \Z^+$. Then for  $\epsilon>0$, there is a 
    $K \in \Z^+$ for which  $d(x_{n_k},x)<\frac{\epsilon}{2}$ for $k \geq K$. Let $N \in \Z^+$, and 
    choose  $n \geq max\{N,K\}$, then  $d(x_n,x) \leq d(x_n,x_{n_k})+d(x_{n_k},d)<\epsilon$.
\end{proof}

\begin{theorem}\label{3.2.2}
    If $\{x_n\}$ is a sequence in a compact metric space  $X$, then some subsequence of 
    $\{x_n\}$ converges to a point  $x$.
\end{theorem}
\begin{proof}
    If $\{x_n\}$ is finite, then thre is an  $x \in \{x_n\}$ and a sequence  $\{n_k\}$ with 
    $n_k<n_{k+1}$ such that  $x_{n_i}=x$ for  $1 \leq i \leq k$, then the subsequence converges 
    to  $x$.

    Now if  $\{x_n\}$ is infinite, there is a limit point  $x \in X$ of $\{x_n\}$, then 
    choose  $n_i$ such that  $d(x,x_i)<\frac{1}{i}$ for $1 \leq i \leq k$. Obtaining  $\{n_k\}$ 
    from this, we see that  $n_k<n_{k+1}$, and so we get that  $\{x_{n_k}\}$ converges to  $x$.
\end{proof}
\begin{corollary}
    Every bounded sequence in $\R^k$ contains a convergent subsequence.
\end{corollary}

\begin{theorem}\label{3.2.3}
    The subsequential limits of $\{x_n\}$ is a metric space  $X$ form a closed subset of  
    $X$.
\end{theorem}
\begin{proof}
    Let $E$ be the set of all subsequential limits of $\{x_n\}$, and let  $x$ be a limit point 
    of  $E$. Choose $n_i$ such that $x_{n_i} \neq x$ and let  $\delta=d(x,x_{n_i})$, for 
    $1 \leq i \leq k$. Then consier the sequence  $\{n_k\}$, since  $x$ is a limit point of  $E$, 
    there is an  $x' \in E$ for which  $d(x,x')<\frac{\delta}{2^i}$. Thus there is an $N_I>n_i$ 
    such that $d(x',x_{n_i})<\frac{\delta}{2^i}$, thus $d(x,x_{n_i})<\frac{\delta}{2^i}$. So 
    $\{x_n\}$ converges to  $x$ and  $x \in E$.
\end{proof}
