\section{Smooth Manifolds}

\begin{definition}
    We call a map $f:\R^n \xrightarrow{} \R^n$ \textbf{$q$-smooth}, or $C^q$, if
    it has continuous partial derivatives of order  $q$. We call  $f$
    \textbf{smooth}, or $C^\infty$, if it has continuous partial derivatives of
    all orders.
\end{definition}

\begin{definition}
    A \textbf{$C^q$-manifold}, wit $q>0$ is a topological manifold with an
    atlas that is  $C^q$. That is, for any charts  $(M_\a,\phi_\a)$ and
    $(M_\b,\phi_\b)$, $\phi_\b \circ \inv{\phi_\a}$ is $C^q$ wherever it is
    defined. We call  $C^\infty$-manifolds  \textbf{smooth manifolds}, or
    \textbf{differentiable manifolds}.
\end{definition}

\begin{example}\label{example_1.6}
    \begin{enumerate}
        \item[(1)] $\R^n$ is a smooth manifold, as are all its open subsets.

        \item[(2)] Consider the $n$-manifold  $S^n$ with charts
            \begin{align*}
                (\com{S^n}{(0, \dots, 0,1)},h) && (\com{S^n}{(0, \dots, 0,-1)},h')  \\
            \end{align*}
            where
            \begin{equation*}
                h(x_1, \dots,x_{n+1})=\frac{1}{1-x_{n+1}}(x_1, \dots x_n) \text{
                and } h'(x_1, \dots,x_{n+1})=\frac{1}{1+x_{n+1}}(x_1, \dots x_n)
            \end{equation*}
            The map $h' \circ \inv{h}$ is smooth. Notice that
            \begin{equation*}
                \inv{h}(y_1, \dots,y_n)=\Big{(}\frac{2y_1}{1+y_1^2+\dots+y_n^2},
                \dots, \frac{2y_n}{1+y_1^2+\dots+y_n^2}\Big{)}
            \end{equation*}
            So that
            \begin{equation*}
                h' \circ \inv{h}=\frac{1}{y_1^2+\dots+y_n^2}(y_1, \dots, y_n)
            \end{equation*}
            Moreover, for all $q>0$, $\partial^q{h' \circ \inv{h}}$ exists,
            which makes $h' \circ \inv{h}$ smooth. This makes $S^n$ a smooth
            manifold.

        \item[(3)] The product of smooth manifolds are smooth manifolds. In
            particular, the torus $T^2=S^1 \times S^1$ is a smooth manifold.
    \end{enumerate}
\end{example}

\begin{definition}
    Let $M$ and $N$ manifolds with atlases  $\{(M_\a,\phi_\a)\}$ and $\{(N_\b,
    \psi_\b)\}$. We call a map $f:M \xrightarrow{} N$ \textbf{$q$-smooth}, or
    $C^q$ if  $\psi_\b \circ \inv{\phi_\a}$ is $C^q$ wherever it is defined. We
    call  $C^q$-maps between manifolds  \textbf{$C^q$-diffeomorphisms}. We call
    $C^\infty$-diffeomorphisms  \textbf{diffeomorphisms}. We call any two
    $C^q$-manifolds  \textbf{diffeomorphic} if there exists a
    $C^q$-diffeomorphism between them.
\end{definition}

\begin{example}\label{example_1.7}
    \begin{enumerate}
        \item[(1)] The map $f:\R \xrightarrow{} \R$ given by $f(x)=x^3$ is a
            smooth map, but it is not a diffeomorphism, since $f'(x)=3x^2$ has a
            singular point at $0$ (elaborate?). It is not even a
            $C^1$-diffeomorphism.

        \item[(2)] The projection map of $T^2=S^1 \times S^1$ onto the second
            factor is a smooth map between manifolds.
    \end{enumerate}
\end{example}

\begin{definition}
    LEt $M$ a  $C^q$-manifold, for some  $q \geq 1$, and let  $x \in M$ and
    $(M_\a,\phi_\a)$ a chart containing $x$. We call  $x$ a  \textbf{critical
    point} of a map $f:M -< \R$ if it is a critical point of  $f \circ
    \inv{\phi_\a}$. If $g:\R^n \xrightarrow{} \R^n$ is a map, we call $x$ a
     \textbf{nondegenerate} critical point if the Hessian of $g$ is nonsingular
     at  $x$, and we call  $x$ a  \textbf{nondegenerate} critical point of $f$
     if it is a nondegenerate critical point of $f \circ \inv{\phi_\a}$.
\end{definition}

\begin{definition}
    We define a \textbf{Morse function} on a manifold $M$ to be a smooth map
    $f:M \xrightarrow{} \R$ such that
    \begin{enumerate}
        \item[(1)] $f$ has nondegenerate critical points.

        \item[(2)] Distinct critical points map to distinct values.
    \end{enumerate}
\end{definition}

\begin{example}\label{1.8}
    The projection map of the Torus $T^2 \subseteq \R^3$ on to the thrid
    coordinate is a map with critical points. It has $1$ maximum value,  $2$
    minimum values, and  $2$ saddle points. Moreover these critical points are
    nondegenerate, so that the projection is a Morse function.
\end{example}
