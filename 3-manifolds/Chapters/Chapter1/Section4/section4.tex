\section{Splitting Fields}

\begin{definition}
    Let $K$ be an extension of a field  $F$. We say a polynomial  $f$ over  $F$
     \textbf{splits completely} over $K$ if  $f$ factors into linear factors
     over  $K$. If  $f$ splits completely over  $K$, and in no other proper
     subfield, then we say  $K$ is the  \textbf{splitting field} of $f$ over
     $F$.
\end{definition}

\begin{theorem}\label{1.4.1}
    If $f$ is a polynomial over a field  $F$, then there exists a splitting
    field  $K$ of  $f$ over  $F$.
\end{theorem}
\begin{proof}
    Let $E$ an extension of  $F$ with  $[E:F]=n$. By induction on $n$, for
    $n=1$, we take  $E=F$ and we are done. Now, for  $n \geq 1$, suppose the
    irreducible factors of  $f$ are of $\deg=1$. Then $f$ has all its roots in
    $F$, and hence splits completely over $F$. Then take $E=F$. On the other
    hand, if  $f$ has at least one irreducible factor of $\deg \geq 2$, then
    there is an extension $E_1$ of $F$ for which $f$ has the factor $(x-\a)$ for
    some root $\a$. Then $f(x)=(x-\a)f_1(x)$ where $\deg{f_1}=n-1$. Therefore by
    the induction hypothesis, there is an extension $E$ of  $E_1$ containing all
    the roots of $f_1$. Hence, it contains all the roots of $f$ and  $f$ splits
    completely over  $E$.

    Now, let $K$ be the intersection of all subfields of $E$ for which $f$
    splite; i.e. all subfields containing the roots of $f$. Then by definition,
     $K$ is the splitting field of  $f$ over  $F$.
\end{proof}

\begin{definition}
    If $K$ is an algebraic extesnion of  $F$ such that it is the splitting field
    for a collection of polynomials over  $F$, then we say that  $K$ is a
     \textbf{normal extension} of $F$.
\end{definition}

\begin{example}\label{example_1.11}
    \begin{enumerate}
        \item[(1)] The splitting field of $x^2-2$ over  $\Q$ is  $\Q(\sqrt{2})$,
            since $x^2-2=(x+\sqrt{2})(x-\sqrt{2})$ and $\pm{\sqrt{2}} \in
            \Q(\sqrt{2})$ and $[\Q(\sqrt{2}):\Q]=2$, so there is no other
            subfield in between.

        \item[(2)] The splitting field for
            $(x^2-2)(x^2-3)=(x+\sqrt{2})(x-\sqrt{2})(x+\sqrt{3})(x-\sqrt{3})$ is
            $\Q(\sqrt{2},\sqrt{3})$. Now, $[\Q(\sqrt{2},\sqrt{3}):Q]=4$ and the
            lattice of fields is
            \[\begin{tikzcd}
                & {\Q(\sqrt{2},\sqrt{3})} \\
                {\Q(\sqrt{2})} & {\Q(\sqrt{6})} & {\Q(\sqrt{3})} \\
                & {\mathbb{Q}}
                \arrow["2", no head, from=1-2, to=2-2]
                \arrow["2", no head, from=2-2, to=3-2]
                \arrow["2"', no head, from=3-2, to=2-3]
                \arrow["2"', no head, from=2-3, to=1-2]
                \arrow["2"', no head, from=1-2, to=2-1]
                \arrow["2"', no head, from=2-1, to=3-2]
             \end{tikzcd}\]

         \item[(3)] Let $\xi=i\frac{\sqrt{3}}{2}$. Notice that $x^3-2$ factors
             into
             $x^3-2=(x-\sqrt[3]{2})(x+\sqrt[3]{2}(-1+\xi))(x+\sqrt[3]{2}(-1-\xi))$.
             Now, $-1+\xi,-1-\xi \notin \Q(\sqrt[3]{2})$, so $\Q(\sqrt[3]{2})$
             is not the splitting field for $x^3-2$. Let  $K$ be the splitting
             field of  $x^3-2$. Then  $K$ conmtains $-1 \pm \xi$, so that
             $i\sqrt{3} \in K$. Thus
             \begin{equation*}
                 K=\Q(\sqrt[3]{2},i\sqrt{3})
             \end{equation*}
             Moreover, $[\Q(\sqrt[3]{2},i\sqrt{3}):\Q(\sqrt[3]{2})] \geq 2$
             and since $\Q(\sqrt[3]{2})$ is not the splitting field,
             $[\Q(\sqrt[3]{2},i\sqrt{3}):\Q(\sqrt[3]{2})]=2$. Hence
             $[\Q(\sqrt[3]{2},i\sqrt{3}):\Q]=6$. We have the following lattice.
             \[\begin{tikzcd}
                & {\Q(\sqrt[3]{2},i\sqrt{3})} \\
                && {\Q(\sqrt[3]{2})} & {\Q(\sqrt[3]{2}(-1+\xi))} & {\Q(\sqrt[3]{2}
                    (-1-\xi))} \\
                {\mathbb{Q}(i\sqrt{3})} \\
                & {\mathbb{Q}}
                \arrow[no head, from=1-2, to=3-1]
                \arrow[no head, from=3-1, to=4-2]
                \arrow[no head, from=4-2, to=2-4]
                \arrow[no head, from=2-4, to=1-2]
                \arrow[no head, from=1-2, to=2-3]
                \arrow[no head, from=2-3, to=4-2]
                \arrow[no head, from=1-2, to=2-5]
                \arrow[no head, from=2-5, to=4-2]
            \end{tikzcd}\]

        \item[(4)] Notice that $x^4+4=(x^2+2x+2)(x^2-2x+2)$ over $\Q$ which is
            irreducible by Eisenstein's criterion. Using the quadratic formula,
            we get $\pm{1}$ and $\pm{i}$ as the roots, moreover, notice that
            $\pm{1},\pm{i} \in \Q(i)$ and since $[\Q(i):\Q]=2$ there are no
            subfields between $\Q$ and  $\Q(i)$ so that $\Q(i)$ is the splitting
            field of $x^4+4$ over  $\Q$.
    \end{enumerate}
\end{example}

\begin{lemma}\label{1.4.2}
    A splitting field of a polynomial of degree $n$ over a field  $F$ is of
    degree at most  $n!$ over  $F$.
\end{lemma}
\begin{proof}
    Let $f \in F[x]$ a polynomial of $\deg{f}=n$. Adjoining one root of $f$ to
    $F$, we have an extension $\faktor{F_1}{F}$ of degree $[F_1:F]=n$. Now,$f$
    over $F_1$ has at leas one linear factor, and so any root of $f$ satisfies a
    polynomial of degree $n-1$. Hence proceeding inductively gives the result.
\end{proof}

\begin{example}\label{example_1.12}
    Consider the polynomial $x^n-1$ over  $\Q$. Then the roots of  $x^n-1$ are
    of the form  $\xi$ where  $\xi^n=1$. Notice, that in  $\C$,
    $\xi=e^{\frac{2i\pi}{n}}$, so that $\C$ contains a splitting field of
    $x^n-1$. Hence  $\Q(\xi) \subseteq \C$ is a splitting field of $x^n-1$ over
     $\Q$. Notice that the set of all roots $\xi$ of  $x^n-1$ forms a cyclic
     group generated by $\xi$.
\end{example}

\begin{definition}
    Consider a field $F$ and the polynomial $x^n-1$ over $F$. We call the roots
     $\xi$ of  $x^n-1$, where  $\xi^n=1$ the  \textbf{primitive $n$-th roots of
     unity} over $F$. We call  $F(\xi)$ the \textbf{cyclotomic field} over $F$.
\end{definition}

\begin{example}\label{exampl_1.13}
    Let $p$ be a prime, and consider the splitting field $x^p-2$ over  $\Q$. If
     $\a$ is a root, then $\a^p=2$ so that $(\xi\a)^p=2$ where $\xi$ is a
     primitive  $p$-th root of unity over  $\Q$. So the roots of $x^2-2$ are
     \begin{equation*}
         \sqrt[p]{2}    \text{ and }  \xi\sqrt[p]{2}
     \end{equation*}
     Notice that $\frac{\xi\sqrt[p]{2}}{\sqrt[p]{2}}=\xi$ so the splitting field
     contains $\Q(\xi,\sqrt[p]{2})$, Moreover, $\Q(\xi,\sqrt[p]{2})$ contains
     all the roots of $x^p-2$ so that  $\Q(\xi,\sqrt[p]{2})$ is the splitting
     field of $x^p-2$ over  $\Q$.

     Notice, that  $\Q(\xi) \subseteq \Q(\xi,\sqrt[p]{2})$ so that
     $[\Q(\xi,\sqrt[p]{2}):\Q(\xi)] \leq p$. not, since $\Q(\sqrt[p]{2})$ is
     also a subfield, we get $[\Q(\xi,\sqrt[p]{2}):Q] \leq p(p-1)$. Since
     $(p,p-1)=1$ (i.e. they are coprime), we have
     $p(p-1)|[\Q(\xi,\sqrt[p]{2}):\Q]$ so that $[p]{2}):\Q]=p(p-1)$. We have the
     follwing lattice.
     \[\begin{tikzcd}
        & {\mathbb{Q}(\xi,\sqrt[p]{2})} \\
        {\mathbb{Q}(\xi)} && {\mathbb{Q}(\sqrt[p]{2})} \\
        & {\mathbb{Q}}
        \arrow[no head, from=1-2, to=2-1]
        \arrow[no head, from=2-1, to=3-2]
        \arrow[no head, from=3-2, to=2-3]
        \arrow[no head, from=2-3, to=1-2]
     \end{tikzcd}\]
\end{example}

\begin{theorem}\label{1.4.3}
    Let $\phi:F \xrightarrow{} F'$ a field isomorphism. Let $f$ and  $f'$
    polynomials over  $F$ and  $F'$, where  $f'$ is obtained by applying  $\phi$
    to the coefficients of  $f$. Let  $E$ and  $E'$ be splitting fields of $f$
    and  $f'$ over $F$ and $F'$, respectively. Then $\phi$ extends to an
    isomorphism betweenn $E$ and $E'$; i.e. $E \simeq E'$.
    \[\begin{tikzcd}
        E & {E'} \\
        F & {F'}
        \arrow[from=1-1, to=1-2]
        \arrow["\phi", from=2-1, to=2-2]
        \arrow[no head, from=1-2, to=2-2]
        \arrow[no head, from=1-1, to=2-1]
        \end{tikzcd}\]
\end{theorem}
\begin{proof}
    Let $\deg{f}=n$. By induction on $n$. If  $f$ has all its roots in $F$, $f$
    splits completely over  $F$, and  $f'$ over  $F'$. Then take  $E=F$ and
    $E'=F'$ and we are done for $n=1$.

    Now, for  $n \geq 1$, suppose the theorem is true. Let $p$ an irreducible
    factor of $f$, and $p'$ an irreducible factor of  $f'$. If  $\a$ and  $\a'$
    are roots of  $p$ and  $p'$, respectively, then extend $\phi$ to  $F(\a)$
    and $F'(\a')$. Then $f(x)=(x-\a)f_1(x)$ and $f'(x)=(x-\a')f_1'(x)$; with
    $\deg{f_1}=\deg{f_1'}=n-1$. Then let $E$ the splitting field of $f_1$ over
    $F(\a)$, and $E'$ the splitting field of  $f_1'$ over $F'(\a')$
    \[\begin{tikzcd}
        E & {E'} \\
        {F(\alpha')} & {F'(\alpha')} \\
        F & {F'}
        \arrow["\phi"', from=3-1, to=3-2]
        \arrow[from=2-1, to=2-2]
        \arrow[from=1-1, to=1-2]
        \arrow[no head, from=1-1, to=2-1]
        \arrow[no head, from=2-1, to=3-1]
        \arrow[no head, from=1-2, to=2-2]
        \arrow[no head, from=2-2,to=3-2]
     \end{tikzcd}\]
     The the roots of $f_1$ and  $f_1'$ are in  $E$ and  $E'$, respectively, and
     hence so are the roots of $f$ and  $f'$. Then by the induction hypothesis,
     we can extend  $\phi$ to  $E$ and  $E'$ so that  $E \simeq E'$.
\end{proof}
\begin{corollary}
    Any two splitting fields of a given polynomial over a field are isomorphic.
\end{corollary}
\begin{proof}
    Take $\phi$ to be the identity map.
\end{proof}
