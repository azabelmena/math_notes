\section{Linear Block Codes}

\begin{definition}
  A \textbf{$q$-ary $[n,k]$-linear block code} $C$ is a $k$ dimensional
  subspace of the vector space $\F_q^n$. If $C$ has minimum distance $d$,
  then we call $C$ an \textbf{$[n,k,d]$-linear block code}.
\end{definition}

\begin{definition}
  Let $C$ be a  $q$-ary  $[n,k]$-linear block code. We define the
  \textbf{generator matrix} $G$ for $C$ to be the $k \times n$ matrix whose
  row space forms a basis for $C$. We say that $G$ is in  \textbf{standard
  form}  if $G=(I_{k \times k} | P)$, where $I_{k \times k}$ is the $k
  \times k$ identity matrix over $\F_q$. If $G$ is in standard form, we
  call the first $k$ entries in any row  \textbf{information symbols}, and
  the remaining $n-k$ entries  \textbf{parity-check symbols}.
\end{definition}

\begin{definition}
  We call two $[n,k]$-linear codes $C_1$ and $C_2$ \textbf{equivalent} if
  for any codeword $(c_1, \dots, c_n) \in C_1$, there exists a permutation
  $\pi:i \xrightarrow{} j$ for which $(c_{\pi(1)}, \dots, c_{\pi(n)}) \in
  C_2$.
\end{definition}

\begin{lemma}\label{lemma_1.2.1}
  Ever linear code is equivalent to a linear code with generator matrix in
  standard form.
\end{lemma}

\begin{definition}
  We call an $[n,k]$-linear code $C$ \textbf{systematic}, or
  \textbf{seperable} if there exists a unique codeword $c \in C$ for all
  possible choice of coordinates in the first $k$ positions.
\end{definition}

\begin{lemma}\label{lemma_1.2.2}
  AN $[n,k]$-linear code is seperable on at least one $k$-tuple over
  $\F_q$.
\end{lemma}

\begin{lemma}\label{lemma_1.2.3}
  An $[n,k]$-linear code has rate $\frac{k}{n}$.
\end{lemma}
\begin{proof}
  Observe since $C \subseteq \F_q^n$ is a subspace of dimension $k$,
  $|C|=q^k$. So the rate of $C$ gives:
  \begin{equation*}
    R=\frac{\log_q{q^k}}{n}=\frac{k}{n}
  \end{equation*}
\end{proof}

\begin{lemma}\label{lemma_1.2.4}
  IIn a linear block code $C$, the minimum distance of  $C$ is equal to the
  minimum weight of $C$.
\end{lemma}
\begin{proof}
  For every $x,y \in C$, $x-y \in C$, and we have $d(x,y)=d(x-y,0)=w(x-y)$.
\end{proof}

\begin{definition}
  Let $C$ be an $[n,k]$-linear code. We define the \textbf{dual} of $C$ to
  be the set:
  \begin{equation}\label{equation_1.7}
    C^\perp=\{y \in \F_q^n : \langle x,y \rangle=0 \text{ for all } x \in C\}
  \end{equation}
  We call $C$ a  \textbf{self-dual code} if $C=C^\perp$.
\end{definition}

\begin{lemma}\label{1.2.5}
  The dual of an $[n,k]$-linear code with generator matrix $G=(I_{k \times
  k} | P)$ is an $[n,n-k]$-linear code with generator matrix:
  \begin{equation}\label{equation_1.8}
    H=(-P^T | I_{(n-k) \times (n-k)})
  \end{equation}
\end{lemma}

\begin{definition}
  oLet $C$be an  $[n,k]$-linear code. We call an $(n-k) \times n$ matrix
  $H$ a  \textbf{parity-check matrix} for $C$, if it is the generator
  matrix for the dual code $C^\perp$. If $x \in \F_q^n$, we call $xH^T$ the
  \textbf{syndrome} of $x$.
\end{definition}

\begin{lemma}\label{lemma_1.2.6}
  The covering radius $\p$ of an $[n,k]$-linear code, with parity-check
  matrix $H$ is the smallest positive integer for which any word in
  $\F_q^{n-k}$ can be written as the sum of at most $\p$ columns of $H$.
\end{lemma}

\begin{lemma}\label{lemma_1.2.7}
  Let $C$ be a linear code with parity-check matrix $H$, and let $c \in
  \F_q^n$. Then $c \in C$ if, and only if $cH^T=0$.
\end{lemma}

\begin{definition}
  Let $C$ be an  $[n,k]$-linear code. We say that two words $x,y \in
  \F_q^n$ are in the same \textbf{coset} if, and only if they share the
  same syndrome, and we call the representatives of those cosets (as group
  cosets) \textbf{coset leaders}.
\end{definition}
