%----------------------------------------------------------------------------------------
%	SECTION 1.1
%----------------------------------------------------------------------------------------

\section{Definitions and Examples.}
\label{section1}

\begin{definition}
    Let $G$ be a group. We call a nonempty subset $H \subseteq G$ a
    \textbf{subgroup} if $H$ is also a group under the binary operation of  $G$.
    We write  $H \leq G$; if $H \neq G$, then we write $H < G$.
\end{definition}

There are two immediate results that we can develop.

\begin{theorem}[The Subgroup Critetrion.]\label{2.1.1}
    Let $G$ be a group. Then a subset  $H \subseteq G$ is a subgroup if, and
    only if:
    \begin{enumerate}
        \item[(1)] For every $a,b \in H$,  $ab \in H$.

        \item[(2)] $\inv{a} \in H$ whenever $a \in H$.
    \end{enumerate}
\end{theorem}
\begin{proof}
    Suppose first that $H \leq G$, then by definition, $(1)$ and $(2)$ are
    satisfied.

    Now suppose that $H$ is closed under the operation on  $G$,  $(1)$, and that
    $H$ has inverses  $(2)$. Immediately, the closure and inverse laws are
    satisfied, moreover, since $H \subseteq G$,  $H$ inherits the associativity
    of  $G$ under the relavent operation. Now, by $(1)$ and $(2)$ we have
    $a\inv{a}=e \in H$, and so the identity law is satisfied. This makes $H \leq
    G$.
\end{proof}
\begin{corollary}
    If $H$ is a finite subset of  $G$, then  $H$ is a subgroup if  $H$ is closed
    under the operation of  $G$.
\end{corollary}
\begin{proof}
    Let $a \in H$, by closure, we have  $a^n \in H$ for $n \in \Z^+$. So,
    consider the infinite collection $\{a^i\}_{i=1} \subseteq H$; since $H$ is
    finite, there are repetitions in the collection  $\{a^i\}$, that is, $
    a^i=a^j$ for some  $i \neq j$ and  $i,j>0$. Then  $a^{i-j}=e$, so
    $a^{i-j-1}=\inv{a}$. Now, since $i-j>0$, we get  $i-j-1 \geq 0$, which makes
     $\inv{a} \in H$. By the above theorem, we get $H \leq G$.
\end{proof}

\begin{example}
    \begin{enumerate}
        \item[(1)] $\Z \leq \Q \leq \R$ under the usual addition.  $\Q^*
            \leq \R^*$ under the usual multiplication. $\R \leq \C$ under
            complex addition, and $\R^* \leq \C^*$ under complex multiplication
            (here, we take $a \in \R$ to have the form  $a+i0$).

        \item[(2)] For any group $G$,  $G \leq G$ and  $\vbrack{e} \leq G$. So
            the minimum number of subgroups that any group has is $2$.

        \item [(3)] Let $H=\{e, r, \dots r^{n-1}\} \subseteq D_{2n}$. Then $H \leq
            D_{2n}$.

        \item[(4)] Let $n\Z=\{na : a \in \Z\}$, for any $n \in \Z$. Then
            $n\Z \leq \Z$ under the usual addition. Let  $na, nb \in n\Z$, then
            $na+nb=n(a+b) \in n\Z$, and $n(-a)=-na \in n\Z$. We will be
            interested in the subgroup $n\Z$ of $\Z$, in particular, for  $n \in
            \Z^+$.

        \item [(5)] Let $Z=\{z \in \C : z^n=1\}$. Then $Z \leq \C^*$ under
            complex multiplication. Notice that  $z,w \in Z$ implies
            $z^nw^n=(zw)^n=1$, and $(\inv{z})^n=\frac{1}{z^n}=1$, so $z^{-n}=1$.
            We call this group the \textbf{roots of unity} in $\C$. If we take
            $n=4$, then we get that  $Z=\{1,i,-1,-i\}$.

        \item[(6)] Let $\Z+i\Z=\{a+ib \in \C : a,b \in \Z\}$. Then $\Z+i\Z \leq
            \C$ under complex addition. We call this group the \textbf{Gaussian
            integers.}

        \item[(7)] Leet $\Q+\sqrt{2}\Q=\{a+b\sqrt{2} : a,b \in \Q\}$. Then
            $\Q+\sqrt{2}\Q \leq \R$ under the usual addition.
    \end{enumerate}
\end{example}

We now give some nonexamples of subgroups.

\begin{example}
    \begin{enumerate}
        \item[(1)] $\Q^* \not\leq \R$ under the usual addition  (why?).

        \item[(2)] $\Z^+ \not\leq \Z$ under the usual addition. Notice the
            identity of $\Z$, $0 \not\in \Z^+$.

        \item[(3)] $D_6 \not\leq D_8$. Notice
            $D_6=\{e,t,r,r^2,rt,r^2t\}$, and $D_8=\{e,t,r,r^2,r^3,rt,r^2t,
            r^3t\}$. One might be tempted to think $D_6 \subseteq D_8$, but
            notice that in $D_6$, $r^3=e$ while in  $D^8$,  $r^4=e$. Thus  $D_6
            \not\subseteq D_8$.
    \end{enumerate}
\end{example}
