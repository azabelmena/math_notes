\section{Algebraic Subsets of The Plane}

\begin{lemma}\label{1.6.1}
    Let $k$ be a field, and let $f,g \in k[x,y]$ polynomials with no common
    factor. Then the set $V(f,g)=V(f) \cap V(g)$ is a finite set of points.
\end{lemma}
\begin{proof}
    Notice that if $f$ and $g$ are coprime in $k[x,y] \simeq k[x][y]$, then they
    are coprime in $k(x)[y]$, where $k(x)$ is the field of fractions of $k[x]$.
    We have that $k(x)[y]$ is a PID, and that the ideal $(f,g)=(1)$. Then there
    exist $r,s \in k(x)[y]$ for which $rf(x,y)+sg(x,y)=1$. There also exists a
    $d \in k[x]$ such that $d(x)r=a(x,y)$ and and $d(x)s=b(x,y)$ in $k[x,y]$.
    Then $a(x,y)f(x,y)+b(x,y)g(x,y)=d(x)(rf(x,y))+d(x)(rg(x,y))=d(x)$. Now, if
    $A,B \in V(f,g)$, then $d(A)=0$. Now, $d$ has finitely many roots in $k$, so
    that there are finitely many $x$-coordinates corresponding to the points of
    $V(f,g)$. Similarly, in the PID $k(y)[x]$, we get that there are finitely
    many $y$-coordinates corresponding to the points of $V(f,g)$. That is
    $V(f,g)$ have finitely many points.
\end{proof}
\begin{corollary}
    If $f$ is irreducible in $k[x,y]$ and $V(f)$ is finite, then $I(V(f))=(f)$,
    and $V(f)$ is an irreducible algebraic set.
\end{corollary}
\begin{proof}
    Suppose that $g \in I(V(f))$, then $V(f,g)$ is infinite, and by the above
    lemma, we get that $g|f$. Then $g \in (f)$, so that $I(V(f))=(f)$. Moreover,
    since $f$ is irreducible in the $k[x,y]$, if $ab \in (f)$, then either $a
    \in (f)$ or $b \in (f)$, which makes $I(V(f))=(f)$ a prime ideal. This makes
    $V(f)$ irreducible by lemma \ref{1.5.1}.
\end{proof}
\begin{corollary}
    Suppose that $k$ an infinite field, then the irreducible algebraic sets of
    $\A^2(k)$ are $\A^2(k)$ itself, the emptyset, point sets, and irreducible
    plane curves $V(f)$, where $f \in k[x,y]$ is irreducible and $V(f)$ is
    infinite.
\end{corollary}
\begin{proof}
    Let $X \subseteq \A^2(k)$ an irreducible algebraic set. If $X$ is finite, or
    $I(X)=(0)$, then it is either $\A^n(k)$, the emptyset, or a finite algebraic
    set (i.e. a set of points). Suppose then, that $X$ is infinite. Then there
    exists a nonconstant polynomial $f \in I(X)$. Now, since $X$ is irreducible,
    $I(X)$ is prime, and hence contains an irreducible factor of $f$; thus,
    suppose without loss of generality that $f$ is irreducible. Then $I(X)=(f)$;
    for otherwise, if $g \in I(X)$ but $g \not\in (f)$, then $X \subseteq
    V(f,g)$ is finite which is a contradiction. This makes $X=V(f)$ as required.
\end{proof}
\begin{corollary}
    If $k$ is an algebraically closed field, and $f$ has the decomposition
    $f=f_1^{n_1} \dots f_m^{n_m}$ into irreducible factore, then $V(f)=V(f_1)
    \cup \dots V(f_m)$ is the decomposition of $V(f)$ into irreducible
    components. Moreover, $I(V(f))=(f_1 \dots f_m)$.
\end{corollary}
\begin{proof}
    By hypothesis, we have that each $f_i$ and $f_j$ are coprime whenever $i
    \neq j$. That is, there exist no inclusions under each $V(f_i)$, so that the
    decomposition $V(f)=V(f_1) \cup \dots \cup V(f_m)$ is the decompostion of
    $V(f)$ into irreducible components. Now, we also have that
    \begin{equation*}
        I(V(f))=\bigcap_{i=1}^m{I(V(f_i))}=\bigcap_{i=1}^m{(f_i)}
    \end{equation*}
    Now, since each polynomial divisible by $f_i$ is also divisible by $f_1
    \dots f_m$, we get that $\bigcap{(f_i)}=(f_1 \dots f_m)$. Lastly, notice
    that since $k$ is algebraically closed, and hence infinite, each $V(f_i)$ is
    infinite.
\end{proof}

\begin{example}\label{example_1.14}
    \begin{enumerate}
        \item[(1)] Consider $f(x,y)=x^2+y^2+1$ over $\R$. We have that $f$ is
            irreducible, and that $V(f)$ is finite (in fact $V(f)=\emptyset$),
            so that $I(V(f))=(f)$. Moreover, since $f$ has no roots in $\R$, we
            observe that $(f)=(1)$. This result could've also been extracted
            using the fact that $I(V(f))=I(\emptyset)=\R[x,y]=(1)$.

        \item[(2)] Consider $X \subseteq \A^2(\R)$ an algebraic set. Then there
            is some $S=(f_1, \dots, f_n) \in \R[x,y]$ for which $X=V(S)=V(f_1,
            \dots, f_n)=V(f_1) \cap \dots \cap V(f_n)$. Now, by the above
            corollories, and assuming that each $f_i$ is pairwise coprime,
            $X=V(S)$ is a finite set of points. Take
            \begin{equation*}
                f(x,y)=\sum_{i=1}^n{f_i^2(x,y)}
            \end{equation*}
            which has finitely many roots as a polynomial in $\R[x][y]$, and as
            a polynomial in $\R[y][x]$. Then $(f)=(f_1, \dots ,f_n)$ so that
            $X=V(f)$. In geneal, we want to work over algebraically closed
            fields to avoid this, since the intersections of hypersurfaces need
            not be finite.
    \end{enumerate}
\end{example}

\begin{example}\label{example_1.14}
    \begin{enumerate}
        \item[(1)] We have that $V(y^2-xy-x^2y+x^3)=V(x-y) \cup V(x^2+y)$ over
            both $\R$ and $\C$.

        \item[(2)] The set $V(y^2-x(x^2-1))$ is irreducible over both $\R$ and
            $\C$ since the polynomial  $y^2-x(x^2-1)$ is an irreducible
            polynomial over both $\R$ and  $\C$.

        \item[(3)] $V(x^3+x-x^2y-y)$ is irreducible over $\R$, but
            $V(x^3+x-x^2y-y)=V(x-i) \cup V(x+i) \cup V(x-y)$ over $\C$, where
            $i^2=-1$.
    \end{enumerate}
\end{example}
