\section{Eculidian Domains.}

\begin{definition}
    Let $R$ be a commutative ring. We call a map  $N:R \xrightarrow{} \N$, with
    $N(0)=0$ a \textbf{norm}, or, \textbf{degree}. If $N(a) \geq 0$, for all $a
    \in R$, then we call $N$ \textbf{nonnegative} If $N(a)>0$ for all $a \in R$
    then we call  $N$ \textbf{positive}.
\end{definition}

\begin{definition}
    Let $R$ be a commutative ring, and  $N:R \xrightarrow{} \N$ a norm. We say
    thay $R$ is a \textbf{Euclidean domain} if for all $a,b \in R$, with $b \neq
    0$, there exist elements  $q,r \in R$ such that
    \begin{equation*}
        a=qb+r \text{ where } r=0 \text{ or } N(r)<N(b)
    \end{equation*}
    We call $q$ the  \textbf{quotient} and $r$ the  \textbf{remainder} of $a$
    when  \textbf{divided} by $b$.
\end{definition}

\begin{example}\label{2.1}
    \begin{enumerate}
        \item[(1)] Let $F$ be any field, and  $N:F \xrightarrow{} \N$ defined by
            $N(a)=0$ for all $a \in F$. Then  $N$ makes  $F$ into a Euclidean
            domain. Take  $a,b \in F$, with  $b \neq 0$, and  $q=a\inv{b}$. Then
            $a=qb+r$ where  $r=0$.

        \item[(2)] The integers $\Z$ is a Euclidean domain with norm $N(a)=|a|$,
            the absolute value of $a$. In fact, the motivation for Euclidean
            rings comes from the division theorem, or Euclid's theorem for
            integers.

        \item[(3)] Let $F$ be a field, and consider  $F[x]$. Let $N:F[x]
            \xrightarrow{} \N$ be defined by $N:f \xrightarrow{} \deg{f}$. Then
            $f$ is a Euclidean domain. If  $F$ is not a field, then it is not
            necesarily true that $F[x]$ be a Euclidean domain.

        \item[(4)] Let $D \in \Z^+$ be squarefree, and consider  $\Z[\sqrt{D}]$.
            Define $N:\Z[\sqrt{D}]=\N$ to be the absolute value of the field
            norm, that is $N(a+b\sqrt{D})=\|a+b\sqrt{D}\|^2=a^2+Db^2$. We notice
            that $\Z[\sqrt{D}]$ is an integral domain, but it is not a Euclidean
            domain. This depends on our choice of $D$. Let  $D=-1$ so tha t
            $\sqrt{D}=i$, and $i^2=-1$. Then the Gaussian integers, $\Z[i]$, is
            a Euclidean domeain. Let $x=a+ib$,  $y=c+id$ with  $y \neq 0$. In
            $\Q[i]$, the field of fractions, we have that $\frac{x}{y}=r+is$,
            where
            \begin{equation*}
                r=\frac{ac+bd}{\|y\|^2} \text{ and } s=\frac{bc-ad}{\|y\|^2}
            \end{equation*}
            Now, let $p$ and  $q$ be the integers closest to $r$ and $s$,
            respectively so that
            \begin{equation*}
                |r-p| \leq \frac{1}{2} \text{ and } |s-q| \leq \frac{1}{2}
            \end{equation*}
            Let $w=(r-p)+i(s-q)$, and take $z=wy$. Then we have  $z=x-(p+iq)y$,
            so that $x=(p+iq)y+z$, moreover, we have $N(w)=(r-p)^2+(q-s)^2 \leq
            \frac{1}{4}+\frac{1}{4}=\frac{1}{2}$. Since $\|\cdot\|$ is
            multiplicative, we have
            \begin{equation*}
                N(w)N(y) \leq \frac{1}{2}N(y)
            \end{equation*}
            which makes $\Z[i]$ into a Euclidean domain.

        \item[(5)] Let $K$ be a field. We define a  \textbf{discrete valuation}
            to be a map $\nu:K^\ast \xrightarrow{} \Z$ such that
            \begin{enumerate}
                \item[(i)] $\nu(ab)=\nu(a)+\nu(b)$.

                \item[(ii)] $\nu$ is onto.

                \item[(iii)] $\min{\{\nu(x),\nu(y)\}} \leq \nu(x+y)$, for all
                    $x,y \in K^\ast$ for which  $x+y \neq 0$.
            \end{enumerate}
            We call the set $\nu{K}=\{x \in K^\ast : \nu(x) \geq 0\}$ the
            \textbf{valuation ring} of $\nu$ and is a subring of  $K^\ast$. We
            call an integral domain  $R$ a  \textbf{discrete valuation ring} if
            there exists a discrete valuation $\nu$ on the field of fractions of
             $R$, having  $\nu{R}$ as its valuation ring.

             It can be shown that discrete valuation rings are Euclidean
             domains by the norm $N:0 \xrightarrow{} 0$ and $N=\nu$ on all
             $R^\ast$.
    \end{enumerate}
\end{example}

\begin{lemma}\label{2.1.1}
    Every ideal in a Euclidean domain $R$, is a principle ideal.
\end{lemma}
\begin{proof}
    If $I=(0)$, we are done. Now, let $N:R \xrightarrow{} \N$ be the norm of $R$,
    and consider the image $N(I)=\{N(a) : a \in I\}$. By the well ordering
    principle, $N(I)$ has a minimum element $N(d)$ for some $d \neq 0$ in  $I$
    Notice that  $(d) \subseteq I$. Now, let $a \in I$. Since $R$ is a Euclidean
    domain, there exist $q,r \in R$ for which
    \begin{equation*}
        a=qd+r \text{ where } r=0 \text{ or } N(r)<N(d)
    \end{equation*}
    Then notice that
    \begin{equation*}
        r=a-qd
    \end{equation*}
    putting $r \in I$ and  $N(r) \in N(I)$. Since $N(d)$ is the minimum element,
    we must have $r=0$ so that  $a=qd$, which puts  $a \in (d)$. Therefore
    $I=(d)$, making $I$ principle.
\end{proof}

\begin{example}\label{2.2}
    \begin{enumerate}
        \item[(1)] The polynomial ring $\Z[x]$ is not a Euclidean domain. The
            ideal $(2,x)$ is not principle.

        \item[(2)] Consider $\Z[\sqrt{-5}]$, i.e. $D=-5$. Suppose the ideal
            $(3,2+\sqrt{-5})$ is a principle ideal, that is
            $(3,2+\sqrt{5})=(a+b\sqrt{-5})$ for some $a,b \in \Z$. Then we get
            that $3=x(a+b\sqrt{-5})$ and $2+\sqrt{-5}=y(a+b\sqrt{-5})$. Then
            notice that $N(x)=a^2+5b^2=9$, and since  $a^2+5b^2 \in \Z^+$, we
            must have that $a^2+5b^2=1,3,9$.
            \begin{enumerate}
            \item[(i)] If $a^2+5b^2=9$, then $N(x)=1$ making $x=\pm 1$ and
                $a+b\sqrt{-5}=\pm3$, which cannot happen since $2+\sqrt{-5}$ is
                not divisible by $3$.

            \item[(ii)] the equation $a^2+5b^2=3$ cannot happen since it has no
                integer solutions. This makes

            \item[(iii)] $a^2+b\sqrt{5}=1$, which makes
                $(a+\sqrt{-5})=\Z[\sqrt{-5}]$, moreover, we get the equation
                $3x+y(2+\sqrt{-5})=1$ for any $x,y \in \Z[\sqrt{-5}]$.
                Multplying both sides by $2-\sqrt{-5}$, we get that
                $3|(2-\sqrt{-5})$ which is impossible.
            \end{enumerate}
            In all three cases, we were led to an impossibility, hence
            $\Z[\sqrt{-5}]$ cannot be a Euclidean domain.
    \end{enumerate}
\end{example}

\begin{definition}
    Let $R$ be a commutative ring, and  $a,b \in R$ with  $b \neq 0$. We say
    that $b$ \textbf{divides} $a$ if there is an  $x \in R$ for which  $a=bx$.
    We write  $b|a$. We also say that $a$ is a \textbf{multiple} of $b$.
\end{definition}

\begin{definition}
    Let $R$ be a commutative ring. We call a nonzero element  $d \in R$ a
    \textbf{greatest common divisor} of elements $a,b \in R$ if
    \begin{enumerate}
        \item[(1)] $d|a$ and  $d|b$.

        \item[(2)] If $c \in R$ is nonzero such that  $c|a$ and  $c|b$, then
            $c|d$.
    \end{enumerate}
    We write $d=(a,b)$.
\end{definition}

\begin{lemma}\label{2.1.2}
    Let $R$ be a commutative ring. For any $a,b \in R$ a nonzero element  $d \in
    R$ is the greatest common divisor if
    \begin{enumerate}
        \item[(1)] $(a,b) \subseteq (d)$.

        \item[(2)] If $c \in R$ is nonzero with $(a,b) \subseteq (c)$, then $(d)
            \subseteq (c)$.
    \end{enumerate}
    In particular, $d=(a,b)$.
\end{lemma}
\begin{proof}
    The first two statements follow from definition, and the last follows lemma
    \ref{2.1.1}.
\end{proof}

\begin{lemma}\label{2.1.3}
    If $R$ is a commutative ring, and $a,b \in R^\ast$, such that  $(a,b)=(d)$
    for some $d \in R^\ast$, then $d$ is the greatest common divisor of $a$ and
     $b$.
\end{lemma}

\begin{lemma}\label{2.1.4}
    Let $R$ be an inetegral domain. If $c,d \in R$ generate the same principle
    ideal, i.e. $(d)=(c)$, then $d=uc$ for some unit  $u \in R$.
\end{lemma}
\begin{proof}
    If $c=0$ or  $d=0$, we are done. Suppose then that  $c,d \neq 0$. Since
    $(d)=(c)$, we have that $d=xc$ and $c=yd$ for some  $x,y \in R$. Then
    $d=(xy)d$, which makes $d(1-xy)=0$. Since $d \neq 0$, we get $xy=1$, making
     $x$ and $y$ units of $R$.
\end{proof}
\begin{corollary}
    If $R$ is commutative, then greatest common divisors are unique.
\end{corollary}

\begin{definition}
    We call an integral domain in which every principle ideal is generated by
    two elements a \textbf{Bezout domain}.
\end{definition}

\begin{lemma}\label{2.1.5}
    Every Euclidean domain is a Bezout domain.
\end{lemma}

\begin{theorem}[The Extended Euclidean Algorithm]\label{2.1.6}
    Let $R$ be a Euclidean and $a,b \neq 0$ elements of  $R$. Let  $d=r_n$ be
    the least nonzero remainder obtained by dividing $a$ from $b$ recursively
    $n+1$ times. Then
    \begin{enumerate}
        \item[(1)] $d=(a,b)$ is the greatest common divisor of $a$ and  $b$.

        \item[(3)] There exist $x,y \in R$ for which  $ax+by=d$.
    \end{enumerate}
\end{theorem}
\begin{proof}
    By lemma \ref{2.1.1}, we get that the ideal $(a,b)$ is principle, so there
    exists a greatest common divisor of $a$ and  $b$. Now, let  $d=r_n$ be
    obtained by dividing $a$ and $b$ recursively $(n+1)$ times. Then the
    $(n+1)^{st}$ equation gives $r_{n-1}=q_{n+1}r_n$, so that $r_n|r_{n-1}$.
    Now, by induction on $n$ if  $r_n|r+{k+1}$ and $r_n|r_k$ then the
    $(k+1)^{st}$ equation gives $r_{k-1}=q_{k+1}r_k+r_{k+1}$, which implies that
    $r_n|r_{k-1}$. Therefore we get in the $1^{st}$ equation that $r_n|b$, and in
    the $0^{th}$ that $r_n|a$. That is,  $d|a$ and  $d|b$.

    Now, notice that $r_0 \in (a,b)$ and that $r_1=b-qr_0 \in (b,r_0) \subseteq
    (a,b)$. By induction on $r_n$, if  $r_{k-1},r_n \in (a,b)$ then
    \begin{equation*}
        r_{k+1}=r_{k-1}-q_{k+1}r_k \in (r_{k-1},r_n) \subseteq (a,b)
    \end{equation*}
    which puts $r_n \in (a,b)$ making $d=(a,b)$ the greatest common divisor.
\end{proof}

\begin{definition}
    Let $R$ be an integral domain, and let
    $\tilde{R}=R^\ast \cup \{0\}$ the set of units together with $0$. We call
    an element  $u \in \com{R}{\tilde{R}}$ a \textbf{universal side divisor} if
    for all $x \in R$, there is a  $z \in \tilde{R}$ such that $u|x-z$.
\end{definition}

\begin{lemma}\label{2.1.7}
    Let $R$ be an integral domain which is not a field. If $R$ is a Euclidean
    domain, then there exist universal side divisors.
\end{lemma}
\begin{proof}
    Notice that since $R$ is not a field, that $\tilde{R} \neq R$ and
    $\com{R}{\tilde{R}}$ is nonempty. Let $N$ be the norm of  $R$, and let  $u
    \in \com{R}{\tilde{R}}$ be of minimal norm. Then for all $x \in R$, take
    $x=qu+r$ with  $r=0$ or  $N(r)<N(u)$> By minimality of $N(u)$, we get $r \in
    \tilde{R}$.
\end{proof}

\begin{example}\label{2.3}
    Notice that $\pm 1$ are the only units in the ring
    $\Z[1+\frac{\sqrt{-19}}{2}]$, so that $\tilde{R}=\{0,1,-1\}$. Suppose that
    $u \in R$ is a universal side divisor, and let $N=\|\cdot\|^2$ be the field
    norm; so that $N(a+(1+\frac{\sqrt{-19}}{2})b)=a^2+ab+5b^2$. If $a,b \in \Z$
    and $b \neq 0$, then we have $a^2+ab+5b^2=(a+\frac{b}{2})^2+\frac{19}{4b^2}
    \geq 5$ so that the smallest nonzero norms are $1$ for $x=1$ and $4$ for $x=2$.
    Now, if $u$ is a universal side divisor, then  $u|2-0$ or  $u|(2 \pm 1)$
    that is $u|2$,  $u|3$ or  $u|1$ making  $u$ a nonunit divsor. If  $2=xy$
    then $4=N(x)N(y)$ and so that $N(x)=1$ or  $N(y)=1$. Hence the only
    divisors of $2$ in  $\Z[1+\frac{\sqrt{-19}}{2}]$ are $\pm 1$ or  $\pm 2$.
    Similarly the only divisors of  $3$ arew  $\pm 1$ or  $\pm 3$ hence  $u=\pm
    2$ or  $u=\pm 3$. Letting  $x=\frac{1+\sqrt{-19}}{2}$, then $x$, nor  $x
    \pm 1$ are divisible by any possible $u$. Therefore
    $\Z[1+\frac{\sqrt{-19}}{2}]$ has no universal side divisors, and cannot be a
    Euclidean domain.
    \end{example}
