\section{Direct Products and Free Modules.}

\begin{definition}
    Let $M$ be an  $R$-module, and $N_1, \dots, N_n$ submodules of $M$. We
    define the  \textbf{sum} of $N_1, \dots, N_n$ to be the set
    \begin{equation*}
        N_1+\dots+N_n=\{a_1+\dots+a_n : a_i \in N_i\}
    \end{equation*}
    For any subset $A \subseteq M$, we call the set
    \begin{equation*}
        RA=\{r_1a_1+r_na_n : r_j \in R \text{ and } a_i \in A\}
    \end{equation*}
    the submodule of $M$  \textbf{generated} by $A$. If $A$ is empty, we write
    $RA=(0)$. If $A$ is finite, we write  $RA=Ra_1+\dots+Ra_n$. We call a
    submodule $N$ of  $M$  \textbf{finitely generated} if there is a finite set
    $A \subseteq M$ for which  $N=RA$. We call  $N$  \textbf{cyclic} if there is
    an $a \in M$ for which  $N=Ra$.
\end{definition}

\begin{example}\label{example_4.6}
    \begin{enumerate}
        \item[(1)] Let $M$ be any  $\Z$-modules, if  $a \in M$, then
            $\Z{a}=\langle a \rangle$ is the cyclic group generated by $a$.
            Moreover, $M$ is generated as a $\Z$-moduled, by a set $A$ if, and
            only if  $A$ is an Abelian group.

        \item[(2)] Let $R$ be a ring with identity, and consider  $R$ as a left
             $R$-module over itself. Then $R$ is finitely generated, and is
             in fact, cyclic. Indeed,  $R=R1$. Now, if $I$ is a cyclic subgroup
             of  $R$, then  $R$ is a principle ideal of  $R$. If  $I$ is
             finitely generated as a submodule of $R$, it is finitely generated
             as an ideal.

        \item[(3)] Let $R$ be a ring with identity, and consider  $R^n$ the free
            module of rank  $n$ ver  $R$, Let  $e_i=(0, \dots, 0,1,0, \dots, 0)$
            the element where $1$ appears in the  $i$-th position, for all  $1
            \leq i \leq n$. Then we have
            \begin{equation*}
                (s_1, \dots, s_n)=s_1e_1+\dots+s_ne_n
            \end{equation*}
            so that $M$ is generated by the set  $\{e_1, \dots, e_n\}$. If $R$
            is commutative, then this is a generating set.

        \item[(4)] Let $F$ be a field, and let  $V$ be a vector space over $F$.
            Let $T:V \xrightarrow{} V$ a linear transformation making $V$ into
            an  $F[x]$-moddule. Then $V$ is a cyclic  $F[x]$-module, with
            generator $v$, if, and only if  $V=\{p(x)v : p \in F[x]\}$. That is,
            elements of $V$ can be written as linear combinations of elements of
            the set
             $\{T^nv : n \geq 0\}$; that is, $\Span{V}=\{v,Tv.T^2v, \dots\}$
    \end{enumerate}
\end{example}

\begin{definition}
    Let $M_1, \dots, M_n$ a a collection of $R$-modules. We define the
    \textbf{external direct product} of $M_1, \dots, M_n$ to be
    \begin{equation*}
        M_1 \times \dots \times M_n=\{(m_1, \dots, m_n) : m_i \in M_i\}
    \end{equation*}
    We define the addition and scalar multiplication on $M_1 \times M_n$ to be
    the addition and scalar multiplication of $M_i$ component-wise for all  $1
    \leq i \leq n$.
\end{definition}

\begin{lemma}\label{4.3.1}
    Let $M_1, \dots, M_n$ be $R$-modules. Then the external direct product  $M_1
    \times \dots \times M_n$ is an $R$-module.
\end{lemma}

\begin{lemma}\label{4.3.2}
    Let $N_1, \dots, N_n$ me submodules of an $R$-module  $M$. The following
    statements are equivalent.
    \begin{enumerate}
        \item[(1)] Then map $\pi:N_1 \times N_n \xrightarrow{} N_1+\dots+N_n$
            defined by $(a_1, \dots, a_n) \xrightarrow{} a_1+\dots+a_n$ is an
            isomorphism.

        \item[(2)] $N_i \cap (N_1+\dots+N_{i-1}+N_{i+1}+\dots+N+_n)=(0)$
            for all $1 \leq i \leq n$.

        \item[(3)] Every $\x \in N_1+\dots+N_n$, can be written uniquely as
            \begin{equation*}
                x=a_1+\dots+a_n \text{ where } a_i \in N_i
            \end{equation*}
    \end{enumerate}
\end{lemma}
\begin{proof}
    First suppose that $N_1 \times \dots \times N_n \simeq N_1+\dots+N_n$ via
    the isomorphism $\pi:(a_1, \dots, a_n) \xrightarrow{} a_1+\dots+\a_n$.
    Suppose now that there exists an index $1 \leq i \leq n$ for which
    \begin{equation*}
        N_i \cap (N_1+\dots+N_{i-1}+N_{i+1}+\dots+N+_n) \neq (0)
    \end{equation*}
    for all $i$. Let  $a_j \in N$ with  $a_j \neq 0$. Then
    \begin{equation*}
        a_i=a_1+\dots+a_{i-1}+a_{i+1}+\dots+a_n
    \end{equation*}
    Then we get $(a_1, \dots, a_{i-1},a_{i+1}, \dots, a_n)$ is a nonzero element
    of $\ker{\pi}$, which contradicts that $\pi$ is an isomorphism.

    Now, suppose that
    \begin{equation*}
        N_i \cap (N_1+\dots+N_{i-1}+N_{i+1}+\dots+N+_n)=(0) \text{ for all } i
    \end{equation*}
    Let $a_i,b_i \in N_i$ and let  $a_1+\dots+a_n=b_1+\dots+b_n$. Then
    \begin{equation*}
        a_i-b_i=(a_1-b_1)+\dots+(a_{i-1}-b_{i-1})+(a_{i+1}-b_{i+1})+\dots+(a_n-b_n)
    \end{equation*}
    Then $a_i-b_i \in N_1$ and
    $(a_1-b_1)+\dots+(a_{i-1}-b_{i-1})+(a_{i+1}-b_{i+1})+\dots+(a_n-b_n) \in N$.
    So we get  $a_i-b_i \in N$ so that  $a_i=b_i$ for all  $1 \leq i \leq n$.

    Now, suppose that $x \in N$ can be written uniquely as $x=a_1+\dots+a_n$
    where $a_i \in N_i$. Then  $x \in N_1+\dots+N_n$. Observe that the map
    $\pi:(a_1, \dots, a_n) \xrightarrow{} a_1+\dots+a_n$ is onto. Then since
    $a_1+\dots+a_n$ uniquely determines elements in $N_1+\dots+N_n$, $\pi$ is
    1--1.
\end{proof}

\begin{definition}
    We call an $R$-module  $M$ an \textbf{internal direct product} of submodules
    $N_1, \dots, N_n$ if $M=N_1+\dots+N_n$ and $M$ satisfies one of the
    conditions of lemma \ref{4.3.2}. We write $M=N_1 \oplus \dots \oplus N_n$.
\end{definition}

\begin{definition}
    We call an $R$-module $F$ \textbf{free} on a subset $A \subseteq F$ if for
    every  $x \in F$, there is are unique nonzero $r_1, \dots ,r_n$ and unique
    $a_1, \dots, a_n \in A$ such that $x=r_1a_1+\dots+r_na_n$ for some $n \in
    \Z^+$. We call $A$ a  \textbf{basis} of $F$. If  $R$ is commutative, we call
     $|A|$ the  \textbf{rank} of $F$ and write  $\rank{F}=|A|$.
\end{definition}

\begin{example}\label{example_4.7}
    The module $\faktor{\Z}{2\Z} \times \faktor{\Z}{2\Z}$ is not a free
    $\Z$-module on the set $\{(1,0),(0,1)\}$. It is not free on any set.
\end{example}

\begin{theorem}\label{4.3.3}
    For any set $A$, there is a free  $R$-module  $F(A)$ on $A$, such that if
    $M$ is any  $R$-module, and  $\phi:A \xrightarrow{} M$ is any map, then
    there is a unique module homomorphism $\Phi:F(A) \xrightarrow{} M$ such that
    $\Phi(a)=\phi(a)$ for all $a \in A$. That is, the following diagram
    commutes.
    \[\begin{tikzcd}
        A & {F(A)} \\
        & M
        \arrow["i", from=1-1, to=1-2]
        \arrow["\Phi", from=1-2, to=2-2]
        \arrow["\phi"', from=1-1, to=2-2]
    \end{tikzcd}\]
    Moreover, if $A=\{a_1, \dots, a_n\}$, then
    \begin{equation*}
        F(A)=Ra_1 \oplus \dots \oplus Ra_n \simeq R^n
    \end{equation*}
\end{theorem}
\begin{proof}
    Let $F(A)=(0)$. If $A$ is empty, and the collection of all set functions
    $f:A \xrightarrow{} R$ such that $f(a)=0$ for all but finitely many $a \in
    A$, if  $A$ is nonempty. Make $F(A)$ into an $R$-module via pointwise
    adition and the action  $R \times F(A) \xrightarrow{} F(A)$ defined by
    $(rf)(a)=r(f(a))$ for all $a \in A$,  $r \in R$, and  $f,g \in F(A)$. Take
    the mape $a \xrightarrow{} f_a$ where
    \begin{equation*}
        f_a(x)=\begin{cases}
            1, x=a  \\
            0, x \neq a \\
        \end{cases}
    \end{equation*}
    This map makes $A \subseteq F(A)$. Then $F(A)$ can be expressed as all
    finite $R$-linear combination of elements of  $A$ by identifying each  $f$
    with  $r_1a_1+\dots+r_na_n$, where
    \begin{equation*}
        f(x)=\begin{cases}
            r_i, x=a_i  \\
            0, x \neq a_i \\
        \end{cases}
    \end{equation*}
    Moreover, each element of $F(A)$ has a unique expression.

    Now, let $\phi:A \xrightarrow{} R$ be a map and define $\Phi:F(A)
    \xrightarrow{} R$ by taking $r_1a_1+\dots+r_na_n \xrightarrow{}
    r_1\phi(a_1)+\dots+r_n\phi(a_n)$. By the uniqueness of expression of
    elements in $F(A)$, $\Phi$ is a well defined $R$-module homomorphism.
    Moreover, by definition, $\Phi|_A=\phi$. Finally, since  $F(A)$ is generated
    by $A$, and the values of any $R$-module homomorphism on $F(A)$ are uniquely
    determined, this makes $\Phi$ the unique extension of  $\phi$ to  $F(A)$.

    Now, let $A=\{a_1, \dots, a_n\}$. THen $F(A)=Ra_1 \oplus \dots \oplus Ra_n$
    by the above lemma. Moreover, since $R \simeq Ra_i$ for all $1 \leq i \leq
    n$, we get $F(A) \simeq R^n$.
\end{proof}
\begin{corollary}
    The following statements are true.
    \begin{enumerate}
        \item[(1)] If $F_1$ and $F_2$ are free modules on $A$, there is a unique
            isomorphism between  $F_1$ and $F_2$ which is the identity
            isomorphism on $A$.

        \item[(2)] IF $F$ is any free $R$-module with basis  $A$, then  $F
            \simeq F(A)$.
    \end{enumerate}
\end{corollary}

\begin{definition}
    We call a free $\Z$-module on a set  $A$ the  \textbf{free Abelian group} on
    $A$ of  \textbf{rank} $|A|=n$ (if $A$ is finite).
\end{definition}
