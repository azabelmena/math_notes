%----------------------------------------------------------------------------------------
%	SECTION 1.2
%----------------------------------------------------------------------------------------

\section{Fields}

\begin{definition}
    A \textbf{field} is a set $F$, together with binary operations  $+$ and  $\cdot$ (called
    \textbf{addition} and \textbf{multiplication}, respectively) such that:
        \begin{enumerate}
            \item[(1)] $F$ forms an abelian group under  $+$.

            \item[(2)] $F \backslash \{0\}$ forms an abelian group under $\cdot$ (where $0$ is the
                additive identity of  $F$).

            \item[(3)] $\cdot$ distributes over  $+$.
        \end{enumerate}
\end{definition}

We now state the following propositions without proof.

\begin{proposition}\label{1.2.1}
    For all $x,y,x \in F$:
        \begin{enumerate}
            \item[(1)] $x+y=x+y$ implies $y=z$

            \item[(2)] $x+y=x$ implies  $y=0$

            \item[(3)] $x+y=0$ implies  $y=-x$

            \item[(4)] $-(-x)=x$.
        \end{enumerate}
\end{proposition}

\begin{proposition}\label{1.2.2}
    For all $x,y,x \in F \backslash \{0\}$:
        \begin{enumerate}
            \item[(1)] $xy=xy$ implies $y=z$

            \item[(2)] $xy=x$ implies  $y=1$

            \item[(3)] $xy=1$ implies  $y=x^{-1}$

            \item[(4)] $(x^{-1})^{-1}=x$.
        \end{enumerate}
\end{proposition}

\begin{proposition}\label{1.2.3}
    For all $x,y,x \in F$:
        \begin{enumerate}
            \item[(1)] $0x=0$

            \item[(2)] $x \neq 0$ and $y \neq 0$ implies $xy \neq 0$

            \item[(3)] $(-x)y=-(xy)=x(-y)$

            \item[(4)] $(-x)(-y)=xy$.
        \end{enumerate}
\end{proposition}

\begin{definition}
    An \textbf{ordered field} is a field $F$ that is also an ordered set, such that:
        \begin{enumerate}
            \item[(1)] $x+y<x+z$ whenever $y<z$, for  $x,yz,z \in F$

            \item[(2)] $xy>0$ whenever  $x>0$ and  $y>0$, for  $x,y \in F$.
        \end{enumerate}
\end{definition}


\begin{proposition}\label{1.2.4}
    Let $F$ be an ordered field, then for any  $x,y,z \in F$, the following hold:
         \begin{enumerate}
             \item[(1)] $x>0$ implies $-x<0$.

             \item[(2)] If $x>0$  and $y<z$, then $xy<xz$.

             \item[(3)] If $x<0$  and $y<z$, then $xz<xy$.

             \item[(4)] If $x \neq 0$, then $x^2>0$, in particular, $1>0$.

             \item[(5)] $0<x<y$ implies that $0<y^{-1}<x^{-1}$.
        \end{enumerate}
\end{proposition}
\begin{proof}
    \begin{enumerate}
        \item[(1)] If $x>0$, then  $0=x+(-x)>0+(-x)$, so  $-x<0$.

        \item[(2)] We have $0<z-y$, so  $0<x(z-y)=xz-xy$, so $xy<xz$.

        \item[(3)] Do the same as  $(2)$,, multiplying  $z-y$ by  $-x$.

        \item[(4)] If  $x>0$, we are done. Now suppose that  $x<0$, then  $-x>0$, so
            $(-x)(-x)=xx=x^2>0$; in particular, we also have that  $1 \neq 0$, and
             $1=1^2$, so  $1>0$.

         \item We have  $0<xy^{-1}<yy^{-1}=1$, then  $0<x^{-1}xy^{-1}=y^{-1}<x^{-1}1=x^{-1}$
    \end{enumerate}
\end{proof}
