%----------------------------------------------------------------------------------------
%	SECTION 1.1
%----------------------------------------------------------------------------------------

\section{Discontinuities.}

\begin{definition}
    Let $X$ and  $Y$ be metric spaces, and let  $f:E \rightarrow Y$ for $E \subseteq X$.
    If there is a point $x\ in E$ for which  $f$ is not continuous, we say that  $f$ is
    textbf{discontinuous} at $x$, and we say that  $f$ has a \textbf{discontinuity} at
    $x$.
\end{definition}

\begin{definition}
    Let $f$ be defined on  $(a,b)$, and let  $x$ be such that  $a \leq x<b$. We write
    $f(x+)=q$ if $f(t_n) \rightarrow q$ for all sequences  $\{t_n\}$ in  $(x,b)$ such that
    $t_n \rightarrow x$. Similarly, if  $x$ is such that  $a<x \leq b$, we write  $f(x-)=q$ if
    $f(t_n) \rightarrow q$ for all  sequences $\{t_n\}$ in  $(a,x)$ such that $t_n \rightarrow x$.
    We call  $f(x+)$ and  $f(x-)$ the \textbf{right handed limit} and \textbf{left handed limit} of  $f$
    at  $x$ respectively, and write  $\lim_{t \rightarrow x^+}{f}=f(x+)$ and  $lim_{t \rightarrow x^-}{f}=f(x-)$.
\end{definition}

\begin{theorem}\label{5.5.1}
    If $x \in (a,b)$, then $\lim{f}$ exists as $t \rightarrow x$ if and only if,
    $f(x+)=f(x-)=lim{f}$.
\end{theorem}
\begin{proof}
    Suppose that $\lim{f}$ exists, by the uniqueness of the limit, and the sequential criterion,
    we get that $f(x+)=f(x-)=\lim{f}$. Conversely, suppose that  $f(x+)=f(x-)=q$. Then
    $f(t_n) \rightarrow q$ for all sequences $\{t_n\}$ in $(x,b)$ and $(a,x)$, then  $f(t_n) \rightarrow q$
    for all sequences  $\{t_n\}$ in  $(a,b)$, thus by the sequential criterion again, $\lim{f}$
    exists, and $\lim{f}=q$.
\end{proof}

\begin{definition}
    Let $f$ be defined on  $(a,b)$. If  $f$ is discontinuous at a point  $x$, and
    $f(x+)$ and  $f(x-)$ exists, we say that  $f$ has a \textbf{removable discontinuity} at  $x$,
    otherwise, we say the  $f$ has an \textbf{infinite discontinuity}.
\end{definition}

\begin{example}
    \begin{enumerate}
        \item[(1)] The function $f(x)=1$ for  $x \in \Q$ and  $f(x)=0$ for  $x \in \com{\R}{\Q}$ has
            an infinite discontinuity at every point $x$.

        \item[(2)] The function $f(x)=x$ for  $x \in \Q$ and  $f(x)=0$ for
            $x \in \com{\R}{\Q}$ is continuous at $x=0$, and has an infinite
            discontinuity at every other point $x$.

        \item[(3)] The function $f(x)=\sin{\frac{1}{x}}$ for  $x \neq 0$ and  $f(x)=0$ for $x=0$,
            has an infinite discontinuity at $x=0$.

        \item[(4)] The function $f(x)=x+2$ for  $-3<x<-2$ and  $0 \leq x<1$ and  $f(x)=-x-2$ for
            $-2 \leq x<0$ has a removable discontinuity at  $x=0$, and is continuous everywhere
            else.
    \end{enumerate}
\end{example}

\begin{remark}
    The discontinuities in examples $(1)$ and  $(2)$ are the result of  $\Q$ and
    $\com{\R}{\Q}$ being dense in  $\R$.
\end{remark}
