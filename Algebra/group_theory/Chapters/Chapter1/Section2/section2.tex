%----------------------------------------------------------------------------------------
%	SECTION 1.1
%----------------------------------------------------------------------------------------

\section{Definitions and Examples.}
\label{section1}

\begin{definition}
    Let $G$ be set, we define a  \textbf{binary operation}, $\ast$, on  $G$ to
    be a map  $\ast:G \times G \rightarrow G$ that takes $(a,b) \rightarrow a
    \ast b$. We say that a binary operation $\ast$ is \textbf{associative} if
    for any $a,b,c \in G$,  $(a \ast b) \ast c=a \ast (b \ast c)$. We say that a
    binary operation, $\ast$ is \textbf{commutative} if for any $a,b \in G$,
    $a \ast b=b \ast a$
\end{definition}
 We also write $ab$ instead of  $a \ast b$ for convinience, and when
context is clear.

\begin{example}
    \begin{enumerate}
        \item[(1)] The usual addition, $+$ is an associative and commutative binary
            operation on the sets $\Z$,  $\Q$, and  $\R$ of integers, rationals,
            and real numbers. The addition of complex numbers, $+$ is an
            associative and commutative binary operation on the complex numbers
            $\C$

        \item[(2)] The usual multiplication $\cdot$ is an associative and
            commutative binary operation on  $\Z^*$,  $\Q^*$ and  $\R^*$.
            Complex multiplication on  $\C^*$ is also an associative,
            commutative binary operation. Note we define
            $F^*=\com{{F}}{\{0\}}$, where $F=\Z,\Q,\R,\C$.

        \item[(3)] The usual subtration, is a noncomutative binary operation on
            $\Z$,  $\Q$, and  $\R$, complex subtraction is noncommutative on
            $\C$. The map $a \rightarrow -a$ is not binary.

        \item[(4)] The usual subtraction is not a binary operation on $\Z^+$,
            $Q^+$, and  $\R^+$, notice that if  $a<b$,  $a-b \notin F$ where
            $F=\Z,\Q,\R$.

        \item[(5)] The cross prodcuct, $\times$ on two vectors in real $3$-space
            is a nonassociative binary operation on $\R^3$.

        \item[(6)] The operation $+_n$ of addition $\mod{n}$ is a binary
            operation on the set of integers  $\mod{n}$, $\faktor{\Z}{n\Z}$.
            Notice that for $k, l, m \in \Z$, that $a \mod{n}$ is of the form
            $a+kn$, thus $b \mod{n}$ and $c \mod{n}$, have the forms $b+ln$ and
            $c+mn$. Thus
            $((a+kn)+(b+ln))+(c+mn)=(a+b)+(l+k)n+(c+mn)=a+b+c+(l+k+m)n=(a+kn)+
            (b+c)+(l+m)n=(a+kn)+((b+ln)+(c+mn))$. This implies that $(a+b)+c
            \mod{n}=a(b+c) \mod{n}$; additionally,
            $(a+kn)+(b+ln)=(a+b)+(k+l)n=(b+a)+(l+k)n=(b+ln)+(a+kn)$, so $a+b
            \mod{n} = b+a \mod{n}$. That is $+_n$ is associative and
            commutative. We abbreviate addition $\mod{n}$ and write $+$ instead
            of  $+_n$.

        \item[(7)] Multiplictation $\mod{n}$ is a binary operation on
            $\faktor{\Z}{n\Z}$ which is associative and commutative.

        \item[(8)] The operation of function composition $\circ$ is a binary
            operation on any set of mappings. We have that for mappings $f$,
            $g$, and  $h$ that  $(f \circ g) \circ h=f \circ (g \circ h)$,
            making $\circ$ associative; but  $f \circ g \neq g \circ f$, making
             $\circ$ noncommutative.
    \end{enumerate}
\end{example}

\begin{definition}
    Let $G$ be a set, and  $H \subseteq G$, and let  $\ast$ be a binary
    operation on  $G$. We say that  $H$ is \textbf{closed} under $\ast$ if
    $\ast|_H$ is a binary operation on  $H$.
\end{definition}

\begin{definition}
    Let $G$ be a nonempty set, and let  $\ast$ be a binary operation on $G$. We
    call the pair  $(G,\ast)$ a \textbf{group} if:
    \begin{enumerate}
        \item[(0)] For every $a,b \in G$,  $ab \in G$. That is  $G$ is closed
            under $\ast$.
        \item[(1)] $(ab)c=a(bc)$ for all $a,b,c \in G$, i.e. $\ast$ is
            associative.

        \item[(2)] There exists an element $e \in G$ called the
            \textbf{identity} element such that  $ae=ea=a$ for all  $a \in G$

        \item[(3)] For each $a \in G$, there is an element  $b \in G$, called
            the \textbf{inverse} of $a$ such that  $ab=ba=e$, where  $e$ is the
            identity element.
    \end{enumerate}
\end{definition}
\begin{remark}
    We make note that property $(0)$ of this definition is impled by stating
    $\ast$ as a binary operation on  $G$, we however list it, becuase when
    verifying a given set is a group, we usually want to check for closure.
\end{remark}
\begin{remark}
    Instead of stating $(G,\ast)$ as a group, we will often just say that $G$ is
    a group under  $\ast$, or simply,  $G$ is a group.
\end{remark}

\begin{example}
    \begin{enumerate}
        \item[(1)] The set $G=\{e\}$ of one element forms a group under any
            operation, and is called the \textbf{trivial} group. We write
            $G=\vbrack{e}$.

        \item[(2)] The sets $\Z$,  $\R$, and  $\Q$ are all groups under the
            usual addition. Here $0$ is the identity, and  $-a$ is the inverse
            of  $a$.  $\C$ is a group under complex addition with $0=0+i0$ the
            identity and $-a-ib$ the inverse of $a+ib$.

        \item[(3)] $\Q$ and  $\R$ are groups under the usual multiplication,
            with identity  $1$ and inverse  $\inv{a}=\frac{1}{a}$. $\Z$ is not a
            group under this operation, as  $\frac{1}{a} \notin \Z$ whenever $a
            \in \Z$.  $\C$ is a group under complex multiplication with identity
             $1=1+i0$ and inverse $\frac{a}{a^2+b^2}-i\frac{b}{a^2+b^2}$ for
             $a+ib$.

         \item[(4)] Consider the set of integers $\mod{n}$, $\faktor{\Z}{n\Z}$
             under addition  $\mod{n}$, $+$. Since  $+$ is a binary operation on
             $\faktor{Z}{n\Z}$, closure is guaranteed. We also see that
             associativity hold. Now, notcie that $n \equiv 0 \mod{n}$, by
             definition, so $a+n=n+a \equiv 0+a \mod{n} \equiv a+0 \mod{n}
             \equiv a$. Moreover, $n-a \equiv 0-a \equiv -a \mod{n}$, and
             $(n-a)+a=n(-a+a)=n \equiv 0 \mod{n}$ and $a+(n-a)=n+(a-a)=n \equiv
             0 \mod{n}$. So $(\faktor{\Z}{n\Z}, +_n)$ is a group, with identity
             element $0 \mod{n}$ and inverse element $-a \mod{n}$ for each $a
             \in \faktor{\Z}{n\Z}$.
    \end{enumerate}
\end{example}

\begin{example}
    Suppose we removed the restriction to be nonempty in the definition of a
    group. We see that if $G=\emptyset$, then  $G$ cannot be a group, since it
    is not closed, trivially; furthurmore, there is no identity, nor inverse to
    each element. Therefore the minimum number of elements a group can have is
    $1$. This makes the trivial group minimal.
\end{example}

\begin{definition}
    We call a group $G$ under a binary operation  $\ast$  \textbf{Abelian}, or
    \textbf{commutative} if for every $a,b \in G$,  $ab=ba$.
\end{definition}

\begin{example}
    The above examples of groups are also examples of abelian groups.
\end{example}

\begin{example}
    Consider $\com{(\faktor{\Z}{n\Z})}{\{0\}}$ under multiplication $\mod{n}$,
    $\cdot_n$  (abbreviated as $\cdot$). This is not a group as not every
    element has an inverse. Specifically, take $n=6$, then in
    $\faktor{\Z}{6\Z}$, $2 \cdot 3=6 \equiv 0 \mod{6} \notin
    \com{(\faktor{\Z}{n\Z})}{\{0\}}$. However, one can still impose a group
    structure with modular multiplication.

    Define the set $U(\faktor{\Z}{n\Z})=\{a \in \faktor{\Z}{n\Z} : (a,n)=1\}$,
    that is it is the set of all integers $\mod{n}$ coprime with $n$. We have
    that $U(\faktor{\Z}{n\Z})$ is closed under $\cdot$. Notice that if
    $(a,n)=1$, and $(b,n)=1$, then $(ab,n)=1$. Also notice that
    $U(\faktor{\Z}{n\Z}) \subseteq \faktor{\Z}{n\Z}$, and so inherits
    associativity. Moreover, notice that $(1,n)=1$, and $a1=1a=a \in
    U(\faktor{\Z}{n\Z})$, so $1 \mod{n}$ is the identity of the set. Now for any
    $a \in U(\faktor{\Z}{n\Z})$, since $(a,n)=1$, there exist $b,m \in \Z$ such
    that $ab+mn=1$, that is $ab \equiv 1 \mod{n}$, and moreover notice that if
    $(ab,n)=1$ and $(a,n)=1$, then $(b,n)=1$, thus $b \in U(\faktor{\Z}{n\Z})$,
    and is an inverse of $a$. Thus we have shown that  $U(\faktor{\Z}{n\Z})$ is
    a group under $\cdot$. We call the group the \textbf{group of units,
    $\mod{n}$}, or simply the \textbf{unit group $\mod{n}$}. Moreover, we see
    that this group is commutative.
\end{example}

\begin{example}
    \begin{enumerate}
        \item[(1)] The vector space axioms for some vector space $V$ specify
            that under vector addition  $+$,  $(V,+)$ forms a group.

        \item[(2)] Let $(A, \ast)$, $(B,\cdot)$ be groups under binary
            operations $\ast$ and  $\cdot$. Consider the product $A \times B$
            and take the map $\circ:(a_1,b_1)(a_2,b_2) \rightarrow (a_1 \ast a_2, b_1
            \cdot b_2)$. Then $\circ$ is a binary operation on  $A \times B$.
            Then  $(A \times B, \circ)$ forms a group. We have that since $A$
            and  $B$ is closed, then so is  $A \times B$. Furthermore, by
            associativity of  $\ast$ and  $\cdot$,  $((a_1,b_1) \circ (a_2,b_2))
            \circ (a_3,b_3)=(a_1,b_1) \circ ((a_2,b_2) \circ (a_3,b_3))$; making
            $\circ$ associative. Now if  $e_1$ and $e_2$ are the identities of
            $A$ and  $B$ respevively, then  $(e_1,e_2)$ is the identiy for $A
            \times B$; finding the inverse of an element  $(a,b)$ follow
            similarly.
    \end{enumerate}
\end{example}

\begin{theorem}\label{1.2.1}
    Let $G$ be a group under a binary operation  $\ast$ then the identy of  $G$
    is unique, and the inverse of  $a \in G$ is unique.
\end{theorem}
\begin{proof}
    Suppose there exists $e,f \in G$ such that for any  $a \in G$  $ae=ea=a$ and
     $af=fa=a$. Then we have that $fe=e$ and  $ef=fe=f$; thus  $e=f$.

     Now let  $a \in G$, and suppose  $a$ has inverses  $b,c \in G$, then
     $ab=ba=e$ and  $ac=ca=e$, where  $e$ is the identity of  $G$. Then we have
     $b=be=b(ac)=(ba)c=ec=c$, thus $b=c$.
\end{proof}
\begin{remark}
    Since the inverse of an element $a$ is unique, we will now denote it
    $\inv{a}$.
\end{remark}
\begin{corollary}
    $\inv{(\inv{a})}=a$ and $\inv{(ab)}=\inv{b}\inv{a}$
\end{corollary}
\begin{proof}
    Since inverses are unique, the $\inv{a} \in G$ has the unique inverse
    $\inv{\inv{a}}$. Then  taking $a\inv{a}=e$, applying inverses, we get
    $a(\inv{a}\inv{(\inv{a})})=e\inv{(\inv{a})}$, $a=\inv{(\inv{a})}$.

    Now Let $a,b \in G$, then $ab\inv{(ab)}=e$. Applying the inverse of $a$ on
    the right to both sides, we get  $b\inv{(ab)}=\inv{a}$; agian with the
    inverse of $b$ yields  $\inv{(ab)}=\inv{b}\inv{a}$.
\end{proof}

\begin{theorem}[Generalized Associativity]\label{1.2.2}
    Let $G$ be a group under a binary operation  $\ast$, then for any  $a_1,
    a_2, \dots, a_n \in G$, the product $a_1 \ast a_1 \ast \dots \ast a_n$ is
    independent of the ordering of any brackets.
\end{theorem}
\begin{proof}
    By induction on $n$, for  $n=1$, we just have the element  $a_1$, for $n=2$
    we have $a_1a_2$ has only one possible bracketing $(a_1b_1)$; and for $n=3$,
    the associativty group law guarantees  $a_1(a_2a_3)=(a_1a_2)a_3$.

    Now for any $k<n$, the the braketing of  $k$ elements  $a_1, \dots, a_k$ is
    can be reduced to the expression
    \begin{equation*}
        a_1 \ast (a_2 \ast (a_3 \ast ( \dots \ast a_k))).
    \end{equation*}
Now we see that $a_1 \ast \dots \ast a_n$ can be bracketed into the products:
    \begin{equation*}
        (a_1 \ast \dots \ast a_k) \ast (a_{k+1} \ast \dots \ast a_n)
    \end{equation*}
which can be bracketed, by hypothesis as:
\begin{equation*}
    (a_1 \ast (a_2 \ast (a_3 \ast ( \dots \ast a_k)))) \ast
    (a_{k+1} \ast (a_{k+2} \ast (a_{k+3} \ast ( \dots \ast a_n))))
\end{equation*}
Therefore, applying the assocaitive group law to this product, we get that
\begin{equation*}
    a_1 \ast \dots \ast a_n=a_1 \ast (a_2 \ast (a_3 \ast ( \dots \ast a_n)))
\end{equation*}
This completes the proof.
\end{proof}

\begin{theorem}[The Cancellation Laws]\label{1.2.3}
    Let $G$ be a group under a binary operation  $\ast$. Then for  $a,b,c \in
    G$, we have:
    \begin{enumerate}
        \item[(1)] $ab=ac$ implies  $b=c$ (Left Cancellation Law).

        \item[(2)] $ba=ca$ implies  $b=c$ (Right Cancellation Law).
    \end{enumerate}
\end{theorem}
\begin{proof}
    Suppose that $ab=ac$, then applying the inverse of $a$ on the left, we get
     $(\inv{a}a)b=(\inv{a}a)c$, hence $eb=ec$, thus  $b=c$. Similarly, we get
     $b=c$ if we apply  $\inv{a}$ to the right in the equation $ba=ca$.
\end{proof}
\begin{corollary}
    For $x,y \in G$, the equations  $ax=b$ and  $ya=b$ have unique solutions.
\end{corollary}
\begin{proof}
    We have $x=b\inv{a}$ and $y=\inv{a}b$. Since inverses are unique, so are the
    solutions $x$ and $y$.
\end{proof}

\begin{definition}
    Let $G$ be a group under a binary opeartion  $\ast$. For any $a \in G$, and
    $n \in \Z^+$, we define the  \textbf{$n$-th power} of $a$ to be:
    \begin{equation*}
        a^n=\underbrace{a \ast \dots \ast a}_{n \text{ times.}}
    \end{equation*}
    We define $a^0=e$ and  $a^{-n}=(\inv{a})^n$.
\end{definition}

\begin{lemma}\label{1.2.4}
    Let $G$ be a group under a binary operation  $\ast$, and let  $a \in G$ and
     $m,n \in \Z^+$. Then:
     \begin{enumerate}
         \item[(1)] $a^ma^n=a^{m+n}$.

         \item[(2)] $(a^m)^n=a^{mn}$.
     \end{enumerate}
\end{lemma}
\begin{proof}
    We have by definition that $a^ma^n=\underbrace{a \ast \dots \ast a}_{m
    \text{ times.}}\underbrace{a \ast \dots \ast a}_{n \text{ times.}}=
    \underbrace{a \ast \dots \ast a}_{m+n \text{ times.}}=a^{m+n}$.

    Likewise, $(a^m)^n=\underbrace{a^m \ast \dots \ast a^m}_{n \text{ times.}}=
    \underbrace{a \ast \dots \ast a}_{nm \text{ times.}}=a^{mn}$.
\end{proof}

We can now, unless context isn't clear enough, drop all mention to the binary
operation of a group.

\begin{definition}
    We define the \textbf{order} of a group $G$ to be the number of elements of
    $G$ and denote it  $\ord{G}$. That is, $\ord{G}=|G|$. If $G$ is infinite,
    then we say that  $G$ has  \textbf{infinite order}; otherwise, we say $G$ is
    of  \textbf{finite order}.
\end{definition}

\begin{definition}
    Let $G$ be a group and let  $a \in G$. We define the \textbf{order} of $a$
    to be the smallest positive integer  $n \in \Z^+$ for which $a^n=e$. If
    there is no such integer $n$, then we say  $a$ has  \textbf{infinite order},
    otherwise, we say $a$ has \textbf{finite order}, and write $\ord{a}=n$.
\end{definition}

We conclude the section with more examples and one last definition.

\begin{lemma}\label{1.2.5}
    Let $G$ be a group, suppose  $a,b \in G$ are elements with  $\ord{a}=m$,
    $\ord{b}=n$, for $m,n \in \Z^+$, and that  $ab=ba$. Then  $\ord{ab}=[m,n]$.
\end{lemma}
\begin{proof}
    Let $\ord{ab}=k$ for $k \in \ \Z^+$. Since  $a$ and  $b$ commute, we get
    $a^k=e$ and  $b^k=e$. Now, by the division theorem, there are integers
    $q_1,q_2,r_1,r_2 \in \Z^+$ such that $k=q_1m+r_1$ and $k=q_2n+r_2$. Then we
    get that $a^k=a^{q_1m+r_1}=a^{q_1m}a^{r_1}=a^{r_1}=e$. Since $\ord{a}=m$,
    this makes $r_1=0$. Similarly, we get that $b^k=b^{r_2}=e$ implies $r_2=0$.
    Therefore, $k=q_1m=q_2n$; moreover, $\ord{ab}=k$ is minimal by definition,
    thus we get $\ord{ab}=k=[m,n]$.
\end{proof}
\begin{corollary}
    For $a_1, \dots, a_k \in G$, with $\ord{a_i}=n_i$ and $a_ia_j=a_ja_i$ for
    all  $1 \leq i,j \leq n$, then  $\ord{a_1a_2 \dots a_k}=[n_1, \dots, n_k]$.
\end{corollary}
\begin{proof}
    By induction, when $k=1$, the case is trivial. Now, for  $k=2$, the result
    follows by the above theorem. Now suppose that $\ord{a_1 \dots a_k}=[n_1,
    \dots, n_k]$. Take $a_{k+1} \in G$ with $\ord{a_{k+1}}=n_{k+1}$ and $(a_1
    \dots a_k)a_{k+1}=a_{k+1}(a_1 \dots a_k)=a_1 \dots a_ka_{k+1}$. Then, by the
    above theorem, we get $\ord{a_1 \dots a_ka_{k+1}}=[[n_1, \dots, n_k],
    n_{k+1}]=[n_1, \dots, n_k, n_{k+1}]$.
\end{proof}

\begin{example}
    \begin{enumerate}
        \item[(1)] $\ord{\vbrack{e}}=1$.
        \item[(1)] In any group $G$,  $\ord{e}=1$, and if $a \in G$ has
            $\ord{a}=1$, then necessarily, $a=e$. That is in any group, the only
            element of order  $1$ is the identity.

        \item[(2)] The additive groups $\Z$,  $\Q$,  $\R$, and  $\C$ have
            infinite order, and their nonzero elements also have infinite order.

        \item[(3)] The multiplicative group $\C^*$ has infinite order, morover,
            since $i^4=1$ and $i^3=-i$, $\ord{i}=4$.

        \item[(4)] $\ord{\faktor{\Z}{n\Z}}=n$, and every element is of finite
            order. $\ord{U(\Z/n\Z)}=\phi(n)$, where $phi$ is the Euler totient
            function. Every element of $U(\Z/n\Z)$ also has finite order.
    \end{enumerate}
\end{example}

\begin{definition}
    Let $G=\{g_1, \dots, g_n\}$ be a finite group of order $n$ with $g_1=e$. We
    define the \textbf{Cayley table}, or \textbf{multiplication table} of $G$ to
    be the  $n \times n$ matrix defined by the entries $(g_{ij})$ where
    \begin{equation*}
        g_{ij}=g_ig_j
    \end{equation*}
    for $1 \leq i,j \leq n$.
\end{definition}
