%----------------------------------------------------------------------------------------
%	SECTION 1.1
%----------------------------------------------------------------------------------------

\section{Perfect Secrecy and The One-Time Pad.}
\label{section1}

We assume a cryptosystem $(\Pc,\Cc,\Kc)$ with cipher $(e,d)$ is used, where a
key $K$ is used for only one encryption. Let  $\Pc$ have a probability
distribution represented by the random variable  $X$, and assume the key $k \in
\Kc$ is chosen according to a probaility distribution represented by $K$. We
define the set of all possible ciphertexts encrypted with  $k$ to be:
\begin{equation}
    C(k)=\{e_k(x) : x \in \Pc\}
\end{equation}

Then we can also define a probaility distribution on $\Cc$ represented by the
random variable $Y$, such that for every $y \in \Cc$,  $P(Y=y)=\sum_{y
\in C(K)}{P(X=x)P(K=k)}$ and
$P(Y=y|X=x)=\sum_{y \in C(K)}{P(K=k)}$. Then by Baye's theorem, we have:
\begin{equation}
    P(X=x|Y=x)=\frac{P(X=x)P(Y=y|X=y)}{p(Y=y)}=\frac{P(X=x)\sum{P(K=k)}}{\sum{P(X=x)
    P(K=k)}}
\end{equation}

\begin{example}
    Let $\Pc=\{a,b\}$ where $P(a)=\frac{1}{4}$ and $P(b)=1-P(a)=\frac{3}{4}$.
    Let $\Kc=\{K_1,K_2,K_3,K_4\}$ with $P(K_1)=\frac{1}{2}$,
    $P(K_2)=P(K_3)=\frac{1}{4}$, and let $\Cc=\{1,2,3,4\}$. Definee the
    encryption rules $e_{K_1}:a \rightarrow 1,b \rightarrow 2$, $e_{K_2}:a
    \rightarrow 2, b \rightarrow 3$, and $e_{K_3}:a \rightarrow 3, b \rightarrow
    4$ we get the following matrix whose $(P(K_i,X))$
    \begin{equation*}
        \begin{pmatrix}
            1   &   2   \\
            2   &   3   \\
            3   &   4   \\
        \end{pmatrix}
    \end{equation*}
    We find the probability distribution on $\Cc$ to be  $P(1)=\frac{1}{8}$,
    $P(2)=\frac{3}{8}+\frac{1}{16}$, $P(3)=\frac{1}{4}$, and
    $P(4)=\frac{1}{16}$. We find the conditional probaility distrobution to be:
    \begin{align*}
        P(a|1)=1                &&  P(b|1)=0                \\
        P(a|2)=\frac{1}{7}      &&  P(b|1)=\frac{6}{7}      \\
        P(a|3)=\frac{1}{7}      &&  P(b|1)=\frac{3}{4}      \\
        P(a|4)=0                &&  P(b|1)=1                \\
    \end{align*}
\end{example}

\begin{definition}
    We say a cryptosystem ha \textbf{perfect secrecy} if $P(x|y)=P(x)$ for all
    possible plaintext elements $x$ and all possible plaintext elements $y$.
\end{definition}

\begin{example}
    In the above example, the cryptosystem $(\Pc,\Cc,\Kc)$ has perfect secrecy
    only when $y=3$; $P(a|3)=P(a)=\frac{1}{4}$.
\end{example}

\begin{theorem}\label{2.1.1}
    Let $\Pc=\Cc=\Kc=\faktor{\Z}{n\Z}$ and suppose $P(K)=\frac{1}{n}$ fo all $K
    \in \faktor{\Z}{n\Z}$. Then for any plaintex distribution, the shift cipher
    has perfect secrecy.
\end{theorem}
\begin{proof}
    Given the encryption rule $e:x \rightarrow x+K \mod{n}$, computing the
    probaility distribution on $\Cc=\faktor{\Z}{n\Z}$, we have
    $P(Y=y)=\sum_K{P(K)P(d_K(y))}=\frac{1}{n}\sum{P(x=y-K)}$. Now if $x$ and $y$
    are given plaintext and ciphertext elements, then  $d_k(y)=x-K \mod{n}$ is a
    permutation on $\faktor{\Z}{n\Z}$, hebce we get
    $\sum{P(x=y-K)}=\sum{P(X=x)}=1$; consequently, $p(y)=\frac{1}{n}$. Now
    $p(y|x)=P(K=y-x \mod{n})=\frac{1}{n}$. Therefore, by Baye's theorem, we get
    $P(x|y)=P(x)$.
\end{proof}
\begin{remark}
    This theorem says that the shift cipher is unbreakable provided a new random
    key is used to encrypt each plaintext element. This is computationally
    inefficient.
\end{remark}

\begin{theorem}\label{2.1.2}
    Let $(\Pc,\Cc,\Kc)$ be a cryptosystem where $|\Kc|=|\Cc|=|\Pc|$. Then
    $(\Pc,\Cc,\Kc)$ has perfect secrecy if, and only if every secret key is used
    with equal probailitiy $\frac{1}{|\Kc|}$ and for all $x \in \Pc$,  $y \in
    \Cc$, there is a unique  $K \in \Kc$ for which  $e_K(x)=y$.
\end{theorem}
\begin{proof}
    Suppose $(\Pc,\Cc,\Kc)$ has perfect secrecy. There is atleast one key $K$
    with  $e_K(x)=y$, so we get $|\Cc| \leq |C(x)| \leq |\Kc|$, where
    $C(x)=\{e_K(x) : K \in \Kc\}$. Then by assumption $|\Cc|=|C(x)|=|\Kc|$ which
    implies that $e_{K_1}(x)=e_{K_2}(x)=y$ only when $K_1=K_2$. Now, again by
    assumption, since $P(x|y)=P(x)$, it follows that $P(K_i)=P(y)$ which implies
    the keys are used with equal probability.

    COnversely, let $n=|\Kc|$ and  $\Pc=\{x_1, \dots, x_n\}$. Let $y \in \Cc$ be
    a ciphertext element, and suppose that  $e_{K_i}(x_i)=y$ for unique $K_i$,
    $1 \leq i \leq n$. Then by Baye's theorem,
    $P(x_i|y)=\frac{P(K=K_i)P(x_i)}{P(y)}$. Now, since every $K_i$ is chosen
    with probability $ \frac{1}{n}$, we see $P (K=K_i)=P(y)=\frac{1}{n}$. Thuse
    $P (x_i|y)=P(x_i)$.
\end{proof}

We now define the one-time pad.

\begin{definition}
    Let $n \in \Z^+$ and let  $\Pc=\Cc=\Kc=(\faktor{\Z}{2\Z})^n$. For $K \in
    (\faktor{\Z}{2\Z})^n$, define the pair $(e,d)$ by the rules $e:x \rightarrow
    x+K \mod{2}=(x_1+K_1, \dotsm x_2+K_2) \mod{2}$ and $d:y \rightarrow y+K
    \mod{2}=(y_1+K_1, \dots, y_n+K_n) \mod{2}$. We call the cipher $(e,d)$ the
    \textbf{one-time pad}.
\end{definition}

\begin{theorem}\label{2.1.3}
    The one-time pad defines a perfectly secure cryptosystem.
\end{theorem}
