%----------------------------------------------------------------------------------------
%	SECTION X.X
%----------------------------------------------------------------------------------------

\section{The Ordered Field Axioms}

\begin{enumerate}[label=(\arabic*)]
    \begin{enumerate}[label=(\arabic*)]
        \item [(1)]
            \begin{enumerate}
                \item We have  $a^++a^-=\frac{1|a|+a-a}{2}=|a|$ , and  $a^+-a^-= \frac{|a|-|a|+a+a}{2}=a$ .                          
                                                                                                                               
              \item Now Let $a \geq 0$, then  $a^+=\frac{a+a}{2}=a$ and $a^-=\frac{a-a}{2}$, and for $a<0$, then               
                   $a^+=\frac{a-a}{2}=0$  and  $a^-=\frac{-a-a}{2}=-a$ .		
            \end{enumerate}
    \end{enumerate}

    \item [(5)]
        \begin{enumerate}
            \item If $a<a<1$, then  $0<1-a<1$ so  $(1-a)^2<1-a$, then $1-a=-(a-1)<\sqrt{q-a}$, hence  $-\sqrt{1-a}<a-1$, 
                adding  $1$ to both sides we hsve  $1-\sqrt{1-a}=b<a$.

            \item If  $2<a$, then $1<a-1$, so $a-1<(a-1)^2$, thus $\sqrt{a-1}<a-1$, adding $1$ again to both sides, we 
                again have $1+\sqrt{a-1}=b<a$.

            \item Let $a \leq b$, then  $0 \leq ab \leq b^2$, so  $G(a,b)=\sqrt{ab} \leq b$. Likewise,  $0 \leq a+b \leq 2b$ 
                so $A(a,b)=\frac{a+b}{2}$. Now if $A(a,b) < G(a,b)$, then  $a+b \leq 2\sqrt{ab}$, so  $a^2+2ab+b^2 < 4ab$, 
                and so  $a^2+b^2 < 2ab$, which is absurd; thus $G(a,b) \leq A(a,b)$. Moreover, if $G(a,b)=A(a,b)$, then 
                $2\sqrt{ab}=a+b$ implying $a=b$, and now if  $a=b$, then  $G(a,b)=\sqrt{b^2}=b$ and  $\frac{b+b}{2}=b$, hence 
                $G(a,b)=A(a,b)$.
        \end{enumerate}

    \item [(7)]
        \begin{enumerate}
            \item $|x| \leq 1$, then  $|x|+1 =|x+1| \leq 2$, multiplying by  $|x-1|$, then we get  $|x+1||x-1|=|(x+1)(x-1)|=
                |x^2-1| \leq 2|x-1|$.

            \item Let  $-1 \leq x \leq 2$, then  $1 \leq x+2 \leq 4$, hence $|x+2| \leq 4$, then again, $|x^2+x-2| \leq 4|x-1|$.

            \item We have $|x|-2=|x-2| \leq -1 < 3$, so  $|(x-2)(x+1)|=|x^2-x-2| \leq 3|x+1|$.

            \item If  $0<|x-1|<1$, adding $1$, then  $0<|x|<2$, by part (b), $|x^2+x-2| \leq 4|x-1|<8|x-1|$. Now if $|x-1|=0$, 
                then $x=1$ and so  $1^2+1-2=0<0$ which is absurd, so this result is not true for  $0 \leq |x-1|$.
        \end{enumerate}

    \item [(9)] We have $(a_1b_1+a_2b_2)^2=(a_1b_1)^2+2a_1b_1a_2b_2+(a_1b_2)^2$, and $(a_1^2+a_2^2)(b_1^2+b_2^2)=(a_1b_1)^2
        +a_1^2b_2^2+a_2b_1+(a_2b_2)^2$. Now $a_1 \leq a_1^2$ and $a_2 \leq a_2^2$, so $a_1b_1 \leq a_1^2b_1^2$ and $a_2b_2 \leq a_2^2b_2^2$. 
        So $a_1b_1a_2b_2 \leq (a_1b_1a_2b_2)^2=(a_1b_2a_2b_1)^2$, by problem $(5c)$  $a_1b_1a_2b_2 \leq \frac{a_1b_2+a_2b_1}{2} \leq 
        \frac{a_1^2b_2^2+a_2^2b_1^2}{2}$, therefore $2a_1b_1a_2b_2 \leq a_1^2b_2^2+a_2^2b_1^2$; and so $(a_1b_1+a_2b_2)^2 
        \leq (a_1^2+a_2^2)(b_1^2+b_2^2)$.

    \item [(10)]
        \begin{enumerate}
            \item Observe that $|x|<|a|+\epsilon$ and  $|y|<|b|+\epsilon$. Also observe that $|(xy-xb)+(xb-ab)| \leq 
                |xy-xb|+|xb-ab|<|(|ab|-b\epsilon)-(|a|-\epsilon)b|+|(|a|-\epsilon)-ab| = |(|ab|-a\epsilon-b\epsilon+\epsilon^2)|+
                    ||ab|-b\epsilon-ab|=|-a\epsilon+\epsilon^2|+||ab|-b\epsilon-ab|=|-a\epsilon|+|-b\epsilon|=|-a\epsilon|
                    +|-b\epsilon|+\epsilon^2=(|a|+|b|)\epsilon+\epsilon^2$.

                \item The method is the same, noting that $x^2=(|a|+\epsilon)$ and  $y^2=(|b|+\epsilon)$.
        \end{enumerate}

\end{enumerate}
