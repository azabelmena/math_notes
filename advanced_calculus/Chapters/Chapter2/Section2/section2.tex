%----------------------------------------------------------------------------------------
%	SECTION X.X
%----------------------------------------------------------------------------------------

\section{Limit Theorems}

There are two problems that are worth discussing, the first is that if we have a sequence, how do we know if it converges? 
The second problem is determining to where it converges, that is to say, what is the limit; if we cannot find the concrete 
value, how can we approximate it?

\begin{theorem}[Sandwhich Theorem]\label{2.2.1}

    Consider real valued sequences $\{x_n\}$,  $\{y_n\}$, and  $\{w_n\}$. Suppose that $\lim{x_n}=\lim{y_n}=a$ and that there 
    is an $N_0 \in \N$ sucht hat  $x_n \leq w_n \leq y_n$ for all  $n \geq N_0$. Then  $\lim_{n \rightarrow \infty}{w_n}=a$. 
\end{theorem}
\begin{proof}
    Let $\epsilon>0$ and let  $\{x_n\}$ and  $\{y_n\}$ both converge to  $a$. Then by definition there are  $N_1,N_2 \in \N$ 
    such that $|x_n-a|<\epsilon$ and  $|y_n-a|<\epsilon$ for  $n \geq N_1,n_2$. Now choose $N=\max{N_0,N_1,N_2}$, if 
    $n \geq N$, we have  $-\epsilon<x_n-a<\epsilon$, and we also have  $x_n-a<w_n-a<y_n-a$, thus we have that:
        \begin{equation*}
            -\epsilon<x_n-a<w_n-s<y_n-a<\epsilon
        \end{equation*}
    Thus we have that $|w_n-a|<\epsilon$.
\end{proof}

\begin{corollary}
    If $x_n \rightarrow \infty$ as  $n \rightarrow \infty$, and  $\{y_n\}$ is bounded, then $x_ny_n \rightarrow 0$ as 
    $n \rightarrow \infty$.
\end{corollary}
\begin{proof}
    We have that $\{y_n\}$ is bounded, hence, there is $M>0$ such that  $|y_n|<M$ for all  $n \in \N$. And since $\{x_n\}$ 
    converges to $0$ we have that for any $\epsilon$ there is an  $N \in \N$ such that for  $n \geq N$,  $|x_n-0|<\frac{\epsilon}{M}$.
    For $|x_ny_n-0|=|x_ny_n|<M|x_n|<M\frac{\epsilon}{M}=\epsilon$. Therefore, $x_ny_n \rightarrow 0$ as  $n \rightarrow \infty$.
\end{proof}

\begin{example}
    Find $\lim_{n \rightarrow \infty} 2^{-n}\cos(n^3-n^2+n-13)$.
\end{example}
\begin{solution}
    We have that $\cos$ is bounded by $1$ for any value it takes, and we also know that  $2^n>n$ hence  $0<2^{-n}<\frac{1}{n}$.
    We know that $\frac{1}{n} \rightarrow 0$ as $n \rightarrow \infty$, thus by the sandwhich theorem,  $2^{-n} \rightarrow 0$.
    Then by the corollary, since  $\cos$ is bounded, we get that $\lim_{n \rightarrow \infty} 2^{-n}\cos(n^3-n^2+n-13)=0$.

\end{solution}

\begin{theorem}\label{2.2.2}

    Let $E \subseteq \R$. If  $E$ has a finite least upper bound  (respectively a finite greatest lower bpund), then there is 
    a sequence $\{x_n\} \subseteq E$ such that  $x_n \rightarrow \sup{E}$ as  $n \rightarrow \infty$
\end{theorem}
\begin{proof}
    If the least upper bound of $E$ is finite, then for every  $\epsilon>0$, there is an  $x \in E$ such that  
    $\sup{E}-\epsilon < x \leq \sup{E}$. For all  $n \geq N$, choose  $\epsilon=\frac{1}{n}$. Then there is one element 
    $x_n \in E$ such that  $\sup{E}-\frac{1}{n}<x_n \leq \sup{E}$. Now construct sequences $\{w_n\}$, and  $\{y_n\}$ such 
    that $w_n=\sup{E}-\frac{1}{n}$ and $y_n=\sup{E}$. Then $\lim{w_n}=\lim{y_n}=\sup{E}$ and  $w_n<x_n \leq y_n$; thus by the 
    sandwhich theorem,  $\lim{x_n}=\sup{E}$. Analogously, we get the same result for $\inf{E}$.
\end{proof}

\begin{theorem}\label{2.2.3}
    Suppose that $\{x_n\}$ and  $\{y_n\}$ are real valued sequence, and let  $\alpha \in \R$. If  $\{x_n\}$ and  $\{y_n\}$ 
    converge,and suppose that $n \rightarrow \infty$. Then:
        \begin{enumerate}[label=(\arabic*)]
            \item $\lim{x_n+y_n}=\lim{x_n}+\lim{y_n}$.

            \item $\lim{\alpha x_n}=\alpha \lim{x_n}$.

            \item $\lim{x_ny_n}=\lim{x_n}\lim{y_n}$.

            \item $\lim{\frac{x_n}{y_n}}=\frac{\lim{x_n}}{\lim{y_n}}$, provided that $y_n \neq 0$ does not converge to $0$.
        \end{enumerate}
\end{theorem}
\begin{proof}
    Suppose that $\{x_n\} \rightarrow x$ and  $\{y_n\} \rightarrow y$ as  $n \rightarrow \infty$, for $x,y \in \R$. Then:
        \begin{enumerate}[label=(\arabic*)]
            \item For any $\epsilon>0$, there exists $N \in \N$ such that for $n \geq N$ we have  $|x_n-x|<\frac{\epsilon}{2}$ 
                and $|y_n-n|<\frac{\epsilon}{2}$. Then $|(x_n-x)+(y_n-y)|=|(x_n+y_n)+(x-y)|< \frac{\epsilon}{2}+
                \frac{\epsilon}{2}=\epsilon$.

            \item For $\epsilon>0$, let  $N \in \N$ such that for  $n \geq N$,  $|x_n-x|<\frac{\epsilon}{|\alpha|}$. Then 
                multiplying the inequality by $|\alpha|$, we get  $|\alpha||x_n-x|=|\alpha x_n-\alpha x|<\epsilon$.

            \item Notice that $y_n-y \rightarrow 0$, and  $x_n$ is bounded (by theorem $2.1.1$), so $x_n(y_n-y) 
                \rightarrow 0$. Similarly, $y_n(x_n-x) \rightarrow 0$. Also notice that $|x_ny_n-xy|=|x_n(y_n-y)+y_n(x_n-x)| 
                \leq |x_n(y_n-y)|+|y_n(x_n-x)|$.

                Now for  $\epsilon>0$, there is some  $N \in \N$ such that for  $n \geq N$,  $|x_n-y|<\frac{\epsilon}{2}$ 
                and $|y_n-y|<\frac{\epsilon}{2}$. Then $|x_n(y_n-y)|+|y_n(x_n-x)|<\epsilon$. Therefore, $|x_ny_n-xy|<\epsilon$.

            \item Provided that $y_n \neq 0$ does not converge to  $0$, notice that:
                \begin{equation*}
                    |\frac{x_n}{y_n}-\frac{x}{y}| &= |\frac{x_ny-xy_n}{yy_n}| = \frac{1}{|yy_n|}|x_n(y-y_n)+y_n(x_n-y)|
                \end{equation*}  
            By the triangle inequality we have:
                \begin{equation}
                    \frac{1}{|yy_n|}|(x_n(y-y_n)|+|y_n(x_n-x)|)=|\frac{x_n}{yy_n}(y-y_n)|+|\frac{1}{y}(x_n-x)|
                \end{equation}
                Now we see that $\frac{1}{y_n}(x_n-x) \rightarrow 0$ we also have that $y-y_n \rightarrow 0$, and $x_n$ 
                converges, so  $x_n$ is bounded. Now since  $y_n,y \neq 0$, we must show that  $ \frac{1}{y_n}$ is bounded. 
                We have that $\lim{|y_n|}=|y|$, then  $|y|>0$, for $\epsilon=\frac{1}{2}y$, there is an $N \in \N$ such that 
                for all $n \geq N$ 
                    \begin{equation*}
                        ||y_n|-|y||<\frac{1}{2}y=\epsilon
                    \end{equation*}
            then we get that $\frac{1}{|y_n|}<\frac{2}{|y|}$; hence $\frac{1}{y_n}$. For $k=1, \dots, N-1$ let
            $M=\frac{2}{|y|}+\max{\frac{1}{|y_1|},\dots,\frac{1}{y_{N-1}}}$, thus $\frac{1}{|y_n|}$ is bounded by $M$.

            Now since  $ \frac{1}{|y_n|}$ is bounded, so is $\frac{x_n}{|yy_n|}$, thus we get that $\frac{x_n}{yy_n}(y-y_n) 
            \rightarrow 0$, thus we get that
                \begin{equation*}
                	\frac{x_ny-xy_n}{yy_n}=\frac{x_n}{y_n}-\frac{x}{y} \rightarrow 0
                \end{equation*}
            This completes the proof.
        \end{enumerate}
\end{proof}

\begin{example}
    Find the limit: $\lim_{n \rightarrow \infty} \frac{n^3+n^2-1}{1-3n^3}$.
\end{example} 
\begin{solution}
    We divide the numerator and the denominator by the highest power of $n$ to get:
        \begin{equation}
            \frac{1+\frac{1}{n}-\frac{1}{n^3}}{\frac{1}{n^3}-3}		
        \end{equation}
    Then we get that as $n \rightarrow \infty$ then
        \begin{equation}
            \frac{1+\frac{1}{n}-\frac{1}{n^3}}{\frac{1}{n^3}-3} \rightarrow \frac{1+0-0}{0-3}=-\frac{1}{3}
        \end{equation}
        Hence, $\lim_{n \rightarrow \infty}{\frac{n^3+n^2-1}{1-3n^3}}=-\frac{1}{3}$
\end{solution}

A sequence $\{n\}$ diverges, so does the sequence  $\{(-1)^n\}$, however, they diverge differently; the sequence  $\{n\}$ 
just gets bigger and bigger as  $n \rightarrow \infty$, however, $\{(-1)^n\}$ oscilates as  $n \rightarrow \infty$.

\begin{definition}
    Let $\{x_n\}$ be a real valued sequence. Then
        \begin{enumerate}[label=(\arabic*)]
            \item $\{x_n\}$ is said to \textbf{diverge} to $+\infty$ if and only if for every $M \in \R$, there is an 
                $N \in \N$ such that for $n \geq N$, $x_n>M$.

            \item $\{x_n\}$ is said to \textbf{diverge} to $-\infty$ if and only if for each $M \in \R$, there is an $N \in \N$ such 
                that for $n \geq N$, $x_n<M$.
        \end{enumerate}
\end{definition}

\begin{theorem}\label{2.2.4}

    Suppose that $\{x_n\}$ and  $\{y_n\}$ are real valued sequence such that  $x_n \rightarrow +\infty$ as $n \rightarrow \infty$.
        \begin{enumerate}[label=(\arabic*)]
            \item If $\{y_n\}$ is bounded below, then  $x_n+y_n \rightarrow +\infty$ as  $n \rightarrow \infty$. Similarly 
                if $\{y_n\}$ is bounded above, then  $x_n+y_n \rightarrow -\infty$.

            \item If $\alpha>0$, then  $\alpha x_n \rightarrow +\infty$ as  $n \rightarrow \infty$. Similarly, if  $\alpha<0$, 
                then  $\alpha x_n \rightarrow -\infty$.

            \item If  $y_n>M_0$ for some $M_0>0$ abd for all $n \in \N$. Then  $x_y_ \rightarrow  +\infty$ as  $n \rightarrow \infty$.

            \item If  $\{y_n\}$ is bounded, and $x_n \neq 0$, then  $ \frac{y_n}{x_n} \rightarrow 0$ as $n \rightarrow \infty$.
        \end{enumerate}
\end{theorem}
We defer the proof.

\begin{corollary} 
    Let $\{x_n\}$ and  $\{y_n\}$ be real valued sequences and  $\alpha, x, y$ be extended real numbers. If  $x_n \rightarrow x$ 
    and  $y_n \rightarrow y$ as  $n \rightarrow \infty$, then  $x_n+y_n \rightarrow x+y$ as  $n \rightarrow \infty$. and 
    $\alpha x_n \rightarrow \alpha x$ and $x_ny_n \rightarrow xy$ ; We disclude of course the indeterminate forms for the 
    extended real numbers.
\end{corollary}

\begin{theorem}[The Comparison Theorem]\label{2.2.5}

    Suppose that $\{x_n\}$ and  $\{y_n\}$ are convergent sequences, if there is an  $N_0 \in \N$ such that $x_n \leq y_n$ 
    for $n \geq N_0$, then $\lim{x_n} \leq \lim{y_n}$ as $n \rightarrow \infty$. In partucular, if $x_n \in [a,b]$ converges 
    to $c$, then  $c \in [a,b]$.
\end{theorem}
\begin{proof}
    Suppose that the statement is false, then $\lim{x_n}>\lim{y_n}$ as  $n \rightarrow \infty$. Let  $\lim{x_n}=x$ and let  
    $\lim{y_n}=y$, and let $\epsilon=\frac{x-y}{4}$, then there is an $N \in N$ such that for  $n \geq N$,  $|x_n-x|<\epsilon$, 
    and $|y_n-y|<\epsilon$ Now we have thay  $x_m>x-\epsilon>y+\epsion>y_n$, a contradiction. Therefore  $\lim{x_n} \leq 
    \lim{y_n}$.

    Now in partucular, we have that if $x_n \in[a,b]$, then  $a \leq x_n \leq b$, letting  $\lim_{n \rightarrow \infty}{x_n} 
    =c$, we get from above that  $a \leq c \leq b$, hence  $v \in [a,b]$.
\end{proof}

\begin{HW} 
    Exercises $3$,  $5$,  and $6$ on page $ 44$.
\end{HW}
