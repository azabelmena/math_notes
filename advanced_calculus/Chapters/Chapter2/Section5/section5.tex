%----------------------------------------------------------------------------------------
%	SECTION 2.5
%----------------------------------------------------------------------------------------

\section{Limits, Least Upperbounds and Greatest Lowerbounds.}\hspace{10mm}

We know that a sequence converges to a point if and only if it is a Cauchy 
sequence. We would like to talk about related concepts of sequences in general.
Considet the sequence $x_1,x_2,x_3, \dots$. And construct the following seqeunce: 
$S_1=\sup\{x_1,x_2,x_3, \dots\}$, $T_1=\inf\{x_1,x_2,x_3, \dots\}$. Cntinuing, 
constructs $S_2=\sup\{x_2,x_3,x_4, \dots\}$ and $T_2=\inf\{x_2,x_3,x_4, \dots\}$.
Continuing allong this line, we get the terms $S_n=\sup\{x_n,x_{n+1}, \dots\}$ and 
$T_n=\inf\{x_n,x_{n+1}, dots\}$. We see that  $\{S_n\}$ is a decreasing sequence 
of extended real numbers, now it could turn out that  $\{S_n\}$ is a complete 
sequence of  $\infty$; if at a certain point, the term  $s_n \neq \infty$, then 
the rest of the sequence is  not  $\infty$.

Similarly,  $\{T_n\}$ is an increasing sequence of extended real numbers. It could 
happen that  $\{T_n\}$ is  $-\infty$ everywhere, however, if at a certain point 
$t_n \neq -\infty$, then the wholw of the sequence is not  $-\infty $.

\begin{definition}
    Let $\{x_n\}$ be a real valued ssequence. Then the \textbf{limit supremum} is 
    the extended real number, $\limsup{x_n}=\lim(\sup_{k \geq n}\{x_k\})$ as , 
    $n \rightarrow \infty$. And the \textbf{limit infimum} is the extended real number $\liminf{x_n} 
    = \lim(\inf_{k \geq n}\{x_k\})$, as $n \rightarrow \infty.$
\end{definition}

We discuss some cases. If $s_n=\infty$ for all  $n \in \N$, then  $\limsup{x_n}=\infty$.
Now if  $s_n \neq \infty$, and if  $\{S_n\}$ is bounded below, then  $\limsup{x_n}$ 
is a real number. Now if  $S_n \neq \infty$ for all  $n \in \N$, and  $\{S_n\}$ 
is not bounded below, then  $\limsup{x_n}=-\infty$.

Similarly, if  $t_n=-\infty$ for all  $n \in \N$, then  $\liminf{x_n}=-\infty$. 
Now if  $t_n \neq -\infty$ fot all  $n \in \N$, and if  $\{T_n\}$ is bounded above, 
then  $\liminf{x_n}$ is a real number. Now if  $t_n \neq -\infty$ for all  $n \in \N$ 
and  $\{T_n\}$ is not bounded above, then  $\liminf{x_n}=\infty$.

\begin{theorem}\label{2.5.1}
    Let $\{x_n\}$ be a real valued sequence, and let  $s=\limsup{x_n}$ and let 
    $t=\liminf{x_n}$. Then there are subsequences  $\{x_{n_k}\}$ and  
    $\{x_{l_j}\}$ such that  $x_{n_k} \rightarrow s$ as $k \rightarrow \infty$, 
    and $x_{l_j} \rightarrow t$ as $j \rightarrow \infty$.
\end{theorem}
\begin{proof}
    We prove it only for the $\liminf$. We know that $t_n=\inf{x_k}$, and that 
    $t_n \rightarrow t$ as  $n \rightarrow \infty$  (for $t_n$ increasing). Now 
    if $t=-\infty$, then $t_n=-\infty$ for all  $n \in \N$. Then $ t_1=-\infty$, 
    so there is an $n_1$ such that $x_{n_1}<-1$. Now $t_{n_1+1}=-\infty$, then 
    there is an $n_2$ such that $x_{n_2}<-2$ and $n_2>n_1$. Continuing along this 
    process, we have $n_1<n_2<n_3< \dots$ and $x_{n_k}<-k$. Then we have that 
    $\lim{x_{n_k}}=-\infty$.

    Now if $t=\infty$, then  $t_n \neq -\infty$ for all  $n \in \N$, and  $\{T_n\}$ 
    is not bounded above. Now without loss of generality, assume that $t_n \neq -\infty$ 
    for all $n \in \N$. For $M=1$, there is an  $n_1$ such that $t_{n_1}>1$. 
    So $x_{n_1} \geq t_{n_1}>1$. Now for $M=2$, there is an  $n_2$ such that 
    $n_2 \geq n_1$, and $t_{n_2}>2$. Continuing this process, we get that 
    $n_1 \leq n_2\leq \dots$ and $x_{n_k}>k$.  


    Suppose that $-\inft<t<\infty$, then  $t$ is not  $t_n \neq -\infty$ for all  $n \in \N$, 
    and  $\lim{t_n}=t$. So  $t_n \rightarrow t$  (where $t_n$ is increasing). Then 
    for every $\epsilon>0$, there is an  $N \in \N$ such that for  $n \geq N$, we have 
    $|t-t_n|=t-t_n<\frac{\epsilon}{2}$. Now if $t_n=\inf\{x_k\}$, sor the same  $\epsilon$,
    there is an  $n_k \geq n$ such that  $t_n \leq x_{n_k}<t_n+\frac{\epsilon}{2}$. 
    Then ny the traingle inequality, we have $|t-t_n|=|t-t_n|+|t_n-x_{n_k}|<\epsilon$.
    Now consider  $\epsilon=\frac{1}{n}$, if $n=1$, then there is an  $N_1 \in \N$, 
    such that for  $n \geq N_1$, $|t-t_n|=t-t_n<\frac{1}{2}$. Then for 
    $t_{N_1}=\inf{x_k}$, there is an $n_1 \geq N_1$, with $t_{N_1} \leq x_{n_1} <
    t_{N_1}+\frac{1}{2}$. Combining these two relations we have $|t-x_{n_1}|<\frac{1}{2}$.
    Now for $n=2$, there is an  $N_2 \in \N$ with $N_2>n_1$ such that for $n\geq N_2$, 
    we have $|t-x_{n_2}|<\frac{1}{4}$. Continuing this process, we find a sequence 
    $\{n_k\}$ such that  $n_1<n_2< \dots$, and $|t-x_{n_k}|<\frac{1}{k}$. Thus 
    we have that $\lim_{k \rightarrow \infty}{x_{n_k}}=t$.
\end{proof}

\begin{theorem}\label{2.5.2}
    Let $\{x_n\}$ be a eal valued sequence and let  $x$ be an extended real number.
    Then $\{x_n\} \rightarrow x$ as  $n \rightarrow \infty$ if and only if:
        \begin{equation*}
            \limsup_{n \rightarrow \infty}{x_n}=
            \liminf_{n \rightarrow \infty}{x_n}=x
        \end{equation*}
\end{theorem}
\begin{proof}
    Suppose that $x_ \rightarrow x$ as  $n \rightarrow \infty$. Then all subsequences  
    $\{x_{n_k}\}$ will converge to  $x$ as  $k \rightarrow \infty$. Then by theorem 
    \ref{2.5.1}, we know that $\limsup{x_n}=\liminf{x_n}=x$. COnversely, suppose 
    that  $\limsup{x_n}=\liminf{x_n}=x$, then we have three cases. If  $x=\infty$, 
    then  $s_n=\infty$, and  $\{T_n\}$ is not bounded above, Thus for all  $M \in \R$, 
    there is an  $N \in \N$ such that  for $n \geq N$,  $t_n=M$, so  $x_n \geq t_n \geq M$.
    Thus  $\lim{x_n}=\infty$.

    Suppose now that  $x=-\infty$, then  $t_n=-\infty$ and  $\{S_n\}$ is not bounded below.
    So for all  $M \in \R$, there is an  $N \in \N$ such that for  $n \geq N$,  $s_n \leq M$. 
    So  $\lim{x_n}=-\infty$. Now suppose that $-\infty<x<\infty$. Then for all  $\epsilon>0$, 
    ther is an  $N \in \N$, such that  $0 \leq s_n-x<\frac{\epsilon}{2}$ and 
    $0 \leq x-t_n<\frac{\epsilon}{2}$. Then fo $m,n \geq N$, assume that  
    $|x_n-x_m|=x_m-x_n \leq s_m-t_n \leq \epsilon$. Hence, $\{x_n\}$ is a Cauchy 
    sequence, so  $x_n \rightarrow x'$ in  $\R$. So all its subsequences converge 
    to  $x'$. But we know that  $\limsup{x_n}=\liminf{x_n}=x$, thus  $x'=x$ and 
    we are done.
\end{proof}

\begin{theorem}\label{2.5.3}
    Let  $\{x_n\}$ be a real valued sequence. Then $\limsup{x_n}$ is the largest 
    value to which some subsequence of $\{x_n\}$ converges, as  $n \rightarrow \infty$. 
    Similiarly,  $\liminf{x_n}$ is the smallest value for which some subsequence 
    of  $\{x_n\}$ converges.  In particular, if $x_{n_k} \rightarrow x$ as  
    $k \rightarrow \infty$, then:
        \begin{equation*}
            \liminf_{n \rightarrow \infty}{x_n} \leq x \leq \limsup_{n \rightarrow \infty}{x_n}.
        \end{equation*} 
\end{theorem}
\begin{proof}
    Assume that a subsequence $\{x_{n_k}\}$ converges to  $x$ as  $k \rightarrow \infty$. 
    Then fix $N \in \N$, and choose  $K$ so that for $k \geq K$, $n_k \geq N$. 
    Now under this condition, $\inf{x_j} \leq x_{n_k} \leq \sup{x_j}$ for $j \geq N$ 
    for all  $k \geq K$. Now we let  $k \rightarrow \infty$, then  $x_{n_k} \rightarrow x$, 
    then by the comparison of the limit, we have  $\inf{x_j} \leq x \leq \sup{x_j}$ 
    for  $j \geq N$, again, by the comparison of limits, letting $N \rightarrow \infty$, 
    we have that  $\inf{x_j}$ is increasing, and $\sup{x_j}$ is decrasing, so 
    we get that $\liminf{x_n} \leq x \leq \limsup{x_n}$.
\end{proof}

\begin{corollary} 
    If $\{x_n\}$ is any real valued sequence, then :
        \begin{equation*}
            \liminf_{n \rightarrow \infty}{x_n} \leq \limsup_{n \rightarrow \infty}{x_n}.
        \end{equation*} 
\end{corollary}
\begin{proof}
    The proof follows form theorem \ref{2.5.3}.		
\end{proof}

\begin{remark} 
    A real valued sequence $\{x_n\}$ is bounded above if and only if  
    $\limsup{x_n}<\infty$ as  $n \rightarrow \infty$; and  $\{x_n\}$ is bounded below 
    if and only if  $\liminf{x_n}>-\infty$.
\end{remark}
\begin{proof}
    If $\{x_n\}$ is bounded above, there is an  $M$ such that  $x_n \leq M$ for 
    all $n$. THen  $\sup{x_j} \leq M$ for  $j \geq N$. Then we get that  
    $\limsup{x_n} \leq M$ as  $n \rightarrow \infty$. Conversely, if  $\limsup{x_n}=M$ 
    then for every  $\epsilon>0$, there exists an  $N \in \N$ such that:
        \begin{equation*}
            |\sup_{j \geq N}{x_j}-M|<\epsilon
        \end{equation*}
    Thus $M-\epsilon<\sup{x_j}<M+\epsilon$. Then choose  $M'=|x_1|+|x_2|+
    \dots+|x_{n-1}|+M+\epsilon$. Then  $x_j \leq M'$ for all  $j \in \N$. The 
    proof is similar for $\liminf{x_n}$.
\end{proof}

\begin{theorem}\label{2.5.4}
    If $x_n \leq y_n$ for suffieciently large  $N$, then  $\limsup{x_n} \leq 
    \limsup{y_n}$, and $\liminf{x_n} \leq \liminf{y_n}$ as  $n \rightarrow \infty$. 
\end{theorem}\
\begin{proof}
    We have $\sup{x_j} \leq \sup{y_j}$ for  $j \leq n$ for  $n$ sufficiently large.
    Then by the comparison theorem of the limit, we will have $\limsu{x_n} \leq 
    \limsup{y_n}$ as $n \rightarrow \infty$. The same holds for  $\liminf$.
\end{proof}

\begin{HW} 
    Problems $2$,  $3$, $4$, $5$, and $6$ on  page $35$.
\end{HW}
