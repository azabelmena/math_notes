%---------------------------------------------------------------------------------------%	SECTION 4.1
%---------------------------------------------------------------------------------------


\section{Riemann Sums.}

\begin{definition}
    Let $f:[a,b] \righarrow \R$. We define the \textbf{Riemann sum} of  $F$ with
    respect to some partition  $P$ of  $[a,b]$ to be a sum of the form:
        \begin{equation}
            \sum_{i=1}^{n}{f(t_i)(x_i-t_i-1)}		
        \end{equation} 
        where $t_i \in [x_{i-1},x_i]$. Now we say that the Riemann sums of $f$
        with respect to  $P$ \textbf{converges} to a point $I(f)$ as  $||P||
        \rightarrow 0$ if and only if for  $\epsilon>0$, there is a partition
        $P_{\epsilon}$ of $[a,b]$ such that:
            \begin{equation*}
                |\sum_{i=1}^{n}{f(t_i)(x_i-t_i-1)}-I(f)|<\epsilon		
             \end{equation*} 
        whenever $P_{\epsilon} \subseteq P$, and we write:
            \begin{equation}
                I(f)=\lim_{||P|| \rightarrow 0}{\sum_{i=1}^{n}{f(t_i)(x_i-t_i-1)}}
            \end{equation} 
\end{definition}

\begin{theorem}\label{5.2.1}
    Let $a,b \in \R$ with  $a<b$, and suppose that  $f:[a,b] \rightarrow \R$ is
    a bounded realvalued function. Then  $f$ is Riemann integrable if on
    $[a,b]$ if and only if  $I(f)$ exists, moreover:
        \begin{equation}
            I(f)=\int_{a}^{b}{f(x) dx}		
        \end{equation} 
\end{theorem}
\begin{proof}
    Suppose that $f$ is integrable on  $[a,b]$, and let $\epsilon>0$. Then there
    is a partition $P_{\epsilon}$ of  $[a,b]$ such that
    $L(f,P_{\epsilon})>\int{f}+\epsilon$ and
    $U(f,P_{\epsilon})<\int{f}+\epsilon$. By the approximation property, we have
    that  $\int{f}=\sup{L}=\inf{U}$. Now for any partiton $P$ finer than
    $P_{\epsilon}$, we have $L(f,P)>L(f,P_{\epsilon})$ and
    $U(f,P)<U(f,P_{\epsilon})$, thus let  $t_i \in [x_{i-1},x_i]$
        \begin{equation*}
             \int_{a}^{b}{f dx}-\epsilon<L(f,P) \leq \sum{f(t_i)(x_i-x_{i-1})} \leq U(f,P)<\int_{a}^{b}{f dx}+\epsilon
        \end{equation*}
    Then we get:
        \begin{equation*}
            |\sum{f(t_i)(x_i-x_{i-1})}-\int_{a}^{b}{f dx}|<\epsilon
        \end{equation*}
        so $I(f)=\int{f}$ exists.

    Conversely suppose that the Riemann sums of $f$ converges to $I(f)$.
    THen for every  $\epsilon>0$, there is a partition  $P_{\epsilon}$ such 
    that $|\sum{f(t_i)(x_i-x_{i-1})}-I(f)|<$ whenever $P_{\epsilon}
    \subseteq P$. Now for each $M_i$ and  $m_i$, there are  $u_i,v_i \in
    [x_{i-1},x_i]$ such that $M_j \geq f(u_i)>M_i-\epsilon$ and  $m_i \leq
    f(v_j)<m_i+\epsilon$. Then $f(u_i)-f(v_i)>M_i-m_i-2\epsilon$ hence: 
        \begin{align*}
            U(f,P)-L(f,P) &= \sum_{i=1}^{n}{(M_i-m_i)(x_i-x_{i-1})} \\
                &<= \sum{(f(u_i)-f(v_i))(x_i-x_{i-1})}+2\epsilon(b-a) \\
                &<\sum{(f(u_i)-f(v_i))(x_i-x_{i-1})}-I(f) \\
                &+I(f)-\sum{(f(u_i)-f(v_i))(x_i-x_{i-1})}+2\epsilon(b-a) \\
                &<(2+2(b-a))\epsilon
        \end{align*}
        Thus, $f$ is Riemann integrable and  $I(f)=\int{f}$.
\end{proof}

\begin{theorem}[Linearity]\label{5.2.2}
    If $f$ and  $g$ are integrable on an interval  $[a,b]$, and  $\alpha \in
    \R$, then $f+g$ and  $\alpha f$ are integrable, and:
        \begin{enumerate}[label=(\arabic*)]
            \item $\int_{a}^{b}{f+g dx}=\int_{a}^{b}{f dx}+\int_{a}^{b}{g dx}$

            \item $\int_{a}^{b}{\alpha f dx}=\alpha\int_{a}^{b}{f dx}$ 
        \end{enumerate}
\end{theorem}
\begin{proof}
    Let $\epsilon>0$ and let $P_{\epsilon}$ be a partition of  $[a,b]$, and for
    any partition  $P$ finer than  $P_{\epsilon}$, an  $t_i \in [x_{i-1},x_]$,
    we have:
        \begin{equation*}
            |\sum{f(t_i)(x_i-x_{i-1})}-\int_{a}^{b}{f dx}|<\epsilon           
        \end{equation*}
    and
        \begin{equation*}
            |\sum{g(t_i)(x_i-x_{i-1})}-\int_{a}^{b}{g dx}|<\epsilon
        \end{equation*}
    Then, adding the inequalities, we have by the triangle inequality:
        \begin{equation*}
            |\sum{(f(t_i)+g(t_i))(x_i-x_{i-1})}-\int_{a}^{b}{f
            dx}-\int_{a}^{b}{g dx}|<2\epsilon
        \end{equation*}
    So $I(f+g)$ exists, anf  $I(f+g)=I(f)+I(g)$.

    The second equality is just part $(1)$ applied $\alpha$ times.
\end{proof}

\begin{theorem}\label{5.2.3}
    If $f$ is integrable on  $[a,b]$, then  $f$ is integrable on each sub
    interval of  $[a,b]$, moreover, if  $a<c<b$, then:
        \begin{equation}
            \int_{a}^{b}{f dx}=\int_{a}^{c}{f dx}+\int_{c}^{b}{f dx}
        \end{equation}
\end{theorem}
\begin{proof}
    Assume that $a<b$, given  $\epsilon>0$, there is a partition  $P$ of
    $[a,b]$ such that  $U(f,P)-L(f,P)<\epsilon$. Then choose  $P'=P \cup
    \{c,d\}$, then let  $P_1=P' \cap [c,d]$ where $c<d$ and  $[c,d] \subseq
    [a,b]$. Then $U(f,P_1)-L(f,P_1) \leq U(f,P')-L(f,P') \leq U(f,P)-L(f,P)$.
    Since $f$ is integrable on  $P$, we have that it is integrable over  $ P_1$,
    that is, it is integrable on $[c,d]$.

    Moreover let  $P$ be a partition of  $[a,b]$, and let  $ P_0=P \cup \{c\}$,
    and let $ P_1=P_0 \cap [a,c]$ and $ P_2=P_0 \cap [c,b]$. Then $ P_0=P_1 \cup
    P_2$. Now we have that $U(f,P) \geq
    U(f,P_0)=U(f,P_0)+U(f,P_1) \geq
    \bar{\int_{a}^{c}}{f}+\bar{\int_{c}^{b}}{f}$, doing the same for lower
    sums, we get
        \begin{equation*}
            \bbar{\int_{a}^{c}}{f}+\bbar{\int_{c}^{b}}{f} \leq
            \int_{a}^{b}{f} \leq
            \bar{\int_{a}^{c}}{f}+\bar{\int_{c}^{b}}{f}
        \end{equation*}
    Which establishes the equality.
\end{proof}

\begin{theorem}[The comparison theorem]\label{5.2.4}
    If $f$ and  $g$ are integrable on an interval  $[a,b]$, and  $f \leq g$ for all  $x \in [a,b]$,
    then: 
        \begin{equation}
            \int_{a}^{b}{f} dx \leq \int_{a}^{b}{g} dx
        \end{equation}
        in particular, if $f$ is bounded between  $m$ and  $M$ for all  $x \in [a,b]$, then we
        have:
        \begin{equation}
            m(b-a) \leq \int_{a}^{b}{f} dx \leq M(b-a)
        \end{equation} 
\end{theorem}
\begin{proof}
    Let $P$ be a partition of  $[a,b]$, since  $f \leq g$, then we know that  $M_i(f) \leq
    M_i(g)$, then it follows that  $U(f,P) \leq U(g,P)$, hence:
        \begin{equation*}    
            \bar{\int}{f} \leq U(g,P)
        \end{equation*}
    taking infimums, we get:
        \begin{equation*}
            \int{f} \leq \int{g}
        \end{equation*}
    Moreover, if $m \leq f \leq M$, we just take intergral of  $m$, $f$ and $M$ to get the desired
    result.
\end{proof}

\begin{theorem}\label{5.2.5}
    If $f$ is integrable on an $[a,b]$, then $|f|$ is integrable on $[a,b]$, and:
        \begin{equation}
            |\int_{a}^{b}{f}dx| \leq \int_{a}^{b}{|f|}dx
        \end{equation}
\end{theorem}
\begin{proof}
    If we know that $|f|$ is integrable, then this reduces to a corollary of theorem \ref{5.2.4},
    since  $-|f| \leq f \leq |f|$ for all  $x \in [a,b]$. That is:
        \begin{equation*}
            -\int{|f|} \leq \int{f} \leq \int{|f|}
        \end{equation*}
        It remains to show that $|f|$ is integrable. Let  $P$ be a partition of  $[a,b]$. 
        We claim that  $M_j(|f|)-m_i(|f|) \leq M_i(f)-m_i(f)$ for all  $1 \leq i \leq n$. Let
        $x,y \in [x_{i-1},x_i]$, Now if $f(x),f(y)>0$, then  $|f(x)|-|f(y)|=f(x)-f(y) \leq
        M_i(f)-m_i(f)$, and we have  $M_i(|f|)-m_i(|f|) \leq M_i(f)-m_i(f)$. 

        The proof is similar for the cases of $f(x(,f(y) \leq 0$, and  $f(x)>0$ and  $f(y)<0$
        (and vice versa). Now given $\epsilon>0$, and a partiton  $P$ of  $[a,b]$ such that
        $U(|f|,P)-L(|f|,P) \leq U(f,p)-L(f,P)<\epsilon$, by definition, thus $|f|$ is integrable
        on  $[a,b]$.
\end{proof}

\begin{theorem}\label{5.2.6}
    If $f$ and  $g$ are integrable on an interval  $[a,b]$, then  $fg$ is also integrable
    on  $[a,b]$.
\end{theorem}
\begin{proof}
    We know that $fg=\frac{(f+g)^2-f^2-g^2}{2}$, we have that $f+g$ is integrable, and we would
    like to show that $f^2$  (consecuently $g^2$) is integrable.

    We have that $M_i(f^2)=(M_i(|f|))^2$, $m_i(f^2)=(m_i(|f|))^2$. So
    $M_i(f^2)-m_i(f^2)=M_i(|f|)^2-m_i_(|f|)^2=(M_i(|f|)-m_i_(|f|))(M_i(|f|)+m_i_(|f|) \leq
    2M(M_i(|f|)-m_i(|f|))$ where $M-\sup{|f}$ for all $x \in [a,b]$. Then taking the upper and
    lower sums on this inequality, we get:
        \begin{equation*}
            U(f^2,P)-L(f^2,P) \leq 2M(U(f,P)-L(f,P))<2M\epsilon.
        \end{equation*}
    Taking $\epsilon$ small enough, then  $f^2$ is integrable. Hence the implication that  $fg$
    is integrable follows.
\end{proof}

\begin{theorem}[The first mean value theorem for Integrals]\label{5.2.7}
    Suppose that $f$ and  $g$ are integrable on an interval  $[a,b]$, with $g \geq 0$ for all  $x
\in [a,b]$. If $m=\inf{f}$ and $M=\sup{f}$ for all $x \in [a,b]$, then there is a number $c
\in [m,M]$such that:
    \begin{equation}
        \int_{a}^{b}{fg(x)} dx=c\int_{a}^{b}{g(x)} dx
    \end{equation} 
in particular, if $f$ is continuous on  $[a,b]$, then there is an $ x_0 \in [a,b]$ satisfying:
    \begin{equation}
        \int_{a}^{b}{fg(x)} dx=f(x_0)\int_{a}^{b}{g(x)} dx
    \end{equation} 
\end{theorem}
\begin{proof}
    We have that $m \leq f \leq M$ for all  $x \in [a,b]$, and we have that $g \geq 0$. So we have
    that  $mg \leq fg \leq Mg$. Notice that all three functions are integrable, then by the
    comparison theorem, we have:
        \begin{equation*}
            m\int{g} \leq \int{fg} \leq M\int{g}
        \end{equation*}
    If $\int_{a}^{b}{g}=0$, then $\int_{a}^{b}{fg}=0$ by the Squeeze theorem, and any $c \in [m,M]$
    is sufficient.

    If $\int{g} \neq 0$, define $c=\frac{\int{fg}}{\int{g}}$, then we have $m\int{g} \leq c\int{g}
\leq M\int{g}$. Then we have (since $\int{g}>0$) that $m \leq c \leq M$.

Now if  $f$ is continuous on  $[a,b]$, then there is an $x_0 \in [a,b]$ for which $ f(x_0)=c$.
\end{proof}
