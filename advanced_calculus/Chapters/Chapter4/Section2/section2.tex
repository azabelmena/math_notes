%----------------------------------------------------------------------------------------
%	SECTION 4.2
%----------------------------------------------------------------------------------------

\section{Differentialbility Theorems.}

\begin{theorem}\label{4.2.1}
    Let $f,g$ be realvalued functions, and let  $\alpha \in \R$. If $f$ and  $g$ are 
    differentiable on  at some point  $a$, then:
         \begin{enumerate}[label=(\arabic*)]
             \item $f+g$ is differentiable at $a$, and $(f+g)'=f'+g'$.

             \item $\alpha f$ is differentiable at $a$, and  $(\alpha f)'=\alpha f'$.

             \item $fg$ is differentiable at $a$, and  $(fg)'=f' g +fg'$

             \item G iven that $g(a) \neq 0$, then $\ \frac{f}{g}$ is differentiable at  $a$,
                 and $( \frac{f}{g})'= \frac{f'g-fg'}{g^2}$.
        \end{enumerate}
\end{theorem}
\begin{proof}
    We only prove the Quotient rule, the others follow easily from definition. We know 
    thi\at if $g(a) \neq 0$, then there is a  $\delta>0$ such that  $g(x) \neq 0$ on the 
    neigheborhood $(a-\delta,a+\delta)$. Then we have that:
        \begin{align*}
            \frac{\frac{f(x)}{g(x)}-\frac{f(a)}{g(a)}}{x-a} &= \frac{f(x)g(a)-f(a)g(x)}
                                                                {g(x)g(a)(x-a)} \\
            &= \frac{f(x)g(a)-f(a)g(a)+f(a)g(a)=f(a)g(x)}{g(x)g(a)(x-a)} \\
            &= \frac{f(x)-f(a)}{g(x)g(a)(x-a)}g(a)-f(a)\frac{g(x)-g(a)}{g(x)g(a)(x-a)} \\
        \end{align*}
    Taking limits, we then get $\frac{f'g-fg'}{g^2}$ as required.
\end{proof}

\begin{theorem}[The Chain Rule]\label{4.2.2}
    Let $f$ and  $g$ be realvalued functions, with  $f$ differentiable at  $a$, 
    and  $g$ differentiable at  $f(a)$. Then  $g \circ f$ is differentiable at  $a$, 
    and  $(g \circ f)'=(g' \circ f)f'$
\end{theorem}
\begin{proof}
    Let $F:I \rightarrow \R$, $G: J \rightarrow \R$ be realvalued functions such that $f(x)=F(x)(x-a)+f(a)$ 
    and  $g(y)=G(y)(y-f(a))+g(f(a))$, then by theorem \ref{4.1.1},  $F(a)=f'(a)$,a nd  $G(f(a))=g'(f(a))$. 
    Observe that $a \in I$, and  $f(a) \in f(I) \subseteq J$. Letting  $y=f(x)$, then we get 
    $g(y)=g(f(x))=G(f(x))(f(x)-f(a)+g(f(a))=G(f(x))F(x)(x-a)+g(f(a))$, so again by theorem 
    \ref{4.1.1}, we get that  $G(f(a))F(a)=g'(f(a))f'(a)$.
\end{proof}

\begin{HW} 
    Exercise $7$, on page  $94$.
\end{HW}
