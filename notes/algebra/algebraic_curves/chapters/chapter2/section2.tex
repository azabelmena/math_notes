\section{Affine Changes of Coordinates}

Let $\phi=(\phi_1, \dots, \phi_m):\A^n(k) \xrightarrow{} \A^m(k)$ be a
polynomial map, and let $F \in k[x_1, \dots, x_n]$. By $F^\phi$, mean $\Phi(F)=F
\circ \phi=F(\phi_1, \dots, \phi_n)$. If $\af$ is an ideal and $V$ is an
algebraic set, then we denote $\af^\phi$ to be the ideal generated by all
$F^\phi$, where $F \in \af$, and $V^\phi=\inv{\phi}(V)=V(\af^\phi)$, where
$\af=I(V)$ in this context.

\begin{example}\label{example_2.3}
  Let $\phi:\A^n(k) \xrightarrow{} \A^m(k)$ be a polynomial map, and $F \in
  k[x_1, \dots, x_n]$ a nonconstant polynomial. Then if $V$ is the hyperplane
  of $F$, then $V^\phi$ is the hyperplane of  $F^\phi$.
\end{example}

\begin{definition}
  We call a polynomial map $\phi:\A^n(k) \xrightarrow{} \A^n(k)$ an
  \textbf{affine change of coordinate} fi $\phi$, is 1--1 and onto, and every
  component polynomial of $\phi$ has total degree $1$.
\end{definition}

\begin{lemma}\label{lemma_2.2.1}
  If $\phi:\A^n(k) \xrightarrow{} \A^n(k)$ is an affine change of coordinates,
  then each component polynomial $\phi_i$ of  $\phi$ has the form:
  \begin{equation}\label{equation_2.2}
    \phi_i(x)=\sum_{j=1}^n{a_{ij}x_j}+b_i
  \end{equation}
  where $a_{ij}, b_i \in k$ for all $1 \leq i,j \leq n$. Moreover, $\phi=\phi''
  \circ \phi'$, where $\phi'$ is a linear map, and $\phi''$ is a translation.
\end{lemma}
\begin{proof}
  Let $\phi=(\phi_1, \dots, \phi_n)$. Then by definition, the total degree of
  each component is $\deg{\phi_i}=1$, so that each monomial term of $\phi_i$
  must consist of a single variable, hence each $\phi_i$ must have the form:
  \begin{equation*}
    \phi_i(x)=\sum_{j=1}^n{a_{ij}x_j}+b_i
  \end{equation*}
  Now, take $\phi_i'(x)=\sum{a_{ij}x_j}$, and let $\phi_i''(x)=x+b_i$. Then each
  component polynomial has the form $\phi_i=\phi_i'' \circ \phi_i'$. Taking
  $\phi'=(\phi_1', \dots, \phi_n')$, and $\phi''=(\phi_1'', \dots, \phi_n'')$,
  we get $\phi=\phi'' \circ \phi'$.
\end{proof}
\begin{corollary}
  If $\phi$ is an affine coordinate transform, then for every $x \in \A^n(k)$:
  \begin{equation}\label{equation_2.3}
    \phi(x)=Ax+B
  \end{equation}
  where $A \in k^{n \times n}$, and $B \in k^{n \times 1}$.
\end{corollary}
\begin{proof}
  Observe that the component polynomials of $\phi$ give a system of linear
  equations over $k$, and that the coeficients of each $\phi_i$ form the entries
  of an $n \times n$ matrix $A$ over $k$, and each constant term of $\phi_i$
  form the  entries of an $n \times 1$ matrix $B$ over $k$. Letting $x=(x_1,
  \dots, x_n)^T$ (the transpose), we get:
  \begin{equation*}
    \phi(x)=
    \begin{pmatrix}
      a_{11}    &   \dots   &   a_{1n}  \\
      \vdots    &   \ddots  &   \vdots  \\
      a_{n1}    &   \dots   &   a_{nn}  \\
    \end{pmatrix}
    \begin{pmatrix}
      x_1 \\  \vdots  \\  x_n \\
    \end{pmatrix}
    +
    \begin{pmatrix}
      b_1 \\  \vdots  \\  b_n \\
    \end{pmatrix}
  \end{equation*}
\end{proof}

\begin{lemma}\label{lemma_2.2.2}
  Let $\phi:\A^n(k) \xrightarrow{} \A^n(k)$ be a polynomial map with the form
  $\phi=\phi'' \circ \phi'$, where $\phi'$ is a linear map, and  $\phi''$ is a
  translation. Then $\phi$ is an affine change of coordinates if, and only if
  $\phi'$ is invertible as a linear transformation.
\end{lemma}
\begin{proof}
  Observe that if $\phi''$ is a translation, then $\phi''$ is invertible. Hence
  it remains to show that $\phi$ is an affine change of coordinate if, and only
  if $\phi'$ is invertible.

  Suppose then that $\phi$ is an affine change of coordinates. Then  $\phi$ is
  1--1 and onto by definition. Since $\phi=\phi'' \circ \phi'$, and $\phi''$ is
  invertible, this makes $\phi'$ invertible. Now, if $\phi'$ is invertible, then
  so is $\phi=\phi'' \circ \phi'$, so that $\phi$ is 1--1 and onto. Now, since
  $\phi'$ is a linear map  (whose monomial terms have degree $1$), and $\phi''$
  is a translation, whose (whose monomial term has degree $1$), then each
  component polynomial of $\phi$ has total degree $1$, which makes $\phi$ an
  affine change of coordinates, by definition.
\end{proof}
\begin{corollary}
  If $\phi(x)=Ax+B$, where $A \in k^{n \times n}$ and $B \in k^{n \times 1}$,
  then $\phi$ is an affine change of coordinates if, and only if $A$ is
  non-singular.
\end{corollary}
\begin{proof}
  We have that $\phi$ is an affine change of coordinates if, and only if $\phi'$
  is invertible as a linear transformation. Since $\phi'(x)=Ax$, then $\phi'$ is
  invertible if, and only if $A$ is non-singular.
\end{proof}
\begin{corollary}
  If $\phi(x)=Ax+B$, then $\phi$ is an affine change of coordinates if, and only
  if $A \in \GL{(k,n)}$, where $\GL{(k,n)}$ is the general linear group of $n
  \times n$ matrices over $k$.
\end{corollary}

\begin{theorem}\label{theorem_2.2.3}
  Let $\phi$ and $\psi$ be affine changes of coordinates on $\A^n(k)$. Then
  $\phi \circ \psi$ and $\inv{\phi}$ are also affine changes of coordinates on
  $\A^n(k)$.
\end{theorem}
\begin{proof}
  Let $\phi(x)=A_1x+B_1$ and $\psi(x)=A_2x+B_2$, where $A_1$ and $A_2$ are
  nonsingular. Then $\phi \circ \psi(x)=(A_1A_2)x+(B_1+A_1B_2)$, and since
  $A_1A_2$ is nonsingular, $\phi \circ \psi$ is an affine change of coordinates.
  Now let  $\phi(x)=Ax+B$, and $y=\phi(x)$. Then
  $\inv{\phi}(y)=\inv{A}y-\inv{A}B$. Since $\inv{A}$ is also nonsingular, this
  makes $\inv{\phi}$ an affine change of coordinates as well.
\end{proof}
\begin{corollary}
  The set of all affine changes of coordinates form a group under function
  composition.
\end{corollary}
\begin{proof}
  Observe that the set of all affine changes of coordinates forms a subset of
  the permutation group on $\A^n(k)$. Moreover, this subset is nonempty, since
  the polynomial map $\i:\A^n(k) \xrightarrow{} \A^n(k)$ defined by
  $\i(x)=Ix+0$, where $I$ is the $n \times n$ identity matrix, and $0$ is the
  $n \times 1$ zero matrix, defines an affine change of coordinates. Then by
  above, we get that this subset forms a subgroup of the permutation group on
  $\A^n(k)$.
\end{proof}

\begin{definition}
  We call an affine variety $V \subseteq \A^n(k)$ a \textbf{linear subvariety}
  if $V=V(F_1, \dots, F_r)$ and $\deg{F_i}=1$ for all $1 \leq i \leq r$.
\end{definition}

\begin{lemma}\label{lemma_2.2.4}
  Let $V \subseteq \A^n(k)$ be a linear subvariety, and $\phi$ an affine change
  of coordinates on $\A^n(k)$. Then $V^\phi$ is also a linear subvariety of
  $\A^n(k)$.
\end{lemma}
\begin{proof}
  Let $V=V(F_1, \dots, F_r)$ where $\deg{F_i}=1$ for all $1 \leq i \leq r$. Let
  $\phi(x)=Ax+B$ an affine coordinate change, with $A \in k^{n \times n}$ and
  $B \in k^{n \times 1}$. Then by theorem \ref{theorem_2.2.3}, $\inv{\phi}$ is
  also an affine coordinate change of the form $\inv{\phi}=\inv{A}y-\inv{A}B$.
  Define then the polynomials $G_1, \dots, G_r$ by
  \begin{equation*}
    G_i=F_i^{\inv{\phi}}
  \end{equation*}
  then $\deg{G_i}=\deg{F_i}=1$, for all $1 \leq i \leq r$, and
  \begin{equation*}
    G_i^{\phi}(P)=(F^{\inv{\phi}})^{\phi}(P)=F_i^{\inv{\phi}\phi}(P)=F_i(P)=0
    \text{ for all } P \in V
  \end{equation*}
  So that $V^\phi=V(G_1, \dots, G_r)=V(F_1^{\inv{\phi}}, \dots,
  F_r^{\inv{\phi}})$.
\end{proof}
\begin{corollary}
  If $V$ is non-empty, then there exists an affine change of coordinates
  $\phi:\A^n(k) \xrightarrow{} \A^n(k)$ such that $V^\phi=V(x_{m+1}, \dots,
  x_n)$ for some $m<n$. Moreover, such an $m$ is independent of the choice of
  $\phi$.
\end{corollary}
\begin{proof}
  Consider $V$ as a subspace of the vector space $\A^n(k)$, and let $\dim{V}=m$.
  Now, observe that $V=V(F_1, \dots, F_r)$ where $\deg{F_i}=1$ for all $1 \leq i
  \leq r$, by definition. Then $\{F_1, \dots, F_r\}$ is a basis for $V$, so that
  $r=m$. Let $\{x_1, \dots, x_m\}$ be linearly independent in $V$, then
  $F_i(x)=0$ for any linear combination $x$ of $\{x_1, \dots, x_m\}$. Now, let
  $\phi:\A^n(k) \xrightarrow{} \A^n(k)$ be a polynomial map of the form
  $\phi(x)=Ax+B$, where $A$ is the  $n \times n$ matrix constructed as:
  \begin{equation*}
    A=
    \begin{pmatrix}
      a_{11}        & \dots   & a_{1n}  \\
      \vdots        & \ddots  & \vdots  \\
      a_{m1}        & \dots   & a_{mn}  \\
      a_{(m+1)1}    & \dots   & a_{(m+1)n}  \\
      \vdots        & \ddots  & \vdots  \\
      a_{n1}        & \dots   & a_{nn}  \\
    \end{pmatrix}
  \end{equation*}
  Where $a_{ij}$ are the components of $x_i$ when  $1 \leq i \leq m$, and
  $a_{ij}$ are the coefficients of $F_{n-i+1}$ when $m+1 \leq i \leq n$, for all
  $1 \leq j \leq n$. Then $A$ consists of linearly independent row vectors, and
  hence $A$ is nonsingular. This makes $\phi$ an affine coordinate change.
  Moreover, for all $1 \leq i \leq m$, we get $F_{n-i+1}(x_i)=0$, so that for
  any $(a_1, \dots, a_n) \in \A^n(k)$
  \begin{equation*}
    \phi:(a_1, \dots, a_m,a_{m+1}, \dots, a_n) \xrightarrow{}
    (0, \dots, 0,a_{m+1}, \dots, a_n)
  \end{equation*}
  so that $V^\phi=V(x_{m+1}, \dots, x_n)$.

  Now, suppose that $m<s$, and that there is an affine coordinate change
  $\phi(x)=Ax+B$ for which $V(x_{m+1}, \dots, x_n)=V(x_{s+1}, \dots, x_n)$. Then
  for every $(0, \dots, 0,a_{m+1}, \dots, a_n)$:
  \begin{equation*}
    \phi:(0, \dots, 0,a_{m+1}, \dots, a_n) \xrightarrow{}
    (0, \dots, 0,a_{s+1}, \dots, a_n)
  \end{equation*}
  so that $\phi:(0, \dots, 0, a_{m+1}-a_{s+1}, 0 \dots, 0) \xrightarrow{} 0$, so
  that $\ker{\phi}=\ker{A} \neq (0)$, so that $A$ is non-singular. This makes
  the set $\{\phi_{m+1}, \dots, \phi_n\}$ linearly dependent. However, since
  $V(x_{s+1}, \dots, x_n)=V(x_{m+1}^\phi, \dots, x_n^\phi)$ generates $\A^n(k)$
  as a vector space, $\{\phi_1, \dots, \phi_n\}$ must be a basis, which is a
  contradiction. This forces $m \leq s$, and hence $m<s \leq s$ so that $m=s$.
\end{proof}
\begin{corollary}
  $V$ is isomorphic as a variety to  $\A^m(k)$.
\end{corollary}

\begin{definition}
  Let $V \subseteq \A^n(k)$ a non-empty affine variety. We define the
  \textbf{dimension} of $V$ to be the integer $m<n$ such that there is an affine
  change of coordinates $\phi$ on $\A^n(k)$ for which $V^\phi=V(x_{m+1}, \dots,
  x_n)$.
\end{definition}

\begin{example}\label{example_2.4}
  Let $P=(a_1, \dots, a_n)$ and $Q=(b_1, \dots, b_n)$ distinct points of
  $\A^n(k)$. We define the \textbf{line} throug $P$ and $Q$ to be the set
  \begin{equation*}
    \bar{PQ}=\{(a_1+t(b_1-a_1), \dots, a_n+t(b_1-a_n)) : t \in k\}
  \end{equation*}
  The following are true:
  \begin{enumerate}
    \item[(1)] If $\phi$ is an affine coordinate change, then
      $\phi(\bar{PQ})=\bar{\phi(P)\phi(Q)}$ is the line through $\phi(P)$ and
      $\phi(Q)$.

    \item[(2)] Lines are linear subvarieties of dimension $1$, and any linear
      subvariety is isomorphic to the line through any two of its points.

    \item[(3)] In $\A^2(k)$, a line is isomorphic to a hyperplane. And if $P,P'
      \in \A^2(k)$ and $L_1$ and $L_2$ are two distinct lines through $P$, and
      $L_1'$ and $L_2'$ are two distinct lines through $P'$, then there is an
      affine coordinate change  $\phi$ on  $\A^2(k)$ for which $\phi(P)=P'$, and
      $\p(L_i)=L_i'$ for $i=1,2$.
  \end{enumerate}
\end{example}

\begin{example}\label{example_2.5}
  Consider $\A^n(\C)=\C^n$ with the usual topology obtained by identifying
  $\C^n$ with  $\R^{2n}$. We call a set $S \subseteq \C^n$
  \textbf{path-connected} if for any two points $P,Q \in S$, there exists a
  continuous mapping $\y:[0,1] \xrightarrow{} S$ for which $\y(0)=P$ and
  $\y(1)=Q$. Then the following are true:
  \begin{enumerate}
    \item[(1)] $\com{\C}{S}$ is path connected for any finite set $S$.

    \item[(2)] If $V \subseteq \C^n$ is an algebraic set, then $\com{\C^n}{V}$
      is path-connected. Indeed, let $\bar{PQ}$ be the line through $P$ and  $Q$
      in $\C^n$. Then observe that $\bar{PQ} \cap V$ is finite, and $\bar{PQ}
      \simeq \C$.
  \end{enumerate}
\end{example}
