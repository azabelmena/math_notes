\section{Field Extensions.}

\begin{definition}
    We define the \textbf{characteristic} of a field $F$ to be the smallest
    positive integer  $p$, such that  $p \cdot 1=0$, where  $1$ is the identity
    of  $F$. We write  $\Char{F}=p$, and if no such $p$ exists, then we write
    $\Char{F}=0$.
\end{definition}

\begin{lemma}\label{1.1.1}
    Let $F$ be a field, then  $\Char{F}$ is either $0$, or a prime integer.
\end{lemma}
\begin{proof}
    Let $\char{F}=p$. If $p=0$, then we are done. Now suppose that  $p=mn$, with
     $m,n \in \Z^+$. Then $p \cdot 1=(mn)1=(n \cdot 1)(m \cdot 1)=mn=0$, which
     makes $m$ and  $n$  $0$ divisors. Since  $F$ is a field, and hence an
     integral domain, this is impossible, and hence  $p$ must be prime.
\end{proof}
\begin{corollary}
    If $\Char{F}=p$, then for all $a \in F$,  $pa=\underbrace{a+\dots+a}_{p
    \text{ times}}$.
\end{corollary}
\begin{proof}
    We have $pa=p(a \cdot 1)=(p \cdot 1)a$.
\end{proof}

\begin{example}\label{}
    \begin{enumerate}
        \item[(1)] Both $\Q$ and  $\R$ have  $\Char=0$. Similarly,
            $\Char{\Z}=0$, even though $\Z$ is just an integral domain.

        \item[(2)] $\Char{\faktor{\Z}{p\Z}}=p$ and
            $\Char{\faktor{\Z}{p\Z}[x]}=p$ for any prime $p$.
    \end{enumerate}
\end{example}

\begin{definition}
    We define the \textbf{prime subfield} of a field $F$ to be the subfield of
    $F$ generated by  $1$.
\end{definition}

\begin{example}\label{}
    \begin{enumerate}
        \item[(1)] The prime subfields of $\Q$ and  $\R$ is  $\Q$.

        \item[(2)] Let $\faktor{\Z}{p\Z}(x)$ the field of rational functions
            over $\faktor{\Z}{p\Z}$. Then the prime subfield of
            $\faktor{\Z}{p\Z(x)}$ is $\faktor{\Z}{p\Z}$. Similarly, the prime
            subfield for $\faktor{\Z}{p\Z}[x]$ is also $\faktor{\Z}{p\Z}$.
    \end{enumerate}
\end{example}

\begin{definition}
    If $K$ is a field containing a field $F$, then we call  $K$  \textbf{field
    extension} over $F$, and write  $\faktor{K}{F}$ (not the quotient field!) or
    denote it by the diagram
    \[\begin{tikzcd}
        K \\
        F
        \arrow[no head, from=1-1, to=2-1]
    \end{tikzcd}\]
\end{definition}

\begin{lemma}\label{1.1.2}
    Every field is a field extension of its prime subfield.
\end{lemma}

\begin{lemma}\label{1.1.3}
    Let $K$ an extension over a field  $F$. Then  $K$ is a vector space over
    $F$.
\end{lemma}

\begin{definition}
    Let $\faktor{K}{F}$ a field extension. We define the \textbf{degree} of $K$
    over $F$, $[K:F]$ to be the dimension of  $\faktor{K}{F}$ as a vector space.
\end{definition}

\begin{definition}
    Let $F$ be a field, and  $f \in F[x]$ a polynomial. We call am element $\a
    \in R$ a  \textbf{root} (or \textbf{zero}) of $f$ if  $f(\a)=0$.
\end{definition}

\begin{lemma}\label{1.1.4}
    Let $\phi:F \xrightarrow{} L$ a field homomorphism. Then either $\phi=0$, or
     $\phi$ is 1--1.
\end{lemma}

\begin{lemma}\label{1.1.5}
    Let $F$ be a field, and  $p \in F[x]$ an irreducible polynomial. Then there
    exists a field $K$ containing an embedding of  $F$, such that  $p$ has a
    root in  $K$.
\end{lemma}
\begin{proof}
    Consider $K=\faktor{F[x]}{(p)}$. Siince $p$ is irreducible in a principle
    ideal domain, $(p)$ is a maximal idea, and hence $K$ is a field. Now
    consider the canonical map  $\pi:F[x] \xrightarrow{} K$ taking $f
    \xrightarrow{} f \mod{(p)}$ and let $\phi=\pi|_{F}$. Then $\phi \neq 0$,
    since  $\pi:1 \xrightarrow{} 1$. Then $\phi$ is 1--1. And so  $\phi(F)
    \simeq F$.

    Now, consider $F$ as a subfield of  $K$. Then $p(x \mod{(p)}) \equiv
    p(x) \mod{(p)} \equiv 0 \mod{(p)}$, so that $x \mod{(p)}$ is a root of
    $p$ in  $K$.
\end{proof}
\begin{corollary}
    There exists a field extension of $F$ containing a root of  $p$.
\end{corollary}

\begin{theorem}\label{1.1.6}
    Let $F$ be a field, and let  $p \in F[x]$ an irreducible polynomial of
    degree $n$, and let  $K=\faktor{F[x]}{(p)}$, and $\th=x \mod{(p)}$. Then
    $\{1, \th, \dots, \th^{n-1}\}$ forms a basis for $K$ as a vector space over
     $F$ and $[K:F]=n$.
\end{theorem}
\begin{proof}
    Let $a \in F[x]$, since $F[x]$ is Euclidean domain, there exist $q,r \in
    F[x]$, $q \neq 0$ for which
    \begin{equation*}
        a(x)=q(x)p(x)+r(x) \text{ where } \deg{r}<n
    \end{equation*}
    Now, since $pq \in (p)$, $a(x) \equiv r(x) \mod{(p)}$, and every element of
    $K$ is a polynomial of degree less than $n$. Then the elements $\{1, \th,
    \dots, \th^{n-1}\}$ span $K$.

    Now, suppose that there are  $b_0, \dots, b_{n-1} \in F$ not all $0$ for
    which
    \begin{equation*}
        b_0+b_1\th+\dots+b_{n-1}\th^{n-1}=0
    \end{equation*}
    Then
    \begin{equation*}
        b_0+b_1\th+\dots+b_{n-1}\th^{n-1} \equiv 0 \mod{(p)}
    \end{equation*}
    so that $p|(b_0+b_1\th+\dots+b_{n-1}\th^{n-1})$ in $F$. But  $\deg{p}=n$ and
    $p$ divides a polynomial of degree  $n-1$, which is a contradiction.
    Therefore we are left with  $b_0=\dots=b_{n-1}=0$.
\end{proof}
\begin{corollary}
    $K=\{\a_0+a_1\th+\dots+a_{n-1}\th^{n-1} : a_i \in F \text{ for all } 1 \leq
    i \leq n-1\}$
\end{corollary}
\begin{corollary}
    If $a(\th),b(\th) \in K$, are elements of degree less than $n$, and the
    operations of polynomial addition, and polynomial multiplication mod
    $(p)$ are defined, then $K$ forms a field.
\end{corollary}

\begin{example}\label{example_1.3}
    \begin{enumerate}
        \item[(1)] Consider the polynomial $x^2+1$ over $\R$. Then one has the
            field
            \begin{equation*}
                \C=\faktor{\R[x]}{(x^2+1)}
            \end{equation*}
            an extension of $\R$ of degree $[\C : \R]=2$. Let $i$ be a root of
            $x^2+1$ in this field, then $i^2=-1$, and the elements of $\C$ are of
            the form $a+ib$ where  $a,b \in \R$. Then we have described the
            field of complex numbers, and the addition and multiplication
            ($\mod{x^2+1}$) of these elements are the addition and
            multiplication of complex numbers.

            One might also construct $\C$ differently by defining the
            isomorphism
            \begin{equation*}
                \faktor{\R[x]}{(x^2+1)} \xrightarrow{} \C \text{ taking }
                a+xb    \xrightarrow{} a+ib
            \end{equation*}

        \item[(2)] Consider again $x^2+1$ over  $\Q$. Then we get the field
            \begin{equation*}
                \Q(i)=\faktor{\Q[x]}{(x^2+1)}
            \end{equation*}
            of degree $[\Q(i):\Q]=2$, and where $i$ is a root of  $x^2+1$, so
            that  $i^2=-1$. Then the elements of  $\Q(i)$ are of the form $a+ib$
            where  $a,b \in \Q$, i.e. it is isomorphic to the set of all complex
            numbers with rational components.

        \item[(2)] Consider $x^2-2$ over  $\Q$. by Eisenstein's criterion for
            $p=2$,  $x^2-2$ is irreducible over  $\Q$. Let $\a$ a root of
            $x^2-2$, so that  $\a^2=2$. Then we have the field
            \begin{equation*}
                \Q(\sqrt{2})=\faktor{\Q[x]}{(x^2-2)}
            \end{equation*}
            of degree $[Q(\sqrt{2}) : \Q]=2$, and whose elements are of the form
            $a+b\sqrt{2}$. One can define an isomorphism between $\Q(i)$ and
            $\Q(\sqrt{2})$ by taking $\sqrt{2} \xrightarrow{} i$.

        \item[(3)] The polynomial $x^3-2$ over  $\Q$ gives us the field
            \begin{equation*}
                \Q(\sqrt[3]{2})=\faktor{\Q[x]}{(x^3-2)}
            \end{equation*}
            of degree $[\Q(\sqrt[3]{2}) : \Q]=3$ over $2$. Here the elements are
            of the form $a+b\xi+c\xi^2$ where $\xi^3=2$.

        \item[(4)] Denote $\F_2$ to be a finite field of $2$ elements. Consider
            the polynomial  $x^2+x+1$ over  $\F_2$ which is irreducible. Then
            the field
            \begin{equation*}
                \F_2(\a)=\faktor{\F_2[x]}{(x^2+x+1)}
            \end{equation*}
            is a field of degree $2$ over  $\F_2$, whose elements are of the
            form $a+b\a$, where  $\a^2=\a+1$. In fact, one can generate this
            field using the fact that  $\a^2=\a+1$.

        \item[(5)] Let $F=K(t)$ the field of rational functions in $t$ over a field
            $K$. Let $p(x)=x^2-t \in F[x]$, then by Eisenstien's criterion with
            the ideal $(t)$, $p$ is irreducible over  $F[x]$. Let $\th$ be a
            root for  $p$, that is  $\th=\sqrt{t}$, then we get the field
            $K(t,\sqrt{t})$ of degree $[K(t,\sqrt{t}) : K]=2$, whose elements are
            of the form $a(t)+b(t)\sqrt{t}$.
    \end{enumerate}
\end{example}

\begin{lemma}\label{1.1.7}
    Let $F$ be a subfield of a field $K$, and let  $\a \in K$. Then there exists
    a unique minimal subfield of  $K$ containing  $F$ and  $\a$; more preciesly,
    it is the intersection of all subfields of  $K$ containing $F$ and $\a$.
\end{lemma}

\begin{definition}
    Let $K$ be any extension of a field $F$, and let  $\a, \b, \dots \in K$.
    Then we define the subfield \textbf{generated} by $\a, \b, \dots$
    over  $F$ to be the unique minimal subfield containing all $\a,\b, \dots$
    and $F$ and we denote it  $F(\a, \b, \dots)$. Moreover, we call $K$ a
    \textbf{simple extension} of $F$ if  $K=F(\a, \b, \dots)$. If $K=(F\a_1,
    \dots, a_n)$ for $\a_1, \dots, a_n \in K$, then it is a \textbf{finitely
    generated} simple extension.
\end{definition}

\begin{theorem}\label{}
    Let $F$ be a field, and  $p \in F[x]$ irreducible, and let $K$ an extension
    of  $F$ containing a root $\a$ of $p$. Then
    \begin{equation*}
        F(\a) \simeq \faktor{F[x]}{(p)}
    \end{equation*}
\end{theorem}
\begin{proof}
    Consider the homomorphism $F[x] \xrightarrow{} F(\a)$ taking $a(x) \xrightarrow{}
    a(\a)$. Since  $p(\a)=0$, $p$ is in the kernel of this homomorphism, and we
    get an induced homomorphsim from $\faktor{F[x]}{(p)} \xrightarrow{} F(\a)$.
    Now, since $p$ is irreducible,  $\faktor{F[x]}{(p)}$ is a field, and since
    the homomorphsim takes $1 \xrightarrow{} 1$, it is 1--1. Then by the first
    isomorphsim theorem for ring homomorphsims these two fields are isomorphic.
\end{proof}
\begin{corollary}
    If $\deg{p}=n$, then $F(\a)=\{a_0+a_1\a+\dots a_{n-1}\a^{n-1} : a_i \in F
    \text{ for all } 1 \leq i \leq n-1\}$ and $[F(\a) : F]=n$.
\end{corollary}

\begin{example}\label{example_1.4}
    \begin{enumerate}
        \item[(1)] The polynomial $x^2-2$ over  $\Q$ also has the root
            $-\sqrt{2}$ in $\R$, so that $\Q(-\sqrt{2})$ is of degree $2$ over
            $\Q$ with elements of the form  $a-b\sqrt{2}$. Notice however that
            $\Q(-\sqrt{2}) \simeq \Q(\sqrt{2})$ by taking $a-b\sqrt{2}
            \xrightarrow{} a+b\sqrt{2}$.

        \item[(2)] The polynomial $x^3-2$ only has the solution
            $\xi=\sqrt[3]{2}$ in $\R$. However, in  $\Q$ it has the solutions
            given by
            \begin{equation*}
                \sqrt[3]{2}(\frac{-1 \pm i\sqrt{3}}{2})
            \end{equation*}
            So that the subfields generated by either of these three elements
            (over $\C$) are isomorphic.
    \end{enumerate}
\end{example}

\begin{theorem}\label{1.1.9}
    Let $\phi:F \xrightarrow{} L$ a field isomorphism and $p \in F[x]$, $q \in
    L[x]$ irreducible polynomials, where $q$ is obtained by applying  $\phi$ to
    the coefficients of $p$. Let  $\a$ a root of  $p$, and  $\b$ a root of $q$.
    Then there exists an isomorphism $F(\a) \xrightarrow{} L(\b)$ taking $\a
    \xrightarrow{} \b$ and extending $\phi$. That is, we have the following
    diagram
    \[\begin{tikzcd}
        {F(\alpha)} & {L(\beta)} \\
        F & E
        \arrow[no head, from=1-1, to=2-1]
        \arrow[no head, from=1-2, to=2-2]
        \arrow[from=1-1, to=1-2]
        \arrow["\phi"', from=2-1, to=2-2]
    \end{tikzcd}\]
\end{theorem}
\begin{proof}
    Notice that $\phi$ induces a ring homomorphism between  $F[x]$ and $L[x]$,
    so that $(p)$ is maximal. Since $q$ is obtained from $p$,  $(q)$ is also
    maximal, so that $\faktor{F[x]}{(p)}$ and $\faktor{L[x]}{(q)}$ are fields.
    Then we have an isomorphsism
    \begin{equation*}
        \faktor{F[x]}{(p)} \simeq \faktor{L[x]}{(q)}
    \end{equation*}
    Then, if $\a$ is a root of $p$, and $\b$ a root of $q$, we obtain the
    isomorphism
    \begin{equation*}
        F(\a) \simeq L(\b)
    \end{equation*}
    moreover, this isomorphism takes $\a \xrightarrow{} \b$.
\end{proof}
