\section{Cayley's Theorem.}

Consider a group $G$ acting on itself via left multiplication; i.e. the action
$G \times G \xrightarrow{} G$ defined by $g \cdot a \xrightarrow{} ga$. if
$\ord{G}=n$, then $G=\{g_0, \dots, g_{n-1}\}$, where $g_0=e$. Conisder then the
permutation $s_g:i \xrightarrow{} j$ if, and only if $gg_i=j$. Moreover, we can
consider $G$ a group, and  $H \leq G$, such that $G$ acts on the factor set
$\faktor{G}{H}$ via the action $g \cdot aH \xrightarrow{} (ga)H$. Then we can
take the permutation above as $s_g:i \xrightarrow{} j$ if, and only if $g \cdot
a_iH=a_jH$.

\begin{example}\label{}
    \begin{enumerate}
        \item[(1)] Consider the Klein $4$-group acting on its self. Let
            $V_4=\{1, a,b,c\}$ and label $a=2,b=3$, and  $c=4$. Define the
            permutation $s_a:i \xrightarrow{} j$. Then $s_a(1)=2$, $s_a(2)=1$,
            $s_a(3)=4$ and $s_a(4)=3$, so we get $s_a=(1 \ 2)(3 \ 4)$. By
            similar computation, we get $s_b=(1 \ 3)(2 \ 4)$ and $s_c=(1 \ 4)(2
            \ 3)$.

        \item[(2)] Consider $D_8$ the dihedral group acting on the factor set
            $\faktor{D_8}{\langle t \rangle}=\{\langle t \rangle, r \langle t
            \rangle, r^2 \langle t \rangle, r^3 \langle t \rangle\} \simeq
            \{1,2,3,4\}$. Taking the permutation $s_t:i \xrightarrow{} j$ if,
            and only if $ti=j\langle t \rangle$, we get $s_t(1)=1$, $s_t(2)=4$,
            $s_t(3)=3$ and $s_t(4)=2$. So $s_t=(2 \ 4)$. Similarly.
    \end{enumerate}
\end{example}

\begin{theorem}\label{4.2.1}
    Let $G$ be a group, and  $H$ a subgroup on $H$. Conisder $G$ acting on
    $\faktor{G}{H}$ via $g \cdot aH \xrightarrow{} (ga)H$, and let $\pi_H$ be
    the permutation representation of this action. Then the following are true:
    \begin{enumerate}
        \item[(1)] $G$ acts faithfully on  $\faktor{G}{H}$.

        \item[(2)] $\stab{H}=H$.

        \item[(3)] $\ker{\pi_H}=\bigcap_{x \in G}{xH\inv{x}}$, and $\ker{\pi_H}$
            is the largest normal subgroup of $G$ contained in $H$.
    \end{enumerate}
\end{theorem}
\begin{proof}
    Let $aH,bH \in \faktor{G}{H}$, and $g=b\inv{a}$. Then $g \cdot
    aH=(ga)H=(b\inv{a}a)H=bH$ so that $aH$ and  $bH$ are in the same orbit.
    Since  $a,b \in G$ are arbitrary, we get there is only one such orbit. So
    $G$ acts faithfully on  $\faktor{G}{H}$.

    Now, notice that $\stab{H}=\{g \in G : g \cdot 1H=1H\}$. So, if $g \in
    \stab{H}$, then $gH=H$, which makes  $g \in H$. On the other hand, if  $g
    \in H$, then  $gH=(g1)H=g \cdot 1H=1H$, which puts $g \in \stab{H}$.
    Therefore $\stab{H}=H$.

    Lastly, notice that $\ker{\pi_H}=\{g \in G : gxH=xH\}$. Now, if $g \in
    \ker{\pi_H}$, then $gxH=xH$ so that  $\inv{x}gxH=H$, so $\inv{x}gx \in H$ so
    that $g \in xH\inv{x}$. Likewise, if $g \in xH\inv{x}$, then $\inv{x}gx \in
    H$ which puts $\inv{x}gx=H$. Therefore $gxH=xH$ which puts $g \in
    \ker{\pi_H}$, for all $x \in G$. That is, $\ker{\pi_H}=\bigcap{xH\inv{x}}$.

    Now, since $\ker{\pi_H} \unlhd G$, if $N \unlhd G$, then  $N=xN\inv{x} \leq
    xH\inv{x}$ for all $x \in G$. This puts $N \leq
    \bigcap{xH\inv{x}}=\kerp{\pi_H}$. This makes $\ker{\pi_H}$ the largest
    normal subgroup of $G$, contained in  $H$.
\end{proof}
\begin{corollary}[Cayley's Theorem]
    Every group is isomorphic to a subgroup of the symmetric group.
\end{corollary}
\begin{proof}
    Let $H=\langle e \rangle$, then $\faktor{G}{\langle e \rangle}=G$ acts on
    intself by the left regular action. Let $\pi_{\langle e \rangle}=\pi:G
    \xrightarrow{} A(G)$ be the permutation representation of this action. By
    above, we get that $\ker{\pi}=\langle e \rangle$, which makes $\pi$ 1--1.
    Moreover, since $\pi(G) \leq A(G)$, we get that $\pi:G \xrightarrow{}
    \pi(G)$ defines an isomorphism of $G$ onto a subgroup of $A(G)$.
\end{proof}
\begin{corollary}
    if $G$ is as group of order $n$, then $G$ is isomorphic to a subgroup of
    $S_n$.
\end{corollary}

\begin{example}\label{}
    \begin{enumerate}
        \item[(1)] $V_4 \simeq \langle (1 \ 2)(3 \ 4),(1 \ 3)(2 \ 4) \rangle
            \leq S_4$.

        \item[(2)] $D_8 \simeq \langle (2 \ 4), (1 \ 2 \ 3 \ 4) \rangle \leq
            S_8$.
    \end{enumerate}
\end{example}

\begin{lemma}\label{4.2.2}
    If $G$ is a finite group of order $n$, and $p$ is the smallest prime
    dividing  $n$, then every subgroup of index $p$, in $G$ is normal.
\end{lemma}
\begin{proof}
    Let $H \leq G$ with  $[G:H]=p$. Let $\pi_H$ the permutation representation
    of the action  $g \cdot aH \xrightarrow{} (ga)H$ and let $K=\ker{\pi_H}$,
    and $[H:K]=k$. Then $[G:K][G:H][H:K]=pk$. Now, since $H$ has  $p$ left
    cosets in  $G$, by the first isomorphism theorem and Cayley's theorem, we get
    $\faktor{G}{K}$ is isomorphic to a subgroup of $S_p$. Now, by Lagrange's
    theorem,  $pk|p!$, i.e.  $k|\frac{p!}{p}=(p-1)!$. However, all prime
    divisors of $(p-1)!$ are less than $p$, so by minimality, every prime
    divisor  $q$ of  $k$ is such that  $p \leq q$. Therefore $k=1$ and $H=K
    \unlhd G$.
\end{proof}
