\section{Group Actions.}
\label{section1}

We now present the notion of a group ``acting'' on a given set. The study of
these ``actions'' will allow us to prove results for groups and also for finding
underlying structures of specific sets.

\begin{definition}
    A \textbf{left} group \textbf{action} of a group $G$ on a set  $A$ is a map
    $\cdot: G \times A
    \rightarrow A$ such that for all $g \in G$ and  $a \in A$:
    \begin{enumerate}
        \item[(1)] $g_1 \cdot (g_2 \cdot a)=(g_1g_2) \cdot a$.

        \item[(2)] $e \cdot a=a$, where  $e$ is the identity of  $G$.
    \end{enumerate}
\end{definition}
\begin{remark}
    As before, we will drop explicit mention of the action $\cdot$ and mearly
    werite $ga$. It should be taken into account that the group action $\cdot$
    is not a binary operation.
\end{remark}
\begin{remark}
    Similarly, one can define \textbf{right} group actions. Our study will
    consist of left group actions, so we drop the indication.
\end{remark}

\begin{lemma}\label{1.7.1}
    Let $G$ be a group acting on a set $S$. For each $g \in G$, define the map
    $\sigma_g:S \rightarrow S$ by $\sigma_g:a \rightarrow ga$. Then:
    \begin{enumerate}
        \item[(1)] For each $g \in G$,  $\sigma_g$ is a permutation of  $S$.

        \item [(2)] The map $G \rightarrow A(S)$ defined by $g \rightarrow
            \sigma_g$ is a homomorphism.
    \end{enumerate}
\end{lemma}
\begin{proof}
    Let $a,b \in S$ and suppose that $\sigma_g(a)=\sigma_g(b)$. Then $ga=gb$,
    and by the cancellation laws,  $a=b$. This makes $\sigma_g$  $1-1$. On the
    otherhand, since $ga \in A$ for all $a \in A$, we get that $\sigma_g$ is
    onto. This makes  $\sigma_g$ a permutation. Now, consider  $\inv{g} \in G$,
    and notice that
    $\sigma_{\inv{g}}\sigma_g(a)=\simga_{\inv{g}}(ga)=\inv{g}(ga)=(\ing{g}g)a=ea=a$.
    Thus we get $\inv{\sigma_g}=\sigma_{\inv{g}}$.

    Now consider the map $\phi:g \rightarrow \sigma_g$. Then
    $\phi(gg')(a)=\sigma_{gg'}(a)=(gg')a=g(g'a)=\sigma_g\sigma_{g'}(a)=
    \phi(g)\phi(g')$.
\end{proof}
\begin{remark}
    The main takeaway of this lemma is that group actions on a set $S$ are
    merely permutations of the elements of  $S$.
\end{remark}

\begin{definition}
    Let $G$ be a group acting on a set  $S$, and define $\sigma_g:a \rightarrow
    ga$, and define $\phi:G \rightarrow A(S)$ by $\phi:g \rightarrow \sigma_a$.
    We call $\phi$ the  \textbf{permutation representation} of $S$ associated
    with  $g$.
\end{definition}

\begin{example}
    Let $G$ be a group, and  $A \neq \emptyset$. Then:

    \begin{enumerate}
        \item[(1)] Define the action $ga=a$ for all  $g \in G$. Then
            $g_1(g_2)a=g_1a=a$ and $(g_1g_2)a=a$, so $g_1(g_2)a=(g_1g_2)a$, and
            $ea=a$, so we indeed have an action. We call this the
            \textbf{trivial} action, and we say that $G$ acts
            \textbf{trivially} on $A$. Define then  $\sigma_g:a \rightarrow
            ga=a$, then $\sigma_g$ is the identity map. So the permutation
            representation associated with  $g$ is the identity map.

        \item[(2)] In the vector space axioms, scalar multiplication $\cdot:F^*
            \times V \rightarrow V$ is an action of $F^*$ on  $V$. We have for any
             $\alpha,\beta \in F$, and  $v \in V$,
             $\alpha(\beta)v=(\alpha\beta)v$, and $1v=v$. Here $F$ is a field,
             and so  $F^*$ forms a multiplicative group under the relavent
             multiplication.

         \item [(3)] For any $S \neq \emptyset$, the symmetric group  $A(S)$
             acts on $S$ via the action  $sa=s(a)$.

         \item[(4)] Consider again the regular $n$-gon. Label its veritces to be
             the set $\faktor{\Z}{n\Z}$, then we can see that the symmetries of
             the $n$-gon act on vertices of the $n$-gon.

             Consider the map $D_{2n} \times \faktor{\Z}{n\Z}\rightarrow
             \faktor{\Z}{n\Z}$ of $D_{2n} \times \faktor{\Z}{n\Z}$ onto
             $\faktor{\Z}{n\Z}$ via the map $(r^jt,i) \rightarrow r^jt(i)$,
             where $j \in \faktor{\Z}{n\Z}$. This map forms a group action of
             $D_{2n}$ on $\faktor{\Z}{n\Z}$. Also notice that distinct
             symmetries induce distinct permutations of the vertices.

         \item[(5)] Let $G$ be any group, and let  $A=G$. Then the binary
             operation on $G$ is a group action of  $G$ onto itself. We have
             $a(bc)=(ab)c$, and so the first prppertiy is satisfied by
             associativity, and $ea=a$ and the second property is satisfied by
             the identity law. We call the binary operation a \textbf{left
             regular} action. Also notice that distinct elements of $G$ induce
             distinct permutations of  $G$.
    \end{enumerate}
\end{example}

We end the section, and the chapter with two more definitions.

\begin{definition}
    Let $G$ be a group acting on a set $A$. We call the action of $G$ on  $A$
     \textbf{faithful} if distinct elements of $G$ induce distinct permutations
     on  $G$. That is if $\phi$ is the permutation representation associated
     with  $G$, then  $\phi$ is  $1-1$.
\end{definition}

\begin{definition}
    Let $G$ be a group acting on a set $A$. We define the \textbf{kernel} of the
    group action on $A$ to be  $\ker{A}=\{g \in G : ga=a \text{ for all } a \in
    A\}$
\end{definition}
