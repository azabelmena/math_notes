\section{Homomorphism.}
\label{section1}

In this section, we relate the structures of groups to each other. The main
reason to do this is to determine which groups are ``equal'', i.e. when two
distinct groups share the same group structure. Doing this will often allow us
to infer properties of one group from the other.

\begin{definition}
    Let $(G, \ast)$ and $(H, \cdot)$ be groups. We call a map $\phi:G
    \rightarrow H$ a group \textbf{homomorphism} if for any $a,b \in G$,
    $\phi(a \ast b)=\phi(a) \cdot \phi(b)$. We call the homomorphism $\phi$ a
    group  \textbf{isomorphism} if $\phi$ is both  $1-1$ and onto. If such an
    isomorphism exists between  $G$ and  $H$, we call  $G$ and  $H$
    \textbf{isomorphic} and write $G \simeq H$.
\end{definition}
\begin{remark}
    Frequently, we will imply the operations on $G$ and  $H$ and write
    $\phi(ab)=\phi(a)\phi(b)$.
\end{remark}

\begin{lemma}\label{1.6.1}
    Isomorphism of groups is an equivalence relation.
\end{lemma}
\begin{proof}
    Let $G$ and  $H$ be groups. First, take $\phi:G \rightarrow G$ by $phi:g
    \rightarrow g$, then $\phi$ is an isomorphism, so $G \simeq G$.

    Now suppose that  $G \simeq H$, then there is an isomorphism  $\phi:G
    \rightarrow H$. Then consider $\inv{\phi}:H \rightarrow G$, we have
    $\inv{\phi}$ is also $1-1$ and onto; moreover
    $\inv{\phi}(\phi(ab))=ab=\inv{\phi}(\phi(a))\inv{\phi}(\phi(b))$. This makes
    $\inv{phi}$ an isomorphism and so $H \simeq G$.

    Lastly, let  $K$ be a group and suppose  $G \simeq H$ and  $H \simeq K$.
    Then there are isomorphisms  $\phi:G \rightarrow H$ and $\psi:H \rightarrow
    K$. Then take $\psi \circ \phi:G \rightarrow K$ which is $1-1$ and onto by
    definition. Then $\psi \circ
    \phi(ab)=\psi(\phi(a)\phi(b))=\psi(\phi(a))\psi(\phi(b))$. Thus $G \simeq
    K$.
\end{proof}

\begin{example}
    \begin{enumerate}
        \item[(1)] The maps $\exp: (\R,+) \rightarrow (\R^+, \cdot)$ and
            $\log: (\R^+, \cdot) \rightarrow (\R,+)$ defined by $\exp:x
            \rightarrow e^x$ and $\log:y \rightarrow \log{y}$. Then $\exp$ and
            $\log$ are homomorphisms. We have $\exp{x+y}=\exp{x}\exp{y}$ and
            $\log{xy}=\log{x}+\log{y}$. Moreover, $\exp$ and  $\log$ are group
            isomorphisms, infact, $\log=\inv{\exp}$.

        \item[(2)] Let $S$ and  $T$ be nonempty finite sets. Then  $A(S) \simeq
            A(T)$ if and only if $|S|=|T|$, i.e. the symmetric groups of  $S$
            and  $T$ are isomorphic if, and only if  $S$ and  $T$ share the same
            cardinality.

            Suppose $S$ and  $T$ are finite, and that  $|S|=|T|=n$. Define  the
            map  $\phi:A(S) \rightarrow A(T)$ by $\phi:s \rightarrow ts\inv{t}$,
            there $t:S \rightarrow T$ is a bujection. Then $\phi$ is $1-1$, for
             $ts\inv{t}=ts'\inv{t}$ implies $s=s'$. Moreover, we have
             $\phi(A(S))=A(T)$, since for any $s \in A(S)$, $ts\inv{t}:T
             \rightarrow T$ defines a bijection from $T$ onto itself; hence
             $ts\inv{t} \in A(T)$. Therefore, we get $\phi$ is an isomorphism
             form  $A(S)$ to $A(T)$. This makes $A(S) \simeq A(T)$.

            On the other hand, if  $A(S) \simeq A(T)$, then $\ord{A(S)}=
            \ord{A(T)}=n!$, for some $n \in \Z^+$, this implies that
            $|S|=|T|=n!$.
    \end{enumerate}
\end{example}

\begin{lemma}\label{1.6.2}
    Let $G$ and  $H$ be groups, and let  $\phi:G \rightarrow H$ be a
    homomorphism. Then the following are true:
    \begin{enumerate}
        \item[(1)] $\phi(e)=e'$ where $e$ and $e'$ are the identites of  $G$ and
             $H$, respectively.

         \item[(2)] $\phi(\inv{a})=\inv{\phi(a)}$.
    \end{enumerate}
\end{lemma}
\begin{proof}
    We have that $\phi(e)=\phi(a\inv{a})=\phi(a)\phi(\inv{a})$. Thus by
    cancellation, we get $\inv{\phi(a)}=\phi(\inv{a})$. By consequence, we also
    get that $\phi(e)=\phi(a)\inv{\phi(a)}=e'$; for any $a \in G$.
\end{proof}
\begin{corollary}
    The following are also true for any $n \in \Z^+$:
    \begin{enumerate}
        \item[(1)]$\phi(a^n)=\phi(a)^n$.

        \item[(2)] $\phi(a^{-n})=\phi(a)^{-n}$.
    \end{enumerate}
\end{corollary}
\begin{proof}
    Firstly, $\phi(a^n)=\underbrace{\phi(a) \dots \phi(a)}_{n \text{
    times}}=\phi(a)^n$. Then we get
    $\phi(a^{-n})=\phi((\inv{a})^n)=\phi(\inv{a})^n=\phi(a)^{-n}$.
\end{proof}

\begin{lemma}\label{1.6.3}
    Let $G$ and  $H$ be groups and suppose  $G \simeq H$. Then the following are
    true:
    \begin{enumerate}
        \item[(1)] $\ord{G}=\ord{H}$.

        \item[(2)] $G$ is Abelian if, and only if  $H$ is abelian.

        \item [(3)] For all $x \in G$,  $\ord{x}=\ord{\phi(x)}$, where $\phi:G
            \rightarrow H$ is the underlying isomorphism.
    \end{enumerate}
\end{lemma}
\begin{proof}
    Let $G \simeq H$,  via the isomorphism  $\phi$. Then since $\phi$ is  $1-1$
    and onto, every element of  $G$ must get mapped to every element of  $H$.
    This makes  $\ord{G}=\ord{H}$.

    Now, suppose $G$ is Abelian, then for every  $a, b \in G$,  $ab=ba$. Thus
    $\phi(ab)=\phi(ba)$. This makes $\phi(a)\phi(b)=\phi(b)\phi(a)$, since $a,b$
    are arbitrary, this makes  $H$ Abelian. The converse is an equivalent
    argument with  $\inv{\phi}$.

    Now suppose that $x \in G$ has order  $\ord{x}=n$. Then $x^n=e$. Thus
    $\phi(x^n)=\phi(e)$, thus $\phi(x)^n=e'$. Now since $n$ is minimal, any
    $m<n$ for which  $\phi(x)^m=e'$ would imply that $x^m=e$, which cannot
    happen. Thus  $\ord{\phi(x)}=n$.
\end{proof}
\begin{corollary}
    Let $\phi:G \rightarrow H$ be a homomorphism. If $H$ is Abelian, and  $\phi$
    is  $1-1$, then  $G$ is Abelian. On the otherhand, if  $G$ is Abelian, and
    $\phi$ is onto, then  $H$ is Abelian.
\end{corollary}
\begin{proof}
    For $a,b \in G$,  $\phi(a)\phi(b)=\phi(b)\phi(a)$, hence
    $\phi(ab)=\phi(ba)$. Since $\phi$ is  $1-1$, this implies  $ab=ba$.

    On the otherhand, if $ab=ba$ for any  $a,b \in G$ and  $\phi$ is onto, then
    we get  $\phi(ab)=\phi(ba)$ hence $\phi(a)\phi(b)=\phi(b)\phi(a)$. Since
    $\phi(G)=H$, this completes the proof.
\end{proof}

\begin{example}
    \begin{enumerate}
        \item[(1)] We have $S_3 \not\simeq \faktor{\Z}{6\Z}$, despite having the
            same order. We have $S_3$ is nonabelian while  $\faktor{\Z}{6\Z}$ is
            Abelian.

        \item[(2)] $(\R,+) \not\simeq (\R^*, \cdot)$ since in $\R^*$,
            $\ord{-1}=2$, while there are no elements of order $2$ in  $\R^+$.
    \end{enumerate}
\end{example}

\begin{lemma}\label{1.6.4}
    Let $G$ and  $H$ be groups with representations; let  $G=\vbrack{S : R_1,
    \dots R_n}$ and $H=\vbrack{T : R_1', \dots, R_n'}$. Then if the relations
    $R_i$ is satisfied by the elements of $H$, for each  $1 \leq i \leq n$, then
    there exits a unique homomorphism  $\phi$ definied by  $\phi:R_i \rightarrow
    R_i'$.
\end{lemma}
\begin{remark}
    We defer the proof of this lemma.
\end{remark}

\begin{example}
    \begin{enumerate}
        \item[(1)] Take $D_{2n}=\vbrack{r,t : r^n=t^2=e, rt=t\inv{r}}$, and take
            $X_{2k}=\vbrack{a,b : a^k=b^2=e, ab=b\inv{a}}$. If $n=km$ for  $m
            \in \Z^+$, then  $a^n=(a^k)^m=e$, so the relations of $D_{2n}$ are
            satisfied by the generators of $X_{2n}$. Thus take homomorhism
            $\phi:D_{2n} \rightarrow X_{2k}$ by $\phi:r,t \rightarrow a,b$.
            Since $X_{2k}=\vbrack{a,b}$, $\phi$ is onto.  $\phi$ is $1-1$
            if, and only if  $n=k$. So, in general, $D_{2n} \not\simeq X_{2k}$.

        \item[(2)] Consider $D_6$ and  $S_3$. Let $a=(1 \ 2 \ 3)$ and $b=(1 \
            2)$. Then $a^3=(1)(2)(3)=(1)$ and $b^2=(1)$; moreover, $ab=(1 \ 2 \
            3)(1 \ 2)=(1 \ 2)(3 \ 2 \ 1)=b\inv{a}$. Thus take $\phi:D_6
            \rightarrow S_3$ by $\phi:r,t \rightarrow (1 \ 2 \ 3), (1 \ 2)$. By
            the above reasoning,  $\phi$ is onto. Now since
            $\ord{D_6}=\ord{S_3}=6$, $\phi$ is  $1-1$ and so  $\phi$ is an
            isomorphism and  $D_6 \simeq S_3$.
    \end{enumerate}
\end{example}

\begin{example}
    \begin{enumerate}
        \item[(1)] $\C^* \not\simeq \R^*$, since $i \in \C^*$ has  $\ord{i}=4$,
            while $\R^*$ has no elements of order  $4$.

        \item [(2)] $\R \not\simeq \Q$, for  $\Q$ is countable, and  $\R$ is
            not. That is if we take  $\ord{\Q}$ and $\ord{\R}$ to be defined and
            assume results from set theory and topology, then $\ord{\Q}<
            \ord{\R}$.

        \item[(3)] $\Z \not\simeq \Q$, for suppose otherwise. If  $\phi:\Z
            \rightarrow \Q$ is an isomorphims, then $\phi(1)=a$, then
            $\phi(\frac{1}{2}+\frac{1}{2})=a$, then $2\phi(\frac{1}{2})=a$,
            likewise $3\phi(\frac{1}{3})=a$, then $2\phi(\frac{1}{2})=
            3\phi(\frac{1}{3})$, implying $\phi(\frac{1}{2})=\phi(\frac{1}{3})$,
            but $\frac{1}{2} \neq \frac{1}{3}$, contradicting the $1-1$ness of
            $\phi$.

        \item[(4)] Notice that if $n<m$, then  $\ord{S_n}=n!<\ord{S_m}=m!$, so
            $S_m \simeq S_n$ if, and only if  $n=m$, for $n,m \in \Z^+$.

        \item[(5)] $D_{24} \not\simeq S_4$. Notice $r \in D_{24}=D_{2 \cdot 12}$
            has $\ord{r}=12$. Now, any permutation in $S_4$ is either a
            $4$-cycle, a  $3$-cycle, or a product of two  $2$-cycles, thus every
             $s \in S_4$ has  $\ord{s} \leq 4$, and so there are no elements of
             order $12$ in  $S_4$.
    \end{enumerate}
\end{example}

 We finish with some results.

\begin{lemma}\label{1.6.5}
    Let $A$ and  $B$ be groups. Then  $A \times B \simeq B \times A$.
\end{lemma}
\begin{proof}
    Take the map $A \times B \rightarrow B \times A$ by taking $(a,b)
    \rightarrow (b,a)$. This map is $1-1$ and onto, for $(b,a)=(b',a')$ implies
    $a=a'$ and  $b=b'$, and  $\ord{A \times B}=\ord{B \times A}$. Lastly,
    notice that $(bb',aa')=(b,a)(b'a')$, which makes it a homomorphism. Thus
    there is an isomorphism from  $A \times B$ onto  $B \times A$.
\end{proof}

\begin{lemma}\label{1.5.7}
    Let $A$,  $B$,  $C$ be groups. Then  $(A \times B) \times C \simeq A \times
    (B \times C)$.
\end{lemma}
\begin{proof}
    Consider the map $(A \times B) \times C \rightarrow A \times (B \times C)$
    by taking $(a,(b,c)) \rightarrow ((a,b),c)$. This map is $1-1$ and onto by
    the same reasoning used in the above lemma, moreover, this map is a
    homomorphsm by closure. This completes the proof.
\end{proof}
