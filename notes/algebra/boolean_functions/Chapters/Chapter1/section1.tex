\section{Definitions and Equivalences of Boolean Functions}\label{section_1.1}

\begin{definition}
  Let $p,m, n \in \Z^+$, where $p$ is prime. By an \textbf{$(n,m,p)$-
  function}, we mean a function $f:\F_p^n \xrightarrow{} \F_p^m$. If
  $p=2$, we call $f$ an \textbf{$(n,m)$-Boolean function}, or an
  \textbf{$S$-box}. If $m=n$, we simply call  $f$ a  \textbf{Boolean
  function}. If  $f(x_1, \dots, x_n)=(f_1(x_1, \dots, x_n), \dots,
  f_m(x_1, \dots, x_n))$, we call each $f_i$ the \textbf{$i$-th
  coordinate} function of $f$.
\end{definition}

\begin{lemma}\label{lemma_1.1.1}
  If $f$ is an  $(n,n,p)$-function, then $f$ can be represented
  uniquely as a univariate polynomial over  $\F_{p^n}$ of the form:
  \begin{equation}\label{equation_1.1}
    f(x)=\sum_{i=0}^{p^n-1}{c_ix^i},  \text{ where } c_i \in \F_{p^n}
    \text{ for all } 0 \leq i \leq p^n-1
  \end{equation}
\end{lemma}
\begin{proof}
  Recall that $\F_p^r \simeq \F_{p^r}$ for any integer $r \in \Z^+$.
  Then any $(n,n,p)$-function can be considered as a function taking
  elements of $\F_{p^n}$ to elements of $\F_{p^n}$, so that there is
  a unique polynomial representation of $f$ of the form in equation
  \ref{equation_1.1}.
\end{proof}

\begin{defintion}
  Let $k$ be an integer where $0 \leq k \leq p^n-1$. We define the
  \textbf{$p$-weight} of $k$ to be
  \begin{equation}\label{equation_1.2}
    w_p(k)=\sum_{s=0}^{n-1}{k_s} \text{ for all } 0 \leq k_s \leq p-1
  \end{equation}
  in the $p$-ary expansion  $\sum_{s=0}^{n-1}{p^sk_s}$. We define the
  \textbf{algebraic degree} of an $(n,m,p)$-function $f$ to be
  \begin{equation}\label{equation_2.3}
    d^\circ(f)=\max{\{w_p(i) : c_i \neq 0 \text{ and } 0 \leq i \leq
    p^n-1\}}
  \end{equation}
  where $f$ is represented by equation \ref{equation_1.1}.
\end{defintion}

\begin{definition}
  We call an $(n,n,p)$-function $f$ is called \textbf{linear} if $f$
  can be represented as a polynomial in $\F_{p^n}[x]$ of degree at
  most $n-1$. We call $f$ \textbf{affine} if it is the sum of linear
  functions and a constant.
\end{definition}

\begin{definition}
  We call an $(n,n,p)$-function \textbf{Dembowski-Ostrom polynomial},
  or \textbf{DO-polynomial} if
  \begin{equation}\label{equation_1.4}
    f(x)=\sum_{0 \leq k,j<n}{a_{kj}x^{p^k+p^j}} \text{ where }
    a_{ij} \in \F_{p^n}
  \end{equation}
  We call $f$ if it is the sum of  DO-polynomials and a constant.
\end{definition}

\begin{defintion}
  Let $m$ be a positive divisor of $n$. We define the trace map
  $\tr_n^m:\F_{p^n} \xrightarrow{} \F_{p^m}$ defined by:
  \begin{equation*}
    \tr_n^m(x)=x+x^{p^m}+ \dots +x^{p^{\frac{nm}{m-1}}}
  \end{equation*}
  When $m=1$, we write $\tr_n(x)$, when $n=2$, we write $\tr^m(x)$,
  and when $n=2$ and $m=1$ we write $\tr(x)$.
\end{defintion}

\begin{lemma}\label{lemma_1.1.2}
  Let  $f$ be an $(n,m,p)$-function where $m|n$, then there exists an
  $(n,n,p)$-function $g$ for which $f(x)=\tr_n^m(g(x))$. Moreover,
  any $(n,1,p)$-functioin $f$ can be written in a nonunique way as
  $F(x)=\tr_n(g(x))$, where $g \in \F_{2^n}[x]$, and the univariate
  representation can be written as:
  \begin{equation}\label{equation_1.5}
    f(x)=\sum_{j \in \Gamma_n}{\tr_{o(j)}(a_jx^j)} \text{ where }
    a_j \in \F_p^{o(j)}
  \end{equation}
  where $\Gamma_n$ is the choice set obtained by choosing one
  representative from each cyclotomic coset of $p \mod{p^n-1}$, and
  $o(j)$ is the size of the cyclotomic coset containing both $j$ and
  $a_j$.
\end{lemma}
\begin{proof}
  Since $m|n$,  $\F_{p^m} \subseteq \F_{p^n}$, so we observe that $f$
  is also an $(n,n,p)$-function, and admits the univariate polynomial
  representation of equation \ref{equation_1.1}. Now, fix $x \in
  \F_{p^n}$, then $\tr_n^m{(f(x))}=f(\tr_n^m{x^i})$
\end{proof}

\begin{lemma}\label{1.1.3}
  Let $n=2k$, and $f$ be an $(n,m,p)$-function. Then there exists a
  unique bivariate polynomial over $\F_{p^k}$ representing $f$ of the
  form:
  \begin{equation}\label{equation_1.6}
    f(x)=\sum_{0 \leq i,j, \leq p^k-1}{a_ijx^iy^j} \text{ where } a_{ij} \in
    \F_{p^k}
  \end{equation}
  Moreover
  \begin{equation*}
    d^\circ(f)=\max{\{w_p(i)+w_p(j) : (i,j)|a_{ij} \neq 0\}}
  \end{equation*}
  and
  \begin{equation*}
    f(x,y)=\tr_k(P(x,y)) \text{ where } P \in \F_{p^k}[x,y]
  \end{equation*}
\end{lemma}

\begin{definition}
  We define the \textbf{algebraic normal form} of an $(n,m,p)$-function
  $f$ to be the representation:
  \begin{equation}\label{equation_1.7}
    f(x)=\sum_{u \in \F_p^n}{a_u\prod_{i=1}^nx_i^{u_i}}, \text{ where }
    a_u \in \F_p^m
  \end{equation}
  We define the \textbf{minimum degree} of $f$ to be
  $d^\circ=\min{d^\circ(f)}$, the minimum degree of all non-zero
  linear combinations of the coordinate functions of $f$.
\end{definition}

\begin{definition}
  We call an $(n,m,p)$-function \textbf{balanced} if every element of
  $\F_p^m$ has the same number $p^{n-m}$ of pre-images.
\end{definition}

\begin{defition}
  Let $\d \in \Z^+$. We call an $(n,m,p)$-function
  \textbf{differentially $\d$-uniform} if for every non-zero $a \in
  \F_p^n$, and every $b \in \F_p^m$, the differnce equation:
  \begin{equation}\label{equation_1.8}
    f(x+a)-f(x)=b
  \end{equation}
  has at most $\d$-solutions. We call $f$ \textbf{perfect nonlinear
  (PN)} if it is differentially $p^{n-m}$-uniform. We define the
  \textbf{derivatve} of $f$ \textbf{in the direction} of $a$ to be
  the function
  \begin{equation*}
    D_a{f(x)}=f(x+a)-f(x)
  \end{equation*}
\end{defition}

\begin{lemma}\label{lemma_1.1.4}
  An $(n,m,p)$-function is perfect nonlinear if, and only if $D_a{f}$
  is balanced for every nonzero $a \in \F_p^n$.
\end{lemma}
\begin{corollary}
  If $n=m$, then $f$ is perfect nonlinear if, and only if $D_a{f}$ is
  a permutation.
\end{corollary}

\begin{definition}
  We call perfect nonlinear $(n,n,p)$-functions \textbf{planar}
  functions.
\end{definition}

\begin{example}
  There are no perfect nonlinear functions on $\F_2^n$, for any $n
  \in \Z^+$. Let $f$ be an $(n,m)$-Boolean function, and let $x_0$ be
  a solution to the equation $D_a{f(x)}=b$, where $a \neq 0$. Then
  $x_0+a$ is also a solution, since $D_a{f(x_0+a)}=f((x_0+a)+a)-
  f(x_0+a)=f(x_0+a)-f(x_0)=b$. So $f$ has at least $2$ solutions in
  $\F_2^n$. However, for  $f$ to be PN, $f$ must be differentially
  $1$-uniform. By the corollary to lemma \ref{lemma_1.1.4}, there are
  also no planar functions in $\F_2^n$.
\end{example}

\begin{definition}
  We call a Boolean function $f:F_2^n \xrightarrow{} \F_2^n$
  \textbf{almost perfect nonlinear (APN)} if $D_a{f}$ is a 2--1
  mapping; that is for every $a,b \in \F_2^n$, with $a \neq 0$, we
  have the equation
  \begin{equation*}
    f(x+a)-f(x)=b
  \end{equation*}
  has either exactly $0$, or  $2$ solutions.
\end{definition}

\begin{definition}
  We define the following relations on $(n,m,p)$-functions:
  \begin{enumerate}
    \item[(1)] We call $f$ and $g$ \textbf{affine equivalent} if
      there exist affine permutations $A_1:\F_p^m \xrightarrow{}
      \F_p^m$ and $A_2:\F_p^n \xrightarrow{} \F_p^n$ for which:
      \begin{equation*}
        g=A_1 \circ f \circ A_2
      \end{equation*}
      We call $f$ and $g$ \textbf{linear} equivalent if $A_1$ and
      $A_2$ are linear permutations.

    \item[(2)] We call $F$ and $g$ \textbf{extended affine equivalent
      (EA-equivalent)} if thre exist affine permutations $A_1:\F_p^m
      \xrightarrow{} \F_p^m$ and $A_2:\F_p^n \xrightarrow{} \F_p^n$,
      and an affine function $A:\F_p^n \xrightarrow{} \F_p^n$ for
      which:
      \begin{equation*}
        g=A_1 \circ f \circ A_2+A
      \end{equation*}

    \item[(3)] We call $f$ and $g$ \textbf{Carlet-Charpin-Zimoviev
        equivalent (CCZ-equivalent)} if for any affine permutation
        $\Lc$ of  $\F_p^n \times \F_p^m$, the image of the graph of
        $f$ is the graph of $g$; that is:
        \begin{equation*}
          \Lc(\F_p^n \times f(\F_p^n))=\F_p^n \times g(\F_p^n)
        \end{equation*}
  \end{enumerate}
\end{definition}

\begin{proposition}\label{proposition_1.1.5}
  The following are true for all $(n,m,p)$-functions:
  \begin{enumerate}
    \item[(1)] Affine equivalence is an equivalence relation.

    \item[(2)] EA-equivalence is an equivalence relation.

    \item[(3)] CCZ-equivalence is an equivalence relation.
  \end{enumerate}
\end{proposition}
\begin{corollary}
  Affine equivalence implies linear equivalence.
\end{corollary}
\begin{corollary}
  EA-equivalence implies affine equivalence. Moreover, the algebraic
  degree of a non-affine function is preserved under EA-equivalence.
\end{corollary}
\begin{corollary}
  CCZ-equivalence implies EA-equivalence. Moreover, every permutation
  is CCZ-equivalent to its inverse.
\end{corollary}

\begin{proposition}\label{proposition_1.1.6}
  Let $f$ and  $g$ be $(n,m,p)$-functions. If $\inv{f}$ exists, then
  $g$ is either EA-equivalent to $f$ or to $\inv{f}$ if, and only if
  there exists a linear permutation $\Lc=(L_1,L_2)$ on $\F_2^n \times
  F_2^m$ for which
  \begin{equation*}
    \Lc(\F_p^n \times f(\F_p^m))=\F_p^m \times g(\F_p^m)
  \end{equation*}
  and $L_1$ depends only on one variable.
\end{proposition}
