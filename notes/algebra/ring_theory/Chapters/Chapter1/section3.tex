%----------------------------------------------------------------------------------------
%	SECTION 1.1
%----------------------------------------------------------------------------------------

\section{Ring Homomorphisms and Factor Rings.}

\begin{definition}
    Let $R$ and  $S$ be rings. We call a map  $\phi:R \xrightarrow{} S$ a
    \textbf{ring homomorphism} if
    \begin{enumerate}
        \item[(1)] $\phi$ is a group homomorphism with respect to addition.

        \item[(2)] $\phi(ab)=\phi(a)\phi(b)$ for any $a,b \in R$.
    \end{enumerate}
    We denote the \textbf{kernel} of $\phi$ to be the kernel of $\phi$ as a
    group homomorphism. That is
    \begin{equation*}
        \ker{\phi}=\{r \in R : \phi(r)=0\}
    \end{equation*}
    Moreover, if $\phi$ is 1--1 and onto, we call $\phi$ an \textbf{isomorphism}
    and say that $R$ and $S$ are \textbf{isomorphic}, and write $R \simeq S$.
\end{definition}

\begin{example}\label{1.9}
    \begin{enumerate}
        \item[(1)] $\phi:\Z \xrightarrow{} \faktor{\Z}{2\Z}$ defined by $n
            \xrightarrow{} 0$ if $n$ is even and  $n \xrightarrow{} 1$ if $n$ is
            odd is a ring homomorphism, with  $\ker{\phi}=2\Z$. Notice that
            $\phi(\Z)=\faktor{\Z}{2\Z}$. $\phi$ is onto, but not 1--1.

        \item[(2)] Let $n \in \Z$ and consider the maps  $\phi_n:\Z
            \xrightarrow{} \Z$ by taking $x \xrightarrow{} nx$. $\phi_n$, in
            general is not a ring homomorphism, as  $\phi(xy)=n(xy)$ but
            $\phi(x)\phi(y)=nxny=n^2(xy)$. $\phi_n$, however is a group
            homomorphism for any $n$.

        \item[(3)] For any ring $R$, define the \textbf{valuation} map
            $\phi:R[x] \xrightarrow{} R$ by taking $f(x) \xrightarrow{} f(0)$;
            i.e. the polynomial $f$ evaluated at $0$.  $\phi$ is a ring
            homomorphism. Moreover, notice that if
            $f(x)=a_0+a_1x+\dots+a_nx^n$, then $f(0)=a_0 \in \R$. So that
            $\phi(R[x])=R$ This makes $\phi$ onto. Now, take  $\phi(f)=0$, Then
            those are all polynomials with constant term $a_0=0$ (this does not
            make $\ker{\phi}=\langle e \rangle$). Again, $\phi$ is onto, but it
            is not 1--1.
    \end{enumerate}
\end{example}

\begin{lemma}\label{1.3.1}
    Let $R$ and $S$ be rings, and $\phi:R \xrightarrow{} S$ a ring homomorphism.
    Then
    \begin{enumerate}
        \item[(1)] $\phi(R)$ is a subring of $S$.

        \item[(2)] $\ker{\phi}$ is a subring of $R$.
    \end{enumerate}
\end{lemma}
\begin{proof}
    Let $s_1,s_2 \in \phi(R)$. Then $s_1=\phi(r_1)$ and $s_2=\phi(r_2)$ for some
    $r_1,r_2 \in R$. Then $s_1s_2=\phi(r_1)\phi(r_2)=\phi(r_1r_2) \in \phi(S)$.
    Additionally, $\inv{s}=\inv{\phi}(r)=\phi(\inv{r})$ for some $s \in S$,  $r
    \in R$. This is sufficient to make  $S$ a subring of  $S$.

    By similar reasoning, if $r_1,r_2 \in \ker{\phi}$, then
    $\phi(r_1)\phi(r_2)=\phi(r_1r_2)=0$ so that $r_1r_2 \in \ker{\phi}$, and
    $\phi(\inv{r})=\inv{\phi}(r)=0$ so $\inv{\phi}\in \ker{\phi}$.
\end{proof}
\begin{corollary}
    For any $r \in R$ and  $a \in \ker{\phi}$, then $ar \in \ker{\phi}$ and $ra
    \in \ker{\phi}$.
\end{corollary}
\begin{proof}
    We have $\phi(ar)=\phi(a)\phi(r)=\phi(a)0=0$ so $ar \in \ker{\phi}$. The
    same happens for $ra$.
\end{proof}

\begin{definition}
    Let $R$ be a ring. We call a subring $I \leq R$ of $R$ a \textbf{left
    ideal} in $R$ if for any $r \in R$ and $a \in I$, we have  $ar \in I$.
    Similarly, we call  $I$ a  \textbf{right ideal} in $R$ if  $ra \in I$. We
    call  $I$ a  (\textbf{two-sided}) \textbf{ideal} in $R$ if it is both a
    left, and a right ideal and we say that the ideals $I$ \textbf{absorb} $r$.
\end{definition}

\begin{lemma}\label{1.3.2}
    If $R$ is a commutative ring, then every left ideal is a right ideal.
\end{lemma}
\begin{proof}
    Notice that $ar=ra$ for all  $a,r \in R$.
\end{proof}

\begin{theorem}\label{1.3.3}
    Let $R$ be aring, and $I$ an ideal in $R$. Let  $\faktor{R}{I}$ be the set
    of all $a+I$ with  $a \in R$. Define operations $+$ and $\cdot$ by
    \begin{align*}
        (a+I)+(b+I) &=  (a+b)+I \\
        (a+I)(b+I)  &=  ab+I    \\
    \end{align*}
    Then $\faktor{R}{I}$ forms a ring under  $+$ and  $\cdot$.
\end{theorem}
\begin{proof}
    Notice that $(a+I)+(b+I)=(a+b)+(I+I)=(a+b)+2I=(a+b)+I$. Moreover,
    $\faktor{R}{I}$ inherits associativity in $+$ from addition in $R$. Now,
    take $0+I=I$ as the additive identity and  $-a+I$ as the inverse of  $a+I$
    for each  $I$.

    Now, notice, that  $(a+I)(b+I)=ab+aI+bI+I^2=ab+(I+I+I)=ab+I$ by distribution
    of multiplication over addition in $R$. Moreover,  $\faktor{R}{I}$ also
    inherits associativity in $\cdot$ of ultiplication in $R$. Now, notice then
    that
    \begin{equation*}
        (a+I)((b+I)+c+I)=(a+I)((b+c)+I)=a(b+c)+I=(ab+ac)+I=(ac+I)+(bc+I)
    \end{equation*}
    and
    \begin{equation*}
        ((a+I)+(b+I))(c+I)=((a+b)+I)(c+I)=(a+b)c+I=(ac_bc)+I=(ac+I)+(bc+I)
    \end{equation*}

    Lastly, notice that $a+I$ is just the left coset of  $a$ by  $I$ in $R$ as a
    group under addition. So that $+$ and  $\cdot$ are coset addition and
    multiplication, which are well defined.
\end{proof}
\begin{corollary}
    If $R$ has identity $1$, then $\faktor{R}{I}$ has identity $1+I$. Moreover
    if  $R$ is commutative, then so is  $\faktor{R}{I}$.
\end{corollary}

\begin{definition}
    Let $R$ be a ring and $I$ an ideal in $R$. We call the ring $\faktor{R}{I}$
    under addition and muiltplication of cosets the \textbf{factor ring} (or
    \textbf{quotient ring}) of $R$ over  $I$.
\end{definition}

\begin{example}\label{1.10}
    \begin{enumerate}
        \item[(1)] We call $(0)=\{0\}$ the \textbf{trivial ideal}, notice also
            that $R$ is also an ideal.

        \item[(2)] For any $n \in \Z$, notice that if $a \in \Z$ and $m \in
            n\Z$, then  $m=nk$, for some  $k \in \Z$ so that  $am=n(ak)=ma \in
            n\Z$. So $n\Z$ is an ideal of $\Z$, with factor ring
            $\faktor{\Z}{n\Z}$. So $\faktor{\Z}{n\Z}$ is a factor ring on top of
            also being a factor group. We call the ring homomorphisme $\phi:\Z
        \xrightarrow{} \faktor{\Z}{n\Z}$ by $a \xrightarrow{} a \mod{n}$ the
        \textbf{reduction homomorphism}.

        \item[(2)] Let $R$ a ring, and consider  $R[x]$. Let $I$ the set of all
            polynomials of degree greater than $2$ together with $0$. Then if
            $f \in I$,  $\deg{f}>2$ or $f=0$. Then for any  $g \in R[x]$,
            $\deg{fg}>2$ or, $fg=0$ and $\deg{gf}>2$ or $gf=0$. This makes  $I$
            an ideal of $R[x]$. Moreover, $p,q \in I$ if and only if they have
            the same constant term. Notice then that $\faktor{t\R[x]}{I}=\{a+bx
            : a,b \in R\}$.

            Now, if $R$ has no zero divisors, it is possible that
            $\faktor{R[x]}{I}$ has zero divisors. Consider $\faktor{\Z[x]}{I}$.

        \item[(3)] Let $A$ a ring, and  $X \neq \emptyset$. For the ring of
            functionss $A^X$, for a given $c \in X$, define the
            \textbf{valuation} map at $c$ by $E_c:f(x) \xrightarrow{} f(c)$.
            Notice that $E_c$ is a ring homomorphism, so that
            $\faktor{A^X}{\ker{E_c}}$ forms a factor ring. IN particular, if
            $A^X=A[x]$ the polynomial ring over $A$, and  $c=0$, then  $E_c$ is
            just the valuation map of polynomials.

            Now, if  $X=(0,1]$, and $R=\R^{(0,1]}$, by the first isomorphism
            theorem, we have $\R \simeq \faktor{\R^{(0,1]}}{\ker{E_c}}$, since
        $E_c(\R^{(0,1]})=\R$.

    \item[(4)] Let $n \geq 2$ and consider  $R^{n \times n}$. Let $J$ an ideal
        of  $R$. Then  $J^{n \times n}=\{(a_{ij}) : a_{ij \in J}\}$ is an ideal
        of $R^{n \times n}$. Take the ring homomorphism
        \begin{align*}
            R^{n \times n}  & \xrightarrow{} (\faktor{R}{J})^{n \times n}   \\
            (a_{ij})    & \xrightarrow{} (a_{ij}+J) \\
        \end{align*}
        Then $J^{n \times n}$ is the kernel of this homomorphism, so that
        \begin{equation*}
            \faktor{R^{n \times n}}{J^{n \times n}} \simeq (\faktor{R}{J})^{n \times n}
        \end{equation*}
        For example, with $n=3$, we have
        \begin{equation*}
            \faktor{\Z^{3 \times 3}}{2\Z^{3 \times 3}} \simeq
            (\faktor{Z}{2\Z})^{3 \times 3}
        \end{equation*}

    \item[(5)] Let $R$ a commutative ring with identity, and  $G$ a finite group
        of order $n$. Define the  \textbf{augmentation} map to be the map
        \begin{align*}
            RG      & \xrightarrow{}    R   \\
            \sum_{i=1}^n{a_ig_i}    & \xrightarrow{} \sum_{i=1}^n{a_i}  \\
        \end{align*}
        We call the kernel of this map the \textbf{augmentation ideal} which is
        the set of all formal sums whose coefficients sum to $0$. Another ideal
        of  $RG$ is the set $I=\{\sum{ag_i} : g_i \in G\}$ the set of all formal
        sums whose coefficients are all equal.
    \end{enumerate}
\end{example}

\begin{theorem}[The First Isomorphism Theorem]\label{1.3.4}
    If $\phi:R \xrightarrow{} S$ is a ring homomorphism from rings $R$ into $S$,
    then $\ker{\phi}$ is an ideal of $R$ and
    \begin{equation*}
        \phi(R) \simeq \faktor{R}{\ker{\phi}}
    \end{equation*}
    \[\begin{tikzcd}
        R &&& S \\
        \\
        \\
        {\faktor{R}{\ker{\phi}}}
        \arrow["\pi"', from=1-1, to=4-1]
        \arrow["\phi", from=1-1, to=1-4]
        \arrow["{\bar{\phi}}"', from=4-1, to=1-4]
    \end{tikzcd}\]
\end{theorem}
\begin{proof}
    By the first isomorphism theorem for groups, $\phi$ is a group isomorphism.
    Now, let $K=\ker{\phi}$ and consider the map $\pi:R \xrightarrow{}
    \faktor{R}{I}$ by $a \xrightarrow{\pi} a+K$. Define the map
    $\bar{\phi}:\faktor{R}{K} \xrightarrow{} \phi(R)$ such that $\bar{\phi}
    \circ \pi=\phi$, then $\bar{\phi}$ defines the ring isomorphism.
\end{proof}
\begin{proof}
    The map $\pi:R \xrightarrow{} \faktor{R}{I}$ defined by $a \xrightarrow{}
    a+I$, for any ideal $I$, is onto, with  $\ker{\pi}=I$.
\end{proof}

\begin{theorem}[The Second Isomorphism Theorem]\label{1.3.5}
    Let $A \subseteq R$ a subring of  $R$, and let  $B$ an ideal in  $R$. Define
     $A+B=\{a+b : a \in A \text{ and } b \in B\}$. Then $A+B \susbeteq R$ is a
     subring and  $A \cap B$ is an ideal in $A$. Then
     \begin{equation*}
         \faktor{A+B}{B} \simeq \faktor{A}{A \cap B}
     \end{equation*}
\end{theorem}

\begin{theorem}[The Third Isomorphism Theorem]\label{1.3.6}
    Let $I$ and  $J$ be ideals in a ring  $R$, with  $I \subseteq J$. Then
    $\faktor{J}{I}$ is an ideal of $\faktor{R}{I}$ and
    \begin{equation*}
        \faktor{R}{J}=\faktor{(\faktor{R}{I})}{(\faktor{J}{I})}
    \end{equation*}
\end{theorem}

\begin{theorem}[The Fourth Isomorphism Theorem]\label{1.3.7}
    Let $I$ an ideal in a ring $R$, then the correspondence between  $A$ and
    $\faktor{A}{I}$, for any subring $A \subseteq R$ is an inclusion preserving
    bijection between subrings of $A$ containing  $I$ and  $\faktor{R}{I}$.
    Moreover, $A$ is an ideal if, and only if  $\faktor{A}{I}$ is an ideal.
\end{theorem}

\begin{example}\label{1.11}
    We have $12\Z$ is an ideal of  $\Z$, and that  $\faktor{\Z}{12\Z}$ has as
    ideals
    \begin{align*}
        \faktor{\Z}{12\Z} && \faktor{2\Z}{12\Z} && \faktor{3\Z}{12\Z} &&
        \faktor{4\Z}{12\Z} && \faktor{6\Z}{12\Z} && \faktor{12\Z}{12\Z}
    \end{align*}
\end{example}

\begin{lemma}\label{1.3.8}
    Let $R$ be a ring with ideals  $I$ and  $J$. Then  $I+J$,  $IJ$ and  $I^n$,
    for any  $n \geq 0$ are ideals of  $R$ and we have the lattice
    \[\begin{tikzcd}
	& R \\
	& {I+J} \\
	I && J \\
	& IJ \\
	& {(0)}
	\arrow[no head, from=1-2, to=2-2]
	\arrow[no head, from=2-2, to=3-1]
	\arrow[no head, from=3-1, to=4-2]
	\arrow[no head, from=4-2, to=3-3]
	\arrow[no head, from=3-3, to=2-2]
	\arrow[no head, from=4-2, to=5-2]
    \end{tikzcd}\]
\end{lemma}

\begin{example}\label{1.12}
    \begin{enumerate}
    \item[(1)] COnsider the ideals $6\Z$ and  $10\Z$ of $\Z$. Then $6\Z+10\Z$ is
        the ideal consisting of all integers of the form  $6x+10y$. Now, for
        $x,y \in \Z$, since $(6,10)=2$, we have that $6\Z+10\Z \subseteq 2\Z$
        since  $6x+10y=2(3x+5y)$. Now, we also have that $2=6 \cdot 2+10 \cdot
        -1$ so that  $2 \in 6\Z+10\Z$ which makes $2\Z \subseteq 6\Z+10\Z$.
        Thus, we have $6\Z+10\Z=2\Z$. In general, we have that $m\Z+n\Z=d\Z$
        where $d=(m,n)$ is the greatest common divisor of $m$ and  $n$. The
        ideal $6\Z10\Z$ gives all integers of the form $6x10y=6 \cdot 10
        (xy)=60(xy)$, so that $6\Z10\Z=60\Z$.

    \item[(2)] Let $I \subseteq \Z[x]$ the ideal of polynomials with even
        constant term. Notce that $2,x=x+0 \in I$ so tht  $4,x^2 \in I^2=II$. So
        that  $4+x^2\in I^2$ which is not in general divisible by elements in $I$.
    \end{enumerate}
\end{example}
