\section{Module Homomorphisms}

\begin{definition}
  Let $R$ be a ring, and $M$ and  $N$  $R$-modules. An
  \textbf{$R$-module homomorphism} of $M$ to $N$ is a map such that
  for all $x,y \in M$, and $r \in R$:
  \begin{enumerate}
    \item[(1)] $\phi(x+y)=\phi(x)+\phi(y)$.

    \item[(2)] $\phi(rx)=r\phi(x)$.
  \end{enumerate}
  If $\phi$ is a 1--1 map of $M$ onto $N$, then we call  $\phi$ an
  \textbf{$R$-module isomorphism}, and we write $M \simeq N$. We
  denote the set of all $R$-module homomorphisms of  $M \xrightarrow{}
  N$ by $\Hom_R{(M,N)}$.
\end{definition}

\begin{definition}
  Let $R$ be a ring, and $M$ and $N$ be $R$-modules, and let $\phi \in
  \Hom_R{(M,N)}$. We define the \textbf{kernel} of $\phi$ to be the
  set
  \begin{equation*}
    \ker{\phi}=\{ x \in M : \phi(x)=0 \}
  \end{equation*}
  We define the \textbf{image} of $\phi$ to simply be
  $\im{\phi}=\phi(M)$.
\end{definition}

\begin{lemma}\label{label_4.2.1}
  Let $M$ and $N$ be $R$-modules, and let  $\phi \in \Hom_R{(M,N)}$.
  Then $\ker{\phi}$ and $\im{\phi}$ are submodules
  of $M$ and $N$ respectively.
\end{lemma}
\begin{proof}
  We have $\phi(0)=\phi(0x)=0\phi(x)=0$. This shows that $\ker{\phi}$
  and $\im{\phi}$ are both nonempty. Fix some $r \in R$. Now,
  let $x,y \in \ker{\phi}$. Then:
  \begin{equation*}
    \phi(x-ry)=\phi(x)-r\phi(y)=0-r0=0
  \end{equation*}
  so $x-ry \in \ker{\phi}$. Likewise, suppose that $x', y' \in
  \im{\phi}$. Then $x'=\phi(x)$ and $y'=\phi(y)$ for some $x,y \in M$.
  Since $M$ is a module, we have $x-ry \in M$. Observe:
  \begin{equation*}
    x'-ry'=\phi(x)-r\phi(y)=\phi(x-ry)
  \end{equation*}
  which puts $x'-ry' \in \im{\phi}$ and we are done.
\end{proof}

\begin{example}\label{example_4.3}
  \begin{enumerate}
    \item[(1)] Consider the $\Z$-module homomorphism  $\phi:x
      \xrightarrow{} 2x$ of $\Z \xrightarrow{} \Z$. Observe that
      $\phi:1 \xrightarrow{} 2 \cdot 1=2$, so that $\phi$ does not
      preserve identities. Hence  $\phi$ is not a ring homomorphism of
       $\Z \xrightarrow{} \Z$.

       Likewise, let $k$ be a field, and consider the $k[x]$-module
       homomorphism $\phi:f(x) \xrightarrow{} f(x^2)$ of $k[x]
       \xrightarrow{} k[x]$. Observe:
       \begin{equation*}
         x^2=\phi(x)=\phi(1x)=x\phi(1)=x(1^2)=x
       \end{equation*}
       which shows that $\phi$ is not a ring homomorphism of  $k[x]
       \xrightarrow{} k[x]$.

     \item[(2)] For any ring $R$, the projection map $\pi_i:R^n
       \xrightarrow{} R$ defined by $\pi_i:(x_1, \dots, x_n)
       \xrightarrow{} x_i$, where $1 \leq i \leq n$, is an $R$-module
       homomorphism of the free module $R^n$ onto $R$ (indeed we can
       verify that $\im{\phi}=R$). $\pi_i$ is not 1--1, so it does not
       define an isomorphism between $R^n$ and $R$ as  $R$-modules.

     \item[(3)] If $R$ is a field, then the  $R$-module homomorphisms
       are just linear transformations over  $R$, by definition.

     \item[(4)] Let $\phi:M \xrightarrow{} N$ be a $\Z$-module
       homomorphism (i.e. $\phi \in \Hom_\Z{(M,N)}$). Since $M$ and
       $N$ are $\Z$-modules, they are (identified with) Abelian
       groups (whose operation is written additively). Let $n \in \Z^+$.
       Observe that
       $\phi(\underbrace{x+\dots+x}_{n-\text{times}})=\phi(nx)=n\phi(x)$
       so that commutativity of the action $R \times M \xrightarrow{}
       M$ under $\phi$ is subsumed by the first property. That is,
       $\phi$ is reduced to a group homomorphism of $M \xrightarrow{}
       N$ as Abelian groups.

     \item[(5)] Let $R$ be a ring, and $I$ an ideal of $R$. Let  $M$
       and  $N$ be  $R$-modules annihilated by  $I$. Then by example
       \ref{example_4.1}(5), any $R$-module homomorphism of $M
       \xrightarrow{} N$ can be made into an $(\faktor{R}{I})$-module
       homomorphism of $M \xrightarrow{} N$, as
       $(\faktor{R}{I})$-modules.

     \item[(6)] Consider the above example. If $A$ is an Abelian
       group, and  $p \in \Z^+$ is a prime for which $px=0$ for all
       $x \in A$, then $A$ is annihilated by the ideal $p\Z$. Then any
       group homomorphism of $A \xrightarrow{} A$ is a
       $(\faktor{\Z}{p\Z})$-module homomorphism. That is, group
       homomorphisms between the same Abelian group are linear
       transformations over the field $\F_p$.
  \end{enumerate}
\end{example}

\begin{theorem}\label{theorem_4.2.2}
  Let $M$, and $N$ be $R$-modules. Then a map $\phi:M \xrightarrow{}
  N$ is an $R$-module homomorphism if, and only if for every  $x,y \in
  M$ and $r \in R$:
  \begin{equation*}
    \phi(rx+y)=r\phi(x)+r\phi(y)
  \end{equation*}
\end{theorem}
\begin{proof}
  If $\phi \in \Hom_R{(M,N)}$, then by definition,
  $\phi(rx+y)=r\phi(x)+\phi(y)$. Suppose now the converse is true.
  Choose $r=1$, so that $\phi(x+y)=\phi(x)+\phi(y)$, and choose $y=0$
  so that $\phi(rx)=r\phi(x)$.
\end{proof}

\begin{theorem}\label{theorem_4.2.3}
  Let $M$ and $N$ be $R$-modules. Then $\Hom_R{(M,N)}$ is an Abelian
  group under function addition. Moreover if $R$ is commutative, then
   $\Hom_R{(M,N)}$ is a left $R$-module under the action given by:
  \begin{equation*}
    (r\phi)(m)=r(\phi(m)) \text{ for all } r \in R \text{ and } m \in M
  \end{equation*}
\end{theorem}
\begin{proof}
  That $\Hom_R{(M,N)}$ is an Abelian group under $+$ (of functions)
  can be seen by observing that the elements are just functions. Now,
  suppose that $R$ is commutative, and define the action $R \times
  \Hom_R{(M,N)} \xrightarrow{} \Hom_R{(M,N)}$ by $(r,\phi)
  \xrightarrow{} r\phi$. Take $m,n \in M$, and denote $m=\phi(x)$ and
  $n=\phi(y)$. Let $r,s \in R$. Then by the $R$-module axioms on $N$,
  we get
  \begin{enumerate}
    \item[(1)] $r(m+n)=rm+rn$

    \item[(2)] $(r+s)m=rm+sm$

    \item[(3)] $(rs)m=r(sm)$
  \end{enumerate}
  which makes $\Hom_R{(M,N)}$ into a left $R$-module, since $m$ and
  $n$ are just images under $\phi$.
\end{proof}
\begin{corollary}
  If $R$ is a commutative ring with identity, then $\Hom_R{(M,N)}$ is
  a unital left $R$-module.
\end{corollary}

\begin{theorem}\label{theorem_4.2.4}
  Let $M$, $N$, and $L$ be  $R$-modules. If  $\phi \in \Hom_R{(L,M)}$
  and $\psi \in \Hom_R{(M,N)}$, then $\psi \circ \phi \in
  \Hom_R{(L,N)}$. That is, the following diagram commutes:
  \[\begin{tikzcd}
    L & M \\
      & N
      \arrow["\phi", from=1-1, to=1-2]
      \arrow["{\psi \circ \phi}"', from=1-1, to=2-2]
      \arrow["\psi", from=1-2, to=2-2]
  \end{tikzcd}\]
\end{theorem}
\begin{proof}
  Let $\phi \in \Hom_r{(L,M)}$ and $\psi \in \Hom_R{(M,N)}$. Then for
  $r \in R$, and $x,y \in L$:
  \begin{align*}
    \psi \circ \phi(rx+y) &=  \psi(\phi(rx+y))  \\
      &=  \psi(r\phi(x)+\phi(y))  \\
      &=  r\psi(\phi(x))+\psi(\phi(y))  \\
      &= r(\psi \circ \phi)(x)+\psi \circ \phi(y)
  \end{align*}
  which puts $\psi \circ \phi \in \Hom_R{(L,N)}$.
\end{proof}

\begin{theorem}\label{theorem_4.2.5}
  Let $M$ be an  $R$-module.  $\Hom_R{(M,M)}$ is a ring with identity
  under function addition and function composition. Moreover, if $R$
  is commutative, then $\Hom_R{(M,M)}$ is an $R$-algebra.
\end{theorem}
\begin{proof}
  Since $\Hom_R{(M,M)}$ consists of $R$-module homomorphism whose
  domains and codomains coincide, functio composition is a well
  defined operation in this set. Now, $\Hom_R{(M,M)}$ is an Abelina
  group under $+$ by \ref{theorem_4.2.3}, and the ring operations
  follow by observing the usual rules of $\circ$.

  Now, suppose that $R$ is a commutative ring, then again by theorem
  \ref{theorem_4.2.3}, $R$ is made into an $R$-module under the action
  $(r,\phi) \xrightarrow{} r\phi$. Defining $r\phi=\phi r$, we make
  $\Hom_R{(M,M)}$ into an $R$-algebra.
\end{proof}

\begin{definition}
  Let $M$ be an  $R$-module. The ring  $\Hom_R{(M,M)}$ is called the
  \textbf{endomorphism ring} of $M$ over $R$, and is denoted
  $\End_R{M}$, or simply just $\End{M}$.
\end{definition}

\begin{lemma}\label{lemma_4.2.6}
  Let $R$ be a commutative ring. Then there is a natural map on
  $\End_R{M}$ defined by $r \xrightarrow{} rI$, where $I$ is an ideal
  of $R$. Moreover if $R$ has identity, then  $\End_R{M}$ is an
  $R$-algebra.
\end{lemma}
