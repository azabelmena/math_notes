\section{Module Homomorphisms and Factor Modules}

We make the conventions that when we say ``ring'', we mean a ring with identity,
and when we say ``module'', we mean unital modules.

\begin{definition}
    Let $R$ be a ring, and $M$ and  $N$ modules. We call a map $\phi:M
    \xrightarrow{} N$ a  \textbf{module homomorphism} over $R$ (or an
    \textbf{$R$-module homomorphism}) if for all $x,y \in M$, and  $r \in R$
    \begin{enumerate}
        \item[(1)] $\phi(x+y)=\phi(x)+\phi(y)$.

        \item[(2)] $\phi(rx)=r\phi(x)$.
    \end{enumerate}
    We call $\phi$ a \textbf{module isomorphism} (or an \textbf{$R$-module
    isomorphism}) if $\phi$ is 1--1 and takes $M$ onto $N$; and we call  $M$ and
     $N$  \textbf{isomorphic} and write $M \simeq N$.
\end{definition}

\begin{definition}
    Let $R$ be a ring,  $M$ and  $N$  $R$-modules, and let $\phi:M
    \xrightarrow{} N$ be a module homomorphism over $R$. We define the
    \textbf{kernel} of $\phi$ to be the set
    \begin{equation*}
        \ker{\phi}=\{m \in M : \phi(m)=0\}
    \end{equation*}
    We denote the set $\Hom_R{(M,N)}$ to be the set of all module homomorphisms
    over $R$ from $M$ into $N$.
\end{definition}

\begin{lemma}\label{4.2.1}
    Let $R$ be a ring, and  $M$, and $N$ module s. let $\phi:M \xrightarrow{} N$
    be a module homomorphism of $R$. Then the following are true
    \begin{enumerate}
        \item[(1)] $\ker{\phi}$ is a submodule of $M$.

        \item[(2)] $\phi(M)$ is a submodule of $N$.
    \end{enumerate}
\end{lemma}
\begin{proof}
    Notice that if $\phi:M \xrightarrow{} N$ is a module homomorphism, then it
    is also a group homomorphism, since $M$ and $N$ are both groups under their
    respective additions. This makes $\ker{\phi}$ and $\phi(M)$ into subgroups
    of $M$ and  $N$ respectively (this also implies that both $\ker{\phi}$ and
    $\phi(M)$ are nonempty).

    Now, let $r,s \in R$, and  $x,y \in \ker{\phi}$. Then we get
    $\phi(rx+sy)=r\phi(x)+s\phi(y)=r0+s0=0$, so that $rx+sy \in \ker{\phi}$.
    Similarly, if $\phi(x), \phi(y) \in \phi(M)$, then
    $r\phi(x)+s\phi(y)=\phi(rx)+\phi(sy)=\phi(rx+sy) \in \phi(M)$. By the
    submodule criterion for unital modules, we are done.
\end{proof}

\begin{example}\label{example_4.4}
    \begin{enumerate}
        \item[(1)] Module homomorphisms need not be ring homomorphisms. Consider
            the ring $\Z$ as a module over itself. Then the map  $x
            \xrightarrow{} 2x$ is a $\Z$-module homomorphism, but not a ring
            homomorphism, since this map takes  $1 \xrightarrow{} 2$, and $2
            \neq 1$ in  $\Z$. Likewise, ring homomorphisms need not be module
            homomorphisms. Let $F$ be a field, and consider the map $f(x)
            \xrightarrow{} f(x^2)$, where $f(x) \in F[x]$. This map is a ring
            homomorphism, but it is not an  $F[x]$-module homomorphism when
            considering $F[x]$ as a module over itself.

        \item[(2)] Let $R$ be a ring, and  $n \in \Z^+$ Consider the \textbf{$i$-th
            projection map} $\pi_i:R^n \xrightarrow{} R$ defined by
            \begin{equation*}
                \pi_i:(x_1, \dots, x_n) \xrightarrow{} x_i
            \end{equation*}
            Then $\pi_i$ is an $R$-module homomorphism. Moreover, we have
            $\pi_i(R^n)=R$ (so that $\pi_i$ is onto) and $\ker{\p_i}=\{(x_1,
            \dots, x_n) : x_i=0\}$.

            \item[(3)] For any field $F$, and any vector spaces $V$ and $W$ over
                 $F$, the usual linear transofrmations  $T:V \xrightarrow{} W$
                 are $F$-module homomorphisms by definition.

             \item[(4)] $\Z$-module homomorphisms are Abelian group
                 homomorphisms. Recall that  module homomorphisms are group
                 homomorphisms, and that $\Z$ is an Abelian group.

             \item[(5)] Let $R$ be a ring, and  $I$ an ideal of  $R$. Suppose
                 that  $M$ and  $N$ are  $R$-modules which are annihilated by
                 $I$. Then any $R$-module homomorphism of $M \xrightarrow{} N$
                 is also an $(\faktor{R}{I})$-module homomorphism.
    \end{enumerate}
\end{example}

\begin{lemma}\label{4.2.2}
    Let $R$ be a ring, and let $M$, $N$, and $L$ be modules over $R$. Then the
    following are true
    \begin{enumerate}
        \item[(1)] A map $\phi:M \xrightarrow{} N$ is a module homomorphism over
            $R$ if, and only if $\phi(rx+sy)=r\phi(x)+s\phi(y)$ for any $x,y
            \in M$ and  $r,s \in R$.

        \item[(2)] $\Hom_R{(M,N)}$ is an Abelian group under function addition.

         \item[(3)] If $\phi \in \Hom_R{(M,N)}$ and $\psi \in \Hom_R{(N,L)}$,
             then $\psi \circ \phi \in \Hom_R{(M,L)}$ where $\circ$ is the usual
             function composition.

         \item[(4)] $\Hom_R{(M,M)}$ is a ring with identity under the operations
             of function addition and function composition.
    \end{enumerate}
\end{lemma}
\begin{proof}
    We break the proof into its respective parts.
    \begin{enumerate}
        \item[(1)] By definition, if $\phi$ is a module homomorphism, then
            $\phi(rx+sy)=\phi(rx)+\phi(sy)=s\phi(x)+y\phi(y)$ for all $x,y \in
            M$ and  $r,s \in R$. Conversely if  $\phi(rx+sy)=r\phi(x)+s\phi(y)$,
            taking $r,s=1$ gives  $\phi(x+y)=\phi(x)+\phi(y)$, and taking $s=0$
            and $y=0$ gives $\phi(rx)=r\phi(x)$.

        \item[(2)] Now, consider $\phi,\psi \in \Hom_R{(M,N)}$. Observing that
            $(\phi+\psi)(x)=\phi(x)+\psi(x)$, we get for all $x,y \in M$ and
            $r,s \in R$ that
            \begin{align*}
                (\phi+\psi)(rx+sy) &=   \phi(rx+sy)+\psi(rx+sy)    \\
                             &=  (r\phi(x)+s\phi(y))+(r\psi(x)+s\psi(y))    \\
                             &= (r\phi(x)+r\psi(x))+(s\phi(y)+s\psi(y)) \\
                             &= r(\phi+\psi)(x)+s(\phi+\psi)(y) \\
            \end{align*}
            Since $x,y \in M$ and  $\phi(x),\phi(y),\psi(x),\psi(y) \in N$, this
            makes $\phi+\psi \in \Hom_R{(M,N)}$. Moreover, notice that $+$ is
            associative and commutative. We get the map $x \xrightarrow{} 0$ as
            the identity homomorphism, and $(-\phi):x \xrightarrow{} -\phi(x)$
            as the additive inverse of $\phi$. This makes  $\Hom_R{(M,N)}$ into
            a group.

        \item[(3)] Let $\phi \in \Hom_R{(M,N)}$ and $\psi \in \Hom_R{(N,L)}$.
            Then for $x,y \in M$ and  $r,s \in R$, and writing $\psi
            \circ \phi$ as  $\psi\phi$ we get
            \begin{align*}
                \psi\phi(rx+sy) &=  \psi(\phi(rx+sy))   \\
                          &=    \psi(r\phi(x)+s\phi(y)) \\
                          &= r\psi\phi(x)+s\psi\psi(y)  \\
            \end{align*}
            Since $x,y \in M$ and  $\psi\phi(x), \psi\phi(y) \in L$, this makes
            $\psi \circ \phi \in \Hom_R{(M,L)}$.

        \item[(4)] Finaly, observe that since $\Hom_R{(M,M)}$ is an Abelian
            group under $+$, and  $\circ$ is well defined for  $\Hom_R{(M,M)}$,
            then it suffies to show the distributive laws and identity. Notice
            however that $\circ$ distributes (on both side) over $+$. Moreover,
             $\io:x \xrightarrow{} x$ serves as the identity element for
             $\Hom_R{(M,M)}$ under $\circ$. That is $\Hom_R{(M,M)}$ is indeed a
             ring with identity.
    \end{enumerate}
\end{proof}
\begin{corollary}
    If $R$ is a commutative ring, then the following are true
    \begin{enumerate}
        \item[(1)] $\Hom_R{(M,N)}$ is a module over $R$, under the action
            $(r\phi)(x)=r(\phi(x))$.

        \item[(2)] $\Hom_R{(M,M)}$ is an $R$-algebra under function addition and
            function composition.
    \end{enumerate}
\end{corollary}
\begin{proof}
    Again, spilitting the proof by parts
    \begin{enumerate}
        \item[(1)] Notice that for $r,s \in R$, that
            \begin{align*}
                (r\phi)(sx) &=  r(\phi(sx)) \\
                         &= r(s\phi(x)) \\
                         &= (rs)\phi(x) \\
                         &= (sr)\phi(x) \\
                         &= s(r\phi(x)) \\
                         &= s((r\phi)(x))   \\
            \end{align*}
            This makes $r\phi \in \Hom_R{(M,N)}$, which makes it into a module
            over $R$.

        \item[(2)] By (4) in the above lemma, we have that $\Hom_R{(M,M)}$ is a
            ring with identity. Moreover, notice that (1) in the above statement
            makes $\Hom_R{(M,M)}$ into a module over $R$. Then taking
            $\phi{r}=r\phi$, we get that $\Hom_R{(M,M)}$ is an $R$-algebra.
    \end{enumerate}
\end{proof}

\begin{definition}
    Let $R$ be a ring, and  $M$ a module over  $R$. We call the ring
    $\Hom_R{(M,M)}$ the \textbf{endomorphism ring} of $M$ and denote it
    $\End_R{M}$, or $\End{M}$ (when context is clear). We call the elements of
    $\End_R{M}$ \textbf{endomorphisms}.
\end{definition}

\begin{example}\label{example_4.5}
    Let $R$ be a ring with ideal $I$, and  $M$ an  $R$-module. The ring
    homomorphism $R \xrightarrow{} \End{M}$ defined by $r \xrightarrow{} rI$
    need not be 1--1. Notice that the homomorphism $\Z \xrightarrow{}
    \End{\faktor{\Z}{2\Z}}$ defined by $2 \xrightarrow{} 2\Z$ is not 1--1.
    Notice that $2x=0$ for any $x \in \faktor{\Z}{2\Z}$, so that the kernel is
    $\faktor{\Z}{2\Z}$. However, if $R$ is a field, then the homomorphism  $r
    \xrightarrow{} rI$ is 1--1. In this case, we call the imbedding of $R$ into
     $\End_R{M}$ the ring of \textbf{scalar transformations}.
\end{example}

\begin{lemma}\label{lemma_4.2.3}
    Let $R$ be a ring, and $M$ a module over $R$, and let $N \subseteq M$ a
    submodule. Then the additive quotient gorup  $\faktor{M}{N}$ is a module
    over $R$, under the action  $R \times \faktor{M}{N} \xrightarrow{}
    \faktor{M}{N}$ defined by
    \begin{equation*}
        r(x+N)=rx+N
    \end{equation*}
\end{lemma}
\begin{proof}
    Since $M$ is an Abelian group under its addition,  $N \unlhd M$, and
    $\faktor{M}{N}$ is an Abelian group. Now, consider $r \in R$, and suppose
    that  $x+N=y+N$, for some  $x,y \in N$. Then  $x-y \in N$; since $N$ is a
    submodule. Then $r(x-y)=rx-ry \in N$, which makes $rx+N=ry+N$. Therefore the
    action  $R \times \faktor{M}{N} \xrightarrow{} \faktor{M}{N}$ given by
    $r(x+N) \xrightarrow{} rx+N$ is well defined.

    Now, since $M$ is an $R$-module, we get that
    \begin{align*}
        (r+s)(x+N)  &=  (r+s)x+N    \\
                    &=  (rx+sx)+N   \\
    \end{align*}
    That
    \begin{align*}
        (rs)(x+N)   &=  (rs)x+N \\
                    &=  r(sx)+N \\
                    &= r(s(x+N))    \\
    \end{align*}
    that
    \begin{align*}
        r((x+N)+(y+N))  &=  r((x+y)+N)  \\
                        &=  r(x+y)+N    \\
                        &=  (rx+ry)+N   \\
    \end{align*}
    This makes $\faktor{M}{N}$ into an $R$-module.
\end{proof}
\begin{corollary}
    The projection $\pi:M \xrightarrow{} \faktor{M}{N}$ given by $\pi:x
    \xrightarrow{} x+N$ is a module homomorphism with $\ker{\pi}=N$.
\end{corollary}
\begin{proof}
    Notice by definition that $\ph:x \xrightarrow{} x+N$ is a group homomorphism
    with $\ker{\pi}=N$. Now, let $r \in R$, ten  $\pi(rx)=rx+N=r(x+N)=r\pi(x)$.
\end{proof}

\begin{definition}
    Let $M$ be a module, and let  $A$ and $B$ submodules of  $M$. We define the
     \textbf{sum} of $A$ and  $B$ to be the set
     \begin{equation*}
         A+B=\{a+b : a \in A \text{ and } b \in B\}
     \end{equation*}
\end{definition}

\begin{lemma}\label{lemma_4.2.4}
    Let $M$ be a module, and  $A$ and  $B$ submodules. Then  $A+B$ is a
    submodule of  $M$.
\end{lemma}

\begin{theorem}[The First Isomorphism Theorem]\label{4.2.5}
    Let $R$ be a ring, and  $M$,  $N$ be modules over  $R$ and let  $\phi:M
    \xrightarrow{} N$ a module homomorphism over $R$. Then  $\ker{\phi}$ and
    $\phi(M)$ are submodules of $M$, and
    \begin{equation*}
        \faktor{M}{\ker{\phi}} \simeq \phi(M)
    \end{equation*}
\end{theorem}

\begin{theorem}[The Second Isomorphism Theorem]\label{4.2.6}
    Let $R$ be a ring and $M$ a module over $M$. Let  $A$ and  $B$ be submodules
    of  $M$. Then
    \begin{equation*}
        \faktor{(A+B)}{B} \simeq \faktor{A}{(A \cap B)}
    \end{equation*}
\end{theorem}

\begin{theorem}[The Third Isomorphism Theorem]\label{4.2.7}
    Let $M$ be a module, and  $A$ and  $B$ submodules of  $M$. Then
    \begin{equation*}
       \faktor{(\faktor{M}{A})}{(\faktor{A}{B})} \simeq \faktor{M}{B}
    \end{equation*}
\end{theorem}

\begin{theorem}[The Fourth Isomorphism Theorem]\label{4.2.8}
    Let $R$ be a ring,  $M$ be a module over  $R$, and  $N$ be a submodule of
    $M$. Then there exists a 1--1 correspondence of submodules of  $M$
    containing  $N$, onto  submodules of $\faktor{M}{N}$ given by the map
    \begin{equation*}
        A \xrightarrow{} \faktor{A}{N}
    \end{equation*}
    for all $N \subseteq A$.
\end{theorem}
