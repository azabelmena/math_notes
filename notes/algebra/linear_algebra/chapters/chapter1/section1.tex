%----------------------------------------------------------------------------------------
%	SECTION 1.1
%----------------------------------------------------------------------------------------

\section{Linear Equations.}
\label{section1}

\begin{definition}
    Let $F$ be a field with  $x_1, \dots, x_n \in F$. let $y_1, \dots, y_m, A_{ij} \in F$ for $1
    \leq i \leq m$ and  $1 \leq j \leq n$. We call a set of equations a  \textbf{system of} $m$
     \textbf{linear equations in} $n$ \textbf{variables} if it has the form:
        \begin{equation}
            \begin{align}
                A_{11}x_1+\dots+A_{1n}x_n &= y_1 \\
                        \vdots             \\
                A_{m1}x_1+\dots+A_{mn}x_n &= y_m \\
            \end{align}
        \end{equation} 
        We call an $n$-tuple  $(a_1, \dots, a_n)$ a \textbf{solution} to the system if it satisfies
        the set of equations in (1.1). If $y_1=\dots=\y_n=0$, then we call the system
        \textbf{homogeneous}.
\end{definition}

\begin{example}
    Consider the following homogeneous system of $2$ linear equations in  $3$ variables.
        \begin{equation*}
            \begin{align*}
                2x_1-x_2+x_3    &=  0 \\
                x_1+3x_2+4x_3   &=  0\\
            \end{align*}
        \end{equation*}
    adding $-2$ times the first equation to the second we get $-7x_2-7x_2=0$, so $x_2=-x_3$. Adding $3$ times the
    first equation to the second, we get $7x_1+7x_3=0$, so $x_1=-x_2$. SO $(-x_3,-x_3,x_3)$ is a
    solution to the system. In fact, the system has as solutions all triples of the form
    $(-a,-,a,a)$.
\end{example} 

\begin{definition}
    Given a system of linear equations of the form (1.1), we call the equation:
        \begin{equation}
            (c_1A_{11}+\dots+c_1A_{m1})+\dots+(c_1A_{1n}+\dots+c_1A_{mn})=c_1y_1+\dots+c_my_m
        \end{equation}
    a \textbf{linear combination} of equations of the system.
\end{definition}

\begin{definition}
    If
        \begin{equation}
            \begin{align}
                B_{11}x_1+ \dots+B_{1n}x_n &= z_1 \\
                \vdots & \\
                B_{k1}x_1+ \dots+B_{kn}x_n &= z_k \\
            \end{align}
        \end{equation}
        is a system of $k$ linear equations in  $n$ variables, we says the systems described by
        (1.1) and (1.3) are \textbf{equivalent} if each equation in each system is a linear
        combination of equations in the other system.
\end{definition}

\begin{theorem}\label{1.1.1}
    Equivalent systems of linear equations have exactly the same solutions.
\end{theorem}
\begin{proof}
    Let 
    \begin{equation}\tag{1}
            \begin{align}
                A_{11}x_1+\dots+A_{1n}x_n &= y_1 \\
                        \vdots             \\
                A_{m1}x_1+\dots+A_{mn}x_n &= y_m \\
            \end{align}
    \end{equation}
    and
        \begin{equation}\tag{2}
            \begin{align}
                B_{11}x_1+ \dots+B_{1n}x_n &= z_1 \\
                \vdots & \\
                B_{k1}x_1+ \dots+B_{kn}x_n &= z_k \\
            \end{align}
        \end{equation}
        be equivalent systems of linear equations. let $(a_1, \dots, a_n)$ be a solution to (1).
        Then $A_{i1}a_1+\dots+A_{in}a_n =
        y_i=(c_1B_{11}+\dots+c_kB_{k1})a_1+\dots+(c_1B_{1n}+\dots+c_kB_{kn})a_n=c_1z_1+\dots+c_kz_k$.
        Which means that $(a_1, \dots, a_n)$ is also a solution to $(2)$.
\end{proof}
