\section{The Nilradical and Jacobson Radical}

\begin{theorem}[The Binomial Theorem]\label{1.8.1}
    Let $A$ be a commutative ring with identity. Then for all  $x,y \in A$, and
     $n \in \Z^+$
     \begin{equation*}
         (x+y)^n=\sum_{r+s=n}{{n \choose r}x^ry^s}
     \end{equation*}
\end{theorem}

\begin{lemma}\label{1.8.2}
    Let $A$ be a commutative ring with identity, and $\Rf$ the set of all
    nilpotent elements of $A$. Then $\Rf$ is an ideal of $A$.
\end{lemma}
\begin{proof}
    If $x \in \Rf$, then there is an  $n \in \Z^+$ for which  $x^n=0$, notice
    that this also implies that  $(-x)^n=0$, so that $-x \in \Rf$. Now, let $x,y
    \in \Rf$. Then for some $m,n \in \Z^+$, we have $x^m=0$ and $y^n=0$.
    Consider then $(x+y)^{m+n-1}$. By the binomial theorem, we have
     \begin{equation*}
         (x+y)^n=\sum_{r+s=m+n-1}{{m+n-1 \choose r}x^ry^s}
     \end{equation*}
     Now, since $r+s=m+n-1$, notice that either  $r<m$, or  $s<n$, but not both.
     This makes  $x^ry^s=0$ for all  $r,s$, so that  $(x+y)^{m+n-1}=0$. This
     makes $\Rf$ a subgroup of  $A$. Lastly, notice that if  $a \in A$, and  $x
     \in \Rf$, then for some  $n \in \Z^+$,  $ax^n=(ax)^n=0$, which makes $\Rf$
     an ideal.
\end{proof}
\begin{corollary}
    $\faktor{A}{\Rf}$ has no nonzero nilpotent elements.
\end{corollary}
\begin{proof}
    Let $x \in A$ be nilpotent, then $x \in \Rf$, so that $x+\Rf=\Rf$ in
    $\faktor{A}{\Rf}$. Therefore, the only nilpotent element of
    $\faktor{A}{\Rf}$ is $\Rf$ itself.
\end{proof}

\begin{definition}
    We define the \textbf{nilradical} of a commutative ring $A$, with identity,
    to be the ideal, $\Nil{A}$, consisting of all nilpotent elements of $A$.
\end{definition}

\begin{lemma}\label{5.1.3}
    Let $A$ be a commutative ring with identity. Then $\Nil{A}$ is the
    intersection of all prime ideals of $A$; i.e.
    \begin{equation*}
        \Nil{A}=\bigcap_{\pf \subseteq A}{\pf} \text{ where } \pf \text{ is a
        prime ideal of } A
    \end{equation*}
\end{lemma}
\begin{proof}
    Let $\Rf$ be the intersection of all prime ideals of $A$. Suppose that  $x
    \in A$ is nilpotent. Then $x^n=0$ for some $n \in \Z^+$, so that  $x^n \in
    \pf$. Since $\pf$ is prime, $x \in \pf$, which puts $x \in \Rf$.

    Conversely, suppose that  $x \in A$ is not nilpotent, and let $\Sigma$ be
    the set of all ideals $\af$ for which $x \notin \af$. Notice that since $0$
    is nilpotent in $A$, $0 \in \Sigma$, so that $\Sigma$ is nonempty.
    Therefore, by Zorn's lemma, $\Sigma$ has a maximal element $\pf$. We claim
    that this $\pf$ is prime. Suppose that  $a,b \notin \pf$, then  $\pf
    \subseteq \pf+(a)$ and $\pf \not\subseteq \pf+(b)$. So $\pf+(a), \pf+(b)
    \notin \Sigma$, by the maximality of $\pf$. This puts  $x^m \in \pf+(a)$ and
    $x^n \in \pf+(b)$, for some $n,m \in \Z^+$. Thus  $x^{m+n} \in \pf+(ab)
    \not\subseteq \Sigma$. Therefore, $ab \notin \pf$, which makes $\pf$ a prime
    ideal for which $x \notin \pf$; i.e. $x \notin \Rf$.
\end{proof}

\begin{definition}
    We define the \textbf{Jacobson radical} of a commutative ring $A$, with
    identity, to be the intersection of all maximal ideals of $A$. We denote it
    by  $\Jac{A}$.
\end{definition}

\begin{lemma}\label{1.8.4}
    Let $A$ be a commutative ring with identity. Then  $x \in \Jac{A}$ if, and
    only if $1-xy$ is a unit in $A$ for some $y \in A$.
\end{lemma}
\begin{proof}
    Suppose that $x \in \Jac{A}$, but that $1-xy$ is not a unit of $A$. Then by
    lemma \ref{1.4.2}, we have $(1-xy) \subseteq \mf$ for some maximal ideal
    $\mf$ of $A$; hence $1-xy \in \mf$. However, since $x \in \Jac{A}$, then $xy
    \in \mf$, which puts $1 \in \mf$, so that $\mf=(1)$ which contradicts that
    $\mf$ is maximal. Therefore,  $1-xy$ must be a unit.

    Conversely, suppose that  $x \in \mf$ for some maximal ideal $\mf$ of $A$.
    THen  $(\mf,x)=(1)$ so that $u+xy=1$ for some  $u \in \mf$ and  $y \in A$.
    This makes  $1-xy \in \mf$ so that $1-\mf$ is not a unit.
\end{proof}
