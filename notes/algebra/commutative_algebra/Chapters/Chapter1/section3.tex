\section{Ring Homomorphisms and Factor Rings}

\begin{definition}
    Let $A$ and  $B$ be commutative rings with identity. We call a map
    $\phi:A \xrightarrow{} B$ a \textbf{ring homomorphism} if
    \begin{enumerate}
        \item[(1)] $\phi$ is a group homomorphism with respect to addition.

        \item[(2)] $\phi(ab)=\phi(a)\phi(b)$ for any $a,b \in A$.

        \item[(3)] $\phi(1_A)=1_B$.
    \end{enumerate}
    We denote the \textbf{kernel} of $\phi$ to be the kernel of $\phi$ as a
    group homomorphism. That is
    \begin{equation*}
        \ker{\phi}=\{r \in A : \phi(r)=0\}
    \end{equation*}
    Moreover, if $\phi$ is 1--1 and onto, we call $\phi$ an \textbf{isomorphism}
    and say that $A$ and $B$ are \textbf{isomorphic}, and write $A \simeq B$.
\end{definition}

\begin{lemma}\label{1.3.1}
    Let $A$ and $B$ be commutative rings with identity, and $\phi:A \xrightarrow{}
    B$ a ring homomorphism. Then
    \begin{enumerate}
        \item[(1)] $\phi(A)$ is a subring of $B$.

        \item[(2)] $\ker{\phi}$ is a subring of $A$.
    \end{enumerate}
\end{lemma}
\begin{proof}
    Let $s_1,s_2 \in \phi(A)$. Then $s_1=\phi(r_1)$ and $s_2=\phi(r_2)$ for some
    $r_1,r_2 \in A$. Then $s_1s_2=\phi(r_1)\phi(r_2)=\phi(r_1r_2) \in \phi(B)$.
    Additionally, $\inv{s}=\inv{\phi}(r)=\phi(\inv{r})$ for some $s \in B$,  $r
    \in A$. This is sufficient to make  $B$ a subring of  $B$.

    By similar reasoning, if $r_1,r_2 \in \ker{\phi}$, then
    $\phi(r_1)\phi(r_2)=\phi(r_1r_2)=0$ so that $r_1r_2 \in \ker{\phi}$, and
    $\phi(\inv{r})=\inv{\phi}(r)=0$ so $\inv{\phi}\in \ker{\phi}$.
\end{proof}
\begin{corollary}
    For any $r \in A$ and  $a \in \ker{\phi}$, then $ar \in \ker{\phi}$ and $ra
    \in \ker{\phi}$.
\end{corollary}
\begin{proof}
    We have $\phi(ar)=\phi(a)\phi(r)=\phi(a)0=0$ so $ar \in \ker{\phi}$. The
    same happens for $ra$.
\end{proof}

\begin{definition}
    Let $A$ be a comutative ring with identity. We call a subset $\af$ of $A$ an
    \textbf{ideal} of $A$ if it is a subgroup under $+$, and for any $r \in A$,
    and  $a \in \af$,  $ra \in \af$.
\end{definition}

\begin{theorem}\label{1.3.3}
    Let $A$ be a commutative ring with identity, and $I\af$ an ideal in $A$.
    Let  $\faktor{A}{\af}$ be the set
    of all $a+\af$ with  $a \in A$. Define operations $+$ and $\cdot$ by
    \begin{align*}
        (a+\af)+(b+\af) &=  (a+b)+\af \\
        (a+\af)(b+\af)  &=  ab+\af    \\
    \end{align*}
    Then $\faktor{A}{\af}$ forms a commutative ring with identity under  $+$ and
    $\cdot$.
\end{theorem}
\begin{proof}
    Notice that $(a+\af)+(b+\af)=(a+b)+(\af+\af)=(a+b)+2\af=(a+b)+\af$. Moreover,
    $\faktor{A}{\af}$ inherits associativity in $+$ from addition in $A$. Now,
    take $0+\af=\af$ as the additive identity and  $-a+I$ as the inverse of
    $a+\af$ for each $\af$.

    Now, notice, that $(a+\af)(b+\af)=ab+a\af+b\af+\af^2=ab+(\af+\af+\af)=
    ab+\af$ by distribution of multiplication over addition in $A$. Moreover,
    $\faktor{A}{\af}$ also inherits associativity and commutativity in $\cdot$
    from multiplication in $A$. Now, notice then
    \begin{equation*}
        (a+\af)((b+\af)+c+\af)=(a+\af)((b+c)+\af)=a(b+c)+\af=(ab+ac)+\af=
        (ac+\af)+(bc+a\f)
    \end{equation*}
    Observe also that if $1$ is the identity of $A$, then $1+\af$ is the
    identity of  $\faktor{A}{\af}$ as $a+$. Since $(a+\af)(1+\af)=a+\af$.

    Lastly, notice that $a+\af$ is just the left coset of  $a$ by  $\af$ in $A$ as a
    group under addition. So that $+$ and  $\cdot$ are coset addition and
    multiplication, which are well defined.
\end{proof}

\begin{definition}
    Let $A$ be a commutative ring with idenity and $\af$ an ideal in $A$. We
    call the ring $\faktor{A}{\af}$ under addition and muiltplication of cosets
    the \textbf{factor ring} (or \textbf{quotient ring}) of $A$ over  $\af$.
\end{definition}

\begin{theorem}[The First Isomorphism Theorem]\label{1.3.4}
    If $\phi:A \xrightarrow{} B$ is a ring homomorphism from rings $A$ into $B$,
    then $\ker{\phi}$ is an ideal of $A$ and
    \begin{equation*}
        \phi(A) \simeq \faktor{A}{\ker{\phi}}
    \end{equation*}
    \[\begin{tikzcd}
        A &&& B \\
        \\
        \\
        {\faktor{A}{\ker{\phi}}}
        \arrow["\pi"', from=1-1, to=4-1]
        \arrow["\phi", from=1-1, to=1-4]
        \arrow["{\bar{\phi}}"', from=4-1, to=1-4]
    \end{tikzcd}\]
\end{theorem}
\begin{proof}
    By the first isomorphism theorem for groups, $\phi$ is a group isomorphism.
    Now, let $K=\ker{\phi}$ and consider the map $\pi:A \xrightarrow{}
    \faktor{A}{\af}$ by $a \xrightarrow{\pi} a+K$. Define the map
    $\bar{\phi}:\faktor{A}{K} \xrightarrow{} \phi(A)$ such that $\bar{\phi}
    \circ \pi=\phi$, then $\bar{\phi}$ defines the ring isomorphism.
\end{proof}
\begin{proof}
    The map $\pi:A \xrightarrow{} \faktor{A}{\af}$ defined by $a \xrightarrow{}
    a+\af$, for any ideal $\af$, is onto, with $\ker{\pi}=\af$.
\end{proof}

\begin{theorem}[The Second Isomorphism Theorem]\label{1.3.5}
    Let $\af \subseteq A$ a subring of  $A$, and let $\bf$ an ideal in $A$. Define
     $\af+\bf=\{a+b : a \in \af \text{ and } b \in \bf\}$. Then
     $\af+\bf \subeteq A$ is a subring and  $\af \cap \bf$ is an ideal in $A$.
     Then
     \begin{equation*}
         \faktor{\af\bf}{\bf} \simeq \faktor{\af}{\af \cap \bf}
     \end{equation*}
\end{theorem}

\begin{theorem}[The Third Isomorphism Theorem]\label{1.3.6}
    Let $\af$ and  $\bf$ be ideals in a ring  $A$, with  $\af \subseteq \bf$. Then
    $\faktor{\bf}{\af}$ is an ideal of $\faktor{A}{\af}$ and
    \begin{equation*}
        \faktor{A}{\bf}=\faktor{(\faktor{A}{\af})}{(\faktor{\bf}{\af})}
    \end{equation*}
\end{theorem}

\begin{theorem}[The Fourth Isomorphism Theorem]\label{1.3.7}
    Let $\af$ an ideal in a ring $A$, then the correspondence between  $A$ and
    $\faktor{A}{\af}$, for any subring of $A$ is an inclusion preserving
    bijection between subrings of $A$ containing  $\af$ and  $\faktor{A}{\af}$.
    Moreover, $A$ is an ideal if, and only if  $\faktor{A}{\af}$ is an ideal.
\end{theorem}

\begin{lemma}\label{1.3.8}
    Let $A$ be a ring with ideals  $\af$ and $\bf$. Then  $\af+\bf$, $\af\bf$ and
    $\af^n$, for any  $n \geq 0$ are ideals of  $A$ and we have the lattice
    \[\begin{tikzcd}
	& A \\
	& {\af+\bf} \\
	\af && \bf \\
	& \af\bf \\
	& {(0)}
	\arrow[no head, from=1-2, to=2-2]
	\arrow[no head, from=2-2, to=3-1]
	\arrow[no head, from=3-1, to=4-2]
	\arrow[no head, from=4-2, to=3-3]
	\arrow[no head, from=3-3, to=2-2]
	\arrow[no head, from=4-2, to=5-2]
    \end{tikzcd}\]
\end{lemma}
