\section{Unique Factorization Domains.}

\begin{definition}
    Let $A$ be an integral domain. A nonzero element  $r \in A$ that is not an
    associate is called \textbf{irreducible} if whenever $r=ab$, then either $a$
    or $b$ are units in $A$; otherwise, we call $r$ \textbf{reducible}.
\end{definition}

\begin{definition}
    Let $A$ be an integral domain. An element $p \in A$ is called \textbf{prime}
    if the ideal $(p)$ is a prime ideal. That is $p$ is not a unit and whenever
     $p|ab$, then either  $p|a$ or $p|b$. We call two elements $a,b \in A$
     \textbf{associates} if $a=ub$ for some unit $u \in A$.
\end{definition}

\begin{lemma}\label{1.7.1}
    In an integral domain, a prime element is always irreducible.
\end{lemma}
\begin{proof}
    Let $(p)$ be a nonzero prime ideal with $p=ab$, for some  $a,b \in A$. Then
     $ab \in (p)$, so that either $a \in (p)$, or $b \in (p)$. Suppose that $a
     \in (p)$. Then $a=pr$ for some  $r \in A$, so that  $p=(pr)b=p(rb)$, so
     that $rb=1$. This makes $b$ a unit. Similarly, we see that $a$ is a unit if
      $b \in (p)$. In either case, $p$ is irreducible.
\end{proof}

\begin{example}\label{2.6}
    \begin{enumerate}
        \item[(1)] In the ring $\Z$ of integers, those elements which are
            irreducible are precisely those which are prime, since the ideals
            $2\Z, 3\Z, \dots, p\Z, \dots$, for  $p$ a prime number are also the
            prime ideals of $\Z$

        \item[(2)] Irreducible elements need not be prime. The element $3 \in
            \Z[\sqrt{-5}]$ is irreducible, as was shown in example \ref{2.2},
            however it is not prime. Notice that
            $3|9=(2+\sqrt{-5})(2-\sqrt{-5})$, but $3 \nmid (2+\sqrt{-5})$ and
            $3 \nmid (2-\sqrt{-5})$.
    \end{enumerate}
\end{example}

\begin{lemma}\label{1.7.2}
    In a principle ideal domain, a nonzero element is prime if, and only if it
    is irreducible.
\end{lemma}
\begin{proof}
    Let $A$ be a PID, and suppose that $p$ is irreducible. Let $(m)$ be the
    principle ideal containing $(p)$, then $p=rm$, and by irreducibility, either
     $r$ or $m$ are units, in either case, we get that either $(p)=(m)$ or
     $(m)=(1)$. This makes $(p)$ a maximal ideal, and hence a prime ideal.
\end{proof}

\begin{example}\label{2.7}
    \begin{enumerate}
        \item[(1)] Since $3$ is not prime in  $\Z[\sqrt{-5}]$, then $(3)$ is not
            a prime ideal in this ring. Therefore $\Z[\sqrt{-5}]$ cannot be a
            PID.

        \item[(2)] Notice that since $\Z$ is a PID, then the fact that
            irreducible and prime elements coincide is guaranteed by lemma
            \ref{1.7.2}.
    \end{enumerate}
\end{example}

\begin{definition}
    We call an integral domain $A$ a  \textbf{unique factorization domain (UFD)}
    if for every nonzero element $r \in A$ which is not a unit, the following
    are true.
    \begin{enumerate}
        \item[(1)] $r$ can be written as the product of, not necessarily distinct,
            irreducible elements. We call this product the
            \textbf{factorization} of $r$.

        \item[(2)] The factorization of $r$ is unique up to associates.
    \end{enumerate}
\end{definition}

\begin{example}\label{2.8}
    \begin{enumerate}
        \item[(1)] All fields are unique factorization domains.

        \item[(2)] Polynomial rings are unique factorization domains whenever
            the ground ring $A$ is a unique factorization domain.

        \item[(3)] The subring $\Z[2i]$ of $\Z[i]$ is an integral domain, but it
            is not a UFD. Notice that both $2$ and  $2i$ are irreducible in
            $\Z[2i]$, but that $4=2 \cdot 2=(2i) \cdot (-2i)$.

        \item[(4)] $\Z[\sqrt{-5}]$ is another example of an integral domain that
            is not a UFD.
    \end{enumerate}
\end{example}

\begin{lemma}\label{1.7.3}
    In a unique factorization domain $A$, a nonzero element is prime if, and
    only if it is irreducible.
\end{lemma}
\begin{proof}
    Since prime elements are irreducible, it remains to show that irreducible
    elements are prime. Let $p$ be irreducible and suppose that  $p|ab$, for
    $a,b \in A$. Then  $ab=pc$ for some  $c \in A$. Writing  $ab$ as a product
    of irreducibles, since  $A$ is a UFD,  $p$ must be associate to one of the
    irreducibles in the factorization of $a$, or to one in the factorization of
$b$. In either case, we get that $p|a$ or $p|b$, and hence  $p$ is prime.
\end{proof}

\begin{lemma}\label{1.7.4}
    Let $a,b \in A$ nonzero elements of a unique factorization domain $A$. If
    $a=up_1^{e_1} \dots p_n^{e_n}$ and $b=vp_1^{f_1} \dots p_n^{f_n}$, where
    $u,v \in A$ are units, then the element
    \begin{equation*}
        d=p_1^{\min\{e_1,f_1\}} \dots p_n^{\min\{e_n,f_n\}}
    \end{equation*}
    os the greatest common divisor of $a$ and  $b$.
\end{lemma}
\begin{proof}
    Notice that by definition of $d$, that  $d|a$ and  $d|b$. Now, let  $c$ be a
    common divisor of  $a$ and  $b$ with the unique prime factorization
    $c=q_1^{g_1} \dots q_m^{g_m}$. Since $q_i|c$ for each  $1 \leq i \leq m$,
    then  $q_i|p_j$ for each prime factor in the factorizations of  $a$ and
    $b$. Since both  $q_i$ and  $p_j$ are irreducible, they are associates. That
    implies that the primes of  $c$ are the primes of  $a$ and  $b$. Moreover
    notice that since each $g_i \leq e_i,f_i$, that  $c|d$, and so  $d=(a,b)$.
\end{proof}

\begin{definition}
    Let $A$ be a principle ideal domain. Let $\{a_n\}$ a sequence of elements of
    $A$. We call the increasing sequence of ideals  $\{(a_n)\}$ an
    \textbf{infinite ascending chain} of ideals in $A$ and write
    \begin{equation*}
        (a_1) \subseteq (a_2) \subseteq \dots \subseteq (a_n) \subseteq \dots
        \subseteq A
    \end{equation*}
    We say that the infinite ascending chain $\{(a_n)\}$ \textbf{stabalizes} if
    for some $k \geq n$, we have  $(a_n)=(a_k)$.
\end{definition}

\begin{lemma}\label{1.7.5}
    In any principle ideal domian, infinite ascending chains of ideals
    stabalize.
\end{lemma}
\begin{proof}
    Let $\af_1 \subseteq \af_2 \subseteq \dots \subseteq A$ an infinite ascending
    chain of ideals and let $\af=\bigcup{\af_k}$. Then $\af$ is an ideal in $A$,
    and since  $A$ is a PID,  $\af=(a)$ for some $a \in A$. This makes  $a \in
    \af_n$ for some  $n$, and hence  $\af_n \subseteq \af$. This makes
    $\af_n=\af$ for some $n \geq 1$, and hence this chain stabalizes.
\end{proof}

\begin{theorem}\label{1.7.6}
    Every principle ideal domain is a unique factorization domain.
\end{theorem}
\begin{proof}
    Let $A$ be a PID, and  $r \in A$ a nonzero element which is not a unit. If
    $r$ is irreducible, we are done. Otherwise, we have  $r=r_1r_2$ fr some
    $r_1,r_2 \in A$. Now, if both $r_1$ and $r_2$ are irreducible, we are done.
    Suppose then, without loss of generality, thart $r_1$ is reducible. Then
    $r_1=r_{11}r_{12}$, and if both $r_{11}$ and $r_{12}$ are irreducible, we
    are done. Suppose then that $r_{11}$ is reducible; continuing this process,
    we arrive at an infinite ascending chain of ideals
    \begin{equation*}
        (r) \subseteq (r_1) \subseteq (r_{11}) \subseteq \dots \subseteq A
    \end{equation*}
    and since $A$ is a PID, this chain stabalizes. Thus $r$ can be factored into
    irreducible elements; since this process terminates.

    Now, by induction on $n$, for  $n=0$, we notice that  $r$ is a unit, and we
    are done. Suppose, then for  $n \geq 1$, that  $r=p_1 \dots p_n=q_1 \dots
    q_m$ for some $m \geq n$, and where each  $p_i$ and  $q_j$ are  (not
    necessarily distinct) irreducibles for all $1 \leq i \leq n$ and $1 \leq j
    \leq m$. Notice that  $p_1|q_1 \dots q_m$, and so $p_1|q_j$ for some $j$.
    This makes  $p_1$ and $q_j$ associates; i.e.  $q_j=p_1u$, with $u \in A$ a
    unit. Cancelling the  $p_1$ from both sides of the equation, we get $p_2
    \dots p_n=q_1 \dots q_{j-1}q_{j+1} \dots q_m$. Repeating this process, we
    get a 1--1 correspondence between associates, and hence the factorization of
    $r$ is unique up to associates. Therefore  $A$ is a UFD.
\end{proof}
\begin{corollary}
    Every Euclidean domain is a unique factorization domain.
\end{corollary}
\begin{proof}
    Notice that Euclidean domains are PIDs by lemma \ref{2.1.1}.
\end{proof}
\begin{corollary}[The Fundamental Theorem of Arithmetic]
    $\Z$ is a unique factorization domain.
\end{corollary}
\begin{proof}
    Notce that $\Z$ is a Euclidean domain.
\end{proof}
\begin{corollary}
    There exists a multiplicative Dedekind-Hasse norm on $A$.
\end{corollary}
\begin{proof}
    If $A$ is a PID, then the theorem tells us it is a UFD. Define the norm $N:A
    \xrightarrow{} \N$ by taking $0 \xrightarrow{} 0$, $u \xrightarrow{} 1$ if
    $u$ is a unit, and  $a \xrightarrow{} 2^n$ where $a=p_1 \dots p_n$, where
    each $p_i$ is irreducible. Notice that for every  $a,b \in A$,
    $N(ab)=N(a)N(b)$. Now, suppose further that $a,b \neq 0$ and consider the
    ideal  $(a,b)=(r)$, for some $r \in A$. UIf  $a \notin (b)$, nether is $r$,
    and hence  $b \nmid r$. Now, since  $b=xr$,  $x \in A$, then  $x$ cannot be
    a unit in  $A$, so that  $N(b)=N(xr)=N(x)N(r)>N(r)$. This completes the
    proof.
\end{proof}
