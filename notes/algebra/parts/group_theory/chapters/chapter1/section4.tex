\section{The General and Special Linear Groups of $n \times n$ Matrices}
\label{section_1.4}

One special class of groups are those that can be defined on matrices. We first
need to define, in an elementary sense what a ``field'' is; though, we
only need their (na\"ive) definition here. We assume familiarity with matrix
algebra such as matrix multiplication and determinants.

\begin{definition}
  Let $F$ be a set together with binary operations  $+$, called
  \textbf{addition} and $\cdot$, called  \textbf{multiplication}. We call
  $(F,+,\cdot)$ a \textbf{field} if:
  \begin{enumerate}
    \item[(1)] $(F,+)$ forms an abelian group.

    \item[(2)] $(F^*,\cdot)$ forms an abelian group; where
      $F^*=\com{F}{\{e\}}$, $e$ is the identity of  $F$ under  $+$.

    \item[(3)] $\cdot$  \textbf{distributes} over $+$; that is, for  $a,b,c
      \in F$,  $a(b+c)=ab+ac$.
  \end{enumerate}
\end{definition}

\begin{example}\label{example_1.9}
  \begin{enumerate}
    \item[(1)] The sets $\Q$ and  $\R$ are fields under the usual addition
      and multiplication.

    \item[(2)]  $\C$ is a field under complex addition and complex
      multiplication. So is  $\R$ if we take all $a \in \R$ to have the
      form  $a+i0$.

    \item[(3)] $\faktor{\Z}{p\Z}$, with $p \in \Z^+$ prime forms a field
      under addition and multiplication modulo $p$.
  \end{enumerate}
\end{example}

\begin{definition}
  Let $F$ be a field. We define  $F^{n \times n}$ to be the field of all $n
  \times n$ matrices with entries in $F$. We define the \textbf{general linear
  group} to be $GL(n,F)=\{A \in F^{n \times n} : \det{A} \neq 0\}$. We define
  the \textbf{special linear group} to be $\SL{(n,F)}=\{A \in F^{n \times n} :
  \det{A}=1\}$. If $F=\faktor{\Z}{p\Z}$, we write
  $\GL{(n,\faktor{\Z}{p\Z})}=\GL{(n,p)}$ and
  $\SL{(n,\faktor{\Z}{p\Z})}=\SL{(n,p)}$.
\end{definition}

\begin{theorem}\label{theorem_1.4.1}
  For any field $F$, and  $n \in \Z^+$, $\GL{(n,F)}$ forms a group under matrix
  multiplication.
\end{theorem}
\begin{proof}
  Let $A,B \in \GL{(n,F)}$ be $n \times n$ matrices. Then  $\det{A} \neq 0$ and
  $\det{B} \neq 0$, so $det{AB}=\det{A}\det{B} \neq 0$, by a well known
  property of determinants. So $\GL{(n,F)}$ is closed. Now since matric
  multiplication is associative, then $\GL{(n,F)}$ satisfies the associative
  law.

  Now consider the $n \times n$ identity matrix $I$, we have for any $A \in
  \GL{(n,F)}$, $AI=IA=A$, moreover,  $\det{I}=1 \neq 0$ making $I \in
  \GL{(n,F)}$. Likewise, since for $A \in \GL{(n,F)}$, $\det{A} \neq 0$, $A$ is
  invertible, by well known properties of matrices, so  $\inv{A}$ exists, and
  $\det{\inv{A}}=\det{A} \neq 0$. Thus $\inv{A} \in \GL{(n,F)}$ and since
  $A\inv{A}=\inv{A}A=I$, this makes $\inv{A}$ the inverse of $A$.
\end{proof}
\begin{corollary}
  $\SL{(n,F)}$ forms a group under matrix multiplication.
\end{corollary}
\begin{proof}
  Notice that $\SL{(n,F)} \subseteq \GL(n,F)$, so $\SL{(n,F)}$ inherits closure (and
  associativity). Now, for $A \in \SL{(n,F)}$, $A \in \GL{(n,F)}$, so $\inv{A}$
  exists. Moreover, $\det{\inv{A}}=\det{A}=1$, making $\inv{A} \in \SL{(n,F)}$.
  This also implies that $I \in \SL{(n,F)}$.
\end{proof}

\begin{example}\label{example_1.10}
  \begin{enumerate}
    \item[(1)]
      \begin{equation*}
        \GL{(2,2)}=\Big{\{}
          \begin{pmatrix}
            1   &   0   \\
            0   &   1   \\
          \end{pmatrix},
          \begin{pmatrix}
            0   &   1   \\
            1   &   0   \\
          \end{pmatrix},
          \begin{pmatrix}
            1   &   1   \\
            0   &   1   \\
          \end{pmatrix},
          \begin{pmatrix}
            1   &   1   \\
            1   &   0   \\
          \end{pmatrix},
          \begin{pmatrix}
            0   &   1   \\
            1   &   1   \\
          \end{pmatrix},
          \begin{pmatrix}
            1   &   0   \\
            1   &   1   \\
          \end{pmatrix}
        \Big{\}}
      \end{equation*}
      Labeling these elements as $I$,  $A$,  $B$,  $C$,  $D$, and  $E$,
      consecutively we find the orders to be:  $|I|=1$, $|A|=2$,
      $|B|=2$, $|C|=3$, $|D|=3$, and $|E|=2$.

    \item[(2)] Consider $\GL{(2,2)}$, and considering the labeling of the
      above  matrix, we compute the Cayeley table to be:
      \begin{equation*}
        \begin{tabular}{c | c c c c c c}
      & I   &   A   &   B   &   C   &   D   &   E   \\
      \hline
          I & I & A & B & C & D & E \\
          A & A & I & D & E & B & C \\
          B & B & C & I & A & E & D \\
          C & C & B & E & D & I & A \\
          D & D & E & I & I & C & I \\
          E & E & D &C  & B & A & I \\
        \end{tabular}
      \end{equation*}
      which is not symmetirc, hence $\GL{(2,2)}$ is not Abelian. In general, for $n,
      p \in \Z^+$ and $p$ prime, for  $A,B \in \GL{(n,p)}$ we have
      \begin{equation*}
        \begin{pmatrix}
          A   &   0   \\
          0   &   1   \\
        \end{pmatrix}
        \begin{pmatrix}
          B   &   0   \\
          0   &   1   \\
        \end{pmatrix}=
        \begin{pmatrix}
          AB  &   0   \\
          0   &   1   \\
        \end{pmatrix}
      \end{equation*}
      while
      \begin{equation*}
        \begin{pmatrix}
          B   &   0   \\
          0   &   1   \\
        \end{pmatrix}
        \begin{pmatrix}
          A   &   0   \\
          0   &   1   \\
        \end{pmatrix}=
        \begin{pmatrix}
          BA  &   0   \\
          0   &   1   \\
        \end{pmatrix}
      \end{equation*} then
      \begin{equation*}
        \begin{pmatrix}
          AB  &   0   \\
          0   &   1   \\
        \end{pmatrix}=
        \begin{pmatrix}
          BA  &   0   \\
          0   &   1   \\
        \end{pmatrix}
      \end{equation*}
      if, and only if $AB=BA$, which is in general, not true for matrices. So
      $\GL{(n,p)}$ is not necesarrily Abelian.
  \end{enumerate}
\end{example}

Now, we would like to observe the order of the group $\GL{(n,p)}$, the order of
$\SL{(n,p)}$ will be derived later.

\begin{example}
  \begin{enumerate}
    \item[(1)] We have that if $F$ is a field with  $|F|=p$, then
      $|\GL(n,F)|<p^{n^2}$, for, notice for any $A \in F^{n \times n}$,
      there are $n^2$ entries, and  $p$ choices for each entry, thus
      $|F^{n \times n}|=n^2$, now, by definition, $\GL{(n,F)}$ excludes
      those with $\det=0$, thus we get the result.

    \item[(2)] Let $A \in \GL{(2,2)}$ where:
      \begin{equation*}
        A=\begin{pmatrix}
          a   &   b   \\
          c   &   d   \\
        \end{pmatrix}
      \end{equation*}
      where $a,b,c,d \in \faktor{\Z}{2\Z}$. Then we have that if $ad-bc
      \neq 0$, then  $ad \neq bc$, thus $a$ is a multiple of  $c$ and  $d$
      is a multiple of  $b$, let us consider the columns. We have that if
      $a=c=0$, then  $\det{A}=0$, thus $a$ and  $c$ cannot be both  $0$,
      also notice that there are $2^2$ possible choices for  $a$
      and  $c$, so the fist column, $\begin{pmatrix}a \\ c
      \\\end{pmatrix}$, hase $2^2-1$ possible choices. Now, obvserving
      column $\begin{pmatrix}b \\ d \\\end{pmatrix}$, we have the $2^2$
      choices for both entries, however, since  $b$ and  $d$ are
      multiples of eachother, we must exclude the $2$ choices for the
      multiples $ad$ and  $bc$. Thus the column $\begin{pmatrix}b \\ d
      \\\end{pmatrix}$ has $2^2-2$ choices. That is,
      $\GL{(2,2)}=(2^2-1)(2^2-2)=2 \cdot 3=6$.
  \end{enumerate}
\end{example}

Observing further, we can see that $|\GL{(n,3)}|=(3^n-1) \dots (3^n-3)$, and so
on. Thus we have:

\begin{theorem}\label{theorem_1.4.2}
  For $n, p \in \Z^+$ and  $p$ prime :
  \begin{equation}
    |\GL{(n,p)}|=\prod_{j=1}^{n-1}{(p^n-p^{n-j})}
  \end{equation}
\end{theorem}
\begin{proof}
  Consider the $n \times n$ matrix $A=(a_{ij}) \in \GL{(n,p)}$.
  Observe that there are $p^n-1$ choices for the first column $
  \begin{pmatrix}
    a_{11}  \\ \vdots  \\ a_{n1} \\
  \end{pmatrix}
  $ since not all $a_{11}, \dots, a_{n1}$ can be zero. Now by
  definition of determinant we have
  \begin{equation*}
    \det{A}=\sum_{l=1}^{j-1}{(-1)^{i+j}a_{il}\det{A_{il}}}
  \end{equation*}
  where $A_{il}$ is the cofactor of $A$ about the entry  $a_{ij}$. So
  for the $j$-th column, we have $
  \begin{pmatrix}
    a_{1j}  \\  \vdots  \\  a_{nj}  \\
  \end{pmatrix}
  $
  then
  \begin{equation*}
    a_{ij}\det{A_{ij}}=\sum_{l=1}^{j-1}{(-1)^{i+j}a_{il}\det{A_{il}}}+
  \sum_{l=1}^{j+1}{(-1)^{i+j}a_{il}\det{A_{il}}}
  \end{equation*}
  for which there are $p^n-p^j$ choices, given that each of the
  $a_{ij}$ entries are multiples of the previous entries, for $1 \leq
  i \leq n$. Taking $2 \leq j \leq n$ (since we have already evaluated
  the first column), we get:
  \begin{equation*}
    \prod_{j=1}^{n-1}{p^n-p^{n-j}}
  \end{equation*}
  choices for every enty in the matrix $A$, and since $A$ is
  arbitrary, this gives us the order of $\GL{(n,p)}$.
\end{proof}

We also need to comment on the order of $\GL(n,F)$ when the field $F$ is
infinite.

\begin{theorem}\label{theorem_1.4.3}
  For any field $F$,  $\GL(n,F)$ is of infinite order if, and only if $F$ is
  of infinite order.
\end{theorem}
\begin{proof}
  We show by contrapositives. Suppose that $F$ is finite with  $|F|=k$.
  Then by the same argument of theorem \ref{theorem_1.4.2}, we find there are
  $\prod{(k^n-k^j)}$ matrices $A \in \GL(n,F)$. Any additional elements
  contradict this result, and so $|GL(n,F)|=\prod{(k^n-k^j)}$.

  On the otherhand, if $|\GL(n,F)|=k$ then there are $k$ $n \times n$
  matrices over  $F$ with $\det \neq 0$. Now, if  $F$ were not finite, then
  there exists a distrinct matrix  $A \in \GL(n,F)$, making $|\GL(n,F)|=k+1$
  a contradiciton. Thus, $F$ must be finite.
\end{proof}

We now introduce a seperate group from the general and special linear groups.

\begin{definition}
  Let $F$ be a field. We define the \textbf{Heisenberg} group over $F$ to be
  the set:
  \begin{equation}
    H(F)= \Big{\{}
      \begin{pmatrix}
        1   &   a   &   b   \\
        0   &   1   &   c   \\
        0   &   0   &   1   \\
      \end{pmatrix} :
      a,b,c \in F
    \Big{\}}
  \end{equation}
  That is, $H(F)$ is the set of all upper triangular matrices over $F$ with
  diagonal entries equal to $1$  (the identity element of $F$).
\end{definition}

\begin{lemma}\label{lemma_1.4.4}
  For any field $F$,  $H(F)$ is a group under matrix multiplication.
\end{lemma}
\begin{proof}
  Let $X=\begin{pmatrix}
    1   &   a   &   b   \\
    0   &   1   &   c   \\
    0   &   0   &   1   \\
  \end{pmatrix}$, and
  $Y=\begin{pmatrix}
    1   &   d   &   e   \\
    0   &   1   &   f   \\
    0   &   0   &   1   \\
  \end{pmatrix}$. Then
  $XY=\begin{pmatrix}
    1   &  a+d  &   e+af+b  \\
    0   &   1   &   f+c     \\
    0   &   0   &   1   \\
  \end{pmatrix}$.
  So $H(F)$ is closed. Additionally, $H(F)$ inherits the associativity of
  matrix multiplication.

  Now, we get that $I=\begin{pmatrix}
    1    &   0   &   0   \\
    0    &   1   &   0   \\
    0    &   0   &   1   \\
  \end{pmatrix}$
  serves as the identity, and the matrix $Y=\begin{pmatrix}
    1   &  -a   &   ac-b    \\
    0   &   1   &   -c      \\
    0   &   0   &   1       \\
  \end{pmatrix}$
  serves as an inverse to $X$. This makes  $H(F)$ into a group.
\end{proof}
\begin{corollary}
  $H(F)$ is non-Abelian.
\end{corollary}
\begin{corollary}
  $|H(F)|=(|F|)^3$.
\end{corollary}
\begin{proof}
  Let $|F|=k$, then we have $n$ choices for  $a$,  $b$, and  $c$, hence
  $n^3$ choices for an arbitrary matrix in  $H(F)$.
\end{proof}
\begin{corollary}
  $H(F)$ is finite if, and only if $F$ is finite.
\end{corollary}
