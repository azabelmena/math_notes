\section{Permutation Groups and the Symmetric Group.}
\label{section1}

\begin{definition}
    Let $S$ be any set, we define $S(A)$ to be the \textbf{symmetric group} on
    $A$ of all permutiations from  $A$ onto itself. If  $A=\faktor{\Z}{n\Z}$ for
    $n \in \Z^+$, we write $S(A)=S_n$.
\end{definition}

\begin{theorem}\label{1.4.1}
    For any set $S$, $A(S)$ is a nonabelian group under function composition.
\end{theorem}
\begin{proof}
    For any $f,g \in A(S)$, we get that $f \circ g$ is also a permutation, so
    $f \circ g \in A(S)$; moreover, $A(S)$ is associative on account of $\circ$.

    Now, the identity map $e:i \rightarrow i$, for any $i \in S$ is the identity
    of $A(S)$, and since $f \in A(S)$ is $1-1$ and onto,  $\inv{f} \in A(S)$.
    This makes $A(S)$ a group. That $A(S)$ is nonabelian follows from the
    noncommutativity of $\circ$.
\end{proof}
\begin{corollary}
    $\ord{S_n}=n!$.
\end{corollary}
\begin{proof}
    The proof of this follows by a combinatorial argument counting the number of
    $1-1$ maps, and the number of onto maps.
\end{proof}

\begin{definition}
    We define the \textbf{cycle} of a permutation $s \in S_n$ to be a string of
    integers of  $\faktor{\Z}{n\Z}$, $(a_0 \ \dots \ a_{n-1})$ where $s:a_i
    \rightarrow a_{i+1}$, where $i \in \faktor{\Z}{n\Z}$. We call two cycles
    \textbf{disjoint} if they share no entries.
\end{definition}

\begin{lemma}\label{1.4.2}
    The cycle decomposition of a given permutation $s \in S_n$ is finite.
\end{lemma}
\begin{proof}
    Let $s=(a_0 \ \dots  \ a_{n-1})$ be the cycle decomposition of $s$. Then by
    definition, we get  $s:a_{n-1} \rightarrow a_{(n-1)+1}=a_n=a_0$.
\end{proof}

\begin{example}
    The cycle $(2 \ 0 \ 1)$ represents the permutation $s:2 \rightarrow 0
    \rightarrow 1$ in $S_3$.
\end{example}

Now, since cycles of permutations of $S_n$ are finite, we can define the
``lenght'' of a cycle.

\begin{definition}
    The \textbf{length} of a cycle of a permutation $s \in S_n$ is the number of
    entries in the cycle. We call a cycle of length  $k \in \Z^+$ a
    \textbf{$k$-cycle}.
\end{definition}

\begin{lemma}\label{1.4.3}
    For any permutation $s \in S_n$, the elements of  $\faktor{\Z}{n\Z}$ can be
    grouped into $k$ cycles of the form:
    \begin{equation}
        (a_0 \ a_2 \ \dots \ a_{m_1})(a_{m_1+1} \ \dots \ a_{m_2}) \dots
        (a_{m_{k-1}} \ \dots \ a_m_k)
    \end{equation}
\end{lemma}
\begin{proof}
    Since $s$ is  $1-1$ and onto,  $s$ will permute through the entirety of
    $\faktor{\Z}{n\Z}$; so every integer $\mod{n}$ will be represented in the
    cycle for $s$.

    Now, find  $x \in \faktor{\Z}{n\Z}$ in a cycle for $s$. If  $x$ is not at
    the end of the cycle, i.e. if  $s(x)$ is not some previous element of cycle
    of $x$, then $s(x)$ is next integer in the cycle of $x$. Otherwise,  $s(x)$
    is the first integer of another cycle of $s$, i.e. if  $x=a_{m_i}$, then
    $s(x)=a_{m_i+1}$. There are $k$ such possible cycles for  $s$, where  $k \in
    \Z^+$.
\end{proof}

\begin{definition}
    For any permutation $s \in S_n$, we  call the concatenation of all cycles
    of $s$ the  \textbf{cycle decomposition} of $s$.
\end{definition}

We introduce a neat algorithm for finding the cycle decomposition of a
permutation.

\begin{algorithm}[The Cycle Decomposition Algorithm]
    Let $s \in S_n$ be a permutation of the elements of  $\faktor{\Z}{n\Z}$.
    \begin{enumerate}
        \item[\textbf{step 1:}] Choose the smallest $i \in \faktor{\Z}{n\Z}$
            which has not appeared in a previous cycle; if there is no previous
            cycle, $i=0$. Start the cycle at $i$.

        \item[\textbf{step 2:}] Compute $s(i)$. If $s(i)=i$, close the cycle,
            and return to \textbf{step 1}. Else, concatenate $s(i)$ to $i$ in
            the cycle.

        \item[\textbf{step 3:}] Repeat  \textbf{step 2} with $s(i)$.

        \item[\textbf{step 4:}] If $\faktor{\Z}{n\Z}$ has been exhausted, go to
            \textbf{step 5}; else return to \textbf{step 3}.

        \item[\textbf{step 5:}] Remove all $1$-cycles and stop.
    \end{enumerate}
\end{algorithm}
\begin{remark}
A neat excersise immediately introduces itself as the problem of how to program
this cycle decomposition algorithm so that one can simply feed a permutation
into a computer and get its cycle decomposition as an output.
\end{remark}

\begin{example}
    Define the permutation $s \in S_{13}$ by:
    \begin{align*}
        s:1 \rightarrow 12 && s:2 \rightarrow 0 && s: 3 \rightarrow 3 \\
        s:4 \rightarrow 1 && s:5 \rightarrow 11 && s:6 \rightarrow 9 \\
        s:7 \rightarrow 5 && s:8 \rightarrow 10 && s:9 \rightarrow 6 \\
        s: 10 \rightarrow 4 && s:11 \rightarrow 7 && s:12 \rightarrow 8 \\
        s:0 \rightarrow 2 \\
    \end{align*}
    Using the cycle decomposition algortithm we get:
    \begin{equation*}
        s=(0 \ 2)(1 \ 12 \ 8 \ 10 \ 4)(5 \ 11 \ 7)(6 \ 9)
    \end{equation*}
\end{example}

The cycle decomposition algorthm provides a neat way of finding the cycle
decomposition $s$; what about  $\inv{s}$, when we are given $s$?

\begin{lemma}\label{1.4.4}
    Let $s \in S_n$ be a permutation with the cycle decomposition  $s=(a_0 \
    a_2 \ \dots \ a_{m_1})(a_{m_1+1} \ \dots \\ a_{m_2}) \dots (a_{m_{k-1}} \
    \dots \ a_m_k)$. Then $\inv{s}$ has the cycle decomposition:
    \begin{equation}
        \inv{s}=(a_{m_1}  \dots \ a_2 \ \ a_0)(a_{m_2} \ \dots \ a_{m_1+1}) \dots
        (a_m_k \ \dots \ a_{m_{k-1}})
    \end{equation}
\end{lemma}
\begin{proof}
    If $s:a_{m_i} \rightarrow a_{m_i+1}$, then $\inv{s}:a_{m_i+1} \rightarrow
    a_{m_i}$. Then by the cycle decomposition algorithm we can derive equation
    $(1.3)$.
\end{proof}

We finally come wish to introduce the ``product'' of cycles.

\begin{definition}
    Let $s,t \in S_n$ be a permutation with a cycle decompositions defined by
    the rules  $s:a_{m_i} \rightarrow a_{m_i+1}$ and $t:b_{m_i} \rightarrow
    b_{m_i+1}$. Then we define the \textbf{product} of cycle decompositions,
    $\circ$, to be: $s \circ t$ whose cycle decomposition is defined by the rule
     $s \circ t:b_{m_i} \rightarrow b_{m_i+1} \rightarrow a_{m_j}$, where
     $a_{m_j}=s(b_{m_i+1})$ and $b_{m_i+1}=t(b_{m_i})$. We define the
     concatenation of cycles to be the product of cycle decompositions.
\end{definition}

\begin{example}
    Consider the cycles $(1 \2 \ 3)$ and $(1 \ 2)(3 \ 0)$ in $S_4$. Then  $(1 \2
    \ 3) \circ (1 \ 2)(3 \ 0)=(1 \ 3 \ 0)$.
\end{example}

We can now refphrase theorem \ref{1.4.1} as:

\begin{theorem}\label{1.4.5}
    Define $S_n$ to be the set of all cycle decompositions of permutations of
    the elements of  $\faktor{\Z}{n\Z}$. Then $S_n$ is a group under cycle
    products.
\end{theorem}
\begin{corollary}
    $S_n$ is nonabelian for  $n \geq 3$.
\end{corollary}
\begin{corollary}
    Disjoint cycles commute.
\end{corollary}

\begin{lemma}\label{1.4.6}
    Let $s \in S_n$ be a $k$-cycle. Then $\ord{s}=k$.
\end{lemma}
\begin{proof}
    We have that $s=(a_1 \ a_2 \ \dots \ a_k)$, thus $s$ maps  $a_i
    \rightarrow a_{i+1}$, for $i \in \faktor{\Z}{k\Z}$; also notice that
    $s:a_{k-1} \rightarrow a_0$. Then $s^k:a_i \rightarrow a_{i+k
    \mod{k}}=a_i$, for all  $i$, thus  $s^k=(1)$. Moreover, if $m \in \Z^+$
    such that $s^m:a_i \rightarrow a_{i+m \mod{k}}$, then $a_i=a_j$ for
    some other  $j \in \faktor{\Z}{n\Z}$, which implies that $s$ is a
    $k-1$-cycle, which cannot happen. Thus $\ord{s}=k$.
\end{proof}
\begin{corollary}
    If $s,t \in S_n$ are  $k$ and $m$-cycles, respectively, then
    $\ord{st}=[k,m]$.
\end{corollary}
\begin{corollary}
    Let $s \in S_n$, then the cycle composition of  $s$ is a product of disjoint
     $m_k$ cycls where $k \in \faktor{\Z}{n\Z}$.
\end{corollary}
