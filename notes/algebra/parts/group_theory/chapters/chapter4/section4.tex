\section{Automorphisms}
\label{section_4.4}

\begin{definition}
  Let $G$ be a group. An \textbf{automorphism} of $G$ is an isomorphism from
  $G$ onto itself. We denote the set of all automorphisms of $G$ as $\Aut{G}$
  and call it the \textbf{automorphism group} of $G$.
\end{definition}

\begin{proposition}\label{proposition_4.4.1}
  Let $G$ be a group, then  $\Aut{G}$ is a group.
\end{proposition}
\begin{proof}
  Let $s,t \in \Aut{G}$. Notice then that $s,t \in A(G)$.
\end{proof}

\begin{proposition}\label{proposition_4.4.2}
  Let $G$ be a group and let  $H \unlhd G$ be a normal subgroup of $G$. Then
  $G$ acts via conjugation on  $H$ as automorphisms. More preciesly, the map
  $H \xrightarrow{} H$ given by $h \xrightarrow{} gh\inv{g}$ is an
  automorphsim of $H$.
\end{proposition}
\begin{proof}
  We have tha the map $H \xrightarrow{} H$ given by $h \xrightarrow{}
  gh\inv{g}$, for some $g \in G$ is onto since  $H \unlhd G$. Moreover, we
  have that  $gh\inv{g}=gh'\inv{g}$ implies $h=h'$, by cancellation; so that
  this map is 1--1 as well. Lastly, notice that
  $ghh'\inv{g}=(gh\inv{g})(gh'\inv{g})$ which makes it an automorphism of $H$.
\end{proof}
\begin{corollary}
  The permutation representation of this action is a homomorphism of $G
  \xrightarrow{} \Aut{H}$ with kernel $C(H)$. In particular, we have that
  $\faktor{G}{C(H)}$ is isomorphic to some subgroup of $\Aut{H}$.
\end{corollary}
\begin{proof}
  Define the permutation representation $\psi:G \xrightarrow{} \Aut{H}$ given
  by $g \xrightarrow{} \phi_g$, where $\phi_g:H \xrightarrow{} H$ is the
  automorphism given by $\phi_g:h \xrightarrow{} gh\inv{g}$. Then $\psi$ is a
  homomorphism, as  $\psi_g'\psi_g=\psi_{g'g}$. Moreover, we have
  $\ker{\psi}=\{g \in G : \phi_g(h)=h\}=\{g \in G : gh\inv{g}=h\}=C(H)$.
  Lastly, by the first isomorphism theorem, we have the following diagram
  \[\begin{tikzcd}
    G &&& {\faktor{G}{C(H)}} \\
    \\
    \\
      &&& {\phi(H)}
      \arrow["\psi", from=1-1, to=1-4]
      \arrow["\phi", from=1-4, to=4-4]
      \arrow[from=1-1, to=4-4]
  \end{tikzcd}\]
  which gives us the isomorphism.
\end{proof}
\begin{corollary}
  If $K$ is a subgroup of  $G$, then  $K \simeq gK\inv{g}$ for any $g \in G$.
\end{corollary}
\begin{corollary}
  For any subgroup $H$ of  $G$, we have  $\faktor{N(H)}{C(H)}$ is isomorphic
  to a subgroup of $\Aut{H}$, in particular, $\faktor{G}{Z(G)}$ is isomorphic
  to a subgroup of $\Aut{G}$.
\end{corollary}
\begin{proof}
  Notice that $H \unlhd N(H)$, so that by above $\faktor{N(H)}{C(H)}$ is
  isomoprphic to a subgroup of $\Aut{H}$. Now take $H=G$, then $N(G)=G$ and
  $C(G)=Z(G)$.
\end{proof}

\begin{definition}
  Let $G$ be a group. We call conjugation by some $g \in G$ an \textbf{inner
  autormorphism} of $G$. We denote the set of all inner automorphisms of $G$
  by  $\Inn{G}$.
\end{definition}

\begin{proposition}\label{proposition_4.4.3}
  For any group $G$,  $\Inn{G} \leq \Aut{G}$.
\end{proposition}
\begin{corollary}
  $\Inn{G} \simeq \faktor{G}{Z(G)}$.
\end{corollary}

\begin{example}\label{example_4.10}
  \begin{enumerate}
    \item[(1)] A group $G$ is Abelian if, and only if every inner
      automorphism of  $G$ is trivial. Let  $H \unlhd G$ be Abelian, such
      that  $H \not\subseteq Z(G)$. Then there exists a $g \in G$ such
      that conjugation by $g$, restricted to $h$  (i.e. taking $\phi|_H$
      where  $\phi:x \xrightarrow{} gx\inv{g}$, $x \in G$) is not an inner
      automorphsim of $H$.

    \item[(2)] Using above, consder the alternating group $A_4$, and
      consider the Klein $4$-group $V_4$ as a subgroup of $A_4$ (writing
      the elements of $V_4$ as even permutations). Now, let $g$ be any
      $3$-cycle of  $A_4$, then conjugation by $g$ restricted to $V_4$ is
      not an inner automorphism.

    \item[(3)] $Z(\Hb)=\langle -1 \rangle$, so $\Inn{\Hb} \simeq V_4 \simeq
      \faktor{\Z}{2\Z} \times \faktor{\Z}{2\Z}$.

    \item[(4)] $Z(D_8)=\langle r^2 \rangle$ so that $\Inn{D_8} \simeq V_4
      \simeq \faktor{\Z}{2\Z} \times \faktor{\Z}{2\Z}$.

    \item[(5)] For $n \geq 3$, $Z(S_n)=\langle (1) \rangle$, so that
      $\Inn{S_n} \simeq S_n$.

    \item[(6)] If $H \simeq \faktor{\Z}{2\Z}$, since $H$ has elements of
      order $1$ and $2$, we get $\Aut{H}=\langle e \rangle$. Then by
      hypothesis, $N(H)=C(H)$. Moreover if $H \unlhd G$, then $H \leq
      Z(G)$.
  \end{enumerate}
\end{example}

\begin{definition}
  We call a subgroup $H$ of a group  $G$  \textbf{characteristic} in $G$ if
  every automorphism of $G$ restricted to  $H$ is onto. That is $s(H)=H$ for
  every $s \in \Aut{G}$. We write $H \Char{G}$
\end{definition}

\begin{proposition}\label{proposition_4.4.4}
  The following are true for any group $G$ and  $H \leq G$.
  \begin{enumerate}
    \item[(1)] If $H \Char{G}$, then $H \unlhd{G}$.

    \item[(2)] If $H$ is the unique subgroup of a given order in  $G$, then
      $H$ is characteristic in  $G$.

    \item[(3)] If $K \leq H \unlhd G$ such that  $K \Char{H}$, then $K
      \unlhd G$.
  \end{enumerate}
\end{proposition}
\begin{proof}
  \begin{enumerate}
    \item[(1)] Let $H \Char{G}$. Then for every $s \in \Aut{G}$, $s(H)=H$.
      In particular, choose $s:G \xrightarrow{} G$ taking $x
      \xrightarrow{} gx\inv{g}$. Then $gH\inv{g}=H$.

    \item[(2)] Let $H$ be the unique subgroup of order $n$ in $G$. Since
      automorphsims of $G$ preserve the order of subgroups, we have
      $|s(H)|=|H|=n$, moreover since $s$ is 1--1 and onto, we get
      $s(H)=H$, for any $s \in \Aut{G}$. Therefore $H \Char{G}$.

    \item[(3)] Lastly, let $K \Char{H}$ where $K \leq H \unlhd G$. By
      statement (1), we have our result.
  \end{enumerate}
\end{proof}

\begin{theorem}\label{proposition_4.4.5}
  The automorphism group of a cyclic group of order $n$ is isomorphic to the
  unit group $\Uc{(\faktor{\Z}{n\Z})}$.
\end{theorem}
\begin{proof}
  Let $Z_n$ be a cyclic group of order $n$, such that $Z_n=\langle x \rangle$.
  Let $\psi \in \Aut{Z_n}$, then $\psi:x \xrightarrow{} x^a$ for some $a \in
  \faktor{\Z}{n\Z}$, since $|x|=n$. Now we have that since $\psi$ is an
  automorphism, it is uniquely determinde by $a$. Let $\psi_a$ be such an
  automorphism. We also get that  $\psi_a:x \xrightarrow{} x^a$ preserves
  order so that $|x|=|x^a|=n$ and by proposition \ref{proposition_2.3.6}, $(a,n)=1$.
  Therefore we have a map $\Psi:\Aut{Z_n} \xrightarrow{} U(\faktor{\Z}{n\Z})$
  taking $\psi_a \xrightarrow{} a$. We have that this map is 1--1 by above,
  and moreover it is onto. Lastly, notice that
  $\psi_a\psi_b(x)=\psi_a(x^b)=x^{ab}=\psi_{ab}(x)$ so $\Psi$ is a
  homomorphism. Therefore we have an isomorphism of $\Aut{Z_n}$ onto
  $U(\faktor{\Z}{n\Z})$.
\end{proof}

\begin{example}\label{example_4.11}
  Let $G$ be a group of order $pq$, where $p,q \in \Z^+$ are primes such that
  $p \leq q$. Suppose that  $p \not{|} q-1$. If $Z(G) \neq \langle e
  \rangle$, then $\faktor{G}{Z(G)}$ is cyclic Lagrange's theorem, which makes
  $G$ cyclic.

  Now suppose that $Z(G)=\langle e \rangle$. Then if every nonidentity
  element of $G$ has order  $p$, by the class equation, we have that
  $pq=1+kq$ for some  $k \in \Z^+$. Now,  $q|pq$ and  $q|kq$, but  $q \not{|}
  1$; so there must be an element of $G$ of order $q$. Let $x$ be such
  element and let  $H=\langle x \rangle$. Then $[G:H]=p$ which is the
  smallest prime dividing $|G|$. By proposition \ref{proposition_4.2.2}, we get that $H
  \unlhd G$. This makes  $C(H)=H$ so that $\faktor{G}{H} \simeq
  \faktor{N(H)}{H}$ is a group of order $p$, isomorphic to a subgroup of
  $\Aut{H}$, by the corllary to proposition \ref{proposition_4.4.2}. Now, by theorem
  \ref{theorem_4.4.5}, we get that $|\Aut{H}|=\phi(q)=q-1$, where $phi$ is
  Euler's Totient function. This makes $p|q-1$ which cannot happen. Therefore
  $Z(G)$ is nontrivial and $G$ must be Abelian.
\end{example}


\begin{theorem}\label{theorem_4.4.6}
  The following are true:
  \begin{enumerate}
    \item[(1)] If $p>2$ is prime and $n \in \Z^+$, then the automorphism
      group of a cyclic group is order $\phi(p)=p-1$ where $\phi$ is
      Euler's Totient function.

    \item[(2)] For all $n \geq 3$, the automorphism group of a cyclic group
      of order  $2^n$ is isomorphic to  $\faktor{\Z}{2\Z} \times
      \faktor{\Z}{2^{n-2}\Z}$. Moreover, it has a cyclic subgroup of index
      $2$.

    \item[(3)] If $p \in \Z^+$ is prime and $V$ is an Abelian group such
      that  $v^p=0$ for all $v \in V$. If  $|V|=p^n$, then $V$ is an
      $n$-dimensional vector space over the field  $\F_p$. Moreover the
      automorphisms of $V$ are nonsingular linear transformations from $V$
      onto itself and  $\Aut{V} \simeq GL(n,\F_p)$.

    \item[(4)] For all $n \neq 6$,  $\Aut{S_n}=\Inn{S_n} \simeq S_n$. If
      $n=6$, then $[\Aut{S_6}:\Inn{S_6}]=2$

    \item[(5)] Notice that $D_8 \unlhd D_{16}$. Then $\Inn{D_{16}} \simeq
      \Aut{D_8}$ and notice that $Z(D_{16})= \langle r^4 \rangle$. So
      $\Aut{D_8} \simeq \faktor{D_{16}}{\langle r^4 \rangle} \simeq D_8$.
      Therefore $|\Aut{D_8}|=8$.

    \item[(6)] Notice that $V_4 \Char D_8$ and that $D_8 \Char S_4$, so we
      get that $V_4 \Char S_4$

    \item[(7)] Notice that $\langle r \rangle$ is the unique subgroup of
      order $n$ of  $D_{2n}$, and so $\langle r \rangle \Char D_{2n}$.
  \end{enumerate}
\end{theorem}

\begin{example}\label{example_4.12}
  Suppose that $G$ is a group of order  $45=3^25$ with a normal subgroup $P$
  of order $3^2$. We get that  $\faktor{G}{C(P)} \simeq P'$, where $P' \leq
  \Aut{P}$, and $|\Aut{P}|=6$ or $|(\Aut{P})|=48$. Since $P$ has
  order the square of a prime,  $P$ must be Abelian so taht  $P \leq C(P)$.
  Then $|C(P)|$ is divisible by $9$ so that $|\faktor{G}{C(P)}|=1$ or
  $|\faktor{G}{C(P)}|=5$. We then get that it must be the former, so that
  $C(P)=G$ and $P \leq Z(G)$. Since $\faktor{G}{Z(G)}$ is cyclic, $G$ must be
  Abelian.
\end{example}

\begin{proposition}\label{proposition_4.4.7}
  For any group $G$,  $\Inn{G} \unlhd \Aut{G}$.
\end{proposition}
\begin{proof}
  We have that $\Inn{G} \leq \Aut{G}$. Now, let $s \in \Aut{G}$, and consider
  the map $\phi_g:x \xrightarrow{} gx\inv{g}$ of conjugation by some $g \in G$.
  Then we have $s\phi_g\inv{s}(x)=s(\phi_g(\inv{s}(x)))=s(g\inv{s}(x)\inv{g})=
  s(g)x\inv{s}(g)$ So that $s(\Inn{G})\inv{s} \subseteq \Inn{G}$.
\end{proof}

\begin{definition}
  Let $G$ be a group. We call the factor group  $\faktor{\Aut{G}}{\Inn{G}}$
  the \textbf{outer automorphism group} of $G$.
\end{definition}

\begin{proposition}\label{proposition_4.4.8}
  Let $G$ be Abelian of order $pq$, where $p,q \in \Z^+$ are distinct primes.
  Then $G$ is cyclic.
\end{proposition}
\begin{proof}
  By Cauchy's theorem, we have that $p|pq$ and  $q|pq$, so there exist
  elements  $x,y \in G$ such that  $|x|=p$ and $|y|=q$. Then we have
  that $((xy)^p)^q=(xy)^{pq}=(x^p)^q(y^q)^p=e$ so that $|xy|=pq$ (since
  $p$ and $q$ are prime, $pq$ is the smalles such integer for which
  $(xy)^{pq}=e$). Then we have that $|G|=|xy|$. Therefore $G$ is
  cyclic.
\end{proof}
