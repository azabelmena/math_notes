\section{Subgroups of Generating Sets}
\label{section_2.4}

We now make precise the notion of when a subgroup of a group $G$ is generated by
a subset $A \subseteq G$. Namely, $A$ will be a set of ``generators''.

\begin{proposition}\label{proposition_2.4.1}
  If $\Hc$ is a nonempty collection of subgroups of  $G$, then the
  intersection of all elements of $\Hc$ forms a subgroup of $G$.
\end{proposition}
\begin{proof}
  Let $\Hc=\{H_i\}$ where $H_i \leq G$ for all  $i \in \Z^+$, and let
  $K=\bigcap{H_i}$. We have $e \in H_i$, for all  $i$, so  $e \in K$.
  Moreover, if  $a,b \in K$, then  $a,b \in H_i$ for all $i$, amking
  $a\inv{b} \in H_i$ for all $i$. Thus  $a\inv{b} \in K$, and so by the
  subgroup criterion, $K \leq G$.
\end{proof}

\begin{definition}
  Let $G$ be a group, and  $\{H_i\}$ the collection of all subgroups of $G$.
  For any subset  $A \subseteq G$, we define the subgroup $\langle A
  \rangle$ \textbf{generated} by $A$ to be the intersection of all
  subgroups $H_i$ containing  $A$; i.e.
  \begin{equation}
    \langle A \rangle=\bigcap_{A \subseteq H_i \leq G}{H_i}
  \end{equation}
\end{definition}

\begin{proposition}\label{proposition_2.4.2}
  Let $A$ be a nonempty subset of a group  $G$. Then  $\langle A
  \rangle \leq G$.
\end{proposition}
\begin{proof}
  Let $\Hc=\{H \leq G : A \subseteq H\}$. Then the result follows from proposition
  \ref{2.4.1}.
\end{proof}

Having $\langle A \rangle$ defined as the intersection of all subgroups
containing $A$, while a useful definition, will not help us much in enumerating
the elements of $\langle A \rangle$. For this, we turn to another subgroup, which
can allow us to enumerate its elements effectively, and prove that it is equal
to $\langle A \rangle$.

\begin{definition}
  Let $G$ be a group and let  $A \subseteq G$. We define the  \textbf{closure}
  of $A$ to be the set $\bar{A}$ of all finite products of elements of $A$.
  That is:
  \begin{equation}
    \bar{A}=\{a_1^{e_1} \dots a_n^{e_n} : n \in \Z^+, a_i \in A, e_i=\pm1
    \text{ for all } i\}
  \end{equation}
  We call the elements of $\bar{A}$ \textbf{words}, and we define
  $\bar{A}=\langle e \rangle$, if $A=\emptyset$.
\end{definition}

\begin{proposition}\label{proposition_2.4.3}
  Let $G$ be a group and  $A \subseteq G$. Then  $\bar{A}=\langle A \rangle$.
\end{proposition}
\begin{proof}
  Notice, that by definition that atleast $e \in \bar{A}$, for if $A$ is empty,
  $\bar{A}=\langle e \rangle$, and if $A$ is not, then  $a_i\inv{a_i}=e \in
  \bar{A}$. Now, if $a,b \in \bar{A}$ are words with $a=a_1^{e_1} \dots
  a_n^{e_n}$, and $b=b_1^{f_1} \dots b_m^{f_m}$, then notice that
  $\inv{b}=b_m^{-f_m} \dots b_1^{-f_1}$. So $\inv{b} \in \bar{A}$, so
  $a\inv{b} \in \bar{A}$. This makes $\bar{A} \leq G$.

  Furthermore, notice that $A \subseteq \bar{A}$, hence $\langle A \rangle
  \subseteq \bar{A}$. However, also notice that $A \subseteq \langle A
  \rangle$, and $\langle A \rangle$ is closed under the operation of $G$, so
  $\langle A \rangle$ contains all finite products of elements of $A$; i.e.
  $\bar{A} \subseteq \langle A \rangle$.
  This establishes equivalence.
\end{proof}
\begin{corollary}
  $\langle A \rangle=\{a_1^{\alpha_1} \dots a_n^{\alpha_n} : a_i \neq a_{i+1},
    n \in \Z^+, \text{ and } \alpha_i \in \Z\}$.
\end{corollary}
\begin{corollary}
  If $G$ is Abelian, then  $\langle A \rangle=\{a_1^{\alpha_1} \dots a_n^{\alpha_n} :
  \alpha_i \in \Z^\}$
\end{corollary}
\begin{corollary}
  If $G$ is Abelian, and if each $a_i$ has order $|a_i|=d_i$, then
  $|\langle A \rangle| \leq d_1 \dots d_n$.
\end{corollary}

The above corollary is not always true, and only holds for Abelian groups. For
example:

\begin{example}\label{example_2.9}
  \begin{enumerate}
    \item[(1)] Consdier $D_8=\langle r, t \rangle$. let $a=t$, and
      $b=rt$ and let $A=\{a,b\}$. Then $D_8=\langle a,b \rangle$. Notice however,
      that $|a|=|b|=2$, but that $|D_8|=8 \geq 4 = 2 \cdot 2$.
      Moreover notice that no element of $D_8$ can be represented in the
      form $a^\alpha b^\beta$. For example,  $aba=trtt=tr$ which cannot be
      written in the form $a^\alpha b^\beta$.

    \item[(2)] Consider $S_n=\langle (1 \ 2), (1 \ 2 \ \dots \ n)
      \rangle$. We have $|(1 \ 2)|=2$ and $|(1 \ 2 \ \dots \ n)|=n$,
      but $|S_n|=n!$

    \item[(3)] In $\GL(2,\R)$, let $A=\begin{pmatrix}0 & 1 \\ 1 & 0
        \\\end{pmatrix}$ and $B=\begin{pmatrix}0 & 2 \\ \frac{1}{2} & 0
      \\\end{pmatrix}$. Then $A^2=B^2=I$, but
        $AB=\begin{pmatrix} \frac{1}{2} & 0 \\ 0 & 2 \\\end{pmatrix}$,
        which has infite order. So $\langle A,B \rangle$ is a subgroup of
        infinite order, generated by elements of finite order.
    \end{enumerate}
  \end{example}
