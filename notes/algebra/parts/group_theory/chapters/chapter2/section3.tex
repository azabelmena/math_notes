\section{Cyclic Groups}
\label{section_2.3}

\begin{definition}
  Let $G$ be a group, and let  $H \leq G$ be a subgroup. We say that  $H$ is
  \textbf{cyclic} if it can be generated by a single element. That is, there
  is some $x \in H$ for which  $H=\{x^n : n \in \Z\}$. We write
  $H=\langle x \rangle$.
\end{definition}
\begin{remark}
  Notice that if $H$ is cyclic, there is a singleton set that generates  $H$,
  i.e.  $H=\langle \{x\} \rangle$.
\end{remark}

\begin{lemma}\label{lemma_2.3.1}
  If $G=\langle g \rangle$, then $G=\langle \inv{g} \rangle$.
\end{lemma}
\begin{proof}
  For any $h \in G$, if  $h=g^n$, then $\inv{h}=x^{-n}=(\inv{x})^n$.
\end{proof}

\begin{lemma}\label{lemma_2.3.2}
  If $G$ is a cyclic group, then  $G$ is Abelian.
\end{lemma}
\begin{proof}
  Let $G=\langle g \rangle$ and for any $m,n \in \Z^+$. Then
  $g^mg^n=g^{m+n}=g^{n+m}=g^mg^n$.
\end{proof}

\begin{example}
  \begin{enumerate}
    \item[(1)] In $D_{2n}$, the set $\{e,r, \dots, r^n\}$ is cyclic, and
      $\langle r \rangle \leq D_{2n}$.

    \item[(2)] $\Z= \langle 1 \rangle=\langle -1 \rangle$.

    \item[(3)] Consider the subgroup $\C_n=\{z \in \C : z^n=1\} \leq \C$ of
      complex primitive $n$-th roots of unity in $\C$, where  $n \in
      \Z^+$. Then $C=\langle z \rangle=\langle \frac{1}{z} \rangle$.

    \item[(4)] $\C_4=\langle i \rangle=\langle -i \rangle$.
  \end{enumerate}
\end{example}

\begin{lemma}\label{lemma_2.3.3}
  Let $G=\langle g \rangle$ be a finite cyclic group, Then $|G|=|g|$.
\end{lemma}
\begin{proof}
  Let $|g|=n$, by the division theorem, there exists $t,q,r \in \Z$ such
  that  $t=nq+r$, with $0 \leq r < n$ ; i.e. $ t \equiv r \mod{n}$, so
  $g^t=g^r$. Thus there are at most  $n$ such elements in  $G$. Now, suppose
  $g^i=g^j$ for  $0 \leq i < j < n$, then  $g^{j-i}=e$, since $j-i < n$, this
  contradicts the order of  $g$. Therefore, there are eaxctly $|g|=n$ such
  element in $G$, since  $G=\langle g \rangle$, these are the only elements, which
  proves the result.
\end{proof}
\begin{corollary}
  If $G=\langle g \rangle$ is a finite cyclic group, then $g$ is of finite order.
\end{corollary}
\begin{proof}
  If $g$ is of infinite order, then there are infinitely many powers of $g$,
  since $g$ generates  $G$, this contradicts the finiteness of  $G$.
\end{proof}

\begin{lemma}\label{lemma_2.3.4}
  Let $G$ be a group and let  $g \in G$ and let  $m,n \in \Z^+$ distinct, such
  that  $g^n=e$ and  $g^m=e$. Then  $g^{(m,n)}=e$. Moreover, $|g|$
  divides $m$.
\end{lemma}
\begin{proof}
  There exist $p,q \in \Z$ such that $mp+nq=(m,n)$. Then by hypothesis,
  $g^{(m,n)}=(g^m)^p(g^n)^q=e$. Morevoer, assuming, without loss of
  generality, that $m<n$, we have by definition of the order of  $g$ that
  either $|g|=m$, or $m=d|g|$ for some $d \in \Z^+$.
\end{proof}

\begin{theorem}\label{lemma_2.3.5}
  If $G$ and  $H$ are finite cyclic groups with  $|G|=|H|$, then $G
  \simeq H$.
\end{theorem}
\begin{proof}
  Let $G=\langle g \rangle$ and $H=\langle h \rangle$. Take the map
  $\phi:\langle g \rangle \rightarrow \langle h \rangle$ via the rule
  $g^k \rightarrow h^k$, for some $k \in \Z^+$. We have that
  $\phi(g^mg^n)\phi(g^{m+n})=h^{m+n}=h^mh^n=\phi(g^m)\phi(g^n)$. So $\phi$
  defines a homomorphism. Moreover, if $|G|=|H|=n$, and $g^s=g^t$ for
  $s,t \in \Z^+$,  $g^{s-t}=e$, so $n|s-t$, that is  $s \equiv t \mod{n}$.
  Thus $\phi(g^s)=\phi(g^t)$. This makes $\phi$ well defined.

  Now, by definition, every element of  $\langle h \rangle$ is of the form
  $h^k=\phi(g^k)$ for some $k \in \N$, this makes  $\phi$ onto. Since  $\phi$
  is onto, and  $|G|=|H|$ we get that $\phi$ is  $1-1$, and so an
  isomorphism.
\end{proof}
\begin{corollary}
  If $\langle g \rangle$ is an infinite cyclic group, then $\Z \simeq
  \langle g \rangle$.
\end{corollary}
\begin{proof}
  Define $\phi:\Z \rightarrow \langle g \rangle$ by $m \rightarrow g^m$. Then
  $\phi(m+n)=g^{m+n}=g^mg^n=\phi(m)\phi(n)$, moreover, if $m=n$, then
  $\phi(m)=g^m=g^n=\phi(n)$; so $\phi$ is a well defined homomorphism.

  Now if  $\phi(m)=\phi(n)$, then $g^m=g^n$, that is  $g^{m-n}=e$, so $m-n=0$,
  hence  $m=n$.  $\phi$ is  $1-1$.  $\phi$ is also onto by definitionm
  therefore  $\phi$ is an isomorphism of  $\Z$ onto $\langle g \rangle$.
\end{proof}

\begin{example}
  \begin{enumerate}
    \item[(1)] In $D_{2n}$, $\langle r \rangle \simeq
      \faktor{\Z}{n\Z}$, and $\langle t \rangle \simeq
      \faktor{\Z}{2\Z}$.

    \item[(2)] If $G$ is any finite cyclic group of order $|G|=n$,
      then $G \simeq \faktor{\Z}{n\Z}$.
  \end{enumerate}
\end{example}

\begin{lemma}\label{lemma_2.3.6}
  If $G$ is a group, and  $g \in G$, and  $k \in \Z^*$, then the following are
  true:
  \begin{enumerate}
    \item[(1)] If $g$ is of infinite order, then so is  $g^k$.

    \item[(2)] If $|g|=n$, then  $|g^k|=\frac{n}{(n,k)}$.
  \end{enumerate}
\end{lemma}
\begin{proof}
  Suppose that $g$ is of infinite order, but that $|g^k|=m$ for some $m
  \in \Z^+$. Then  $(g^k)^m=g^{km}=e$. Now, either $km>0$, or  $-km>0$, thus,
  we get  $|g| \leq km$ or $|g| \leq -km$. Both contradict the
  infinite order of $g$.

  Now let  $|g|=n$, and let $h=g^k$ and define  $d=(n,k)$. Then $n=dm$ and
  $k=dl$ for  $m,l \in \Z$,  $m>0$. Then we have $na+kb=d$, for $a,b \in \Z$;
  this implies that  $ma+lb=1$, so  $(m,l)=1$. Now, let $|h|=p$, then
  $h^m=g^{km}=g^{dlm}=(g^{dm})^l=(g^n)^l=e$, so $p|m$. On the other hand,
  since  $(m,l)=1$, we get $m|p$, thus  $p=|h|=m$. Since $m=\frac{n}{d}$,
  we get the result.
\end{proof}
\begin{corollary}
  If $k|n$, then  $|g^k|=\frac{n}{k}$.
\end{corollary}
\begin{proof}
  If $k|n$, then  $(n,k)=k$.
\end{proof}

\begin{lemma}\label{lemma_2.3.7}
  Let $\langle g \rangle$ be a cyclic group, then:
  \begin{enumerate}
    \item[(1)] If $g$ is of infinite order, then  $\langle g
      \rangle=\langle g^k \rangle$ if, and only if $k=\pm{1}$.

    \item[(2)] If $|g|=n$, then $\langle g \rangle=\langle g^k
      \rangle$ if, an only if $(n,k)=1$.
  \end{enumerate}
\end{lemma}
\begin{proof}
  First, if $k=\pm 1$, tjem  $g^k=g$ or  $g^k=\inv{g}$. By lemma \ref{2.3.1},
  we get the result. Now suppose $\langle g \rangle=\langle g^k \rangle$ for
  $k>1$. Then $g=g^k$ for some  $k>1$. If  $k$ is odd, then  $k=2l+1$ and if
  $k$ is even, $k=2l$ for  $l \in \Z^+$. Then  $g=g^{2l+1}=g^{2l}g$, making
  $g^{2l}=e$. On the otherhand, if $g=g^k=g^{2l}$, then $g^{2l}\inv{g}=g^{2l-1}=e$.
  Both these cases contradict the infinite order of $g$. So  $k \leq 1$. Now,
  since $\langle g \rangle=\langle e \rangle$ cannot happen, $k \neq 0$. Now if
  $k<-1$, then we get the same result using  $-k$. Thus either  $k=1$ or  $k=-1$.

  Now suppose that  $|g|=n$. Then $g^k$ generates a subgroup of
  $|g^k|=\frac{n}{(n,k)}$. Now $\langle g \rangle=\langle g^k \rangle$ if, and
  only if $|g|=|g^k|$. That is, if and only if  $n(n,k)=n$, i.e. if, and only
  if $(n,k)=1$.
\end{proof}
\begin{corollary}
  The number of generators of $\langle g \rangle$ is $\phi(n)$, the
  Euler-$\phi$ function.
\end{corollary}

\begin{example}
  \begin{enumerate}
    \item[(1)] Any $k \in \faktor{\Z}{n\Z}$, coprime with $n$ generates
      $\faktor{\Z}{n\Z}$. So the generators of $\faktor{\Z}{n\Z}$ are the
      elements of $\Uc{(\faktor{\Z}{n\Z})}$.

    \item[(2)] $\faktor{\Z}{6\Z}=\langle 1 \rangle=\langle 5 \rangle$.

    \item[(3)] $\faktor{\Z}{12\Z}=\langle 1 \rangle=\langle 5
      \rangle=\langle 7 \rangle=\langle 11 \rangle$.
  \end{enumerate}
\end{example}

\begin{theorem}\label{theorem_2.3.8}
  Let $G=\langle g \rangle$ be a cyclic group. The following are true:
  \begin{enumerate}
    \item[(1)] Every subgroup of $\langle g \rangle$ is cyclic, and has the form
      $\langle g^d \rangle$, with $0 \leq d < |g|$.

    \item[(2)] If $\langle g \rangle$ is infinite, then for any $m,n \in \Z^+$
      distinct,  $\langle g^m \rangle \neq \langle g^n \rangle$. Moreover,
      $\langle g^m \rangle=\langle g^{|m|} \rangle$, and there is a $1-1$ map of
      subgroups of $\langle g \rangle$ onto $\Z^+$.

    \item[(3)] If $|\langle g \rangle|=n$, then for each divisor $k$ of  $n$,
      there is a unique subgroup of order  $k$, which is  $\langle g^d
      \rangle$, where $d=\frac{n}{k}$.
  \end{enumerate}
\end{theorem}
\begin{proof}
  First, let $H \leq \langle g \rangle$, if $K=\langle e \rangle$, we are done.
  Otherwise, there is some $k \neq 0$ with  $g^k \in K$. If  $k<0$, then
  $g^{-k} \in K$. Now, define $P=\{l \in \Z^+ : g^l \in K\}$. We have by above
  that $P$ is nonempty, thus by the Well Ordering Principle,  $P$ has a
  least element, $d$. Now,  $g^d \in K$ and  $K \leq \langle g \rangle$, so
  $\langle g^d \rangle \leq K$. Now, for any $g^k \in K$, by the division
  theorem, we have  $k=qd+r$, $0 \leq r < d$, with  $q, r \in \Z$. Then
  $g^r=g^{k-qd}=g^k(g^d)^{-q}$. Since $g^k,g^d \in K$, by the minimality of
  $d$, we must have  $r=0$. So $k=qd$, this make $g^k=(g^d)^q \in
  \langle g^d \rangle$ and hence $K \leq \langle g^d \rangle$. Thus $K$ is
  cyclic. Moreover, if $|g|=n$, and if $d>n$, then by the division theorem
  $d \equiv r \mod{n}$, hence there are  $n$ subgroups of $\langle g
  \rangle$.

  Now, if $\langle g \rangle$ is infinite, and if $\langle g^m
  \rangle=\langle g^n \rangle$, then $g^m=g^n$, so $g^{m-n}=e$, implying that
  $g$ has finite order; hence $\langle g \rangle$ has finite order. This cannot
  happen, so  $\langle g^m \rangle \neq \langle g^n \rangle$. Morevoer, we get
  $\langle g^m \rangle=\langle g^{-m} \rangle$, so $\langle g^m
  \rangle=\langle g^{|m|} \rangle$.

  Now, define $\phi:m \rightarrow \langle g^m \rangle$. By above, we get $\phi$
  is $1-1$ and onto, so we have established  a  $1-1$ correspondance between
  the subgroups of  $\langle g \rangle$ onto $\Z^+$.

  Finally, let $|\langle g \rangle|=n$, and let $k|n$. Letting
  $d=\frac{n}{(n,k)}=\frac{n}{k}$, we have $\langle g^d \rangle$ is a
  subgroup of order $|\langle g^d \rangle|=k$. Now, suppose that $K$
  is any other subgroup of order  $|K|=k$. Then $K=\langle g^l
  \rangle$ for some $l \in \Z^+$ is the smallest such integer of the
  set $P$ in the above arguments. Then
  $|K|=\frac{n}{(n,l)}=\frac{n}{d}$, so $d=(n,l)$, i particular, $d|l$
  so  $g^l \in \langle g^d \rangle$. Since $|K|=k$, this makes
  $K=\langle g^d \rangle$.
\end{proof}

\begin{example}\label{example_2.8}
  \begin{enumerate}
    \item[(1)] The subgroups of $\faktor{\Z}{12\Z}=\langle 1
      \rangle=\langle 5 \rangle= \langle 7 \rangle=\langle 11 \rangle$ are:
      \[\begin{tikzcd}
        {\langle 1 \rangle} \\
        {\langle 2 \rangle=\langle 10 \rangle} \\
        {\langle 3 \rangle=\langle 9 \rangle} \\
        {\langle 4 \rangle=\langle 8 \rangle} \\
        {\langle 6 \rangle} \\
        {\langle 0 \rangle}
        \arrow[no head, from=1-1, to=2-1]
        \arrow[no head, from=2-1, to=3-1]
        \arrow[no head, from=3-1, to=4-1]
        \arrow[no head, from=4-1, to=5-1]
        \arrow[no head, from=5-1, to=6-1]
      \end{tikzcd}\]

    \item[(2)] Let $G$ be any group, and let  $g \in G$. Then
      $C(g)=C(\langle g \rangle)$, and $\langle g \rangle \leq
      N(\langle g \rangle)$.
  \end{enumerate}
\end{example}
