\section{Composition Series, Simple Groups, and Solvable Groups}
\label{section_3.5}

The bulk of this section is concerned primarily with results, and not their
proofs. As such, the majority of the theorems are preseneted without proof.

\begin{theorem}\label{theorem_3.5.1}
  Let $G$ be a finite Abelian group, and let  $p \divides |G|$ a prime. Then
  $G$ contains an element of order  $p$.
\end{theorem}
\begin{proof}
  By induction, let $|G|>1$, then there exists an $x \in G$ with  $x \neq
  e$. Now, suppose that $|G|=p$. By  Lagrange's theorem, $|x|=p$.

  Now suppose that $|G|>p$. If $p \divides |x|$, then $|x|=pn$ for some
  $n \in \Z^+$. Thus by proposition \ref{2.3.6} $|x^p|=\frac{p}{(n,p)}=p$, and
  we are done.

  Now, assume that $p \not{\divides} |x|$, and consider the cyclic
  subgroup $\langle x \rangle$. Since $\langle x \rangle$ is Abelian,
  we have that $\langle x \rangle \unlhd G$, so by Lagrange's theorem,
  we have $[G: \langle x \rangle]=\frac{|G|}{|x|}$; moreover since
  $\langle x \rangle \neq \langle e \rangle$, $[G: \langle x \rangle]
  < |G|$. Since $p \not{\divides} |x|$, we have $p \divides [G :
  \langle x \rangle]$. By hypothesis then, we have $\faktor{G}{\langle
  x \rangle}$ contains an element of order $p$,  $g\langle x \rangle$.
  Then $(g\langle x \rangle)^p=g^p\langle x \rangle=\langle e \rangle$
  so that $g^p \in \langle x \rangle$. Moreover, $\langle g^p \rangle
  \neq \langle g \rangle$. Therefore, $G$ contains an element of order
  $p$.
\end{proof}
\begin{remark}
  This theorem and its proof illustrates that one can prove certain results
  about groups using normal subgroups and quotients. However, they usually
  hinge on these structures being nontrivial. It is not always the case that a
  group will have nontrivial normal subgroups.
\end{remark}

\begin{definition}
  We call a group $G$ \textbf{simple} if its only normal subgroups are
  $\langle e \rangle$ and $G$ itself.
\end{definition}
\begin{proof}
  These simple groups consitute something analogous to prime numbers
  for $\Z$.
\end{proof}

\begin{example}\label{example_3.8}
  If $|G|=p$, then by Lagrange's theorem, $G$ is simple.
\end{example}

With a prime analog for groups, one would be motivated to try and construct
something analogous to ``prime'' factorization of groups. This motivates the
following definition.

\begin{definition}
  Let $G$ be a group and $\{H_i\}_{i=}^k$ a sequence of subgroups with
  $H_0=\langle e \rangle$, $H_k=G$, and  $H_i \leq H_{i+1}$, If $H_i \unlhd H_{i+1}$
  and $\faktor{H_{i+1}}{H_i}$ is simple for all $0 \leq i \leq k$, then we
  call the sequence  $\{H_i\}$ a \textbf{composition series}. We call the
  quotients $\faktor{H_{i+1}}{H_i}$ \textbf{composition factors} of $G$.
\end{definition}

\begin{example}\label{example_3.9}
  $D_8$ has two composition series:
  \begin{align*}
    \langle e \rangle & \unlhd \langle t \rangle \unlhd \langle t, r^2
    \rangle \unlhd D_8  \\
    \langle e \rangle & \unlhd \langle r^2 \rangle \unlhd \langle r
    \rangle \unlhd D_8  \\
  \end{align*}
\end{example}

\begin{theorem}[Jordan-H\"older]\label{theorem_3.5.2}
  Let $G$ be a nontrivial finite group. Then:
  \begin{enumerate}
    \item[(1)] $G$ has a composition series.

    \item[(2)] The composition factors of  $G$ are uniquely determined. That
      is, if $\langle e \rangle=N_0 \leq N_1 \leq \dots \leq N_k=G$ and
      $\langle e \rangle=M_0 \leq M_1 \leq \dots \leq M_s=G$ are composition
      series of $G$, then  $r=s$ and for some permutation  $\pi \in S_r$,
      the quotients:
      \begin{equation*}
        \faktor{M_{\pi(i)+1}}{M_{\pi(i)}} \simeq \faktor{N_{i+1}}{N_i}
      \end{equation*}
  \end{enumerate}
\end{theorem}
\begin{remark}
  This theorem serves as the group theoretic analog for the fundamental theorem of
  arithmetic. In enssence, every group has a unique ``prime'' factorizations.
  The uniqueness of the factorization comes from the fact that the composition
  factors turn out to be isomorphic.
\end{remark}

The study of simple groups, particulalry, finite simple groups motivates the
following problem: to classify all finite simple groups. Another problem
motivated by the study of simple groups is to construct other groups from simple
groups.

\begin{theorem}\label{theorem_3.5.3}
  There are $18$ inifinite families of simple groups and $26$ simple groups
  not belonging to any one family, such that every finite simple group is
  isomorphic to one of the groups in the families.
\end{theorem}
\begin{remark}
  We call these $26$ groups the \textbf{sporadic} simple groups.
\end{remark}

\begin{example}\label{example_3.10}
  \begin{enumerate}
    \item[(1)] One of the $18$ families is the collection
      $\{\faktor{\Z}{p\Z} : p \text{ is prime}\}$.

    \item[(2)] Another familiy is the collection
      $\{\faktor{\SL(n,\F_q)}{Z(\SL(n, \F_q))} : n \in \Z^+, n \geq 2,\F_q \text{
      a finite field} \}$
  \end{enumerate}
\end{example}

\begin{theorem}[Feit-Thompson]\label{theorem_3.5.4}
  If $G$ is a simple group of odd order, then $G \simeq \faktor{\Z}{p\Z}$ for
  some prime $p$.
\end{theorem}

Likewise, in addition to simplicity of groups, we also have the notion of
solvability. Solvable groups play a role in the theory of polynomials and in
certain geometric constructions.

\begin{definition}
  A group $G$ is \textbf{solvable} if there exists a sequence of normal
  subgroups $\langle e \rangle=G_0 \unlhd G_1 \unlhd \dots \unlhd G_k=G$ such that
  $\faktor{G_{i+1}}{G_i}$ is Abelian for each $0 \leq i \leq k-1$.
\end{definition}

\begin{theorem}\label{theorem_3.5.5}
  A finite group is solvable if and only if for every $n$ dividing
  $|G|$, with $(n,\frac{|G|}{n})=1$, $G$ has a subgroup of order $n$.
\end{theorem}

We conclude with a proposition.

\begin{proposition}\label{proposition_3.5.6}
  Let $G$ be a group and  $N \unlhd G$. If  $N$ and  $\faktor{G}{N}$ are
  solvable, then so is $G$.
\end{proposition}
\begin{proof}
  Let $\bar{G}=\faktor{G}{N}$, and let $\langle e \rangle=N_0 \unlhd N_1
  \unlhd \dots \unlhd N_n=N$ a sequence of normal subgroups with
  $\faktor{N_{i+1}}{N_i}$ Abelian. Let $\langle e \rangle=\bar{G_0} \unlhd
  \bar{G_1} \unldh \dots \unlhd \bar{G_m}=\bar{G}$ be another such sequence.
  By the fourth isomporphism theorem, there are subgroups $G_i \leq G$ with
  $N \leq G_i$ such that $\faktor{G_i}{N}=\bar{G_i}$ and $G_i \unlhd G_{i+1}$.
  Then by the third isomorphism theorem, we have
  \begin{equation*}
    \faktor{\bar{G_{i+1}}}{\bar{G_i}}=\faktor{G_{i+1}/N}{G_i/N} \simeq
    \faktor{G_{i+1}}{G_i}
  \end{equation*}
  Thus, $\langle e \rangle=N_0 \unlhd N_1 \unlhd \dots \unlhd N_n=N=G_0 \unlhd G_1
  \unlhd \dots \unlhd G_m=G$ is a sequence of subgroups whose subsequent
  quotients are Abelian. That is, $G$ is solvable.
\end{proof}
