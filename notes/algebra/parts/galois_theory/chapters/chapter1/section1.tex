\section{Definitions and Examples.}

\begin{definition}
    An isomorphism of a field $K$ onto itself is called an
    \textbf{automorphism}. We denote the set of all automorphisms of $K$
    $\Aut{K}$, and for $\s \in \Aut{K}$,  we write $\s\a$ to mean  $\s(\a)$. We
    say an automorphism $\s$ of  $K$  \textbf{fixes} an element $\a \in K$ if
    $\s\a=\a$. We say  $\s$  \textbf{fixes} a subset $F \subseteq K$ if
    $\s\a=\a$ for all  $\a \in F$. We denote  $\Aut{\faktor{K}{F}}$ to be the
    set of all automorphisms of $K$ that fix $F$, where $\faktor{K}{F}$ is a
    field extension.
\end{definition}

\begin{proposition}\label{2.1.1}
    Let $K$ be a field. Then  $\Aut{K}$ is a group. Moreover, if $K$ is an
    extension of a field $F$, then $\Aut{\faktor{K}{F}} \leq \Aut{K}$.
\end{proposition}

\begin{proposition}\label{2.1.2}
    Let $K$ be an extension of  $F$, and let  $\a \in K$ algebraic over  $F$.
    Then for every  $\s \in \Aut{\faktor{K}{F}}$, $\s\a$ is a root of the
    minimal polynomial of  $\a$ over  $F$; that is,  $\Aut{\faktor{K}{F}}$
    permutes the roots of irreducible polynomials.
\end{proposition}
\begin{proof}
    Suppose that $a_0+a_1\a+\dots+a_{n-1}\a^{n-1}+\a^n=0$, where $a_i \in F$
    for all  $1 \leq i \leq n$, and  $a_n=1$. Notice that if $\s$ is an
    automorphism of $K$, then it is a homomorphism, moreover, since $\s$ fixes
    $F$, and $a_i \in F$, we get $ \s(a_i\a^i)=a_i\s\a^i$. Therefore,
    \begin{align*}
        \s(a_0+a_1\a+\dots+a_{n-1}\a^{n-1}+\a^n)  &=  \s0=0   \\
        \s(a_0)+\s(a_1\a)+\dots+\s(a_{n-1}\a^{n-1})+\s(\a^n)  &=  0   \\
        a_0+a_1\s\a+\dots+a_{n-1}\s\a^{n-1}+\s\a^n  &=  0   \\
    \end{align*}
    which makes $\s\a$ a root.
\end{proof}

\begin{example}\label{example_2.1}
    \begin{enumerate}
        \item[(1)] The identity map is an automorphism called the
            \textbf{trivial automorphsim}, and just maps elements of a field
            onto themselves. We denote this automorphism by $\i$. Notice
            additionally, that if $\s$ is an automorphism of a field, then
            $\s:1 \xrightarrow{} 1$ and $\s:0 \xrightarrow{} 0$, so that $\s
            a=a$ for any element  $a$ in the prime subfield. That is, the
            automorphism group of a field fixes its prime subfield. In
            partucular, notice that $\Q$ and  $\F_p$ have only the trivial
            automorphism, so that  $\Aut{\Q}=\langle \i \rangle$ and
            $\Aut{\F_p}=\langle \i \rangle$.

        \item[(2)] If $\t \in
            \Aut{\Q(\sqrt{2})}=\Aut{\faktor{\Q(\sqrt{2})}{\Q}}$, then
            $\t\sqrt{{2}}=\pm\sqrt[]{2}$. Then $\t$ fixes $\Q$, and we have
            that it sends elements $\t:a+b\sqrt{2} \xrightarrow{} a \pm
            b\sqrt{2}$. In the case of addition, we have that $\t=\i$ the
            identity. The latter case of subtraction gives  $\t=a+b\sqrt{2}
            \xrightarrow{} a-b\sqrt{2}$, so that
            $\Aut{\Q(\sqrt{2})}=\langle \t \rangle$ a cyclic group of order
            $2$ generated by  $\t$.
    \end{enumerate}
\end{example}

\begin{proposition}\label{2.1.3}
    Let $H \leq \Aut{K}$ for some field $K$. Then the collection  $F$ of
    elements of  $K$ fixed by $H$ is a subfield of $K$.
\end{proposition}
\begin{proof}
    LEt $h \in H$, and $a,b \in F$. Then $ha=a$,  $hb=b$, so that  $h(a \pm b)=a
    \pm b$, and $h(ab)=ab$, and $h(\inv{a})=\inv{(ha)}=\inv{a}$.
\end{proof}

\begin{definition}
    Let $K$ be a field. If $H \leq \Aut{K}$, we define the \textbf{fixed field}
    of $H$ to be the subfield of $K$ fixed by  $H$, and we denote it  $\Fc(H)$.
\end{definition}

\begin{proposition}\label{2.1.4}
    If $F_1 \subseteq F_2 \subseteq K$ are subfields of a field $K$, then
    $\Aut{\faktor{K}{F_2}} \subseteq \Aut{\faktor{K}{F_1}}$. Moreover,
    if $H_1 \leq H_2 \leq \Aut{K}$, then $\Fc(H_2) \subseteq \Fc(H_1)$.
\end{proposition}

\begin{example}\label{example_2.2}
    \begin{enumerate}
        \item[(1)] The fixed field of $\Aut{\Q(\sqrt{2})}$ is the field
            \begin{equation*}
                F=\{a+b\sqrt{2} \in \Q(\sqrt{2}) :
                \s(a+b\sqrt{2})=a+b\sqrt{2}\}
            \end{equation*}
            by definition. Then $a-b\sqrt{2}=a+b\sqrt{2}$ so that $b=0$. Therefore
            $F=\Q$ and  $\Q$ is the fixed field.

        \item [(2)] Consider $\Aut{\Q(\sqrt[3]{2})}=\langle \i \rangle$. Then
            the fixed field of $\Aut{\faktor{\Q(\sqrt[3]{2})}{\Q}}$ is
            $\Q(\sqrt[3]{2})$.
    \end{enumerate}
\end{example}

\begin{proposition}\label{2.1.5}
    Let $E$ be the splitting field over a field $F$ of a polynomial $f(x)$ over
    $F$. Then
    \begin{equation*}
        \ord{\Aut{\faktor{E}{F}}} \leq [E:F]
    \end{equation*}
\end{proposition}
\begin{proof}
    By induction on $[E:F]$. If $[E:F]=1$, then $E=F$, and we are done. Now, for
     $[E:F] \geq 1$, $f(x)$ has atleast one irreducible factor $p(x)$ of degree
     $\deg{p}>1$. Now, let $F'$ be the corresponding field to $F$ with splitting
     field $E'$, corresponding to  $E$. Let  $f'(x)$ be the polynomial over $F'$
     the polynomial corresponding to $f$ over $F$, with irreducible factor
     $p'(x)$ corresponding to the irreducible factor $p$. Now, let  $\a$ be a
     root of  $p$. IF  $\s$ is an extension of an isomorphism $\phi$ to $E$,
     then the restriction $\t=\s|_{F(\a)}$ is an isomorphism of $F(\a)$ onto a
     subfield of $E'$. Since $\a$ generates $F(\a)$, $\t$ is completely
     determined by its action on  $\a$; i.e.  $\t\a$, so that  $\t\a$ is a root
     of  $p'$. We then get the following diagram:
     \[\begin{tikzcd}
        E & {E'} \\
        {F(\alpha)} & {F'(\tau\alpha)} \\ F & {F'}
        \arrow["\sigma", from=1-1, to=1-2]
        \arrow["\tau", from=2-1, to=2-2]
        \arrow["\phi", from=3-1, to=3-2]
        \arrow[no head, from=1-1, to=2-1]
        \arrow[no head, from=2-1, to=3-1]
        \arrow[no head, from=1-2, to=2-2]
        \arrow[no head, from=2-2,to=3-2]
    \end{tikzcd}\]

    COnversly, let  $\b$ be a root of  $p'$. Then there exist extensions  $\t$
    and  $\s$ of the isomorphism  $\phi$ giving the above diagram  (replace
    $\t\a$ with  $\b$). Now, the number of extensions of $\phi$ to  $\t$ is
    equal to the number of distinct roots of  $p'$. Since
    $\deg{p}=\deg{p'}=[F(\a):F]$, the number of extensions to $\t$ is at most
    $[F(\a):F]$.

    Now, notice that $[E:F(\a)]<[E:F]$. Therefore, by the induction hypothesis,
    the number of extensions of $\t$ to  $\s$ is at most $[E:F(\a)]$. Therefore,
    the number of extensions of $\phi$ to  $\s$ is at most
    $[E:F(\a)][F(\a):F]=[E:F]$.

    Finally, if $F=F'$, we have  $f=f'$ (and $p=p'$), and so $\phi$ is the
    identity map and  $E=E'$. THis makes $\s$ an automorphism of $E$ which fixes
     $F$. The proof is complete.
\end{proof}
\begin{corollary}
    If $K$ is the splitting field of a seperable polynomial $f(x)$ over a field
    $F$, then $\ord{\Aut{\faktor{K}{F}}}=[K:F]$.
\end{corollary}

\begin{definition}
    We call a finite field extension $\faktor{K}{F}$ a \textbf{Galois extension}
    if $\ord{\Aut{\faktor{K}{F}}}=[K:F]$. We call $\Aut{\faktor{K}{F}}$ the
    \textbf{Galois group} of $\faktor{K}{F}$, and write $\Gal{\faktor{K}{F}}$.
\end{definition}

\begin{proposition}\label{2.1.6}
    An extension $K$ over a field $F$ is Galois over if and only if it is normal
    and seperable.
\end{proposition}
\begin{proof}
    If $K$ is Galois over $F$, the result follws by definition. Now, let $K$ be
    normal and seperable. Let $\a \in K$. Then the minimal polynomial $m$ of
    $\a$ over $F$ is seperable. Moreover, $\a$ is a root of $K$, and since $K$
    is normal, $m$ splits completely over $K$. Thus $K$ contains the splitting
    field of $m$, but since $m$ is minimal and irreducible, that makes  $K$ the
    splitting field of some polynomial $f$ over $F$, having $m$ as a factor. By
    the above corollary, this makes $K$ Galois over  $F$.
\end{proof}

\begin{example}\label{example_2.3}
    \begin{enumerate}
        \item[(1)] $\faktor{\Q(\sqrt{2})}{\Q}$ is Galois, and
            $\Gal{\faktor{\Q(\sqrt{2})}{\Q}}=\langle \s \rangle \simeq
            \faktor{\Z}{2\Z}$, where $\s:a+b\sqrt{2} \xrightarrow{}
            a-b\sqrt{2}$.

        \item[(2)] Any quadratic extension field $K$ over  $F$ is Galois over
            $F$, provided  $\Char{F} \neq 2$. Then any quadratic extension $K$
            of  $F$, of degree  $[K:F]=2$ is of the form $F(\sqrt{D})$, where $D
            \in \Z^+$ is squarefree. Hence $K=F(\sqrt{D})$ is the splitting
            field of the polynomial $x^2-D$.

        \item[(3)] $\Q(\sqrt[3]{2})$ is not Galois over $\Q$, since
            $\ord{\Aut{\faktor{\Q(\sqrt[3]{2})}{\Q}}}=1$, but
            $[\Q(\sqrt[3]{2}):\Q]=3$.

        \item[(4)] $\Q(\sqrt{2},\sqrt{3})$ is the splitting field of the
            seperable polynomial $(x^2-2)(x^2-3)$ over $\Q$. Hence
            $\Q(\sqrt{2},\sqrt{3})$ is Galois over $\Q$, and has Galois group
            $\Gal{\faktor{\Q(\sqrt{2},\sqrt{3})}{\Q}}$ of order $6$. Moreover,
            since the automorphisms of this group are completely determined by
            the roots $\sqrt{2}$ and $\sqrt{3}$, we get the possible
            automorphisms are given by the maps
            \begin{equation*}
                \begin{array}{lcl}
                    \sqrt{2}    \xrightarrow{} \sqrt{2} && \sqrt{3}
                                                \xrightarrow{} -\sqrt{3} \\

                 \sqrt{2}    \xrightarrow{} -\sqrt{2} && \sqrt{3}
                                            \xrightarrow{} \sqrt{3} \\

                    \sqrt{2}    \xrightarrow{} \sqrt{2} && \sqrt{3}
                                                \xrightarrow{} -\sqrt{3} \\

                    \sqrt{2}    \xrightarrow{} -\sqrt{2} && \sqrt{3}
                                                \xrightarrow{} -\sqrt{3} \\
                \end{array}
            \end{equation*}
            Now, let $\s:\sqrt{2} \xrightarrow{} -\sqrt{2}, \sqrt{3}
            \xrightarrow{} \sqrt{3}$ and $\t:\sqrt{2} \xrightarrow{} \sqrt{2},
            \sqrt{3} \xrightarrow{} -\sqrt{3}$. Then $\s\t:\sqrt{2}
            \xrightarrow{} -\sqrt{2}, \sqrt{3} \xrightarrow{} -\sqrt{3}$.
            Therefore we have
            \begin{equation*}
                \Gal{\faktor{\Q(\sqrt{2},\sqrt{3})}{\Q}}=\langle \s,\t \rangle
                \simeq V_4
            \end{equation*}
            where $V_4$ is the Klein $4$-group.

            We can also determine the fixed fields correspongidng to each
            subgroup of  $\langle \s \t \rangle$. That is, $\Fc(\langle \s\t
            \rangle)$ is the set of all elements fixed by $\s\t$ and has
            elements of the form  $a+b\sqrt{6}$. So $\Fc(\langle \s\t
            \rangle)=\Q(\sqrt{6})$. The table below lists the fixed fields of
            the Galois group considered.
            \begin{equation*}
                \begin{array}{lcl}
                    \text{subgroup}    &&   \text{fixed field}   \\
                    \langle \i \rangle  &&   \Q(\sqrt{2},\sqrt{3})   \\
                    \langle \s \rangle  &&   \Q(\sqrt{3})   \\
                    \langle \s\t \rangle  &&   \Q(\sqrt{6})   \\
                    \langle \t \rangle  &&   \Q(\sqrt{2})   \\
                    \langle \s,\t \rangle  &&   \Q   \\
                \end{array}
            \end{equation*}

        \item[(5)] The roots of $x^3-2$ over  $\Q$ are given by
            \begin{align*}
                \sqrt[3]{2} && \xi\sqrt[3]{2}   &&  \xi^2\sqrt[3]{2}    \\
            \end{align*}
            where $\xi^3=1$ is the  $3$-rd root of unity. Additionally, the
            splitting field of  $x^3-2$ over  $\Q$ is  $\Q(\sqrt[3]{2},
            \xi\sqrt[3]{2})$ of degree $6$. Now,  $x^3-2$ is irreducible over
            $\Q$, and hence seperable over  $\Q$. This makes  $\Q(\sqrt[3]{2},
            \xi\sqrt[3]{2})$ Galois over $\Q$, of order  $6$.

            Consider now the set of generators $\sqrt[3]{2}$ and $\xi$. Then an
            automorphism  $\s$ takes  $\sqrt[3]{2} \xrightarrow{} \sqrt[3]{2},
            \xi\sqrt[3]{2}$, or $\xi^2\sqrt[3]{2}$, and takes  $\xi
            \xrightarrow{} \xi$ or $\xi^2$. Since these are the roots of the
            cyclotomic polynomial $\Phi_3(x)=x^2+x+12+x+1$, $\s$ is completely
            determined by the actions on  $\sqrt[3]{2}$ and  $\xi$. Hence there
            are  $6$ possible automorphisms.

            Let
            \begin{equation*}
                \begin{array}{lcl}
                    \s: \sqrt[3]{2} \xrightarrow{} \xi\sqrt[3]{2}    &&  \xi
                                                    \xrightarrow{} \xi  \\

                    \t: \sqrt[3]{2} \xrightarrow{} \sqrt[3]{2}    &&  \xi
                                                    \xrightarrow{} \xi^2  \\
                \end{array}
            \end{equation*}

            We obtain then the elements
            \begin{align*}
                \i  &&  \s^2    &&  \t\s^2=\s\t \\
            \end{align*}
            and we get the additional relations
            \begin{equation*}
                \s^2=\t^2=\i
            \end{equation*}
            so that
            \begin{equation*}
                \Gal{\faktor{\Q(\sqrt[3]{2}, \xi\sqrt[3]{2})}{\Q}}=
                \langle \s,\t \rangle \simeq S_3
            \end{equation*}
            The fixed field of $\langle \s^2 \rangle$ is $\Q(\xi)$.

        \item[(6)] $\Q(\sqrt[4]{2})$ is not Galois over $\Q$. We have
            $[\Q(\sqrt[4]{2}) : \Q]=4$ but that any automorphism takes
            $\sqrt[4]{2}$ onto $\pm \sqrt[4]{2}$, or $\pm i\sqrt[4]{2}$. But
            $\pm i\sqrt[4]{2} \notin \Q(\sqrt[4]{2})$. Notice however that
            $\Q(\sqrt[4]{2})$ is Galois over $\Q(\sqrt{2})$, and $\Q(\sqrt{2})$
            is Galois over $\Q$.

        \item[(7)] The extension field $\F_{p^n}$ is Galois over $\F_p$.  Recall
            that $\F_{p^n}$ is the splitting field of the seperable polynomial
            $x^{p^n}-x$ over $\F_p$. Then
            $\ord{\Gal{\faktor{\F_{p^n}}{\F_p}}}=n$ and the Frobenius
            automorphism given by
            \begin{equation*}
                \s:\a \xrightarrow{} \a^p
            \end{equation*}
            generates the Galois group, making it $\langle \s \rangle$, a cyclic
            group of order $n$.

        \item[(8)] The extension $\F_2(x)$ is not Galois over $\F_2(t)$, since
            $x^2-t$ is not seperable. Moreover, any automorphism of
            $\Aut{\faktor{\F_2(x)}{\F_2(t)}}$ sends $x$ to the only root of
            $x^2-t$, making it the trivial group.
    \end{enumerate}
\end{example}
