\section{Irreducible Algebraic Sets}\label{section_10.3}

\begin{definition}
  Let $k$ be a field. We call an algebraic set $X \subseteq \A^n(k)$
  \textbf{reducible} if it can be written as the union of two algebraic sets;
  that is, there exist $X_1,X_2 \subseteq \A^n(k)$ such that $X=X_1 \cup X_2$.
  We call an algebraic set \textbf{irreducible} if it is not reducible.
\end{definition}

\begin{proposition}\label{proposition_10.3.1}
  An algebraic set is irreducible if, and only if its ideal is prime.
\end{proposition}
\begin{proof}
  Let $X \subseteq \A^n$. Suppose that the ideal $I(X)$ is not prime.
  Let $f_1f_2 \in I(X)$, but $f_1,f_2 \not\in I(X)$. Then:
  \begin{equation*}
    X=(X \cap V(f_1)) \cup (X \cap V(f_2))
  \end{equation*}
  and $X \cap V(f_1) \subseteq X$ and $X \cap V(f_2) \subseteq X$. This makes
  $X$ reducible, by definition.

  Conversely, suppose that $X$ is reducible, and that $X=X_1 \cup X_2$ for
  $X_1, X_2 \subseteq \A^n(k)$. Then $I(X) \subseteq I(X_1)$ and $I(X)
  \subseteq I(X_2)$. Let $f_1 \in I(X_1)$ and $f_2 \in I(X_2)$, but $f_1,f_2
  \not\in I(X)$. Then $f_1f_2 \in I(X)$, but $f_1,f_2 \not\in I(X)$, so that
  $I(X)$ is not prime.
\end{proof}

\begin{example}\label{example_10.5}
  \begin{enumerate}
    \item[(1)] The algerbraic set $V(y-x^2)$ is irreducible in
      $\A^2(\C)$. Let $\phi$ be the homomorphism from $\C[x,y]$ onto
      $\C[x]$ defined by the rules:
      \begin{align*}
        x & \xrightarrow{} x \\
        y & \xrightarrow{} y^2 \\
      \end{align*}
      then $\ker{\phi}=(x-y^2)$ and by the first isomoprhism theorem
      we get
      \begin{equation*}
        \faktor{\C[x,y]}{(y-x^2)} \simeq \C[x]
      \end{equation*}
      which makes $(y-x^2)$ a prime ideal. Indeed,
      $\Ic((V(y-x^2)))=(y-x^2)$.

    \item[(3)] The set $V(y^4-x^2, y^4-x^2y^2+xy^2-x^3)$ is reducible
      in $\A^2(\C)$. We have
      \begin{equation*}
        V(y^4-x^2, y^4-x^2y^2+xy^2-x^3)=
        V(y^4-x^2) \cap V(y^4-x^2y^2+xy^2-x^3)
      \end{equation*}
      Now, $y^4-x^2=(y^2-x)(y^2+x)$ and $y^4-x^2y^2+xy^2-x^3=
      y^2(y^2-x^2)+x(y^2-x^2)=(y^2+x)(y^2-x^2)=(y^2+x)(y-x)(y+x)$. So
      \begin{equation*}
        V(y^4-x^2, y^4-x^2y^2+xy^2-x^3)=
        V(y^2-x) \cup V(y^2+X) \cup V(y-x) \cup V(y-x)
      \end{equation*}
      Observe also that eaach of these components are irreducible.

    \item[(4)] We recall that an irreducible element in an integral
      domain need not be prime. Indeed, $\R[x,y]$ is not a PID, so
      proposition \ref{proposition_6.3.2} fails to hold.

      Let $f(x,y)=y^2+x^2(x-1)^2$ in $\R[x,y]$. Viewing $f(x,y)$ as a
      polynomial in $y$, set $x^2(x-1)^2=a$, then $f(x,y)=y^2+a$ has
      no real roots, and so is irreducible in $\R[y]$. Now, set
      $a=y^2$ so that $f(x,y)=x^2(x-1)^2+a$. Let $z=x(x-1)$, then
      $f(x,y)=z^2+a$ which is irreducible in $\R[z]$, and hence must
      also be irreducible over $\R[x]$. Therefore $f(x,y)$ is
      irreducible in $\R[x,y]$. However, notice that the only points
      in $V(y^2+x^2(x-1)^2)$ are the points $(0,0)$ and $(1,0)$. This
      makes $V(y^2+x^2(x-1)^2)$ reducible in $\A^2(\R)$

    \item[(5)] Let $f(x,y)=y^2+x^2(x-1)^2$ in $\C[x,y]$. Observe that
      \begin{equation*}
        f(x,y)=(x^2-x-iy)(x^2-x+iy)
      \end{equation*}
      so that $V(y^2+x^2(x-1)^2)$ is also reducible in $\C[x,y]$, and:
      \begin{equation*}
        V(y^2+x^2(x-1)^2)=V(x^2-x-iy) \cup V(x^2-x+iy)
      \end{equation*}
      Now, the componennts $V(x^2-x-iy)$ and $V(x^2-x+iy)$ are
      irreducible components, since $x^2-x \pm y$ is irreducible over
       $\C$, and $(x^2-x-iy)$ and $(x^2-x+iy)$ are prime ideals in
       $\C[x,y]$ (since $\C$ is algebraically closed).

    \item[(6)] Let $k$ be an infinite field. Then $\A^n(k)=(0)$ which
      is prime in $k[x_1, \dots, x_n]$ since $k[x_1, \dots, x_n]$ is
      an integral domain. This makes $\A^n(k)$ irreducible.

    \item[(7)] For any finite field $\F_q$,  $\A^n(\F_q)$ is
      reducible. Indeed, $\A^n(\F_q)$ is the finite disjoint union of
      $q$ points. Similarly, any algebraic set of $\F_q$ must also be
      reducible.
  \end{enumerate}
\end{example}

\begin{proposition}\label{proposition_10.3.2}
  Any non-empty collection of algebraic sets has a minimal member.
\end{proposition}
\begin{proof}
  Let $k$ be the underlying field, and let $\{X_\a\}$ be a non-empty
  collection of algerbaic sets in $\A^n$, ordered by inclusion. then
  the collections of ideals $\{\Ic(X_\a)\}$ is ordered under the
  reverse inclusion by proposition \ref{proposition_10.2.1}. Now
  since $k[x_1, \dots, x_n]$ is a Noetherian ring, and $\{\Ic(X_\a)\}$
  is non-empty, then by theorem \ref{theorem_5.9.2} $\{\Ic(X_\a)\}$
  has a maximal member. Therefore, $\{X_\a\}$ must have a minimal
  member.
\end{proof}

\begin{theorem}\label{theorem_10.3.3}
  Any algebraic set can be uniquely expressed as the disjoint union of
  irreducible algebraic sets.
\end{theorem}
\begin{proof}
  We first show that such a decomposition exists for every algebraic set in
  $\A^n$. Let  $\Sc$ be the collection of all algebraic sets which cannot be
  expressed as a (not necessarily disjoint) union of (not necessarily unique)
  irreducible algebraic sets. Let $X$ be a minimal element of $\Sc$. Then
  $X$ is not irreducible. Hence there exist $X_1, X_2 \subseteq \A^n$ for
  which $X=X_1 \cup X_2$; suppose further that $X_1,X_2 \subseteq X$. By the
  minimality of $X$,  $X_1, X_2 \not\in \Sc$, so that
  \begin{equation*}
    X_i=\bigcup_{j=1}^{m_i}{X_{ij}}
  \end{equation*}
  where each $X_{ij}$ is irreducible. This makes
  \begin{equation*}
    X=\bigcup_{i=1,j=1}^{m,m_i}{X_{ij}}
  \end{equation*}
  which contradicts that $X \in \Sc$. Therefore $\Sc$ must be empty, and every
  algebraic set can be expressed as the union of irreducible algebraic sets.

  Now, take  $X=X_1 \cup \dots X_m$, where each $X_i$ is irreducible, and
  discard all those $X_i$ for which  $X_i \subseteq X_j$ for all $i \neq j$.
  This makes $X$ a disjoint union. Now, suppose that  $X=Y_1 \cup \dots Y_r$.
  Then
  \begin{equation*}
    X_i=\bigcup_{j=1}^r{(Y_j \cap X_i)}
  \end{equation*}
  so that $X_i \subseteq Y_j$ for some $j$. Similarly, we get that  $Y_j
  \subseteq X_k$ for some $k$. Thus $X_i \subseteq X_k$, but since $X$ is
  already a disjoint union, this makes  $i=k$ so that $X_i=Y_j$ and $m=r$.
  Thus the decomposition of $X$ into mutually disjoint irreducible algebraic
  sets is unique.
\end{proof}
\begin{corollary}
  If $X,Y \in \A^n$ are algebraic such that $X \subseteq Y$, then each
  irreducible component of $X$ is contained in an irreducible component of
  $Y$.
\end{corollary}
\begin{proof}
  Let
  \begin{equation*}
    X=\bigcup_{i=1}^r{X_j} \text{ and } Y=\bigcup_{j=1}^l{Y_j}
  \end{equation*}
  the decompositions of $X$ and $Y$ where $\{X_i\}_{i=1}^r$ and
  $\{Y_j\}_{j=1}^l$ are the irreducible components of $X$ and $Y$,
  resepectivel and where  $X \subseteq Y$. Let $\Sc$ the collection of
  all irreducible components of  $X$ not contained in an irreducible
  component of $Y$. Then by proposition \ref{proposition_10.3.2},
  $\Sc$ has a minimal member $X'$. Then $X' \not\subseteq Y_j$ for any
  $1 \leq j \leq l$, so $X \not\subseteq Y$. But $X' \subseteq X
  \subseteq Y$ which contradicts the assumption. Therefore no such
  $X'$ exists and $\Sc$ must be empty.
\end{proof}
