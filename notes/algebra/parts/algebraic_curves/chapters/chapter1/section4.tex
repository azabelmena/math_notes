\section{Algebraic Subsets of The Plane}\label{section_10.6}

\begin{proposition}\label{proposition_10.4.1}
  Let $k$ be a field. If $f(x,y)$ and $g(x,y)$ are coprime in
  $k[x,y]$, then $V(f,g)$ is finite.
\end{proposition}
\begin{proof}
  If $f(x,y)$ and $g(x,y)$ are coprime in $k[x,y]=k[x][y]$, then
  they are coprime in $k(x)[y]$. Then $(f,g)=(1)$, since $k(x)[x]$ is
  a PID. Then there exist $r(y),s(y) \in k(x)[y]$ for which:
  \begin{equation*}
    r(y)f(x,y)+s(y)g(x,y)=1
  \end{equation*}
  Now, there also exists a non-zero $d(x) \in k[x]$ for which
  $a(x,y)=d(x)r(y)$ and $b(x,y)=d(x)s(y)$ in $k[x,y]$. Then
  \begin{equation*}
    a(x,y)f(x,y)+b(x,y)g(x,y)=d(x)
  \end{equation*}
  Now, if $(a_1,a_2) \in V(f,g)$, then
  \begin{equation*}
    a(a_1,a_2)f(a_1,a_2)+b(a_1,a_2)g(a_1,a_2)=d(a_1)=0
  \end{equation*}
  Since $d(x)$ has finitely many roots, there are finitely many $a_1$
  for which $(a_1,a_2) \in V(f,g)$. By similar reasoning, there are
  finitely many $a_1$ for which $(a_1,a_2) \in V(f,g)$. Hence
  $|V(f,g)|<\infty$.
\end{proof}
\begin{corollary}
  If $f(x,y)$ is irreducible in $k[x,y]$ and $V(f)$ is infinite, then
  $\Ic(V(f))=(f)$ and $V(f)$ is irreducible.
\end{corollary}
\begin{proof}
  Let $g \in \Ic(V(f))$. Since $V(f)$ is infinite, $g \divides f$ by
  above so that $g \in (f)$. Now, since $f$ is irreducible in
  $k[x,y]$, $(f)$ is prime so that $V(f)$ is irreducible by
  proposition \ref{proposition_10.3.1}.
\end{proof}
\begin{corollary}
  If $k$ is an algerbaically closed field, and $f(x,y)$ is irreducible
  in $k[x,y]$, then $\Ic(V(f))=(f)$ and $V(f)$ is irreducible.
\end{corollary}
\begin{proof}
  By above, since $k$ is algerbaically closed, $V(f)$ is infinite in
  $\A^2(k)$.
\end{proof}
\begin{corollary}
  If $k$ is infinite, then the irreducible algerbaic sets of $\A^2(k)$
  are $\A^2(k)$ itself, the empty-set, points, and irreducible
  plane curves.
\end{corollary}
\begin{proof}
  Let $V \subseteq \A^2(k)$ be an irreducible algebraic set. If
  $\Ic(V)=(0)$ or $\Ic(V)=k[x,y]$ then $V=\A^2(k)$ or $V=\0$ as
  required. Now, if $\Ic(V)$ contains a non-constant polynomial
  $f(x,y)$, then since $\Ic(V)$ is prime, any irreducible factor of
  $f(x,y)$ is contained in $\Ic(V)$; so we can assume that $f(x,y)$ is
  irreducible. Then $\Ic(V)=(f)$. Lastly, if $P$ is a point, then
  $\{P\}$ is irreducible in $\A^2(k)$ by definition.
\end{proof}
\begin{corollary}
  If $k$ is algebraically closed, and $f(x,y) \in k[x,y]$ has the
  factorization:
  \begin{equation*}
    f(x,y)=f_1(x,y)^{n_1} \dots f_r(x,y)^{n_r}
  \end{equation*}
  into irreducible factors, then:
  \begin{equation*}
    V(f)=V(f_1) \cup \dots V(f_r)
  \end{equation*}
  where each $V(f_i)$ is irreducible. Moreover $\Ic(V)=(f_1, \dots,
  f_r)$.
\end{corollary}
\begin{proof}
  Since each $f_i$ is irreducible, no $f_i$ divides any other $f_j$
  for all $i \neq j$. Since $k$ is also algerbaically closed we get
  that each $V(f_i)$ is irreducible, and so $V(f_i) \not\subseteq V(f_j)$
  for all $i \neq j$. Thus $V(f_i)$ are the required irreducible
  components in the decomposition of $V(f)$.

  Now
  \begin{equation*}
    \Ic(V(f))=\bigcap_{i=1}^r{\Ic(V(f_i))}=\bigcap_{i=1}^r{(f_i)}
  \end{equation*}
  Now, any polynomial divisible by $f_i$ is also divisible by $f_1
  \dots f_r$ so $\bigcap{(f_i)}=(f_1, \dots, f_r)$.
\end{proof}

\begin{example}\label{example_10.6}
  \begin{enumerate}
    \item[(1)] Let $f(x,y)=x^2+y^2+1$. Now, if $f(x,y)=0$ then we
      have $x^2=-y^2-1$. Now suppose $(a,b) \in \R^2$ such that
      $a^2=-b^2-1$. Then we have  $a^2=-(b^2+1)$. Now, there are no
      such $a \in \R$ for which this equation holds for any $b \in \R$
      hence $(a,b) \notin V(f)$. Thus $V(x^2+y^2+1)=\0$ in $\A^2(\R)$,
      hence $\Ic(V(x^2+y^2+1))=\R[x,y]$.

    \item[(2)] Since $\R$ is infinite, by the above corollary to
      proposition \ref{proposition_10.4.1}, the only irreducible
      algerbaic sets of $\A^2(\R)$ are $\A^2(\R)$ itself, the
      empty-set, points, or irreducible plane curves. Then we have the
      following forms for an irreducible algebraic set in $\A^2(\R)$:
      \begin{enumerate}
        \item[(i)] $V(0)$ or $V(1)$ where $0(x,y), 1(x,y)$ are the
          zero-polynomial and constant polynomail $1$, respectively.

        \item[(ii.)] $V(x-a_1,y-a_2)$ where $P=(a_1,a_2)$ is a point
          in $\R^2$.

        \item[(iii.)] $V(f)$ where $f(x,y)$ is irreducible in
          $\R[x,y]$.
      \end{enumerate}

    \item[(3)] Let $f(x,y)=y^2-xy-x^2y+x^3$ in $\R[x,y]$. Then
      $f(x,y)=(y+x^2)(y-x)$, so $V(f)=V(y+x^2) \cup V(y-x)$ in
      $\A^2(\R)$. Now, we habe that $y+x^2=(x-iy)(x+iy)$ in $\C[x,y]$,
      so that $V(f)=V(x-iy) \cup V(x+iy) \cup V(y-x)$.

    \item[(4)] Let $f(x,y)=y^2-x(x^2-1)$ in $\R[x,y]$ then $f(x,y)$ is
      irreducible in both $\R[x,y]$ and $\C[x,y]$, hence $V(f)$ is
      irreducible.

    \item[(5)] Let $f(x,y)=x^3+x-x^2y-y$ in $\R[x,y]$. Then
      $f(x,y)=(x-y)(x^2+1)$, so $V(f)=V(x-y) \cup V(x^2-1)$ in
      $\A^2(\R)$ and $V(f)=V(x-y) \cup V(x-i) \cup V(x+i)$ in
      $\A^2(\C)$.
  \end{enumerate}
\end{example}
