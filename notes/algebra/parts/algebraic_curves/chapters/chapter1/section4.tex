\section{Algebraic Subsets of The Plane}\label{section_1.6}

\begin{lemma}\label{1.6.1}
    let $k$ be a field, and let $f,g \in k[x,y]$ polynomials with no common
    factor. then the set $v(f,g)=v(f) \cap v(g)$ is a finite set of points.
\end{lemma}
\begin{proof}
    notice that if $f$ and $g$ are coprime in $k[x,y] \simeq k[x][y]$, then they
    are coprime in $k(x)[y]$, where $k(x)$ is the field of fractions of $k[x]$.
    we have that $k(x)[y]$ is a pid, and that the ideal $(f,g)=(1)$. then there
    exist $r,s \in k(x)[y]$ for which $rf(x,y)+sg(x,y)=1$. there also exists a
    $d \in k[x]$ such that $d(x)r=a(x,y)$ and and $d(x)s=b(x,y)$ in $k[x,y]$.
    then $a(x,y)f(x,y)+b(x,y)g(x,y)=d(x)(rf(x,y))+d(x)(rg(x,y))=d(x)$. now, if
    $a,b \in v(f,g)$, then $d(a)=0$. now, $d$ has finitely many roots in $k$, so
    that there are finitely many $x$-coordinates corresponding to the points of
    $v(f,g)$. similarly, in the pid $k(y)[x]$, we get that there are finitely
    many $y$-coordinates corresponding to the points of $v(f,g)$. that is
    $v(f,g)$ have finitely many points.
\end{proof}
\begin{corollary}
    if $f$ is irreducible in $k[x,y]$ and $v(f)$ is finite, then $i(v(f))=(f)$,
    and $v(f)$ is an irreducible algebraic set.
\end{corollary}
\begin{proof}
    suppose that $g \in i(v(f))$, then $v(f,g)$ is infinite, and by the above
    lemma, we get that $g|f$. then $g \in (f)$, so that $i(v(f))=(f)$. moreover,
    since $f$ is irreducible in the $k[x,y]$, if $ab \in (f)$, then either $a
    \in (f)$ or $b \in (f)$, which makes $i(v(f))=(f)$ a prime ideal. this makes
    $v(f)$ irreducible by lemma \ref{1.5.1}.
\end{proof}
\begin{corollary}
    suppose that $k$ an infinite field, then the irreducible algebraic sets of
    $\a^2(k)$ are $\a^2(k)$ itself, the emptyset, point sets, and irreducible
    plane curves $v(f)$, where $f \in k[x,y]$ is irreducible and $v(f)$ is
    infinite.
\end{corollary}
\begin{proof}
    let $x \subseteq \a^2(k)$ an irreducible algebraic set. if $x$ is finite, or
    $i(x)=(0)$, then it is either $\a^n(k)$, the emptyset, or a finite algebraic
    set (i.e. a set of points). suppose then, that $x$ is infinite. then there
    exists a nonconstant polynomial $f \in i(x)$. now, since $x$ is irreducible,
    $i(x)$ is prime, and hence contains an irreducible factor of $f$; thus,
    suppose without loss of generality that $f$ is irreducible. then $i(x)=(f)$;
    for otherwise, if $g \in i(x)$ but $g \not\in (f)$, then $x \subseteq
    v(f,g)$ is finite which is a contradiction. this makes $x=v(f)$ as required.
\end{proof}
\begin{corollary}
    if $k$ is an algebraically closed field, and $f$ has the decomposition
    $f=f_1^{n_1} \dots f_m^{n_m}$ into irreducible factore, then $v(f)=v(f_1)
    \cup \dots v(f_m)$ is the decomposition of $v(f)$ into irreducible
    components. moreover, $i(v(f))=(f_1 \dots f_m)$.
\end{corollary}
\begin{proof}
    by hypothesis, we have that each $f_i$ and $f_j$ are coprime whenever $i
    \neq j$. that is, there exist no inclusions under each $v(f_i)$, so that the
    decomposition $v(f)=v(f_1) \cup \dots \cup v(f_m)$ is the decompostion of
    $v(f)$ into irreducible components. now, we also have that
    \begin{equation*}
        i(v(f))=\bigcap_{i=1}^m{i(v(f_i))}=\bigcap_{i=1}^m{(f_i)}
    \end{equation*}
    now, since each polynomial divisible by $f_i$ is also divisible by $f_1
    \dots f_m$, we get that $\bigcap{(f_i)}=(f_1 \dots f_m)$. lastly, notice
    that since $k$ is algebraically closed, and hence infinite, each $v(f_i)$ is
    infinite.
\end{proof}

\begin{example}\label{example_1.14}
    \begin{enumerate}
        \item[(1)] consider $f(x,y)=x^2+y^2+1$ over $\r$. we have that $f$ is
            irreducible, and that $v(f)$ is finite (in fact $v(f)=\emptyset$),
            so that $i(v(f))=(f)$. moreover, since $f$ has no roots in $\r$, we
            observe that $(f)=(1)$. this result could've also been extracted
            using the fact that $i(v(f))=i(\emptyset)=\r[x,y]=(1)$.

        \item[(2)] consider $x \subseteq \a^2(\r)$ an algebraic set. then there
            is some $s=(f_1, \dots, f_n) \in \r[x,y]$ for which $x=v(s)=v(f_1,
            \dots, f_n)=v(f_1) \cap \dots \cap v(f_n)$. now, by the above
            corollories, and assuming that each $f_i$ is pairwise coprime,
            $x=v(s)$ is a finite set of points. take
            \begin{equation*}
                f(x,y)=\sum_{i=1}^n{f_i^2(x,y)}
            \end{equation*}
            which has finitely many roots as a polynomial in $\r[x][y]$, and as
            a polynomial in $\r[y][x]$. then $(f)=(f_1, \dots ,f_n)$ so that
            $x=v(f)$. in geneal, we want to work over algebraically closed
            fields to avoid this, since the intersections of hypersurfaces need
            not be finite.
    \end{enumerate}
\end{example}

\begin{example}\label{example_1.14}
    \begin{enumerate}
        \item[(1)] we have that $v(y^2-xy-x^2y+x^3)=v(x-y) \cup v(x^2+y)$ over
            both $\r$ and $\c$.

        \item[(2)] the set $v(y^2-x(x^2-1))$ is irreducible over both $\r$ and
            $\c$ since the polynomial  $y^2-x(x^2-1)$ is an irreducible
            polynomial over both $\r$ and  $\c$.

        \item[(3)] $v(x^3+x-x^2y-y)$ is irreducible over $\r$, but
            $v(x^3+x-x^2y-y)=v(x-i) \cup v(x+i) \cup v(x-y)$ over $\c$, where
            $i^2=-1$.
    \end{enumerate}
\end{example}
