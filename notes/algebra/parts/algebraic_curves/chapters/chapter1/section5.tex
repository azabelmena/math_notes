\section{Hilbert's Nullstellensatz}\label{section_10.7}

\begin{proposition}[The Weak Nullstellensatz]\label{proposition_10.5.1}
  Let $k$ be an algebraically closed field. If $\af$ is a proper ideal
  of $k[x_1, \dots, x_n]$, then $V(\af)$ is non-empty.
\end{proposition}
\begin{proof}
  Suppose without loss of generality that $\af$ is a maximal ideal in
   $k[x_1, \dots, x_n]$. Then
   \begin{equation*}
     l=\faktor{k[x_1, \dots, x_n]}{\af}
   \end{equation*}
   is a field containing $k$ as a subfield. Now, suppose that $k=l$,
   then for all  $1 \leq i \leq n$, there exist $a_i \in k$ such that
   $x_i-a_i \in \af$. Now, $(x_1-a_1, \dots, x_n-a_n)$ is a maximal
   ideal in $k[x_1, \dots, x_n]$, so that $\af=(x_1-a_1, \dots, x_n-a_n)$,
   and $V(\af)=\{(a_1, \dots, a_n)\}$. Now, if $\bf$ is any other
   ideal in $k[x_1,\dots,x_n]$, then $\bf$ is contained in a maximal
   ideal, say $\af$, so $V(\af) \subseteq v(\bf)$, and $V(\bf)$
   contains at-least one point.
\end{proof}

\begin{theorem}[Hilbert's Nullstellensatz]\label{theorem_10.5.2}
  Let $k$ be an algebraically closed field, and let $\af$ an ideal of
  $k[x_1, \dots, x_n]$. Then $\Ic(V(\af))=\rad{\af}$.
\end{theorem}
\begin{proof}
  By proposition \ref{proposition_10.2.3}, $\rad{\af} \subseteq
  \Ic(V(\af))$. Since every ideal in $k[x_1, \dots, x_n]$ is finitely
  generated, let $\af=(f_1, \dots, f_r)$, and take $g \in \Ic(V(f_1,
  \dots, f_r))$. Now, let $\bf=(f_1, \dots, f_r, x_{n+1}g-1)$. Then
  for any $P \in V(f_1, \dots, f_r)$, $g(P)=0$ so that $x_{n+1}g-1
  \neq 0$. This makes $V(\bf)$ empty. Therefore by the weak
  nullstellensatz, $1 \in \bf$. Therefore
  \begin{equation*}
    1=a_1f_1(x_1, \dots, x_n)+\dots+a_rf_r(x_1, \dots, x_n)+
    a_{r+1}(x_{n+1}g(x_1, \dots, x_n)-1)
  \end{equation*}
  where $a_1, \dots, a_r,a_{r+1} \in k[x_1, \dots, x_{n+1}]$. Set
  $y=\frac{1}{x_{n+1}}$. Then for large enough $N \in \Z^+$:
  \begin{equation*}
    y^N=c_1f_1(x_1, \dots, x_n)+\dots+c_rf_r(x_1, \dots, x_n)+
    c_{r+1}(x_{n+1}g(x_1, \dots, x_n)-y)
  \end{equation*}
  for $c_1, \dots, c_r,c_{r+1} \in k[x_1, \dots, x_{n+1}]$.
  Substituing $g$ for $y$ in the above equation gives $g^N \in \af$ in
  $k[x_1, \dots, x_n]$, so that $g \in \rad{\af}$.
\end{proof}
\begin{corollary}
  If $\af$ is a radical ideal, then $\Ic(V(\af))=\af$. Moreover, there
  exists a 1--1 correspondence between radical ideals and algebraic
  sets.
\end{corollary}
\begin{proof}
  Define the map:
  \begin{align*}
    \Ic: 2^{\A^n(k)} & \xrightarrow{} 2^{k[x_1, \dots, x_n]} \\
        V(S) & \xrightarrow{} \Ic(V(S)) \\
  \end{align*}
  where $S$ is the generating set of $V(S)$. Then $\Ic$ maps algebraic
  sets $V(S)$ to their ideals $\Ic(V(S))$. Moreover, this map is a
  1--1 correspondence between the algebraic sets and their ideals, by
  proposition \ref{proposition_10.2.1}. Now let $\af=(S)$ so
  $V(S)=\af$. By Hilbert's nullstellensats, $\Ic(V(\af))=\rad{\af}$,
  so that $\Ic$ defines a 1--1 correspondence between $V(\af)$ and the
  radical ideal $\rad{\af}$.

  Now, if $\af$ is radical in $k[x_1, \dots, x_n]$ then by the
  nullstellensatz again, $\Ic(V(\af))=\rad{\af}=\af$.
\end{proof}
\begin{corollary}
  If $\af$ is a prime ideal, then  $V(\af)$ is irreducible. Moreover,
  there exists a 1--1 correspondence between prime ideals and
  irreducible algebraic sets, where the maximal ideals correspond to
  points.
\end{corollary}
\begin{proof}
  Again, via the map $\Ic:V(\af) \xrightarrow{} \Ic(V(\af))$, if
  $V(\af)$ is irreducible, then $\Ic(V(\af))$ is a prime ideal.
  Moreover the irreducible algebraic set $V=\{(a_1, \dots, a_n)\}$
  corresponds to the ideal $\Ic(V)=\rad{(x_1-a_1, \dots, x_n-a_n)}=
  (x_1-a_1, \dots, x_n-a_n)$ which is maximal.
\end{proof}
\begin{corollary}
  If $f \in k[x_1, \dots, x_n]$ has the decomposition:
  \begin{equation*}
    f(x_1, \dots, x_n)=
    f_1(x_1, \dots, x_n)^{n_1} \dots f_r(x_1, \dots, x_n)^{n_r}
  \end{equation*}
  into irreducible factors, then $V(f)$ decomposes as:
  \begin{equation*}
    V(f)=V(f_1) \cup \dots \cup V(f_r)
  \end{equation*}
  into irreducible components. Moreover, there exists a 1--1
  correspondence between irreducible polynomials and irreducible
  hypersurfaces.
\end{corollary}
\begin{proof}
  By theorem \ref{theorem_10.3.3} $V(f)$ decomposes uniquely as
  \begin{equation*}
    V(f)=V_1 \cup \dots \cup V_r
  \end{equation*}
  where $V_i$ is irreducible for all $1 \leq i \leq r$ and $V_i
  \not\subseteq V_j$ for any $i \neq j$. Now, consider the
  decomposition $f=f_1^{n_1} \dots f_r^{n_r}$ where each $f_i$ is an
  irreducible factor. Then $(f_i)$ is prime, so that $V(f_i)$ is
  irreducible. Indeed:
  \begin{equation*}
    V(f)=V(f_1) \cup \dots \cup V(f_r)
  \end{equation*}
  where $V(f_i)=V_j$ for some $1 \leq j \leq r$. Suppose without loss
  of generality that $V(f_i)=V_i$. Then since each $f_i$ is
  irreducible, we get that  $V(f_i) \not\subseteq V(f_j)$ for any $i
  \neq j$ and we have the required decomposition.

  Now, suppose that $f$ is an irreducible polynimial in $k[x_1, \dots,
  x_n]$. Then $\Ic:V(f) \xrightarrow{} \Ic(V(f))$, since $f$ is
  irreducible, $\Ic(V(f))$ is prime so that $\Ic(V(f))=(f)$.
\end{proof}

\begin{proposition}\label{proposition_10.5.3}
  Let $k$ be an algebraically closed field, and $\af$ an ideal in
  $k[x_1, \dots, x_n]$. Then $V(\af)$ is finite if, and only if
  $\faktor{k[x_1, \dots, x_n]}{\af}$ is finite dimensional as a
  vector space over $k$. Moreover:
  \begin{equation*}
    |V(\af)| \leq \dim_k{\faktor{k[x_1, \dots, x_n]}{\af}}
  \end{equation*}
\end{proposition}
\begin{proof}
  Let $K=\faktor{k[x_1, \dots, x_n]}{\af}$, and suppose $K$ is finite
  dimensional when viewed as a vector space over $k$. Let $P_1, \dots,
  P_r \in V(\af)$, and choose $f_1, \dots, f_r \in k$ such that
  $f_i(P_j)=0$ for all $i \neq j$ and $f_i(P_i)=1$. Take $g_i \equiv
  f_i \mod{\af}$ in $K$. Then if, for some $\l_1, \dots, \l_r \in k$
  \begin{equation*}
    \l_1g_1(x_1, \dots, x_n)+\dots+\l_rg_r(x_1, \dots, x_n)=0
  \end{equation*}
  we get $\l_1f_1(x_1, \dots, x_n)+\dots+\l_rf_r(x_1, \dots, x_n) \in \af$,
  so that, for all $i \neq j$:
  \begin{equation*}
    \l_j = \Big{(} \sum_{i \neq j}{\l_if_i(x_1, \dots, x_n)} \Big{)}(P_j)=0
  \end{equation*}
  which makes the collection $\{g_1, \dots, g_r\}$ linearly
  independent in $K$. This makes $r \leq \dim_k{K}<\infty$ so that
  $V(\af)$ is finite. Indeed, this makes $|V(\af)|=r$.

  Now, suppose that $V(\af)=\{P_1, \dots, P_r\}$. Take $P_i=(a_{i1},
  \dots, a_{in})$, and define:
  \begin{equation*}
    f_j(x_1, \dots, x_n)=\prod_{i=1}^r{(x_j-a_{ij})}
  \end{equation*}
  then $f_j \in \Ic(V(\af))$, so that for sufficiently large $N \in
  \Z^N$, $f_j^N \in \af$ for all $1 \leq j \leq r$. Take $g_j \equiv
  f_j \mod{\af}$, then $g_j^N=0$ for all $1 \leq j \leq r$. Take $y_i
  \equiv x_i \mod{\af}$. Then $y_i^{rN}$ is a linear combination of
  the set $\{1, y, \dots, y^{rN-1}\}$ so by induction, for some $s \in
  \Z^+$, $y_i^s$ of $\{1, y, \dots, y^{rN-1}\}$. Hence the set
  \begin{equation*}
    \{y_1^{m_1}, \dots, y_n^{m_n} : m_i < rN\}
  \end{equation*}
  generates the vector space $K$, and makes it finite dimensional.
\end{proof}
