\section{Ideals of Algebraic Sets}\label{section_1.2}

\begin{definition}
  Let $k$ be a field. We define the  \textbf{ideal} of a set $X
  \subseteq \A^n(k)$ to be the set:
  \begin{equation*}
    \Ic(X)=\{ f \in k[x_1, \dots, x_n] : f(P)=0 \text{ for all } P \in X \}
  \end{equation*}
  That is, $\Ic(X)$ is the set of all polynomials over $k$ which
  vanish on $X$.
\end{definition}

\begin{proposition}\label{proposition_10.2.1}
  Let $k$ be a field. The following are true:
  \begin{enumerate}
    \item[(1)] For any $X \subseteq \A^n(k)$ $\Ic(X)$ is a radical
      ideal in $k[x_1, \dots, x_n]$.

    \item[(2)] If $X \subseteq Y$ in $\A^n(k)$ then $\Ic(Y) \subseteq
      \Ic(X)$ in $k[x_1, \dots, x_n]$.

    \item[(3)] $\Ic(\0)=k[x_1, \dots, x_n]$ and if $k$ is an infinite
      field then $\Ic(\A^n(k))=(0)$. Moreover, for any $a_1, \dots,
      a_n \in k$:
      \begin{equation*}
        \Ic(\{a_1, \dots, a_n\})=(x_1-a_1, \dots, x_n-a_n)
      \end{equation*}

    \item[(4)] For any set of polynomials $S$ over $k$:
      \begin{equation*}
        S \subseteq \Ic(V(S)) \text{ and } V(S)=V(\Ic(V(S)))
      \end{equation*}

    \item[(5)] For any set of $X$ of points in $\A^n(k)$ :
      \begin{equation*}
        X \subseteq V(\Ic(X)) \text{ and } \Ic(X)=\Ic(V(\Ic(X)))
      \end{equation*}
  \end{enumerate}
\end{proposition}
\begin{proof}
  Let $X \subseteq \A^n(k)$. Observe first that $\Ic(X)$ is an ideal.
  Now, we have by proposition \ref{proposition_5.5.14} that
  $\Ic(X) \subseteq \rad{\Ic(X)}$. Now, let $f(x_1, \dots, x_n) \in
  \rad{\Ic(X)}$. Then for some $n \in \Z^+$, we have $f(x_1, \dots,
  x_n)^n \in \Ic(X)$, so that for every $P \in X$, $f(P)^n=0$. But
  that makes $f(P)=0$, so that $f(x_1, \dots, x_n) \in \Ic(X)$.
  Hence $\rad{\Ic(X)} \subseteq \Ic(X)$.

  Now, statements (2) and (3) follow directly from proposition
  \ref{proposition_10.1.2} (generalized to any set of points). Indeed
  suppose $X \subseteq Y$ in $\A^n(k)$, and let $f(x_1, \dots, x_n)
  \in \Ic(Y)$. Then for any point $P \in Y$, $f(P)=0$. Hence by
  inclusion, $f(Q)=0$ for any point $Q \in X$, so $f(x_1, \dots, x_n)
  \in \Ic(X)$. Observe also that since $k[x_1, \dots, x_n]=(1)$, which
  vanishes on no set of $\A^n(k)$, that $\Ic(\0)=(1)$. Now, let $k$ be
  infinite, then the zero polynomial $0(x_1, \dots, x_n)$ vanishes on
  all $\A^n(k)$ so that $\Ic(\A^n(k))=(0)$. Indeed, if $k$ were a
  finite field with $q$ elements, then $\Ic(\A^n(k))$ cannot be the
  trivial ideal since every subset of $\A^n(k)$ is algebraic in this
  case.

  Now, let $S \subseteq k[x_1, \dots, x_n]$. Then for any $f(x_1,
  \dots, x_n) \in S$ and any $P \in V(S)$ $f(P)=0$ so that $f \in
  \Ic(V(S))$. Now, let $X=V(S)$ and $\af=\Ic(X)$. Let $P \in X$ so
  that for any $f \in \af$, $f(P)=0$. Then $P \in V(\af)$. This gives
  $V(S) \subseteq V(\Ic(V(S)))$. Now, (letting again $X=V(S)$ and
  $\af=\Ic(X)$), let $P \in V(X)$. Then for any $f \in \af$, $f(P)=0$.
  But $\af=\Ic(X)$ and $X=V(S)$, so that $f$ vanishes on $V(S)$. This
  puts $P \in V(S)$ and we get the reverse inclusion. That is:
  $V(S)=V(\Ic(V(S)))$.

  Now, reversing the arguments from statement (4) gives the resulting
  proofs for statement (5). Indeed, let $P \in X$, then for any
  $f \in \Ic(X)$, $f(P)=0$ so that $P \in V(\Ic(X))$. This puts
  $X \subseteq V(\Ic(X))$. Now, let $\af=\Ic(X)$. Then for any
  $f \in \af$, $f(P)=0$ for all $P \in X$. This puts $\af \subseteq
  \Ic(V(\af))$. Likewise for any $f \in \Ic(V(\af))$, $f(P)=0$ for
  any $P \in V(\af)$, so that $f \in \af$, and we get the resuilting
  equality.
\end{proof}

\begin{proposition}
  Let $V$ and $W$ be algebraic sets in $\A^n$. Then $V=W$ if, and only
  if  $\Ic(V)=\Ic(W)$.
\end{proposition}
\begin{proof}
  If $V=W$, then  $V \subseteq W$ so that $\Ic(W) \subseteq \Ic(V)$,
  and $W \subseteq V$ so that $\Ic(V) \subseteq \Ic(W)$. Conversely,
  suppose that $\Ic(V)=\Ic(W)$. Let $S$ and $T$ be sets of
  polynomials and let $\af=(S)$ and $\bf=(T)$ such that
  $V=V(\af)$ and $W=V(\bf)$. Then $\Ic(V(\af)) \subseteq \Ic(V(\bf))$,
  so that $V(\Ic(V(\bf)))=V(\bf) \subseteq V(\af)=V(\Ic(V(\af)))$,
  and $\Ic(V(\bf)) \subseteq \Ic(V(\af))$ so that $V(\Ic(\af))=
  V(\af) \subseteq V(\bf)=V(\Ic(V(\bf)))$. That is, $V \subseteq W$
  and $W \subseteq V$.
\end{proof}

\begin{example}\label{example_10.4}
  Let $k$ be any field unless otherwise specified.
  \begin{enumerate}
    \item[(1)] Let $V$ be algerbaic in $\A^n$, and let $P \in
      \com{\A^n}{V}$, and let $W=V \cup \{P\}$. Observe that $V
      \subseteq W$ so that $\Ic(W) \subseteq \Ic(V)$. Now, suppose the
      reverse inclusion holds, that is for any $f \in \Ic(V)$,
      $f(Q)=0$ for all $Q \in W$. Then $f(P)=0$. But $P \notin V$
      which contradicts that $f$ vanishes on all $V$; indeed we have
      implied that $V=V \cup \{P\}$. Therefore, there is a polynomial
      $f(x_1, \dots, x_n)$ over $k$ for which $f(Q)=0$ for all $Q \in
      V$ but $f(P) \neq 0$. Indeed, define the polynomial $g(x_1,
      \dots, x_n)$ by
      \begin{equation*}
        g(x_1, \dots, x_n)=\inv{(f(P))}f(x_1, \dots, x_n)
      \end{equation*}
      then $g \in \Ic(V)$, and $g(P)=1$.

    \item[(2)] By above, if $P_1, \dots, P_r \in \com{\A^n}{V}$. Fix
      $j$ so $\Ic(V \cup \{P_j : i \neq j\}) \subseteq \Ic(V)$.
      Hence there are polynomials $f_1, \dots, f_r \in \Iv(V)$ for which
      $f_i(P_j)=0$ for all $i \neq j$ but $f_i(P_i)=1$.

      Now, let $P_1, P_2 \notin V$, and define:
      \begin{equation*}
        f(x_1, \dots, x_n)=a_1f_1(x_1, \dots, x_n)+a_2f_2(x_1, \dots,
        x_n)
      \end{equation*}
      where $f_1, f_2 \in \Ic(V)$ such that $f_1(P_2)=0$ and
      $f_1(P_1)=1$ and $f_2(P_1)=0$ and $f_2(P_2)=1$. Then
      $f(P_1)=f_1(P_1)=1$ and $f(P_2)=f_2(P_2)=1$.

    \item[(3)] Recall that the ideal $(x^2+1)$ is maximal in $\R[x]$,
      hence it is prime and hence radical in $\R[x]$. However
      $(x^2+1)$ is not the ideal of any set in $\A^1(\R)$, since the
      equation $x^2-1=0$ has no real solutions.
  \end{enumerate}
\end{example}

\begin{proposition}\label{proposition_10.2.3}
  Let $k$ be a field, and let $\af$ an ideal in $k[x_1, \dots, x_n]$.
  Then $V(\af)=V(\rad{\af})$ and $\rad{\af} \subseteq \Ic(V(\af))$.
\end{proposition}
\begin{proof}
  We have that $\af \subseteq \rad{\af}$, so that $V(\rad{\af})
  \subseteq V(\af)$. Now, let $P \in V(\rad{\af})$, then for any $f
  \in \rad{\af}$, $f(P)=0$. By definition, for some $n \in \Z^+$, $f^n
  \in \af$. Let $g=f^n$. Since  $f(P)=0$ then $(f(P))^n=g(P)=0$ so
  that $P \in V(\af)$. Indeed, this also shows that if $f \in
  \rad{\af}$, then $f \in \Ic(V(\af))$.
\end{proof}
