\section{Algebraic Extensions}
\label{section_8.2}

\begin{definition}
  Let $F$ be a field, and $K$ an extension of $F$. An element $\a \in K$
  is said to be \textbf{algebraic} over $F$ provided there exists a
  nonzero polynomial $f(x)$ over $F$ for which $\a$ is a root over
  $K$; i.e. $f(\a)=0$. If no such polynomial $f$ exists, we call $\a$
  \textbf{transcendental} over $F$.
\end{definition}

\begin{definition}
  Let $F$ be a field. We call an extension $K$ \textbf{algebraic} over
  $F$ if every element of $K$ is algebraic over $F$.
\end{definition}

\begin{proposition}\label{proposition_8.2.1}
  If $\a$ is algebraic over a field $F$, then it is algebraic over any
  extension of $F$.
\end{proposition}
\begin{proof}
  Let $K$ be an extension for which $\a \in K$ is algebraic over $F$.
  Then there is a polynomial $f(x) \in F[x]$ for which $f(\a)=0$ over
  $K$. Now, consider the tower
  \begin{equation*}
    F \subseteq E \subseteq K
  \end{equation*}
  Since $f(\a)=0$ in $K[x]$, $f(\a)=0$ in $E[x]$, and observe that $K$
  is an extension of $E$. Then $\a$ is algeraic over $E$. On the
  otherhand, if we have the tower
  \begin{equation*}
    F \subseteq K \subseteq E
  \end{equation*}
  since $\a \in K$, we have $\a \in E$ so that $f(\a)=0$ over $E$, and
  hence $\a$ must also be algebraic over $K$.
\end{proof}

\begin{proposition}\label{proposition_8.2.2}
  If $\a$ is algebraic over a field $F$, then there exists a unique
  monic irreducible polynomial $m_{\a,F}(x)$ over $F$ for which $\a$
  is a root. Moreover, if $f(x)$ is a polynomial over $F$ having $\a$ as
  a root, then $m_{\a,F}(x)$ divides $f(x)$.
\end{proposition}
\begin{proof}
  Let $K$ be an extension of $F$, and $\a \in K$. Let $g(x) \in F[x]$
  be of minimal degree, such that $g(\a)=0$. Suppose also, without
  loss of generality, that $g(x)$ is monic. Let $a(x),b(x) \in F[x]$
  such that $g(x)=a(x)b(x)$. Then $\deg{a} \leq \deg{g}$ and
  $\deg{b} \leq \deg{g}$. Moreover, $g(\a)=a(\a)b(\a)=0$. Since $K$
  is an integral domain, either $a(\a)=0$, or $b(\a)=0$. This
  contradicts that $g(x)$ is the polynomial of minimal degree for
  which $g(\a)=0$. Therefore, $g(x)$ must be irreducible.

  Now, let $f(x) \in F[x]$ such that $f(\a)=0$. By the division
  theorem, there exist polynomials $q(x),r(x) \in F[x]$ for which
  \begin{equation*}
    f(x)=q(x)g(x)+r(x) \text{ where } r(x)=0 \text{ or }
    \deg{r}<\deg{g}
  \end{equation*}
  Suppose that $r(x) \neq 0$. Since $f(\a)=0$ and $g(\a)=0$, it
  follows that $r(\a)=0$. But $\deg{r}<\deg{g}$ which contradicts the
  minimality of $g(x)$. Therefore $r(x)=0$ over $F$, and  $g|f$.

  Lastly, suppose there is another monici irreducible polynomial
  $g'(x) \in F[x]$ of minimal degree for which $g'(\a)=0$. Then by
  above, we have that $g'|g$ and $g|g'$. By monic irreduciblity, we
  must have that  $g(x)=g'(x)$.
\end{proof}
\begin{corollary}
  If $\faktor{L}{F}$ is an extension, and  $m_{\a,F}(x) \in F[x]$, and
  $m_{\a,L} \in L[x]$ are the monic irreducible polynomials described
  above for some $\a \in L$, then $m_{\a,L}(x)$ divides $m_{\a,F}(x)$
  in $L[x]$.
\end{corollary}

\begin{definition}
  Let $F$ be a field, and $K$ be an extension of $F$, and let $\a \in
  K$ be algebraic over $F$. We define the \textbf{minimal polynomial}
  of $\a$ over $F$ to be the unique monic irreducible polynomial
  $m_{\a,F}(x)$ over $F$ for which  $m_{\a,F}(\a)=0$. We define the
  \textbf{degree} of $\a$ over $F$ to be $\deg_F{\a}=\deg{m_{\a,F}}$.
  When the extension $K$ is understood, we simply write $m_\a(x)$ and
  $\deg{\a}$.
\end{definition}

\begin{proposition}\label{proposition_8.2.3}
  Let $\a$ be algebraic over a field  $F$. Then
  \begin{equation*}
    F(\a) \simeq \faktor{F[x]}{(m_\a)}
  \end{equation*}
\end{proposition}
\begin{proof}
  Since $m_\a(x)$ is monic and irreducible over $F$, $(m_\a)$ is a
  maximal ideal in $F[x]$. Thus
  \begin{equation*}
    \faktor{F[x]}{(m_\a)}
  \end{equation*}
  is a field. Now, by by definition of $\a$ as an algebraic element,
  and by theorem \ref{theorem_8.1.8}, the result follows.
\end{proof}
\begin{corollary}
  $[F(\a):F]=\deg{m_\a}=\deg{\a}$.
\end{corollary}

\begin{example}\label{example_8.5}
  \begin{enumerate}
    \item[(1)] $m_{\sqrt{2}}(x)&= x^2-2$ over $\Q$ and
      $[\Q(\sqrt{2}):\Q]=\deg{m_{\sqrt{2}}}=\deg{\sqrt{2}}=2$.
      Similarly, $m_{\sqrt[3]{2}}(x)=x^3-2$ over $\Q$, and
      $[\Q(\sqrt[3]{2}):\Q]=\deg{m_{\sqrt[3]{2}}}=\deg{\sqrt[3]{2}}=3$.
      We can also observer that $m_{\xi\sqrt[3]{2}}(x)=
      m_{\bar{\xi}\sqrt[3]{2}}(x)=m_{\sqrt[3]{2}}(x)$ over $\Q$, where
      \begin{equation*}
        \xi=-\frac{1}{2}+i\frac{\sqrt{3}}{2}
      \end{equation*}
      and $\bar{\xi}$ is the corresponding complex conjugate.

    \item[(2)] For every $n>1$,  $x^n-2$ is monic irreducible over
      $\Q$ by the Eisenstein-Sch\"onemann criterion. Now, denote
      $\a=\sqrt[n]{2}$, then:
      \begin{align*}
        \Q(\sqrt[n]{2}) &\simeq \faktor{\Q[x]}{(x^n-2)} \\
          &= \{a_0+a_1\sqrt[n]{2}+a_2\sqrt[n]{2^2}+
            \dots+a_{n-1}\sqrt[n]{2^{n-1}} :
          a_0, \dots, a_{n-1} \in \Q\}  \\
      \end{align*}
      Let $m(x)$ is the minimal polynomial of $\sqrt[n]{2}$ over
      $\Q$. Then $m(x)|x^n-2$, so that $\deg{\sqrt[n]{2}} \leq n$.
      Now, since $m(\sqrt[n]{2})=0$, and $m(x)$ is monic and
      irreducible, we get
      \begin{align*}
        \Q(\sqrt[n]{2}) &\simeq \faktor{\Q[x]}{(m)} \\
          &= \{b_0+b_1\sqrt[n]{2}+b_2\sqrt[n]{2^2}+
            \dots+b_{m}\sqrt[n]{2^{m}} :
          a_0, \dots, a_{n-1} \in \Q\}  \\
      \end{align*}
      for some $0 \leq m \leq n-1$. We then get that the two sets
      \begin{align*}
        \{ 1, \sqrt[n]{2}, \dots, \sqrt[n]{2^{n-1}} \} &&
        \{ 1, \sqrt[n]{2}, \dots, \sqrt[n]{2^{m}} \}  \\
      \end{align*}
      span $\Q(\sqrt[n]{2})$. Since $[\Q(\sqrt[n]{2}):\Q]=n$, we get
      $m=n-1$, so $\deg{m}=\deg{\sqrt[n]{2}}=n$ over $\Q$. Since $m(x)$
      and divides $x^n-2$, it follows that $m(x)=x^n-2$. Thereofore,
      $x^n-2$ is irreducible over  $\Q$.
      Now, observe that over $\R$, $m(x)=x-\sqrt[n]{2}$, so that
      $[R(\sqrt[n]{2}):\R]=1$, hence $\R(\sqrt[n]{2})=\R$.

    \item[(3)] Take $p(x)=x^3-3x-1$ over $\Q$ which is irreducible by
      the Eisenstein-Sch\"onemann criterion. Let $\a$ a root of
      $p(x)$, then $[\Q(\a):\Q]=3$.
  \end{enumerate}
\end{example}

\begin{proposition}\label{proposition_8.4.2}
  Let $F$ be a field. An element $\a$ in some extension of $F$ is
  algebraic over $F$ if, and only if the simple extension
  $\faktor{F(\a)}{F}$ is finite.
\end{proposition}
\begin{proof}
  Suppose that $\a$ is algebraic over $F$, and let $m_\a(x)$ the
  minimal polynomial of $\a$ over $F$. Then by proposition
  \ref{proposition_8.2.3}
  \begin{equation*}
    F(\a) \simeq \faktor{F[x]}{(m_\a)}
  \end{equation*}
  and $[F(\a):F]=\deg{m_\a}<\infty$. This makes $\faktor{F(\a)}{F}$ a
  finite simple extension.

  Conversely, suppose that $[F(\a):F]=n$, for some $n \geq 1$. Since
  $\{ 1, \a, \dots, \a^{n-1} \}$ is a basis for $F(\a)$ as a vector
  space over $F$, the elements
  \begin{align*}
    1 &&  \a  &&  \dots &&  \a^{n-1}  &&  \a^n  \\
  \end{align*}
  are linearly dependent, and hence there exist $b_0, \dots, b_n \in
  F$, not all $0$ for which
  \begin{equation*}
    b_0+b_1\a+\dots+b_{n-1}\a^{n-1}+b_n\a^n=0
  \end{equation*}
  define $m(x) \in F[x]$ by $m(x)=b_0+b_1x+\dots+b_nx^n$. Then
  $f(\a)=0$ by above, and hence $\a$ is by definition algebraic over
  $F$.
\end{proof}
\begin{corollary}
  If $\faktor{K}{F}$ is a finite extension, then $\faktor{K}{F}$ is
  algebraic.
\end{corollary}
\begin{proof}
  Let $\a \in K$. Then we have the tower $F \subseteq F(\a) \subseteq
  K$, and $[F(\a):F] \leq [K:F]<\infty$. This make $\a$ algebraic over
  $F$.
\end{proof}

\begin{example}\label{example_8.6}
  \begin{enumerate}
    \item[(1)] Let $F$ be a field of  $\char \neq 2$, and let
      $\faktor{K}{F}$ an extension with $[K:F]=2$. Take $\a \in K$,
      since $[F(\a):F] \leq 2$, we have $\deg{\a} \leq 2$. Now, let
      $m_\a(x)$ the minimal polynomial of $\a$ over  $F$. If
      $\deg{m_\a}=1$ then
      \begin{equation*}
        m_\a(x)=x-\a
      \end{equation*}
      which puts $\a \in F$, making $K=F$, which is impossible since
      $[K:F]=2$. Therefore $\deg{m_\a}=2$ and $m_\a(x)$ has the form:
      \begin{equation*}
        m_\a(x)=x^2+bx+c \text{ where } b,c \in F
      \end{equation*}
      solving for $\a$ in the equation $m_\a(\a)=0$, we get:
      \begin{equation*}
        \a=-\frac{b}{2} \pm \frac{\sqrt{b^2-4c}}{2}
      \end{equation*}
      Now, if $b^2-4c=y^2$ for some  $y \in F$, we get
      \begin{equation*}
        \a=-\frac{b}{2} \pm \frac{y}{2} \in F
      \end{equation*}
      which is impossible. So $b^2-4c$ is not square in  $F$. Now, let
       $\b=\sqrt{b^2-4c}$. We get
       \begin{equation*}
         b^2-(b^2-4c)=0
       \end{equation*}
       which is monic irreducible in $F$; more over there does not
       exist a polynomial of lower degree having $\b$ as a root,
       therefore:
       \begin{equation*}
         m_\b(x)=x^2-(b^2-4c)
       \end{equation*}

       Now, we claim that $F(\a)=F(\b)$. Certainly, $\a \in F(\b)$
       since
       \begin{equation*}
         \a=-\frac{b}{2} \pm \frac{\b}{2}
       \end{equation*}
       By simmilar reasoning, $\b=2\a \pm b^2$ so that $\b \in F(\a)$.
       Therefore $F(\a)=F(\b)$. Now since $\a \in K$, this also puts
       $\b \in K$.

       Now, let $D=b^2-4c$ so that  $F(\b)=F(\sqrt{D})$ (by similar
       computations to above). We call fields of the form
       $F(\sqrt{D})$ \textbf{quadratic extensions} of the field $F$.
       Indeed, we proved that if $\faktor{K}{F}$ is an extension of
       degree $2$, then it is a quadratic extension. We call
       $D=b^2-4c$ the \textbf{discriminant} of $F(\sqrt{D})$.
  \end{enumerate}
\end{example}

\begin{theorem}\label{theorem_8.2.5}
  Let $F$ be a field, and $\faktor{L}{K}$ and $\faktor{K}{F}$ be
  extensions. Then
  \begin{equation*}
    [L:F]=[L:K][K:F]
  \end{equation*}
\end{theorem}
\begin{proof}
  Take $[L:K]=m$ and $[K:F]=n$, and let $\{ \a_1, \dots, \a_m\}$ and
  $\{ \b_1, \dots, \b_n\}$ be bases for the extensions $\faktor{L}{K}$
  annd $\faktor{K}{F}$, respectively. We claim that $\{\a_1\b_1,
  \dots, \a_m\b_m\}$ is a basis for the extension $\faktor{L}{F}$.

  Indeed, observe that every element of $L$ has the form:
  \begin{equation*}
    a_1\a_1+\dots+a_m\a_m \text{ where } a_1, \dots, a_m \in K
  \end{equation*}
  and every element $a_i \in K$ has the form:
  \begin{equation*}
    a_i=b_{i1}\b_1+\dots+b_{in}\b_n \text{ where } b_{i1}, \dots, b_{in} \in F
  \end{equation*}
  therefore, every element of $L$ has the form
  \begin{equation*}
    \sum{b_{ij}\a_i\b_j} \text{ where } b_{ij} \in F
  \end{equation*}
  Therefore $\Span_F{L}=\{\a_1\b_1, \dots, \a_m\b_n\}$.

  Now, suppose that
  \begin{equation*}
    \sum{b_{ij}\a_i\b_j}=0 \text{ for } b_{ij} \in F
  \end{equation*}
  taking $a_i=b_{i1}\b_1+\dots+b_{in}\b_n$ in $K$, since $\{\a_1,
  \dots, \a_m\}$ is linearly independent over $K$ we get
  \begin{equation*}
    a_1\a_1+\dots+a_m\a_m=0 \text{ for } a_1, \dots, a_m \in K
  \end{equation*}
  implies $a_1=\dots=a_m=0$. Hence
  \begin{equation*}
    b_{i1}\b_1+\dots+b_{in}\b_n=0 \text{ for } b_{i1}, \dots, b_{in} \in F
  \end{equation*}
  for all $1 \leq i \leq m$. Since $\{\b_1, \dots, \b_n\}$ is
  linearly independent over $F$, we get $b_{i1}=\dots=b_{in}=0$ for
  all $1 \leq i \leq m$. This makes the spanning set $\{\a_1\b_1,
  \dots, \a_m\b_n\}$ linearly independent over $F$. Therefore
  $\{\a_1\b_1, \dots, \a_m\b_n\}$ is a basis for $\faktor{L}{F}$. Now,
  $\{\a_1\b_1, \dots, \a_m\b_n\}$ has $mn$ elements since $\{\a_1,
  \dots, a_m\}$ has $m$ elements and  $\{\b_1, \dots, \b_n\}$ has $n$
  elements. Therefore, as a vector space, $\dim_F{L}=mn=\dim_K{L} \cdot
  \dim_F{K}$. That is, $[L:F]=[L:K][K:F]$.
\end{proof}
\begin{corollary}
  The extension $\faktor{L}{F}$ is finite if both $\faktor{L}{K}$ and
  $\faktor{K}{F}$ are finite. Moreover, we have $[K:F] \divides
  [L:F]$.
\end{corollary}

\begin{example}\label{example_8.7}
  \begin{enumerate}
    \item[(1)] Let $\a$ be a root of $x^3-3x-1$ over $\Q$. Then
      $\sqrt{2} \notin \Q(\a)$. Now, $[\Q(\sqrt{2}):\Q]=2$
      and $[\Q(\a):\Q]=3$. Since $2 \nmid 3$, $\Q(\sqrt{2})
      \not\subseteq \Q(\a)$.

    \item[(2)] Let $\sqrt[6]{2}$ the root of $x^6-2$ over $\Q$. Then
      $[\Q(\sqrt[6]{2}):\Q]=6=2 \cdot 3$. Now,
      $(\sqrt[6]{2})^3=\sqrt{2}$, so that $\Q(\sqrt{2}) \subseteq
      \Q(\sqrt[6]{2})$, and we have the tower
      \[\begin{tikzcd}
        {\Q(\sqrt[6]{2})} \\
        {\Q(\sqrt{2})} \\
        \Q
        \arrow[no head, from=2-1, to=1-1]
        \arrow[no head, from=3-1, to=2-1]
      \end{tikzcd}\]
      where $[\Q(\sqrt[6]{2}):\Q(\sqrt{2})]=3$.
  \end{enumerate}
\end{example}

\begin{definition}
  Let $F$ be a field. We call an extension $K$ of $F$
  \textbf{finitely generated} if there exist $\a_1, \dots, \a_n \in K$
  for which $K=F(\a_1, \dots, \a_n)$.
\end{definition}

\begin{lemma}\label{lemma_8.2.6}
  Let $F$ be a field, and $\a$ and $\b$ elements of some extension of
  $F$. Then $F(\a,\b)=(F(\a))(\b)$.
\end{lemma}
\begin{proof}
  Observe that $F(\a) \subseteq F(\a,\b)$ and $\b \in F(\a,\b)$ so
  that we have $(F(\a))(\b) \subseteq F(\a,\b)$. Now, $F \subseteq
  (F(\a))(\b)$ and $\a,\b \in (F(\a))(\b) \subseteq F(\a,\b)$. Since
  $F(\a,\b)$ is the minimal such field containing $F$, and $\a$, and
  $\b$ we get the reverse inclusion  $F(\a,\b) \subseteq (F(\a))(\b)$.
\end{proof}
\begin{corollary}
  If $\a_1, \dots, \a_{n+1}$ are elements of some extension of $F$,
  then $F(\a_1, \dots, \a_n, \a_{n+1})=(F(\a_1, \dots,
  \a_n))(\a_{n+1})$.
\end{corollary}
\begin{proof}
  We proceed recursively from above.
\end{proof}
\begin{corollary}
  If $\faktor{K}{F}$ is an extension, and $\a_1, \dots, \a_n \in K$,
  then there exists a tower of extensions:
  \begin{equation*}
    F=F_0 \subseteq F_1 \subseteq \dots \subseteq F_n=K
  \end{equation*}
  wehere $F_{i+1}=F_i(\a_{i+1})$. Moreover, if $\a_1, \dots, \a_n$ are
  algebraic over $F$, with $[F_{i+1}:F_i]=m_i$, then
  $\faktor{F_{i+1}}{F_i}$ is a simple extension and $[K:F]=m_1 \dots
  m_n$.
\end{corollary}

\begin{example}\label{example_8.8}
  \begin{enumerate}
    \item[(1)] $\Q(\sqrt[6]{2}, \sqrt{2})=\Q(\sqrt[6]{2})$ since
      $\sqrt{2}=(\sqrt[6]{2})^3$.

    \item[(2)] Observe $\Q(\sqrt{2},\sqrt{3})$ over $\Q$. We have
      $\deg{\sqrt{3}}=2$, and $[\Q(\sqrt{2},\sqrt{3}):\Q(\sqrt{2})]
      \leq 2$. Indeed, if $x^2-3$ is irreducible over $\Q(\sqrt{2})$,
      then $[\Q(\sqrt{2},\sqrt{3}):\Q(\sqrt{2})]=2$. Now, suppose that
      $\sqrt{3} \in \Q(\sqrt{2})$. Then
      \begin{equation*}
        \sqrt{3}=a+b\sqrt{2}
      \end{equation*}
      squaring both sides, we get
      \begin{equation*}
        3=(a^2+2b^2)+2ab\sqrt{2}
      \end{equation*}
      Taking $ab \neq 0$ we get
      \begin{equation*}
        \sqrt{2}=\frac{3}{ab}-\frac{a^2+2b^2}{ab}
      \end{equation*}
      which puts $\sqrt{2} \in \Q$, impossible. Now, if $b=0$, then
      $3=a^2$ so that $\sqrt{3} \in \Q$, again impossible. If $a=0$,
      then  $3=2b^2$ so that $\sqrt{6}=2b \in \Q$, which is again
      impossible. Therefore $\sqrt{3} \notin \Q(\sqrt{2})$. Hence
      $x^3-2$ is irreducible over $\Q(\sqrt{2})$. This make
      $[\Q(\sqrt{2},\sqrt{3}):\Q(\sqrt{2})]=2$. Then we have
      \begin{equation*}
        [\Q(\sqrt{2},\sqrt{3}):\Q]=[\Q(\sqrt{2},\sqrt{3}):\Q(\sqrt{2})]
        [\Q(\sqrt{2}):\Q]=2 \cdot 2=4
      \end{equation*}
      and we have the lattice of fields labeled with the corresponding
      degrees:
      \[\begin{tikzcd}
  & {\Q(\sqrt{2},\sqrt{3})} \\
        {\Q(\sqrt{2})} && {\Q(\sqrt{3})} \\
                       & \Q
                       \arrow["2", no head, from=2-1, to=1-2]
                       \arrow["2"', no head, from=2-3, to=1-2]
                       \arrow["4", no head, from=3-2, to=1-2]
                       \arrow["2", no head, from=3-2, to=2-1]
                       \arrow["2"', no head, from=3-2, to=2-3]
      \end{tikzcd}\]
  \end{enumerate}
\end{example}

\begin{theorem}\label{theorem_8.2.7}
  Let $F$ be a field. The extension $\faktor{K}{F}$ is finite if, and
  only if $K$ is generated by finitely many algebraic elements over
  $F$.
\end{theorem}
\begin{proof}
  Suppose that $\faktor{K}{F}$ is finite, with $[K:F]=n$, and let $\{
  \a_1, \dots, \a_n \}$ be a basis for $\faktor{K}{F}$. Construct the
  tower of fields
  \begin{equation*}
    F=F_0 \subseteq F_1 \subseteq \dots \subseteq F_n=K
  \end{equation*}
  where $F_{i+1}=F_i(\a_{i+1})$ for all $0 \leq i \leq n-1$. Then
  $[F_{i+1}:F] \divides [K:F]$, so the extension $\faktor{F_{i+1}}{F}$ is
  finite. Moreover, since $\faktor{F_{i+1}}{F}$ is a simple extension,
  this makes $\a_{i+1}$ algebraic over $F$ for all $0 \leq i \leq
  n-1$.

  Now, suppose we have a finite set $\{\a_1, \dots, \a_n\}$ of
  elements of $K$ which are algebraic over $F$. Suppose that
  $K=F(\a_1, \dots, \a_n)$. Then by definition, we get the
  corresponding tower of fields
  \begin{equation*}
    F=F_0 \subseteq F_1 \subseteq \dots \subseteq F_n=K
  \end{equation*}
  where $F_{i+1}=F_i(\a_{i+1})$ for all $0 \leq i \leq n-1$. Now, the
  extensions $\faktor{F_{i+1}}{F_i}$ are simple extensions, take
  $[F_{i+1}:F_i] \leq m_i$, where $\deg{\a_i}=m_i$. Then we get
  \begin{equation*}
    [K:F]=[F_n:F_{n-1}]\dots[F_1,F_0]\leq m_1 \dots m_n
  \end{equation*}
  which makes $\faktor{K}{F}$ a finite extension.
\end{proof}
\begin{corollary}
  The set of all elements algebraic over $F$ forms an extension field
  of $F$.
\end{corollary}
\begin{proof}
  Let $K$ be the collection of all elements which are algerbaic over
  $F$ (over any given extension). Then $F \subseteq K$. Now, observe
  for any two elements $\a$ and $\b$ algebraic over $F$, that $\a \pm
  \b$, $\a\b$ are algebraic over $F$. Additionally, $\a\inv{\b}$ is
  algebraic over $F$, provided  $\b \neq 0$ and $\inv{\a}$ is
  algebraic over $F$, provided that $\a \neq 0$.
\end{proof}

\begin{example}\label{example_8.9}
  \begin{enumerate}
    \item We define the \textbf{field of algebraic numbers} over $\Q$
      to be the set $\bar{\Q}$ of all elements of $\C$ which are
      algebraic over $\Q$.

      Now, $\bar{\Q}$ is algebraic over $\Q$ by definition. In
      particular, observe for any $n \in \Z^+$, that $\sqrt[n]{2}
      \in \bar{\Q}$, so that $[\bar{\Q}:\Q] \geq n$ and
      $\faktor{\bar{\Q}}{\Q}$ is an infinite algebraic extension of
      $\Q$.

    \item[(2)] Recall that $\Q$ is countable, hence $\Q[x]$ is also
      countable. Now, let $f(x) \in \Q[x]$ a polynomial of
      $\deg{f}=n$. Then $f(x)$ has at most $n$ roots in $\R$. Now, let
      $\bar{\Q}_n$ the set of all algebraic elements (in $\R$) over
      $\Q$ of degree $n$. Then $\bar{\Q}_n$ is countable, and observe
      that
      \begin{equation*}
        \bar{\Q} \cap \R=\bigcup_{n \in \Z^+}{\bar{\Q}_n}
      \end{equation*}
      is a countable union of countable sets. This makes $\bar{\Q}
      \cap \R$ countable. Now, observe that since $\R$ is uncountable,
      there exist uncountably many numbers in  $\R$ which are not
      algebraic over $\Q$.
  \end{enumerate}
\end{example}

\begin{theorem}\label{theorem_8.2.8}
  Let $F$ be a field. If $K$ is an algebraic extension over $F$, and
  $L$ an algebraic extension over  $K$, then  $L$ is an algerbaic
  extension over  $F$.
\end{theorem}
\begin{proof}
  Let $\a \in L$. Then $\a$ is algebraic over  $K$, so there exists a
  polynomial $f(x)=a_0+a_1x+\dots+a_nx^n$ over $K$, for which
  \begin{equation*}
    a_0+a_1\a+\dots+a_n\a^n=0 \text{ where } a_0, \dots, a_n \in K
  \end{equation*}
  Since each $a_i$ is algebraic over $F$, consider the field
  $F''=F(\a,a_0,\dots,a_n)$. Let $F'=F(a_0, \dots, a_n)$. Then the
  extension $\faktor{F''}{F}$ is finite. Moreover, since $\deg{\a}
  \leq n$, we have:
  \begin{equation*}
    [F'':F]=[F'':F'][F':F]
  \end{equation*}
  is also finite, hence $\faktor{F'}{F}$ is algebraic. Since $\a$ is
  algebraic over $F$ this makes the extension $\faktor{L}{F}$
  an algebraic extension.
\end{proof}

\begin{definition}
  Let $K_1$ and $K_2$ be subfields of a field $K$. We define the
  \textbf{composite field} of $K_1$ and $K_2$ to be the smallest
  subfield of $K$ containing both $K_1$ and $K_2$. We denote the
  composite field by $K_1K_2$.
\end{definition}

\begin{example}\label{example_8.10}
  Since $(\sqrt[6]{2})^3=\sqrt{2}$ and $(\sqrt[6]{2})^2=\sqrt[3]{2}$,
  we get the composite field
  $\Q(\sqrt[6]{2})=\Q(\sqrt{2})\Q(\sqrt[3]{2})$.
\end{example}

\begin{proposition}\label{proposition_8.2.9}
  If $K_1$ and $K_2$ are finite extensions of a field $F$, and  $\{
  \a_1, \dots, \a_m \}$ is a basis of $\faktor{K_1}{F}$ and $\{ \b_1,
  \dots, \b_n \}$ is a basis of $\faktor{K_2}{F}$, then
  \begin{equation*}
    K_1K_2=F(\a_1, \dots, \a_m,\b_1, \dots, \b_n)
  \end{equation*}
  and
  \begin{equation*}
    [K_1K_2:F] \leq [K_1:F][K_2:F]
  \end{equation*}
  where equality holds if, and only if the span of $K_1K_2$,
  $\{\a_1\b_1, \dots, \a_m\b_n\}$ is linearly independent.
\end{proposition}
\begin{proof}
  By hypothesis, we have $[K_1:F]=m$ and $[K_2:F]=n.$ Now since both
  extensions $\faktor{K_1}{F}$ and $\faktor{K_2}{F}$ are finite, they
  are finitely generated by their respective bases:
  \begin{align*}
    K_1 &=  F(\a_1, \dots, \a_m)  \\
    K_2 &=  F(\b_1, \dots, \b_n)  \\
  \end{align*}
  so we have
  \begin{align*}
    K_1K_2  &=  F(\a_1, \dots, \a_m)F(\b_1, \dots, \b_n)  \\
            &= F(\a_1, \dots, \a_m)(\b_1, \dots, \b_n) \\
            &= F(\a_1, \dots, \a_m, \b_1, \dots, \b_n) \\
  \end{align*}
  Now, observe that
  \begin{align*}
    \a_i^k=a_1\a_1+\dots+a_m\a_m \text{ where } a_1 \dots, a_m \in F \\
    \b_j^k=b_1\b_1+\dots+b_n\b_n \text{ where } b_1 \dots, b_n \in F \\
  \end{align*}
  taking $c_{ij}=a_ib_j$, for all $1 \leq i \leq m$ and $1 \leq j \leq n$,
  we get
  \begin{equation*}
    (\a_i\b_j)^k=\sum_{i,j=1}^{m,n}{c_{ij}\a_ib_j}
  \end{equation*}
  so that $\Span_F{K_1K_2}=\{\a_1\b_1, \dots \a_m\b_n\}$. Therefore
  $[K_1K_2:F] \leq mn$.

  Lastly, if $[K_1K_2:F]=mn$, then $\{\a_1\b_1, \dots, \a_m\b_n\}$ is
  linearly independent, and hence a basis for $K_1K_2$ over $F$. This
  makes $\{\a_1, \dots, \a_m\}$ linearly independent over $K_2$ and
  $\{\b_1, \dots, \b_n\}$ linearly independent over $K_1$. Now,
  observing that $K_1K_2=K_1(\b_1, \dots, \b_n)$, $\{\b_1 \dots,
  \b_n\}$ spans $K_1K_2$ and hence is a basis over $K_1K_2$ over $K_1$.
  Similarly $\{\a_1, \dots, \a_m\}$ is a basis for $K_1K_2$ over
  $K_2$. The converse follows easily, considering the lattice with the
  corresponding degrees:
  \[\begin{tikzcd}
  & {K_1K_2} \\
    {K_1} && {K_2} \\
          & F
          \arrow["{\leq n}", from=2-1, to=1-2]
          \arrow["{\leq m}"', from=2-3, to=1-2]
          \arrow["m", from=3-2, to=2-1]
          \arrow["n"', from=3-2, to=2-3]
  \end{tikzcd}\]
\end{proof}
\begin{corollary}
  If $(m,n)=1$, then $[K_1K_2:F]=[K_1:F][K_2:F]$.
\end{corollary}
\begin{proof}
  Observe that $m,n \divides [K_1K_2:F]$, so that $mn \divides
  [K_1K_2:F]$. Since $[K_1K_2:F] \leq mn$ we are done.
\end{proof}

\begin{example}\label{example_8.11}
  $[\Q(\sqrt[6]{2}):Q]=[\Q(\sqrt{2}):\Q][\Q(\sqrt{3}):\Q]=2 \cdot
  3=6$.
\end{example}
