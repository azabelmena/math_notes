\section{Algebraic Extensions}
\label{section_8.2}

\begin{definition}
  Let $\faktor{K}{F}$ be a field extension. We say that an element $\a \in K$
  is  \textbf{algebraic} over $F$, provided there exists a polynomial over $F$
  having $\a$ as a root. Otherwise we call $\a$  \textbf{transcendental}. If
  every $\a \in K$ is algebraic, we call $K$ \textbf{algebraic} and
  $\faktor{K}{F}$ an \textbf{algebraic extension}.
\end{definition}

\begin{lemma}\label{lemma_8.2.1}
  Let $\a$ be algebraic over a field  $F$. Then there exista a unique monic
  irreducible polynomial  $m \in F[x]$ having $\a$ as a root. Moreover, if  $f
  \in F[x]$ is a polynomial, then $f$ has $\a$ as a root if, and only if $m|f$.
\end{lemma}
\begin{proof}
  Let $m$ a polynomial of minimal degree having  $\a$ as a root. Suppose, also
  that  $,$ is monic. Now, if  $m$ were reducible, then $m(x)=a(x)b(x)$ for
  some $a,b \in F[x]$ polynomials both of degree less than $\deg{m}$. Then we
  also have that $a(\a)=b(\a)=0$, which contradicts that $m$ is the polynomial
  of minimal degree satisfying that condition. Hence,  $m$ is irreducible.

  Now, let  $f \in F[x]$ have $\a$ as a root, then by the divison theorem,
  there exist  $q,r \in F[x]$, with $q \neq 0$ for which
  \begin{equation*}
    f(x)=q(x)m(x)+r(x) \text{ where } \deg{r}<\deg{m}
  \end{equation*}
  Now, since $f(\a)=q(\a)m(\a)+r(\a)=0$, then $r(x)=0$ for all $x$ lest we
  contradict the minimality of $m$. Hence $m|f$. Conversely, if $m|f$, then
  $f$ has $\a$ as a root.

  Now, let $g$ a polynomial of minimal degree for which  $g(\a)=0$. Then by
  above, we have that $\deg{g}=\deg{m}$, and that moreover, $m|g$ and  $g|m$.
  therefore  $g=m$ and uniqueness is established.
\end{proof}
\begin{corollary}
  Let $\faktor{L}{F}$ be an extension, and $\a$ algebraic over $F$. Let
  $m_{\a,F}$ the unique monic irreducible polynomial over $F$ having  $\a$ as
  root, and  $m_{\a,L}$ the unique monic irreducible polynomial over $L$
  having  $\a$ as root. Then $m_{\a,L}|m_{\a,F}$ in $L[x]$.
\end{corollary}

\begin{definition}
  Let $F$ be a field, and $\a$ algebraic over $F$. We define the
  \textbf{minimal polynomial} $m_{\a,F}$, to be the polynomial over $F$ of
  minimal degree having  $\a$ as a root. If the field is clear, we instead
  write  $m_\a$, or even just  $m$ when the root itself is also clear. We
  define the  \textbf{degree} of $\a$ to be  $\deg{\a}=\deg{m_{\a}}$.
\end{definition}

\begin{lemma}\label{lemma_8.2.2}
  Let $\a$ algebraic over  $F$. Then
  \begin{equation*}
    F(\a) \simeq \faktor{F[x]}{(m_{\a,F})}
  \end{equation*}
\end{lemma}
\begin{corollary}
  $[F(\a) : F]=\deg{m_\a}=\deg{\a}$.
\end{corollary}

\begin{example}\label{example_7.5}
  \item[(1)] The minimal polynomial for $\sqrt{2}$ over $\Q$ is $x^2-2$.

  \item[(3)] The minimal polynomial for $\sqrt[3]{2}$ over $\Q$ is $x^3-2$.

  \item[(3)] Let $n>1$, then by the Eisenstein-Sch\"omann criterion, $x^n-2$
    is irreducible over  $\Q$. Moreover, $x^n-2$ has as root in  $\R$
    $\sqrt[n]{2}$. Then $\Q(\sqrt[n]{2})$ is a field of degree
    $[\Q(\sqrt[n]{2}) : \Q]=2$. Moreover $x^n-2$ is the minimal polynomial
    of  $\sqrt[n]{2}$. Notice, that over $\R$,  $\deg{\sqwrt[n]{2}}=1$, and
    that $m_{\sqrt[n]{2},\R}(x)=x-\sqrt[n]{2}$.

  \item[(4)] Consider $p(x)=x^3-3x-1$ over $\Q$. Notice that  $p$ is
    irreducible over  $\Q$ and let $\a$ a root of  $p$. Then
    $[\Q(\a):\Q]=3$.
  \end{example}

  \begin{lemma}\label{lemma_8.2.3}
    An element $\a$ is algebraic over a field  $F$ if, and only if the simple
    extension  $\faktor{F(\a)}{F}$ is finite.
  \end{lemma}
  \begin{proof}
    If $\a$ is algebraic over $F$ then  $[F(\a):F]=\deg{\a} \leq n$ if $\a$
    satisfies a polynomial of degree  $n$. Conversely, if  $\a$ is an element
    of the finite extenson  $\faktor{K}{F}$, of degree $n$, then the set
    $\{1,\a, \dots, \a^n\}$ is linearly dependent over $F$. Hence there exist
    $b_0, \dots, b_n \in F$ not all $0$ for which
    \begin{equation*}
      b_0+b_1\a+\dots+a_n\a^n=0
    \end{equation*}
    making $\a$ a root of a nonzero polynomial over $F$ of degree $\deg \leq n$.
  \end{proof}
  \begin{corollary}
    If an extension $\faktor{K}{F}$ is finite, then it is algebraic.
  \end{corollary}
  \begin{proof}
    If $\a \in K$ is algebraic, then  $\faktor{K}{F}$ implies that
    $\faktor{F(\a)}{F}$ is finite, since $F(\a) \subseteq K$.
  \end{proof}

  \begin{example}\label{example_8.6}
    Let $F$ a field of  $\Char{F} \neq 2$, and let $K$ an extension field of
    $F$ of degree $[K:F]=2$. Let $\a \in K$  not in $F$, then  $\a$ satisfies
    an polynomial of at most degree  $2$ over $F$. Now, since  $\a \notin F$,
    this polynomial must have degree greater than $1$. Hence it satisfies a
    polynomial of degree $2$. Then the minimal polynomial of  $\a$ is a
    quadratic
    \begin{equation*}
      m_\a(x)=x^2+bx+c \text{ with } b,c \in F
    \end{equation*}
    Since $F \subseteq F(\a) \subseteq K$, and $F(\a)$ is a vector space over
    $F$ of dimension  $2$, then we must have  $K=F(\a)$; that is
    $\faktor{K}{F}$ is simple.

    Now, the roots of $m_\a$ are
    \begin{equation*}
      \a=\frac{-b \pm \sqrt{b^2-4c}}{2}
    \end{equation*}
    Since $\a \notin F$,  $b^2-4c$ is not a square in $F$, and
    $\sqrt{b^2-4c}$ is a root of the equation $x^2-(b^2-4c)=0$ in $K$.

    Conversely,  $\sqrt{b^2-4c}=\pm (b+2\a)$ which puts $\sqrt{b^2-4c} \in
    F(\a)$. That is $F(\sqrt{b^2-4c})=\F(\a)$. Moreover, $x^2-(b^2-4c)$ does
    not have solutions in $K$.

    We call field extensions  $\faktor{K}{F}$ of degree $2$ \textbf{quadratic
    field extension}, where $K=F(\sqrt{D})$, and $D$ is a squarefree element of
    $F$.
  \end{example}

  \begin{theorem}\label{theorem_8.2.4}
    Let $F \subseteq K \subseteq L$. Then  $[L:F]=[L:K][K:F]$.
  \end{theorem}
  \begin{proof}
    Let $[L:K]=m$ and $[K:F]=n$. Let $\{\a_1, \dots, \a_m\}$ and $\{\b_1,
    \dots, \b_n\}$ be bases for the extensions $\faktor{L}{K}$ and
    $\faktor{K}{F}$. Now, the elements of $L$ over  $K$ are of the form
    \begin{equation*}
      a_1\a_1+\dots+a_m\a_m \text{ where } a_i \in K \text{ for all } 1 \leq
      i \leq m
    \end{equation*}
    Since each $a_i \in K$, which is an extension over  $F$, they have the form
    \begin{equation*}
      a_i=b_{i1}\b_{i1}+\dots+b_{in}\beta_{in} \text{ where } b_{ij} \in F
      \text{ for all } 1 \leq j \leq n
    \end{equation*}
    That is, every element of $L$, as a vector space over $F$ are of the form
    \begin{equation*}
      \sum{b_{ij}\a_i\b_j}
    \end{equation*}
    So the set $\{\a_1\b_1, \dots \a_m\b_n\}$ spans $L$. It remains to show that
    this set is linearly in dependent.

    Suppose that
    \begin{equation*}
      \sum{b_{ij}\a_i\b_j}=0
    \end{equation*}
    for some $b_{ij} \in F$. Since $\{\a_1, \dots, \a_m\}$ are linearly
    indpendent in $L$ over  $K$, we have that the coefficients
    $a_1=\dots=a_n=0$ which makes
    \begin{equation*}
      a_i=b_{i1}\b_{i1}+\dots+b_{in}\b_{in}=0
    \end{equation*}
    Now, since $\{\b_1, \dots, \b_n\}$ is linearly independent in $K$ over
    $F$, this implies that $b_{i1}=\dots=b_{in}=0$ which makes the collection
    $\{\a_1\b_1, \dots \a_m\b_n\}$ linearly independent, and hence, a basis.
    Moreover, notice that this basis has size $mn$.
  \end{proof}

  \begin{example}\label{example_8.7}
    \begin{enumerate}
      \item[(1)] The element $\sqrt{2} \notin \Q(\a)$, where $\a$ is the root
        of  $x^3-3x-1$; since  $[\Q(\sqrt{2}) : \Q]=2$, and
        $[\Q(\a):\Q]=3$.

      \item[(2)] We have $[\Q(\sqrt[6]{2}):\Q]=6$, and since
        $(\sqrt[6]{2})^3=\sqrt{2}$, we observe that $\Q(\sqrt{2}) \subseteq
        \Q(\sqrt[6]{2})$. Moreover, notice that by theorem \ref{8.2.4}
        $[\Q(\sqrt[6]{2}):Q(\sqrt{2})]=3$. Then we have the following
        tower of fields for
        \begin{equation*}
          \Q \subseteq \Q(\sqrt{2}) \subseteq \Q(\sqrt[6]{2})
        \end{equation*}
        \[\begin{tikzcd}
          {\Q(\sqrt[6]{2})} \\
          {\Q(\sqrt{2})} \\
          {\Q}
          \arrow[no head, from=1-1, to=2-1]
          \arrow[no head, from=2-1, to=3-1]
        \end{tikzcd}\]
    \end{enumerate}
  \end{example}

  \begin{lemma}\label{lemma_8.2.5}
    Let $\a,\b$ be algebraic over a field  $F$. Then  $F(\a,\b)=(F(\a))(\b)$.
  \end{lemma}
  \begin{proof}
    By definition, $F(\a,\b)$ contains $F$, and  $\a$, and hence contains
    $F(\a)$. It also contains $\b$ so that  $(F(\a))(\b) \subseteq  F(\a,\b)$.
    By the same argument, $(F(\a))(\b)$ contains $F$,  $\a$ and  $\b$ so that
    $F(\a,\b) \subseteq (F(\a))(b)$.
  \end{proof}
  \begin{corollary}
    The elements of $F(\a,\b)$ are of the form $\sum{a_{ij}\a^ib^j}$, where $1
    \leq i \leq \deg{\a}$ and $1 \leq j \leq \deg{\b}$.
  \end{corollary}

  \begin{example}\label{example_8.8}
    Consider $\Q(\sqrt{2},\sqrt{3})$ generated by $\sqrt{2}$ and $\sqrt{3}$.
    Notice that $\deg{\sqrt{3}}=2$ over $\Q$ so that $[\Q(\sqrt{2},\sqrt{3}) :
    \Q(\sqrt{2})] \leq 2$. Now $[\Q(\sqrt{2},\sqrt{3}):\Q(\sqrt{2})]=2$ if, and
    only if the polynomial  $x^2-3$ is irreducible over  $\Q(\sqrt{2})$. Then
    it is irreducible if, and onyl if $\sqrt{2} \in \Q(\sqrt{3})$. It can be
    shown that this is not the case by trying to find $a,b \in \Q$ for which
    $\sqrt{3}=a+b\sqrt{2}$. Moreover we have
    \begin{equation*}
      [\Q(\sqrt{2},\sqrt{3}):\Q]=4
    \end{equation*}
  \end{example}

  \begin{theorem}\label{theorem_8.2.6}
    An extension field $\faktor{K}{F}$ is finite if, and only if it is
    generated by finitely many algebraic elements over $F$.
  \end{theorem}
  \begin{proof}
    Let $\faktor{K}{F}$ finite of degree $n$, and  $\{\a_1, \dots, \a_n\}$ a
    basis. Then by theorem \ref{8.2.4},  $[F(\a_i):F]|[K:F]$ for all $1 \leq i
    \leq n$. So each  $\a_i$ is algebraic over  $F$. Then  $K$ is generated by
    finitely many algebraic elements.

    Conversely, let $K=F(\a_1, \dots, \a_k)=(F(\a_1, \dots a_{k-1}))(\a_k)$ We
    obtain $K$ by taking the extensions $\faktor{F_{i+1}}{F_i}$ iteratively,
    where $F_{i+1}=F_i(\a_{i+1})$, and obtain the sequence
    \begin{equation*}
      F=F_0 \subseteq \dots \subseteq F_k=K
    \end{equation*}
    Now, if the elements $\a_1, \dots, \a_k$ are algebraic over $F$, each of
    $\deg{\a_i}=n_i$ for $1 \leq i \leq k$, then the extension
    $\faktor{F_{i+1}}{F_i}$ is a simple extension, and
    $[F_{i+1}:F_i]=\deg{m_{\a_{i+1}}} \leq \deg{\a_{i+1}}=n_{i+1}$. Then we
    have
    \begin{equation*}
      [K:F]=[F_k:F_{k-1}] \dots [F_1,F] \leq n_1 \dots n_k
    \end{equation*}
    which makes $\faktor{K}{F}$ a finite extension.
  \end{proof}
  \begin{corollary}
    If $\a,\b$ are algebraic over $F$, then so are $\a \pm \b$,  $\a\b$, and
    $\a\inv{\b}$ (for $\b \neq 0$).
  \end{corollary}
  \begin{corollary}
    If $\faktor{L}{F}$ is an extension, then the collection of elements of $L$
    which are algebraic over  $F$ forms a subfield of  $L$.
  \end{corollary}

  \begin{example}\label{example_8.9}
    \begin{enumerate}
      \item[(1)] Consider the extension $\faktor{\C}{\Q}$, and let $\cl{\Q}$
        the subfield of all elements of $\C$ which are algebraic over
        $\Q$. Then $\sqrt[n]{2} \in \cl{Q}$ for all $n \geq 1$, so that
        $[\cl{\Q}:\Q] \geq n$. This makes $\cl{\Q}$ an infinite algebraic
        extension, and we call $\cl{\Q}$ the \textbf{field of algebraic
        numbers}.

      \item[(2)] Consider $\cl{\Q} \cap \R$ as a subfield of $\R$ (i.e. the
        subfield of all algebraic elements of $\Q$). Since  $\Q$ is
        countable, so is the field  $\Q[x]$, and each polynomial in $\Q[x]$
        has at most $n$ roots in  $\R$, hence the number of all algebraic
        elements of $\R$ over $\Q$ is also countable. This means that
        $\cl{\Q}$ must also be countable. Now, since $\R$ is uncountable,
        then there exist uncountably transcendental numbers of  $\R$ over
        $\Q$. Most notably the irrational numbers  $\pi$ and  $e$ are
        transcendental.
    \end{enumerate}
  \end{example}

  \begin{theorem}\label{theorem_8.2.7}
    If $K$ is algebraic over  $F$, and  $L$ algebraic over  $K$, then  $L$ is
    algebraic over  $F$.
  \end{theorem}
  \begin{proof}
    Let $\a \in L$, since  $L$ is algebraic over  $K$, there exists a  $p \in
    K[x]$ having $\a$ as root. Let $p(x)=a_0+a_1x+\dots+a_nx^n$. Consider then
    $F(\a,a_0, \dots, a_n)$. Since $\faktor{K}{F}$ is algebraic, $a_0, \dots,
    a_n$ are algebraic over $F$, and so  $F(\a, a_0, \dots, a_n)$ is a finite
    extension over $F$. Then $\a$ generates an extension field of degree less
    than $n$, and we get
    \begin{equation*}
      [F(\a, a_0, \dots, a_n):F]=
      [F(\a, a_0, \dots, a_n):F(a_0, \dots, a_n)][F(a_0, \dots,a_n):F]
    \end{equation*}
    is finite, and $F(\a,a_0, \dots, a_n)$ is algebraic over $F$. That is,
    $\a$ is algebraic over  $F$, and so  $L$ is algbraic over  $F$.
  \end{proof}

  \begin{definition}
    Let $K_1$ and $K_2$ subfields of a field $K$. The  \textbf{composite field}
    $K_1K_2$ is the smallest subfield of $K$ containing both  $K_1$ and $K_2$.
  \end{definition}

  \begin{example}\label{example_8.10}
    The composite field of $\Q(\sqrt[3]{2})$ and $\Q(\sqrt{2})$ is
    $\Q(\sqrt[6]{2})$.
  \end{example}

  \begin{lemma}\label{lemma_8.2.8}
    Let $K_1$ and $K_2$ be extensions of a field $F$ contained in a field  $K$.
    Then  $[K_1K_2:F] \leq [K_1:F][K_2:F]$ with equality holding if, and only
    if a basis of $F$ in the other field is linearly independent. Moreover if
    $\{\a_1, \dots, \a_m\}$ and $\{\b_1, \dots, \b_n\}$ are bases for $K_1$ and
    $K_2$, then $\{\a_1,\b_1, \dots, \a_m\b_n\}$ span $K_1$ and $K_2$.
    \[\begin{tikzcd}
        & {K_1K_2} \\
      K_1 && K_2 \\
          & F
          \arrow["{\leq m}"', no head, from=1-2, to=2-1]
          \arrow["n"', no head, from=2-1, to=3-2]
          \arrow["m"', no head, from=3-2, to=2-3]
          \arrow["{\leq n}"', no head, from=2-3,to=1-2]
    \end{tikzcd}\]
  \end{lemma}
  \begin{corollary}
    If $[K_1:F]=m$, and $[K_2:F]=n$ with $m$ and  $n$ coprime, then
    $[K_1K_2:F]=[K_1:F][K_2:F]$.
  \end{corollary}
  \begin{proof}
    We have that $m,n|[K_1K_2:F]$ and since $K_1,K_2 \subseteq K_1K_2$ are
    subfields of $K_1K_2$, we get the least common multiple $[m,n]|[K_1K_2:F]$.
    Now, since $(m,n)=1$, we get $[m,n]=mn$ so that $mn \leq =[K_1K_2:F]$.
  \end{proof}
