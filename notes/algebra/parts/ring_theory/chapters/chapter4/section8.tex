\section{Multivariate Polynomials over Fields}

\begin{theorem}\label{theorem_8.8.1}
  Let $F$ be a field, and $\af$ a proper ideal in $k[x_1, \dots,
  x_n]$. Then the canonical map $\pi:F[x_1, \dots, x_n] \xrightarrow{}
  \faktor{F[x_1, \dots, x_n]}{\af}$ restricts to a homomorphism from
  $F$ into $\faktor{F[x_1, \dots, x_n]}{\af}$.
\end{theorem}
\begin{proof}
  For any $c \in F$, consider the map $\phi(c)=c+\sum{0x_1^{d_1} \dots
  x_n^{d_n}}$ in $F[x_1, \dots, x_n]$. Then $\phi:F \xrightarrow{}
  F[x_1, \dots, x_n]$ defines a field homomorphism. Now observe that
  $\pi \circ \phi(c)=(c+\sum{0x_1^{d_1} \dots x_n^{d_n}}) \mod{\af}
  \equiv c \mod{\af}$. Indeed, we observe that $\pi \circ \phi$ is a
  homormophism, and that $\pi|_F=\pi \circ \phi$.
\end{proof}
\begin{corollary}
  $\faktor{F[x_1, \dots, x_n]}{\af}$ contains $F$ as a subfield. In
  particular, $\faktor{F[x_1, \dots, x_n]}{\af}$ is a vector space
  over $F$.
\end{corollary}
\begin{proof}
  Observe that $\pi \circ \phi$ is a field homomorphism, and hence
  1--1 by proposition \ref{proposition_8.1.4}. The vector field
  axioms follow.
\end{proof}

\begin{definition}
  Let $F$ be a field, and let $f(x_1, \dots ,x_n) \in F[x_1, \dots, x_n]$
  be written as:
  \begin{equation*}
    f(x_1, \dots, x_n)=\sum_{m=0}^d{f_mx_i^{m}}
  \end{equation*}
  where $f_m(x_1, \dots, x_{i-1},x_{i+1}, \dots x_n) \in F[x_1, \dots,
  x_{i-1},x_{i+1}, \dots, x_n]$. We define the \textbf{derivative} of
  $f(x_1, \dots, x_n)$ with respect to $x_i$ to be the polynomial
  $D_{x_i}{f(x_1, \dots, x_n)} \in F[x_1, \dots, x_n]$ of the form:
  \begin{equation*}
    D_{x_i}{f(x_1, \dots, x_n)}=\sum_{m=0}^{d-1}{mf_mx_i^{m-1}}
  \end{equation*}
\end{definition}

\begin{proposition}\label{proposition_8.8.2}
  Let $F$ be a field. Then the following hold for all $f(x_1, \dots,
  x_n), g(x_1, \dots, x_n) \in F[x_1, \dots, x_n]$:
  \begin{enumerate}
    \item[(1)] $D_{x_i}{(af(x_1, \dots, x_n)+bg(x_1, \dots, x_n))}=
      aD_{x_i}{f(x_1, \dots, x_n)}+bD_{x_i}{g(x_1, \dots, x_n)}$.

    \item[(2)] $D_{x_i}{(f(x_1, \dots, x_n)g(x_1, \dots, x_n))}=
      f(x_1, \dots, x_n)D_{x_i}{g(x_1, \dots, x_n)}+
      g(x_1, \dots, x_n)D_{x_i}{g(x_1, \dots, x_n)}$.

    \item If $D_{x_i}{f(x_1, \dots, x_n)}=0$ for all $1 \leq i \leq n$
      then $f(x_1, \dots, x_n)$ is constant.
  \end{enumerate}
\end{proposition}
\begin{proof}
  Set $S=R[x_1, \dots, x_{i-1},x_{i+1}, \dots, x_n]$. Then $R[x_1,
  \dots, x_n]=S[x_i]$ and $f(x_1, \dots, x_n) \in S[x_i]$ is a
  polynomial in $x_i$ with coefficients being polynomials in $S$.
  Hence $D_{x_i}{f(x_1, \dots, x_n)}=D{f(x_i)}$ in $S[x_i]$, so that
  (1) and (2) follows from proposition \ref{proposition_8.6.1}. Now,
  suppose for some $1 \leq i \leq n$ that $D_{x_i}{f(x_1, \dots,
  x_n)}=0$, and write
  \begin{equation*}
    f(x_1, \dots, x_n)=\sum_{m=0}^{d}{f_mx_i^m}
  \end{equation*}
  where $f_m(x_1, \dots, x_{i-1},x_{i+1}, \dots, x_n) \in S$. Then
  $f_m(x_1, \dots, x_{i-1},x_{i+1}, \dots, x_n)=0$ for all $1 \leq m
  \leq d$, where $d$ is the degree of  $f$ as a polynomial in
  $S[x_i]$. This makes $f(x_1, \dots, x_n)$ constant in $x_i$. Hence
  if $D_{x_i}{f(x_1, \dots, x_n)}=0$ for all $1 \leq i \leq n$ then
  $f(x_1, \dots, x_n)$ is constant in all $x_i$; therefore $f(x_1,
  \dots, x_n)$ must be a constant polynomial.
\end{proof}
\begin{corollary}
  The following are also true:
  \begin{enumerate}
    \item[(1)] For any $k \in \Z^+$, $D_{x_i}{f(x_1, \dots,
      x_n)^k}=kf(x_1, \dots, x_n)^{k-1}D_{x_i}{f(x_1, \dots, x_n)}$

    \item[(2)] If $g_1(x), \dots, g_n(x) \in F[x]$, then
      \begin{equation*}
        D{f(g_1(x), \dots, g_n(x))}=
        \sum_{i=1}^n{(D_{x_i}{f})(g_1(x), \dots, g_n(x))D{g_i(x)}}
      \end{equation*}
  \end{enumerate}
\end{corollary}
\begin{proof}
  (1) follows immediately from the above proposition. Now, observe for
  $g_1(x), \dots, g_n(x) \in F[x]$:
  \begin{equation*}
    f(g_1(x), \dots, g_n(x))=
    \sum_{m=0}^d{f_m^gg_i(x)^m}
  \end{equation*}
  where $f_m^g=f_m(g_1(x), \dots, g_{i-1}(x),g_{i+1}(x), \dots.
  g_n(x)) \in F[x]$.

  Indeed, observe that $f(g_1(x), \dots, g_n(x)) \in F[x]$ so that $d=n$.
  Take then $m=i$. Then we have
  \begin{align*}
    D{f(g_1(x), \dots, g_n(x))}=
    \sum_{i=1}^n{if_i^gg_i(x)^{i-1}D{g_i(x)}}
    =\sum_{i=1}^n{(D_{x_i}{f})(g_1(x), \dots, g_n(x))D{g_i(x)}}
  \end{align*}
  as required.
\end{proof}
\begin{corollary}
  If
  \begin{equation*}
    f(x_1, \dots, x_n)=\sum{a_dx^d}
  \end{equation*}
  where $d=(d_1, \dots, d_n)$ and $x^d=x_1^{d_1} \dots x_n^{d_n}$
  then:
  \begin{equation*}
    f(x_1, \dots, x_n)=\sum{d_ia_dx_1^{d_1} \dots x_i^{d_i-1} \dots x_n^{d_n}}
  \end{equation*}
\end{corollary}
\begin{proof}
  This follows directly from definition by expanding the coefficient
  terms in $S=F[x_1, \dots, x_{i-1},x_{i+1}, \dots, x_n]$.
\end{proof}

\begin{proposition}\label{proposition_8.8.3}
  Let $F$ be a field. Then for any $f(x_1, \dots, x_n) \in F[x_1,
  \dots, x_n]$:
  \begin{equation*}
    D_{x_j}{(D_{x_i}{f(x_1, \dots, x_n)})}=
    D_{x_i}{(D_{x_j}{f(x_1, \dots, x_n)})}
  \end{equation*}
\end{proposition}
\begin{proof}
  Let now $S=F[x_1, \dots, x_{i-1},x_{i+1}, \dots x_{j-1},x_{j+1},
  \dots,x_n]$, and consider $f(x_1, \dots, x_n) \in S[x_i,x_j]$. Then
  $f(x_1, \dots, x_n)$ has the form:
  \begin{equation*}
    f(x_1, \dots, x_n)=\sum_{l,m=0}^{d_1,d_2}{f_{l,m}x_i^{l}x_j^m}
  \end{equation*}
  Where $f_{l,m}(x_1, \dots, x_{i-1},x_{i+1}, \dots x_{j-1},x_{j+1},
  \dots,x_n) \in S$
  Then
  \begin{equation*}
    D_{x_i}{f(x_1, \dots, x_n)}=
    \sum_{l=1m=0}^{d_1,d_2}{lf_{l,m}x_i^{l-1}x_j^{m}}
  \end{equation*}
  so that:
  \begin{equation*}
    D_{x_j}{(D_{x_i}{f(x_1, \dots, x_n)})}=
    \sum_{l=1m=1}^{d_1,d_2}{lmf_{l,m}x_i^{l-1}x_j^{m-1}}
  \end{equation*}
  Likewise, reversing the order of $D_{x_i}$ and $D_{x_j}$, we get:
  \begin{equation*}
    D_{x_j}{f(x_1, \dots, x_n)}=
    \sum_{l=0m=1}^{d_1,d_2}{mf_{l,m}x_i^{l}x_j^{m-1}}
  \end{equation*}
  so that:
  \begin{equation*}
    D_{x_i}{(D_{x_j}{f(x_1, \dots, x_n)})}=
    \sum_{l=1m=1}^{d_1,d_2}{lmf_{l,m}x_i^{l-1}x_j^{m-1}}
  \end{equation*}
  which gives us the required result.
\end{proof}

\begin{theorem}[Euler's Theorem]\label{theorem_8.8.4}
  Let $F$ be a field. If  $f(x_1, \dots, x_n)$ is a form of degree $m$
  over $F$, then:
  \begin{equation*}
    mf(x_1, \dots, x_n)=\sum_{i=1}^n{x_iD_{x_i}{f(x_1, \dots, x_n)}}
  \end{equation*}
\end{theorem}
\begin{proof}
  Let $d=(d_1, \dots, d_n)$, and take $x^d=x_1^{d_1} \dots x_n^{d_n}$.
  Then
  \begin{equation*}
    x_iD_{x_i}{f(x_1, \dots, x_n)}=\sum{d_ia_dx^d}
  \end{equation*}
  so that: $x_iD_{x_i}{f(x_1, \dots, x_n)}=d_if(x_1, \dots, x_n)$.
  Since $f(x_1, \dots, x_n)$ is a form of degree $m$,
  $d_1+\dots+d_n=m$. Then we get
  \begin{align*}
    \sum_{i=1}^n{x_iD_{x_i}{f(x_1, \dots, x_n)}} &=  \sum_{i=1}^n{d_if(x_1, \dots, x_n)}\\
    &= (d_1+\dots+d_n)f(x_1, \dots, x_n) \\
    &= mf(x_1, \dots, x_n) \\
  \end{align*}
  and we are done.
\end{proof}

\begin{theorem}\label{theorem_8.8.5}
  Let $F$ be a field, and let $f(x_1, \dots, x_n) \in F[x_1, \dots, x_n]$.
  Then for some $a_1, \dots, a_n \in F$:
  \begin{equation*}
    f(x_1,\dots,x_n)=\sum{\l_{(i)}(x_1-a_1)^{i_1} \dots (x_n-a_n)^{i_n}}
  \end{equation*}
\end{theorem}
\begin{corollary}
  If $f(a_1, \dots, a_n)=0$ then
  \begin{equation*}
    f(x_1, \dots, x_n)=\sum_{i=1}^n{(x-a_i)g_i(x_1, \dots, x_n)}
  \end{equation*}
   For some $g_i(x_1, \dots, x_n) \in F[x_1, \dots, x_n]$ not
   necessarily unique.
\end{corollary}
\begin{proof.g}
  If $f(a_1, \dots, a_n)=0$, then $f(x_1, \dots, x_n) \in (x_1-a_1,
  \dots, x_n-a_n)$ by above.
\end{proof.g}
