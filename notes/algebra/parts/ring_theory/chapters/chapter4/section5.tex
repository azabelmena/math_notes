\section{Algebraic Closures}
\label{section_8.5}

\begin{definition}
  We define the \textbf{algebraic closure} of a field $F$ to be the algebraic
  extension, $\cl{F}$, over $F$ for which every polynomial over  $F$ splits.
  We call a field  $K$  \textbf{algebraically closed} if every polynomial over
  $K$ has at least one root in  $K$.
\end{definition}

\begin{lemma}\label{lemma_8.5.1}
  A field $K$ is algebraically closed if, and only if every polynomial over
  $K$ has all of its roots in  $K$.
\end{lemma}
\begin{proof}
  Certainly, if a polynomial $f$ over  $K$ contains all of its roots in  $K$,
  then  $K$ is algebraically closed, by definition.

  Now, suppose that  $K$ is algebraically closed, and let  $f$ a polynomial
  over  $K$. Then  $f$ contains at least one root in $K$. Hence
  $f(x)=(x-\a)f_1(x)$ for some root $\a$ of $f$, and where $f_1 \in K[x]$. But
  then by definition again, $f_1$ contains at least one root in $K$. Hence,
  we proceed until we exhaust all the roots of  $f$, and obtain that every
  root of  $f$ lies in  $K$.
\end{proof}
\begin{corollary}
  $K$ is algebraically closed if, and only if  $\cl{K}=K$.
\end{corollary}

\begin{lemma}\label{lemma_8.5.2}
  Let $F$ be a field, and  $\cl{F}$ its algebraic closure. Then $\cl{F}$ is
  algebraically closed; i.e. $\cl{(\cl{F})}=\cl{F}$.
\end{lemma}
\begin{proof}
  Let $f \in \cl{F}[x]$, and $\a$ a root of  $f$. Then  $\a$ generates all of
  $\cl{F}(\a)$, making $\cl{F}$ algebraic over $F$. Hence  $\a$ is algebraic
  over  $F$, but  $\a \in \cl{F}$, so that $\cl{(\cl{F})}=\cl{F}$.
\end{proof}

\begin{lemma}\label{lemma_8.5.3}
  For every field $F$, there exists an algebraically closed set containing
  $F$.
\end{lemma}
\begin{proof}
  Consider the polynomial ring $F[\dots, x_n, \dots]$ where $f(x_n)$ is a
  nonconstant polynomial over $f$. Consider the ideal  $(f)$. Then, if
  $(f)=(1)$, then
  \begin{equation*}
    g_1f_1(x_1)+\dots+g_nf_n(x_n)=1
  \end{equation*}
  where $g_i \in F[x_i]$. Then we get
  \begin{equation*}
    g_1(x_1, \dots, x_m)f_1(x_1)+\dots+g_n(x_1, \dots, x_m)f_n(x_n)=1
  \end{equation*}
  Now, let $F'$ an extension of  $F$ containing a root  $\a_i$ of  $f_i$. Then
  we observe that  $0=1$ in the above equation which is a blatant
  contradiciton. S o $(f)$ must be a proper ideal.

  Now, by Zorn's lemma, there exists a maximal ideal $M$ containing $I$. Then
  the quotient
  \begin{equation*}
    K_1=\faktor{F[\dots, x_n, \dots]}{M}
  \end{equation*}
  is a field containing an imbedding of $F$. Moreover,  $f$ has a root in
  $K_1$, so that $f(x_n) \in  (f) \subseteq M$. Then $K_1$ is a field in which
  every polynomial over $F$ has a root. Proceeding as before with  $K_1$, we
  obtain $K_2$ in which every polynomial over $K_1$ has a root. Hence,
  proceeding recursively, we obtain the sequence
  \begin{equation*}
    F=K_0 \subseteq K_1 \subseteq K_n \subseteq \dots
  \end{equation*}
  in which everypolynomial over $K_n$ has all its roots in $K_{n+1}$. Now, let
  \begin{equation*}
    K=\bigcup{K_n}
  \end{equation*}
  Then $F \subseteq K$, and every polynomial over $K$ has a root in  $K_N$,
  for  $N$ large enough; but  $K_N \subseteq K$, so that  $K$ is algebraically
  closed.
\end{proof}

\begin{lemma}\label{lemma_8.5.4}
  Let $K$ be algebraically closed, and let  $F \subseteq K$. Then the
  collection of elements of the algbraic closure $\cl{F}$ of $K$ that are
  algebraic over $F$ is an algebraic closure of $F$.
\end{lemma}
\begin{proof}
  By definition, $\faktor{\cl{F}}{F}$ is algebraic. Then every polynomial $f$
  over  $F$ splits over  $K$ into linear factors $(x-\a)$, where $\a$ is a
  root of $f$. So  $\a$ is algebraic over  $F$, and hence  $\a \in \cl{F}$.
  then all linear factors have a coefficient in $\cl{F}$, so that $f$ splits
  completely over  $\cl{F}$.
\end{proof}
\begin{corollary}
  Algebraic closures are unique up to isomorphism.
\end{corollary}

\begin{theorem}[The Fundamental Theorem of Algebra]\label{theorem_8.5.5}
  $\C$ is algebraically closed.
\end{theorem}
\begin{corollary}
  $\C$ contains the an algebraic closuder of any of its subfields.
\end{corollary}
