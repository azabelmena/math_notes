\section{Algebraic Closures}
\label{section_8.5}

\begin{definition}
  Let $F$ be a field. We call an extension field $K$ over $F$ an
  \textbf{algebraic closure} of $F$ if it is algerbaic over $F$, and
  any polynomial over $F$ splits completely over $K$.
\end{definition}

\begin{definition}
  A field $K$ is said to be  \textbf{algebraically closed} provided
  any polynomial over $K$ has at least one root in $K$.
\end{definition}

\begin{proposition}
  Let $K$ be a field. Then $K$ is algebraically closed if, and only if
  every polynomial over $K$ contains all its roots in $K$.
\end{proposition}
\begin{proof}
  Let $f(x) \in K[x]$, and let $\deg{f}=n$. Now, suppose  $K$ is
  algebraically closed. Then $f(x)$ has a root $\a_1 \in K$ and
  $f(x)$ splits as
  \begin{equation*}
    f(x)=(x-\a)f_1(x)
  \end{equation*}
  where $f_1(x) \in K[x]$ and $\deg{f_1}=n-1$. Now, by definition
  again, $f_1(x)$ has a root $\a_2 \in K$ so that $f(x)$ splits as:
  \begin{equation*}
    f(x)=(x-\a_1)(x-\a_2)f_2(x)
  \end{equation*}
  where $f_2(x) \in K[x]$ and $\deg{f_2}=n-2$. Proceeding recursively
  on each subsequent factor, we exhaust all the roots of $f(x)$, and
  obtain them as elements of $K$ by definition. Conversely, if every
  polynomial over $K$ contains all its roots in $K$, by definition $K$
  is algebraically closed.
\end{proof}
\begin{corollary}
  $K$ is algebraically closed if, and only if every polynomial over
  $K$ splits completely over $K$.
\end{corollary}
\begin{corollary}
  If $f(x) \in K[x]$ has degree $\deg{f}=n$, then $f(x)$ has exactly
  $n$-roots in $K$.
\end{corollary}

\begin{proposition}\label{proposition_8.5.2}
  If $K$ is algebraically closed, then  $K$ is an algebraic closure
  of itself.
\end{proposition}
\begin{proof}
  By the above proposition, we observe that if $K$ is algebraically
  closed, then every polynomial over $K$ splits completely over $K$,
  and that $K$ is algebraic over itself.
\end{proof}

\begin{proposition}\label{propostion_8.5.3}
  Let $K$ be an algebraic closure of a field $F$. Then $K$ is
  algebraically closed.
\end{proposition}
\begin{proof}
  Let $f(x) \in K[x]$, and let $\a$ a root of $f(x)$ (in some
  extension, not necessarily $K$). Then $\a$ generates the algebraic
  extension $K(\a)$ over $K$. Now by definition, since $K$ is
  algebraic over $F$,  we get that $K(\a)$ is algebraic over $F$, and
  hence $\a$ is algerbaic over $F$. This puts $\a \in K$, and hence
  $K$ must be algebraically closed.
\end{proof}

\begin{proposition}\label{proposition_8.5.4}
  Let $F$ be a field. Then there exists an algebraically closed field
  containing $F$.
\end{proposition}
\begin{proof}
  For any $f(x) \in F[x]$, denote $x_f$ an indeterminate. Now,
  consider the ideal $\ff \subseteq F[\dots, x_f,\dots]$ generated by
  those $f(x_f)$. If $\ff$ is not a proper ideal, so that $1 \in \ff$,
  then we get
  \begin{equation*}
    g_1(\dots, x_{f_1}, \dots)f_1(x_{f_1})+
    \dots+g_n(\dots, x_{f_n},\dots)f_n(x_{f_n})=1
  \end{equation*}
  where $g_i(\dots, x_{f_i}, \dots) \in F[\dots, x_f, \dots]$. Now,
  let $F'$ a finite extension containing roots $\a_{f_i}$ of
  $f_i(x_{f_i})$, then we get:
  \begin{equation*}
    g_1(\dots, \a_{f_1}, \dots)f_1(\a_{f_1})+\dots+
    g_n(\dots, \a_{f_n}, \dots)f_n(\a_{f_n})=0
  \end{equation*}
  which gives $1=0$ in $F$, which is impossible since $F$ is a field.
  Therefore $1 \notin \ff$ and  $\ff$ is a proper ideal. Hence by
  Zorn's proposition, $\ff$ is contained in a maximal ideal $\mf$. Now
  letting
  \begin{equation*}
    K_1=\faktor{F[\dots, x_f, \dots]}{\mf}
  \end{equation*}
  $K_1$ is a field containing an isomorphic copy of $F$. Moreover, a
  polynomial $f \in F[x_f]$ has a root in $K_1$, since $f \in \ff
  \subseteq \mf$.

  Repeating the above process with $K_1$, we obtain $K_2$, and
  proceeding inductively we obtaine the tower
  \begin{equation*}
    F=K_0 \subseteq K_1 \subseteq K_2 \subseteq \dots
    \subseteq K_j \subseteq \dots
  \end{equation*}
  where every polynomial iver $K_j$ has roots in $K_{j+1}$. Now, let:
  \begin{equation*}
    K=\bigcup_{j=0}^\infty{K_j}
  \end{equation*}
  Then $K$ is a field containing an isomorphic copy of $F$. Now if
  $h(x) \in K[x]$, then $h(x)$ has coefficients in some $K_N$ for $N
  \geq 1$ large enough, and hence has roots in $K_{N+1}$. That is
  $h(x)$ has roots in $K$. Therefore $K$ is algebraically closed.
\end{proof}

\begin{proposition}\label{proposition_8.5.5}
  Let $K$ be an algebraically closed field, and $F \subseteq K$ a
  subfield. Then the collection $\bar{F}$ of elements of $K$ which are
  algebraic over $F$ is an algebraic closure of $F$.
\end{proposition}
\begin{proof}
  By definition, $\bar{F}$ is algebraic over $F$. Now, observe that
  every polynomial $f(x)$ over $K$ splits completely over  $K$ into
  linear factors of the form  $x-\a$, with  $\a$ a root of  $f(x)$.
  This makes $\a$ algebraic over  $F$, and hence  $\a \in \bar{F}$.
  That is, $f(x)$ splits completely over $\bar{F}$, so that $\bar{F}$
  is an algebraic closure of $F$.
\end{proof}

\begin{theorem}\label{theorem_8.5.6}
  Let $F$ be a field. Then any two algebraic closures of $F$ are
  isomorphic.
\end{theorem}
\begin{proof}
  Let $K$ and $K'$ be algebraic closures of $F$. Define $\io:F
  \xrightarrow{} F$ to be the identity map on $F$. Now, by the
  corollary to theorem \ref{theorem_8.4.4}, any two splitting fields
  of a polynomial $f(x) \in F[x]$ are isomorphic. Now, let $\{E_f\}$
  the collection of all splitting fields over $F$, indexed by
  polynomials in $F[x]$. Define the maps $\io_f:E_f \xrightarrow{} K'$
  such that $\io_f|_F=\io$. Now, define
  \begin{equation*}
    E=\bigcup_{f(x) \in F[x]}{E_f}
  \end{equation*}
  and define $\io_E:E \xrightarrow{} K'$ such that
  $\io_E|_{E_f}=\io_f$. Then $\io_E|_F=\io$. Now, we get the following
  tower of fields for any polynomial $f(x)$ over $F$:
  \begin{equation*}
    F \subseteq E_f \subseteq E
  \end{equation*}
  so that $E$ is an upperbound for the collection $\{E_f\}$ ordered by
  inclusion. Therefore, by Zorn's proposition, the collection $\{E_f\}$ has
  a maximal element.

  Take $E$ to be the maximal element of $\{E_f\}$. Now, we claim that
  $E=K$. Indeed, suppose not, and observe that since $K$ is an
  algebraic closure of $F$, $E \subseteq K$. Now, take $\a \in
  \com{K}{E}$, and let $m_\a(x)$ the minimal polynomial of $\a$ over
  $F$, and construct the field $E(\a)$. Since $K$ is algebraic over
  $F$, $K$ must also be algebraic over $E$. Now, since $K'$ is also an
  algebraic closure of $F$, $m_\a(x)$ has a root $\b \in K'$. Define
  now the map $\io_E':E(\a) \xrightarrow{} K'$ by $\io_E'|_E=\io_E$,
  and $\io_E'(\a)=\b$. Observe that $E(\a)$ is the splitting field of
  $m_\a(x)$ over $F$, so that $E(\a) \subseteq \{E_f\}$; moreover, $E
  \subseteq E(\a)$. But that contradicts the maximality of $E$,
  therefore we must have that $E=K$. Therefore, we take
  $\io_E=\io_K:K \xrightarrow{} K'$, and we have extended the
  identity $\io:F \xrightarrow{} F$ to a 1--1 homomorphism $\io_K$.

  Lastly, let $\b \in K'$, and let $m_\b(x)$ the minimal polynomial of
  $\b$ over $F$. Then $m_\b(x)$ splits completely over $K'$ as:
  \begin{equation*}
    m_\b(x)=(x-\b_1) \dots (x-\b_d) \text{ where } \deg{m_\b}=d
  \end{equation*}
  in particular, $\b=\b_j$ for some $1 \leq j \leq d$. Now, applying
  $\io_K$ to $m_\b(x)$ over $F$, and observing that $\io_K|_F=\io$, we
  get that $\io_K \circ m_\b(x)=m_\b(x)$, and $m_\b(x)$ splits over
  $K$ as:
  \begin{equation*}
    m_\b(x)=(x-\io_K(\a_1)) \dots (x-\io_K(\a_d))
  \end{equation*}
  where $\a_1, \dots, \a_d \in K$ are roots of $m_\b(x)$ over $K$.
  Therefore we have that $\b_j=\io_K(\a_i)$ for all $1 \leq i,j \leq
  d$. This show that $\io_K$ is onto. Therefore, $\io:F \xrightarrow{}
  F$ extends to an isomorphism $\io_K:K \xrightarrow{} K'$.
\end{proof}

\begin{theorem}[The Fundamental Theorem of Algebra]\label{theorem_8.5.7}
  The field of complex numbers $\C$ is algebraically closed.
\end{theorem}
\begin{corollary}
  $\C$ contains the algebraic closure of any of its subfields.
\end{corollary}
\begin{proof}
  This is immediate from proposition \ref{proposition_8.5.5}.
\end{proof}
