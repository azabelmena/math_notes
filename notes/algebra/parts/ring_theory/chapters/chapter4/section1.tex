\section{Field Extensions}
\label{section_8.1}

\begin{definition}
  Let $D$ be an integral domain. We define the \textbf{characteristic}
  of $D$ to be the smallest  $p \in \Z^+$ for which $p \cdot 1=0$. We
  write $\char{D}=p$. If no such $p$ exists, then we write
  $\char{D}=0$.
\end{definition}

\begin{proposition}\label{proposition_8.1.1}
  Let $D$ be an integral domain. Then $\char{D}$ is either $0$, or
  prime in $\Z^+$.
\end{proposition}
\begin{proof}
  Suppose that $\char{D}=m \neq 0$, and suppose $m=ab$ for some  $a,b
  \in \Z^+$. Then
  \begin{equation*}
    m \cdot 1=(ab) \cdot 1=a(b \cdot 1)=ab=0
  \end{equation*}
  Since $D$ is an integral domain, this makes  $a=0$ or $b=0$, hence
  $m=0$ which cannot happen. Therefore, $m$ must be prime in $\Z^+$.
  In particular, if $m$ isn't prime, it must follow that $m=0$.
\end{proof}
\begin{corollary}
  If $F$ is a field, then either $\char{F}$ is $0$ or is prime.
\end{corollary}
\begin{corollary}
  For any $\a \in D$, with $\char{D}=p$
  \begin{equation*}
    p \cdot \a=\underbrace{\a+\dots+\a}_{p-\text{times}}=0
  \end{equation*}
\end{corollary}
\begin{proof}
  Observe $p \cdot \a=p(\a \cdot 1)=(p \cdot 1)\a$.
\end{proof}

\begin{example}\label{example_8.1}
  \begin{enumerate}
    \item[(1)] The integral domain $\Z$ and the fields $\Q$, $\R$, and
       $\C$ all have characteristic $0$. There are no $p$ in any of
       these structures for which $p \cdot 1=0$.

     \item[(2)] Let $p \in \Z^+$ be a prime. Then
       $\char{\faktor{\Z}{p\Z}}=p$, since $p \cdot 1 \equiv p \mod{p}
       \equiv 0 \mod{p}$.

     \item[(3)] Let $p \in \Z^+$ a prime. In $\F_p$, the field with
       $p$ elements, $\char{\F_p}=p$. Moreover, $\char{\F_p[x]}=p$ and
       $\char{\F_p(x)}=p$.
  \end{enumerate}
\end{example}

\begin{proposition}\label{proposition_8.1.2}
  Let $F$ be a field. Then there exists a smallest subfield of $F$
  containing the identity $1 \in F$. Moreover, this field is
  isomoprhic to either $\Q$ or $\faktor{\Z}{p\Z}$.
\end{proposition}
\begin{proof}
  Define the map:
  \begin{align*}
    \phi: \Z & \xrightarrow{} F \\
        n & \xrightarrow{} n \cdot 1  \\
  \end{align*}
  Then $\phi$ is a ring homomorphism of $\Z$ into $F$. Now, let
  $\char{F}=p$, then $\ker{\phi}=p\Z$. Now, by proposition
  \ref{proposition_8.1.1} $p=0$, or $p$ is prime. So $\phi$ is a 1--1
  map of either $\Z$ into $F$ or to $\faktor{\Z}{p\Z}$ into $F$.
  Constructing the field of fractions of $\Z$ or  $\faktor{\Z}{p\Z}$,
  by lemma \ref{lemma_5.7.3}, $F$ has an isomoprhic copy of either
  $\Q$ or $\faktor{\Z}{p\Z}$ as a subfield, and denote it $R$. Suppose
  that there is another subfield $R'$ of $F$ containing  $1$ such tha
   $R' \subseteq R$. Since $\char{F}=p$, then $\char{R'}=p$, so that
   $R' \simeq \Q$ or $R \simeq \faktor{\Z}{p\Z}$. In either case, $R'
   \simeq R$ and since $R' \subseteq R$, this makes $R'=R$.
\end{proof}

\begin{definition}
  We define the \textbf{prime subfield} of a field $F$ to be the
  smallest subfield of $F$ containing $1 \in F$.
\end{definition}


\begin{example}\label{example_8.2}
  \begin{enumerate}
    \item[(1)] By definition, and by proposition
      \ref{proposition_8.1.2}, the prime subfield of any field $F$
      is isomoprhic to either $\Q$  (if $\char{F}=0$) or to
      $\faktor{\Z}{p\Z}$ (if $\char{F}=p$, $p \in \Z^+$ a prime).

    \item[(2)] The prime subfields of $\Q$ and $\R$ and  $\C$ are all
      $\Q$, while the prime subfield of $\faktor{\Z}{p\Z}$ is itself.
      $\Z$ has no prime subfield since it is just an integral domain.

    \item[(3)] The prime subfields of $\F_p$ and of $\F_p(x)$ are both
      $\faktor{\Z}{p\Z}$, where $p \in \Z^+$ is a prime.
  \end{enumerate}
\end{example}

\begin{definition}
  Let $F$ be a field. A field $K$ is said to be a \textbf{field
  extension} of $F$, provided  $F \subseteq K$. We write
  $\faktor{K}{F}$ to denote $K$ as a field extension of $F$, and we
  call $F$ the \textbf{base-field} of the extension. We also
  visualize the field extension $\faktor{K}{F}$ as the lattice
  \[\begin{tikzcd}
    K \\
    F
    \arrow[no head, from=2-1, to=1-1]
  \end{tikzcd}\]
  ordered under inclusion.
\end{definition}

\begin{proposition}\label{proposition_8.1.3}
  Let $F$ be a field, and  $K$ a field extension of  $F$. Then  $K$ is
  a vector space over $F$.
\end{proposition}

\begin{definition}
  Let $F$ be a field, and $K$ an extension over $F$. We define the
   \textbf{degree} of $K$ over  $F$ to be
   \begin{equation*}
     [K:F]=\dim_F{K}
   \end{equation*}
   where $\faktor{K}{F}$ is seen as a vector space. We call $K$ a
   \textbf{finite extension} over $F$ if  $[K:F]$ is finite;
   otherwise, we call $K$ an \textbf{infinite extension} over $F$.
\end{definition}

\begin{proposition}\label{proposition_8.1.4}
  Let $\phi:F \xrightarrow{} F'$ be a field homomorphism. Then either
  $\phi$ is identically the  $0$ homomorphism, or  $\phi$ is 1--1.
\end{proposition}
\begin{proof}
  Observe as a ring homomorpsism, either $\ker{\phi}=(0)$ or
  $\ker{\phi}=(1)$.
\end{proof}

\begin{theorem}\label{theorem_8.1.5}
  Let $F$ be a field, and $p(x) \in F[x]$ an irreducible polynomial.
  Then there exists a field extension $K$ containing a subfield
  isomoprhic to $F$, and for which $p(x)$ has a root.
\end{theorem}
\begin{proof}
  Define
  \begin{equation*}
    K=\faktor{F[x]}{(p)}
  \end{equation*}
  Since $p(x)$ is irreducible over $F$, and  $F[x]$ is a PID, $(p)$ is
  maximal in $F[x]$, and hence $K$ is a field.

  Now, the map
  \begin{align*}
    \pi: F[x] & \xrightarrow{} K \\
      f(x)  & \xrightarrow{} f(x) \mod{p(x)}
  \end{align*}
  is a field homomorphism. Since $\pi:1 \xrightarrow{} 1$, $\pi$ must
  be 1--1, and hence imbeds $F \xrightarrow{} \pi(F)$ into $K$. Now,
  identify $F$ with $\pi(F)$, and let $\a \in K$. Then
  $p(\pi(\a))=p(\a \mod{p(x)})=p(\a) \mod{p(x)} \equiv 0 \mod{\p(x)}$,
  so that $p(x)$ has a root in $K$.
\end{proof}

\begin{theorem}\label{theorem_8.1.6}
  Let $F$ be a field, and $p(x)$ be an irreducible polynomial of
  degree $n$ over $F$. Define $K=\faktor{F[x]}{(p)}$, and let
  $\a \in K$ be a root of $p(x)$. Then the set
  \begin{equation*}
    \{1, \a, \dots, \a^{n-1}\}
  \end{equation*}
  forms a basis of $K$ over $F$ as a vector space.
\end{theorem}
\begin{proof}
  Let $a(x) \in F[x]$ a polynomial of arbitary degree. Since $F[x]$ is
  a Euclidean domain, there exist $q(x), r(x) \in F[x]$ such that
  \begin{equation*}
    a(x)=q(x)p(x)+r(x) \text{ and } \deg{r}<n
  \end{equation*}
  then $a(x) \equiv r(x) \mod{p(x)}$ so that $a(x)$ is a polynomial of
  degree less than $\deg{p}=n$. Indeed, since $\a$ is a root of $p$,
  setting $\a=x \mod{p}$, we have
  \begin{equation*}
    a(x) \equiv a_0+a_1\a+\dots+a_{n-1}\a^{n-1}=a(\a)
  \end{equation*}
  in $K$. That is  $\span{K}=\{1, \a \dots, \a^{n-1}\}$.

  Now, suppose there exist $b_0,b_1, \dots, b_{n-1} \in F$ not all $0$
  for which
  \begin{equation*}
    b_0+b_1\a+\dots+b_{n-1}\a^{n-1}=0 \text{ in }  K
  \end{equation*}
  Then
  \begin{equation*}
    b_0+b_1x+\dots+b_{n-1}x^{n-1} \equiv 0 \mod{p(x)}
  \end{equation*}
  so $p(x)|(b_0+b_1x+\dots+b_{n-1}x^{n-1})$, hence
  \begin{equation*}
    b_0+b_1x+\dots+b_{n-1}x^{n-1}=p(x)q(x)
  \end{equation*}
  and $\deg{p}+\deg{q}=n-1$. But $\deg{p}=n$, so that $\deg{q}=-1$
  which is impossible in $F[x]$. Therefore, no such $b_0,b_1, \dots,
  b_n \in F$ exist, so $\{1, \a, \dots, \a^{n-1}\}$ is linearly
  independent, and hence forms a basis for $K$ over $F$.
\end{proof}
\begin{corollary}
  $K=\{a_0+a_1\a+\dots+a_{n-1}\a^{n-1}: a_0, a_1, \dots, a_{n-1} \in
  F\}$, and $[K:F]=\deg{p}=n$.
\end{corollary}
\begin{corollary}
  If $a(\a),b(\a) \in K$ are of degree less than $n$, then
  $ab(\a)=r(\a)$ where $r(x) \equiv a(x)b(x) \mod{p(x)}$.
\end{corollary}

\begin{example}\label{example_8.3}
  \begin{enumerate}
    \item[(1)] Consider $p(x)=x^2+1$ over $\R$.  $p(x)$ is irreducible
      over $\R$, so we obtain the field
      \begin{equation*}
        \faktor{\R[x]}{(x^2+1)}
      \end{equation*}
      of degree $2$ over $\R$. Moreover, letting $\a$ be a root of
      $p(x)$, we have
      \begin{equation*}
        \faktor{\R[x]}{(x^2+1)}=\{a+\a{b} : a,b \in \R\}
      \end{equation*}
      where $\a^2+1=0$. Taking the map
      \begin{align*}
        \faktor{\R[x]}{(x^2+1)} & \xrightarrow{} \C \\
        a+bx  & \xrightarrow{} a+ib \\
      \end{align*}
      we obtain the isomorphism, $\C \simeq \faktor{\R[x]}{(x^2+1)}$.

    \item[(2)] Take $p(x)=x^2+1$ over $\Q$. Since $p$ is irreducible
      over $\R$, and  $\Q \subseteq \R$, $p$ is irreducible over
      $\Q$. Letting $i$ be a root of $p(x)$ so that $i^2+1=0$ we
      obtain the field
      \begin{equation*}
        \faktor{\Q[x]}{(x^2+1)}=\{a+ib : a,b \in \Q\}
      \end{equation*}
      of degree $2$ over $\Q$. We observe this is a subfield of $\C$.

    \item[(3)] Take $p(x)=x^2-2$ over $\Q$. Then  $p(x)$ is
      irreducible over $\Q$ and we obtain the field extension
      \begin{equation*}
        \faktor{\Q[x]}{(x^2-2)}=\{a+b\a : a,b \in \Q\} \text{ where }
        \a^2-2=0
      \end{equation*}
      We note that this field is not isomoprhic to
      a subfield of $\C$. The degree of this field over $\Q$ is also $2$.

    \item[(4)] Consider $p(x)=x^3-2$ over $\Q$. Then $p(x)$ is
      irreducible, and we obtain the field
      \begin{equation*}
        \faktor{\Q[x]}{(x^3-2)}=\{a+b\a+b\a^2 : a,b,c \in \Q\} \text{
        where } \a^3-2=0
      \end{equation*}
      Solving the equation $p(\a)=0$ in $\R$ we get $\a=\sqrt[3]{2}$.
      The degree of this field over $\Q$ is $3$.

    \item[(5)] Take $p(x)=x^2+x+1$ over $\F_2$. Observe $p(0)=1$ and
      $p(1)=1$ so that $p(x)$ is irreducible over $\F_2$. Letting $\a$
      be a root of  $p(x)$ we obtain the extension
      \begin{equation*}
        \F_4=\faktor{\F_2[x]}{(x^2+x+1)}=\{a+b\a : a,b \in \F_2\}
        \text{ where } \a^2+\a+1=0
      \end{equation*}
      Then $[\F_4:\F_2]=2$, and $|\F_4|=4$. Indeed, observe that
      \begin{equation*}
        \F_4=\{0, 1, \a, \a+1\}
      \end{equation*}

    \item[(6)] Let $k$ be a field, and define $p(x)=x^2-t$ over the
      field of rational functions $k(t)$. Observe that $(t)$ is a
      prime ideal in $k[t]$, so by the Eisenstein-Shc\"onemann
      criterion, $p(x)=x^2-t$ is irreducible in $k(t)[x]$. We hence
      obtain the field
      \begin{equation*}
        K=\faktor{k(t)[x]}{(x^2-t)}=\{a(t)+b(t)\a : a(t),b(t) \in
        k(t)\} \text{ where }  \a^2-t=0
      \end{equation*}
      We get the following extensions
      \[\begin{tikzcd}
        {k(t,\a)} \\
        {k(t)} \\
        k
        \arrow[no head, from=2-1, to=1-1]
        \arrow[no head, from=3-1, to=2-1]
      \end{tikzcd}\]
      Observe then that $k(t,\a)$ is the smallest field extension of
      $k$ containing $k$, and both $t$ and $\a$.
  \end{enumerate}
\end{example}

\begin{proposition}\label{proposition_8.1.7}
  Let $F$ be a field, and $K$ be an extension of $F$. Let  $\a \in K$.
  Then the collection of subfields of $K$ containing $F$ and $\a$ is
  nonempty. Moreover, the intersection of all such subfields is the
  smallest subfield of $K$ containing $F$ and $\a$.
\end{proposition}
\begin{proof}
  We have that $F \subseteq K$ and $\a \in K$, so that $K$ itself is a
  subfield containing $F$ and $\a$; hence the collection is non-empty.
  Now, let $\{F_m\}$ be the collection of all such subfields of $K$
  which contain  $F$ and  $\a$, and take
  \begin{equation*}
    F'=\bigcap_{m}{F_m}
  \end{equation*}
  Then $F \subseteq F'$, and $\a \in F'$. Now, suppose there exists a
  subfield $F'' \subseteq K$ containing $F$ and $\a$, and for which
  $F \subseteq F'' \subseteq F'$. Then $F'' \subseteq \{F_m\}$ for
  some $m$, and hence $F' \subseteq F''$, so that $F'=F''$.
\end{proof}
\begin{corollary}
  Let $\a,\b, \dots K$. Then there exists a smallest subfield of $K$
  containing $F$, and all $\a,\b, \dots$.
\end{corollary}

\begin{definition}
  Let $F$ be a field, and $K$ an extension of $F$. We define the field
  $F(\a,\b, \dots)$ \textbf{generated} by $F$, and $\a,\b \in K$ to be
  the smalles subfield of $K$ containing $F$ and $\a,\b,\dots$.
\end{definition}

\begin{definition}
  Let $F$ be a field. We call an extension $K$ a \textbf{simple
  extension} of $F$ if  $K=F(\a)$ for some $\a \in K$, and we call
  $\a$ a \textbf{primitive element} of $K$.
\end{definition}

\begin{theorem}\label{theorem_8.1.8}
  Let $F$ be a field, and $p(x)$ an irreducible polynomial over $F$.
  If $\a$ is a root of $p(x)$, then
  \begin{equation*}
    F(\a) \simeq \faktor{F[x]}{(p)}
  \end{equation*}
\end{theorem}
\begin{proof}
  Consider the natural map
  \begin{align*}
    \pi: F[x] & \xrightarrow{} F(\a)  \\
      a(x)  & \xrightarrow{} a(\a)  \\
  \end{align*}
  where $\pi|_F$ is the identity map, and  $\pi:x \xrightarrow{} \a$.
  Then $\pi$ extends to a homomorphism on $F[x]$. Now, since
  $p(\a)=0$, $p(\a) \in \ker{\pi}$ and we induce the map
  \begin{equation*}
    \pi:\faktor{F[x]}{(p)} \xrightarrow{} F(\a)
  \end{equation*}
  and since $p(x)$ is irreducible, $\faktor{F[x]}{(p)}$ is a field.
  Since $\pi:1 \xrightarrow{} 1$, we get $\pi$ is 1--1, so by the
  first isomoprhism theorem for rings
  \begin{equation*}
    \faktor{F[x]}{(p)} \simeq \pi\Big{(}\faktor{F[x]}{(p)}\Big{)} \subseteq F(\a)
  \end{equation*}
  and $\pi\Big{(}\faktor{F[x]}{(p)}\Big{)}$ is a field containing $F$, therefore
  by definition $\pi\Big{(}\faktor{F[x]}{(p)}\Big{)}=F(\a)$.
\end{proof}
\begin{corollary}
  $F(\a) \simeq \{a_0+a_1\a+\dots+a_{n-1}\a^{n-1}: a_0,a_1, \dots, a_n
  \in F\}$.
\end{corollary}
\begin{corollary}
  If $K$ is an extension of $F$, and  $\a_1, \dots, a_m \in K$, then
  \begin{equation*}
    F(\a_1, \dots, \a_m) \simeq \faktor{F(\a_1, \dots, \a_{m-1})[x]}{(p)}
  \end{equation*}
  where $p(x)$ is a polynomial over $F(\a_1, \dots, \a_{m-1})$ having
  $\a_m$ as a root.
\end{corollary}
\begin{proof}
  This proceeds by recursion on $F(\a_1, \dots, \a_m)$ applying the
  above theorem successively.
\end{proof}

\begin{example}\label{example_8.4}
  Referring to example \ref{example_8.3} we get:
  \begin{enumerate}
    \item[(1)] Let $p(x)=x^2+1$ over $\R$, and let $\a$ be a root of
      $p(x)$. Then
      \begin{equation*}
        \R(\a) \simeq \faktor{\R[x]}{(x^2+1)}=\{a+b\a : a,b \in \R\}
      \end{equation*}
      Taking the map $a+b\a \xrightarrow{} a+ib$ where $i^2+1=0$, we
      get $\R(\a) \simeq \C$. Indeed, observe that $\C=\R(i)$.

    \item[(2)] Let $p(X)=x^2+1$ over $\Q$, and take $i^2+1=0$ a root
      of $p(x)$. Then $\Q(i) \simeq \{a+ib : a,b \in \Q\}$. Indeed,
      $\Q(i) \subseteq \C$ is a subfield.

    \item[(3)] Solving the equation$p(\a)=\a^2-2$ in $\R$, we get
      $p(\a)=(\a-\sqrt{2})(\a+\sqrt{2})$ so that we get the
      isomoprhisms
      \begin{equation*}
        \Q(\sqrt{2}) \simeq \faktor{\Q[x]}{(x^2-2)} \simeq \Q(-\sqrt{2})
      \end{equation*}
      where $[\Q(\sqrt{2}):\Q]=[\Q(-\sqrt{2}):\Q]=2$. Moreover we get
      that
      \begin{align*}
        \Q(\sqrt{2})  &=  \{a+b\sqrt{2} : a,b \in \Q\}  \\
        \Q(-\sqrt{2}) &=  \{a-b\sqrt{2} : a,b \in \Q\}
      \end{align*}
      taking the map $a+b\sqrt{2} \xrightarrow{} a-b\sqrt{2}$, we can
      establish the isomorphism $\Q(\sqrt{2}) \simeq \Q(-\sqrt{2})$.

    \item[(4)] Take $p(x)=x^3-2$ over $\Q$. Then $p$ has as a root in
      $\R$ $\a=\sqrt[3]{2}$. So
      \begin{equation*}
        \Q(\sqrt[3]{2}) \simeq
      \{a+b\sqrt[3]{2}+c\sqrt[3]{4} : a,b,c \in \Q\}
      \end{equation*}
      Now, observe that factoring $p(x)$ over $\C$, we get the roots
      $\a=\xi\sqrt[3]{2}$ and $\b=\tbar{\xi}\sqrt[3]{2}$; where
      \begin{equation*}
        \xi=\frac{-1+\sqrt{3}}{2}
      \end{equation*}
      with complex conjugate $\tbar{\xi}$. Observe that neither
      $\Q(\xi\sqrt[3]{2})$ nor $\Q(\tbar{\xi}\sqrt[3]{2})$ are
      subfields of $\R$, but that
      \begin{equation*}
        \Q(\xi\sqrt[3]{2}) \simeq \Q(\sqrt[3]{2}) \simeq
        \Q(\tbar{\xi}\sqrt[3]{2})
      \end{equation*}

    \item[(5)] Let $k$ be a field, and $p(x)=x-t$ over $k$. Then  $p$
      is irreducible over $k$, and we have
      \begin{equation*}
        k(t) \simeq \faktor{k[x]}{(x-t)}
      \end{equation*}
      as the field of rational functions over $k$. Then consider the
      polynomial  $q(y)=y^2-t$ over $k(t)$. Again, $q(y)$ is
      irreducible over $k(t)$. Letting $\a^2-t=0$, we have:
      \begin{equation*}
        k(t,\a) \simeq \faktor{k(t)[y]}{(q)} \simeq
        \faktor{k[x,y]}{(p,q)}
      \end{equation*}
  \end{enumerate}
\end{example}

\begin{theorem}\label{theorem_8.1.9}
  Let $\phi:F \xrightarrow{} F'$ be an isomorphism of fields, and let
  $p(x)$ be an irreducible polynomial over $F$. Define  $p'(y)$ over
  $F'$ by $p'=\phi \circ p$, then $p'(y)$ is irreducible over $F'$.
  Moreover, if $\a$ is a root of $p(x)$ and $\b$ is a root of $p'(x)$,
  then there exists an isomorphism
  \begin{align*}
    \s: F(\a) & \xrightarrow{} F(\b)  \\
    \a  & \xrightarrow{} \b \\
  \end{align*}
  extending $\phi$.
\end{theorem}
\begin{proof}
  Observe first that if $p(x)=a_0+a_1x+\dots+a_{n-1}x^{n-1}$, then
  \begin{equation*}
    \phi(p(x))=\phi(a_0)+\phi(a_1)y_1+\dots+\phi(a_{n-1})y^{n-1}
    \text{ where } y=\phi(x)
  \end{equation*}
  Now, since $\phi$ is a ring homomorphism taking
  $1 \xrightarrow{} 1$, $\phi$ is 1--1. Moreover since $p$ is irreducible
  over $F$,  $(p)$ is maximal in $F[x]$ so that $(p')$ is maximal in
  $F'$, and hece $p'(x)$ is also irreducible. Now, extend $\phi$ to
  $\Phi:F[x] \xrightarrow{} F'[y]$ by taking $f(x) \xrightarrow{} \phi(f(x))$
  Then we get the field isomorphism
  \begin{equation*}
    \faktor{F[x]}{(p)} \simeq \faktor{F'[y]}{(p')}
  \end{equation*}
  Letting $\a$ be a root of $p(x)$ and $\b$ a root of $p'(y)$, define
  $\s:F(\a) \xrightarrow{} F(\b)$ by $\s=\Phi$ on $F[x]$, and $\s:\a
  \xrightarrow{} \b$. Then we get the isomorphism of the figure
  \[\begin{tikzcd}
    {F(\a)} & {\faktor{F[x]}{(p)}} & {\faktor{F'[y]}{(p')}} & {F(\b)} \\
            & F & {F'}
            \arrow["\simeq"{description}, draw=none, from=1-1, to=1-2]
            \arrow["\s", from=1-2, to=1-3]
            \arrow["\simeq"{description}, draw=none, from=1-4, to=1-3]
            \arrow[no head, from=2-2, to=1-2]
            \arrow["\phi", from=2-2, to=2-3]
            \arrow[no head, from=2-3, to=1-3]
  \end{tikzcd}\]
  Now, since $\s=\Phi$ on $F[x]$, and $\Phi=\phi$ on $F$, we get that
  $\s=\phi$ on $F$.
\end{proof}
