\section{Splitting Fields}
\label{section_8.4}

\begin{definition}
  Let $F$ be a field, and $K$ an extension over $F$. We say a
  polynomial $f(x) \in F[x]$ \textbf{splits} over $K$ if there exists
  factors of $f(x)$ over $K$. That is:  $f(x)=g(x)h(x)$ where
  $g(x),h(x) \in K[x]$. We say that $f(x)$ \textbf{splits completely}
  over $K$ if $f(x)$ splits into linear factors over $K$.
\end{definition}

\begin{definition}
  Let $F$ be a field. We call an extension $K$ of $F$ a
  \textbf{splitting field} for a polynomial $f(x) \in F[x]$ provided
  it is the smallest extension of $F$ over which $f(x)$ splits
  completely.
\end{definition}

\begin{proposition}\label{proposition_8.4.1}
  Let $F$ be a field. Then a polynomial $f(x)$ over $F$ of degree at
  most $n$ has at most $n$ roots over $F$. In particular, $f(x)$ has
  exactly $n$ roots over $F$ if, and only if $f(x)$ splits completely
  over $F$.
\end{proposition}

\begin{theorem}\label{theorem_8.4.2}
  For any field $F$, and any polynomial $f(x)$ over $F$, there exists
  a splitting field for $f(x)$ over $F$.
\end{theorem}
\begin{proof}
  Let $\deg{f}=n$. If $n=1$, then $f(x)$ is its own linear factor and
  hence splits completely over $F$. Take the splitting field $K=F$.

  Now, for $n \geq 1$, if $f(x)$ has all irreducible factors of degree
  $1$, then by definition $f(x)$ splits completely over $F$, hence
  take the splitting field $K=F$.

  Now, if $f(x)$ does not have irreducible factors all of degree $1$,
  there exists an irreducible factor $p(x)$ of $\deg{p} \geq 2$. Now,
  by theorem \ref{theorem_8.1.5}, there is an extension $E_1$ of $F$
  containing a root  $\a$ of  $p(x)$. So $f(x)$ splits over $E_1$ as:
  \begin{equation*}
    f(x)=(x-\a)f_1(x) \text{ where } \deg{f_1}=n-1
  \end{equation*}
  where $f_1(x) \in E_1[x]$. Therefore, by induction on $n$, there
  exists an extension $\faktor{E}{E_1}$ over which $f_1(x)$ splits
  completely. Now, $\a \in E'$ so that $f(x)$ splits completely over
  $E$. Consider all $F \subseteq E' \subseteq E$ for which $f(x)$
  splits completely over $E'$, and let:
  \begin{equation*}
    K=\bigcap_{F \subseteq E' \subseteq E}{E'}
  \end{equation*}
  then by definition, $K$ is a splitting field for $f(x)$ over $F$.
\end{proof}

\begin{definition}
  Let $F$ be a field, and $K$ an extension of $F$. We call $K$ a
  \textbf{normal extension} of $F$ if it is algebraic over $F$, and it
  is a splitting field for a collection of polynomials over $F$.
\end{definition}

\begin{example}\label{example_8.12}
  \begin{enumerate}
    \item[(1)] The splitting field of $x^2-2$ over $\Q$ is
      $\Q(\sqrt{2})$, since $x^2-2=(x-\sqrt{2})(x+\sqrt{2})$, and
      $\pm{\sqrt{2}} \in \Q(\sqrt{2})$.

    \item[(2)] Observe that $x^2-2$ and  $x^2-3$ over $\Q$ have as
      splitting fields $\Q(\sqrt{2})$ and $\Q(\sqrt{3})$,
      respectively. Now, $x^2-2$ does not split completely over
      $\Q(\sqrt{3})$, and $x^2-3$ does not split completely over
      $\Q(\sqrt{2})$ (see example \ref{example_8.8}(2)). Observe that
      $\pm{\sqrt{2}},\pm{\sqrt{3}} \in \Q(\sqrt{2},\sqrt{3})$, and
      that
      \begin{equation*}
        (x^2-2)(x^2-3)=(x-\sqrt{2})(x+\sqrt{2})(x-\sqrt{3})(x+\sqrt{3})
      \end{equation*}
      Observe also that $\Q(\sqrt{6}) \subseteq \Q(\sqrt{2},\sqrt{3})$
      but that $(\sqrt{6}^2-2)(\sqrt{6}^2-3)=3 \cdot 2=6 \neq 0$ in
      $\Q$, so  $\sqrt{6}$ is not a root of $(x^2-2)(x^2-3)$. Hence
      the splitting field for $(x^2-2)(x^2-3)$ is
      $\Q(\sqrt{2},\sqrt{3})$. We have the following lattice with the
      corresponding degrees:
      \[\begin{tikzcd}
  & {\Q(\sqrt{2}, \sqrt{3})} \\
        {\Q(\sqrt{2})} & {\Q(\sqrt{6})} & {\Q(\sqrt{3})} \\
                       & \Q
                       \arrow["2", no head, from=2-1, to=1-2]
                       \arrow["2"', no head, from=2-2, to=1-2]
                       \arrow["2"', no head, from=2-3, to=1-2]
                       \arrow["2", no head, from=3-2, to=2-1]
                       \arrow["2"', no head, from=3-2, to=2-2]
                       \arrow["2"', no head, from=3-2, to=2-3]
      \end{tikzcd}\]
      Observe this is the complete lattice as compared to the lattice
      found in example \ref{example_8.8}(2).

    \item[(3)] Consider $x^3-2$ over  $\Q$ which has roots
      \begin{align*}
        \sqrt[3]{2} &&  \xi\sqrt[3]{2}  &&  \bar{\xi}\sqrt[3]{2}  \\
      \end{align*}
      where
      \begin{equation*}
        \xi=-\frac{1}{2}+i\frac{\sqrt{3}}{2}
      \end{equation*}
      with complex conjugate $\bar{\xi}$ and where $i^2+1=0$.
      Observe $\xi, \bar{\xi} \notin \Q(\sqrt[3]{2})$, however $\xi,
      \bar{\xi} \in \Q(\sqrt[3]{2}, i\sqrt{3})$ is the smalles subfield
      containing all the roots of $x^3-2$ over $\Q$. So the splitting
      field of $x^3-2$ over  $\Q$ is $\Q(\sqrt[3]{2}, i\sqrt{3})$. We get
      \begin{equation*}
        [\Q(\sqrt[3]{2}, i\sqrt{3}):\Q]=
        [\Q(\sqrt[3]{2}, i\sqrt{3}):\Q(i\sqrt{3})][\Q(i\sqrt{3}):Q]=6
      \end{equation*}
      so that $[\Q(\sqrt[3]{2}, i\sqrt{3}):\Q(i\sqrt{3})]=3$ and we have
      the following lattice:
      \[\begin{tikzcd}
    &&& {\Q(\sqrt[3]{2}, i\sqrt{3})} \\
        {\Q(\bar{\xi})} & {\Q(\xi)} & {\Q(\sqrt[3]{2})} \\
                        &&&& {\Q(i\sqrt{3})} \\
                        &&& \Q
                        \arrow[no head, from=2-1, to=1-4]
                        \arrow[no head, from=2-2, to=1-4]
                        \arrow[no head, from=2-3, to=1-4]
                        \arrow[no head, from=3-5, to=1-4]
                        \arrow[no head, from=4-4, to=2-1]
                        \arrow[no head, from=4-4, to=2-2]
                        \arrow[no head, from=4-4, to=2-3]
                        \arrow[no head, from=4-4, to=3-5]
      \end{tikzcd}\]

    \item[(4)] $x^4+4=(x^2+2x+2)(x^2-2x+2)$ over $\Q$, and where the
      polynomials $x^2 \pm 2x+2$ are irredcucible over $\Q$ by the
      Eisenstein-Shc\"onemann criterion. Now, we get
      \begin{equation*}
        x^2 \pm 2x+2=x \pm 1, x \pm i
      \end{equation*}
      where $i^2+1=0$, so that
      \begin{equation*}
        x^4+4=(x-1)(x+1)(x+i)(x-i)
      \end{equation*}
      and the splitting field for $x^4+4$ over $\Q$ is the quadratic
      extension $\Q(i)$ of discriminant $-1$.
  \end{enumerate}
\end{example}

\begin{proposition}\label{proposition_8.4.3}
  Let $F$ be a field, and $f(x)$ a polynomial over $F$. If
  $\deg{f}=n$, then any splitting field of $f(x)$ over $F$ has degree
  at most $n!$.
\end{proposition}
\begin{proof}
  Let $\a$ be a root of $f(x)$, and let $K$ be a splitting field of
  $f(x)$ over $F$. Then
  \begin{equation*}
    f(x)=(x-\a)g(x)
  \end{equation*}
  over $F(\a)$ and $F(\a)
  \subseteq K$, and $\deg{g} \leq n-1$. Appending a root $\b$ of
  $g(x)$ over $F(\a)$, we get
  \begin{equation*}
    f(x)=(x-\a)(x-\b)h(x)
  \end{equation*}
  over $F(\a,\b) \subseteq K$, where $\deg{h} \leq n-2$. Recursing
  over the roots of $f(x)$, since $f(x)$ splits completely over $K$,
  we get
  \begin{equation*}
    [K:F] \leq n(n-1)(n-2)\dots 2 \cdot 1=n!
  \end{equation*}
\end{proof}
\begin{corollary}
  If $f(x)$ is irreducible over $F$, then  $[K:F]=n!$.
\end{corollary}

\begin{example}[The Cyclotomic Field over $\Q$]\label{example_8.13}
  \begin{enumerate}
    \item[(1)] Let $x^n-1$ over $\Q$ whose roots are the $n$-th roots
      of unity $\xi^n=1$. If  $K$ is the splitting field of  $x^n-1$
      over $\Q$, then the set of all $n$-th roots of unity form a
      cyclic group. Moreover, by above, we get $[K:\Q]=n!$, since
      $x^n-1$ is irreducible
      over  $\Q$. Now, observe that
      \begin{equation*}
        \Q \subseteq K \subseteq \C
      \end{equation*}
      so that in $\C$ the $n$-th roots of unity of $\Q$ have the form
      \begin{equation*}
        \xi_k=\exp{\frac{2i\pi{k}}{n}} \text{ where } 1 \leq k \leq n
      \end{equation*}
      and where $i^2+1=0$. We call the generators of the cyclic group of
      $n$-th roots of unity  \textbf{primitive} over  $\Q$. Indeed, the
      splitting field of $x^n-1$ over $\Q$ is $\Q(\xi)$, and is called the
      \textbf{cyclotomic field} of $n$-th roots of unity over $\Q$.

    \item[(2)] Let $p \in \Z^+$ a prime, and let $K$ be the splitting
      field of  $x^p-2$ over $\Q$. If $\a$ is a root of  $x^2-2$ over
      $\Q$, then  $\a^p=2$. Now, letting $\xi$ a $p$-th root of unity
      over $\Q$, we have
      \begin{equation*}
        (\xi\a)^p=\xi^p\a^p=\a^p=2
      \end{equation*}
      so that $\xi\a$ is also a root. So the roots of  $x^p-2$ are of
      the form $\xi^k\a$ where $\a=\sqrt[p]{2}$, $\xi^p=1$ and $1
      \leq k \leq p$. Therefore the splitting field of $x^p-2$ over
      $\Q$ is  $\Q(\xi, \sqrt[p]{2})$.

      Now, observe that $\Q(\xi),\Q(\sqrt[p]{2}) \subseteq
      \Q(\xi, \sqrt[p]{2})$, and
      \begin{equation*}
        [\Q(\xi):\Q]=p-1 \text{ and } [\Q(\sqrt[p]{2}):\Q]=p
      \end{equation*}
      so $[\Q(\xi, \sqrt[p]{2}):\Q]=p(p-1)$. This shows that $x^p-2$
      is irreducible over $\Q$ and that $\Q(\xi)$ and
      $\Q(\sqrt[p]{2})$ are the only proper non-trivial subfields of
      $\Q(\xi,\sqrt[p]{2})$. Hence we have the lattice:
      \[\begin{tikzcd}
  & {\Q(\xi, \sqrt[p]{2})} \\
        {\Q(\xi)} && {\Q(\sqrt[p]{2})} \\
                  & \Q
                  \arrow[no head, from=2-1, to=1-2]
                  \arrow[no head, from=2-3, to=1-2]
                  \arrow[no head, from=3-2, to=2-1]
                  \arrow[no head, from=3-2, to=2-3]
      \end{tikzcd}\]
  \end{enumerate}
\end{example}

\begin{theorem}\label{theorem_8.4.4}
  Let $\phi:F \xrightarrow{} F'$ an isomorphsm of fields, and let
  $f(x) \in F[x]$, and define $f'(x) \in F'[x]$ by $f'(x)=\phi \circ
  f(x)$. Let $K$ and $K'$ be splitting fields of $F$ and $F'$,
  respectively, then $\phi$ extends to a field isomorphism $\s:K
  \xrightarrow{} K'$ given by the following diagram:
  \[\begin{tikzcd}
    K & {K'} \\
    F & {F'}
    \arrow["\s", from=1-1, to=1-2]
    \arrow[no head, from=2-1, to=1-1]
    \arrow["\phi"', from=2-1, to=2-2]
    \arrow[no head, from=2-2, to=1-2]
  \end{tikzcd}\]
\end{theorem}
\begin{proof}
  Let $\deg{f}=n$, so that $\deg{f'}=n$. By induction on $n$. If $n=1$
  and $f(x)$ has all it's roots in $F[x]$, then $f(x)$ splits
  completely over $F$. Hence $f'(x)$ must also split completely over
  $F'$. Take then the splitting fields to be $K=F$ and $K'=F'$, and
  take $\s:K \xrightarrow{} K'$ to be $\phi$.

  Now, suppose that the theorem is true for any polynomial of degree
  less than or equal to $n$. let $f(x) \in F[x]$ be of degree
  $\deg{f}=n+1$. Since $f'(x)=\phi \circ f$, $\deg{f'}=n+1$ over $F'$.
  Now, let $p(x) \in F[x]$ and $p'(x) \in F'[x]$ be irreducible
  factors of $f(x)$ and $f'(x)$, respectively, each of degree greater
  than or equal to $2$ (indeed, notice $p'(x)=\phi \circ p(x)$). Let
  $\a \in K$ a root of $p(x)$ and $\b \in K'$ a root of $p'(x)$. Now
  extend $\phi$ to a field isomorphism  $\s':F(\a) \xrightarrow{}
  F'(\b)$ as shown:
  \[\begin{tikzcd}
    {F(\a)} & {F'(\b)} \\
    F & {F'}
    \arrow["{\s'}", from=1-1, to=1-2]
    \arrow[no head, from=2-1, to=1-1]
    \arrow["\phi"', from=2-1, to=2-2]
    \arrow[no head, from=2-2, to=1-2]
  \end{tikzcd}\]
  Now
  \begin{equation*}
    f(x)=(x-\a)f_1(x) \text{ and } f'(x)=(x-\b)f_1'(x)
  \end{equation*}
  over $F(\a)$ and $F'(\b)$, respectively. Moreover $\deg{f_1}=n$ and
  $\deg{f_1'}=n$. Lastly observe the towers
  \begin{equation*}
    F \subseteq F(\a) \subseteq K \text{ and }
    F' \subseteq F'(\b) \subseteq K'
  \end{equation*}
  so by the induction hypothesis, $\s'$ extends to a field isomorphism
  $\s:K \xrightarrow{} K'$ shown below:
  \[\begin{tikzcd}
    K & {K'} \\
    {F(\a)} & {F'(\b)} \\
    F & {F'}
    \arrow["\s", from=1-1, to=1-2]
    \arrow[no head, from=2-1, to=1-1]
    \arrow["{\s'}", from=2-1, to=2-2]
    \arrow[no head, from=2-2, to=1-2]
    \arrow[no head, from=3-1, to=2-1]
    \arrow["\phi", from=3-1, to=3-2]
    \arrow[no head, from=3-2, to=2-2]
  \end{tikzcd}\]
  Restricting $\s$ to $F$, we get $\phi$.
\end{proof}
\begin{corollary}
  Any two splitting fields of a polynomial over $F$ are isomorphic.
\end{corollary}
\begin{proof}
  Take $F'=F$ and let $\phi:F \xrightarrow{} F$ the identity
  isomoprhism.
\end{proof}
