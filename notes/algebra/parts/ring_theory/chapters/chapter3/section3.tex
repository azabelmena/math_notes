\section{Irreducibility of Polynomials}
\label{section_7.3}

\begin{definition}
    Let $R$ be a ring, and  $p \in R[x]$. We call an element $\al \in R$ a
    \textbf{root} (or \textbf{zero}) of $p$ if  $p(\a)=0$.
\end{definition}

\begin{lemma}\label{lemma_7.3.1}
    Let $k$ be a field, and  $p \in k[x]$. Then $p$ has a linear factor if, and
    only if  $p$ has a root in  $k$.
\end{lemma}
\begin{proof}
    If $p$ has a linear factor, then it is of the form  $(x-\a)$ (assuming it is
    monic), for some $\a \in k$. But then  $p(\a)=(\a-\a)q(\a)=0$ making $\a$ a
    root.

    Conversely, suppose that $p$ has a root  $\a \in k$. By the division
    theorem, there exist $q, r \in k[x]$, $q(x) \neq 0$ for which
    \begin{equation*}
        p(x)=q(x)(x-\a)+r(x) \text{ and } r(x) \text{ is constant}
    \end{equation*}
    Then $p(\a)=q(\a)(\a-\a)+r(\a)$ so that $r(\a)=0$, which makes $r(x)=0$, and
    hence we have a linear factor $(x-\a)$ of $p$.
\end{proof}
\begin{corollary}
    A polynomial $p$ over a field $k$ of degree $\deg{p}=2,3$ is irreducible if,
    and only if it has no root in $k$.
\end{corollary}
\begin{proof}
  If $p(x)$ is reducible of $\deg{p}=2,3$, then it has a linear
  factor, so that $p(x)=(x-\a)q(x)$. where $\a$ is a root of $p$ and
  $\deg{q}=1,2$.
\end{proof}

\begin{lemma}\label{lemma_7.3.2}
    Let $R$ be a unique factorization domain, and
    \begin{equation*}
        p(x)=a_0+a_1x+\dots+a_nx^n
    \end{equation*}
    a polynomial of degree $n$ over $R$. Let $k$ be the field of fractions of
    $R$,and $\frac{r}{s} \in k$ with $r$ and $s$ coprime. If $\frac{r}{s}$ is a
    root of $p$, then  $r|a_0$ and $s|a_n$.
\end{lemma}
\begin{proof}
    We have that
    $p(\frac{r}{s})=0=a_0+a_1(\frac{r}{s})+\dots+a_n(\frac{r}{s})^n$.
    Multiplying both sides by $s^n$, we have
    \begin{equation*}
        s^na_0+s^na_1(\frac{r}{s^{n-1}})+\dots+a_nr^n=0
    \end{equation*}
    so that
    \begin{equation*}
        a_nr^n=s(-a_0s^{n-1}- \dots -a_{n-1}r^{n-1})
    \end{equation*}
    which makes $s|a_n$, since $(r,s)=1$. By simnilar reasoning, we conclude that
    $r|a_0$.
\end{proof}

\begin{example}\label{example_7.4}
    \begin{enumerate}
        \item[(1)] The polynomial $x^3+3x-1$ is irreducible in  $\Z[x]$. By
            Gauss' lemma, it suffices to show that it has no roots in $\Q$.
            Indeed, the only possible roots for this polynomial are  $\pm{1}$,
            and notice $1^2+3(1)-1=1+3-1=3 \neq 0$, and
            $(-1)^3+3(-1)-1=-1-3-1=-5 \neq 0$.

        \item[(2)] For every prime $p$,  $x^2-p$ and  $x^3-p$ are irreducible in
             $\Q[x]$. Notice that since they are monic, the only possible roots
             are $\pm{1}$ and $\pm{p}$, none of which satisfy the polynomials.

         \item[(3)] $x^2+1$ is reducible in  $\faktor{\Z}{2\Z}[x]$. Notice that
             $1^2+1=1+1 \equiv 0 \mod{2}$. Then $x^2+1=(x+t)(x+1)=(x+1)^2$ in
             $\faktor{\Z}{2\Z}[x]$. Similarly, we can observe that $x^3+x+1$ is
             irreducible in  $\faktor{\Z}{2\Z}[x]$.
    \end{enumerate}
\end{example}

\begin{lemma}\label{lemma_7.3.3}
    Let $\af \neq (1)$ be a proper ideal of an integral domain $R$, and  $p \in
    R[x]$, is a nonnegative monic polynomial. If $p \mod{\af}$ cannot be factored
    in $\faktor{R}{\af}[x]$, then $p$ is irreducible.
\end{lemma}
\begin{proof}
    Suppose that $p$ fails to factor in  $\faktor{\R}{\af}[x]$, but that it is
    reducible in $R[x]$. Then there exist $a,b \in R[x]$ monic and nonconstant
    polynomials for which $p(x)=a(x)b(x)$. Then $p \equiv ab \mod{\af}$ which is a
    factorization in $\faktor{R}{\af}[x]$; a contradiction!
\end{proof}
\begin{remark}
    The converse is not true.
\end{remark}

\begin{example}\label{example_7.5}
    \begin{enumerate}
        \item[(1)] Let $p(x)=x^2+x+1 \in \Z[x]$. Then $p \mod{2}$ is irreducible
            in $\faktor{\Z}{2\Z}$, so that $p$ is irreducible in  $\Z[x]$.

        \item[(2)] Notice that $x^2+1$ is irreducible in  $\faktor{\Z}{3\Z}[x]$,
            so that it is irreducible in $\Z[x]$.

        \item[(3)] The polynomial $x^2+xy+1$ is irreducible in  $\Z[x,y]$. Take
            the ideal $(y)$, and notice that $x^2+xy+1 \mod{(y)} \equiv x^2+1$
            in $\faktor{\Z[x,y]}{(y)} \simeq \Z[x]$ which is irreducible.

        \item[(4)] The polynomial $xy+x+y+1=(x+1)(y+1)$ is reducible, but is
            irreducible $\mod{(x)}$ and $\mod{(y)}$ as well. This occures since
            non-unit polynomials in $\Z[x,y]$ can reduce to units in the
            quotient. Hence, to determine irreduciblity in $\Z[x,y]$ using
            ideals, it is necessary to first observe which elements reduce to
            quotients in the quotient ring.
    \end{enumerate}
\end{example}

\begin{theorem}[The Eisenstein-Sch\"onemann Criterion]\label{theorem_7.3.4}
    Let $\pf$ a prime ideal of an integral domain $R$, and let
    \begin{equation*}
        f(x)=a_0+a_1x+\dots+a_nx^n
    \end{equation*}
    a polynomial in $R[x]$ of degree $n \geq 1$. If $a_0, \dots, a_{n-1} \in
    \pf$, and $a_0 \notin \pf^2$, then $f$ is irreducible.
\end{theorem}
\begin{proof}
    Suppose that $f$ is reducible; i.e.  $f(x)=a(x)b(x)$, where $a,b \in R[x]$
    are nonconstant polynomials. Reducing modulo $\pf$, and by the fact that
    $a_0, \dots, a_{n-1} \in P$, we get $x^n \equiv ab \mod{\pf}$ in
    $\faktor{R}{\pf}[x]$. Since $\pf$ is prime,  $\faktor{R}{\pf}$ is an integral
    domain, so that either $a \mod{\pf}$ or $b \mod{\pf}$. have $0$ constant term.
    constant term. constant term. constant term, by supposition. However, $a_0$
    is the product of the constant terms of $a$ and $b$, so that $a_0 \in \pf^2$,
    which is a contradiciton.
\end{proof}
\begin{corollary}
    Let $p \in \Z$ be prime, and let
    \begin{equation*}
        f(x)=a_0+a_1x+\dots+a_nx^n
    \end{equation*}
    a polynomial in $\Z[x]$ of degree $n \geq 1$. If $p|a_0, \dots, a_{n-1}$ and
    $p^2 \nmid a_0$, then $f$ is irreducible in  $\Z[x]$, and in $\Q[x]$.
\end{corollary}

\begin{example}\label{example_7.6}
    \begin{enumerate}
        \item[(1)] By the Eisenstein-Sh\"onemann criterion for $p=5$,
          $x^4+10x+5$ is irreducible in  $\Z[x]$.

        \item[(2)] If $a \in \Z$, and  $p$ is a prime such that  $p|a$, but $p^2
            \nmid a$, then the polynomial $x^n-a$ is irreducible in  $\Z[x]$ for
            all $n \geq 1$.

        \item[(3)] COnsider $f(x)=x^4+1$ in $\Z[x]$. Let
            $g(x)=f(x+1)=x^4+4x^3+6x^2+4x+2$, and take $p=2$. Then  $p|2,4,6$, but
            $4 \nmid 2$, so tht  $g(x)$ is irreducible. This implies that $f$ is
            irreducible.

        \item[(4)] Let $p$ a prime, and let
            \begin{equation*}
                \Phi_n(x)=\frac{x^p-1}{x-1}=x^p+x^{p-1}+\dots+x+1
            \end{equation*}
            Take $\Phi_p(x+1)=x^{p-1}+px^{p-1}+\dots+\frac{p(p-1)}{2}x+p \in \Z[x]$.
            By Eisenstein's criterion, $\Phi_p$ is irreducible.

        \item[(2)] Consider the polynomial $y^n-x \in \Q[x,y]$ for all $n \geq
            0$. Notice that  $(x)$ is prime in $\Q[x]$ and that
            $\faktor{\Q[x]}{(x)} \simeq \Q$, we have $y^n-x$ is irreducible in
            $\Q[x,y]=\Q[y][x]$.

    \end{enumerate}
\end{example}
