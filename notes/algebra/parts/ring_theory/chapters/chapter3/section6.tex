\section{Gr\"obner Bases}
\label{section_7.6}

\begin{definition}
  Let $R$ be a Noetherian ring, and $\leq$ a monomial ordering on
  $R[x_1, \dots, x_n]$. We define the \textbf{leading term} of a
  polynomial $f(x_1, \dots, x_n)$ to be the maximum element of the
  ideal $(f)$, and denote it $LT(f)$. If $\af$ is an ideal in $R[x_1,
  \dots, x_n]$, then we define the \textbf{ideal of leading terms} of
  $\af$ to be the ideal generated by all leading terms of polynomials
  of $\af$; i.e.
  \begin{equation*}
    LT(\af)=(LT(f) : f \in \af)
  \end{equation*}
  If $\af=(f_1, \dots, f_m)$, we write $LT(f_1, \dots, f_m)$. If
  $\af=(f)$, we keep the previous convention $LT(\af)$.
\end{definition}

\begin{definition}
  Let $f(x_1, \dots, x_n)$ a multi-variate polynomial over a
  Noetherian ring. We define the \textbf{multi-degree} of $f$ to be
  the multi-degree of $LT(f)$, and we denote it $\partial{f}$.
\end{definition}

\begin{proposition}\label{proposition_7.6.1}
  Let $f_1, \dots, f_m$ be multivariate polynomials over a Noetherian
  ring. Then
  \begin{equation*}
    (LT(f_1), \dots, LT(f_m)) \subseteq LT(f_1, \dots, f_m)
  \end{equation*}
\end{proposition}
\begin{proof}
  Observe that since $f_i \in \af$ for each $1 \leq i \leq m$, then
  $LT(f_i) \in LT(\af)$ for each $1 \leq i \leq m$.
\end{proof}

\begin{proposition}\label{proposition_7.6.2}
  Let $k$ be field, and $f,g \in k[x_1, \dots, x_n]$. Then
  the following are true:
  \begin{enumerate}
    \item[(1)] $LT(fg)=LT(f)LT(g)$.

    \item[(2)] $LT(f+g) \leq \max{\{LT(f), LT(g)\}}$ with equality
      holding if $LT(f) \neq LT(g)$.
  \end{enumerate}
\end{proposition}
\begin{proof}
  Let $AX^\a=LT(f)$ and $BX^\b=LT(g)$, and make the convention $f(x_1,
  \dots, x_n)=f(X)$. Then we have for some $h,h' \in R[x_1, \dots,
  x_n]$ that $f(X)=AX^\a+h(X)$ and $g=BX^\b+h'(X)$. Then
  \begin{align*}
    fg(X) &=  (AX^\a+h(X))(BX^\b+h'(X))  \\
          &= (AB)X^\a X^\b+(AX^\a h'(X)+BX^\b h(X)+h(X)h'(X))  \\
          &=  (AB)X^{\a+\b}+h''(X)  \\
  \end{align*}
  so that $LT(fg)=LT(f)LT(g)$.

  Now, we also observe that
  \begin{equation*}
    f+g(X)=(AX^\a+BX^\b)+(h(X)+h'(X))
  \end{equation*}
  if $\a=\b$, and $B=-A$, then $f+g(X)=h(X)+h'(X)$ and $LT(f+g)<AX^\a$
  and $LT(f+g)<BX^\b$, so that $LT(f+g)<\max{\{AX^\a,BX^\b\}}$. Now, we have
  if  $\a \neq \b$, then
  \begin{align*}
    f+g(X)=AX^\a+(BX^\b+h(X)+h'(X))  & \text{ if } \b < \a  \\
    f+g(X)=AX^\b+(BX^\a+h(X)+h'(X))  & \text{ if } \a < \b  \\
  \end{align*}
  in either case either $LT(f+g)=\max{\{X^\a,X^\b\}}$.
\end{proof}
\begin{corollary}
  The following are true:
  \begin{enumerate}
    \item[(1)] $\partial{(fg)}=\partial{f}+\partial{g}$.

    \item[(2)] $\partial{(f+g)} \leq \max{\{\partial{f},
      \partial{g}\}}$, with equality holding if $\partial{f} \neq
      \partial{g}$.
  \end{enumerate}
\end{corollary}

\begin{proposition}\label{proposition_7.6.3}
  Let $R$ be a Noetherian ring. Let $\af=(X^{\a_1}, \dots, X^{\a_m})$ in $R[x_1,
  \dots x_n]$ and ideal and $f \in R[x_1, \dots, x_n]$. Then $f \in
  \af$ if, and only if each monomial term of $f$ is a multiple of some
  monomial  $X^\a_i$ generating $\af$.
\end{proposition}
\begin{proof}
  Suppose that $f \in \af$. Then
  \begin{equation*}
    f(x_1, \dots, x_n)=f(X)=\sum_{i=1}^m{A_mX^{\a_i}}
  \end{equation*}
  and notice that each monomial $X^{\a_i}$ is a multiple of itself.

  Conversely suppose that
  \begin{equation*}
    f(X)=\sum_{j=1}^n{A_jX^{\b_j}}
  \end{equation*}
  and that there is an $X^{\a_i}$ for which $X^{\a_i}|X^{\b_j}$ for
  all $1 \leq j \leq n$. Then $X^{\b_j} \in (X^{\a_i}) \subseteq
  (X^{\a_1}, \dots, X^{\a_m})$. Then $f \in \af$.
\end{proof}

\begin{example}\label{example_7.8}
  Let $k$ be a field.
  \begin{enumerate}
    \item[(1)] Choose the lexicographic order $x>y$ on $k[x,y]$. Let
      $f(x,y)=x^3y-xy^2+1$ and $g(x,y)=-y^3+x^2y^2-1$. Then
      $LT(f)=x^3y$ with $\partial{f}=(3,1)$, and $LT(g)=x^2y^2$ with
      $\partial{g}=(2,2)$. Then
      \begin{align*}
        LT(fg)=x^5y^3 & \text{ and } \partial{(fg)}=(5,3) \\
        LT(f+g)=x^3y  & \text{ and } \partial{(f+g)}=(3,0)  \\
      \end{align*}
      Now, take $\af=(f,g)$, then $(x^3y,x^2y^2) \subseteq LT(f,g)$.
      However, observe that $yf(x,y)-xg(x,y)=x+y \in \af$, and
      $LT(yf+xg)=x \in LT(f,g)$. However $x \notin
      (x^3y,x^2y^2)$. That is, in general, if $\af=(f_1,
      \dots, f_m)$, then $LT(f_1, \dots, f_m) \neq (LT(f_1), \dots
      LT(f_m))$.

    \item[(2)] Choose the lexicographic ordering $y>x$ in  $k[x,y]$.
      Let again $f(x,y)=x^3y-xy^2+1$ and $g(x,y)=-y^3+x^2y^2-1$. Then
      $LT(f)=xy^2$ with $\partial{f}=(1,2)$ and $LT(g)=y^3$ with
      $\partial{g}=(0,3)$. Then
      \begin{align*}
        LT(fg)=xy^5 & \text{ and } \partial{(fg)}=(1,5) \\
        LT(f+g)=-y^3 & \text{ and } \partial{(f+g)}=(03) \\
      \end{align*}
      Again, we have $(xy^2, y^3) \subseteq LT(f,g)$, but $LT(f,g)
      \neq (xy^2,y^3)$. We observe that $y \in LT(f,g)$ but $y \notin
      (xy^2,y^3)$.

    \item[(3)] Choose any order on $k[x,y]$, and let $f(x,y) \neq 0$.
      Let $\af=(f)$, then $LT(\af)=(LT(f))$, since every monomial term
      of $f$ is a multiple of  $LT(f)$.
  \end{enumerate}
\end{example}

\begin{definition}
  Let $k$ be a field, and $\af$ an ideal of $k[x_1, \dots, x_n]$. We
  call a finite set of generators $G=\{g_1, \dots, g_m\}$ of $\af$ a
  \textbf{Gr\"obner basis} for $\af$ if
  \begin{equation*}
    LT(\af)=(LT(g_1), \dots, LT(g_m))
  \end{equation*}
\end{definition}
