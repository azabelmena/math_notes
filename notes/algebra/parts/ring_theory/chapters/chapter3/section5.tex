\section{Gr\"obner Bases}
\label{section_7.5}

\begin{definition}
  We call a commutative ring with identity $1 \neq 0$
  \textbf{Noetherian} if every ideal is finitely generated.
\end{definition}

\begin{theorem}[Hilbert's Basis Theorem]\label{theorem_7.5.1}
  If $R$ is a Noetherian ring, then so is $R[x]$.
\end{theorem}
\begin{proof}
  Let $\af$ be an ideal of $R[x]$, and let $\lf$ be the set of all
  leading coefficients of elements of $\af$; that is:
  \begin{equation*}
    \lf=\{ a_n : f(x)=a_nx^n+h(x), \text{ for all }
    f \in \af, \text{ and } h(x) \in R[x]  \}
  \end{equation*}
  Then $\lf$ is an ideal of $R$. Indeed, observe that $0=0x^n+\dots
  \in \af$ so that $0 \in \lf$. Now let $f,g \in \af$ have the
  form $f(x)=ax^d+h(x)$ and $g(x)=bx^e+h'(x)$. Then $x^ef(x)-x^dg(x)
  \in \af$. Now
  \begin{equation*}
    x^ef(x)-x^dg(x)=(a-b)x^{d+e}+(x^eh(x)-x^dh'(x))
  \end{equation*}
  so that $a-b \in \lf$. Likewise, let $r \in R$, then $rf \in \af$,
  and
  \begin{equation*}
    rf(x)=(ra)x^d+(rh(x))
  \end{equation*}
  so that $ra \in \lf$. Since $R$ is Noetherian, and $\lf$ is an ideal
  of $R$,  $\lf=(a_1, \dots, a_n)$ for some $a_1, \dots, a_n \in R$.

  Now, let $f_1, \dots, f_n \in R[x]$ be polynomials with leading
  coefficients $a_1, \dots, a_n$ (associated above by $\lf$). Take
  $\deg{f_i}=e_i$ for all $1 \leq i \leq n$, and let $N=\max{\{e_1,
  \dots, e_n\}}$. Now, for $d \in \faktor{\Z}{N\Z}$, let $\lf_d$ be
  the set of all leading coefficients of polynomials in $\af$ with
  degree equal to $d$. By the same reasoning we used for $\lf$,
  $\lf_d$ is also an ideal of $R$, and hence $\lf_d=(b_{d,1}, \dots,
  b_{d,n})$ for some $b_{d,1}, \dots b_{d,n} \in R$. Now, let
  $f_d,i(x) \in \af$ have the form $f_{d,i}(x)=b_{d,i}x^d+h(x)$.

  Let $\af'=(f_1, \dots, f_n) \cup \{f_{d,i} : 1 \leq i \leq n_d
  \text{ and } d \in \faktor{\Z}{d\Z}\}$, where
  $f_i(x)=a_ix^{e_i}+\dots$ (recall that $\lf=(a_1, \dots, a_n)$). We
  show that $\af=\af'$. Indeed, since $f_i,f_{d,i} \in \af$, we get
  that $\af' \subseteq \af$. Suppose now that $\af \neq \af$, and
  choose some $f \in \com{\af}{\af'}$ of minimum degree $d'$, and let
  $f$ have the form $f(x)=ax^d+h(x)$.

  Suppose first that $d' \geq N$, and let $a \in \lf$, so that
  $a=r_1a_1+\dots+r_na_n$ for some $r_1, \dots r_n \in R$. Define
  \begin{equation*}
    g(x)=r_1x^{d'-e_1}f_1(x)+\dots+r_nx^{d'-e_n}f_n(x)
  \end{equation*}
  Then $g \in \af'$, and $\deg{g}=d'$. Moreover, $g$ has leading
  coefficient $a$. Then $f-g \in \af$, and $\deg{f-g}<d'$. Since $f'$ is
  of minimum degree in  $\af$, $f-g=0$ so that  $f=g$ and  $f \in
  \af'$. A contradiction!

  Likewise, suppose that $d'<N$, and  $a \in \lf_d$. Then
  $a=r_1b_{d,1}+\dots+r_nb_{d,n}$. Take
  \begin{equation*}
    g(x)=r_1f_{d,1}(x)+\dots+r_nf_{d,n}(x)
  \end{equation*}
  Then $g \in \af'$ and $\deg{g}=d'$, and $g$ has leading coefficient
  $a$. Then $f-g \in \af$, so that $f=g$ in $\af$ by the above
  reasoning which puts  $f \in \af'$ and gives us the same
  contradiction. In either case, we must have that $\af=\af'$.
  Therefore $\af$ is finitely generated in $R[x]$ wich makes $R[x]$
  Noetherian.
\end{proof}
\begin{corollary}
  Any field $k$ is Noetherian. In particular, the rings $k[x]$ and
  $k[x_1, \dots, x_n]$ are Noetherian.
\end{corollary}
\begin{proof}
  If $k$ is a field, then its only ideals are $(0)$ and $(1)$, which
  makes it Noetherian; hence $k[x]$ is Noetherian by Hilbert's basis
  theorem. Lastly, observe that $k[x_1, \dots, x_n]=(k[x_1, \dots,
  x_{n-1}])[x_n]$ and apply the above argument recursively.
\end{proof}
\begin{corollary}
  Let $k$ be a field. If $\af$ is an ideal in $k[x_1, \dots, x_n]$ generated
  by a set $S$, then $\af$ is generated by finitely many polynomials in $S$.
\end{corollary}
\begin{proof}
  Since $k$ is a field, $k[x_1, \dots, x_n]$ is Noetherian, and
  $\af=(f_1, \dots, f_n)$ where $f_i(x_1, \dots ,x_n) \in k[x_1,
  \dots, x_n]$. Likewise, since $\af=(S)$ is finitely generated, $S$
  is finite. Then since $(f_1, \dots, f_n)=(S)$, $f_i \in S$ for all
  $1 \leq i \leq n$ so that
  \begin{equation*}
    f_i(x)=r_1g_{i,1}(n_1, \dots, x_n)+\dots+r_ng_{i,m}(n_1, \dots, x_n)
  \end{equation*}
  and so $\af=(f_1, \dots, f_n)=(g_{1,1}, \dots ,g_{1,m}, \dots,
  g_{n,1}, \dots g_{n,m})$.
\end{proof}

\begin{definition}
  Let $R$ be a ring, and  $M$ the set of monomials in  $R[x_1, \dots,
  x_n]$. We define the \textbf{monomial ordering} $\leq$ on $M$ to be
  the lexicographic ordering on $M$. That is: identifying $(m,n)$ with
  $mn$
  \begin{equation*}
    m_1n_1 \leq m_2n_2 \text{ if, and only if }
    m_1=m_2 \text{ or }  n_1 \leq n_2
  \end{equation*}
\end{definition}

\begin{lemma}\label{lemma_7.5.2}
  Let $R$ be a ring and  $M$ the set of all monomials in  $R[x_1,
  \dots, x_n]$. Then $M$ is a partially ordered set under the monomial
  ordering.
\end{lemma}
\begin{proof}
  Exercise.
\end{proof}
\begin{corollary}
  $M$ is a totally ordered set.
\end{corollary}
\begin{proof}
  Exercise.
\end{proof}
\begin{corollary}
  $M$ is well-ordered.
\end{corollary}

\begin{lemma}\label{7.5.3}
  Let $R$ be a ring, and $M$ the set of monomials over $R[x_1, \dots,
  x_n]$. Define the relation $\leq$ on $\Z^n$ by: for $\a,\b,\y \in
  \Z^n$
  \begin{equation*}
    \a+\y \leq \b+\y \text{ if, and only if } \a \leq \b
  \end{equation*}
  where $\a=(a_1, \dots ,a_n), \b=(b_1, \dots, b_n), \y=(c_1, \dots,
  c_n)$ correspond to the monomials $Ax_1^{a_1} \dots x_n^{a_n},
  Bx_1^{b_1} \dots x_n^{b_n}$, and $Cx_1^{c_1} \dots x_n^{c_n}$ in
  $M$. Then $\leq$ is equivalent to the monomial ordering on $M$.
\end{lemma}
