\section{Gr\"obner Bases}
\label{section_7.5}

\begin{definition}
  Let $R$ be a ring, and  $M$ the set of monomials in  $R[x_1, \dots,
  x_n]$. We define the \textbf{monomial ordering} $\leq$ on $M$ to be
  the lexicographic ordering on $M$. That is: identifying $(m,n)$ with
  $mn$
  \begin{equation*}
    m_1n_1 \leq m_2n_2 \text{ if, and only if }
    m_1=m_2 \text{ or }  n_1 \leq n_2
  \end{equation*}
\end{definition}

\begin{lemma}\label{lemma_7.5.2}
  Let $R$ be a ring and  $M$ the set of all monomials in  $R[x_1,
  \dots, x_n]$. Then $M$ is a partially ordered set under the monomial
  ordering.
\end{lemma}
\begin{proof}
  Exercise.
\end{proof}
\begin{corollary}
  $M$ is a totally ordered set.
\end{corollary}
\begin{proof}
  Exercise.
\end{proof}
\begin{corollary}
  $M$ is well-ordered.
\end{corollary}

\begin{lemma}\label{7.5.3}
  Let $R$ be a ring, and $M$ the set of monomials over $R[x_1, \dots,
  x_n]$. Define the relation $\leq$ on $\Z^n$ by: for $\a,\b,\y \in
  \Z^n$
  \begin{equation*}
    \a+\y \leq \b+\y \text{ if, and only if } \a \leq \b
  \end{equation*}
  where $\a=(a_1, \dots ,a_n), \b=(b_1, \dots, b_n), \y=(c_1, \dots,
  c_n)$ correspond to the monomials $Ax_1^{a_1} \dots x_n^{a_n},
  Bx_1^{b_1} \dots x_n^{b_n}$, and $Cx_1^{c_1} \dots x_n^{c_n}$ in
  $M$. Then $\leq$ is equivalent to the monomial ordering on $M$.
\end{lemma}
