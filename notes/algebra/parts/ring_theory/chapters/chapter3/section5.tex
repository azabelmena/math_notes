\section{Monomial Orderings}
\label{section_7.5}

Throughout this section, we implicitly work in a multivariate
polynomial ring, unless otherwise specified. We consider monomials
over this ring, and omit all mention to it, unless otherwise
specified.

\begin{definition}
  Let $x_1^{a_1} \dots x_n^{a_n}$ be a monomial term in a multivariate
  polynomial ring. We define the \textbf{multi-degree} of $x_1^{a_1} \dots
  x_n^{a_n}$ to be the $n$-tuple $\a=(a_1, \dots, a_n)$ in $\N^n$. We
  denote the monomials of a multivariate polynomial ring, usinge the
  multi-degree by
  \begin{equation*}
    X^\a=x_1^{a_1} \dots x_n^{a_n}
  \end{equation*}
\end{definition}

\begin{definition}
  We define a \textbf{monomial ordering} on the collection of
  monomials of a multivariate polynomial ring to be the relation
  $\leq$ given by: For all $X^\a$ and  $X^\b$,
  \begin{equation*}
    X^\a \leq X^\b \text{ if, and only if }
    X^\a X^\y \leq X^\b X^\y \text{ for all } X^\y
  \end{equation*}
\end{definition}

\begin{lemma}\label{lemma_7.5.1}
  The collection of all monomials over a Noetherian ring is a
  partially ordered set.
\end{lemma}
\begin{proof}
  Fix $X^\y$. Then  $X^\a \leq X^\a$, since $X^\a X^\y=X^\a X^\y$.
  Now, suppose that  $X^\a \leq X^\b$ and $X^\b \leq X^\a$. Then $X^\a
  X^\y \leq X^\b X^\y$  and $X^\b X^\y \leq X^\a X^\y$, so that
  $X^\a X^\y=X^\b X^\y$. This makes $X^\a=X^\b$.

  Now, suppose that $X^\a \leq X^\b$ and $X^\b \leq X^\d$. Then $X^\a
  X^\y \leq X^\b X^\y$ and $X^\b X^\y \leq X^\d X^\y$. So
  $X^\a X^\y \leq X^\d X^\y$. Hence $X^\a \leq X^\d$.
\end{proof}
\begin{corollary}
  Define the relation $\leq$ on the collection of multi-degrees of
  monomials by: For all $\a, \b$
  \begin{equation*}
    \a \leq \b \text{ if, and only if }
    \a+\y \leq \b+\y \text{ for all } \y
  \end{equation*}
  Then $\leq$ is a partial order. Moreover, it is equivalent to the
  monomial ordering.
\end{corollary}
\begin{proof}
  We have (under monomial ordering), that $X^\a \leq X^\b$ if, and only
  if $X^\a X^\y \leq X^\b X^\y$, if, and only if $X^{\a+\y} \leq X^{\b+\y}$.
  This establishes the equivalence between $\leq$ (on multi-degrees)
  and the monomial ordering above.
\end{proof}

\begin{lemma}\label{lemma_7.5.2}
  The monomials over a Noetherian ring are totally ordered under
  monomial ordering.
\end{lemma}
\begin{proof}
  Suppose that $X^\a \nleq X^\b$, and $X^\b \nleq X^\a$. Then there
  exist $X^\y$ and  $X^\d$ for which  $X^\a X^\y \nleq X^\b X^\y$ and
   $X^\b X^\d \nleq X^\a X^\d$. Observe that $X^\a \neq X^\b$.
   Then we have:
   \begin{equation*}
     X^\b X^\y < X^\a X^\y \text{ and } X^\a X^\d < X^\b X^\d
   \end{equation*}
   which implies $X^\a<X^\b$ and $X^\b < X^\a$ so that $X^\a=X^\b$; a
   contradiction! Hence the monomial ordering must be totally ordered.
\end{proof}
\begin{corollary}
  $1 \leq X^\a$. In particular, the monomials of a multivariate
  polynomial ring are well-ordered.
\end{corollary}
\begin{proof}
  Take $\x=(0, \dots, 0)$, and $1=X^\x$. By total ordering, we have
  either $1 \leq X^\a$ or $X^\a \leq 1$ for any $X^\a$. Now, suppose
  that  $X^\a \leq 1$. Then for any $X^\y$
  \begin{equation*}
    X^\a X^\y \leq 1X^\y=X^\x X^\y
  \end{equation*}
  so that $\a+\y \leq \x+\y$, and hence $\a \leq \x$. Letting
  $\a=(a_1, \dots, a_n)$, this makes $a_i=0$ in $\N$ for all
  $1 \leq i \leq n$. Therefore $\a=\x$ and  $X^\a=X^\x$ in this case.
  This proves that $1 \leq X^\a$ for all $X^\a$.

  Now, obesever that by Hilbert's basis theorem, ideals generated my
  monomials are finitely generated. Consier the polynomial:
  \begin{equation*}
    f(x_1, \dots, x_n)=\sum_{i=1}^m{A_ix_1^{a_{i1}} \dots x_n^{a_{in}}}
  \end{equation*}
  Take $X=x_1 \dots x_n$, and $\a_i=(a_{i1}, \dots, a_{in})$. Then
  \begin{equation*}
    f(x_1, \dots x_n)=f(X)=\sum_{i=1}^m{A_iX^{\a_i}}
  \end{equation*}
  then $(f) \subseteq (X^{\a_1}, \dots, X^{\a_m})$. So that any
  multivariate polynomial is characterized by an ideal finitely
  generated by monomials. Now, let $M$ be the set of monomial terms of
  $f$. We have $M$ is nonempty, since if $f(X)=0$ then
  \begin{equation*}
    f(X)=\sum_{i=1}^m{0X^{\a_i}} \in (X^{\a_1}, \dots, X^{\a_m})
  \end{equation*}
  Now, take the convention $X^{\a_1} \leq \dots \leq X^{\a_m}$. We
  have that $M \subseteq (X^{\a_1}, \dots, X^{\a_m})$, so that $M$ has
  an upper-bound, (possibly $X^\a_m$, and possible $X^{\a_m} \in M$).
  So $M$ is a non-empty totally-ordered set bounded above. Therefore
  by Zorn's lemma, $M$ has a maximum element. Therefore $M$ is a
  well-ordered.
\end{proof}

\begin{definition}
  We define the \textbf{lexicographic order} on monomials of a
  multivariate polynomial ring to be the relation $\leq$ defined by:
  For all $X^\a=x_1^{a_1} \dots x_n^{a_n}$ and $X^\b=x_1^{b_1} \dots
  x_n^{b_n}$
  \begin{equation*}
    X^\a \leq X^\b \text{ if, and only if }
    x_2^{a_2}=x_1^{b_1} \text{ and }
    x_2^{a_2} \dots x_n^{a_n} \leq x_2^{b_2} \dots x_n^{b_n} \text{
    under a monomial ordering}
  \end{equation*}
\end{definition}

\begin{lemma}\label{lemma_7.5.3}
    The lexicographic order on monomials is a monomial ordering.
\end{lemma}
\begin{proof}
  Take $\a=(a_1, \a'), \b= (b_1,\b')$ and $\y=(c_1,\y')$ where
  $\a'=(a_2, \dots, a_n), \b=(b_2, \dots, b_n)$ and $\y'=(c_2, \dots,
  c_n)$. Then $X^\a \leq X^\b$ if, and only if $x_1^{a_1}=x_1^{b_1}$,
  and $X^{\a'} \leq X^{\b'}$. Then observe that
  \begin{equation*}
    x_1^{a_1}x_1^{c_1}=x_1^{a_1+c_1} \leq x_1^{b_1+c_1}=x_1^{b_1}x_1^{c_1}
  \end{equation*}
  and
  \begin{equation*}
    X^{\a'}X^{\y'}=X^{\a'+\y'} \leq X^{\b'+\y'}=X^{\b'}X^{\y'}
  \end{equation*}
  so that $X^\a X^\y \leq X^\b X^\y$.
\end{proof}
