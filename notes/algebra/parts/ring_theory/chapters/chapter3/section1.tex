\section{Multivariate Polynomial Rings}
\label{section_7.1}

\begin{theorem}\label{theorem_7.1.1}
  Let $\af$ be an ideal of a ring $R$ and $\af[x]$ the ideal of $R[x]$ generated by
  $\af$. Then
  \begin{equation*}
    \faktor{R[x]}{\af[x]} \simeq \faktor{R}{\af}[x]
  \end{equation*}
  Moreover, if $\af$ is a prime ideal in $R$, then $\af[x]$ is a prime ideal in
  $R[x]$.
\end{theorem}
\begin{proof}
  Consider the map $\pi:R[x] \xrightarrow{} \faktor{R}{\af}[x]$ given by $f
  \xrightarrow{} f \mod{\af}$. That is, reduce $f$ modulo $\af$ by
  taking $f+\af$. Then $\pi$ is a ring homomorphism with kernel
  $\ker{\pi}=\af[x]$. By the first isomorphism theorem, we get
  \begin{equation*}
    \faktor{R[x]}{\af[x]} \simeq \faktor{R}{\af}[x]
  \end{equation*}
  Now, let $\af$ be a prime ideal in $R$, Then we have that
  $\faktor{R}{\af}$ is an integral domain, hence so is $\faktor{R}{\af}[x]$,
  which makes $\af[x]$ a prime ideal of $R[x]$.
\end{proof}

\begin{example}\label{example_7.1}
  Consider the ideal $n\Z$ in  $\Z$. By above, we have
  \begin{equation*}
    \faktor{\Z[x]}{n\Z[x]} \simeq \faktor{\Z}{n\Z}[x]
  \end{equation*}
  with natural map reduction of polynomials modulo $n$. If $n$ is composite,
  then the ring $\faktor{\Z}{n\Z}[x]$ is not an integral domain. If $n=p$ a
  prime, then  $\faktor{\Z}{n\Z}[x]$ is an integral domain.
\end{example}

\begin{definition}
  We define the \textbf{polynomial ring} in $n$  \textbf{variables} $x_1,
  \dots, x_n$ with \textbf{coefficients} in $R$ recursively to be
  \begin{equation*}
    R[x_1, \dots, x_n]=R[x_1, \dots, x_{n-1}][x_n]
  \end{equation*}
  and is the set of all \textbf{multivariate polynomials} of the form $f(x_1,
  \dots,x_n)=\sum{ax_1^{d_1} \dots x_n^{d_n}}$. We call the monic term
  $x_1^{d_1} \dots x_n^{d_n}$ of $f$ a  \textbf{monomial}. We define the
  \textbf{degree} of a monomial to be $\deg{(x_1^{d_1} \dots
  x_n^{d_n})}=d_1+\dots+d_n$ and we define the \textbf{total degree}
  of $f$ to be $\deg{f}=\max{\{\deg{x_1^{d_1} \dots x_n^{d_n}}\}}$
  (i.e. the maximum degree of all monomials of $f$). If all the
  monomials of $f$ have the same degree, we call  $f$ a \textbf{form}.
\end{definition}

\begin{definition}
  Let $R$ be a ring. We define the \textbf{addition}, $+$, on
  $R[x_1, \dots, x_n]$ recursively to be the addition of polynomials
  in $(R[x_1, \dots, x_{n-1}])[x_n]$. We define
  \textbf{multiplication}, $\cdot$, on $R[x_1, \dots, x_n]$ recursively
  to be the multiplication of polynomials in $(R[x_1, \dots,
  x_{n-1}])[x_n]$.
\end{definition}

\begin{proposition}\label{proposition_7.1.2}
  Let $R$ be a ring. Then  $R[x_1, \dots, x_n]$ is a ring.
\end{proposition}
\begin{proof}
  By definition $R[x_1, \dots, x_n]=(R[x_1, \dots, x_{n-1}])[x_n]$.
  Recurse the definition until we get $(((R[x_1])[x_2]) \dots
  [x_{n-1}])[x_n]$. The result follows by observing that $R[x_1]$ is a
  ring.
\end{proof}
\begin{corollary}
  $R[x_1, \dots, x_n]$ is a ring under the addition and multiplication
  of multivariate polynomials.
\end{corollary}

\begin{example}\label{example_7.2}
  \begin{enumerate}
    \item[(1)] Consider the polynomial ring $\Z[x,y]$ in two variables $x$ and
      $y$ with integer coefficients. Then $p(x,y)=2x^3+xy-y^2$ and has
      $\deg{p}=3$. The polynomial $q(x,y)=-3xy+2y^2+x^2y^3$ has $\deg{q}=5$.
      The sum
      \begin{equation*}
        p+q(x,y)=2x^3-2xy+y^2+x^2y^3 \text{ has degree } \deg{p+q}=5
      \end{equation*}
      and the product
      \begin{equation*}
        pq(x,y)=-6x^4y+4x^3y^2+2x^5y^3-3x^2y^2+5xy^3+x^3y^4-2y^4-x^2y^5
      \end{equation*}
      had degree $\deg{pq}=8$.

    \item[(2)] The polynomial $p(x,y,z)=4y^2z^5-3xy^3z+2x^2y$ over
      $\Z[x,y,z]$ has degree $\deg{p}=7$ and the polynomial
      $q(x,y,z)=5x^2y^3z^4 -9x^2z+7x^2$ has degree $\deg{q}=9$. The
      polynomials
      \begin{equation*}
        p+q(x,y,z)  &=  5x^2y^3z^4+4y^2z^5-3xy^3z+2x^2y-9x^2z+7x^2   \\
      \end{equation*}
      and
      \begin{align*}
        pq(x,y,z)   &=  20x^2y^5z^9-15x^3y^6z^5+10x^4y^4z^4 \\
                    & -36x^2y^2z^6+28x^2y^2z^5+ 27x^3y^3z^2-21x^3y^3z-18x^4yz+14x^4y  \\
      \end{align*}
      have degrees $\deg{(p+q)}=9$
      and $\deg{pq}=16$, respectively.

    \item[(3)] Consider the polynomials $p$ and $q$ of the above example
      over  $\faktor{\Z}{3\Z}$, i.e. as polynomials in
      $\faktor{\Z}{3\Z}[x,y,z]$. Then we have
      \begin{align*}
        p(x,y,z)    &=  xy^2z^5+2x^2y    \\
        q(x,y,z)    &=  2x^2y^3z^4+x^2  \\
      \end{align*}
      which makes
      \begin{equation*}
        p+q(x,y,z)=2x^2y^3z^4+y^2z^5+2x^2y+x^2
      \end{equation*}
      and
      \begin{equation*}
        pq(x,y,z)=2x^2y^5z^9+1x^4y^4z^4+1x^2y^2z^5+14x^4y
      \end{equation*}
      of degrees $\deg{(p+q)}=9$ and $\deg{pq}=16$, still.

    \item[(4)] Let $R$ be an integral domain. Then if $f(x_1, \dots,
      x_n), g(x_1, \dots, x_n)$ are forms of degree $r$ and $s$,
      respectively, over $R$ then $fg(x_1, \dots, x_n)$ is a form of
      degree $r+s$ over $R$. Indeed every monomial term of $f(x_1,
      \dots, x_n)$ is of the form $a_dx_1^{d_1} \dots x_n^{d_n}$ and
      every monomial term of $g(x_1, \dots, x_n)$ is of the form
      $b_cx_1^{c_1} \dots x_n^{c_n}$ (where $d=(d_1, \dots, d_n)$ and
      $c=(c_1, \dots, c_n)$). Then every monomial term of $fg(x_1,
      \dots, x_n)$ is of the form $t_ex_1^{e_1} \dots x_n^{e_n}$,
      where $e=(e_1, \dots, e_n)$. Now by the multiplication,
      $e_i=d_i+c_i$ for all $1 \leq i \leq n$. Now, since $f(x_1,
      \dots, x_n)$ is a form of degree $r$ and  $g(x_1, \dots, x_n)$
      is a form of degree $s$, then each monomial term of $fg(x_1,
      \dots, x_n)$ has degree $e_1+\dots+e_n=(d_1+c_1)+\dots+(d_n+c_n)=
      (d_1+\dots+d_n)+(c_1+\dots+c_n)=r+s$; which makes $fg(x_1,
      \dots, x_n)$ a from of degree $r+s$.

    \item[(5)] Again, let $R$ be an integral domain, and let $f(x_1,
      \dots, x_n)$ be a form of degree $r$ over $R$. Suppose $f(x_1,
      \dots, x_n)$ has a factor $g(x_1, \dots, x_n)$. Then
      \begin{equation*}
        f(x_1, \dots, x_n)=g(x_1, \dots, x_n)h(x_1, \dots, x_n)
      \end{equation*}
      indeed, let $g(x_1, \dots, x_n)$ and $h(x_1, \dots, x_n)$ have
      monomial terms of the form $a_dx_1^{d_1} \dots x_n^{d_n}$ and
      $b_cx_1^{c_1} \dots x_n^{c_n}$. Then $f(x_1, \dots, x_n)=
      gh(x_1, \dots, x_n)$ have monomial terms of the form $t_ex_1^{e_1} \dots x_n^{e_n}$ where
      each $e_i=d_i+c_i$. Observe then $e_1+\dots+e_n=
      (d_1+\dots+d_n)+(c_1+\dots+c_n)=r$ for all monomial terms. Now,
      if $g(x_1, \dots, x_n)$ were not a form, then there is some
      monomial term of $g(x_1, \dots, x_n)$ for which
      $(d_1+\dots+d_n)+(c_1+\dots+c_n) \neq r$, which contradicts that
      $f(x_1, \dots, x_n)$ is a form. Therefore we must have that
      $g(x_1, \dots, x_n)$ (and respectively $h(x_1, \dots, x_n)$)
      must be forms over $R$.
  \end{enumerate}
\end{example}

\begin{proposition}\label{proposition_7.1.3}
  Let $R$ be a commutative ring, and  $\pi$ a permutation of the set  $\{1,
  \dots n\}$. Then $R[x_1, \dots, x_n] \simeq R[x_{\pi(1)}, \dots,
  x_{\pi(n)}]$. That is, multivariate polynomial rings are independent of the
  ordering of their variables.
\end{proposition}
\begin{proof}
  Define the map $\Pi:R[x_1, \dots, x_n] \xrightarrow{} R[x_{\pi(1)}, \dots,
  x_{\pi(n)}]$ termwise by first sending $x_1 \dots x_n \xrightarrow{}
  x_{\pi(1)} \dots x_{\pi(n)}$. Then notice that $\Pi$ defines a ring
  homomorphism, and moreover, for any  $f \in R[x_1, \dots, x_n]$, $\Pi$
  permutes the terms of $f$. So that  $\Pi$ dictates the required isomorphism.
\end{proof}

\begin{example}\label{label_7.3}
\begin{enumerate}
\item[(1)] Consider the ideals $(x)$ and $(x,y)$ in $\Q[x,y]$. We have
that $(x)$ is a prime ideal in $\Q[x,y]$, since $\Q[x,y] \simeq
\Q[y,x]=\Q[y][x]$. Moreover, let $fg \in (x,y)$ so that
$fg(x,y)=xyr(x,y)$ for some $r \in \Q[x,y]$. Then $xy|fg$ which
makes  $xy|f$ or  $xy|g$, so that $f \in (x,y)$ or $g \in (x,y)$.
This makes $(x,y)$ a prime ideal. Notice, however, that $(x)
\subseteq (x,y)$, so that $(x)$ is not maximal. $(x,y)$, however is
a maximal ideal in $\Q[x,y]$.

\item[(2)] Notice that $(x,y)$ is a prime ideal in $\Z[x,y]$, since
$\Z[x,y]$ is a subring of $\Q[x,y]$, and $(x,y)$ is prime in
$\Q[x,y]$. Similarly, $(2,x,y)$ is prime in $\Z[x,y]$. Notice
however that $(x,y) \subseteq (2,x,y)$ so that $(x,y)$ is not
maximal in $\Z[x,y]$; $(2,x,y)$ is maximal in $\Z[x,y]$.

\item[(3)] Notice that $(x,y)$ is not a principle ideal in $\Q[x,y]$.
  Indeed, suppose there is an $f(x,y) \in \Q[x,y]$ for which
  $(x,y)=(f)$. Then $x \in (f)$ and $y \in (f)$, so that $f \divides
  x$ and  $f \divides y$. Hence $f \divides (x+y)$, so
  $x+y=f(x,y)g(x,y)$ for some $g \in \Q[x,y]$ not a unit. Now,
  $\deg{f} \leq 1$, and since $g$ is not a unit, $\deg{g}=1$, which
  makes $\deg{f}=0$. Therefore $f(x,y)$ is a unit in $\Q[x,y]$ so that
  $(f)=(1)=\Q[x,y]$ which is impossible.
\end{enumerate}
\end{example}

\begin{proposition}\label{proposition_7.1.4}
    Let $k$ be an infinite field, and suppose  $f \in k[x_1, \dots, x_n]$ is a
    polynomial such that $f(a_1,\dots,a_n)=0$ for all $a_1, \dots, a_n \in k$.
    Then $f$ must be the zero polynomial.
\end{proposition}
\begin{proof}
    Observe that
    \begin{equation*}
        f(x_1, \dots, x_n)=\sum{f_i(x_1,\dots,x_{n-1})x_n^i}
    \end{equation*}
    where $f \in k[x_1, \dots, x_{n-1}]$. Now, for $n=1$, if we have  $f \in
    k[x_1]$, then if $f(a_1)=0$ for all $a_1 \in k$, then $f(x_1)=0$ in $k$.
    Suppose now that the the polynomial $f(x_1,\dots,x_n)$ satisfies
    \begin{equation*}
        f(a_1, \dots, a_n)=0 \text{ for all } a_1,\dots,a_n \in k
    \end{equation*}
    and consider a the polynomial $f$ as a polynomial in $k[x_1, \dots,
    x_n,x_{n+1}] \simeq k[x_1,\dots,x_n][x_{n+1}]$, then we have
    \begin{equation*}
        f(x_1,\dots, x_n, x_{n+1})=\sum{f_i(x_1, \dots, x_n)x_{n+1}^i}
    \end{equation*}
    Now, if $f(a_1, \dots, a_n,a_{n+1})=0$ for all $a_1, \dots, a_n,a_{n+1} \in
    k$, we have that
    \begin{equation*}
        f(a_1,\dots, a_n, a_{n+1})=\sum{f_i(a_1, \dots, a_n)a_{n+1}^i}=0
    \end{equation*}
    Now, $f$ has finitely many roots when considered as a polynomial in $k
    [x_1, \dots, x_n]$, and if $ a_1, \dots, a_n$ are roots of $f_i$, by
    hypothesis, $f_i=0$ for each  $i$, so that
    \begin{equation*}
        f(a_1,\dots, a_n, a_{n+1})=\sum{f_i(a_1, \dots,
        a_n)a_{n+1}^i}=\sum{0a_{n+1}^i}=0
    \end{equation*}
    which makes $f(x_1, \dots, x_{n+1})=0$ in $k[x_1, \dots, x_{n+1}]$.
\end{proof}

\begin{proposition}\label{proposition_7.1.5}
    If $k$ is a field, then there exist infinitely many monic irreducible
    polynomials in $k[x]$.
\end{proposition}
\begin{proof}
    Suppose that there exist finitely many monic irreducible polynomials in
    $k[x]$, and let $p_1, \dots, p_n$ an enumeration of all of them. Consider
    the polynomial
    \begin{equation*}
        p(x)=p_1(x) \dots p_n(x)+1
    \end{equation*}
    Then for every $p_i$, we have that  $p \equiv 1 \mod{p_i}$ so that no $p_i$
    divides  $p$.  Since ech  $p_i$ is monic and irreducible, this makes  $p$
    irreducible, contradiction the given enumeration.
\end{proof}

\begin{theorem}\label{theorem_7.1.6}
  Let $\phi:R \xrightarrow{} S$ be a ring homomorphism, and let $s_1,
  \dots, s_n \in S$. Then $\phi$ extends to a ring homomorphism
  $\Phi:R[x_1, \dots, x_n] \xrightarrow{} S$ such that $\Phi(x_i)=s_i$
  for all $1 \leq i \leq n$. Moreover, such an extension is unique.
\end{theorem}
\begin{proof}
  Define the homomorphism $\Phi:R[x_1, \dots, x_n] \xrightarrow{} S$
  by $\Phi:f(x_1, \dots, x_n) \xrightarrow{} f(s_1, \dots, s_n)$, and
  $\Phi(x)=\phi(x)$ for all $x \in R$. Then for $f(x_1, \dots, x_n),
  g(x_1, \dots, x_n) \in R[x_1, \dots, x_n]$ we have:
  \begin{align*}
    \Phi(f(x_1, \dots, x_n)+g(x_1, \dots, x_n)) &= \Phi((f+g)(x_1, \dots, x_n)) \\
    &= (f+g)(s_1, \dots, s_n) \\
    &= f(s_1, \dots, s_n)+g(s_1, \dots, s_n) \\
    &= \Phi(f(x_1, \dots, x_n))+\Phi(g(x_1, \dots, x_n)) \\
  \end{align*}
  and
  \begin{align*}
    \Phi(f(x_1, \dots, x_n)g(x_1, \dots, x_n)) &= \Phi((fg)(x_1, \dots, x_n)) \\
    &= (fg)(s_1, \dots, s_n) \\
    &= f(s_1, \dots, s_n)g(s_1, \dots, s_n) \\
    &= \Phi(f(x_1, \dots, x_n))\Phi(g(x_1, \dots, x_n)) \\
  \end{align*}
  moreover, setting $m_i(x_1, \dots, x_n)=x_i$, we have for all $1
  \leq i \leq n$ $\Phi(x_i)=\Phi(m_i(x_1, \dots, x_n))=m_i(s_1, \dots,
  s_n)=s_i$. Lastly, Taking for any $c \in R$ the constant
  polynomial $c(x_1, \dots, x_n)=c+\sum{0x_1^{d_1} \dots x_n^{d_n}}$,
  we get $\Phi(c)=\Phi(c(x_1 ,\dots, x_n))=\phi(c)+\sum{0s_1^{d_1}
  \dots s_n^{d_n}}$, so that $\Phi(c)=\phi(c)$ for all $c \in R$.
  Indeed, $\Phi$ is a required extension for $\phi$.

  Now, suppose there is another homomorphism $\Psi:R[x_1, \dots, x_n]
  \xrightarrow{} S$ for which $\Psi|_R=\phi$ and $\Psi(x_i)=s_i$ for
  all $1 \leq i \leq n$. Define the the homomorphism $\Phi-\Psi:R[x_1,
  \dots, x_n] \xrightarrow{} S$ by $\Phi-\Psi(f(x_1, \dots,
  x_n))=\Phi(f(x_1, \dots, x_n))-\Psi(f(x_1, \dots, x_n))$. Observe
  that since $\Phi|_R=\phi=\Psi|_R$, then $\Phi=\Psi$ on $R$. Now, by
  definition, we have $\Phi-\Psi(x_i)=\Phi(x_i)-\Psi(x_i)=s_i-s_i=0$,
  so $x_i \in \ker{(\Phi-\Psi)}$. Then observe if $f(x_1, \dots,
  x_n)=\sum{ax^{d_1} \dots x_n^{d_n}}$, then
  \begin{align*}
    \Phi(f(x_1, \dots, x_n))=\sum{a\Phi(x_1)^{d_1} \dots \Phi(x_n)^{d_n}}
    &= \sum{as_1^{d_1} \dots s_n^{d_n}} \\
  \end{align*}
  and
  \begin{align*}
    \Psi(f(x_1, \dots, x_n))=\sum{a\Psi(x_1)^{d_1} \dots \Psi(x_n)^{d_n}}
    &= \sum{as_1^{d_1} \dots s_n^{d_n}} \\
  \end{align*}
  so that $\Phi-\Psi(f(x_1, \dots, x_n))=\Phi(f(x_1, \dots,
  x_n))\Psi(f(x_1, \dots, x_n))=0$ hence $\ker{(\Phi-\Psi)}=R[x_1,
  \dots, x_n]$. That is, for every $f(x_1, \dots, x_n) \in R[x_1,
  \dots, x_n]$, $\Phi-\Psi(f(x_1, \dots, x_n))=0$. That is,
  $\Phi(f(x_1, \dots, x_n))=\Psi(f(x_1, \dots, x_n))$, and we are
  done.
\end{proof}
