\section{Multivariate Polynomial Rings.}

\begin{theorem}\label{3.1.1}
    Let $I$ be an ideal of  $R$ and  $I[x]$ the ideal of $R[x]$ generated by
    $I$. Then
    \begin{equation*}
        \faktor{R[x]}{I[x]} \simeq \faktor{R}{I}[x]
    \end{equation*}
    Moreover, if $I$ is a prime ideal in $R$, then $I[x]$ is a prime ideal in
    $R[x]$.
\end{theorem}
\begin{proof}
    Consider the map $\pi:R[x] \xrightarrow{} \faktor{R}{I}[x]$ given by $f
    \xrightarrow{} f \mod{I}$. That is, reduce $f$ modulo  $I$.  Then $\pi$ is a
    ring homomorphism with kernel $\ker{\pi}=I[x]$. By the first isomorphism
    theorem, we get
    \begin{equation*}
        \faktor{R[x]}{I[x]} \simeq \faktor{R}{I}[x]
    \end{equation*}
    Now, let $I$ be a prime ideal in $R$, Then we have that $\faktor{R}{I}$ is
    an integral domain, hence, so is $\faktor{R}{I}[x]$, which makes $I[x]$ a
    prime ideal of $R[x]$.
\end{proof}

\begin{example}\label{3.1}
    Consider the ideal $n\Z$ in  $\Z$. By above, we have
    \begin{equation*}
        \faktor{\Z[x]}{n\Z[x]} \simeq \faktor{\Z}{n\Z}[x]
    \end{equation*}
    with natural map reduction of polynomials modulo $n$. If $n$ is composite,
    then the ring $\faktor{\Z}{n\Z}[x]$ is not an integral domain. If $n=p$ a
    prime, then  $\faktor{\Z}{n\Z}[x]$ is an integral domain.
\end{example}

\begin{definition}
    We define the \textbf{polynomial ring} in $n$  \textbf{variables} $x_1,
    \dots, x_n$ with \textbf{coefficients} in $R$ inductively to be
    \begin{equation*}
        R[x_1, \dots, x_n]=R[x_1, \dots, x_{n-1}][x_n]
    \end{equation*}
    and is the set of all \textbf{multivariate polynomials} of the form $f(x_1,
    \dots,x_n)=\sum{ax_1^{d_1} \dots x_n^{d_n}}$. We call the monic term
    $x_1^{d_1} \dots x_n^{d_n}$ of $f$ a  \textbf{monomial}. We define the
    \textbf{degree} of a monomial to be $\deg{x_1^{d_1} \dots
    x_n^{d_n}}=d_1+\dots+d_n$ and we define the \textbf{degree} of $f$ to be
    $\deg{f}=\max{\{\deg{x_1^{d_1} \dots x_n^{d_n}}\}}$ (i.e. the maximum degree
    of all monomials of $f$). If all the monomials of $f$ have the same degree,
    we call  $f$  \textbf{homogeneous}, or, a \textbf{form}.
\end{definition}

\begin{lemma}\label{3.1.2}
    Let $R$ be a ring. Then  $R[x_1, \dots, x_n]$ is a ring.
\end{lemma}

\begin{example}\label{3.2}
    \begin{enumerate}
        \item[(1)] Consider the polynomial ring $\Z[x,y]$ in two variables $x$ and
            $y$ with integer coefficients. Then $p(x,y)=2x^3+xy-y^2$ and has
            $\deg{p}=3$. The polynomial $q(x,y)=-3xy+2y^2+x^2y^3$ has $\deg{q}=5$.
            The sum
            \begin{equation*}
                p+q(x,y)=2x^3-2xy+y^2+x^2y^3 \text{ has degree } \deg{p+q}=5
            \end{equation*}
            and the product
            \begin{equation*}
                pq(x,y)=-6x^4y+4x^3y^2+2x^5y^3-3x^2y^2+5xy^3+x^3y^4-2y^4-x^2y^5
            \end{equation*}
            had degree $\deg{pq}=8$.

        \item[(2)] The polynomial $p(x,y,z)=4y^2z^5-3xy^3z+2x^2y$ over
            $\Z[x,y,z]$ has degree $\deg{p}=7$ and the polynomial
            $q(x,y,z)=5x^2y^3z^4 -9x^2z+7x^2$ has degreee $\deg{q}=9$. The
            polynomials
            \begin{equation*}
                p+q(x,y,z)  &=  5x^2y^3z^4+4y^2z^5-3xy^3z+2x^2y-9x^2z+7x^2   \\
            \end{equation*}
            and
            \begin{align*}
                pq(x,y,z)   &=  20x^2y^5z^9-15x^3y^6z^5+10x^4y^4z^4-36x^2y^2z^6+ \\
                                28x^2y^2z^5+ 27x^3y^3z^2-21x^3y^3z-18x^4yz+14x^4y
            \end{align*}
            have degrees $\deg{(p+q)}=9$
            and $\deg{pq}=16$, respectively.

        \item[(3)] Consider the polynomials $p$ and $q$ of the above example
            over  $\faktor{\Z}{3\Z}$, i.e. as polynomials in
            $\faktor{\Z}{3\X}[x,y,z]$. Then we have
            \begin{align*}
                p(x,y,z)    &=  xy^2z^5+2x^2y    \\
                q(x,y,z)    &=  2x^2y^3z^4+x^2  \\
            \end{align*}
            which makes
            \begin{equation*}
                p+q(x,y,z)=2x^2y^3z^4+y^2z^5+2x^2y+x^2
            \end{equation*}
            and
            \begin{equation*}
                pq(x,y,z)=2x^2y^5z^9+1x^4y^4z^4+1x^2y^2z^5+14x^4y
            \end{equation*}
            of degrees $\deg{(p+q)}=9$ and $\deg{pq}=16$, still.
    \end{enumerate}
\end{example}

\begin{lemma}\label{3.1.3}
    Let $R$ be a commutative ring, and  $\pi$ a permutation of the set  $\{1,
    \dots n\}$. Then $R[x_1, \dots, x_n] \simeq R[x_{\pi(1)}, \dots,
    x_{\pi(n)}]$. That is, multivariate polynomial rings are independent of the
    ordering of their variables.
\end{lemma}
\begin{proof}
    Define the map $\Pi:R[x_1, \dots, x_n] \xrightarrow{} R[x_{\pi(1)}, \dots,
    x_{\pi(n)}]$ termwise by first sending $x_1 \dots x_n \xrightarrow{}
    x_{\pi(1)} \dots x_{\pi(n)}$. Then notice that $\Pi$ defines a ring
    homomorphism, and moreover, for any  $f \in R[x_1, \dots, x_n]$, $\Pi$
    permutes the terms of $f$. So that  $\Pi$ dictates the required isomorphism.
\end{proof}

%\begin{example}\label{3.3}
    %\begin{enumerate}
        %\item[(1)] Consider the ideals $(x)$ and $(x,y)$ in $\Q[x,y]$. We have
            %that $(x)$ is a prime ideal in $\Q[x,y]$, since $\Q[x,y] \simeq
            %\Q[y,x]=\Q[y][x]$. Moreover, let $fg \in (x,y)$ so that
            %$fg(x,y)=xyr(x,y)$ for some $r \in \Q[x,y]$. Then $xy|fg$ which
            %makes  $xy|f$ or  $xy|g$, so that $f \in (x,y)$ or $g \in (x,y)$.
            %This makes $(x,y)$ a prime ideal. Notice, however, that $(x)
            %\subseteq (x,y)$, so that $(x)$ is not maximal. $(x,y)$, however is
            %a maximal ideal in $\Q[x,y]$.

        %\item[(2)] Notice that $(x,y)$ is a prime ideal in $\Z[x,y]$, since
            %$\Z[x,y]$ is a subring of $\Q[x,y]$, and $(x,y)$ is prime in
            %$\Q[x,y]$. Similarly, $(2,x,y)$ is prime in $\Z[x,y]$. Notice
            %however that $(x,y) \subseteq (2,x,y)$ so that $(x,y)$ is not
            %maximal in $\Z[x,y]$; $(2,x,y)$ is maximal in $\Z[x,y]$.

        %\item[(3)] Notice that $(x,y)$ is not a principle ideal in $\Q[x,y]$.
            %Suppose that it were, then $(x,y)=(f)$ for some $f(x,y) \in
            %\Q[x,y]$. Then we have that $x \in (f)$ and $y \in (f)$ so that
            %$f|x$ and  $f|y$. That is, $x=f(x,y)r(x,y)$ and $y=f(x,y)q(x,y)$.
            %Then $x+y=f(x,y)(r(x,y)+q(x,y))$. Notice also that $\deg{f} \leq 1$.
            %Then if $\deg{f}=0$, $f$ is a unit, and we get  $(f)=\Q[x,y]$. On
            %the other hand, if $\deg{f}=1$, and since
            %$x+y=f(x,y)(r(x,y)+q(x+y))$, we have that
    %\end{enumerate}
%\end{example}
