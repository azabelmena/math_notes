\section{Extensions and Contractions of Ideals}
\label{section_5.6}

In this section, we assume that all rings are commutative rings with
identity $1 \neq 0$.

\begin{definition}
  Let $f:R \xrightarrow{} S$ a ring homomorphism. The extension of the ideal
  $\af$ in $R$ is the ideal $\af^e$ generated by $f(\af)$ in $S$. That is,
  \begin{equation*}
    \af^e=Sf(\af)=
    \Sig{\{} \sum{y_if(x_i) : y_i \in S \text{ and } x_i \in \af} \Sig{\}}
  \end{equation*}
\end{definition}

\begin{definition}
  Let $f:R \xrightarrow{} S$ be a ring homomorphism. We define the
  \textbf{contraction} of the ideal $\bf$ in $S$ to be the preimage
  $\inv{f}(\bf)$, and denote it $\bc^c$.
\end{definition}

\begin{proposition}\label{proposition_5.6.2}
    Let $f:R \xrightarrow{} S$ a ring homomorphism, and $\bf$ and ideal of
    $\bf$. Then $\bf$ is an ideal of $R$. Moreover, if $\bf$ is prime in $S$,
    then  $\bf^c$ is prime in $R$.
\end{proposition}
\begin{proof}
    Let $x \in \bf^c$. Then  $f(x) \in \bf$, so that $-f(x)=f(-x) \in
    \bf$, which puts $-x \in \bf^c$; similarly, we get $x+y \in \bf^c$ whenever
    $x,y \in \bf^c$. Lastly, notice that if  $a \in R$, and  $x \in \bf^c$, then
     $f(a)f(x)=f(ax) \in \bf$, so that $\bf^c$ is an ideal.

     Now, suppose that  $\bf$ is prime. Then since  $\bf \neq (1_S)$, $\bf^c
     \neq (1_R)$. Now, let $ab \in \bf^c$. Then  $f(ab)=f(a)f(b) \in
     \bf$. Since $\bf$ is prime, this puts  $f(a) \in \bf$ or $f(b) \in
     \bf$; that is, $a \in \bf^c$ or  $b \in \bf^c$. Therefore, $\bf^c$ must
     also be prime.
\end{proof}

\begin{example}\label{example_1.21}
    Let $f:R \xrightarrow{} S$ a ring homomorphism. We have that for any prime
    ideal $\bf$ of $S$, $\bf^c$ is prime. The same is not true for extensions.
    If $\af$ is prime in $R$,  $\af^e=Sf(\af)$ need not be prime in $S$.
\end{example}

\begin{proposition}\label{proposition_5.6.3}
    Let $f:R \xrightarrow{} S$ be a ring homomorphism, with $f=\i \circ \pi$,
    where  $\pi$ is onto and  $\i$ is 1--1. Then there exists a 1--1
    correspondece between the ideals $f(R)$ and the ideals of $R$ containing
     $\ker{f}$. Moreover, prime ideals correspond to prime ideals.
     \[\begin{tikzcd}
        R \\
        {f(R)} & S
        \arrow["\pi"', from=1-1, to=2-1]
        \arrow["\iota"', from=2-1, to=2-2]
        \arrow["{f=\iota \circ \pi}", from=1-1, to=2-2]
      \end{tikzcd}\]
\end{proposition}

\begin{example}\label{example_1.22}
    Consider the map $\Z \xrightarrow{} \Z[i]$ where $i^2=-1$. R prime ideal
    $(p)=p\Z$ may or may not be prime when extended to $\Z[i]$. Now, $\Z[i]$ is
        a PID, so that we have the following.
        \begin{enumerate}
            \item[(1)] $(2)^e=((1+i)^2)$ in $\Z[i]$; that is, it is the square
                of a prime ideal in $\Z[i]$.

            \item[(2)] If $p \equiv 1 \mod{4}$, then $(p)^e$ is the product of
                two prime ideals in $\Z[i]$, and if $p \equiv 3 \mod{4}$,
                $(p)^e$ is a prime ideal in $\Z[i]$.
        \end{enumerate}
\end{example}

\begin{proposition}\label{proposition_5.6.4}
    Let $f:R \xrightarrow{} S$ be a ring homomorphism. Then the following are
    true for ideals $\af$ and  $\bf$ of  $R$ and  $S$, respectively.
    \begin{enumerate}
        \item[(1)] $\af \subseteq \af^{ec}$ and $\bf^{ce} \subseteq \bf$.

        \item[(2)] $\af^e=\af^{ece}$, and $\bf^c=\bf^{cec}$.

        \item[(3)] If $C$ is the set of all contracted ideals in $R$, and  $E$
            is the set of all extended ideals in  $S$, then
            \begin{equation*}
                C=\{\af \subseteq R : \af=\af^{ec}\} \text{ and }
                E=\{\bf \subseteq S : \bf=\bf^{ce}\}
            \end{equation*}
            Moreover, there exists a 1--1 correspondence of $C$ onto $E$  given
            by the map $\af \xrightarrow{} \af^e$.
    \end{enumerate}
\end{proposition}
\begin{proof}
    First, consider $\af$ in  $R$. Then  $\af^e=Sf(\af)$, so that if $x \in
    \af$, then  $f(x) \in f(\af)$, that is $x \in \af^{ec}$. Similarly, we
    get $\bf^{ce} \subseteq \bf$.

    Now, for the second assertion, we have
    \begin{equation*}
        (\bf^{ce})^c \subseteq \bf^c \subseteq (\bf^c)^{ec}
    \end{equation*}
    so that $\bf^c=\bf^{cec}$. Similarly, we get $\af^e=\af^{ece}$.

    Finally, let $\af \in C$. Then there is a $\bf$ in $S$ for which
    $\af=\bf^c$. Then  $\af^e=\bf^{ce}=\bf^{cec}=\af^{ec}$. Conversely, if
    $\af=\af^{ec}$, then $\af$ is the contraction of $\af^e$. We use a similar
    argument to prove the result for $E$.
\end{proof}

\begin{proposition}\label{proposition_5.6.5}
    If $\af_1,\af_2$ are ideals of $R$ and  $\bf_1,\bf_2$ are ideals of $S$,
    then the following are true.
    \begin{enumerate}
        \item[(1)] $(\af_1+\af_2)^e=\af_1^e+\af_2^e$ and $\bf_1^c+\bf_2^c
            \subseteq (\bf_1+\bf_2)^c$.

        \item[(2)] $(\af_1 \cap \af_2)^e \subseteq \af_1^e \cap \af_2^e$ and
            $\bf_1^c \cap \bf_2^c=(\bf_1 \cap \bf_2)^c$.

        \item[(3)] $(\af_1\af_2)^e=\af_1^e\af_2^e$ and $\bf_1^c\bf_2^c
            \subseteq (\bf_1\bf_2)^c$.

        \item[(4)] $(\af_1:\af_2)^e \subseteq (\af_1^e:\af_2^e)$ and
            $(\bf_1:\bf_2)^c=(\bf_1^c:\bf_2^c)$.

        \item[(5)] $(\sqrt{\af})^e \subseteq \sqrt{\af^e}$ and
            $(\sqrt{\bf})^c=\sqrt{\bf^c}$.
    \end{enumerate}
\end{proposition}
\begin{corollary}
  Let $C$ be the set of all contracted ideals and $E$ the set of all extended
  ideals. Then $C$ is closed under the ideal operations of intersections,
  quotient, and radicals, and  $E$ is closed under the ideal operations of
  addition and products.
\end{corollary}
\begin{proof}
  Let $\af,\bf \in C$, then $(\af \cap \bf)^{ec} \subseteq (\af^e \cap
  \bf^e)^c=(\af^{ec} \cap \bf^{ec})=(\af \cap \bf) \subseteq (a \cap b)^{\ec}$;
  and $(\sqrt{\af})^{ec} \subseteq (\sqrt{\af^e})^c=\sqrt{\af^{ec}} \subseteq
  \sqrt{\af} \subseteq (\sqrt{\af})^{ec}$. Finally, $(\af:\bf) \subseteq
  (\af:\bf)^{ec} \subseteq (\af^{ec}:\bf^{ec})=(\af:\bf)$.

  Now, let $\af, \bf \in E$. Then $\af+\bf=\af^{ce}+\bf^{ce}=(\af^c+\bf^c)^e
  \subseteq (\af+\bf)^{ce} \subseteq \af+\bf$. Likewise, we get the same result
  for $\af\bf$ using the same sequence of operations.
\end{proof}
