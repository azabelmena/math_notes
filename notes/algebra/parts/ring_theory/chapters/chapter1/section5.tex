\section{Operations on Ideals}
\label{section_5.5}

\begin{theorem}[The Binomial Theorem]\label{5.5.1}
    Let $R$ be a commutative ring with identity. Then for all  $x,y \in R$, and
     $n \in \Z^+$
     \begin{equation*}
         (x+y)^n=\sum_{r+s=n}{{n \choose r}x^ry^s}
     \end{equation*}
\end{theorem}

\begin{proposition}\label{proposition_5.5.2}
    Let $R$ be a commutative ring with identity, and $\rf$ the set of all
    nilpotent elements of $R$. Then $\rf$ is an ideal of $R$.
\end{proposition}
\begin{proof}
    If $x \in \rf$, then there is an  $n \in \Z^+$ for which  $x^n=0$, notice
    that this also implies that  $(-x)^n=0$, so that $-x \in \rf$. Now, let $x,y
    \in \rf$. Then for some $m,n \in \Z^+$, we have $x^m=0$ and $y^n=0$.
    Consider then $(x+y)^{m+n-1}$. By the binomial theorem, we have
     \begin{equation*}
         (x+y)^n=\sum_{r+s=m+n-1}{{m+n-1 \choose r}x^ry^s}
     \end{equation*}
     Now, since $r+s=m+n-1$, notice that either  $r<m$, or  $s<n$, but not both.
     This makes  $x^ry^s=0$ for all  $r,s$, so that  $(x+y)^{m+n-1}=0$. This
     makes $\rf$ a subgroup of  $R$. Lastly, notice that if  $a \in R$, and  $x
     \in \rf$, then for some  $n \in \Z^+$,  $ax^n=(ax)^n=0$, which makes $\rf$
     an ideal.
\end{proof}
\begin{corollary}
    $\faktor{R}{\rf}$ has no nonzero nilpotent elements.
\end{corollary}
\begin{proof}
    Let $x \in R$ be nilpotent, then $x \in \rf$, so that $x+\rf=\rf$ in
    $\faktor{R}{\rf}$. Therefore, the only nilpotent element of
    $\faktor{R}{\rf}$ is $\rf$ itself.
\end{proof}

\begin{definition}
    We define the \textbf{nilradical} of a commutative ring $R$, with identity,
    to be the ideal, $\Nil{R}$, consisting of all nilpotent elements of $R$.
\end{definition}

\begin{proposition}\label{proposition_5.1.3}
    Let $R$ be a commutative ring with identity. Then $\Nil{R}$ is the
    intersection of all prime ideals of $R$; i.e.
    \begin{equation*}
        \Nil{R}=\bigcap_{\pf \subseteq R}{\pf} \text{ where } \pf \text{ is a
        prime ideal of } R
    \end{equation*}
\end{proposition}
\begin{proof}
    Let $\rf$ be the intersection of all prime ideals of $R$. Suppose that  $x
    \in R$ is nilpotent. Then $x^n=0$ for some $n \in \Z^+$, so that  $x^n \in
    \pf$. Since $\pf$ is prime, $x \in \pf$, which puts $x \in \rf$.

    Conversely, suppose that  $x \in R$ is not nilpotent, and let $\Sigma$ be
    the set of all ideals $\af$ for which $x \notin \af$. Notice that since $0$
    is nilpotent in $R$, $0 \in \Sigma$, so that $\Sigma$ is nonempty.
    Therefore, by Zorn's proposition, $\Sigma$ has a maximal element $\pf$. We claim
    that this $\pf$ is prime. Suppose that  $a,b \notin \pf$, then  $\pf
    \subseteq \pf+(a)$ and $\pf \not\subseteq \pf+(b)$. So $\pf+(a), \pf+(b)
    \notin \Sigma$, by the maximality of $\pf$. This puts  $x^m \in \pf+(a)$ and
    $x^n \in \pf+(b)$, for some $n,m \in \Z^+$. Thus  $x^{m+n} \in \pf+(ab)
    \not\subseteq \Sigma$. Therefore, $ab \notin \pf$, which makes $\pf$ a prime
    ideal for which $x \notin \pf$; i.e. $x \notin \rf$.
\end{proof}

\begin{definition}
    We define the \textbf{Jacobson radical} of a commutative ring $R$, with
    identity, to be the intersection of all maximal ideals of $R$. We denote it
    by  $\Jac{R}$.
\end{definition}

\begin{proposition}\label{proposition_5.5.4}
    Let $R$ be a commutative ring with identity. Then  $x \in \Jac{R}$ if, and
    only if $1-xy$ is a unit in $R$ for some $y \in R$.
\end{proposition}
\begin{proof}
    Suppose that $x \in \Jac{R}$, but that $1-xy$ is not a unit of $R$. Then by
    proposition \ref{proposition_5.4.3}, we have $(1-xy) \subseteq \mf$ for some maximal ideal
    $\mf$ of $R$; hence $1-xy \in \mf$. However, since $x \in \Jac{R}$, then $xy
    \in \mf$, which puts $1 \in \mf$, so that $\mf=(1)$ which contradicts that
    $\mf$ is maximal. Therefore,  $1-xy$ must be a unit.

    Conversely, suppose that  $x \in \mf$ for some maximal ideal $\mf$ of $R$.
    Then  $(\mf,x)=(1)$ so that $u+xy=1$ for some  $u \in \mf$ and  $y \in R$.
    This makes  $1-xy \in \mf$ so that $1-\mf$ is not a unit.
\end{proof}

\begin{definition}
  We define the \textbf{sum} of two ideals $\af$ and  $\bf$ to be the set
  \begin{equation*}
    \af+\bf=\{x+y : x \in \af \text{ and } y \in \bf\}
  \end{equation*}
  If $\{\af_\a\}$ is an arbitrary (not necesarily countable) collection of
  ideals, then we define the \textbf{sum} of $\{\af_\a\}$ to be
  \begin{equation*}
    \sum{\af_\a}=
    \Big{\{} \sum{x_\a} : x_\a \in \af_\a \text{ and almost all but finitely
    many } x_\a=0 \Big{\}}
  \end{equation*}
\end{definition}

\begin{proposition}\label{proposition_5.5.5}
  The following are true for any ideals $\af$ and $\bf$:
  \begin{enumerate}
    \item[(1)] $\af+\bf$ is an ideal, and is the smallest ideal containing both
      $\af$ and  $\bf$.

    \item[(2)] $\af \cap \bf$ is an ideal, and is the largest ideal
      contained in both $\af$ and $\bf$.
  \end{enumerate}
\end{proposition}
\begin{corollary}
  Let $R$ be a ring with ideals $\af$ and $\bf$. Then we have the lattice
  \[\begin{tikzcd}
  & (1) \\
  & {\af+\bf} \\
    \af && \bf \\
        & \af \cap \bf \\
        & {(0)}
        \arrow[no head, from=1-2, to=2-2]
        \arrow[no head, from=2-2, to=3-1]
        \arrow[no head, from=3-1, to=4-2]
        \arrow[no head, from=4-2, to=3-3]
        \arrow[no head, from=3-3, to=2-2]
        \arrow[no head, from=4-2, to=5-2]
  \end{tikzcd}\]
\end{corollary}
\begin{corollary}
  The following are true:
  \begin{enumerate}
    \item[(1)] If $\{\af_\a\}$ is a collection of ideals, then $\sum{\af_\a}$ is
      an ideal, and is the smallest ideal containing all $\af_\a$.

    \item[(2)] $\bigcap{\af_\a}$ is an ideal, and is the largest ideal
      containing all $\af_\a$.
  \end{enumerate}
\end{corollary}

\begin{definition}
  We define the \textbf{product} of two ideals $\af$ and $\bf$
  \textbf{generated} all products $xy$ with $x \in \af$ and $b \in \bf$ to be
  the set
  \begin{equation*}
    \af\bf=
    \Big{\{} \sum{x_\a y_\b} : x_\a \in \af \text{ and } y_\b \in \bf \Big{\}}
  \end{equation*}
  If $\{\af_i\}_{i=1}^n$ is a finite collection of ideals, then the
  \textbf{product} is defined to be
  \begin{equation*}
    \prod_{i=1}^{n}{\af_i}=
    \Big{\{} \sum{\prod_{i=1}^n{x_i}} : x_i \in \af_i \Big{\}}
  \end{equation*}
  We define the \textbf{$n$-th power} of an ideal $\af$ to be
  \begin{equation*}
    \af^0=(1) \text{ for } n=0 \text{ and }
    \af^n=\Big{\{} \sum{\prod_{i=1}^n{x_i}} : x_i \in \af  \Big{\}}
    \text{ for } n>0
  \end{equation*}
\end{definition}

\begin{example}\label{example_5.17}
  \begin{enumerate}
    \item[(1)] Let $m,n \in \Z^+$, then $m\Z+n\Z$ is the ideal generated by
      the greatest common divisor $(m,n)$, of $m$ and $n$, and $m\Z \cap n\Z$
      is the ideal generated by the least common multiple $[m,n]$, of $m$ and
      $n$. Moreover, $(mn)=[m,n]$ if, and only if $m$ and $n$ are coprime; i.e.
      $(m,n)=1$. Moreover, if $(m,n)=1$, the ideals $m\Z$ and $n\Z$
      with $m\Z+\n\Z=\Z$ is just the set of all Diophantine equations
      of the form
      \begin{equation*}
        mx+ny=1
      \end{equation*}

    \item[(2)] Consider the ideals $6\Z$ and  $10\Z$ of $\Z$. Then $6\Z+10\Z$ is
      the ideal consisting of all integers of the form  $6x+10y$. Now, for
      $x,y \in \Z$, since $(6,10)=2$, we have that $6\Z+10\Z \subseteq 2\Z$
      since  $6x+10y=2(3x+5y)$. Now, we also have that $2=6 \cdot 2+10 \cdot
      -1$ so that  $2 \in 6\Z+10\Z$ which makes $2\Z \subseteq 6\Z+10\Z$.
      Thus, we have $6\Z+10\Z=2\Z$. In general, we have that $m\Z+n\Z=d\Z$
      where $d=(m,n)$ is the greatest common divisor of $m$ and  $n$. The
      ideal $6\Z10\Z$ gives all integers of the form $6x10y=6 \cdot 10
      (xy)=60(xy)$, so that $6\Z10\Z=60\Z$.

    \item[(3)] Let $I \subseteq \Z[x]$ the ideal of polynomials with even
      constant term. Notce that $2,x=x+0 \in I$ so tht  $4,x^2 \in I^2=II$. So
      that  $4+x^2\in I^2$ which is not in general divisible by elements in $I$.
  \end{enumerate}
\end{example}

\begin{theorem}\label{theorem_5.5.6}
  The following are true for any ideals $\af$,  $\bf$, and $\cf$ in a
  commutative ring with identity:
  \begin{enumerate}
    \item[(1)] $\af+(\bf+\cf)=(\af+\bf)+\cf$, $\af(\bf\cf)=(\af\bf)\cf$, and
      $\af (\bf \cap \cf)=(\af \cap \bf) \cap \cf$.

    \item[(2)] $\af+\bf=\bf+\af$, $\af\bf=\bf\af$, and  $\af \cap \bf=\bf \cap
      \af$.

    \item[(3)] $\af(\bf+\cf)=\af\bf+\af\cf$.
  \end{enumerate}
\end{theorem}
\begin{proof}
  Let $x \in \af$, $y \in \bf$, and $z \in \cf$, by associativity and
  commutativity we have $x+(y+z)=(x+y)+z$, $x(yz)=(xy)z$; and where the
  associativity of intersection follows from set theory. By similar reasoning ,
  we also get  $\af+\bf=\bf+\af$,  $\af\bf=\bf\af$, and  $\af \cap \bf=\bf \cap
  \af$.

  Now, let  $h \in \af(\bf+\cf)$, then
  $h=\sum{x_i(y_i+z_i)}=\sum(x_iy_i+x_iz_i)=\sum{x_iy_i}+\sum{x_iz_i} \in
  \af\bf+\af\cf$. On the other hand, suppose $h \not\in \af(\bf+\cf)$, then by
  similar reasoning, we get $h \not\in \af\bf+\af\cf$.
\end{proof}

\begin{example}\label{example_5.18}
  Intersection and addition of ideals distribute over each other in $\Z$, but
  this is not true in general.
\end{example}

\begin{theorem}[Modularity]\label{theorem_5.5.7}
  If $\af$ and $\bf$, and $\cf$ are ideals in a commutative ring with
  identity, and either $\bf \subseteq \af$, or
  $\cf \subseteq \af$, then:
  \begin{equation*}
    \af \cap (\bf+\cf)=(\af \cap \bf)+(\af \cap \cf)
  \end{equation*}
\end{theorem}
\begin{proof}
  Suppose that $\bf \subseteq \af$, and take $h \in \af \cap (\bf+\cf)$. So $h
  \in \af$, and $h=y+z$ with  $y \in \bf$ and $z \in \cf$. Then $y \in \af \cap
  \bf \subseteq \af$, and $y+z \in \af$, so by group closure, $z \in \af$; i.e.
  $z \in \af \cap \cf$. Therefore, $h \in (\af \cap \bf)+(\af \cap \cf)$.
 in a commutative ring with identity
  COnverselt, let $h=y+z$ with  $y \in \af \cap \bf$ and $z \in \af \cap \cf$.
  Then $y \in \af$ and $y \in \bf$, and $z \in \af$ and $z \in \cf$. This makes
  $h \in \af$ and $h \in \bf+\cf$, so $h \in \af \cap (\bf+\cf)$.
\end{proof}

\begin{theorem}\label{theorem_5.5.8}
  For any ideals $\af$ and $\bf$:
  \begin{equation}
    (\af+\bf)(\af \cap \bf) \subseteq \af\bf
  \end{equation}
\end{theorem}
\begin{proof}
  Let $h \in (\af+\bf)(\af \cap \cf)$. Then $h=\sum{(x_i+y_i)z_i}=
  \sum{(x_iz_i+y_iz_i)} \subseteq \af\bf$.
\end{proof}
\begin{corollary}
  If $\af+\bf=(1)$, then $\af\bf=\af \cap \bf$.
\end{corollary}

\begin{definition}
  We call two ideals $\af$ and $\bf$ in a commutative ring with
  identity \textbf{coprime} if $\af+\bf=(1)$.
\end{definition}

\begin{definition}
  Let $R_1, \dots, R_n$ be commutative rings. The \textbf{direct product} of
  $R_1, \dots, R_n$ is defined to be the set $\prod_{i=1}^n{R_i}$ of all tuples
  $x=(x_1, \dots, x_n)$, with $x_i \in R_i$. We define addition and
  multiplication of elements of $\prod{R_i}$ component-wise.
\end{definition}

\begin{proposition}\label{proposition_5.5.9}
  The direct product of commutative rings is a commutative ring. Moreover, if
  each ring has identity, so does their direct product.
\end{proposition}

\begin{theorem}\label{theorem_5.5.10}
  Let $R$ be a commutative ring with ideals $\af_1, \dots, \af_n$. Define the
  map $\pi:R \xrightarrow{} \prod_{i=1}^n{R}$ by $\pi:x \xrightarrow{}
  (x_1+\af_1, \dots, x_n+\af_n)$. Then $\pi$ is a homomorphism, and we have
  \begin{enumerate}
    \item[(1)] If $\af_i$ and  $\af_j$ are coprime whenever $i \neq j$, then
      \begin{equation*}
        \prod_{i=1}^n{\af_i}=\bigcap_{i=1}^n{\af_i}
      \end{equation*}

    \item[(2)] $\pi$ is onto if, and only if $\af_i$ and $\af_j$ are coprime
      whenever  $i \neq j$.

    \item[(3)] $\pi$ is 1--1 if, and only if  $\bigcap{\af_i}=(0)$.
  \end{enumerate}
\end{theorem}
\begin{proof}
  By induction on $n$ it was shown that for $n=2$ that if $\af_1,\af_2$ are
  coprime, then $\af_1\af_2=\af_1 \cap \af_2$. Now, suppose that
  \begin{equation*}
    \bf=\prod_{i=1}^n{\af_i}=\bigcap_{i=1}^n{\af_i}
  \end{equation*}
  for all $n \geq 2$ and consider the case for $n+1$. Since
  $\af_i+\af_{n+1}=(1)$ (they are coprime by hypothesis), we have $x_i+y_i=1$
  where  $x_i \in \af_i$ and  $y_i \in \af_{n+1}$. Hence notice that
  \begin{equation*}
    \prod_{i=1}^n{x_i}=\prod_{i=1}^n{1-y_i} \equiv 1 \mod{\af_{n+1}}
  \end{equation*}
  so that $\bf+\af_{n+1}=(1)$. Hence $\bf\af_{n+1}=\bf \cap \af_{n+1}$ which
  completes the proof.

  Suppose now, that $\phi$ is onto. Then there exists an  $x \in R$ such that
  $\phi(x)=(1,0, \dots, 0)$ so that $x \equiv 1 \mod{\af_1}$ and $x \equiv 0
  \mod{a_i}$ for $i>1$. Hence  $1=(1-x)+x \in \af_1+\af_i$ for all $i>1$.
  This makes  $\af_1$ and $\af_i$ coprime. We can repeat this argument for
  any inex  $j \neq i$. Conversely suppose that $\af_1$ and $\af_i$ are
  coprime. Then $\af_1+\af_i=(1)$ for all $i>1$ and we have  $u_i+v_i=1$ for
  some  $u_i \in \af_1$ and $v_i \in \af_i$. Take then  $x=\prod{v_i}$. Then
  \begin{equation*}
    x=\prod{1-u_i} \equiv 1 \mod{\af_1} \text{ and } x \equiv 0 \mod{\af_i}
  \end{equation*}
  thus $\phi(x)=(1,0, \dots, 0)$. repeating for each index $j \neq i$, we get
  that  $\phi$ is onto. Finally, observe that
  \begin{equation*}
    \ker{\phi}=\{x \in R : (x+\af_1, \dots,x+\af_n)=(\af_1, \dots, \af_n)\}=
    \bigcap_{i=1}^n{\af_n}
  \end{equation*}
  Which gives us the equivalent condition for $\phi$ to be 1--1.
\end{proof}

\begin{example}\label{example_5.19}
  In general, the union of ideals is not an ideal. Take $2,3 \in \Z$, then
  $2+3=5 \not\in 2\Z \cup 3\Z$.
\end{example}


\begin{theorem}\label{theorem_5.5.11}
  The following are true:
  \begin{enumerate}
    \item[(1)] If $\pf_1, \dots, \pf_n$ are prime ideals, and $\af$ is an ideal
      contained in  $\bigcup{\pf_i}$, then $\af \subseteq \pf_i$ for some $i$.

    \item[(2)] If $\af_1, \dots, \af_n$ are ideals, and $\pf$ is a prime ideal
      containing  $\bigcap{\af_i}$, then $\af_i \subseteq \pf$ for some $i$.
      Moreover, if  $\pf=\bigcap{\af_i}$, then $\pf=\af_i$.
  \end{enumerate}
\end{theorem}
\begin{proof}
  For the first assertion, the result is vacaciously true for $n=1$. Now
  suppose the result is true for all  $n \geq 1$. Then for every  $1 \leq i
  \leq n$, there is an  $x_i \in \af$ such that  $x_i \in \pf_j$ whenever  $i
  \neq j$. Now, if  $x_i \notin \pf_i$, we are done. Otherwise, $x_i \in \pf$,
  and consider
  \begin{equation*}
    y=\sum_{i=1}^n{x_1x_2 \dots x_{i-1}x_{i+1} \dots x_n}
  \end{equation*}
  Then $y \in \af$ but $y \notin \pf_i$, which puts  $\af \not\subseteq
  \pf_i$, for all $1 \leq i \leq n$, hence $\af \subseteq \pf_{n+1}$ and we
  are done.

  For the second assertion, suppose that $\af_i \not\subseteq \pf$ for all
  $1 \leq i \leq n$. Then let  $x_i \in \af$ such that  $x_i \notin\pf$. Then
  we have
  \begin{equation*}
    \prod{\x_i} \in \prod{\af_i}
  \end{equation*}
  but $\prod{x_i} \notin \pf$, hence $\bigacap{\af_i} \not\subseteq \pf$.
\end{proof}
\begin{corollary}
  If $\pf=\bigcap{\af_i}$, then $\pf=\af_i$ for some  $1 \leq i \leq n$.
\end{corollary}

\begin{definition}
  We define the \textbf{ideal quotient} of two ideals $\af$ and $\bf$ to be
  \begin{equation*}
    (\af:\bf)=\{x \in R : x\bf \subseteq \af\}
  \end{equation*}
  We call the ideal quotient $(0:\bf)$ the \textbf{annihilator} of $\bf$.
\end{definition}

\begin{example}\label{example_5.20}
  Take $m,n \in \Z^+$. By unique factorization $m=\prod{p^{u_i}}$, and
  $n=\prod{p^{v_j}}$. Take $w_k=\max{\{u_i-v_j,0\}}=u_i-\min{\{u_i,v_j\}}$, and
  take $q=\prod{p^{w_k}}$. Then $(m\Z:n\Z)=q\Z$, so that $q=\frac{mn}{(m,n)}$.
\end{example}

\begin{proposition}\label{proposition_5.5.12}
  The following are true for any ideals $\af$,  $\bf$ and  $\cf$.
  \begin{enumerate}
    \item[(1)] $\af \subseteq (\af:\bf)$.

    \item[(2)] $(\af:\bf)\bf \subseteq \af$

    \item[(3)] $((\af:\bf):\cf)=(\af:\bf\cf)=((\af:\cf):\bf)$.

    \item[(4)] $(\bigcap{\af_\a}:\bf)=\bigcap{(\af_\a:\bf)}$ for any collection
      $\{\af_\a\}$ of ideals.

    \item[(5)] $(\af:\sum{\bf_\a})=\bigcap{(\af:\bf_\a)}$ for any collection
      $\{\bf_\a\}$ of ideals.
  \end{enumerate}
\end{proposition}

\begin{definition}
  If $\af$ is an ideal, we define its \textbf{radical} to be
  \begin{equation*}
    \rad{\af}=\{x \in R : x^n \in \af \text{ for some } n \in \Z^+\}
  \end{equation*}
\end{definition}

\begin{proposition}\label{proposition_5.5.13}
  If $\af$ is an ideal, then so is $\rad{\af}$.
\end{proposition}
\begin{proof}
  Take the natural map $\pi:R \xrightarrow{} \faktor{R}{\af}$. Then
  $\rad{\af}=\inv{\pi}(\Nil{\faktor{R}{\af}})$.
\end{proof}

\begin{proposition}\label{proposition_5.5.14}
  The following are true for any ideals $\af$ and  $\bf$.
  \begin{enumerate}
    \item[(1)] $\af \subseteq \rad{\af}$.

    \item[(2)] $\rad{\rad{\af}}=\rad{\af}$.

    \item[(3)] $\rad{\af\bf}=\rad{\af \cap \bf}=\rad{\af} \cap \rad{\bf.}$

    \item[(4)] $\rad{\af}=(1)$ if, and only if $\af=(1)$.

    \item[(5)] $\rad{\af+\bf}=\rad{\rad{\af}+\rad{\bf}}$.

    \item[(6)] If $\pf$ is a prime ideal, then $\rad{\pf^n}=\pf$ for any $n \in
      \Z^+$.
  \end{enumerate}
\end{proposition}

\begin{proposition}\label{proposition_5.5.15}
  If $\af$ is an ideal, then $\rad{\af}$ is the intersection of all prime
  ideals containing $\af$.
\end{proposition}
\begin{proof}
  $\rad{\af}=\inv{\pi}(\Nil{\faktor{R}{\af}})=\inv{\pi}(\bigcap{\pf_\a})=
  \bigcap{\inv{\pi}\pf_\a}$, and  $\inv{\pi}(\pf_\a)$ is a prime ideal
  containing $\af$.
\end{proof}

\begin{proposition}\label{proposition_5.5.16}
  Let $D$ be the set of all zero-divisors in a commutative ring. Then:
  \begin{equation}
    D=\bigcup_{\af \neq (0)}{(0:\af)}
  \end{equation}
\end{proposition}
