\section{Ring Homomorphisms and Factor Rings}
\label{section_5.3}

\begin{definition}
  Let $R$ and  $S$ be rings. We call a map  $\phi:R \xrightarrow{} S$ a
  \textbf{ring homomorphism} if
  \begin{enumerate}
    \item[(1)] $\phi$ is a group homomorphism with respect to addition.

    \item[(2)] $\phi(ab)=\phi(a)\phi(b)$ for any $a,b \in R$.
  \end{enumerate}
  We denote the \textbf{kernel} of $\phi$ to be the kernel of $\phi$ as a
  group homomorphism. That is
  \begin{equation*}
    \ker{\phi}=\{r \in R : \phi(r)=0\}
  \end{equation*}
  Moreover, if $\phi$ is 1--1 and onto, we call $\phi$ a \textbf{ring
  isomorphism} and say that $R$ and $S$ are \textbf{isomorphic},
  and write $R \simeq S$.
\end{definition}

\begin{example}\label{example_5.9}
  \begin{enumerate}
    \item[(1)] $\phi:\Z \xrightarrow{} \faktor{\Z}{2\Z}$ defined by $n
      \xrightarrow{} 0$ if $n$ is even and  $n \xrightarrow{} 1$ if $n$ is
      odd is a ring homomorphism, with  $\ker{\phi}=2\Z$. Notice that
      $\phi(\Z)=\faktor{\Z}{2\Z}$. $\phi$ is onto, but not 1--1.

    \item[(2)] Let $n \in \Z$ and consider the maps  $\phi_n:\Z
      \xrightarrow{} \Z$ by taking $x \xrightarrow{} nx$. $\phi_n$, in
      general is not a ring homomorphism, as  $\phi(xy)=n(xy)$ but
      $\phi(x)\phi(y)=nxny=n^2(xy)$. $\phi_n$, however is a group
      homomorphism for any $n$.

    \item[(3)] For any ring $R$, define the \textbf{evaluation} map
      $\ev_0:R[x] \xrightarrow{} R$ by taking $f(x) \xrightarrow{} f(0)$;
      i.e. the polynomial $f$ evaluated at $0$.  $\ev$ is a ring
      homomorevsm. Moreover, notice that if
      $f(x)=a_0+a_1x+\dots+a_nx^n$, then $f(0)=a_0 \in \R$. So that
      $\ev_0(R[x])=R$ This makes $\ev_0$ onto. Now, take  $\ev(f)=0$, Then
      those are all polynomials with constant term $a_0=0$ (this does not
      make $\ker{(\ev_0)}=\langle e \rangle$). Again, $\ev_0$ is onto, but it
      is not 1--1.
  \end{enumerate}
\end{example}

\begin{proposition}\label{proposition_5.3.1}
  Let $R$ and $S$ be rings, and $\phi:R \xrightarrow{} S$ a ring homomorphism.
  Then
  \begin{enumerate}
    \item[(1)] $\phi(R)$ is a subring of $S$.

    \item[(2)] $\ker{\phi}$ is a subring of $R$.
  \end{enumerate}
\end{proposition}
\begin{proof}
  Let $s_1,s_2 \in \phi(R)$. Then $s_1=\phi(r_1)$ and $s_2=\phi(r_2)$ for some
  $r_1,r_2 \in R$. Then $s_1s_2=\phi(r_1)\phi(r_2)=\phi(r_1r_2) \in \phi(S)$.
  Additionally, $\inv{s}=\inv{\phi}(r)=\phi(\inv{r})$ for some $s \in S$,  $r
  \in R$. This is sufficient to make  $S$ a subring of  $S$.

  By similar reasoning, if $r_1,r_2 \in \ker{\phi}$, then
  $\phi(r_1)\phi(r_2)=\phi(r_1r_2)=0$ so that $r_1r_2 \in \ker{\phi}$, and
  $\phi(\inv{r})=\inv{\phi}(r)=0$ so $\inv{\phi}\in \ker{\phi}$.
\end{proof}
\begin{corollary}
  For any $r \in R$ and  $a \in \ker{\phi}$, then $ar \in \ker{\phi}$ and $ra
  \in \ker{\phi}$.
\end{corollary}
\begin{proof}
  We have $\phi(ar)=\phi(a)\phi(r)=\phi(a)0=0$ so $ar \in \ker{\phi}$. The
  same happens for $ra$.
\end{proof}

\begin{definition}
  Let $R$ be a ring. We call a subgroup $\af \leq R$ of $R$ a \textbf{left
  ideal} in $R$ if for any $r \in R$ and $a \in \af$, we have  $ar
  \in \af$. Similarly, we call  $\af$ a  \textbf{right ideal} in $R$ if
  $ra \in \af$. We call  $\af$ a  (\textbf{two-sided}) \textbf{ideal} in
  $R$ if it is both a left, and a right ideal and we say that the ideals
  $\af$ \textbf{absorb} $r$.
\end{definition}

\begin{proposition}\label{proposition_5.3.2}
  If $R$ is a commutative ring, then every left ideal is a right ideal.
\end{proposition}
\begin{proof}
  Notice that $ar=ra$ for all  $a,r \in R$.
\end{proof}

\begin{theorem}\label{theorem_5.3.3}
  Let $R$ be a ring, and $\af$ an ideal in $R$. Let  $\faktor{R}{\af}$ be the set
  of all $x+\af$ with  $x \in R$. Define operations $+$ and $\cdot$ by
  \begin{align*}
    (x+\af)+(y+\af) &=  (x+y)+\af \\
    (x+\af)(y+\af)  &=  xy+\af    \\
  \end{align*}
  Then $\faktor{R}{\af}$ forms a ring under $+$ and $\cdot$.
\end{theorem}
\begin{proof}
  Observe first that $+$ and $\cdot$ are well-defined. Indeed, let
  $x,y \in R$ and suppose $x+\af=x'+\af$ and $y+\af=y'+\af$. Then
  $(x-x')+\af=\af$ and $(y-y')+\af=\af$ so $x-x',y-y' \in \af$. Then
  there are $z,z' \in \af$ for which $x-x'=z$ and $y-y'=z'$, so that
  $(x-x')+(y-y')=(x+y)-(x'+y')=z+z'$. Then $(x+y)-(x'+y') \in \af$, so
  that $(x+y)-(x'+y')+\af=\af$. That is $(x+y)+\af=(x'+y')+\af$.

  Similarly, $xy=(x'+z)(y'+z')=(x'y')+(x'z'+zy'+zz')$, so
  $xy-x'y'=x'z'+zy'+zz'$, i.e. $xy-x'y' \in \af$, so that
  $xy+\af=x'y'+\af$.

  Now, observe that under $+$, the ideal $\af$ is a normal subgroup of
  $R$, as an Abelian group. So, we have that $\faktor{R}{\af}$ is the
  quotient group of $R$ under $+$.

  Observe now:
  \begin{align*}
    (x+\af)(y+\af)  &=  xy+a\af+b\af+\af \\
                    &=  xy+(\af+\af+\af)  \\
                    &=  xy+\af  \\
  \end{align*}
  By distribution of multiplication over addition in $R$. Moreover,
  $\faktor{R}{\af}$ also inherits associativity in $\cdot$ of ultiplication in
  $R$. Finally:
  \begin{align*}
    (x+\af)((y+\af)+(z+\af))  &=  (x+\af)((y+z)+\af)  \\
        &= x(y+z)+\af \\
        &=  (xy+xz)+\af
        &=  (xy+\af)+(xz+\af) \\
  \end{align*}
  and
  \begin{align*}
    ((x+\af)+(y+\af))(z+\af)  &=  ((x+y)+\af)(z+\af)  \\
      &=  (x+y)z+\af  \\
      &=  (xz+yz)+\af \\
      &=  (xz+\af)+(yz+\af) \\
  \end{align*}
  This makes $\faktor{R}{\af}$ a ring as required.
\end{proof}
\begin{corollary}
  If $R$ has identity $1$, then $\faktor{R}{\af}$ has identity
  $1+\af$. Moreover if  $R$ is commutative, then so is $\faktor{R}{\af}$.
\end{corollary}

\begin{definition}
  Let $R$ be a ring and $\af$ an ideal in $R$. We call the ring
  $\faktor{R}{\af}$
  under addition and muiltplication of cosets the \textbf{factor ring}
  of $R$ over  $\af$.
\end{definition}

\begin{example}\label{example_5.10}
  \begin{enumerate}
    \item[(1)] We call $(0)=\langle 0 \rangle=\{0\}$ the \textbf{trivial ideal},
      notice also that $R$ is also an ideal. If $R$ has identity $1$,
      we write $R=(1)$ when considering it as an ideal.

    \item[(2)] For any $n \in \Z$, notice that if $a \in \Z$ and $m \in
      n\Z$, then  $m=nk$, for some  $k \in \Z$ so that  $am=n(ak)=ma \in
      n\Z$. So $n\Z$ is an ideal of $\Z$, with factor ring
      $\faktor{\Z}{n\Z}$. So $\faktor{\Z}{n\Z}$ is a factor ring on top of
      also being a factor group. We call the ring homomorphisme $\phi:\Z
      \xrightarrow{} \faktor{\Z}{n\Z}$ by $a \xrightarrow{} a \mod{n}$ the
      \textbf{reduction homomorphism}.

    \item[(2)] Let $R$ a ring, and consider  $R[x]$. Let $\af$ the set of all
      polynomials of degree greater than $2$ together with $0$. Then if
      $f \in \af$,  $\deg{f}>2$ or $f=0$. Then for any  $g \in R[x]$,
      $\deg{fg}>2$ or, $fg=0$ and $\deg{gf}>2$ or $gf=0$. This
      makes  $\af$ an ideal of $R[x]$. Moreover, $p,q \in \af$ if and
      only if they have the same constant term. Notice then that
      $\faktor{t\R[x]}{\af}=\{a+bx : a,b \in R\}$.

      Now, if $R$ has no zero divisors, it is possible that
      $\faktor{R[x]}{\af}$ has zero divisors. Consider
      $\faktor{\Z[x]}{\af}$.

    \item[(3)] Let $A$ a ring, and  $X$ a non-empty set. For the ring of
      functions $A^X$, given $c \in X$, define the
      \textbf{evaluation} map at $c$ by $\ev_c:f(x) \xrightarrow{} f(c)$.
      Notice that $\ev_c$ is a ring homomorphism, so that
      $\faktor{A^X}{\ker{(\ev_c)}}$ forms a factor ring. In particular, if
      $A^X=A[x]$ the polynomial ring over $A$, and  $c=0$, then $\ev_c$ is
      just the evaluation map of polynomials.

      Now, if  $X=(0,1]$, and $A=\R^{(0,1]}$, by the first isomorphism
      theorem for groups, we have $\R \simeq \faktor{\R^{(0,1]}}{\ker{(\ev_c)}}$,
      since $\ev_c(\R^{(0,1]})=\R$.

    \item[(4)] Let $n \geq 2$ and consider  $R^{n \times n}$. Let $\bf$ an ideal
      of  $R$. Then  $\bf^{n \times n}=\{(a_{ij}) : a_{ij} \in \bf\}$ is an ideal
      of $R^{n \times n}$. Take the ring homomorphism
      \begin{align*}
        R^{n \times n}  & \xrightarrow{} (\faktor{R}{\bf})^{n \times n}   \\
        (a_{ij})    & \xrightarrow{} (a_{ij}+\bf) \\
      \end{align*}
      Then $\bf^{n \times n}$ is the kernel of this homomorphism, so that
      \begin{equation*}
        \faktor{R^{n \times n}}{\bf^{n \times n}} \simeq (\faktor{R}{\bf})^{n \times n}
      \end{equation*}
      For example, with $n=3$, we have
      \begin{equation*}
        \faktor{\Z^{3 \times 3}}{2\Z^{3 \times 3}} \simeq
        (\faktor{Z}{2\Z})^{3 \times 3}
      \end{equation*}

    \item[(5)] Let $R$ a commutative ring with identity, and  $G$ a finite group
      of order $n$. Define the \textbf{augmentation} map to be the map
      \begin{align*}
        RG      & \xrightarrow{}    R   \\
        \sum_{i=1}^n{a_ig_i}    & \xrightarrow{} \sum_{i=1}^n{a_i}  \\
      \end{align*}
      We call the kernel of this map the \textbf{augmentation ideal} which is
      the set of all formal sums whose coefficients sum to $0$. Another ideal
      of  $RG$ is the set
      \begin{equation*}
        \af= \Big{\{}\sum{ag_i} : g_i \in G \Big{\}}
      \end{equation*}
      of all formal sums whose coefficients are all equal.
  \end{enumerate}
\end{example}

\begin{theorem}[The First Isomorphism Theorem]\label{theorem_5.3.4}
  If $\phi:R \xrightarrow{} S$ is a ring homomorphism from rings $R$ into $S$,
  then $\ker{\phi}$ is an ideal of $R$ and
  \begin{equation*}
    \phi(R) \simeq \faktor{R}{\ker{\phi}}
  \end{equation*}
  \[\begin{tikzcd}
    R &&& S \\
    \\
    \\
    {\faktor{R}{\ker{\phi}}}
    \arrow["\pi"', from=1-1, to=4-1]
    \arrow["\phi", from=1-1, to=1-4]
    \arrow["{\Phi}"', from=4-1, to=1-4]
  \end{tikzcd}\]
\end{theorem}
\begin{proof}
  By the first isomorphism theorem for groups, $\phi$ is a group isomorphism.
  Now, let $K=\ker{\phi}$ and consider the map $\pi:R \xrightarrow{}
  \faktor{R}{I}$ by $a \xrightarrow{\pi} a+K$. Define the map
  $\Phi:\faktor{R}{K} \xrightarrow{} \phi(R)$ such that $\Phi
  \circ \pi=\phi$, then $\Phi$ defines the ring isomorphism.
\end{proof}
\begin{proof}
  The map $\pi:R \xrightarrow{} \faktor{R}{I}$ defined by $a \xrightarrow{}
  a+I$, for any ideal $I$, is onto, with  $\ker{\pi}=I$.
\end{proof}

\begin{theorem}[The Second Isomorphism Theorem]\label{theorem_5.3.5}
  Let $A \subseteq R$ a subring of  $R$, and let $\bf$ an ideal in $R$. Define
  $A+\bf=\{a+b : a \in A \text{ and } b \in \bf\}$. Then $A+\bf \susbeteq R$ is a
  subring and  $A \cap \bf$ is an ideal in $A$. Then
  \begin{equation*}
    \faktor{(A+\bf)}{\bf} \simeq \faktor{A}{(A \cap \bf)}
  \end{equation*}
\end{theorem}

\begin{theorem}[The Third Isomorphism Theorem]\label{theorem_5.3.6}
  Let $\af$ and  $\bf$ be ideals in a ring  $R$, with  $\af \subseteq \bf$.
  Then
  $\faktor{\bf}{\af}$ is an ideal of $\faktor{R}{\af}$ and
  \begin{equation*}
    \faktor{R}{\bf}=\faktor{(\faktor{R}{\af})}{(\faktor{\bf}{\af})}
  \end{equation*}
\end{theorem}

\begin{theorem}[The Fourth Isomorphism Theorem]\label{theorem_5.3.7}
  Let $\af$ an ideal of $R$. There is an inclusion preserving 1--1
  correspondence between the subrings $A \subseteq R$ containing
  $\af$, and subrings of $\faktor{R}{\af}$. Moreover, any subring $A
  \subseteq R$ is an ideal if, and only if $\faktor{A}{\af}$ is an
  ideal.
\end{theorem}

\begin{example}\label{example_5.11}
  We have $12\Z$ is an ideal of  $\Z$, and that  $\faktor{\Z}{12\Z}$ has as
  ideals
  \begin{align*}
    \faktor{\Z}{12\Z} && \faktor{2\Z}{12\Z} && \faktor{3\Z}{12\Z} &&
    \faktor{4\Z}{12\Z} && \faktor{6\Z}{12\Z} && \faktor{12\Z}{12\Z}
  \end{align*}
\end{example}

\begin{remark}
  From here on, we abbreviate the elements of the factor ring
  $\faktor{R}{\af}$ by $x \mod{\af}$ instead of $x+\af$. We read $x
  \mod{\af}$ ``$x$ modulo $\af$''. This notational convention is
  justified by recognizing the factor ring $\faktor{R}{\af}$ as a
  factor group with additional structure.
\end{remark}
