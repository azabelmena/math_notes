\section{Noetherian Rings}
\label{section_5.9}

In this section, we assume that all rings are commutative rings with
identity $1 \neq 0$, unless otherwise specified.

\begin{definition}
  We say a sequence of ideals $\{\af_n\}$ in a ring is an
  \textbf{ascending chain} of ideals if $\af_n \subseteq \af_{n+1}$
  for all $n \in \Z^+$. We say that the the the chain $\{\af_n\}$
  \textbf{stabalizes} if there exists some $k \geq n$ for which
  $\af_k=\af_n$. We say a general ring $R$ (not necessarily commutative
  nor with identity) satisfies the \textbf{ascending chain condition}
  on ideals if every ascending chain of ideals of $R$ stabalizes.
\end{definition}

\begin{definition}
  We call a commutative ring with identity \textbf{Noetherian} if it
  satisfies the ascending chain condition on ideals.
\end{definition}

\begin{lemma}\label{lemma_5.9.1}
  If $\af$ is an ideal of a Noetherian ring  $R$, then the factor ring
  $\faktor{R}{\af}$ is also Noethrian. In particular, the image of a Noetherian
  ring under any ring homomorphism is Noetherian.
\end{lemma}
\begin{proof}
    By the fourth isomorphism theorem, there is a 1--1 inclusion
    preserving correspondence between the ideals of $R$ and the ideals
    of $\faktor{R}{\af}$.
\end{proof}

\begin{theorem}\label{lemma_5.9.2}
  The following are equivalent for any ring $R$.
  \begin{enumerate}
    \item[(1)] $R$ is Noetherian.

    \item[(2)] Every nonempty collection of ideals of $R$ contains a maximal
      element under inclusion.

    \item[(3)] Every ideal of $R$ is finitely generated.
  \end{enumerate}
\end{theorem}
\begin{proof}
  Let $R$ be Noetherian, and let  $\Ac$ an nonempty collection of ideals of
  $R$. Choose an ideal  $\af_1 \in \Ac$. If $\af_1$ is maximal, we are done. If
  not, then there is an ideal $\af_2 \in \Ac$ for which  $\af_1 \subseteq \af_2$.
  Now, if $\af_2$ is maximal, we are done. Otherwise, proceeding inductively, if
  there are no maximal ideals of $R$ in $\Ac$, then by the axiom of choice,
  construct the infinite strictly increasing chain
  \begin{equation*}
    \af_1 \subseteq \af_2 \subseteq \dots
  \end{equation*}
  of ideal of $R$. This contradicts that  $R$ is Noetherian, so  $\Ac$ must
  contain a maximal element.

  Now, suppose that any nonempty collection of ideals of  $R$ contains
  a maximal element. Let  $\Ac$ the collection of all finitely generated
  ideals of $R$, and let $\af$ be any ideal of  $R$. By hypothesis, $\Ac$ has a
  maximal element  $\af'$. Now suppose that $\af \neq \af'$, and choose an
  $x \in \com{\af}{\af'}$, then the ideal generated by $\af'$ and  $x$ is finitely
  generated, and so is in  $\Ac$; but that contradicts the maximality of  $\af'$.
  Therefore we must have  $\af=\af'$.

  Finally, suppose every ideal of $R$ is finitely genrated, and let
  $\af=(a_1, \dots, a_n)$. Let
  \begin{equation*}
    \af_1 \subseteq \af_2 \subseteq \dots
  \end{equation*}
  an ascending chain of ideals of $R$ for which
  \begin{equation*}
    \af=\bigcup_{n \in \Z^+}{\af_n}
  \end{equation*}
  Since $a_i \in \af$ for each  $1 \leq j \leq n$, we have that  $a_i \in
  \af_{i_j}$ and $i \in \Z^+$. Now, let  $m=\max{\{j_1, \dots, j_n\}}$ and
  coinsider the ideal $\af_m$. Then  $a_i \in \af_m$ for each $i$, which makes
  $\af \subseteq \af_m$. That is, $\af_n=\af_m$ for some  $n \geq m$; which
  makes $R$ Noetherian.
\end{proof}

\begin{theorem}[Hilbert's Basis Theorem]\label{theorem_5.9.3}
  If $R$ is a Noetherian ring, then so is the polynomial ring $R[x]$.
\end{theorem}
\begin{proof}
  Let $\af$ be an ideal of $R[x]$, and let $\lf$ be the set of all
  leading coefficients of elements of $\af$; that is:
  \begin{equation*}
    \lf=\{ a_n : f(x)=a_nx^n+h(x), \text{ for all }
    f \in \af, \text{ and } h(x) \in R[x]  \}
  \end{equation*}
  Then $\lf$ is an ideal of $R$. Indeed, observe that $0=0x^n+\dots
  \in \af$ so that $0 \in \lf$. Now let $f,g \in \af$ have the
  form $f(x)=ax^d+h(x)$ and $g(x)=bx^e+h'(x)$. Then $x^ef(x)-x^dg(x)
  \in \af$. Now
  \begin{equation*}
    x^ef(x)-x^dg(x)=(a-b)x^{d+e}+(x^eh(x)-x^dh'(x))
  \end{equation*}
  so that $a-b \in \lf$. Likewise, let $r \in R$, then $rf \in \af$,
  and
  \begin{equation*}
    rf(x)=(ra)x^d+(rh(x))
  \end{equation*}
  so that $ra \in \lf$. Since $R$ is Noetherian, and $\lf$ is an ideal
  of $R$,  $\lf=(a_1, \dots, a_n)$ for some $a_1, \dots, a_n \in R$.

  Now, let $f_1, \dots, f_n \in R[x]$ be polynomials with leading
  coefficients $a_1, \dots, a_n$ (associated above by $\lf$). Take
  $\deg{f_i}=e_i$ for all $1 \leq i \leq n$, and let $N=\max{\{e_1,
  \dots, e_n\}}$. Now, for $d \in \faktor{\Z}{N\Z}$, let $\lf_d$ be
  the set of all leading coefficients of polynomials in $\af$ with
  degree equal to $d$. By the same reasoning we used for $\lf$,
  $\lf_d$ is also an ideal of $R$, and hence $\lf_d=(b_{d,1}, \dots,
  b_{d,n})$ for some $b_{d,1}, \dots b_{d,n} \in R$. Now, let
  $f_d,i(x) \in \af$ have the form $f_{d,i}(x)=b_{d,i}x^d+h(x)$.

  Let $\af'=(f_1, \dots, f_n) \cup \{f_{d,i} : 1 \leq i \leq n_d
  \text{ and } d \in \faktor{\Z}{d\Z}\}$, where
  $f_i(x)=a_ix^{e_i}+\dots$ (recall that $\lf=(a_1, \dots, a_n)$). We
  show that $\af=\af'$. Indeed, since $f_i,f_{d,i} \in \af$, we get
  that $\af' \subseteq \af$. Suppose now that $\af \neq \af$, and
  choose some $f \in \com{\af}{\af'}$ of minimum degree $d'$, and let
  $f$ have the form $f(x)=ax^d+h(x)$.

  Suppose first that $d' \geq N$, and let $a \in \lf$, so that
  $a=r_1a_1+\dots+r_na_n$ for some $r_1, \dots r_n \in R$. Define
  \begin{equation*}
    g(x)=r_1x^{d'-e_1}f_1(x)+\dots+r_nx^{d'-e_n}f_n(x)
  \end{equation*}
  Then $g \in \af'$, and $\deg{g}=d'$. Moreover, $g$ has leading
  coefficient $a$. Then $f-g \in \af$, and $\deg{f-g}<d'$. Since $f'$ is
  of minimum degree in  $\af$, $f-g=0$ so that  $f=g$ and  $f \in
  \af'$. A contradiction!

  Likewise, suppose that $d'<N$, and  $a \in \lf_d$. Then
  $a=r_1b_{d,1}+\dots+r_nb_{d,n}$. Take
  \begin{equation*}
    g(x)=r_1f_{d,1}(x)+\dots+r_nf_{d,n}(x)
  \end{equation*}
  Then $g \in \af'$ and $\deg{g}=d'$, and $g$ has leading coefficient
  $a$. Then $f-g \in \af$, so that $f=g$ in $\af$ by the above
  reasoning which puts  $f \in \af'$ and gives us the same
  contradiction. In either case, we must have that $\af=\af'$.
  Therefore $\af$ is finitely generated in $R[x]$ wich makes $R[x]$
  Noetherian.
\end{proof}
\begin{corollary}
  Any field $k$ is Noetherian. In particular, the rings $k[x]$ and
  $k[x_1, \dots, x_n]$ are Noetherian.
\end{corollary}
\begin{proof}
  If $k$ is a field, then its only ideals are $(0)$ and $(1)$, which
  makes it Noetherian; hence $k[x]$ is Noetherian by Hilbert's basis
  theorem. Lastly, observe that $k[x_1, \dots, x_n]=(k[x_1, \dots,
  x_{n-1}])[x_n]$ and apply the above argument recursively.
\end{proof}
\begin{corollary}
  Let $k$ be a field. If $\af$ is an ideal in $k[x_1, \dots, x_n]$ generated
  by a set $S$, then $\af$ is generated by finitely many polynomials in $S$.
\end{corollary}
\begin{proof}
  Since $k$ is a field, $k[x_1, \dots, x_n]$ is Noetherian, and
  $\af=(f_1, \dots, f_n)$ where $f_i(x_1, \dots ,x_n) \in k[x_1,
  \dots, x_n]$. Likewise, since $\af=(S)$ is finitely generated, $S$
  is finite. Then since $(f_1, \dots, f_n)=(S)$, $f_i \in S$ for all
  $1 \leq i \leq n$ so that
  \begin{equation*}
    f_i(x)=r_1g_{i,1}(n_1, \dots, x_n)+\dots+r_ng_{i,m}(n_1, \dots, x_n)
  \end{equation*}
  and so $\af=(f_1, \dots, f_n)=(g_{1,1}, \dots ,g_{1,m}, \dots,
  g_{n,1}, \dots g_{n,m})$.
\end{proof}

\begin{example}\label{example_5.24}
  The multivariate polynomial ring $\Z[x_1, x_2, \dots]$ is not
  Noetherian, since the ideal $(x_1, x_2, \dots)$ is not finitely
  generated.
\end{example}

\begin{definition}
  Let $k$ be a field. We call a ring  $R$ a \textbf{$k$-algebra} if $k$ is
  contained in the center of  $R$  (i.e. $k \subseteq Z(R)$), and $1_k=1_R$.
  We call  $R$ a  \textbf{finitely generated} $k$-algbera if $R$ is generated
  by  $k$ together with a finite set $\{r_1, \dots, r_n\}$ of elements of
  $R$.
\end{definition}

\begin{definition}
  Let $k$ be a field and $R$ and $S$ $k$-algebras. We call a map  $\phi:R
  \xrightarrow{} S$ a \textbf{$k$-algebra homomorphism} if $\phi$ is a ring
  homomorphism, and  $\phi$ is the identity on $k$.
\end{definition}

\begin{lemma}\label{5.9.4}
  Let $k$ be a field. Then a ring $R$ is a finitely generated $k$-algebra if,
  and only if there exists a $k$-algebra homomorphism $\phi:k[x_1, \dots,
  x_n] \xrightarrow{} R$ taking $k[x_1, \dots, x_n]$ onto $R$.
\end{lemma}
\begin{proof}
  If $R$ is generated by elements  $r_1, \dots, r_n$ as a $k$-algebra, then
  define the map $\phi:k[x_1, \dots, x_n] \xrightarrow{} R$ by taking $x_i
  \xrightarrow{} r_i$, for all $1 \leq i \leq n$, and  $\phi(a)=a$ for all $a
  \in k$. Then  $\phi$ extends to a ring homomorphism of  $k[x_1, \dots,
  x_n]$ onto $R$.

  Conversly, let  $\phi$ be a $k$-algebra homomorphism of $k[x_1, \dots,
  x_n]$ onto $R$,  such that the images $\phi(x_1), \dots \phi(x_r)$ generate
  $R$ as a  $k$-algebra. Then $R$ is finitely generated, and since  $k[x_1,
  \dots, x_n]$ is Notherian by the corollary to Hilbert's basis theorem, $R$
  is a quotient of a Noetherian ring, and hence  $R$ is Noetherian. This
  makes  $R$ a finitely generated  $k$-algebra.
\end{proof}

\begin{example}\label{example_5.25}
  Let $R$ be a  $k$-algebra, for some field  $k$, viewed as a finite
  dimensional vector space over  $k$. In particular, let
  $R=\faktor{k[x]}{(f(x))}$, where $f(x)$ is a nonzero polynomial over $k$.
  Then  $R$ is a finitely generated  $k$-algebra, since it has a finite
  basis, and that basis serves as a generator for $R$ as a  $k$-algebra.
  Thus, we have the ideals of $R$ are $k$-subspaces. Moreover, any ascending
  chain of ideals of $R$ has at most  $\dim_k{R}-1$ distinct terms, which
  means that $R$ satisfies the ascending chain condition.
\end{example}

 \begin{theorem}\label{theorem_5.9.5}
     Let $\af$ be an ideal in a ring $R$, and consider the natural map $\pi:R
     \xrightarrow{} \faktor{R}{\af}$. The following are true.
     \begin{enumerate}
         \item[(1)] For every ideal $\bf'$ of $\faktor{R}{\af}$,
             $\inv{\pi}(\bf')=\bf$ is an ideal of $R$ containing $\af$.
             Moreover, for any ideal $\bf$ of $R$ containing $\af$, then
             $\pi(\bf)=\bf'$.

         \item[(2)] The ideal $\bf'$ of $\faktor{R}{\af}$ is a radical ideal if,
             and only if $\bf$ is a radical ideal in $R$.

         \item[(3)] If $\bf$ is finitely generated in $R$, then $\bf'$ is
             finitely generated in $\faktor{R}{\af}$. Moreover,
             $\faktor{R}{\af}$ is Noetherian if $R$ is Noetherian.
     \end{enumerate}
 \end{theorem}
 \begin{proof}
     Let $\bf'$ be an ideal of $\faktor{R}{\af}$. Since the natural map $\pi$ is
     onto, there is an ideal $\bf \in R$ for which $\bf=\inv{\pi}(\bf')$. Now,
     let $a,b \in \bf$, then $\pi(a),\pi(b) \in \bf'$, so that $\pi(a+b) \in
     \bf'$ and $-\pi(a) \in \bf'$. Moreover, if $a \in \bf$, and $r \in R$, then
     $r\pi(a)=\pi(ra) \in \bf'$, since $\bf'$ is an ideal. Now, since
     $\ker{\pi}=\af$, we have that $\af \subseteq \bf$. So that $\bf$ is an
     ideal containing $\af$. By similar reasoning, if $\bf$ is an ideal
     containing $\af$, then $\bf'=\pi(\bf)$ is also an ideal.

     Now, suppose that $\bf$ is a radical ideal. That is, $\bf=\sqrt{\bf}$. Since
     $\bf=\inv{\pi}(\bf')$, we have $\inv{\pi}(\bf')=\sqrt{\inv{\pi}(\bf')}$.
     Now, suppose that $\bf$ is a prime ideal, then if $ab \in \bf$, either $a
     \in \bf$ or $b \in \bf$. This implies whenever $\pi(ab) \in \bf'$, either
     $\pi(a) \in \bf'$ or $\pi(b) \in \bf'$. This makes $\bf'$ prime. Similarly,
     if $\bf'$ is prime so is $\bf$. Finally, by definition of a maximal idea,
     $\bf$ is maximal if, and only if $\bf'$ is maximal.

     Finally, suppose that $\bf$ is finitely generated, then $\bf=(a_1, \dots,
     a_n)=\inv{\pi}(\bf')$ for $a_1, \dots, a_n \in R$. Then every element of
     $\bf$ is the sum of $a_1, \dots, a_n$. That is, $b=r_1a_1+\dots+r_na_n$ for
     every $b \in \bf$, and $r_1, \dots, r_n \in R$. Now, since $b \in
     \bf=\inv{\pi}(\bf')$, then $\pi(b)=r_1\pi(a_1)+\dots+r_n\pi(a_n) \in \bf'$,
     so that $\bf'=(\pi(a_1), \dots, \pi(a_n))$. This makes $\bf'$ finitely
     generated. We can then conclude that if $R$ is Noetherian, by theorem
     \ref{1.4.2}, $\faktor{R}{\af}$ must also be Noetherian.
 \end{proof}
 \begin{corollary}
     Let $k$ be a field and $\af$ an ideal of $k[x_1, \dots, x_n]$. Any ring of
     the form $\faktor{k[x_1, \dots, x_n]}{\af}$ is a Noetherian ring.
 \end{corollary}
