\section{Polynomail Rings, Matrix Rings, and Group Rings}
\label{section_5.2}

\begin{theorem}\label{theorem_5.2.1}
    Let $R$ be a commutative ring with identity, and define $R[x]=
    \{f(x)=a_0+a_1x+\dots+a_nx^n : a_0, \dots a_n \in R\}$. Define the
    operations $+$ and  $\cdot$ on $R[x]$ for $f(x)=a_0+a_1x+\dots+a_nx^n$ and
    $g(x)=b_0+b_1x+\dots+b_nx^n$ by:
    \begin{align*}
        f+g &=  (a_0+b_0)+(a_1+b_1)x+\dots+(a_n+b_n)x^n \\
        fg  &=  c_0+c_1x+\dots+c_kx^k \text{ where }
        c_j=\sum_{i=0}^j{a_ib_{j-i}} \text{ and } k=n+m  \\
    \end{align*}
    Then $R[x]$ is a commutative ring with identity.
\end{theorem}

\begin{definition}
    Let $R$ be a commutative ring with identity. We call the ring $R[x]$ the
    \textbf{ring of polynomials} in $x$ with \textbf{coefficients} in $R$ whose
    elements of the form
    \begin{equation*}
        f(x)=a_0+a_1x+\dots+a_nx^n
    \end{equation*}
    where $n \geq 0$ are called \textbf{polynomails}. If $a_n \neq 0$, then the
     \textbf{degree} of $f$ is  denoted $\deg{f}=n$, and $f$ is called
     \textbf{monic} if $a_n=1$. We call $+$ and $\cdot$ the  \textbf{addition}
     and \textbf{multiplication} of polynomials.
\end{definition}

\begin{example}\label{example_5.6}
    \begin{enumerate}
        \item[(1)] Take $R$ any commutative ring with identity and form $R[x]$.
            One can verify that the polynomial $0(x)=0+0x+\dots+0x^n+\dots=0$,
            in this case we call $0$ the \textbf{zero polynomial}. Similarly,
            the additive inverse of $f(x)=a_0+a_1x^1+\dots+a_nx^n$ is the
            polynomial $-f(x)=-a_0-a_1x^1-\dots-a_nx^n$. Now, since $R[x]$ has
            identity, the \textbf{identity} polynomial is $1(x)=1+0x+\dots=1$,
            that is, it is the identity in $R$. Lastly, we call a polynomial $f$
            with  $\deg{f}=0$ a \textbf{constant polynomial}. Notice that $0$
            and  $1$ are constant polynomials.

        \item[(2)] $\Z[x]$, $\Q[x]$, $\R[x]$ and $\C[x]$ are the polynomial
            rings in $x$ with coeffiients in $\Z$, $\Q$, $\R$, and $\C$
            respectively.

        \item[(3)] Notice that the rings $\Z[\omega]$ and $\Z[i]$ are polynomial
            rings in $\omega$ and  $i$, respectively, with coefficients in $\Z$,
            and where $\omega=\sqrt{D}$ if $D \not\equiv 1 \mod{4}$ or
            $\omega=\frac{1+\sqrt{D}}{2}$ otherwise, and $i^2=-1$. Notice that
            the highest degree a polynomial in  $\Z[i]$ can achieve is $\deg=1$;
            however, one may be able to form polynomial rings in other variables
            with coefficients in $\Z[i]$, i.e. take $Z[x]$, where $Z=\Z[i]$.

        \item[(4)] $\faktor{\Z}{3\Z}[x]$ is the polynomial ring with
            coefficients in $\faktor{\Z}{3\Z}$.
    \end{enumerate}
\end{example}

\begin{theorem}\label{theorem_5.2.2}
    Let $R$ be an integral domain, and let  $p,q \neq 0$ be polynomials in
    $R[x]$. Then the following are true:
    \begin{enumerate}
        \item[(1)] $\deg{pq}=\deg{p}+\deg{q}$.

        \item[(2)] The units of $R[x]$ are precisely the units of $R$

        \item[(3)] $R[x]$ is an integral domain.
    \end{enumerate}
\end{theorem}
\begin{proof}
    Consider the leading terms $a_nx^n$ and  $b_mx^m$ of  $p$ and  $q$
    respectively. Then  $a_nb_mx^{m+n}$ is the leading term of $pq$; moreover we
    require $a_nb_m \neq 0$. Now, if $\deg{pq}<m+n$, then $ab=0$, making $a$ and
    $b$ zero divisors of  $R$; impossoble. Therefore  $ab \neq 0$. It also
    follows that since no term of $p$ is a zero divisor, then $p$ cannot be a
    zero divisor of  $R[x]$. Lastly, if $pq=1$, then $\deg{p}+\deg{q}=0$, so
    that $pq$ is a constant polynomial. Noticing that constant polynomials are
    simply just elements of $R$, then $p$ and $q$ are units.
\end{proof}

\begin{theorem}\label{theorem_5.2.3}
    Let $R$ be a ring. Let  $R^{n \times n}$ be the set of all $n \times n$
    matrices with entries in $R$ and define the operations $+$ and  $\cdot$ by:
    \begin{align*}
        (a_{ij})+(b_{ij})   &= (a_{ij}+b_{ij})  \\
        (a_{ij})(b_{ij})    &=  (c_{ij}), \text{ where }
        c_{ij}=\sum_{k=1}^n{a_{ik}b_{kj}}   \\
    \end{align*}
    Then $R^{m \times n}$ forms a ring under $+$ and $\cdot$.
\end{theorem}

\begin{definition}
    For any ring $R$, we call the ring  $R^{n \times n}$ the  \textbf{matrix
    ring} of $n \times n$ matrices with entries in $R$.
\end{definition}

\begin{example}\label{example_5.7}
    \begin{enumerate}
        \item[(1)] Note that if $R$ is a commutative ring, then for $n \geq 2$,
            $R^{n \times n}$ need not be commutative.

        \item[(2)] We call matrices of $R^{n \times n}$, for $n \in \Z^+$
            \textbf{square matrices}. We call a matrix $(a_{ij}) \in R^{n \times
            n}$ \textbf{scalar} if $a_{ii}=1$ for all $1 \leq i \leq n$ and
            $a_{ij}=0$ whenever $i \neq j$.

        \item[(3)] If $R$ has identity, then so does  $R^{n \times n}$. We call
            the identity of $R^{n \times n}$ the \textbf{identity matrix} and
            denote it as the $n \times n$ scalar matrix  $I$ with $1$ across the
            diagonal. We call the units of $R^{n \times n}$ \textbf{invertible}
            matrices, and denote the unit group of invertible matrices to be
            $GL(n,R)$ the general linear group of degree $n$ over $R$.

        \item[(4)] Notice that $2\Z^{n \times n} \subseteq \Z^{n \times n}
            \subseteq \Q^{n \times n} \subseteq \R^{n \times n} \subseteq \C^{n
            \times n}$.

        \item[(5)] Let $R$ be a ring, and  $R^{n \times n}$ its matrix ring.
            Let $U^{n \times n}=\{(a_{ij}) : a_{pq}=0 \text{ whenever } p>q\}$
            the set of \textbf{upper triangular matrices}. Then $U^{n \times n}
            \subseteq R^{n \times n}$ is a subring.
    \end{enumerate}
\end{example}

\begin{theorem}\label{theorem_5.2.4}
    Let $R$ be a ring with identity, and let $G$ be a finite group of order $n$.
    Let $RG$ the set of all sums $a_1g_1+\dots+a_ng_n$, where $a_i \in R$ for
    all  $1 \leq i \leq n$. Define the operations $+$ and $\cdot$ by:
    \begin{align*}
        (a_1g_1+\dots+a_ng_n)+(b_1g_1+\dots+b_ng_n) &= (a_1+b_1)g_1+\dots+(a_n+b_n)g_n \\
        (a_1g_1+\dots+a_ng_n)(b_1g_1+\dots+b_ng_n)  &=  c_1g_1+\dots+c_ng_n,
        \text{ where } c_k=\sum_{g_k=g_ig_j}{a_ib_j}
    \end{align*}
    Then $RG$ forms a ring with identity under  $+$ and  $\cdot$. Moreover, $RG$
    is commutative if, and only if $G$ is abelian.
\end{theorem}

\begin{definition}
    Let $R$ be a ring with identity, and let  $G$ be a finite group of order
    $n$. We call the ring $RG$ the \textbf{group ring} of $G$. We call the
    elements of  $RG$  \textbf{formal sums} of the elements of $G$.
\end{definition}

\begin{example}\label{example_5.8}
    \begin{enumerate}
        \item[(1)] Consider $D_8=\langle r,t : r^4=t^2=1, rt=t\inv{r} \rangle$
            and $\Z$. Let $a,b \in \Z{D_8}$ where $a=r+r^2-2t$ and  $b=-3r^2+rt$.
            Then
            \begin{align*}
                a+b &=  r-2r^2+rt-t \\
                ab  &=  -5r^3+r^3t+7r^2t-3  \\
            \end{align*}

        \item[(2)] For any ring with identity $R$, and finite group $G$, $R
            \subseteq RG$, for take the elements of  $R$ to be the sums
            $a_1+\dots+a_n$. $G \subseteq RG$, for  $g_i=1g_i$; moreover, each
            $g_i$ has an inverse in  $RG$, so we call  $G$ the subgroup of units
            of $RG$.

        \item[(3)] Let $G$ be a group with $\ord{G}>1$. Let $g \in G$ with
            $\ord{g}=m$. Notice that the elements $(1-g),(1+g+\dots+g^{m-1}) \in
            RG$ are nonzero, but that
            \begin{equation*}
                (1-g)(1+g+\dots+g^{m-1})=1-g^m=1-1=0
            \end{equation*}
            which makes $1-g$ a zero divisor. In general, the ring $RG$ will
            always have zero divisors.

        \item[(4)] Let $G$ be a finite group. We call the rings  $\Z{G}$,
            $\Q{G}$, $\R{G}$, and $\C{G}$ the \textbf{integral},
            \textbf{rational}, \textbf{real}, and \textbf{complex} group rings
            of $G$, respectively. Notice that $\Z{G} \subseteq \Q{G} \subseteq
            \R{G} \subseteq \C{G}$. Moreover, if $H \leq G$ is a subgroup of
            $G$, then  $RH \subseteq RG$ is a subring.
    \end{enumerate}
\end{example}
