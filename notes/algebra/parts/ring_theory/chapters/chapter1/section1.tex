\section{Definitions and Examples}
\label{section_5.1}

\begin{definition}
  A \textbf{ring} $R$ is a set together with two binary operations  $+:(a,b)
  \xrightarrow{} a+b$ and $\cdot:(a,b) \xrightarrow{} ab$ called
  \textbf{additon} and \textbf{multiplication} such that:
  \begin{enumerate}
    \item[(1)] $R$ is an Abelian group over $+$, where we denote the
      identity element as $0$ and call it the \textbf{zero}-element,
      and the inverse of each $a \in R$ as $-a$.

    \item[(2)] For any $a,b \in R$,  $ab \in R$ and $a(bc)=(ab)c$. That is,
      $R$ is closed under multiplication, and  multiplication is
      associative.

    \item[(3)] $a(b+c)=ab+ac$ and $(a+b)c=ac+bc$.
  \end{enumerate}
  We call $R$ commutative if for all $a,b \in R$, $ab=ba$. If there
  exists an element $1 \in R$ such that $a1=1a=R$ for all $a \in R$,
  then we call $R$ a ring with  \textbf{identity}.
\end{definition}

\begin{definition}
  A ring $R$ with identity $1 \neq 0$ is called a \textbf{division ring} if
  for all $a \in R$, where  $a \neq 0$, there exists a  $b \in R$ such that
  $ab=ba=1$. We call a commutative division ring a \textbf{field}.
\end{definition}

\begin{example}\label{example_5.1}
  \begin{enumerate}
    \item[(1)] Let $R$ be an abelian group under an operation $+$, define the operation
      $\cdot$ by  $(a,b) \xrightarrow{} ab=0$ for all $a,b \in R$. Then $R$ is a
      ring under $+$ and  $\cdot$, called the  \textbf{trivial ring}. If
      $R=\langle 0 \rangle$, the trivial group, then we call $R$ the  \textbf{zero
      ring}. We write $R=(0)$ to denote the zero-ring.

    \item[(2)] The integers $\Z$ form a ring under the usual addition and
      muiltiplication.

    \item[(3)] The sets of rational numbers $\Q$ and the set of real numbers
      $\R$ are rings under their usual addition and multiplication; in fact,
      they are fields. The complex numbers $\C$ also form a field under
      complex addition and complex multiplication, where
      \begin{align*}
        +:(a+ib,c+id)   &   \xrightarrow{} (a+c)+i(b+d)   \\
        \cdot:(a+ib,c+id)   &   \xrightarrow{} (ac-bd)+i(ad+bc)   \\
      \end{align*}

    \item[(4)] The factor group of integers modulo $n$, $\faktor{\Z}{n\Z}$ is a
      commutative ring under addition modulo $n$, and multiplication modulo
      $n$, $\faktor{\Z}{n\Z}$ has identity $1 \mod{n}$. $\faktor{\Z}{n\Z}$
      forms a field if, and only if $n=p^r$, where $p$ is a prime.

    \item[(5)] We define the \textbf{real quaternions} to be the set:
      \begin{equation*}
        \Hb=\{a+ib+jc+kd : a,b,c,d \in \R, i^2=j^2=k^2=-1 \text{ and }
        ij=k, jk=i, \text{ and } ki=j\}
      \end{equation*}
      $\Hb$ is a ring under addition and
      multiplication are defined for all $x=a+ib+jc+kd$ and $y=e+if+jg+kh$ to be:
      \begin{align*}
        +(x,y): &   \xrightarrow{} x+y=(a+e)+i(b+f)+j(c+g)+k(d+h)   \\
        \cdot(x,y):  &   \xrightarrow{} xy= (a+ib+jc+kd)(e+if+jg+kh) \\
      \end{align*}

    \item[(6)] Let $A$ be a ring and $A^X$ the set of all maps $f:X \xrightarrow{}
      A$. Then $R$ forms a ring under function addition  $f+g(x)=f(x)+g(x)$ and
      function multiplication $fg(x)=f(x)g(x)$. Notice that $A^X$ is commutative if,
      and only if  $A$ is, moreover, $A^X$ has identity if, and only if  $A$ has
      identity.

    \item[(7)] We say a real-valued function $f:\R \xrightarrow{} \R$ has
      \textbf{compact support} if there exist $a,b \in \R$ such that $f(x)=0$
      for all $x \notin [a,b]$. The set of all functions with compact support
      forms a ring without identity under function addition and function
      multiplication.

    \item[(8)] Let $X,Y \subseteq \R$. We denote the set of all continuous
      functions  $f:X \xrightarrow{} Y$ by $C(X,Y)$. Then $C(X,Y)$ forms a
      commutative ring with identity under function addition and function
      multiplication.
  \end{enumerate}
\end{example}

\begin{proposition}\label{proposition_5.1.1}
  Let $R$ be a ring. Then the following are true for all  $a,b \in R$.
  \begin{enumerate}
    \item[(1)] $0a=a0=0$.

    \item[(2)] $(-a)b=a(-b)=-(ab)$.

    \item[(3)] $(-a)(-b)=ab$

    \item[(4)] If $R$ has identity  $1 \neq 0$, then $1$ is unique and
      $-a=(-1)a$.
  \end{enumerate}
\end{proposition}
\begin{proof}
  \begin{enumerate}
    \item[(1)] Notice $0a=(0+0)a=0a+0a$, so that $0a=0$. Likewise, $a0=0$ by
      the same reasoning.

    \item[(2)] Notice that $b-b=0$, so $a(b-b)=ab+a(-b)=0$, so that
      $a(-b)=-(ab)$. The same argument with $(a-a)b$ gives $(-a)b=-(ab)$.

    \item[(3)] By the inverse laws of addition in $R$, we have
      $-(a(-b))=-(-(ab))$, so that $(-a)(-b)=ab$.

    \item[(4)] Suppose $R$ has identity $1 \neq 0$, and suppose there is an
      element $2 \in R$ for which  $2a=a2=a$ for all $a \in R$. Then we have
      that $1 \cdot 2=1$ and  $1 \cdot 2=2$, making  $1=2$; so  $1$ is unique.
      Now, we have that $a+(-a)=0$, so that $1(a+(-a))=1a+1(-a)=1a+(-a)=0$ So
      $(-a)=-(1a)=(-1)a$ by (2).
  \end{enumerate}
\end{proof}

\begin{definition}
  Let $R$ be a ring. We call a non-zero element $a \in R$ a
  \textbf{zero-divisor} if there exists a non-zero element $b \in R$
  such that $ab=0$. We all $a$ a unit if there is a $b \in R$ such
  that $ab=ba=1$.
\end{definition}

\begin{example}\label{example_5.2}
  Notice if $R$ is a ring with identity $1$, then $1$ is a unit of $R$ by
  definition.
\end{example}

\begin{definition}
  Let $R$ be a ring. We call the set of all units in  $R$ the \textbf{group
  of units} and denote it $\Uc(R)$.
\end{definition}

\begin{proposition}\label{proposition_5.1.2}
  Let $R$ be a ring with identity $1 \neq 0$. Then the group of units
  $\Uc(R)$ forms a group under multiplication.
\end{proposition}
\begin{proof}
  Let $a,b \in R$ be units in $R$. Then there are $c,d \in R$ for which
  $ac=ca=1$ and  $bd=db=1$. Consider then $ab$. Then  $ab(dc)=a(bd)c=ac=1$ and
  $(dc)ab=d(ca)b=db=1$ so that $ab$ is also a unit in $R$. Moreover $\Uc{(R)}$
  inherits the associativity of  $\cdot$ and $1$ serves as the identity
  element of $\Uc{(R)}$. Lastly, if $a \in \Uc{(R)}$ is a unit there is a $b
  \in R$ for which $ab=ba=1$. This also makes $b$ a unit in $R$, and the
  inverse of $a$.
\end{proof}
\begin{corollary}
  $a$ is a zero divisor if, and only if it is not a unit.
\end{corollary}

\begin{proof}
  Suppose that $a \neq 0$ is a zero divisor. Then there is a  $b \in R$ such
  that $b \neq 0$ and $ab=0$. Then for any $v \in R$,  $v(ab)=(va)b=0$ so that
  $a$ cannot be a unit. On the other hand let  $a$ be a unit, and  $ab=0$ for
  some  $b \neq 0$. Then there is a  $v \in R$ for which
  $v(ab)=(va)b=1b=b=0$. Then $b=0$ which is a contradiction.
\end{proof}
\begin{corollary}
  If $R$ is a field, then it has no zero divisors.
\end{corollary}
\begin{proof}
  Notice by definition of a field, every element is a unit, except for $0$.
\end{proof}

\begin{example}\label{example_5.3}
  \begin{enumerate}
    \item[(1)] $\Z$ has no zero divisors, and has as units the elements $-1$
      and $1$.


    \item[(2)] For any $n \in \Z^+$, the unit group of the ring
      $\faktor{\Z}{n\Z}$ is the group $\Uc{(\faktor{\Z}{n\Z})}$.
      Observe that $a \in \Uc{(\faktor{\Z}{n\Z})}$ if, and only if
      $(a,n)=1$, so that the group of units of the ring
      $\faktor{\Z}{n\Z}$ coincides with the group definition.

    \item[(3)] Let $D \in \Q$ be squarefree. Define
      $\Q(\sqrt{D})=\{a+b\sqrt{D} : a,b \in \Q\}$. Then $\Q(\sqrt{D})$ is
      a field called the \textbf{quadratic field} under the operations
      \begin{align*}
        +:(a+b\sqrt{D}, c+d\sqrt{D})    & \xrightarrow{} (a+c)+(b+d)\sqrt{D} \\
        \cdot((a+b\sqrt{D}, c+d\sqrt{D}))   &   \xrightarrow{}
        (ac-bdD)+(ad-bc)\sqrt{D} \\
      \end{align*}
      Since $\Q(\sqrt{D})$ is a field, every element is a unit.
  \end{enumerate}
\end{example}

\begin{definition}
  A commutative ring with identity $1 \neq 0$ is called an \textbf{integral
  domain} if it has no zero divisors.
\end{definition}

\begin{proposition}\label{proposition_5.1.3}
  Let $R$ be a ring, and $a$ not a zero divisor. Then if $ab=ac$, then either
  $a=0$, or  $b=c$.
\end{proposition}
\begin{proof}
  Notice that $ab=ac$ implies  $ab-ac=a(b-c)=0$. Since $a$ is not a zero
  divisor, either  $a=0$ or  $b-c=0$.
\end{proof}
\begin{corollary}
  Any finite integral domain is a field.
\end{corollary}
\begin{proof}
  Let $R$ be a finite integral domain and consider the map on $R$, by $x
  \xrightarrow{} ax$. By above, this map is 1--1, moreover since $R$ is
  finite, it is also onto. So there is a $b \in R$ for which $ab=1$, making
  $a$ a unit. Since $a$ is abitrarily chosen, this makes $R$ a field.
\end{proof}
\begin{corollary}
  If $R$ is a field it is a (not necessarily finite) integral domain.
\end{corollary}

\begin{example}\label{example_5.4}
  We have that fields are integral domains, and finite integral domains are
  fields. However, notice that not every integral domain need be a field. $\Z$
  is an integral domain that is not a field. Moreover, so are the real
  quaternions  $\Hb$.
\end{example}

\begin{definition}
  A \textbf{subring} of a ring $R$ is a subgroup of $R$ closed under
  multiplication.
\end{definition}

\begin{example}\label{example_5.5}
  \begin{enumerate}
    \item[(1)] We have the following sequence of subgrings $\Z \subseteq \Q
      \subseteq \R \subseteq \C$.

    \item[(2)] The factor group $\faktor{\Z}{n\Z}$ is not a subgring of
      $\Z$, well the multiplication and addition of $\Z$ is different from
      that of $\faktor{\Z}{n\Z}$.

    \item[(3)] The set $\Z+i\Z+j\Z+k\Z \subseteq \Hb$ is a subring of $\Hb$.

    \item[(4)] If $k$ is a field, then any subring of $k$ is also an
      integral domain by inheretence.

    \item[(5)] The set $\Z[\sqrt{D}]=\{a+b\sqrt{D} : a,b \in \Z\}$ is a
      subring of the quadratic field $\Q(\sqrt{D})$. Moreover if $D \equiv
      1 \mod{4}$, then the set
      \begin{equation*}
        \Z\Big{[}\frac{1+\sqrt{D}}{2}  \Big{]}
        =\Big{\{} a+b\frac{1+\sqrt{D}}{2} : a,b \in \Z \Big{\}}
      \end{equation*}
      is also a subgring of $\Q(\sqrt{D})$. We call the subgring
      $\Z[\omega]$, where
      \begin{equation*}
        \omega=     \begin{cases}
          \sqrt{D}, \text{ if } D \not\equiv 1 \mod{4}    \\
          \frac{1+\sqrt{D}}{2}, \text{ if } $D \equiv 1 \mod{4}$  \\
        \end{cases}
      \end{equation*}
      the \textbf{ring of integers} in the quadratic field. When $D=-1$,
      we get the ring  $\Z[i]$, with $i^2=-1$ and call it the
      \textbf{Gaussian integers}. Notice then that $\Z[i]$ is a subring of
      $\C$; in fact, it is field in $\C$.

    \item[(6)] Consider $\Q(\sqrt{D})$ where $D$ is squarefree. We define
      the  \textbf{field norm} to be the map:
      \begin{align*}
        N: \Q(\sqrt{D}) & \xrightarrow{} D  \\
        (a+b\sqrt{D}) & \xrightarrow{} (a+b\sqrt{D})(a-b\sqrt{D})=a^2-Db^2 \\
      \end{align*}
      If $D=i^2=-1$, then  $N:a+ib \xrightarrow{} a^2+b^2$ which is the
      modulus of complex number restricted to $\Q$.

      Notice that if  $z=a+b\sqrt{D}$, $w=c+d\sqrt{D}$, then
      $N(zw)=N(z)N(w)$ moreover,
      \begin{equation*}
        N(a+\omega b)=\begin{cases}
          a^2-Db^2, \text{ if } D \equiv 2, 3 \mod{4} \\
          a^2+ab+\frac{1-D}{4}, \text{ if } D \equiv 1 \mod{4} \\
        \end{cases}
      \end{equation*}
      where
      \begin{equation*}
        \omega=\begin{cases}
          \sqrt{D}, \text{ if } D \not\equiv 1 \mod{4}   \\
          \frac{1+\sqrt{D}}{2}, \text{ if } D \equiv 1 \mod{4}    \\
        \end{cases}
      \end{equation*}
      In either case, $N:\Z[\omega] \xrightarrow{} \Z$.
  \end{enumerate}
\end{example}

\begin{proposition}\label{proposition_5.1.4}
  Let $\omega=\begin{cases} \sqrt{D}, \text{ if } D \not\equiv 1 \mod{4}   \\
    \frac{1+\sqrt{D}}{2}, \text{ if } D \equiv 1 \mod{4}
  \end{cases}$
  where $D \in \Z^+$ is squarefree. Then an element of $z \in \Z[\omega]$ is a
  unit if, and only if $N(z)=\pm1$
\end{proposition}
\begin{proof}
  Let $z=a+\omega b$ such that $N(z)=\pm 1$. Then we have
  \begin{equation*}
    \inv{z}=\pm (a+\bar{\omega}b) \in \Z[\omega]
  \end{equation*}
  making it a unit. On the other hand, if $N(zw)=N(z)N(w)=\pm 1$, then since
  $N(z),N(w) \in \Z$, we must have that both $N(z)=\pm 1$ and $N(w)=\pm 1$.
\end{proof}
