\section{Ideals}
\label{section_5.4}

\begin{definition}
  Let $R$ be a commutative ring with identity. We call the smallest ideal
  containing a nonempty subset $A$ in $R$ the  \textbf{ideal generated} by
  $A$, and we write $(A)$. We say that the ideal $(A)$ is
  \textbf{finitely generated} if $|A|$ is finite, and if
  $A=\{a_1, \dots, a_n\}$, then we denote $(A)=(a_1, \dots, a_n)$.
\end{definition}

\begin{definition}
  Let $R$ be a commutative ring with identity. We call a proper ideal
   $\af \in R$ \textbf{principle} if it is generated by a single
   element of $R$; that is:  $\af=(a)$ for some $a \in R$.
\end{definition}

\begin{example}\label{example_5.13}
  \begin{enumerate}
    \item[(1)] In any commutative ring with identity, the trivial ideal and
      $R$ are the ideals generated by $0$ and $1$, respectively, so we
      write them as $(0)$ and $R=(1)$.

    \item[(2)] In $\Z$, we can write the ideals  $n\Z=(n)=(-n)$. Notice that
      every ideal in $\Z$ is a principle ideal. Moreover, for $m,n \in
      \Z$, $n|m$ if, and only if  $n\Z \subseteq n\Z$. Notice that
      $m\Z+n\Z=d\Z$ is the ideal generated by $n$ and $n$, where $d=(m,n)$
      is the greatest common divisor of $m$ and  $n$. Indeed, by
      definition, $d|m,n$ so that $d\Z \subseteq m\Z+n\Z$, and if
      $c|m,n$, then  $c|d$, making  $m\Z+n\Z \subseteq d\Z$. Then
      $m\Z+n\Z=d\Z$ is the ideal generated by the greatest common divisor
      $(m,n)$ and consists of all diophantine equations of the form
      \begin{equation*}
        mx+ny=(m,n)
      \end{equation*}
      In general, we can define the \textbf{greatest common divisor} for
      integers $n_1, n_2, \dots, n_m$ to be the smallest such integer $d$
      generating the ideal $n_1\Z+\dots+n_m\Z=d\Z$. We then write
      $d=(n_1, \dots, n_m)$.

    \item[(3)] Consider the ideal $(2,x)$ of $\Z[x]$. $(2,x)$ is not a
      principle ideal. Indeed, suppose that $(2,x)=(a)$ for some
      polynomial $a \in \Z[x]$, then $2 \in (a)$, so that $2=p(x)a(x)$,
      of degree $\deg{p}+\deg{a}$. This makes $p$ and $a$ constant
      polynomials in $\Z[x]$. Now, since $2$ is prime in $\Z$, then only
      values for $p$ and  $q$ are  $p=\pm1$ and $a=\pm2$. If $a(x)=\pm1$,
      then every polynomial in $\Z[x]$ can be written as a polynomial in
      $(a)$, so that $(a)=\Z[x]$, impossible. If $a(x)=\pm2$, then since
      $x \in (a)$, we get $x=2q(x)$ where $q \in \Z[x]$. This cannot
      happen, so that $(a) \neq (2,x)$.


    \item[(4)] Consider $\R^{[0,1]}$ the ring of all functions $f:[0,1]
      \xrightarrow{} \R$. Let $\mf=\{f : f(\frac{1}{2})=0\}$. Then
      $\mf$ is an ideal in $\R^{[0,1]}$, in fact, notice that it is the
      kernel of the evaluation map at $\frac{1}{2}$. Define $g:[0,1] \xrightarrow{}
      \R$ by:
      \begin{equation*}
        g(x)=\begin{cases}
          1, \text{ if } x \neq \frac{1}{2}  \\
          0, \text{ if } x=\frac{1}{2}   \\
        \end{cases}
      \end{equation*}
      then $f=fg$ by definition of both  $f$ and $g$. So that $\mf=(g)$
      which makes $\mf$ a principle ideal. $\mf$ is not principle in general,
      consider $C^{[0,1]}$ the set of all realvalued continuous functions
      on $[0,1]$.

    \item[(5)] Let $G$ be a finite group and  $R$ a commutative ring with
      identity. The augmentation ideal in $RG$ is generated by the set
      $\{g-1 : g \in G\}$, and we write $(g_1-1, \dots, g_n-1)$ where
      $\ord{G}=n$. If $G$ is cyclic, then the augmentation ideal is just
      $(g-1)$, and is principle.
  \end{enumerate}
\end{example}

\begin{lemma}\label{lemma_5.4.1}
  Let $\af$ an ideal in ring $R$ with identity. Then
  \begin{enumerate}
    \item[(1)] $\af=(1)$ if, and only if $\af$ contains a unit.

    \item[(2)] If $R$ is commutative, then $R$ is a field if, and only if
      its only ideals are $(0)$ and $(1)$.
  \end{enumerate}
\end{lemma}
\begin{proof}
  Recall that $R=(1)$. Now, if $\af=(1)$, then $1 \in \af$, and  $1$ is a unit.
  Conversly, suppose that $u \in I$ with $u$ a unit. By definition, we have
  that  $r=r \cdot 1=r(uv)=r(vu)=(rv)u$, so that $1 \in \af$. This makes
  $\af=(1)$.

  Now, if $R$ is a field, then it is a commutative ring, moreover every  $r
  \neq 0$ is a unit in $R$, which makes $r \in \af$ for some ideal
  $\af \neq (0)$. This makes every $\af \neq (0)$ equal to $(1)$. Conversly,
  if $(0)$ and $(1)$ are the only ideals of the commutative ring $R$, then
  every $r \neq 0 \in (1)$, which makes themn units. Hence all nonzero $r$ is a
  unit in $R$. This makes $R$ into a field.
\end{proof}
\begin{corollary}
  If $R$ is a field, then any nonzero ring homomorphism $\phi:R \xrightarrow{}
  S$ is 1--1.
\end{corollary}
\begin{proof}
  If $R$ is a field, then either $\ker{\phi}=(0)$ or $\ker{\phi}=(1)$. Now,
  since $\ker{\phi} \neq R$, we must have $\ker{\phi}=(0)$.
\end{proof}

\begin{definition}
  We call a ring $D$ with identity a \textbf{division ring} if its only left
  and right ideals are $(0)$ and $(1)$ respectively.
\end{definition}

\begin{example}\label{example_5.14}
  For any field $F$, the only two sided ideals of  $F^{n \times n}$ are $(0)$
  and $(1)$, so that $F^{n \times n}$ is a division ring.
\end{example}

\begin{definition}
  We call an ideal $\mf$ of a ring  $R$  \textbf{maximal} if $\mf \neq
  (1)$, and if $\af$ is an ideal such that
  \begin{equation*}
    \mf \subseteq \af \subseteq (1)
  \end{equation*}
  then either $\mf=\af$ or $\af=(1)$.
\end{definition}

\begin{lemma}\label{lemma_5.4.2}
  If $R$ is a ring with identity, every proper ideal is contained in a maximal
  ideal.
\end{lemma}
\begin{proof}
  Let $\af$ a proper ideal of $R$. Let $\Sc=\{\bf : \bf \neq (1) \text{ is a proper
  ideal, and } \af \subseteq \bf \}$. Then $\Sc \neq \emptyset$, as
  $\af \in \Sc$, and the relation $\subseteq$ partially orders $\Sc$. Let
  $\Cc$ be a chain in $\Sc$ and define
  \begin{equation*}
    \jf=\bigcup_{\af \in \Cc}{\af}
  \end{equation*}
  We have that $\jf \neq \emptyset$ since  $(0) \in \jf$. Now, let
  $a,b \in \jf$ , then we have that either  $(a) \subseteq (b)$ or
  $(b) \subseteq (a)$, but not both. In either case, we have $a-b \in \jf$
  so that  $\jf$ is closed under additive inverse. Moreover, since
  $\af \in \Cc$ is an ideal, by definition, $\jf$ is closed with respect to
  absorbption. This makes  $\jf$ an ideal.

  Now, if  $1 \in \jf$, then $J=(1)$ and $\jf$ is not proper, and $\af=(1)$ by
  definition of $\jf$. This is a contradiction as $\af$ must be
  proper. Therefore $\jf$ must also be a proper ideal. Therefore,
  $\Cc$ has an upperbound in $\Sc$, therefore, by Zorn's lemma,
  $\Sc$ has a maximal element $\mf$.
\end{proof}

\begin{lemma}\label{lemma_5.4.3}
  Let $R$ be a commutative ring. An ideal $\mf$ is maximal if, and only if
  $\faktor{R}{\mf}$ is a field.
\end{lemma}
\begin{proof}
  If $\mf$ is maximal, then there is no ideal $\af \neq (1)$ for which
  $\mf \subseteq \af \subseteq (1)$ By the fourth isomorphism theorem,
  the ideals of  $R$ containing $\mf$ are in an inclusion preserving 1--1
  correspondence with the those of $\faktor{R}{\mf}$. Therefore $\mf$ is
  maximal if, and only if the only ideals of $\faktor{R}{M}$ are
  $\mf)$ and $1+\mf$.
\end{proof}

\begin{example}\label{example_5.15}
  \begin{enumerate}
    \item[(1)] Let $n \geq 0$ an integer. The ideal  $n\Z$ is maximal in
      $\Z$ if and only if  $\faktor{\Z}{n\Z}$ is a field. Therefore $n\Z$
      is maximal if, and only if  $n=p$ a prime in $\Z$. So the maximal
      ideals of $\Z$ are those  $p\Z$ where  $p$ is prime.

    \item[(2)] $(2,x)$ is not principle in $\Z[x]$, but it is maximal in
      $\Z[x]$, as $\faktor{\Z[x]}{(2,x)} \simeq \faktor{\Z}{2\Z}$ which
      is a field.

    \item[(3)] The ideal $(x)$ is not maximal in $\faktor{\Z}{n\Z}$, since
      $\faktor{\Z}{(x)} \simeq \Z$, which is not a field. Moreover, $(x)
      \subseteq (2,x) \subseteq \Z[x]$. We construct this isomorphism by
      identifying $x=0$, then all polynomials of $\faktor{\Z[x]}{(x)}$ only
      have constant term in $\Z$.

    \item[(4)] Let $a \in [0,1]$, and $\mf_a=\{f : f(a)=0\}$ the kernel of the
      evaluation map at $a$. Then  $M$ is principle, moreover, we also
      have that since  $f(a) \in \R$, then $\faktor{\R^{[0,1]}}{M_a}
      \simeq \R$ which makes $\mf_a$ maximal.

    \item[(5)] If $F$ is a field and  $G$ a finite group of order $n$, then
      the augmentation ideal  $(g_1-1, \dots, g_n-1)$ is maximal in $FG$.
      Let $\pi:\sum{g_ia_i} \xrightarrow{} \sum{a_i}$, then
      $\ker{\pi}=(g_1-1, \dots, g_n-1)$ and $\pi(FG)=F$. This makes
      $\faktor{FG}{(g_1-1, \dots, g_n-1)} \simeq F$.
  \end{enumerate}
\end{example}

\begin{definition}
  We call an ideal $\pf$ in a commutative ring $R$ with identity  $1
  \neq 0$ \textbf{prime} if $\pf \neq (1)$ and if $ab \in \pf$ implies
  either $a \in \pf$ or $b \in \pf$.
\end{definition}

\begin{lemma}\label{lemma_5.4.4}
  Let $R$ be a commutative ring with identity $1 \neq 0$. An ideal
  $\pf \neq (1)$ of $R$ is prime if, and only if $(ab) \subseteq \pf$
  implies $(a) \subseteq \pf$ or $(b) \in \pf$.
\end{lemma}
\begin{proof}
  This follows directly from definition.
\end{proof}

\begin{example}\label{example_5.16}
  The prime ideals of $\Z$ are $p\Z$ with  $p$ prime together with $(0)$.
\end{example}

\begin{lemma}\label{lemma_5.4.5}
  An ideal $\pf$ in a commutative ring $R$, is prime if, and
  only if $\faktor{R}{\pf}$ is an integral domain.
\end{lemma}
\begin{proof}
  Suppose that $\pf$ is prime, and let  $(a+\pf)(b+\pf)=ab+\pf=\pf$. This gives us
  that $ab \in \pf$ and hence  $a \in \pf$ or  $b \in \pf$. Then either $a+\pf=\pf$
  or $b+\pf=\pf$ in $\faktor{R}{\pf}$. Conversly, if $\faktor{R}{\pf}$ is an
  integral domain, then for any $a+\pf,b+\pf$  $ab+\pf=\pf$ implies that either
  $a+\pf=\pf$ or  $b+\pf=\pf$. Then $a \in \pf$ or  $b \in \pf$, only when
  $ab \in \pf$. This makes  $\pf$ prime.
\end{proof}
\begin{corollary}
  Every maximal ideal is a prime ideal.
\end{corollary}

\begin{example}\label{example_5.17}
  \begin{enumerate}
    \item[(1)] The prime ideals of $\Z$ are $p\Z$, where $p$ is prime, which
      are the maximal ideals of $\Z$.

    \item[(2)] The ideal $(x)$ in $\Z[x]$ is a prime ideal, but it is not
      maximal as $(x) \subseteq (2,x) \subseteq \Z[x]$.
  \end{enumerate}
\end{example}
