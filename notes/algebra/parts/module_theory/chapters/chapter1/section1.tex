\section{Definitions and Examples}

\begin{definition}
  Let $R$ be a ring. A  \textbf{left module} over $R$ (or \textbf{left
  $R$-module}) is a set $M$ together with a binary operation $+:M
  \times M \xrightarrow{} M$, called \textbf{addition}, and a left
  action $R \times M \xrightarrow{} M$, taking $(r,m) \xrightarrow{}
  rm$, called \textbf{scalar multiplication} such that $M$ is a group
  under $+$ and for every $r,s \in R$ and $m,n \in M$:
  \begin{enumerate}
    \item[(1)] $r(m+n)=rm+rn$

    \item[(2)] $(r+s)m=rm+sm$

    \item[(3)] $(rs)m=r(sm)$
  \end{enumerate}
   We define a \textbf{right module} (\textbf{right $R$-module}) over
   $R$ similarly, where the action is a right action  $M \times R
   \xrightarrow{} M$ taking $(m,r) \xrightarrow{} mr$. If $R$ is
   a commutative ring, we simply call $M$ a \textbf{module} over
   $R$ (or an \textbf{$R$-module}).
\ends{definition}

\begin{definition}
  If $R$ is a ring with identity $1$, we call a left (or right )
  $R$-module unital if $1m=m$ (similarly $m1=m$) for all $m \in M$.
\end{definition}

\begin{note}
  When context is clear, we will also simply just call left (or right)
  $R$-modules just $R$-modules.
\end{note}

\begin{definition}
  Let $R$ be a ring, and $M$ an $R$-module. We call a subgroup $N$ of
  $M$ a  \textbf{submodule} if it is closed under the action $R \times
  M \xrightarrow{} M$ when restricted to $N$; i.e. for every  $r \in
  R$ and $n \in N$, $rn \in N$.
\end{definition}

\begin{example}\label{example_4.1}
  \begin{enumerate}
    \item[(1)] Any ring $R$ is a left module over itself, and
      similarly, also a right module over itself. The actions $R
      \times M \xrightarrow{} M$ and $M \times R \xrightarrow{} M$
      need not coincide however.

    \item[(2)] If $k$ is a filed, then any vector space $V$ over $k$
      is a $k$-module by definition. Now, let $n \in \Z^+$ and define
      \textbf{affine $n$-space} over $k$ to be:
      \begin{equation*}
        \A^n(k)=\{(x_1, \dots, x_n) : x_i \in k\}
      \end{equation*}
      i.e. $\A^n(k)=k^n$. It can be shown that $\A^n(k)$ is a vector
      space over $k$.

    \item[(3)] Let $R$ be a ring with identity $1$. define
      \begin{equation*}
        R^n = \{(x_1, \dots, x_n) : x_i \in R\}
      \end{equation*}
      define $+:R^n \times R^n \xrightarrow{} R^n$ to be componentwise
      addition on $R$, and  $R \times R^n \xrightarrow{} R^n$ by
      $r(x_1, \dots, x_n)=(rx_1, \dots, rx_n)$. Then $R^n$ is a left
      $R$-module, called the \textbf{left free module} of
      \textbf{rank} $n$ on $R$. We define the \textbf{right free
      module} on $R$ similarly with the action $R^n \times R
      \xrightarrow{} R^n$ given by $(x_1, \dots, x_n)r=(x_1r, \dots,
      x_nr)$. Again, if $R$ is commutative, we simply call
      $R^n$ the \textbf{free module} of \textbf{rank} $n$ on $R$.

    \item[(4)] If $M$ is an $R$-module, and $S \subseteq R$ is a
      subring of $R$, then $M$ can also be made into an $S$-module by
      restricting the action $R \times M \xrightarrow{} M$ to the
      action $S \times M \xrightarrow{} M$.

    \item[(5)] Let $R$ be a ring,  $I$ an ideal of  $R$, and let  $M$
      be an $R$-module. We say that $M$ is \textbf{annihiliated} by
      $I$ if for every $a \in I$, and  $m \in M$, $am=0$.

      Now, define the action $\faktor{R}{I} \times M \xrightarrow{} M$
      by $(r+I)m=rm$. Then $M$ is made into an
      $(\faktor{R}{I})-module$. If $I$ is a maximal ideal, then  $M$
      becomes a vector space over the field  $\faktor{R}{I}$.

    \item[(6)] Let $A$ be an Abelian group, and define the action $\Z
      \times A \xrightarrow{} A$ by:
      \begin{equation*}
        na=
        \begin{cases}
          a^n,  & \text{ if } n>0  \\
          e,    & \text{ if } n=0  \\
          (\inv{a}) ^n, & \text{ if } n<0 \\
        \end{cases}
      \end{equation*}
      Then this action makes $A$ into a $\Z$-module. Moreover, any
      $\Z$-module defines an Abelian group, so that there is a 1--1
      correspondence between Abelian groups and $\Z$-modules.

      Now, if  $A$ is an Abelian group, and  $x \in A$ is an element
      of order $|x|=n$, then  $na=0$, moreover if $|A|=m$, then by
      Lagrange's theorem, $ma=0$ for every $a \in A$, so that $A$ can
      be made into a  $(\faktor{\Z}{m\Z})-module$. If $m=p$ a prime,
      then we call $A$ as a $(\faktor{\Z}{p\Z})$ an \textbf{elementary
      $p$-group}.

    \item[(7)] Let $k$ be a field, and $k[x]$ the polynomial ring of
      $k$ in $x$. Let $V$ be a vector space over $k$, and $T:V
      \xrightarrow{} V$ a linear transformation. We take $V$ into an
      $k[x]$-module by defining the action $k[x] \times V
      \xrightarrow{} V$ given by:
      \begin{equation*}
        p(x)v=p \circ T(v)=(a_0v+a_1Tv+\dots+a_nT^nv)
      \end{equation*}
      Now, since $k \subseteq k[x]$ is a subring one can view the
      action $k[x] \times V \xrightarrow{} V$ as extending the
      action $k \times V \xrightarrow{} V$, taking $V$ as a
      $k$-module to $V$ as a  $k[x]$-module. This extension depends on
      the choice of linear transformation $T$.

    \item[(8)] Consider affine $n$-space $\A^n(k)$ over the field $k$.
      Define the \textbf{shift operator} $T(x_1, \dots, x_n)=(x_2,
      \dots, x_n, 0)$. We can determine this action as an action of a
      polynomial in $k[x]$ on a vector of $\A^n(k)$.
  \end{enumerate}
\end{example}

\begin{lemma}\label{lemma_4.1.1}
  Let $k$ be a field. There exists a 1--1 correspondence between
  $k[x]$-modules and pairs $(V,T)$, where $V$ is a vector space over
  $k$, and $T:V \xrightarrow{} V$ is a linear transformation.
\end{lemma}
\begin{proof}
  Let $V$ be a $k[x]$-module, then $V$ is also a $k$-module (and hence
  a vector space), and observe that the action $k \times V
  \xrightarrow{} V$ by definition defines a linear transformation $T:V
  \xrightarrow{} V$. Then the action of $k$ on $V$ uniquely determines
  the action of $k[x]$ on $V$, which gives the required bijection.
\end{proof}

\begin{definition}
  Let $V$ be a vector space, and  $T:V \xrightarrow{} V$ a linear
  transformation. A subspace $W \subseteq V$ is said to be
  \textbf{$T$-stable} if $T(W) \subseteq W$.
\end{definition}

\begin{lemma}\label{lemma_4.1.2}
  If $W$ is a $T$-stable subspace of a vector space, then $T^n(W)
  \subseteq W$ for all $n \in \Z^+$.
\end{lemma}

\begin{lemma}\label{lemma_4.1.3}
  Let $k$ be a field. There exists a 1--1 correspondence between
  $k[x]$-submodules and $T$-stable subspaces of vector spaces.
\end{lemma}
\begin{proof}
  Let $V$ be a $k[x]$-module, and $T:V \xrightarrow{} V$ the linear
  transformation given by the action of $x$ on  $V$. Let  $W$ be a
  submodule of $V$. Then $W$ is a subspace of $V$, by definition, and
  (provided that $T$ is characterized by the action of $x$ on $V$),
  $T$ restricts to a transformation  $T:W \xrightarrow{} W$. That is,
  $T(W) \subseteq W$, so that $W$ is  $T$-stable.

  Now, let $W$ be $T$-stable subspace of $V$ as a vector space. Then
  since $T(W) \subseteq W$, we have for any $w \in W$:
  \begin{equation*}
    a_0w+a_1Tw+\dots+a)nT^nw \in W
  \end{equation*}
  so that $T$ is characterized by the action  $k[x] \times W
  \xrightarrow{} W$, which makes $W$ into a $k[x]$-module.
\end{proof}

\begin{theorem}[The Submodule Criterion]\label{theorem_4.1.4}
  Let $R$ be a ring, and  $M$ an $R$-module. A nonempty subset $N
  \subseteq M$ is a \textbf{submodule} of $M$ if, and only if  $x+ry
  \in N$ for all $r \in R$ and $x,y \in N$.
\end{theorem}
\begin{proof}
  If $N$ is a submodule of $M$, then $N$ is a subgroup of $M$ so that
  $N$ is nonempty by definition. Moreover $N$ is closed under the
  action $R \times M \xrightarrow{} M$, so that $x+ry \in N$.

  Now, suppose that $N$ is nonempty, and that  $x+ry \in N$. Take
  $r=-1$ (as a group element) in $R$. Then $x-y \in R$ so that $N \leq
  M$ by the subgroup criterion. Now, taking $x=0$, then  $0+ry=ry \in
  N$, which makes $N$ into a submodule of $M$.
\end{proof}

\begin{definition}
  Let $R$ be a commutative ring with identity. An
  \textbf{$R$-algebra} is a ring $A$ with identity together with a
  ring homomorphism $f:R \xrightarrow{} A$, taking $1_R \xrightarrow{}
  1_A$ such that $f(R) \subseteq Z(A)$, where $Z(A)$ is the center of
  $A$.
\end{definition}

\begin{lemma}\label{lemma_1.4.5}
  If $A$ is an  $R$-algebra, then $A$ is an left (respectively right)
  $R$-module.
\end{lemma}
\begin{proof}
  Define the action $R \times A \xrightarrow{} A$ by $ra=f(r)a$. Then
  $A$ is made into a left $R$-module. Similarly, defining the action
  $A \times R \xrightarrow{} A$ by $ar=af(r)$ makes $A$ into a right
  $R$-module. Observe lastly that since $f(R) \subseteq Z(A)$, then
  $f(r)a=af(r)$ so that the actions $R \times A \xrightarrow{} A$ and
  $A \times R \xrightarrow{} A$ coincide at the center of $A$.
\end{proof}

\begin{definition}
  Let $A$ and  $B$ be $R$-algebras. An \textbf{$R$-algebra
  homomorphism} is a ring homomorphism $\phi:A \xrightarrow{} B$,
  taking $1_A \xrightarrow{} 1_B$ such that for every $r \in R$ and
  $a \in A$, $\phi(ra)=r\phi(a)$.
\end{definition}

\begin{example}\label{example_4.2}
  Let $R$ be a commutative ring with identity.
  \begin{enumerate}
    \item[(1)] Any ring with identity is a $\Z$-algebra.

    \item[(2)] For any ring $A$ with identity, if  $R \subseteq Z(A)$
      is a subring containing $1_A$, then  $A$ is also an $R$-algebra.
      In particular, if $A$ is commutative, then any subring  $R$ with
       $1_A$ makes  $A$ into an  $R$-algebra.

     \item[(3)] Any $R[x]$-algebra is also an $R$-algebra. Likewise,
       if  $G$ is a group, then the group ring $RG$ is also an
       $R$-algebra.

     \item[(4)] If $A$ is an $R$-algebra, then $A$ as an  $R$-module
       depends only on the ring  $f(R)$. So every algerba $A$ comes
       from a subring containing $1_A$ in the center of $A$, up to
       ring homomorphism.

     \item[(5)] Let $k$ be a field, then  $k \simeq f(k)$ by the
       first ring isomorphism theorem. Then if $A$ is a  $k$-algebra,
       it contains  $k$ in its center, and  $1_k=1_A$.
  \end{enumerate}
\end{example}
