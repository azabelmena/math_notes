\section{Definitions and Examples}

\begin{definition}
  Let $R$ be a ring. A \textit{left $R$-module} (or \textit{left
  module over $R$}) is a set $M$ together with a binary operation
  $+$ for which $M$ is an Abelian group under $+$, and a left action
  $R \times M \xrightarrow{} M$ defined by $(r,m) \xrightarrow{} rm$
  such that the following hold for all $m,n \in M$ and $r,s \in R$:
  \begin{enumerate}
    \item[(1)] $(r+s)m=rm+sm$.

    \item[(2)] $(rs)m=r(sm)$.

    \item[(3)] $r(m+n)=rm+rn$.
  \end{enumerate}
\end{definition}

\begin{definition}
  Let $R$ be a ring. A \textit{right $R$-module} (or \textit{right
  module over $R$}) is a set $M$ together with a binary operation
  $+$ for which $M$ is an Abelian group under $+$, and a right action
  $M \times R \xrightarrow{} M$ defined by $(m,r) \xrightarrow{} mr$
  such that the following hold for all $m,n \in M$ and $r,s \in R$:
  \begin{enumerate}
    \item[(1)] $m(r+s)=mr+ms$.

    \item[(2)] $m(rs)=(mr)s$.

    \item[(3)] $(m+n)r=mr+nr$.
  \end{enumerate}
\end{definition}

\begin{remark}
  We work almost exclusively with left modules over a given ring,
  and specify right modules when needed. We also note that the
  definitions for right modules are analogous to those of left
  modules.
\end{remark}

\begin{lemma}\label{lemma_9.1.1}
  Let $R$ be a commutative ring, and let $M$ be a left $R$-module.
  Then $M$ can be made into a right $R$-module via a compatible
  action.
\end{lemma}
\begin{proof}
  Observe the action $(r,m) \xrightarrow{} rm=mr$ since $R$ is
  commutative. Define then the action $(m,r) \xrightarrow{} mr$.
  Then the two actions above are compatible.
\end{proof}

\begin{definition}
  Let $R$ be a ring with identity $1$, and $M$ be a left
  $R$-module, and $N$ a right $R$-module. We call $M$ and $N$
   \textit{unital} provided that:
   \begin{enumerate}
     \item[(1)] $1m=m$ for all $m \in M$.

     \item[(2)] $n1=n$ for all $n \in N$.
   \end{enumerate}
\end{definition}

\begin{remark}
  From here on, we only state definitions in terms of left modules.
  Definitions for right modules will follow analogously.
\end{remark}

\begin{definition}
  Let $R$ be a ring and $M$ a left $R$-module. A \textit{left
  $R$-submodule} (or \textit{left submodule over $R$}) is a subgroup
  $N$ of $M$ which is closed under the action of $R$ on $M$. That
  is:
  \begin{equation*}
    rn \in N \text{ for all } r \in R \text{ and } n \in N
  \end{equation*}
\end{definition}

\begin{example}\label{example_9.1}
  \begin{enumerate}
    \item[(1)] Let $F$ be a field. Then any $F$-module is just a
      vector space over $F$.  $F$-submodules of an $F$-modules just
      correspond to subspaces of a vector space.

    \item[(2)] Every $R$-module $M$ has two submodules; $M$ itself,
      and  $(0)$, which we call the \textit{trivial submodule}.

    \item[(3)] Let $R$ be any ring. Then $R$ is a left module
      over itslef. We define the action  $R \times R \xrightarrow{}
      R$ as just being the ring mutliplication. Moreover, $R$ is
      also a right module over itself by the same reasoning.

      However, if $R$ is noncommutative, then the left and right
      module structures of $R$ over itself need not be the same.

    \item[(4)] Let $F$ be any field. Consider the set
      \begin{equation*}
        F^n=\{(a_1, \dots, a_n) : a_i \in F \text{ for all } 1 \leq
        i  \leq n\}
      \end{equation*}
      then $F^n$ is also a vector space over $F$, and hence an
      $F$-module.

    \item[(5)] Let $R$ be any ring with identity $1$, and define:
      \begin{equation*}
        R^n=\{(a_1, \dots, a_n) : a_i \in R \text{ for all } 1 \leq
        i  \leq n\}
      \end{equation*}
      Then $R^n$ is a unital left $R$-module under the action:
      \begin{align*}
        R \times R^n &  \xrightarrow{} R^n  \\
        (r,a)  & \xrightarrow{} ra=r(a_1, \dots, a_n)=(ra_1, \dots, ra_n) \\
      \end{align*}
      where $a=(a_1, \dots, a_n) \in R^n$. We call $R^n$ the
      \textit{free module over $R$} of \textit{rank} $n$.
      We can make $R^n$ a unital right $R$-module under the action:
      \begin{align*}
        R \times R^n &  \xrightarrow{} R^n  \\
        (a,r)  & \xrightarrow{} ar=(a_1, \dots, a_n)r=(a_1r, \dots, a_nr) \\
      \end{align*}

    \item[(6)] Let $R$ be a ring and $M$ an $R$-module. Let $\af
      \subseteq R$ be an ideal such that $am=0$ for all  $a \in
      \af$ and for all $m \in M$. Then we say that $M$ is
      \textit{annihilated} by $\af$. We can make  $M$ into an
      $(\faktor{R}{\af})$-module given the action:
      \begin{align*}
        \faktor{R}{\af} \times M  & \xrightarrow{} M \\
        (r+\af, m)  & \xrightarrow{} (r+\af)m=rm  \\
      \end{align*}
  \end{enumerate}
  one checks that this action is well defined, and the module
  properties follow.
\end{example}

\begin{example}[$\Z$-modules]\label{example_9.2}
  \begin{enumerate}
    \item[(1)] Let $G$ be any Abelian grooup whose operation is written
    additively. We make $G$ into a $\Z$-module through the following
    action $\Z \times G \xrightarrow{} G$:
    \begin{equation*}
      (n,g) \xrightarrow{} ng=
      \begin{cases}
        \underbrace{g+\dots+g}_{n-\text{times}},  & n > 0  \\
        0,  &  n=0  \\
        \underbrace{(-g)+\dots+(-g)}_{n-\text{times}},  & n < 0  \\
      \end{cases}
    \end{equation*}
    where $0$ is the (additive) identity of $G$. It's easy to show
    that $G$ is a $\Z$-module via this action.

    Now, let $G$ be any $\Z$-module. Then by definition $G$ is an
    Abelian group under the underlying binary operation $+$. Hence,
    there is an equivalence between $\Z$-modules and Abelian groups.
    We also observe that $\Z$-submodules of $\Z$-modules are
    equivalent to subgroups of groups by the closure condition.

    Now, if $G$ is an Abelian group, and  $g \in G$ is an element of
    finite order $|g|=m$, then $mg=0$ and the ideal $m\Z$ generated by
    $m$ annihilates the cyclic subgroup $\langle g \rangle$, generated
    by $g$. Moreover, if $|G|=m$, then  $mg=0$ for all $g \in G$ and
    the ideal $m\Z$ annihilates $G$. Hence $G$ can be made into a
    $\faktor{\Z}{m\Z}$-module by example \ref{example_9.1}(6).
    Specifically, when $m=p$ is a prime, we call the
    $\faktor{\Z}{p\Z}$-modules \textit{elementary Abelian $p$-groups}.

    \item[(2)] Observe the Kelin $4$-group  $V_4$ of order $4=2^2$
      is a $2$-dimensional vector space over the finite field
      $\F_2$.
  \end{enumerate}
\end{example}

\begin{example}[$F\left{[}x\right{]}$-modules]\label{example_9.3}
  \begin{enumerate}
    \item[(1)] Let $F$ be a field, and $V$ a vector space over $F$.
      let $T$ be a linear operator on $V$. Then $V$ can be made into an
      $F[x]$-module, via the operator $T$, using the following action
      $F[x] \times V \xrightarrow{} V$:
      \begin{align*}
        p(x)v &=  (a_nT^n+a_{n-1}T^{n-1}+\dots+a_1T+a_0)(v) \\
              &=  a_nT^n(v)+a_{n-1}T^{n-1}(v)+\dots+a_1T(v)+a_0
      \end{align*}
      where $p(x)=a_nx^n+a_{n-1}x^{n-1}+\dots+a_1x+a_0$.

      Now, since $F \subseteq F[x]$ is a subring, restricting this
      action to $F$ recovers the action of $F$ on $V$ (as an
      $F$-module). That is, this action extends $V$ as an $F$-module
      to $V$ as an $F[x]$-module.

    \item[(2)] Let $V$ and $F$ be as above. Then extending $V$ from
      an  $F$-module to $V$ as an $F[x]$-module depends on the
      choice of linear operator. Take $T=0$ the zero-operator. Then
       $V$ is made into an $F[x]$-module via the action
       \begin{equation*}
         p(x)v=a_0v
       \end{equation*}
       where $p(x)=a_nx^n+\dots+a_1x+a_0$. This is just
       multiplication by a constant in $F$, hence the operator
       $T=0$ preserves the $F$-module structure precisely.

     \item[(3)] Let $F$ be a field, and consider $F^n$. Define
       $T:F^n \xrightarrow{} F^n$ by:
       \begin{equation*}
         T(x_1, \dots, x_n)=(x_2, \dots, x_n, 0)
       \end{equation*}
       then $T$ is a linear operator on $F^n$ called the
       \textit{left shift operator}. Now, let $e_i$ be the $i$-th
       basis vector of $F^n$. Then:
       \begin{equation*}
         T^k(e_i)=
         \begin{cases}
           e_{i-k}, & i>k \\
           0, & \text{otherwise} \\
         \end{cases}
       \end{equation*}
       then for every $p(x)=a_mx^m+\dots+a_1x+a_0 \in F[x]$, the
       resulting action gives:
       \begin{equation*}
         p(x)e_n=(0, \dots, 0, a_m, a_{m-1}, \dots, a_0)
       \end{equation*}
       and we can define the action of any polynomial in $F[x]$ on
       any vector of  $V$.
  \end{enumerate}
\end{example}

\begin{theorem}\label{theorem_9.1.2}
  There is a $1--1$ correspondence between $F[x]$-modules, and
  vector spaces over $F$ together with a given linear operator.
\end{theorem}
\begin{proof}
  This is given by the action $F[x] \times V \xrightarrow{} V$.
\end{proof}

\begin{definition}
  Let $V$ be a vector space, and $T$ a linear operator on $V$. We
  call a subspace $U \subseteq V$ \textit{$T$-stable} if $T(U)
  \subseteq U$.
\end{definition}

\begin{lemma}\label{lemma_9.1.3}
  Let $V$ be a vector space and $T$ a linear operator on $V$. If $U$
  is a $T$-stable subspace of $V$, then $T^n(U) \subseteq U$ for all
  $n \in \Z^+$.
\end{lemma}
\begin{proof}
  By induction on $n$, for $n=1$ since $U$ is  $T$-stable, we have
  by definition that $T(U)=T^1(U) \subseteq U$.

  Now, suppose that $T^n(U) \subseteq U$ for all $n \geq 1$. Observe
  that $T^{n+1}(U)=T(T^n(U)) \subseteq T(U) \subseteq U$, and we are
  done.
\end{proof}
\begin{corollary}
  There is a $1--1$ correspondence between $F[x]$-submodules and
  $T$-stable subspaces of vector spaces.
\end{corollary}

\begin{example}\label{exampl_9.4}
  Consider again $F^n$ and let $T$ be the left shift operator on
  $F^n$. Take for $0 \leq k \leq n$:
  \begin{equation*}
    U_k=\{(x_1, \dots, x_n, 0, \dots, 0) : x_i \in F \text{ for all
    } 0 \leq i \leq n\}
  \end{equation*}
  then $U_k$ is a $T$-stable subspace, and hence an $F[x]$-submodule
  of $F^n$ (under the left shift operator).
\end{example}

\begin{proposition}(The Submodule Criterion)\label{proposition_9.1.4}
  Let $R$ be a ring, and let $M$ be an $R$-module. A non-empty
  subset $N \subseteq M$ is an $R$-submodule of $M$ if, and only if
  for every $r \in R$, and $x,y \in N$, $x+ry \in N$.
\end{proposition}
\begin{proof}
  Suppose that $N$ is a submodule of $M$. Then by definition, $N$ is
  non-empty, and is a subgroup of $M$ (as an Abelian group).
  Moreover, by the closure of the action of $R$ on $M$, for every
  $x,y \in N$ and $r \in R$ we have that $x+ry \in N$.

  Conversely, suppose that $N$ is non-empty, and for every $r
  \in R$ and $x,y \in N$ that $x+ry \in N$. Take $r=-1$ so $x-y \in
  N$ for every $x,y \in N$. Hence $N \leq M$. Now, taking $x=0$
  gives $ry \in N$. This makes $N$ a submodule of $M$.
\end{proof}
