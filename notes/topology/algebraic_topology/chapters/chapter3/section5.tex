%----------------------------------------------------------------------------------------
%	SECTION 1.1
%----------------------------------------------------------------------------------------

\section{Tychonoff's Theorem}

This section is entirely devoted to proving that arbitrary  (or infinite) products of compact spaces
are also compact. Unlike its finite counterpart, the proof for this is not so well behaved. It can
be achieved using the Well ordering principle, or using Zorn's lemma. We present the proof using
Zorn's lemma, which we state without proof.

\begin{lemma}[Zorn's Lemma]\label{3.5.1}
    Every strictly partially ordered set, whose simply oredered sets have upperbounds, has a maximal
    element.
\end{lemma}

\begin{lemma}\label{3.5.2}
    Let $X$ be a set, and  $\Ac$ be a collection of subsets satisfying the finite intersection
    property. There exists a collection  $\Dc$ of subsets of  $X$ such that  $\Ac \subseteq \Dc$ and
    $\Dc$ has the finite intersection property, and is maximal with respect to it.
\end{lemma}
\begin{proof}
    We construct $\Dc$ using Zorn's lemma on a set whose elements are collections of subsets of
    $X$; let us denote is as a  \textbf{superset}. Let $\Ac \subseteq 2^X$ satisfy the FIP. Define a
    partial order $\subset$ on supersets by  $\Ac \subset \Bc$ if  $\Ac \subseteq \Bc$ and  $\Ac
    \neq \Bc$. Let  $\A$ be the superset of all collections  $\Bc \subseteq 2^X$ such that  $\Ac
    \subset \Bc$ and  $\Bc$ satisfies the FIP.

    Let  $\B \subseteq \A$ be ordered by inclusion and let  $\Cc=\bigcup_{\Bc \in \B}{\Bc}$. Since
    $\B \subseteq \A$, we get  $\Ac \subset \Cc$. Now let $C_1, \dots, C_n \in \Cc$. By definition,
    there is a $\Bc_i \in \B$ such that  $C_i \in \Bc_i$ for  $1 \leq i \leq n$. Then we have the
    superset  $\{\Bc_i\}_{i=1}^n \subseteq \B$ is finite, hence there is a largest element, i.e.
    there is a $\Bc_k$ such that  $\Bc_i \subset \Bc_k$ for all  $1 \leq i \leq n$,  $i \neq k$.
    Then all  $C_i \in \Bc_k$, and since  $\Bc_k$ has the FIP, we get  $\bigcap_{i=1}^n{C_i} \neq
    \emptyset$. So $\Cc$ also has the finite intersection property, putting  $\Cc \in \A$. Also
    notice that  $\Cc$ is an upper bound of  $\B$.
\end{proof}

\begin{lemma}\label{3.5.3}
    Let $X$ be a set and let  $\Dc \subseteq 2^X$ be maximal with respect to the finite intersection
    property. Then:
    \begin{enumerate}
        \item[(1)] Any finite interesection of elements of $\Dc$ is an element of  $\Dc$.

        \item [(2)] If $A \subseteq X$ intersects every element of  $\Dc$, then  $A \in \Dc$.
    \end{enumerate}
\end{lemma}
\begin{proof}
    Let $B=\bigcap_{i=1}^n{D_i}$ where $D_i \in \Dc$ for all  $1 \leq i \leq n$. Define a collection
     $\Ec=\Dc \cup \{B\}$. Now take $\{E_j\}_{j=1}^m \subseteq \Ec$. If $E_j \neq B$ for all  $j$,
     then  $\bigcap{E_j} \neq \emptyset$, by the FIP of  $\Dc$. If  $E_j=B$ for some $1 \leq j \leq
     m$, then  $\bigcap{E_j}=E_1 \cap \dots \cap E_m \cap B=\bigcap{E_j} \cap \bigcap{D_i} \neq
     \emptyset$. Either way $\Ec$ satisfies the FIP. Now since  $\Dc \subseteq \Ec$ and  $\Dc$ is
     maximal with respect to the FIP, this makes  $\Dc=\Ec$.

     Now let  $A \subseteq X$ and define  $\Ec=\Dc \cup \{A\}$. Take $\{E_i\}_{i=1}^n \subseteq
     \Ec$. If $E_i \neq A$ for all $i$, then  $\bigcap{E_i}$ is a finite intersection consisting of
     elements of $\Dc$ and  (nonempty) intersections of elements of $\Dc$ and  $A$, so
     $\bigcap{E_i} \neq \emptyset$. On the other hand, if $E_i=A$ for some  $1 \leq i \leq n$, then
      $\bigcap{E_i}=E_1 \cap \dots \cap E_n \cap A$. Since $A$ intersects every element of  $\Dc$,
      we get  $\bigcap{E_i} \neq \emptyset$ again. HEnce $\Ec$ has the FIP, which makes  $A \in
      \Dc$.
\end{proof}

\begin{theorem}[Tychonoff's Theorem]\label{3.5.4}
    Arbitrary products of compact spaces are compact in the product topology.
\end{theorem}
\begin{proof}
    Let $X=\prod{X_\alpha}$ with $X_\alpha$ compact for all  $\alpha$. Take  $\Ac$ the collection of
    all subsets of  $X$ satisfying the finite intersection property, and consider  $\bigcap_{A \in
    \Ac}{\cl{A}}$.

    By lemma \ref{3.5.2}, there is a collection $\Dc \subseteq 2^X$ such that  $\Ac \subset \Dc$
    (refer to the partial order defined in lemma \ref{3.5.2}) and $\Dc$ is maximal with respect to
    the FIP.

    Let  $\pi_\alpha:X \rightarrow X_\alpha$ be the projection of $x \rightarrow x_\alpha$ and
    consder the collection $\{\pi_\alpha(D)\}_{D \in \Dc}$ of subsets of $X_\alpha$. Since  $\Dc$
    has the FIP,  so does  $\{\pi_\alpha(D)\}$. Now by the compactness of  $X_\alpha$, take
    $x_\alpha \in X_\alpha$ such that  $x_\alpha \in \bigcap{\cl{\pi_\alpha(D)}}$ for each $\alpha$.
    Let  $x \in X$. We wish to show  $x \in \cl{D}$.

    Let $U_\beta$ be a neighbourhood of  $x_\beta \in X_\beta$, and considet the subbasis element
    $\pi^{-1}_\beta(U_\beta)$. Since $x_\beta \in \cl{\pi_\beta(D)}$, we have $U_\beta \cap
    D=\pi_\beta(y)$, with $y \in D$. Then by lemma \ref{3.5.3}, we have that every basis element
    containing $x$ is in  $\Dc$. Then every basis element containing  $x$ intersects every $D \in
    \Dc$, thus  $x \in \cl{D}$. This shows that $X$ is compact.
\end{proof}
