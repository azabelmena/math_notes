%----------------------------------------------------------------------------------------
%	SECTION 1.1
%----------------------------------------------------------------------------------------

\section{Affine Spaces.}

\begin{definition}
    We call a subset $X \subseteq \R^n$  \textbf{affine} if for every $x,y \in
    X$, the line  $l(x,y)$ passing through $x$ and $y$ is contained in $X$.
\end{definition}

\begin{lemma}\label{3.1.1}
    Affine sets are convex.
\end{lemma}
\begin{proof}
    Note that the line $l(x,y)$ contains the segment $l[x,y]$ which is in $X$
    for every $x,y \in X$.
\end{proof}

\begin{theorem}\label{3.1.2}
    If $\{X_\alpha\}$ is a collection of affine (or convex) sets in $\R^n$,
    then the intersection of all $X_\alpha$ is affine (or convex) in $\R^n$.
\end{theorem}
\begin{proof}
    Let $X=\bigcap{X_\alpha}$ and let $x,y \in X$. let  $l(x,y)$ be the line
    passing through $x$ and $y$, then $l(x,y) \in X_\alpha$ for every $\alpha$,
    since  $x,y \in X_\alpha$  which is affine. This makes  $l(x,y) \in X$,
    which makes $X$ affine in  $\R^n$. The proof for convexity of  $X$ is the
    same except using the line segment  $l[x,y]$.
\end{proof}

\begin{definition}
    An \textbf{affine combination} of points $x_0, \dots,x_m \in \R^n$ is a
    point $x \in \R^n$ such that
    \begin{equation*}
        x=t_0x_1+\dots+t_mx_m
    \end{equation*}
    Where $\sum{t_i}=1$. A \textbf{convex combination} is an affine combination
    in which each $t_i \geq 0$ for $o \leq i \leq m$.
\end{definition}

\begin{example}\label{}
    The line $tx+(1-t)y$ is a convex combination in $\R^n$.
\end{example}

\begin{definition}
    We say a subset $X \subseteq \R^n$  \textbf{spans} an affine set $[X]$ if
    $[X]$ is the intersection of all affine subsets containing $X$. Similarly,
    we say  $X$  \textbf{spans} a convex set $[X]$ if $[X]$ is the intersection
    of all convex subsets containing $X$. We call these the affine and convex
    \textbf{hulls}, respectively.
\end{definition}

\begin{theorem}\label{3.1.3}
    If $x_0, \dots, x_m \in \R^n$, then the convex hull $[x_0, \dots, x_m]$ is
    the set of all convex combinations of $x_0, \dots, x_m$.
\end{theorem}
\begin{proof}
    Let $S$ be the set of all convex combinations of  $x_0, \dots, x_m$, then
    $[x_0, \dots, x_m] \subseteq S$. Now, let $t_j=1$ and $t_i=0$, then  $x_i
    \in S$ for all  $j$. Moreovoer, let  $\alpha=\sum{a_ix_i}$ and
    $\beta=\sum{b_ix_i}$ where $\sum{a_i}=\sum{b_i}=1$. Then for $t \in [0,1]$
    we have
    \begin{equation*}
        t\alpha+(1-t)\beta=t\sum{a_ix_i}+(1-t)\sum{b_ix_i}=\sum{(t(a_ix_i)+(1-t)b_ix_i)}
    \end{equation*}
    moreover, $t\sum{a_i}+(1-t)\sum{b_i}=1$ and $ta_i+(1-t)b_i \geq 0$ for all
    $0 \leq i \leq m$, so  $t\alpha+(1-t)\beta$ is a convex combination in $S$.

    Now, let  $X$ be any convex set containing  $\{x_0, \dots, x_m\}$. By
    induction on $m$, for  $m=0$,  $S=\{x_0\}$. Now let $m \geq 0$ and  $t_i
    \geq 0$ with  $\sum{t_i}=1$. Assume without loss of generality that $t_0
    \neq 1$. Then
    \begin{equation*}
        y=(\frac{t_1}{1-t_0})x_0+\dots+(\frac{t_m}{1-t_0})x_m \in X
    \end{equation*}
    which makes $x=t_0x_0+(1-t_0)y \in X$ This makes $S \subseteq [x_0, \dots,
    x_m]$.
\end{proof}

\begin{definition}
    We call points $x_0, \dots, x_m \in \R^n$ \textbf{affinely independent} if
    $\{x_1-x_0, \dots, x_m-x_0\}$ is linearly independent in $\R^n$ as a vector
    space. We say the points $x_0 ,\dots, x_m$ are \textbf{affinely dependent}
    if they are not affinely independent.
\end{definition}

\begin{theorem}\label{3.1.4}
    For any points $x_0, \dots, x_m \in \R^n$, the following are equivalent:
    \begin{enumerate}
        \item[(1)] $x_0, \dots, x_m$ are affinely independent.

        \item[(2)] If $s_0, \dots, s_m \in \R$ such that $\sum{s_ix_i}=0$ and
            $\sum{s_i}=0$, then $s_0=\dots=s_m=0$.

        \item[(3)] If $A$ is an affine set spanned by  $x_0, \dots, x_m$, then
            every $x \in A$ can be written as a unique affine combination of
            $x_0, \dots, x_m$.
    \end{enumerate}
\end{theorem}
\begin{proof}
    Suppose, that $x_0, \dots, x_m$ are affinely independent. Let $a_0, \dots,
    a_m \in \R$ such that $\sum{a_i}=0$ and $\sum{a_ix_i}=0$. We see that
    \begin{equation*}
        \sum{a_ix_i}=\sum{a_ix_i-0 \cdot x_0}=\sum{a_ix_i-x_0\sum{a_i}}
        =\suma_i(x_i-x_0)=0
    \end{equation*}
    Now, since $x_0, \dots, x_m$ are affinelt independent, $x_1-x_0, \dots,
    x_m-x_0$ are linearly indpendent which implies that $a_0=\dots=a_m=0$.

    Now, suppose that if $\sum{a_i}=0$ and $\sum{a_ix_i}=0$, then
    $a_0=\dots=a_m=0$. Let $A$ be an affine set spanned by $x_0, \dots, x_m$ and
    suppose there is an $x \in A$ for which $x=\sum{a_ix_i}$ and
    $x=\sum{b_ix_i}$. Then
    \begin{equation*}
        \sum{a_ix_i}=\sum{(b_ix)}
    \end{equation*}
    so that $\sum{(a_i-b_i)x_i}=0$. Notice also that
    $\sum{a_i-b_i}=\sum{a_i}-\sum{b_i}=1-1=0$, so that by hypothesis,
    $a_i-b_i=0$ for each  $i$. That is  $a_i=b_i$.

    Finally, suppose that  $A$ is an affine set spanned by  $x_0, \dots, x_m$
    for which every $x \in A$ can be written uniquely as an affine combination
    of these points. That is $x=\sum{a_ix_i}$ where $\sum{a_i}=1$. Now, if
    $m=0$, we get each  $x=a_0x_0$ which is trivially affinely independent. Now,
    suppose $m \geq 0$ and by induction on  $m$, suppose that  $x_1-x_0, \dots,
    x_m-x_0$ are linearly dependent. Then there exists $a_0, \dots, a_m \in \R$
    not all $0$ such that  $\sum{a_i(x_i-x_0)}=0$. Choose $r_j \neq 0$ then
    \begin{equation*}
        \sum{\frac{r_i}{r_j}(x_i-x_0)}=0
    \end{equation*}
    Suppose then, without loss of generality that $r_j=1$. Then  $x_j \in
    \{x_0. \dots, x_m\}$ which gives $x_j$ two affine combinations:
    \begin{align*}
        x_j     &=      1 \cdot x_j     \\
        x_j     &=      -\sum{a_ix_i}+(1+\sum{a_i})x_0      \\
    \end{align*}
    This contradicts that each $x \in A$ has a unique representation as an
    affine combination, hence $x_1-x_0, \dots, x_m-x_0$ have to be linearly
    independent, making $x_0, \dots, x_m$ affinely indpendent.
\end{proof}
\begin{corollary}
    Given points $x_0, \dots, x_m \in \R^n$, affine independence on the points
    is independent of their ordering.
\end{corollary}
\begin{corollary}
    If $A \subseteq \R^n$ is an affine set spanned by affinely independent
    points  $x_0, \dots, x_m$, then it is the translation of an $m$-dimensional
    subspace  $V$ of  $\R^n$ as a vector space.
\end{corollary}
\begin{proof}
    Let $p_0=x_0$ and $V$ subspace of  $\R^n$ as a vector space with basis
    $\{x_1-x_0, \dots, x_m-x_0\}$. If $z \in A$, then  $z=\sum{a_ix_i}$ where
    $\sum{a_i}=1$ Then
    $z=\sum{a_ix_i}+a_0x_0=\sum{a_ix_i}-\sum{a_ix_0}+(a_0+\sum{a_i})x_0=
    \sum{a_i(x_i-x_0)}+x_0 \in V+x_0$. By similar reasoning, if we have $z \in
    V+p_0$, then $z \in A$.
\end{proof}

\begin{definition}
    We say a set of points $a_1, \dots, a_k \in \R^n$ are in \textbf{general
    position} if every $n+1$ of its points are affinely independent.
\end{definition}

\begin{theorem}\label{3.1.5}
    Given $k \geq 0$,  $\R^n$ contains  $k$ points in general position.
\end{theorem}
\begin{proof}
    For $0 \leq k \leq n+1$, take the origin $0$ together with any $k-1$
    elements of a basis of $\R^n$. These points are in general position.

    Now, suppose that $k>n+1$, and choose  $r_1, \dots, r_k \in \R$ and define
    \begin{equation*}
        a_i=(r_i,r_i^2, \dots, r_i^n) \text{ for } 1 \leq i \leq k
    \end{equation*}
    Suppose additionally that the points $a_1, \dots, a_k$ are not in general
    position. Then $n+1$ of the points $a_{i_0}, \dots, a_{i_n}$ which are
    affinely dependent. Then $a_{i_1}-a_{i_0}, \dots, a_{i_n}-a_{i_0}$ are
    linearly dependent. Then there exist $s_0, \dots, s_n$, not all $0$ such
    that
    \begin{equation*}
        \sum{s_j(a_{i_j}-a_{i_0})}=0
    \end{equation*}
    Consider now, the $n \times n$ south east block,  $V^*$ of the  $(n+1)
    \times (n+1)$ Vandermonde matrix obtained from $r_{i_0}, \dots, r_{i_n}$.
    Let $\sigma=(s_0, \dots, s_m)$, then the equation above give the matrix
    equation
    \begin{equation}
        V^*\sigma^T=0
    \end{equation}
    Now, since $V^*$ is nonsingular, and each of the  $r_{i_j}$ is distinct, we
    get that $\sigma=0$, which contradicts our assumption that  $a_{i_0}, \dots,
    a_{i_n}$ are affinely independent.
\end{proof}

\begin{definition}
    Let $x_0, \dots, x_m \in \R^n$ be affinely independent and let $x \in \R^n$
    be such that  $x=\sum{t_ix_i}$. We call the $(m+1)$-tuple $(t_0,x_0, \dots,
    t_m,x_m)$ the \textbf{barycentric coordinates} fo $x$.
\end{definition}

\begin{definition}
    Let $x_0, \dots, x_m \in \R^n$ be affinely independent. We call the convex
    set spanned by each of these points, $[x_0, \dots, x_m]$ an $m$-simplex. We
    call the  $n$-simplex $[e_0, \dots, e_n]$ of $\R^{n+1}$, where $\{e_0,
    \dotsm e_m\}$ is the standard basis of $\R^N{n+1}$ the \textbf{standard}
    $m$-simplex, and we denote it $\Delta^n$.
\end{definition}

\begin{theorem}\label{3.1.6}
    If $x_0, \dots, x_m \in \R^n$ are affinely independent then each $x \in
    [x_0, \dots, x_m]$ is the unique convex combination of barycentric
    coordinates.
\end{theorem}
\begin{proof}
    Note that barycentric coordinates are unuqiue by theorem \ref{3.1.4}.
\end{proof}

\begin{definition}
    If $x_0, \dots, x_m \in \R^n$ are affinely independent, the
    \textbf{barycenter} of $[x_0, \dots, x_m]$ is the point
    $\frac{\sum{x_i}}{m+1}$ of $[x_0, \dots, x_m]$.
\end{definition}

\begin{example}\label{}
    \begin{enumerate}
        \item[(1)] $[x_0]$ is a $0$-simplex with  $x_0$ as its barycenter.

        \item[(2)] The $1$-simplex  $[x_0,x_1]$ has as its barycenter the point
            $\frac{x_0+x_1}{2}$, which is the midpoint of a closed line segment
            between $x_0$ and $x_1$.

        \item[(3)] The $2$-simplex $[x_0,x_1,x_2]$ has barycenter
            $\frac{x_0+x_1+x_2}{3}$ which is the geometric barycenter of a
            triangle.

        \item[(4)] Let $\Delta^n$ the standard  $n$-simplex. Every point $x \in
            \Delta^n$ has the form $\sum{t_ie_i}$, which is represented as
            $(t_0, \dots, t_n)$ in $\R^{n+1}$ as a vector space. Therefore the
            barycentric coordinates of any point in $\Delta^n$ are precisely its
            cartesian coordinates.
    \end{enumerate}
\end{example}

\begin{definition}
    Let $[x_0, \dots, x_m]$ be an $m$-simplex. We define the \textbf{face
    opposite} of $x_i$ to be the set
    \begin{equation*}
        [x_0, \dots \hat{x_i}, \dots x_m]=\{\sum{t_jx_j} : t_j \geq 0,
        \sum{t_j}=0, \text{ and }, t_i=0\}
    \end{equation*}
    We define a \textbf{$k$-face} of $[x_0, \dots, x_m]$ to be a $k$-simplex
    spanned by $k+1$ vertices of  $[x_0, \dots, x_n]$. We define the
    \textbf{boundry} of $[x_0, \dots, x_m]$ to be the union of all faces
    opposite $x_i$ for all  $0 \leq i \leq m$, and we write  $\partial{[x_0,
    \dots, x_m]}$.
\end{definition}

\begin{example}\label{}
    \begin{enumerate}
        \item[(1)] Note that $\partial{[e_0, \dots, e_n]}=\partial{\Delta^n}$.

        \item[(2)] Given any $m$-simplex, it has  ${m+1 \choose k+1}$ $k$-faces.
    \end{enumerate}
\end{example}

\begin{theorem}\label{3.1.7}
    Let $S=[x_0, \dots, x_n]$ be an $n$-simplex. The following are true
    \begin{enumerate}
        \item[(1)] If $u,v \in S$, then $\|u-v\| \leq \sup_{i}{\|u-x_i\|}$

        \item[(2)] $\diam{S}=\sup_{i,j}{\|x_i-x_j\|}$

        \item[(3)] If $b$ is the barycenter of $S$, then $\|b-x_i\| \leq
            \frac{n}{n+1}\diam{S}$.
    \end{enumerate}
\end{theorem}
\begin{proof}
    Let $u,v \in S$, and  $v=\sum{t_ix_i}$ where $t_i \geq 0$ and
    $\sum{t_i}=1$. Then $\|u-v\|=\|u-\sum{t_ix_i}\|=\|u\sum{t_i}-\sum{t_ix_i}\|
    \leq \sum{t_i\|u_i-x_i\|} \leq \sum{t_i\sup{\|u-x_i\|}}=\sup{\|u-x_i\|}$. It
    also follow that the second statement is true using the properties of
    least upperbounds.

    Now, let $b=\frac{x_0+\dots+x_n}{n+1}$ the barycenter of $S$. Then
    \begin{align*}
        \|b-x_i\|       &=      \|\frac{1}{n+1}\sum{x_j}-x_i\|    \\
                      &=      \|frac{1}{n+1}\sum{x_j}-\frac{1}{n+1}\sum{x_i}\|  \\
                      &=      \|\frac{1}{n+1}\sum{x_j-x_i}\|        \\
                 \leq &       \frac{1}{n+1}\sum{\|x_j-x_i\|}    \\
                 \leq &       \frac{n}{n+1}\sup{\|x_j-x_i\|}    \\
                      &=      \frac{n}{n+1}\diam{S}     \\
    \end{align*}
\end{proof}
