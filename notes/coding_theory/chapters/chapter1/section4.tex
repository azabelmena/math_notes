\section{Hilbert's Basis Theorem}

\begin{definition}
    Let $R$ be a ring . We say a sequence of ideals $\{\af_n\}$ is an
    \textbf{ascending chain} of ideals if $\af_n \subseteq \af_{n+1}$ for all
    $n \in \Z^+$. We say that the chain  $\{\af_n\}$ \textbf{stabalizes} if there
    exists some $k \geq n$, $\af_k=\af_n$.
\end{definition}

\begin{definition}
    Let $R$ be a ring. We call $A$ \textbf{Noetherian} if every ascending chain
    of ideals of $A$ stabalizes. We say that $A$ satisfies the \textbf{ascending
    chain condition} on ideals.
\end{definition}

\begin{lemma}\label{1.4.1}
    If $\af$ is an ideal of a Noetherian ring $R$, then the factor ring
    $\faktor{A}{\af}$ is also Noethrian. In particular, the image of a Noetherian
    ring under any ring homomorphism is Noetherian.
\end{lemma}
\begin{proof}
    This follows by the isomorphism theorems for ring homomorphisms.
\end{proof}

\begin{theorem}\label{1.4.2}
    The following are equivalent for any ring $R$.
    \begin{enumerate}
        \item[(1)] $R$ is Noetherian.

        \item[(2)] Every nonempty collection of ideals of $R$ contains a maximal
            element under inclusion.

        \item[(3)] Every ideal of $R$ is finitely generated.
    \end{enumerate}
\end{theorem}
\begin{proof}
    Let $R$ be Noetherian, and let  $\Sc$ an nonempty collection of ideals of
    $R$. Choose an ideal  $\af_1 \in \Sc$. If $\af_1$ is maximal, we are done. If
    not, then there is an ideal $\af_2 \in \Ac$ for which  $\af_1 \subseteq \af_2$.
    Now, if $\af_2$ is maximal, we are done. Otherwise, proceeding inductively, if
    there are no maximal ideals of $R$ in $\Sc$, then by the axiom of choice,
    construct the infinite strictly increasing chain
    \begin{equation*}
         \af_1 \subseteq \af_2 \subseteq \dots
    \end{equation*}
    of ideal of $R$. This contradicts that  $R$ is Noetherian, so  $\Sc$ must
    contain a maximal element.

    Now, suppose that any nonempty collection of ideals of  $R$ contains
    a maximal element. Let  $\Sc$ the collection of all finitely generated
    ideals of $R$, and let $\af$ be any ideal of  $R$. By hypothesis, $\Sc$ has a
    maximal element  $\af'$. Now suppose that $\af \neq \af'$, and choose an
    $x \in \com{\af}{\af'}$, then the ideal generated by $\af'$ and  $x$ is finitely
    generated, and so is in  $\Sc$; but that contradicts the maximality of  $\af'$.
    Therefore we must have  $\af=\af'$.

    Finally, suppose every ideal of $R$ is finitely genrated, and let
    $\af=(a_1, \dots, a_n)$. Let
    \begin{equation*}
        \af_1 \subseteq \af_2 \subseteq \dots
    \end{equation*}
    an ascending chain of ideals of $R$ for which
    \begin{equation*}
        \af=\bigcup_{n \in \Z^+}{\af_n}
    \end{equation*}
    Since $a_i \in \af$ for each  $1 \leq j \leq n$, we have that  $a_i \in
    \af_{i_j}$ and $i \in \Z^+$. Now, let  $m=\max{\{j_1, \dots, j_n\}}$ and
    coinsider the ideal $\af_m$. Then  $a_i \in \af_m$ for each $i$, which makes
    $\af \subseteq \af_m$. That is, $\af_n=\af_m$ for some  $n \geq m$; which
    makes $R$ Noetherian.
\end{proof}
\begin{corollary}
    Every proper ideal of a Noetherian ring is contained in a maxima ideal.
\end{corollary}
\begin{proof}
    Consider again the Noetherian ring $R$ from above. Let $\af$ be a proper ideal
    of  $R$, and consider the collection $\Sc$ of proper ideals of $R$
    containing  $\af$. Then $\Sc$ contains a maximal member $\bf$ with $\af
    \subseteq \bf$. Now, sine $\bf$ is maximal in $\Sc$ there is no other proper
    ideal $\bf'$ containing $\af$ for which $\bf \subseteq \bf' \subseteq R$.
    This by definition makes $\bf$ a maximal ideal.
\end{proof}

\begin{example}\label{example_1.9}
    \begin{enumerate}
        \item[(1)] Every principle ideal domain (PID) is Noetherian, since any
            collection of ideals has a maximal element. Moreover PIDs satisfy the
            ascending chain condition.

        \item[(2)] The rings $\Z$, $\Z[i]$, and $k[x]$ (where $k$ is a field)
            are Noetherian.

        \item[(3)] The multivariate polynomial ring $\Z[x_1, x_2, \dots]$ is not
            Noetheria, since the ideal $(x_1, x_2, \dots)$ is not finitely
            generated.
    \end{enumerate}
\end{example}

\begin{example}\label{example_1.10}
    Let $k$ be any field, and consider the collection  $\Sc=\{(x_1-a_1), \dots,
    (x_m-a_m)\}$ of ideals in $k[x_1, \dots, x_n]$. Then each $(x_i-a_i)$ is a
    maxmial member of $\Sc$, however none of them are maximal ideals. Observe
    the polynomial  $f(x_1, \dots, x_n)=(x_1-a_1) \dots (x_m-a_m)$. Then
    $(x_i-a_i) \subseteq (f)$ for each $1 \leq i \leq m$.
\end{example}

\begin{definition}
    We call a ring in which every ideal is finitely generated a
    \textbf{Noetherian ring}.
\end{definition}

\begin{theorem}[Hilbert's Basis Theorem]\label{1.4.3}
    If $R$ is a Noetherian ring, then so is the polynomial ring $R[x]$.
\end{theorem}
\begin{proof}
    Let $\af$ be an ideal of  $R[x]$, and let $L$ be the set of all leading
    coefficients of polyonimials in $\af$. Notice that since  $0 \in \af$, then  $0
    \in L$, so that $L$ is nonempty. Moreover, let $f(x)=ax^d+\dots$ and
    $g(x)=bx^e+\dots$ polynomials in $\af$ of degree  $\deg{f}=d$ and $\deg{g}=e$,
    with leading coefficients $a, b \in R$. Then for any  $r \in R$, we have the
    coefficient  $ra-b=0$, or  $ra-b$ is the leading coefficient of the
    polynomial  $rx^ef-x^dg \in \af$. In either case, we get  $ra-b \in L$. This
    makes  $L$ an ideal of  $R$. Now, since  $R$ is Noetherian  $L$ is finitely
    generated ; let $L=(a_1, \dots, a_n)$. Then for every $1 \leq  i \leq n$,
    let  $f_i \in \af$ the polynomial of degree  $\deg{f_i}=e_i$ whose leading
    coefficient is $a_i$. Take, then  $N=\max{\{e_1, \dots, e_n\}}$. Then for
    any $d \in \faktor{\Z}{N\Z}$, let $L_d$ be the set of all leading
    coefficients of polynomials in  $\af$, of degree $d$, together with $0$. Let
    $f_{di} \in \af$ a polynomial of degree $\deg{f_{di}}=d$ with leading
    coefficient $b_{di}$. We wish to show that
    \begin{equation*}
        \af=(f_1, \dots, f_n) \cup (f_{d1}, \dots f_{nd})
    \end{equation*}

    Let $\af'=(f_1, \dots, f_n) \cup (f_{d1}, \dots f_{nd})$. By construction,
    since the generators were chosen from $\af$,  $\af' \subseteq \af$. Now, if
    $\af \neq \af'$. Then there is a nonzero polynomial $f \in \af$ of minimum degree
     not contained in $\af'$  (i.e $f \notin \af'$). Let $\deg{f}=d$, and let $a$ be
     the leading coefficient of  $f$. Suppose that  $d \geq N$. Since  $a \in
     L$, $a$ is an  $R$-linear combination of the generators of  $L$; i.e.
     \begin{equation*}
         a=r_1a_1+\dots+r_na_n
     \end{equation*}
     where $r_1, \dots, r_n \in R$. Let
     \begin{equation*}
        g=r_1x^{d-e_1}f_1+\dots+r_nx^{d-e_n}f_n
     \end{equation*}
     then $g \in \af'$ and has degree $\deg{g}=d$ and leading coefficient $a$.
     Hence  $f-g \in \af'$ is of smaller degree, and by the minimality of  $f$,
     $f-g=0$, which makes  $f=g \in \af'$; a contradiction. THerefore $\af=\af'$

     Now, if $d<N$, then  $a \in L_d$, and so is an  $R$-linear combniation of
     generators of  $L_d$; that is
     \begin{equation*}
         a=r_1b_{d1}+\dots+r_nb_{dn}
     \end{equation*}
     where $r_1, \dots, r_n \in R$. Then let
     \begin{equation*}
         g=r_1f_{d1}+\dots+r_nf_{dn}
     \end{equation*}
     then $g \in \af'$ is a polynomial of degree  $\deg{g}=d$ and leading
     coefficient $a$; which gives us the above contraditction.

     Therefore, $\af=\af'$, and since $\af'$ is finitely generated,  $R[x]$ is
     Noetherian.
 \end{proof}
 \begin{corollary}
     Let $k$ be a field. Then the polynomial ring in  $n$ variables  $k[x_1,
     \dots, x_n]$ is Noetherian.
 \end{corollary}

 \begin{theorem}\label{1.4.4}
     Every algebraic set is the intersection of a finite number of
     hypersurfaces.
 \end{theorem}
 \begin{proof}
     Let $\af$ be an iddeal in the ring $k[x_1, \dots, x_n]$ for some field $k$,
     and consider the set $V(\af)$. Since $k[x_1, \dots, x_n]$ is Noetherian,
     then $\af=(f_1, \dots, f_n)$, so that
     \begin{equation*}
         V(\af)=V(f_1) \cap \dots \cap V(f_n)
     \end{equation*}
 \end{proof}

 \begin{theorem}\label{1.4.4}
     Let $\af$ be an ideal in a ring $R$, and consider the natural map $\pi:R
     \xrightarrow{} \faktor{R}{\af}$. The following are true.
     \begin{enumerate}
         \item[(1)] For every ideal $\bf'$ of $\faktor{R}{\af}$,
             $\inv{\pi}(\bf')=\bf$ is an ideal of $R$ containing $\af$.
             Moreover, for any ideal $\bf$ of $R$ containing $\af$, then
             $\pi(\bf)=\bf'$.

         \item[(2)] The ideal $\bf'$ of $\faktor{R}{\af}$ is a radical ideal if,
             and only if $\bf$ is a radical ideal in $R$.

         \item[(3)] If $\bf$ is finitely generated in $R$, then $\bf'$ is
             finitely generated in $\faktor{R}{\af}$. Moreover,
             $\faktor{R}{\af}$ is Noetherian if $R$ is Noetherian.
     \end{enumerate}
 \end{theorem}
 \begin{proof}
     Let $\bf'$ be an ideal of $\faktor{R}{\af}$. Since the natural map $\pi$ is
     onto, there is an ideal $\bf \in R$ for which $\bf=\inv{\pi}(\bf')$. Now,
     let $a,b \in \bf$, then $\pi(a),\pi(b) \in \bf'$, so that $\pi(a+b) \in
     \bf'$ and $-\pi(a) \in \bf'$. Moreover, if $a \in \bf$, and $r \in R$, then
     $r\pi(a)=\pi(ra) \in \bf'$, since $\bf'$ is an ideal. Now, since
     $\ker{\pi}=\af$, we have that $\af \subseteq \bf$. So that $\bf$ is an
     ideal containing $\af$. By similar reasoning, if $\bf$ is an ideal
     containing $\af$, then $\bf'=\pi(\bf)$ is also an ideal.

     Now, suppose that $\bf$ is a radical ideal. That is, $\bf=\sqrt{\bf}$. Since
     $\bf=\inv{\pi}(\bf')$, we have $\inv{\pi}(\bf')=\sqrt{\inv{\pi}(\bf')}$.
     Now, suppose that $\bf$ is a prime ideal, then if $ab \in \bf$, either $a
     \in \bf$ or $b \in \bf$. This implies whenever $\pi(ab) \in \bf'$, either
     $\pi(a) \in \bf'$ or $\pi(b) \in \bf'$. This makes $\bf'$ prime. Similarly,
     if $\bf'$ is prime so is $\bf$. Finally, by definition of a maximal idea,
     $\bf$ is maximal if, and only if $\bf'$ is maximal.

     Finally, suppose that $\bf$ is finitely generated, then $\bf=(a_1, \dots,
     a_n)=\inv{\pi}(\bf')$ for $a_1, \dots, a_n \in R$. Then every element of
     $\bf$ is the sum of $a_1, \dots, a_n$. That is, $b=r_1a_1+\dots+r_na_n$ for
     every $b \in \bf$, and $r_1, \dots, r_n \in R$. Now, since $b \in
     \bf=\inv{\pi}(\bf')$, then $\pi(b)=r_1\pi(a_1)+\dots+r_n\pi(a_n) \in \bf'$,
     so that $\bf'=(\pi(a_1), \dots, \pi(a_n))$. This makes $\bf'$ finitely
     generated. We can then conclude that if $R$ is Noetherian, by theorem
     \ref{1.4.2}, $\faktor{R}{\af}$ must also be Noetherian.
 \end{proof}
 \begin{corollary}
     Let $k$ be a field and $\af$ an ideal of $k[x_1, \dots, x_n]$. Any ring of
     the form $\faktor{k[x_1, \dots, x_n]}{\af}$ is a Noetherian ring.
 \end{corollary}
