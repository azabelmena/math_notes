\section{Metacategories}

\begin{definition}
    A \textbf{metagraph} is a collection of objects together with a collection
    of \textbf{arrows} between objects, and operations $\dom$ and  $\cod$ such
    that if  $f$ is an arrow, then $a=\dom{f}$ and $b=\cod{f}$ are objects. We
    write $f:a \xrightarrow{} b$, and call $a$ the  \textbf{domain}, and $b$ the
     \textbf{codomain}.

\begin{definition}
    A \textbf{metacategory} is a metagraph together with an operation $\circ$,
    called \textbf{composition}, and arrows $\Id_a$ corresponding to each object
    $a$, such that
    \begin{enumerate}
        \item[(1)] $\circ$ corresponds the pair of arrows $(g,f)$ to the arrow
            $g \circ f$, called the  \textbf{composition}, where
            $\cod{f}=\dom{g}$; i.e. if $f:a \xrightarrow{} b$, and $g:b
            \xrightarrow{} c$, then $g \circ f:a \xrightarrow{} c$.

        \item[(2)] $\dom{\Id_a}=\cod{\Id_a}=a$. Moreover, if $f:a \xrightarrow{}
            b$, and $g:b \xrightarrow{} c$, then $\Id_b \circ f=f$ and  $g \circ
            \Id_b=g$.

        \item[(3)] If $f:a \xrightarrow{} b$, $g:b \xrightarrow{} c$, and $h:c
            \xrightarrow{} d$ are arrows, then the compositions $k \circ (g
            \circ f)$ and $(k \circ g) \circ f$ are equal whenever one of them
            is defined.
    \end{enumerate}
\end{definition}

\begin{definition}
    A \textbf{diagram} is a directed graph whos vertices consists of objects of
    a metacategory, and whos edjes are arrows between those objects. We say a
    diagram \textbf{commutes} if for any pair $c,c'$ of vertices, any two paths
    from  $c$ to  $c'$ gives, by composition of the labels, equal arrows.
\end{definition}

\begin{example}\label{example_1.1}
    \begin{enumerate}
        \item[(1)] We represent arrows of metagraphs pictorally by the diagram
                 \[\begin{tikzcd}
                        a & b
                        \arrow["f", from=1-1, to=1-2]
                    \end{tikzcd}\]

        \item[(2)] The following is a diagram of the composition law of a
            metacategory
            \[\begin{tikzcd}
                    a \\ b & c
                \arrow["f"', from=1-1, to=2-1]
                \arrow["g"', from=2-1, to=2-2]
                \arrow["{g \circ f}", from=1-1, to=2-2]
            \end{tikzcd}\]
            Moreover, the associative law, and identity laws of metagategories
            are represented by diagrams as
            \[\begin{tikzcd}
                a & d && a & b \\
                b & c &&& b & c
                \arrow["f"', from=1-1, to=2-1]
                \arrow["g"', from=2-1, to=2-2]
                \arrow["h"', from=2-2, to=1-2]
                \arrow["{h \circ (g \circ f)=(h \circ g) \circ f}", from=1-1, to=1-2]
                \arrow["{h \circ g}"{pos=0.7},from=2-1, to=1-2]
                \arrow["{g \circ f}"'{pos=0.6}, from=1-1, to=2-2]
                \arrow["f", from=1-4,to=1-5]
                \arrow["f"', from=1-4,to=2-5]
                \arrow["g",from=1-5, to=2-6]
                \arrow["g"',from=2-5,to=2-6]
                \arrow["{\Id_b}"{description},from=1-5,to=2-5]
            \end{tikzcd}\]

        \item[(3)] The metacategory of sets has as objects all sets, and as
            arrows, all functions. The identity arrow is the identity function,
            and the composition law is the usual function composition.
    \end{enumerate}
\end{example}

\begin{definition}
    An \textbf{arrows-only metacategory} is a collection of arrows together with
    pairs of arrows called  \textbf{composable pairs} and an operation assigning
    the composable pair $(g,f)$ to the arrow $gf$ such that if $(hg)f$ and
    $h(gf)$ are defined, then they are equal, and moreover $hgf$ is defined
    whenever $hg$ and $gf$ are defined. We define the \textbf{identity arrow} to
    be the arrow $u$ such that  $fu=f$ and  $ug=g$, whenever defined. Moreover,
    for every arrow $g$, there are arrows  $u,u'$ such that  $u'g$ and  $gu$ are
    defined.
\end{definition}
