%----------------------------------------------------------------------------------------
%	SECTION 1.1
%----------------------------------------------------------------------------------------

\section{Upper and Loweer Limits.}

Let $\{\x_n\}$ be a sequence in  $\R$ such that for all  $M>0$, there is an  $N \in \Z^+$
for which  $n \geq N$ implies that either  $x_n \geq M$, or  $x_n \leq M$. Then we write
 $x_n \rightarrow \infty$ and  $x_n \rightarrow -\infty$, respectively.

 \begin{definition}
     Let $\{x_n\}$ be a sequence in $\R$, and let $E$ tbe the set of all extended real
     numbers $x$  such that $x_{n_k} \rightarrow x$ for some subseqence  $\{x_{n_k}\}$. Then
     $E$ contains all subsequential limits of  $\{x_n\}$, and possible  $\pm\infty$. We then
     call  $\sup{E}$ the  \textbf{upper limit} of $E$, and  $\inf{E}$ the \textbf{lower limit} of
     $E$.
 \end{definition}

 \begin{theorem}\label{4.1.1}
     Let $\{x_n\}$ be a sequence in  $\R$, and let  $E$ be the set of all extended real
     numbers  $x$, let  $s=\sup{E}$ and  $s'=\inf{E}$. Then the following hold:
         \begin{enumerate}
             \item[(1)] $s,s' \in E$.

             \item[(2)] If $x>s$, and $x'>s'$, there is an $N \in \Z^+$ such that $n \geq N$
                 implies that $x'<x_n<x$.
        \end{enumerate}
 \end{theorem}
 \begin{proof}
     We prove the theorem for the case of $s$, since it is analogous for  $s'$.
        \begin{enumerate}
            \item[(1)] If $s=\infty$, then $E$ is not bounded above, so neither is  $\{x_n\}$, and
                there is a subsequence for which $x_n \rightarrow \infty$. Now if $s \in \R$, then
                 $E$ is bounded above, and has at least one subsequential limit. Then  $s \in E$.
                 Now if  $s=-\infty$, then  $E$ contains only $-\infty$, and so by definition
                 $x_n \rightarrow -\infty$.

             \item[(2)] Suppose there is an $x>s$, such that  $x_n \geq x$ for all  $n$. Then there
                 is a  $y \in E$ such that  $y \geq x \geq s$, a contradiction of the definition of  $s$.
        \end{enumerate}
 \end{proof}

 \begin{example}
     \begin{enumerate}
         \item[(1)] Let $\{x_n\}$ be a sequence in  $\Q$, then every real number is a subsequential
             limit, and $\limsup{x_n}=\infty$ and  $\liminf{x_n}=-\infty$.

         \item[(2)] Let  $\{x_n\}=\{\frac{(-1)^n}{1+\frac{1}{n}}\}$; then $\limsup{x_n}=1$ and
             $\liminf{X_n}=-1$ as  $n \rightarrow \infty$.

         \item[(3)] For a sequence  $\{x_n\}$ in  $\R$,  $\lim{x_n}=x$ if and only if
             $\limsup{x_n}=\liminf{x_n}=x$ as  $n \rightarrow \infty$.
     \end{enumerate}
 \end{example}

 \begin{theorem}\label{3.4.2}
     If $x_n \leq y_n$, for  $n \geq N>0$, then  $\liminf{x_n} \leq \liminf{y_n}$ and
     $\limsup{x_n} \leq \limsup{y_n}$ as  $n \rightarrow \infty$.
 \end{theorem}
