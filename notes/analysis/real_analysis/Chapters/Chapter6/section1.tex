%----------------------------------------------------------------------------------------
%	SECTION 1.1
%----------------------------------------------------------------------------------------

\section{The Derivative of Real valued Functions.}

\begin{definition}
    Let $f:[a,b] \rightarrow \R$ be a realvalued function defined on  $[a,b]$. The \textbf{derivative}
    of $f$ at a point  $x \in (a,b)$ is the function $f':(a,b) \rightarrow \R$ defined by
        \begin{equation}
            f'(x)=\lim_{t \rightarrow x}{\frac{f(t)-f(x)}{t-x}}
        \end{equation}
        If $f'$ is defined at $x \in [a,b]$, then we say that $f$ is \textbf{differentiable} at $x$,
        and if $f'$ is  defined for all $x \in (a,b)$, we say that $f$ is \textbf{differentiable}
        on $(a,b)$.
\end{definition}

\begin{theorem}\label{6.1.1}
    Let $f:[a,b] \rightarrow \R$ be a realvalued function. If  $f$ is differentiable at a point
    $x \in (a,b)$, then $f$ is continuous.
\end{theorem}
\begin{proof}
    As $t \rightarrow x$, we get $|f(t)-f(x)|=|\frac{f(t)-f(x)}{t-x}||t-x| \rightarrow f'(x)0=0$,
    thus $f(t) \rightarrow f(x)$.
\end{proof}

\begin{theorem}\label{6.1.2}
    Suppose $f,g:[a,b] \rightarrow  \R$ are realvalued functiond differentiable at a point
    $x \in (a,b)$. Then $f+g$,  $fg$, and  $ \frac{f}{g}$ are differentiable at $x$, and
    as $t \rightarrow x$:
        \begin{enumerate}
            \item[(1)] $(f+g)'=f'+g'$.

            \item[(2)] $(fg)'=f'g+fg'$.

            \item[(3)] $(\frac{f}{g})'=\frac{f'g-fg'}{g^2}$, provided that $g(x) \neq 0$.
        \end{enumerate}
\end{theorem}
\begin{proof}
    $(1)$ follows directly from the definiton. Now notice that $fg(t)-fg(x)=f(t)(g(t)-g(x))+
    g(t)(f(t)+f(x))$, then dividing by  $t-x$, the result follows by definition.

    Now also notice that $ \frac{\frac{f}{g}(t)-\frac{f}{g}(x)}{t-x}=\frac{1}{g(t)g(x)}
    (g(x)\frac{f(t)-f(x)}{t-x}-f(x)\frac{g(t)-g(x)}{t-x})$, and the result again follows by
    definition.
\end{proof}

\begin{example}
    \begin{enumerate}
        \item[(1)] The derivative of constant functions are alway $0$, and the derivative of the
            identity function is always $1$.

        \item[(2)] Let  $f(x)=x^n$, for  $n \in \Z$, and  $x \neq 0$ for  $n<0$, then  $f$
            is differentiable and  $f'(x)=nx^{n-1}$.

        \item[(3)] Polynomial functions are differentiable, and so are rational functions  $ \frac{p}{q}$,
            provided that $q \neq 0$.
    \end{enumerate}
\end{example}

\begin{theorem}[Caratheodory's Theorem]\label{6.1.3}
    Let $f:[a,b] \rightarrow \R$ be a continuous realvalued function. Then  $f$ is differentiable at
    a point  $x \in (a,b)$ if and only if there is a continuous function $\phi:(a,b) \rightarrow \R$
    such that $f(t)-f(x)=\phi(t)(t-x)$; moreover, $\phi=f'$.
\end{theorem}
\begin{proof}
    Suppose $f'$ exists at  $x$, and define  $\phi:(a,b) \rightarrow \R$ by  $\phi(t)=\frac{f(t)-f(x)}{t-x}$
    when $t \neq x$, and $\phi(t)=f'(x)$ at  $t=x$. Then by the contiunity of  $f$, $\phi$ is
    continuous at  $x$, moreover, at  $t \neq x$ we see that  $f(t)-f(x)=\phi(t)(x-t)$.

    Conveersesly, sup[ose there is a  $\phi$, continuous at $x$ such that  $f(t)-f(x)=\phi(t)(x-t)$,
    then clearly, $\lim{\phi}=f'(x)$ as $t \rightarrow x$, and since  $\phi$ is continuous, $\phi(x)=f'(x)$.
\end{proof}

\begin{theorem}[The Chain Rule]\label{6.1.4}
    Suppose that $f:[a,b] \rightarrow \R$ and  $g:I \rightarrow \R$ are continuous, where  $f([a,b])
    \subseteq I \subseteq [a,b]$, and suppose that $f$ is differentiable at  $x$, and that $g$
    is differentiable at $f(x)$. Then  $g \circ g$ is differentiable at  $x$, and  $(g \circ f)'=(g' \circ f)f'$.
\end{theorem}
\begin{proof}
    We have by Caratheodory's theorem that  $f(t)-f(x)=(t-x)(f'(x)-u(t))$, and
    $g(s)-g(y)=(s-y)(g'(y)-v(s))$. Then letting $y=f(x)$, and  $s \rightarrow y$ as  $t \rightarrow x$,
    we see that  $u,v \rightarrow 0$, and we get that  $g(f(t))-g(f(x))=g'(f(t)f(t))-g'(f(x))f(x)$,
    dividing by  $t-x$ give the desired result.
\end{proof}

\begin{example}
    \begin{enumerate}
        \item[(1)] Let $f(x)=\sin{\frac{1}{x}}$ at $x \neq 0$, and  $f(x)=0$ at  $x=0$. We
            have at $x \neq 0$, that  $f'(x)=\sin{\frac{1}{x}}-\frac{1}{x}\cos{\frac{1}{x}}$, but
            at $x=0$, we must appeal to the definition, and we get  $f(t)=\sin{\frac{1}{t}}$, which
            diverges at $t \rightarrow 0$, thus  $f'(0)$ does not exist.

        \item[(2)] Let $f(x)=x^2\sin{\frac{1}{x}}$ at $x \neq 0$, and  $f(x)=0$ at  $x=0$. For
            $x \neq 0$, we get  $f'(x)=2x\sin{\frac{1}{x}}-\cos{\frac{1}{x}}$, and at $x = 0$,
            we notice that  $|t\sin{\frac{1}{t}}| \leq |t|$, so by the sandwhich theorem, $f'(0)=0$
            as  $t \rightarrow 0$.
    \end{enumerate}
\end{example}
