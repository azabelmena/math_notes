\section{Properties of Lebesgue Measurable Functions}

\begin{lemma}\label{9.1.1}
    Let $f$ a real-valued function on a measurable domain $E$. Then the following
    statements are equivalent for any $c \in \R$.
    \begin{enumerate}
        \item[(1)] $\inv{f}((c,\infty))=\{x \in E : f(x)>c\}$ is measurable.

        \item[(2)] $\inv{f}([c,\infty))=\{x \in E : f(x)\geq c\}$ is measurable.

        \item[(3)] $\inv{f}((-\infty,c))=\{x \in E : f(x)<c\}$ is measurable.

        \item[(4)] $\inv{f}((-\infty,c])=\{x \in E : f(x) \leq c\}$ is measurable.
    \end{enumerate}
\end{lemma}
\begin{proof}
    It suffices to show that statements (1) and (2) are equivalent. Suppose that
    (1) holds. Notice that
    \begin{equation*}
        \inv{f}([c,\infty))=\bigcap_{k=1}^\infty{\{x \in E : f(x)>c-\frac{1}{k}\}}
    \end{equation*}
    which is a countable intersection of measurable sets, which makes
    $\inv{f}([c,\infty))$ measurable. Conversely, if (2) holds, then we have
    \begin{equation*}
        \inv{f}((c,\infty))=\bigcup_{k=1}^\infty{\{x \in E : f(x) \geq
        c-\frac{1}{k}\}}
    \end{equation*}
    which is a countable union of measurable sets, which makes
    $\inv{f}((c,\infty))$ measurable.
\end{proof}
\begin{corollary}
    If $f$ is an extended real-valued function, then  $\inv{f}(\infty)$ is
    measurable.
\end{corollary}
\begin{proof}
    Notice that
    \begin{equation*}
        \inv{f}(\infty)=\bigcap_{k \in \Z^+}{\{x \in E : f(x)>k\}}
    \end{equation*}
    so that $\inv{f}(\infty)$ is measurable by the above lemma.
\end{proof}

\begin{definition}
    We call a real-valued function $f$ on a measurable domain $E$
    \textbf{Lebesgue measurable} if for any $c \in \R$ and $I$ a bounded
    interval of the form  $(c,\infty)$, $(-\infty,c)$, $[c,\infty)$, or
    $(-\infty,c]$, then $\inv{f}(I)$ is measurable.
\end{definition}

\begin{lemma}\label{9.1.2}
    Let $f$ be a real-valued function defined on a measurable domain  $E$. Then
    $f$ is Lebesgue measurable if, and only if for any  $U$ open in $\R$,
    $\inv{f}(U)$ is measurable.
\end{lemma}
\begin{proof}
    Let $U$ be open in  $\R$, and $\inv{f}(U)$ be measurable. Since $(c,\infty)$
    is open, then $\inv{f}((c,\infyy))$ is measurable by hypothesis, which makes
    $f$ measurable.

    Conversely, suppose that $f$ is measurable, and let  $U$ be open in  $\R$.
    Then
    \begin{equation*}
        U=\bigcup{I_k}
    \end{equation*}
    where $\{I_k\}$ is a countable collection of open bounded intervals.
    Moreover, we have each $I_k=A_k \cap B_k$ where  $A_k=(a_k,\infty)$ and
    $B_k=(-\infty,b_k)$. Since $f$ is measurable, then  $\inv{f}(A_k)$ and
    $\inv{f}(B_k)$ are measurable. Hence
    \begin{equation*}
        \inv{f}(U)=\inv{f}\Big{(} \bigcup{(A_k \cap B_k)} \Big{)}=
        \bigcup{\inv{f}(A_k) \cap \inv{f}(B_k)}
    \end{equation*}
    which is a countable union of measurable sets, which makes $\inv{f}(U)$
    measurable.
\end{proof}
\begin{corollary}
    If $f$ is continuous on its domain, then it is measurable.
\end{corollary}
\begin{proof}
    Let $f$ be continuous on $E$, and let $U$ be open. Then  $\inv{f}(U)$ is
    open and $\inv{f}(U)=E \cap V$ with some $V$ open in  $\R$. Now, $E$ and $V$
    are measurable sets, which makes  $\inv{f}(U)$ measurable.
\end{proof}

\begin{lemma}\label{9.1.3}
    Monotone functions defined on an interval are measurable.
\end{lemma}

\begin{lemma}\label{9.1.4}
    Let $f$ and $g$ be extended real-valued functions on a measurable domain $E$.
    Then the following are true.
    \begin{enumerate}
        \item[(1)] If $f$ is Lebesgue measurable, and $f=g$ almost everywhere on
            $E$, then $g$ is Lebesgue measurable.

        \item[(2)] If $D$ is a measurable subset of  $E$, then  $f$ is
            Lebesgue measurable if, and  only if $f$ is Lebesgue measurable when
            restricted to $D$ and to  $\com{E}{D}$; i.e. both $f|_D$ and
            $f|_{\com{E}{D}}$ are Lebesgue measurable.
    \end{enumerate}
\end{lemma}
\begin{proof}
    Suppose that $f$ is measurable and that $f=g$ almost everywhere on $E$. Let
    $A=\{x\ in E : f(x) \neq g(x)\}$. Then notice that
    \begin{equation*}
        \inv{g}((c,\infty))=\{x \in A : g(x)>c\} \cup ((\{x \in E : f(x)>c\})
        \cap (\com{E}{A}))
    \end{equation*}
    Now, since $f=g$ almost everywhere on $E$, then  $m(A)=0$, so that the set
    $\{x \in A : g(x)>c\}$ is measurable. Now, $\inv{f}((c,\infty))$ is
    measurable since $f$ is measurable, which makes  $\inv{g}((c,\infty))$
    measurable.

    Now, let $D \subseteq E$ a measurable subset. Observe that for every  $c \in
    \R$
    \begin{equation*}
        \inv{f}((c,\infty))=\{x \in D : f(x)>c\} \cup \{x \in \com{E}{D} :
        f(x)>c\}
    \end{equation*}
    hence the equivalence is proved.
\end{proof}

\begin{theorem}\label{9.1.5}
    Let $f$ and  $g$ be Lebesgue measruable functions defined on a measurable
    domain $E$, and finite almost everywhere on  $E$. Then for all  $\a, \b \in
    \R$
    \begin{enumerate}
        \item[(1)] $\a{f}+\b{g}$ is Lebesgue measurable.

        \item[(2)] $fg$ is Lebesgue measurable.
    \end{enumerate}
\end{theorem}
\begin{proof}
    Suppose, without loss of generality, that $f$ and  $g$ are finite on all of
     $E$. Now, for $\a=0$, we have that $\a{f}=0$ is measurable. Now, if $\a
     \neq 0$, observe that
     \begin{equation*}
         \inv{(\a{f})}((c,\infty))= \begin{cases}
                                    \{x \in E : f(x)>\frac{c}{\a}\}, \a>0   \\
                                    \{x \in E : f(x)<\frac{c}{\a}\}, \a<0   \\
                                \end{cases}
     \end{equation*}
     Now, since $f$ is measurable, both these sets are measurable. This makes
     $\a{f}$ measurable. Now, consider the function $f+g$. If  $f(x)+g(x)<c$ for
     all $x \in E$, then  $f(x)<c-g(x)$. Now, by the density of $\Q$ in  $\R$,
     there is a rational  $q \in \Q$ for which  $f(x)<q<c-g(x)$. Thus
     \begin{equation*}
         \inv{(f+g)}((c,\infty))=\bigcup_{q \in \Q}{(\inv{g}((-\infty,c-q)) \cap
         \inv{f}(-\infty,q))}
     \end{equation*}
     By the countability of $\Q$ we have a countable union of measurable sets,
     which makes $\inv{(f+g)}((c,\infty))$ measurable. Hence $f+g$ is
     measurable.

     Lastly, notice that $fg=\frac{1}{2}((f+g)^2-f^2-g^2)$. It suffices to show
     then that $f^2$ is measurable. For  $c \geq 0$, we have that
     \begin{equation*}
         \inv{(f^2)}((c,\infty))=\{x \in E : f(x)>\sqrt{c}\} \cup \{x \in E :
         f(x)<-\sqrt{c}\}
     \end{equation*}
     and for $c<0$, we have  $\inv{(f^2)}((c,\infty))=\{x \in E : f^2(x)>c\}=E$.
     In either case we get measurable sets, which makes $f^2$ measurable.
\end{proof}

\begin{example}\label{example_9.1}
    Consider the strictly increasing function $\psi$ on  $[0,1]$ defined by
    $\psi(x)=\phi(x)+x$, where $\phi$ is the Cantor-Lebesgue function. Then
    there exists a measurable subset $A$ of $[0,1]$ for which $\psi(A)$ is
    nonmeasurale. Now, extend $\psi$ to the continuous function  $\Psi:\R
    \xrightarrow{} \R$. Then $\inv{\Psi}$ is continuous, and hence measurable.
    Now, consider the function
    \begin{equation*}
        \chi_A(x)=  \begin{cases}
                     1, \text{ if } x \in A \\
                     0, \text{ if } x \notin A \\
                 \end{cases}
    \end{equation*}
    and take $f(x)=\chi_A \circ \inv{\Psi}(x)$. We claim that $f$ is not a
    measurable function. Take  $I$ an open interval with  $1 \in I$ and  $0
    \notin I$. Then $\inv{f}(I)=\Psi(\inv{\chi_A}(I))=\Psi(A)=\psi(A)$ which is
    not measurable. Hence $f$ fails to be a measurable function. In general,
    compostions of Lebesgue measurable functions need not be Lebesgue measurable.
\end{example}

\begin{theorem}\label{9.1.6}
    Let $g$ be a lebesgue measurable function on a domain  $E$, and let  $f$ be a
    continuous real-valued function on all  $\R$. Then  $f \circ g$ is Lebesgue
    measurable on  $E$.
\end{theorem}
\begin{proof}
    Let $U$ be open in  $\R$. Then  $\inv{(f \circ g)}(U)=\inv{g}(\inv{f}(U))$.
    Now, since $f$ is continuous, we get  $f(U)$ is open in $\R$, and since  $g$
    is measurable, this makes  $\inv{g}(\inv{f}(U))$ measurable by lemma
    \ref{9.1.2}. Therefore $f \circ g$ is Lebesgue measurable.
\end{proof}
\begin{corollary}
    $|f|^p$ is Lebesgue measurable for any  $p>0$.
\end{corollary}

\begin{lemma}\label{9.1.7}
    Let $\{f_k\}_{k=1}^n$ be a finite sequence of Lebesgue measurable functions
    on a common measurable domain $E$. Then the functions
    \begin{align*}
        \bar{f}(x)  &=  \max{\{f_1(x), \dots, f_n(x)\}} \\
        \bbar{f}(x)  &=  \min{\{f_1(x), \dots, f_n(x)\}} \\
    \end{align*}
    are Lebesgue measurable.
\end{lemma}
