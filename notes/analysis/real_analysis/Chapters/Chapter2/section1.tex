%----------------------------------------------------------------------------------------
%	SECTION 1.1
%----------------------------------------------------------------------------------------

\section{Finite, Countable, and Uncountable Sets}

\begin{definition}
    Let $A$ be a set, and let  $E \subseteq \N$. We say that  $A$ is \textbf{finite} if there
    exists a 1-1 mapping of $A$ ont $E$, we say $A$ is \textbf{countable} if  $E=\N$, and
    we say  $A$ is \textbf{atmost countable} if $A$ is either finite or countable.
\end{definition}

\begin{example}
    The set of all integers $\Z$ is countable. Take  $f:\N \rightarrow \Z$ such that
    $f(n)=2$ if  $n$ is even, and  $f(n)=-n$ if  $n$ is odd.
\end{example}

\begin{definition}
    Let $A$ be a set, and let  $E \subseteq \N$. A \textbf{sequence} in  $A$ is a mapping
    $f:E \rightarrow A$ such that $f(n)=x_n$, for  $x_n \in A$. We call the values of  $f$
    \textbf{terms} of the sequence. We denote sequences by  $\{x_n\}_{i=1}^{n}$, and
    when  $E=\N$, we denote them simply by  $\{x_n\}$.
\end{definition}

\begin{theorem}\label{2.1.1}
    Every infinite subset of a countable set is countable.
\end{theorem}
\begin{proof}
    Let $A$ be countable, and let  $E \subseteq A$ be infinite. Arrange the elements of  $A$
    into a sequence $\{x_n\}$, and construct a sequence  $\{n_k\}$ such that  $n_1$ is the
    least term for which  $\{x_{n_k}\} \in E$, and  $n_k$ is the least term greater than
    $n_{k-1}$ for which  $x_{n_k} \in E$. Let $f(k)=x_{n_k}$, and we get a 1-1 mapping
    of $\N$ onto  $E$.
\end{proof}

\begin{theorem}\label{2.1.2}
    Let $\{E_n\}$ be a sequence of countable sets. Then  $S=\bigcup{E_n}$ is also countable.
\end{theorem}
\begin{proof}
    Arrange every set $E_n$ in a sequence  $\{x_{nk}\}$, and consider the infinite array
    $(x_{ij})$, in which the elements of  $E_n$ form the  $n$-th row. Then  $(x_{ij})$ contains
    all the elements of  $S$, and we can arrange them is a sequence
        \begin{equation*}
            x_{11}, (x_{21},x_{12}), (x_{31},x_{22},x_{13}), \dots
        \end{equation*}
    Moreover, if $E_j \cap E_j \neq \emptyset$, for  $i \neq j$, then the elements of  $E_i \cap E_j$
    appear more than once in the sequence of  $S$; so taking  $T \subseteq \N$, we get
    a 1-1 mapping of  $T$ onto  $S$, hence  $S$ is atmost countable, and since $E_i \subseteq S$
    for  $i \in \N$, is infinite, by theorem \ref{2.1.1},  $S$ is infinite, thus  $S$ is
    countable.
\end{proof}

\begin{figure}
    \centering
    \includegraphics[scale = 0.05]{Figures/diagonalizationArray.png}
    \caption{The infinite array $(x_{ij})$}
    \label{fig_2.1}
\end{figure}

\begin{corollary}
    Let $A$ be atmost countable, and suppose for all  $\alpha \in A$ that the sets
    $B_{\alpha}$ are atmost countable. Then
        \begin{equation*}
            T=\bigcup_{\alpha \in A}{B_{\alpha}}
        \end{equation*}
    is atmost countable.
\end{corollary}

\begin{theorem}\label{2.1.3}
    Let $A$ be countable, and let  $B_n$ be the set of all  $n$-tuples  $(a_1, \dots, a_n)$
    such that $a_i \in A$ for  $1 \leq i \leq n$. Then  $B_n$ is countable.
\end{theorem}
\begin{proof}
    By induction on $n$, we have that  $B_1=A$, which is countable. Now suppose that
    $B_n$ is countable, and consider  $B_{n+1}$ whose elements are of the form  $(b,a)$
    where  $b \in B_n$ and  $a \in A$. Fixing  $b$, we get a 1-1 correspondence between the
    elements of  $B_{n+1}$ and  $A$; therefore  $B$ is countable.
\end{proof}

\begin{corollary}
    $\Q$ is countable.
\end{corollary}
\begin{proof}
    For every rational $\frac{p}{q} \in \Q$, represent $\frac{p}{q}$ as $(p,q)$. Then
    the countability of $\Q$ follows from theorem \ref{2.1.3}.
\end{proof}

\begin{theorem}\label{2.1.4}
    Let $A$ be the set of all sequences of $0$ and  $1$; then  $A$ is uncountable.
\end{theorem}
\begin{proof}
    Let $E \usbseteq A$ be countable, and let $E$ consist of all the sequences  of $0$ and
    $1$,  $s_1,s_2,s_3, \dots$. Construct the sequence $s$ such that if the  $n$-th term of
    the sequence  $s_i$ is  $0$, then the  $n$-th term of  $s$ is  $1$, and vice versa,
    for $i \in \Z^+$. Then the sequence  $s$ differs from the sequence  $s_i$ at atleast
    one place; thus  $s \notin E$, but  $s \in A$. Therefore  $E \subset A$, which establishes
    the uncountablitiy of $A$.
\end{proof}
