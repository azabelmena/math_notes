\section{Absolute Continuity}

\begin{definition}
    A real-valued function $f$ on a closed bounded interval  $[a,b]$ is said to
    be absolutely continuous on $[a,b]$ provided for each $\e>0$, there is a
    $\d>0$, and a finite disjoint collection  $\{(c_k,d_k)\}_{k=1}^n$ of open
    subintervals of $(a,b)$ for which
    \begin{equation*}
        \sum_{k=1}^n{|f(d_k)-f(c_k)|}<\e \text{ whenever }
        \sum_{k=1}^n{d_k-c_k}<\d
    \end{equation*}
\end{definition}

\begin{lemma}\label{11.3.1}
    Absolutely continuous functions are continuous.
\end{lemma}
\begin{proof}
    Exercise
\end{proof}

\begin{example}\label{example_11.2}
    The Cantor-Lebesgue function $\phi$ defined on $[0,1]$ is continuous on
    $[0,1]$, but not absolutely continous on $[0,1]$. Let $n \in \Z^+$ and
    consider the disjoint collection  $\{[c_k,d_k]\}_{k=1}^n$ of $2^n$ closed
    subintervals of  $[0,1]$ each with $l([c_k,d_k])=\frac{1}{3^n}$. Then $\phi$
    is constant on the complement of each $[c_k,d_k]$ in $[0,1]$, and since
    $\phi$ is increasing, we have $\phi(1)-\phi(0)=1$. Observe, then that
    \begin{equation*}
        \sum_{k=1}^n{d_k-c_k}=\frac{2^n}{3^n} \text{ where }
        \sum_{k=1}^n{\phi(d_k)-\phi(c_k)}=1
    \end{equation*}
\end{example}

\begin{lemma}\label{11.3.2}
    Lipschitz continuous functions on a closed bounded interval are absolutely
    continuous.
\end{lemma}
\begin{proof}
    Let $f$ be Lipschitz on the closed bounded interval  $[a,b]$, and let $c>0$
    be its Lipschitz constant; that is for any  $u,v \in [a,b]$
    \begin{equation*}
        |f(u)-f(v)| \leq c|u-v|
    \end{equation*}
    Now, let $\e>0$ and take $\d=\frac{\e}{c}$. Then
    \begin{equation*}
        |f(u)-f(v)|<\e \text{ whenever } |u-v|<\d
    \end{equation*}
    This makes $f$ absolutely continuous for $n=1$ and the collection $(u,v)$ in
    $[a,b]$ for any $u,v \in [a,b]$.
\end{proof}

\begin{example}\label{example_11.3}
    The function $f(x)=\sqrt{x}$ on $[0,1]$ is absolutely continuous, but not
    Lipschitz continuous.
\end{example}

\begin{theorem}\label{11.3.3}
    Let $f$ be absolutely continuous on a closed bounded interval  $[a,b]$. Then
    $f$ is the difference of absolutely continuous increasing functions on
    $[a,b]$. In particular, $f$ is a function of bounded variation.
\end{theorem}
\begin{proof}
    We first show that $f$ is a function of bounded variation. Let  $\e=1$ and
    take $\d>0$ so that $f$ is absolutely continuous on $[a,b]$. Let $P$ be a
    partitiono of $[a,b]$ into $N>0$ closed subintervals
    $\{[c_k,d_k]\}_{k=1}^n$, each of lenght $l([c_k,d_k])<\d$. Since
    $d_k-c_k<\d$ impleis that  $|f(c_k)-f(d_k)|<1$, we get $V(f|_{[c_k,d_k]})
    \leq 1$ for all $1 \leq k \leq n$. Moreover
    \begin{equation*}
        V(f)=\sum_{k=1}^N{V(f|_{[c_k,d_k]})} \leq N
    \end{equation*}
    so that $f$ is a function of bounded variation. Now, by lemma \ref{11.2.1},
    since $f$ is of bounded variation, it can be expressed as a difference of
    the increasing functions $f(x)-V(f)(x)$ and $V(f)(x)$, where
    $V(f)(x)=V(f|_{[a,x]})$. Therefore, it suffices to show that $V(f)(x)$ is
    absolutely continuous.

    Let $\e>0$ and choose  $\d=\frac{\e}{2}$. Let $P_k$ be a partition of
    $[c_k,d_k]$ for all $1 \leq k \leq n$. Then by the choice of $\d$, and the
    absolute continuity of $f$ on  $[a,b]$,
    \begin{equation*}
        \sum_{k=1}^n{v(f|_{[c_k,d_k]},P_k)}<\frac{\e}{2}
    \end{equation*}
    so that, taking least upper bounds,
    \begin{equation*}
        \sum_{k=1}^n{V(f|_{[c_k,d_k]})} \leq \frac{\e}{2}<\e
    \end{equation*}
    Notice also that $V(f|_{[c_k,d_k]})=V(f|_{a,d_k})-V(f|_{a,c_k})=
    V(f)(d_k)-V(f)(c_k)$. Therefore, we have
    \begin{equation*}
        \sum_{k=1}^n{V(f)(d_k)-V(f)(c_k)}<\e \text{ whenever }
        \sum_{k=1}^n{d_k-c_k}<\d
    \end{equation*}
    This makes $V(f)(x)$ absolutely continuous.
\end{proof}

\begin{theorem}\label{11.3.4}
    Let $f$ be a continuous function on a closed bounded interval $[a,b]$. Then
    $f$ is absolutely continuous if, and only if the collection of divided
    difference functions  $\{\Diff_h{f}\}_{0<h \leq 1}$ is uniformly integrable
    on $[a,b]$.
\end{theorem}
\begin{proof}
    Suppose first that $\{\Diff_h{f}\}$ is uniformly integrable on $[a,b]$,
    where $0<h \leq 1$. Let  $\e>0$ and take $\d>0$ for which
    \begin{equation*}
        \int_E{|\Diff_h{f(x)}| \ dm}<\frac{\e}{2} \text{ if } m(E)<\d \text{ and }
        0<h \leq 1
    \end{equation*}
    Let $\{(c_k,d_k)\}_{k=1}^n$ a disjoint collection of open subintervals of
    $(a,b)$ for which $\sum_{k=1}^n{d_k-c_k}<\d$. Then for $0<h \leq 1$, and  $1
    \leq k \leq n$,
    \begin{equation*}
        \Av_h{f(d_k)}-\Av_h{f(c_k)}=\int_{d_k}^{c_k}{\Diff_h{f(x)} \ dx}
    \end{equation*}
    therefore,
    \begin{equation*}
        \sum_{k=1}^n{|\Av_h{f(d_k)}-\Av_h{f(c_k)}|} \leq
        \sum_{k=1}^n{\int_{d_k}^{c_k}{|\Diff_h{f(x)}| \ dx}} \leq
        \int_E{|\Diff_h{f(x)}| \ dm}
    \end{equation*}
    where $E=\bigcup_{k=1}^n{(c_k,d_k)}$ has measure $m(E)<\d$. Therefore,
    \begin{equation*}
        \sum_{k=1}^n{|\Av_h{f(d_k)}-\Av_h{f(c_k)}|}<\frac{\e}{2} \text{ for all
        } 0<h \leq 1
    \end{equation*}
    taking $h \xrightarrow{} 0+$, we obtain
    \begin{equation*}
        \sum_{k=1}^n{|f(d_k)-f(c_k)|} \leq \frac{\e}{2}<\e
    \end{equation*}
    which makes $f$ absolutely continuous on $[a,b]$.

    Conversely, suppose now that $f$ is absolutely continuous, moreover, by
    theorem \ref{11.3.3}, suppose that $f$ is increasing, so that
    $\Diff_h{f(x)} \geq 0$ for all $0<h \leq 1$. Take  $\e>0$ and $\d>0$, and
    let $E$ be a measurable set for which  $m(E)<\d$. Then, by outer
    approximation (theorem \ref{8.4.1}), there is a $G_\d$ set  $G$, containing
     $E$ with  $m(\com{G}{E})=0$. However, $G$ is the intersection of a
     descending collection of open sets, each of which is the union of an
     ascending collection of open sets, each of which is the union of finite
     disjoint collection of open intervals; i.e. $G$ is the intersection of
     finite disjoint unions of open intervals. By the continuity of the Lebegue
     integral, choose $\d>0$ in response to  $\frac{\e}{2}$ and let
     $\{(c_k,d_k)\}_{k=1}^d$ a disjoint collection of open subinterval of
     $(a,b)$ with $E=\bigcup_{k=1}^n{(c_k,d_k)}$. Then by the absolute
     continuouty of $f$ on  $[a,b]$, and a change of variables and cancellation
     in the Riemann,
     \begin{equation*}
         \int_u^v{\Diff_h{f(x)} \ dx}=\frac{1}{h}\int_0^h{g(x) \ dx}
     \end{equation*}
     where $g(x)=f(v+x)-f(u+x)$ for $0 \leq x \leq 1$ and $a \leq u<v \leq b$.
     Then
     \begin{equation*}
         \int_E{\Diff_h{f(x)} \ dm}=\frac{1}{h}\int_0^h{g(x) \ dx} \text{ with }
         E=\bigcup_{k=1}^n{(c_k,d_k)} \text{ and }
         g(x)=\sum_{k=1}^n{f(d_k+x)-f(c_k)+x}
     \end{equation*}
     for all $0 \leq x \leq 1$. Thus, if $\sum_{k=1}^n{d_k-c_k}<\d$, then for $0
     \leq x \leq 1$,
     \begin{equation*}
        \sum_{k=}^n{(d_k+x)-(c_k+x)}<\d
     \end{equation*}
     and hence $g(t)<\frac{\e}{2}$. This makes
     \begin{equation*}
         \int_E{\Diff_h{f(x)} \ dm}<\frac{\e}{2}
     \end{equation*}
     which makes the collection $\{\Diff_h{f}\}$ uniformly integrable, where
     $0<h \leq 1$.
\end{proof}
