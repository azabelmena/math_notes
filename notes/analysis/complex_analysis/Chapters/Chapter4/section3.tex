\section{Analytic Functions}

\begin{definition}
    Let $f:U \xrightarrow{} \C$ be a complex valued function on an open set $U$
    of  $\C$. We call $f$ \textbf{analytic} at $z_0$ if $f$ has a power series
    representation converging on the open ball  $B(z_0,R)$, with $R>0$; that is,
    \begin{equation*}
        f(z)=\sum_{n=0}^\infty{a_n(z-z_0)^n} \text{ for all } |z-z_0|<R
    \end{equation*}
    and $R$ is the radius of convergence of the series. We call $f$
    \textbf{analytic} on $U$ if it is analytic at ever point of  $U$. We call
    $f$  \textbf{entire} if it is analytic on all $\C$.
\end{definition}

\begin{theorem}[Leibniz's Rule]\label{theorem_4.3.1}
    Let $\phi:[a,b] \times [c,d] \xrightarrow{} \C$ be a continuous complex
    valued function, and define $g:[c,d] \xrightarrow{} \C$ by
    \begin{equation*}
        g(t)=\int_a^b{\phi(s,t) \ ds}
    \end{equation*}
    Then $g$ is continous, moreover if  $D_t{\phi}$ exists and is continuous on
    $[a,b] \times [c,d]$, then $g$ is of class  $C^1$ and
    \begin{equation*}
        g'(t)=\int_a^b{D_t{\phi(s,t)} \ ds}
    \end{equation*}
\end{theorem}
\begin{proof}
    That $\phi$ is continuous implies that  $g$ is continuous. Now, fix
    $t_0 \in [c,d]$ and let $\e>0$, and let  $\phi_t=D_t{\phi}$. Since $\phi_t$
    is continuous it is uniformly continuous on $[a,b] \times [c,d]$, thus,
    there exists a $\d>0$ for which  $|\phi_t(s',t')-\phi_t(s,t)|<\e$ whenever
    $(s-s')^2+(t-t')^2<\d^2$. In particular, we have
    $|\phi_t(s,t)-\phi(s,t_0)|<\e$ whenever $|t-t_0|<\d$ and $a \leq s \lea b$.
    Then we get
    \begin{equation*}
        \Big{|} \int_{t_0}^t{\phi_t(s,\t)-\phi_t(s,t_0) \ dt} \Big{|}<\e|t-t_0|
    \end{equation*}
    whenver $|t-t_0|<\d$, for $s$ fixed. Now, let
    \begin{equation*}
        \Phi(t)=\phi(s,t)-\phi_t(s,t_0)
    \end{equation*}
    then $\Phi$ is a primitive of  $\phi_t(s,t)-\phi_t(s,t_0)$. Thus, by the
    fundamental theorem of calculus, it follows that
    \begin{equation*}
        |\phi(s,t)-\phi(s,t_0)-(t-t_0)\phi_t(s,t_0)|<\e|t-t_0|
    \end{equation*}
    for every $s \in [a,b]$, when $|t-t_0|<\e$. This gives
    \begin{equation*}
        \Big{|} \frac{g(t)-g(t_0)}{t-t_0}-\int_a^b{\phi_t(s,t_0) \ ds} \Big{|}
        \leq \e(b-a)
    \end{equation*}
    whenever $0<|t-t_0|<\d$. Since $g$ is differentiable and  $D_t(\phi)$ is
    continuous, we get that $g'$ is continuous.
\end{proof}

\begin{example}\label{example_4.2}
    Let $\phi(s,t)=\frac{e^{is}}{e^{is}-tz} \ ds$ for $0 \leq t \leq 1$ and  $0
    \leq s \leq 2\pi$. Let  $|z|<1$. Then  $\phi$ is of class $C^1$. Then
    \begin{equation*}
        g(t)=\int_0^{2\pi}{\frac{e^{is}}{e^{is}-tz} \ ds}
    \end{equation*}
    Then $g(0)=2\pi$, and now
    \begin{equation*}
        g'(t)=\int_0^{2\pi}{\frac{ze^{is}}{(e^{is}-tz)^2} \ ds}
    \end{equation*}
    with $t$ fixed. Then
    \begin{equation*}
        \Phi(s)=\frac{iz}{e^{is}-tz}
    \end{equation*}
    has as derivative
    \begin{equation*}
        \Phi'(s)=\frac{ze^{is}}{(e^{is}-tz)^2}
    \end{equation*}
    hence $g'(z)=\Phi(2\pi)-\Phi(0)=0$ making $g$ constant. Hence  $g(1)=2\pi$
    and we get
    \begin{equation*}
        \int_0^{2\pi}{\frac{e^{is}}{e^{is}-tz} \ ds}=2\pi
    \end{equation*}
    for all $|z|<1$.
\end{example}

\begin{lemma}\label{4.3.2}
    Let $f:U \xrightarrow{} \C$ be holomorphic on an open set $U$ of  $\C$, and
    suppose that  $\bar{B}(z_0,r) \subseteq U$ with $r>0$. If
    $\y(t)=z_0+re^{it}$ on $0 \leq t \leq 2\pi$, then
    \begin{equation*}
        f(z)=\frac{1}{2i\pi}\int_\y{\frac{f(w)}{w-z} \ dw}
    \end{equation*}
    for every $|z-z_0|<r$
\end{lemma}
\begin{proof}
    Consider $U_1=\{\frac{z-z_0}{r} : z \in U\}$ and let $g(z)=f(z_0+rz)$. Then
    without loss of generality, suppose that $z_0=0$, and $r=1$, so that we work
    with the unit ball  $B(0,1)$.

    Fix $|z|<1$ and let
    \begin{equation*}
        \phi(s,t)=\frac{f((1-t)z+te^{is})e^{is}}{e^{is}-z}-f(z)
    \end{equation*}
    for all $0 \leq t \leq 1$, and  $0 \leq s \leq 2\pi$. Notice that
    $|(1-t)z+te^{is}| \leq 1$ so that $g$ is of class  $C^1$.

    Now, we have
    \begin{equation*}
        g(0)=\int_0^{2\pi}{\phi(s,0) \
        ds}=f(z)\int_0^{2\pi}{\frac{e^{is}}{e^{is}-z} \ ds}-2\pi{f(z)}=0
    \end{equation*}
    so that we get
    \begin{equation*}
        \int_0^{2\pi}{\frac{e^{is}}{e^{is}-z} \ ds}=2\pi
    \end{equation*}
    now, by leibniz's rule, we get
    \begin{equation*}
        g'(t)=\int_0^{2\pi}{D_t{\phi(s,t)} \ ds}
    \end{equation*}
    however, for $0 \leq t \leq 1$, we have
    \begin{equation*}
        \Phi(s)=-i\frac{f(z+t(e^{is}-z))}{t}
    \end{equation*}
    is a primitive of $D_t{\phi}$, and that $g'(t)=\Phi(2\pi)-\Phi(0)=0$ for all
    $0 \leq t \leq 1$, which makes  $g$ constant. So we get
    \begin{equation}
        f(z)=\int_0^{2\pi}{\frac{f(e^{is})e^{is}}{e^{is}-z} \ ds}
        =\frac{1}{2i\pi}\int_\y{\frac{f(w)}{w-z} \ dw}
    \end{equation}
\end{proof}

\begin{lemma}\label{4.3.3}
    Let $\y$ be a rectifiable path in  $\C$. Then, if $\{f_n\}$ is a sequence of
    continuous complex valued functions, defined on $\{\y\}$, converging
    uniformly to a continuous complex valued function $f$ defined on $\{\y\}$,
    then we have
    \begin{equation*}
        \int_\y{f_n}=\int_\y{f}
    \end{equation*}
\end{lemma}
\begin{proof}
    Let $\e>0$. Then there exists an $N \in \Z^+$ such that
    \begin{equation*}
        |f_n(z)-f(z)|<\frac{\e}{V(\y)}
    \end{equation*}
    for ever $z \in \{\y\}$, and $n \geq N$. Then we get
    \begin{equation*}
        \Big{|} \int_\y{f_n}-\int_{y}{f} \Big{|}=
        \Big{|} \int_\y{(f_n-f)} \Big{|} \leq
        \int_\y{|f_n(z)-f(z)| \ |dz|} \leq \frac{\e}{V(\y)}V(\y) \leq \e
    \end{equation*}
\end{proof}

\begin{theorem}\label{4.3.4}
    Let $f:U \xrightarrow{} \C$ be a complex valued function on an open set $U$
    of $\C$. Then  $f$ is holomorphic if, and only if it is analyitic.
\end{theorem}
\begin{proof}
    It was shown by theorem \ref{3.2.4} that functions which are analytic on
    their domain are holomorphic on their domain. It remains to show the
    converse.

    Suppose first that $f$ is holomorphic. Let $0<r<R$ so that  $\bar{B}(z_0,r)
    \subseteq B(z_0,R)$ and take $y(t)=z_0+re^{it}$ on $0 \leq t \leq 2\pi$.
    Then we have that
    \begin{equation*}
        f(z)=\frac{1}{2i\pi}\int_\y{\frac{f(w)}{w-z} \ dw} \text{ for all }
        |z-z_0|<r
    \end{equation*}
    since $w \in \{\y\}$, and $|z-z_0|<r$, we have
    \begin{equation*}
        \frac{|f(w)||z-z_0|^n}{|w-z_0|^{n+1}} \leq
        \frac{M}{r}\Big{(} \frac{z-z_0}{r} \Big{)}^n
    \end{equation*}
    with $M=\max{\{|f(w)|\}}$ for all $|w-z_0|<r$. Then we get
    \begin{equation*}
        \frac{|z-z_0|}{r}<1
    \end{equation*}
    so by the Weierstrass $M$-test, the series
    \begin{equation*}
        \sum_{n=0}^\infty{f(z)\frac{(z-z_0)^n}{(w-z_0)^{n+1}}}
    \end{equation*}
    converges uniformly on $\{\y\}$. Using a geometric series, we get
    \begin{equation*}
        \frac{1}{w-z_0}=\frac{1}{1-\frac{z-z_0}{w-z_0}}=
        \frac{1}{w-z_0}\sum_{n=0}^\infty{\Big{(} \frac{z-z_0}{w-z_0} \Big{)}^n}
    \end{equation*}
    Since $|z-z_0|<r$. Now, multiplying both sides by $(\frac{f(z)}{2i\pi})$ and
    integrating along the circle defined by $|w-z_0|<r$ gives us $f(z)$. Now, by
    lemma \ref{4.3.3}, and integrating under the sum, we get
    \begin{equation*}
        f(z)=\sum_{n=0}^\infty{a_n(z-z_0)^n}
    \end{equation*}
    where
    \begin{equation*}
        a_n=\frac{1}{2i\pi}\int_\y{\frac{f(w)}{(w-z_0)^{n+1}} \ dw}
    \end{equation*}
    Therefore, $f$ converges for  $|z-z_0|<r$, and moreover each $a_n$ is
    independent of the path $\y$, so that we get the required power series
    converging on  $|z-z_0|<R$.
\end{proof}
\begin{corollary}
    If $f$ is analytic on $U$, then for every $z_0 \in U$ with $|z-z_0|<R$,
    \begin{equation*}
        f(z)=\sum_{n=0}^\infty{a_n(z-z_0)^n}
    \end{equation*}
    where the radius of convergence $R>0$ is gven by the distance between $z_0$
    and $\partial{U}$.
\end{corollary}
\begin{corollary}
    Analytic functions are of class $C^\infty$.
\end{corollary}
\begin{corollary}
    If $f$ is analytic on $U$ and $\bar{B}(z_0,r) \subseteq U$, then
    \begin{equation*}
        f^{(n)}(z_0)=\frac{n!}{2i\pi}\int_\y{\frac{f(w)}{(w-z)^{n+1}} \ dw}
    \end{equation*}
    where $\y(t)=z_0+re^{it}$ on $0 \leq t \leq 2\pi$.
\end{corollary}
\begin{corollary}
    A function is entire if and only if it is holomorphic on all $\C$.
\end{corollary}
\begin{remark}
    This theorem shows that the definitions for analytic functions and
    holomorphic functions are equivalent. Therefore, we make the convention of
    calling functions holomorphic when we want to refer to their differential
    properties, and analytic when we want to refer to them as power series
    expansions.
\end{remark}

\begin{theorem}[Cauchy's Estimate]\label{4.3.5}
    Let $f:U \xrightarrow{} \C$ be analytic on an open set $U$ of  $\C$, whose
    power series expansion converges on the ball $B(z_0,R)$. If there is
    an $M>0$ such that  $|f(z)| \leq M$ for every $z \in B(z_0,R)$, then
    \begin{equation*}
        |g^{(n)}(z_0)| \leq \frac{n!M}{R^n}
    \end{equation*}
\end{theorem}
\begin{proof}
    By the above corollory, we have
    \begin{equation*}
        |f^{(n)}(z_0)| \leq
        \frac{n!}{2\pi}\int_\y{\Big{|} \frac{f(w)}{(w-z)^N{n+1}} \Big{|} \ dw}
        \leq \frac{n!}{2\pi}\frac{M}{R^{n+1}}2\pi{r}=\frac{n!M}{R^n}
    \end{equation*}
\end{proof}

\begin{lemma}\label{4.3.6}
    Let $f:U \xrightarrow{} \C$ be analytic on an open set $U$ of  $\C$, and
    suppose that $\y$ is a closed rectifiable path on the ball $B(z_0,R)$, with
    $R>0$ the radius of convergence of $f$ at $z_0$. Then $f$ has a primitive
    and
    \begin{equation*}
        \int_\y{f}=0
    \end{equation*}
\end{lemma}
\begin{proof}
    We have that
    \begin{equation*}
        f(z)=\sum{a_n(z-z_0)^n} \text{ for all } |z-z_0|<R
    \end{equation*}
    Let
    \begin{equation*}
        F(z)=\sum{\frac{a_n}{n+1}(z-z_0)^{n+1}}
    \end{equation*}
    Since $\sqrt{n+1} \xrightarrow{} 1$ as $n \xrightarrow{} \infty$, $F$ has
    the same radius of convergenc of  $F$, and hence converges on the ball
    $B(z_0,R)$. Moreover, notice that $F'=f$ for every  $|z-z_0|<R$.
\end{proof}
