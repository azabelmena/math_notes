\section{M\"obius Transformations}

\begin{definition}
    We call a complex valued function $S$ on $C_\infty$, defined by
    \begin{equation*}
        S(z)=\frac{az+b}{cx+d} \text{ where } a,b,c,d \in \C
    \end{equation*}
    a \textbf{M\"obius transformation} if $ad-cd \neq 0$.
\end{definition}

We have the following results for M\"obius transformations.

\begin{theorem}\label{3.3.1}
    The set of all M\"obius transformations forms a group under function
    compositon.
\end{theorem}
\begin{proof}
    Since $\circ$ is associative, it suffices to show the identity, closure, and
    inverse laws. Indeed, notice that the function
    \begin{equation*}
        I=\frac{1z+0}{0z+1}=z
    \end{equation*}
    is a M\"obius transformation, and that for any M\"obius transformation, $S$,
    $S=S \circ I=I \circ S$.

    Now, Let $S$ and $T$ be M\"obius transformations. Then
    \begin{equation*}
        S(z)=\frac{az+b}{cz+d} \text{ and } T(z)=\frac{fz+g}{hz+l}
    \end{equation*}
    for $a,b,c,d,f,g,h,l \in \C$. Then we get
    \begin{equation*}
        S \circ T(z)=\frac{aT(z)+b}{cT(z)+b}
    \end{equation*}
    and since $T(z) \in \C_\infty$, for all values of $\z$, then $S \circ T$ is a
    M\"obius transformation.

    Finally, let
    \begin{equation*}
        \inv{S}(z)=\frac{dz-b}{cz-a}
    \end{equation*}
    Then $S \circ \inv{S}=\inv{S} \circ S(z)=I(z)$, and we are done.
\end{proof}

\begin{definition}
    Let $a \in \C$. We call a M\"obius trasnsformation of the form $T(z)=z+a$ a
    \textbf{translation} of $z$ by  $a$. We call a M\"obius transformation
    $D(z)=az$ a \textbf{dilation} of $z$ by  $a$.  Let $0 \leq t \leq 2\pi$.
    Then we call the M\"obius transformation  $R(z)=e^{it}z$ a \textbf{rotation}
    of $z$ about  $t$, and we call the M\"obius transformation
    $S(z)=\frac{1}{z}$ an \textbf{inversion} of $z$.
\end{definition}

\begin{lemma}\label{3.3.2}
    If $S$ is a M\"obius transformation, then  $S$ is the composition of
    translations, dilations, and inversions.
\end{lemma}
\begin{proof}
    Let
    \begin{equation*}
        S(z)=\frac{az+b}{cz+d}
    \end{equation*}
    Suppose that $c=0$, so that  $S(z)=\frac{a}{d}z+b$. Then $S=S_2 \circ S_1$
    where $S_1(z)$ is a translation by $b$ and $S_2(z)$ is a dilation by
    $\frac{a}{d}$.

    Now, if $c \neq 0$, then let  $S_1(z)=z+\frac{d}{c}$,
    $S_2(z)=\frac{1}{z}$, $S_3(z)=\frac{bc-ad}{c^2}z$ and
    $S_4(z)=z+\frac{a}{c}$. Then $S=S_4 \circ S_3 \circ S_2 \circ S_1$.
\end{proof}
\begin{corollary}
    Rotations are compositions of translations, dilations, and inversions.
\end{corollary}

\begin{definition}
    Let $S$ be a M\"obius transformation. We cal a point $z \in \C$ a
    \textbf{fixed point} of $S$ if  $S(z)=z$.
\end{definition}

\begin{lemma}\label{3.3.3}
    If
    \begin{equation*}
        S(z)=\frac{az+b}{cz+d}
    \end{equation*}
    is a M\"obius transformation, and $z$ is a fixed point, then
    $cz^2+(d-a)z-b=0$ and $S$ has at most two fixed points; unless it is the
    identity transformation.
\end{lemma}
\begin{proof}
    Suppose that $S \neq I$, and consider the equation
    \begin{equation*}
        z=\frac{az+b}{cz+d}
    \end{equation*}
    to obtain a quadratic polynomial over $\C$, which has at most two roots in
    $\C$.
\end{proof}

\begin{lemma}\label{3.3.4}
    Let $S$ be a M\"obius transformation on  $\C_\infty$, and let  $a,b,c,d \in
    \C_\infty$ distinct points with  $\a=S(a)$, $\b=S(b)$ and $\y=S(c)$. If $T$
    is another M\"obius transformation with this property, then  $S=T$.
\end{lemma}
\begin{proof}
    Notice by hypothesis that the transformation $\inv{T} \circ S$ has $a$,
    $b$, and  $c$ as fixed points, which forces  $\inv{T} \circ S=I$.
\end{proof}

\begin{definition}
    Let $z_2,z_3,z_4 \in \C_\infty$ and define the M\"obius transformation
    $S:\C_\infty \xrightarrow{} \C_\infty$ by
    \begin{align*}
        S(z)    &=  \frac{\Big{(} \frac{z-z_3}{z-z_4} \Big{)}}{\Big{(}
                        \frac{z_2-z_3}{z_2-z_4} \Big{)}} \text{ if } z_2,z_3,z_4
                                \in \C   \\
        S(z)    &= \frac{z-z_3}{z-z_4} \text{ if } z_2=\infty   \\
        S(z)    &= \frac{z_2-z_4}{z-z_4} \text{ if } z_3=\infty   \\
        S(z)    &= \frac{z_2-z_3}{z_2-z_3} \text{ if } z_4=\infty   \\
    \end{align*}
    and where $S(z_2)=1$, $S(z_3)=0$, and $S(z_4)=\infty$. Then if $z_1 \in
    \C_\infty$, we define the \textbf{cross ratio}, $(z_1,z_2,z_3,z_4)$ of $z_1$
    to be $S(z_1)$.
\end{definition}

\begin{example}\label{example_3.8}
    $(z_2,z_2,z_3,z_4)=1$, $(z_3,z_2,z_3,z_4)=0$, and $(z_4,z_2,z_3,z_4)=0$, by
    definition. Now, if $M$ is aany M\"obius transformation, and  $w_2,w_3,w_4$
    are points on $M$ such that $M(w_1)=1$, $M(w_3)=0$, and $M(w_4)=\infty$,
    then $M(z)=(z,w_2,w_3,w_4)$.
\end{example}

\begin{theorem}\label{3.3.5}
    IF $z_2,z_3,z_4 \in \C_\infty$ are distinct points, and $T$ is a M\"obius
    transformation, then  $(z,z_2,z_3,z_4)=(T(z),T(z_2),T(z_3),T(z_4))$ for all
    $z \in \C_\infty$. That is, the cross ratio is invariant under
    transformations.
\end{theorem}
\begin{proof}
    Let $S=(z,z_2,z_3,z_4)$, then $S$ is a M\"obius transformation. Now, if
    $M=S \circ \inv{T}$, then $M(T(z_2))=1$, $M(T(z_3))=0$, and
    $M(T(z_4))=\infty$, which makes $S \circ \inv{T}=(z,T(z_2),T(z_3),T(z_4))$.
\end{proof}

\begin{lemma}\label{3.3.6}
    If $z_2,z_3,z_4 \in \C_\infty$ and $w_2,w_3,w_4 \in \C_\infty$ are all
    distinct points, then there exists one, and only one M\"obius transformation
    $S$ for which $S(z_2)=w_2$, $S(z_3)=w_3$, and $S(z_4)=w_4$.
\end{lemma}
\begin{proof}
    Let $T(z)=(z,z_2,z,3,z_4)$ and $M=(z,w,2,w,3,w_4)$. Put $S=\inv{M} \circ T$.
    Then $s(Z_2)=w_2$, $S(z_3)=z_3$,and $S(z_4)=w_4$. Now, if $R$ is another
    M\"obius transformation having this property, then  $\inv{R} \circ S$ has
    $3$ fixed points, which makes $\inv{R} \circ S=I$.
\end{proof}

\begin{lemma}\label{3.3.7}
    Let $z_1,z_2,3,z_4 \in \C_\infty$ be distinct points. Then
    $(z,1,z_2,z_3,z_4) \in \R$ if, and only if all the points lie on a circle.
\end{lemma}
\begin{proof}
    Define $S:\C_\infty \xrightarrow{} \C_\infty$ by $S(z)=(z,z_2,z_3,z_4)$.
    Then it suffices to show that the preimage $\inv{S}(\R_\infty)$ is a circle.

    Let
    \begin{equation*}
        S(z)=\frac{az+b}{cz+d}
    \end{equation*}
    If $z \in \R$, and  $w=\inv{S}=(z) \neq \infty$ then $x=S(w)$, so that
    $S(w)=\bar{S(w)}$. Thus
    \begin{equation*}
        \frac{aw+b}{cw+b}=\frac{\bar{a}\bar{w}+\bar{b}}{\bar{c}\bar{w}+\bar{d}}
    \end{equation*}
    this gives
    \begin{equation*}
        (a\bar{c}-\bar{a}c)|w|^2+(a\bar{d}-\bar{b}c)w+(b\bar{c}-d\bar{a})\bar{w}+
        (b\bar{d}-\bar{b}d)=0
    \end{equation*}
    If $a\bar{c} \in \R$, then $a\bar{c}-a\bar{c}=0$. putting
    $\a=2(a\bar{d}-\bar{b}c)$, and $\b=i(b\bar{d}-\bar{b}d)$, and multuplying by
    the above equation by $i$ gives
    \begin{equation*}
        0=\im{(\a w+\b)}
    \end{equation*}
    Since $\b \in \R$,  $w$ lies on this line determined by  $\a$ and  $\b$.
    Now, if  $\a\bar{c} \notin \R$, then we get
    $|w|^2+\y\bar{\y}w+w\bar{\y}-\d=0$, for $\y \in \C$ and  $\d \in \R$
    constants. Hence  $|w+\y|=\l$, where
     \begin{equation*}
         \l=\sqrt{|\y^2+\d}=\Big{|} \frac{ad-bc}{\bar{a}c-a\bar{c}} \Big{|}>0
     \end{equation*}
     Since $\y$ and  $\l$ are independent of  $x$, this defines a circle, and we
     are done.
\end{proof}

\begin{theorem}\label{3.3.8}
    M\"obius transformations take circles onto circles.
\end{theorem}
\begin{proof}
    Let $\Gamma$ be a circle in  $\C_\infty$, and  $S$ a M\"obius
    transformation. Let  $z_2,z_3,z_4 \in \Gamma$, diestinct, and put
    $w_2=S(z_2)$, $w_3=S(z_3)$, and $w_4=S(z_4)$. then $w_1$ $w_2$ and $w_3$
    determine a circle $\Gamma'$. Now, for every  $z \in \C_\infty$, we have
    that  $(z,z_2,z_3,z_4)=(S(z),w_2,w_3,w_4)$, so that if $z \in \Gamma$,
    $S(z) \in \Gamma'$; i.e. that $S(\Gamma)=\Gamma'$.
\end{proof}

\begin{lemma}\label{3.3.9}
    For any two circle $\Gamma$ and  $\Gamma'$ in $\C_\infty$, there is a
    M\"obius transformation  $T$, such that  $T(\Gama)=\Gamma'$, and $T$ takes
    any three points on  $\Gamma$ to any three points on  $\Gamma'$.
\end{lemma}

\begin{definition}
    Let $\Gamma$ be a circle wthrough the discinct points $z_2,z_3,z_4 \in
    \C_\infty$. We call the points $z,z' \in \C_\infty$  \textbf{symmetric} with
    respect to $\G$ if  $(z',z_2,z_3,z_4)=\bar{(z,z_2,z_3,z_4)}$.
\end{definition}

\begin{example}\label{example_3.9}
    If $\Gamma$ is a straight line, choosing  $z_4=\infty$, we get that
    \begin{equation*}
        \frac{z'-z_3}{z_2-z_4}=\frac{\bar{z}-\bar{z_3}}{\bar{z_2}-\bar{z_3}}
    \end{equation*}
    Giving $|z'-z_3|=|z-z_3|$. Which shows that $z$ and  $z'$ are equidistant
    from each other. Moreover, we have that
    \begin{equation*}
        \im{\frac{z'-z_3}{z_2-z_4}}=-\im{\frac{z-z_3}{z_2-z_3}}
    \end{equation*}
    hence, with the exception that $z \in \Gamma$, $z$ and $z'$ lie on different
    half planes determined by  $\y$. That is, lhe line segmenent $[z,z'] \perp
    \Gamma$.

    Now, let $\Gamma=\{z \in \C : |z-z_0|=R\}$ (the circle of radius $R$
   centered about  $z_0$) for some $a \in \C$, and $0<R<\infty$. Let $z_2,z_3,z_4
    \in \Gamma$ distinct points on $\Gamma$. Then we have that
    \begin{align*}
        (z',z_2,z_3,z_4)    &=  \bar{(z,z_2,z_3,z_4)}   \\
                            &=  \bar{(z-z_0,z_2-z_0,z_3-z_0,z_4-z_0)}   \\
                            &= (\bar{z}-\bar{z_0},\frac{R^2}{z_2-z_0},
                                    \frac{R^2}{z_3-z_0},\frac{R^2}{z_4-z_0}) \\
                            &=(\frac{R^2}{\bar{z}-\bar{z_0}},z_2-z_0,z_3-z_0,z_4-z_0) \\
                            &=(\frac{R^2}{\bar{z}-\bar{z_0}}+z_0,z_2,z_3,z_4) \\
    \end{align*}
    so that $z'=\frac{R^2}{z-z_0}$ or $(z'-z_0)(\bar{z}-\bar{z_0})=0$, hence
    \begin{equation*}
        \frac{z'-z_0}{z-a}=\frac{R^2}{|z-z_0|^2}>0
    \end{equation*}
    and $z'$ lies on the ray  $[z,\infty)=\{z_0+t(z-z_0) : 0<t<\infty\}$ from
    $z_0$ to $z$. Now, suppose that  $z$ is inside of  $\Gamma$, let  $L$ be the
    ray from  $z_0$ trhough $z$, and construct the perpendicular  $P \perp L$.
    Then $P$ cuts  $\Gamma$ at a point $z''$. Construct the tangent through
    the point $z''$ and $z'$. Then  $z_0$ and $\infty$ are symetrc to  $\Gama$,
    and we obtain  $z'$ from  $z$ in this manner.
\end{example}

\begin{theorem}[The Symmetric Principle]\label{3.3.10}
    If a M\"obius transformation $T$ takes a circle  $\Gamma_1$ onto a circle
    $\Gamma_2$, then pairs of symmetric points with respect to $\Gamma_1$ are
    mapped onto pairs of symmetric points with respect to $\Gamma_2$.
\end{theorem}
\begin{proof}
    LEt $z_2,z_3,z_4 \in \Gamma$. If $z$ and  $z'$ are symmetric with respect to
     $\Gamma_1$, then
     $(T(z'),T(z_2),T(z_3),T(z_4))=(z',z_2,z_3,z_4)=\bar{(z,z_2,z_3,z_4)}=
     \bar{(T(z),T(z_2),T(z_3),T(z_4))}$. Which makes $T(z')$ symmetric to $z$
     with respect to  $\Gamma$.
\end{proof}

\begin{definition}
    We define the \textbf{orientation} of a circle $\Gamma$ to be an ordered
    triple $(z_1,z_2,z_3)$ of points on $\Gamma$.
\end{definition}

\begin{example}\label{example_3.10}
    Consider $\R$ as a circle, with  $z_1,z_2,z_3 \in \R$. Then put
    $T(z)=(z,z_1,z_2,z_3)=\frac{az+b}{cz+d}$. Since $T(\R_\infty)=\R$, choose
    $a,b,c,d \in \R$ and we get
    \begin{equation*}
        T(z)=\frac{az+b}{cz+d}(c\bar{z}+d)=\frac{ad-bc}{|cz+d|^2}\im{z}
    \end{equation*}
    If $ad-bc>0$, or  $ad-bc<0$, then the set  $H_<=\{z \in \C :
    \im{(z,z_1,z_2,z_3)<0}\}$ is the upper, or lower half of the plane.
\end{example}

\begin{definition}
    If $(z_1,z_2,z_3)$ is an orientaton of a circle $\Gamma$, then we define the
     \textbf{right side} of $\Gamma$ to be $\Gamma^>=\{z \in \C :
     \im{(z,z_1,z_2,z_3)}>0\}$. Similarly, we define the \textbf{left side} of
     $\Gamma$ to be $\Gamma^>=\{z \in \C : \im{(z,z_1,z_2,z_3)}<0\}$.
\end{definition}

\begin{theorem}[The Orientation Principle]\label{3.3.11}
    IF $\Gamma_1$ and $\Gamma_2$ are circles in $\C_\infty$, and  $T$ is a
    M\"obius transformation, taking  $\Gamma_1$ onto $\Gamma_2$, then $T$
    preserves orientation. That is, if  $(z_1,z_2,z_3)$ is an orientation of
    $\Gamma_1$, $(T(z_1),T(z_2),T(z_3))$ is an orientation of $\Gamma_2$.
    Moreover, $T$ takes the right and left sides of  $\Gamma_1$ onto the right
    and left sides of $\Gamma_2$, respectively.
\end{theorem}

\begin{example}\label{example_3.11}
    \begin{enumerate}
        \item[(1)] Consider the orientation $(1,0,\infty)$ of $\R$. Since
            $(z,1,0,\infty)=z$, the right side of $\R$ with respect to
            $(1,0,\infty)$ is the upper half plane. Now, to find an
            function $f:U \xrightarrow{} \C$, with $U=\{z \in \C : \re{z}>0\}$
            such that $f(U)=B(0,1)=\{z \in \C : |z|<1\}$ (i.e. the open unit
            ball), it is sufficient to find a M\"obius transformaton mapping the
            imaginary axis $i\R$ onto the unit circle $S^1$. By the orientation
            principle, this transformation takes  $U$ onto  $B(0,1)$. Choose the
            orientation $(-i,0,i)$ on $\R$, then the set $\{z:\re{z}>0\}$ is
            on the right side of $i\R$. Hence $i\R^>=\{z \in \C : \re{z}>0\}$.

            Taking $S(z)=\frac{z_2}{z-i}$, and $R(z)=\frac{z-1}{z+1}$, then the
             function
            \begin{equation*}
                g(z)=\frac{e^z-1}{e^z+1}
            \end{equation*}
            maps the strip $\{z \in \C : \Im{z}<\frac{\pi}{2}\}$ onto $B(0,1)$.

        \item[(2)] Let $U_1$ and $U_2$ be regions in $\C$, and we would like to
            find an  functnion $f:U_1 \xrightarrow{} U_2$ such that
            $f(U_1)=f(U_2)$. Now, map $U_1$ and $U_2$ onto the ball $B(0,1)$ via
            a M\"obius transformation. If this is possible, then we can obtain
            $f$ by taking the composition of two  functions  $f_1 \circ
            \inv{f_2}$.

        \item[(3)] Let $U$ be the region insid the intersection of two circles
            $\Gamma_1$ and $\Gamma_2$; intersecting at ppints $a,b \in \C$, with
             $a \neq b$. Let  $L$ be the line segment $[a,b]$ with the
             orientation $(\infty,b,a)$ an dlet
             \begin{equation*}
                 T(z)=(z,\infty,b,a)=\frac{z-a}{b-a}
             \end{equation*}
             Then $T$ maps  $L$ onto the closed interval $[0,1]$, and since $T$
             maps circles onto circles, $T$ maps $\Gamma_1$ onto $\Gamma_2$
             through the points $0$ and  $\infty$. SO that $T(\Gamma_1)$ and
             $T(\Gamma_2)$ are straight lines in $\C_\infty$. By the orientation
             principle we get  $T(U)=\{\w \in \C : -\a<\arg{\w}<\a\}$ for some
             $\a>0$, or  $T(U)$ is the complement of a closed sector.

             Now, let $f_1(z)=e^{it}z^a$, with $a$ an appropriate power and let
             $f_2(z)=\frac{z-1}{z+1}$. Then $f_1 \circ f_2$ takes $U$ onto
             $B(0,1)$.
    \end{enumerate}
\end{example}
