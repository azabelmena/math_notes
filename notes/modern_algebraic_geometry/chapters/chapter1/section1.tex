\section{Categories and Functors}\label{section_1.1}

\begin{definition}
  A \textit{category} $\Cc$ is a collection of \textit{objects},
  denoted $\obk{\Cc}$ together with a collection of sets
  between pairs objects, denoted $\Mor{(a,b)}$ given $a,b \in
  \obj{\Cc}$, such that the following data is given:
  \begin{enumerate}
    \item[(1)] For any objects $a,b,c \in \obj{\Cc}$, there exist
      \textit{composition maps}
      \begin{equation*}
        \circ:\Mor{(b,c)} \times \Mor{(a,b)} \xrightarrow{} \Mor{(a,c)}
      \end{equation*}
      associating to each pair of
      morphisms $(g,f)$ a morphism $g \circ f$, called the
      \textit{composition} of the morphism $g$ with the morphism $f$.

    \item[(2)] For any objects $a,b,c,d \in \obj{\Cc}$, and for any
      morphisms $f \in \Mor{(a,b)}$, $g \in \Mor{(b,c)}$, and $h \in
      \Mor{(c,d)}$, the compositions between morphisms associate
      wherever they exist. That is:
      \begin{equation*}
        h \circ (g \circ f)=(h \circ g) \circ f
      \end{equation*}
      whenever either $h \circ (g \circ f)$ or $(h \circ g) \circ f$
      exists as a morphism in $\Cc$.

    \item[(3)] For any object $a \in \obj{\Cc}$, there is a morphism
      $\id_a \in \Mor{(a,a)}$, called the \textit{identity morphism},
      such that for any objects $b,c \in \obj{\Cc}$ and morphisms $f
      \in \Mor{(a,b)}$ and $g \in \Mor{(c,a)}$ the following holds:
      \begin{equation*}
        f \circ \id_a=f \text{ and } \id_a \circ g=g
      \end{equation*}
  \end{enumerate}
  We usually denote the collection of objects of the category
  $\Cc$ by $\Cc$ itself, instead of $\obj{\Cc}$. Given objects
  $a,b \in \Cc$, we can denote a morphism $f$ between $a$ and $b$
  by $f:a \xrightarrow{} b$ instead of $f \in \Mor{(a,b)}$. We
  denote the collection of all morphisms of the category $\Cc$ by
  $\Mor{\Cc}$.
\end{definition}

\begin{definition}
  Let $\Cc$ be a category. We call objects $a,b \in \Cc$
  \textit{isomorphic} provided there exists morphisms $f \in
  \Mor{(a,b)}$ and $g \in \Mor{(b,a)}$ such that:
  \begin{equation*}
    g \circ f=\id_a \text{ and } f \circ g=\id_b
  \end{equation*}
\end{definition}

\begin{lemma}\label{lemma_1.1.1}
  Let $\Cc$ be a category, and let $a,b \in \Cc$ be objects. If $f:a
  \xrightarrow{} b$ is an isomorphism, then there exists a unique
  morphism $g:b \xrightarrow{} a$ such that:
  \begin{equation*}
    g \circ f=\id_a \text{ and } f \circ g=\id_b
  \end{equation*}
  Moreover, $g:b \xrightarrow{} a$ is also an isomorphism.
\end{lemma}

\begin{example}\label{example_1.1}
  \begin{enumerate}
    \item[(1)] Consider the collection $\Cc$ whose objects are defined
      to be sets, and whose moprhisms are defined to be functions
      between sets. That is, any set $A$ is an object of $\Cc$ and
      given any sets $A$ and $B$, a function $f:A \xrightarrow{} B$ is
      a morphism of $\Cc$. Define the composition of morphisms,
      $\circ$, to be the usual function composition of functions
      between sets. Then such compositions are guaranteed to exist,
      moreover they are associative. Indeed, let $A,B,C,D$ be sets
      (objects), and let $f:A \xrightarrow{} B$, $g: B \xrightarrow{}
      C$, and $h:C \xrightarrow{} D$ be functions (morphisms). Fix any
      element $x \in A$. Then:
      \begin{equation*}
        h \circ (g \circ f)(x)=h \circ g(f(x))=h(g(f(x))) \text{ and }
        (h \circ g) \circ f(x)=(h \circ g)(f(x))=h(g(f(x)))
      \end{equation*}
      Lastly, define the identity function $\id_A:A \xrightarrow{} A$
      by $\id_A(x)=x$ for all $x \in A$. Then $\id_A$ is the identity
      morphism on $A$. Indeed, let $B,C$ be sets and $f:A
      \xrightarrow{} B$ and $g:C \xrightarrow{} A$ be functions. Fix
      $x \in A$ and $y \in C$. Then:
      \begin{equation*}
        f \circ \id_A(x)=f(\id_A(x))=f(x) \text{ and }
        \id_A \circ g(y)=\id_A(g(y))=g(y)
      \end{equation*}
      So $\Cc$ defines a category on sets. Indeed we denote this
      category by $\Set$.

      The isomorphisms of the category $\Set$ are those functions
      which are bijectcions. Indeed, let $f:A \xrightarrow{} B$ be an
      isomprhism of $\Set$. Then there is a funcion $g:B
      \xrightarrow{} A$ for which $g \circ f=\id_A$, which makes $f$
      injective (or 1--1), and $f \circ g=\id_B$, which makes $f$
      surjective (or onto). Therefore $f$ must be a bijection between
       $A$ and $B$.

     \item[(2)] Fix a field $k$. We denote the category $\Vec_k$ to be
       the category having as objects vector spaces over $k$, and
       havin as morphisms linear tranformations between vector spaces
       over $k$. The isomorphisms of $\Vec_k$ are linear operators.

     \item[(3)] We define a \textit{groupoid} to be a category in
       which every morphism is an isomorphism. We define a
       \textit{monoid} to be a category having only one objects.

       We claim that any group $G$ is a groupoid which is also a
       monoid; that is, it is a groupoid having only one obeject.
       Indeed, define the category $\Gc$ having as its only object the group
       $G$, i.e. $\obj{\Gc}=\{G\}$. Define the morphisms of $\Gc$ to
       be the elements of $G$; i.e. $\Mor{\Gc}=G$, and write the
       elements of $G$ as $x:G \xrightarrow{} G$. Define the
       composition of morphisms of $\Gc$ to be the group operation
       $\cdot:G \times G \xrightarrow{} G$ taking $(x,y)
       \xrightarrow{} xy$. By the axiom of closure for groups, and the
       associative law for groups, the compositions $xy$ exist in
       $\Gc$, and we have for any morphisms $x,y,z:G \xrightarrow{}
       G$:
       \begin{equation*}
         x(yz)=(xy)z
       \end{equation*}
       Observing the identity law for groups, the identity morphism of $G$ is the
       identity element $e:G
       \xrightarrow{} G$, for wich given any element $x:G
       \xrightarrow{} G$, we get:
       \begin{equation*}
         xe=x \text{ and } ex=x
       \end{equation*}
       This makes $\Gc$ a category, having  $\obj{\Gc}=\{G\}$ and
       $\Mor{\Gc}=G$; moreover, since $\Gc$ has only one object, $\Gc$
       is a monoid.

       Lastly, observe the inverse law for groups; given an element
       $a:G \xrightarrow{} G$, there is an element $b:G \xrightarrow{}
       G$ such that
       \begin{equation*}
         ab=e \text{ and } ba=e
       \end{equation*}
       Which makes $a$ an isomorphism of $\Gc$ (moreover, the
       corresponding element $b:G \xrightarrow{} G$ is also an
       isomorphsm and is unique by lemma \ref{lemma_1.1.1}). This
       makes every morphism of $\Gc$ an isomorphism, and hence $\Gc$ is
       a groupoid. Since $\obj{\Gc}=\{G\}$ and $\Mor{\Gc}=G$, we
       simply associate $\Gc$ with $G$ itself, so that $G$ is a
       groupoid with only one element.
  \end{enumerate}
\end{example}
