\section{Hasse Diagrams of Posets}
\label{section_1.3}

\begin{definition}
  Let $P$ be a poset and  $x,y \in P$. We say that $y$
  \textbf{covers} $x$ if  $x < y$, and  $x \leq z < y$ implies $x=z$
  for any  $z \in P$. We write $x \covers y$.
\end{definition}

\begin{lemma}\label{lemma_1.3.1}
  If $P$ is a finite poset, then  $x < y$ in $P$ if, and only if there
  exists a finite sequence of coverings $x=x_0 \covers x_1 \covers
  \dots \covers x_n=y$.
\end{lemma}

\begin{example}\label{example_1.6}
  \begin{enumerate}
    \item[(1)] In $\N$, we have  $m \covers n$ if, and only if
      $n=s(m)$, where $s$ is the successor function. Indeed, suppoes
      that  $m \covers n$. Then $m < n$ and  $m \leq k<n$ implies that
      $m=k$, so that $s(m)=s(k)$. Now, suppose that $s(k) \neq n$,
      since $k < n$ there is some $k<k'<n$ for which $s(k')=n$, but
      then we have $m<k'<n$ which contradicts that $m \covers n$. Thus
      we must have that $s(k)=n$, so that $s(m)=n$. On the otherhand,
      by definition of $s$, if  $s(m)=n$, so that $m<n$. Now, if there
     is some $k \in \N$ for which $m<k<n$, then $s(m)<s(k) \leq n$. Now
     since $s(m)=n$, we get $s(m)<s(k) \leq s(m)$, so that
     $s(m)=s(k)$. Hence $m=k$ and we get  $m \covers n$.

   \item[(2)] Since $\R$ is a totally ordered set under it's order,
     and since  $\Q$ is dense in  $\R$, there are no  $x,y \in \R$ for
     which $x \covers y$.

   \item[(3)] Let $X$ be a set, and let  $A,B \subseteq C$. Suppose
     that $A \covers B$. Then $A \subseteq B$ with $A \neq B$, and if
     $A \subseteq C \subseteq B$, (with $C \neq B$) then $A=C$. Then
     there must be some element  $b \in \com{X}{A}$ for which $B=A
     \cup \{b\}$. Conversely, suppose that $B=A \cup \{b\}$, for some
     $b \in \com{X}{A}$. Then $A \subseteq B$ is strictly contained.
     Now, suppose there is a $C \subseteq X$ for which $A \subseteq C
     \subseteq B$, and $C \neq B$. Then $C \neq A \cup \{b\}$, and
     since $A \subseteq C$, it must be that $b \notin C$. This puts
     $C \subseteq A$, so that $C=A$.
  \end{enumerate}
\end{example}

\begin{definition}
  Let $P$ be a finite Poset. We define the  \textbf{Hasse diagram} of
  $P$ to be the directed graph  $G(P)=(V(P),E(P))$, toghether with a
  bijection $p:P \xrightarrow{} V(P)$, taking elements of $P$ to
  vertices of  $G(P)$, and such that for all $x,y \in P$, the following
  hold:
  \begin{enumerate}
    \item[(1)] If $x \covers y$ then $p(x)p(y) \in E(G)$ is a directed
      edge.

    \item[(2)] For $z \in P$, the vertex $p(z)$ is incident to the
      edge $p(x)p(y)$ if, and only if $x=z$ or $y=z$.
  \end{enumerate}
\end{definition}

\begin{example}\label{example_1.7}
  \begin{enumerate}
    \item[(1)] Let $P=\{a,b,c,d\}$. The two diagrams below are Hasse
      diagrams for $P$:
      \[\begin{tikzcd}
        c & d && c && d \\
        a & b &&& b \\
          &&&& a
          \arrow[from=2-1, to=1-1]
          \arrow[from=2-1, to=1-2]
          \arrow[from=2-2, to=1-1]
          \arrow[from=2-2, to=1-2]
          \arrow[no head, from=2-5, to=1-4]
          \arrow[no head, from=2-5, to=1-6]
          \arrow[no head, from=3-5, to=1-4]
          \arrow[no head, from=3-5, to=1-6]
          \arrow[no head, from=3-5, to=2-5]
      \end{tikzcd}\]

    \item[(2)] The following are also Hasse diagrams on differen sets.
      \[\begin{tikzcd}
        \bullet & \bullet & \vdots \\
        \bullet & \bullet & \bullet & \bullet && \bullet && \bullet \\
        \bullet & \bullet & \bullet && \bullet && \bullet \\
        \bullet && \bullet \\
        \\
        \\
        \\
        \\
        \\
        &&&& \bullet
        \arrow[from=2-1, to=1-1]
        \arrow[from=2-1, to=1-2]
        \arrow[from=2-2, to=1-2]
        \arrow[from=2-3, to=1-3]
        \arrow[from=2-4, to=3-5]
        \arrow[from=2-6, to=3-7]
        \arrow[from=3-1, to=2-1]
        \arrow[from=3-1, to=2-2]
        \arrow[from=3-2, to=2-2]
        \arrow[from=3-3, to=2-3]
        \arrow[from=3-5, to=2-6]
        \arrow[dashed, from=3-7, to=2-8]
        \arrow[from=4-1, to=3-1]
        \arrow[from=4-3, to=3-3]
      \end{tikzcd}\]

    \item[(3)] The following is not a Hasse diagram since the vertex
      $c$ is incident to the edge $ad$, but niether coincides witht
      the vertices $a$ or $d$.
      \[\begin{tikzcd}
          && d \\
          & c \\
        a && b
        \arrow[from=2-2, to=1-3]
        \arrow[no head, from=3-1, to=2-2]
        \arrow[from=3-3, to=1-3]
        \arrow[from=3-3, to=2-2]
      \end{tikzcd}\]

    \item[(4)] The following is also a Hasse diagram
      \[\begin{tikzcd}
        & \bullet \\
        \bullet & \bullet & \bullet \\
        \bullet && \bullet \\
                & \bullet \\
                & \bullet \\
        \bullet && \bullet \\
        \bullet & \bullet & \bullet \\
        & \bullet
        \arrow[no head, from=2-1, to=1-2]
        \arrow[no head, from=2-1, to=3-1]
        \arrow[no head, from=2-2, to=4-2]
        \arrow[no head, from=2-3, to=1-2]
        \arrow[no head, from=2-3, to=3-3]
        \arrow[no head, from=3-1, to=5-2]
        \arrow[no head, from=3-3, to=4-2]
        \arrow[no head, from=4-2, to=6-3]
        \arrow[no head, from=5-2, to=7-2]
        \arrow[no head, from=6-1, to=5-2]
        \arrow[no head, from=6-1, to=7-1]
        \arrow[no head, from=6-3, to=7-3]
        \arrow[no head, from=7-1, to=8-2]
        \arrow[no head, from=8-2, to=7-3]
      \end{tikzcd}\]
  \end{enumerate}
\end{example}

\begin{remark}
  Since Hasse diagrams are directed graphs, it is natural to write the
  arrows with arrows pointing in one directions. However, we will be
  in the convention of drawing the Hasse diagrams for posets in a
  ``top-down'' manner which suggests the underlying order of the
  elements. In this case, we can drop the arrows in our graphs and
  draw our Hasse diagrams with unmarked edges.
\end{remark}

\begin{lemma}\label{lemma_1.3.2}
  Let $P$ and $Q$ be posets, and $\phi:P \xrightarrow{} Q$ a
  bijection. Then the following statements are equivalent:
  \begin{enumerate}
    \item[(1)] $\phi$ is an order-isomorphism.

    \item[(2)] $x \covers y$ in $P$ if, and only if $\phi(x) \covers
      \phi(y)$ in $Q$.
  \end{enumerate}
\end{lemma}
\begin{proof}
  Suppose that $\phi$ is an order isomorphism, and that $x \covers y$
  in $P$. Then $x<y$ in $P$ so that  $\phi(x)<\phi(y)$ in $Q$. Suppose
  there exists a $w \in Q$ such that $\phi(x) \leq w<\phi(y)$.
  Since $\phi$ is onto, there is a $z \in P$ for
  which $w=\phi(z)$, moreover by order preservation we get that $x
  \leq z<y$ Since $x \covers y$ $x=z$, so $\phi(z)=\phi(x)=w$,
  therefore $\phi(x) \covers \phi(y)$. The converse follows by a
  similar argument since $\phi$ is order-preserving.

  Conversely, suppose that $x \covers y$ in  $Q$ if, and only if
  $\phi(x) \covers \phi(y)$ in $Q$. Let $x<y$ in $P$. By lemma
  \ref{lemma_1.3.1} there is a sequence of coverings $x=x_0 \covers
  x_1 \covers \dots \covers x_n=y$. Then by hypothesis
  $\phi(x)=\phi(x_0) \covers \phi(x_1) \covers \dots \covers
  \phi(x_n)=\phi(y)$. By lemma \ref{lemma_1.3.1}, this makes
  $\phi(x)<\phi(y)$, and hence $\phi$ is an order-isomorphism.
\end{proof}

\begin{proposition}\label{proposition_1.3.3}
  Two finite posets are order-isomorphic if, and only if their Hasse
  diagrams are graph isomorphic.
\end{proposition}
\begin{proof}
  Let $P \simeq Q$ via the order-isomoprhism  $\phi:P \xrightarrow{}
  Q$. Let $G(P)$ and $G(Q)$ be the Hasse diagrams of $P$ and $Q$
  respectively, and let $p:P \xrightarrow{} V(P)$ and $q:Q
  \xrightarrow{} V(Q)$ be the the bijections taking the elements of
  $p$ onto the vertives of $G(P)$ and the elements of $Q$ onto the
  vertices of $G(Q)$, respectively. Observe that if $x \in P$ then
  $\phi(x) \in Q$, so that $p(x) \in V(P)$ is a vertex of $G(P)$ and
  $q(\phi(x)) \in G(Q)$ is a vertex of $G(Q)$. Now, by lemma
  \ref{lemma_1.3.2} $x \covers y$ in $P$ if, and only if $\phi(x)
  \covers \phi(y)$ in $Q$. Then $p(x)p(y) \in E(P)$ is an edge in
  $G(P)$ if, and only if $q(\phi(x))q(\phi(y)) \in E(Q)$ is an edge in
  $G(Q)$. Therefore we have established a 1--1 correpsondence between
  the vertices of $G(P)$ and the vertices of $G(Q)$, and a 1--1
  correspondence between the edges of $G(P)$ and the edges of $G(Q)$.
  This makes $G(P) \simeq G(Q)$ as directed graphs.
\end{proof}

\begin{figure}[h]
  \centering
  \[\begin{tikzcd}
    {G(P)} & {p(a)} && {G(Q)} & {q(\phi(a))} \\
    {p(b)} & {p(e)} & {p(d)} & {q(\phi(b))} & {q(\phi(e))} & {q(\phi(d))} \\
    & {p(c)} &&& {q(\phi(c))}
    \arrow["\Phi", curve={height=-24pt}, dashed, tail reversed, from=1-1, to=1-4]
    \arrow[no head, from=1-2, to=2-1]
    \arrow[no head, from=1-2, to=2-2]
    \arrow[no head, from=1-5, to=2-4]
    \arrow[no head, from=1-5, to=2-5]
    \arrow[no head, from=2-1, to=3-2]
    \arrow[no head, from=2-2, to=3-2]
    \arrow[no head, from=2-3, to=1-2]
    \arrow[no head, from=2-4, to=3-5]
    \arrow[no head, from=2-5, to=3-5]
    \arrow[no head, from=2-6, to=1-5]
    \arrow[no head, from=3-2, to=2-3]
    \arrow[no head, from=3-5, to=2-6]
  \end{tikzcd}\]
  \caption{A graph-isomorphism $\Phi$ between the Hasse diagrams
  $G(P)$ and $G(Q)$, where $P$ and $Q$ are $5$ element posets.}
  \label{figure_1.1}
\end{figure}

We conclude this section with some more examples of Hasse diagrams.

\begin{example}\label{exmaple_1.8}
  \begin{enumerate}
    \item[(1)] The following Hasse diagrams are not graph-isomorphic.
      \[\begin{tikzcd}
  & 3 &&&&&& \bullet \\
  & 3 &&& \bullet && \bullet \\
        2 & 2 && \bullet & \bullet & \bullet & \bullet & \bullet \\
        1 & 1 && \bullet & \bullet & \bullet & \bullet & \bullet \\
        {\mathbf{2}} & {\mathbf{4}} &&& \bullet &&& \bullet \\
        1 & 2 & 3 &&&& \bullet \\
          & {\tbar{\mathbf{3}}}
          \arrow[no head, from=2-2, to=1-2]
          \arrow[no head, from=2-7, to=1-8]
          \arrow[no head, from=2-7, to=3-8]
          \arrow[no head, from=2-7, to=4-8]
          \arrow[no head, from=3-2, to=2-2]
          \arrow[no head, from=3-4, to=2-5]
          \arrow[no head, from=3-5, to=2-5]
          \arrow[no head, from=3-5, to=4-4]
          \arrow[no head, from=3-5, to=4-6]
          \arrow[no head, from=3-6, to=2-5]
          \arrow[no head, from=3-7, to=1-8]
          \arrow[no head, from=3-7, to=3-8]
          \arrow[no head, from=3-7, to=5-8]
          \arrow[no head, from=4-1, to=3-1]
          \arrow[no head, from=4-2, to=3-2]
          \arrow[no head, from=4-4, to=3-4]
          \arrow[no head, from=4-5, to=3-4]
          \arrow[no head, from=4-5, to=3-6]
          \arrow[no head, from=4-6, to=3-6]
          \arrow[from=4-7, to=1-8]
          \arrow[no head, from=4-7, to=4-8]
          \arrow[no head, from=4-7, to=5-8]
          \arrow[no head, from=5-5, to=4-4]
          \arrow[no head, from=5-5, to=4-5]
          \arrow[no head, from=5-5, to=4-6]
          \arrow[no head, from=6-7, to=3-8]
          \arrow[no head, from=6-7, to=4-8]
          \arrow[no head, from=6-7, to=5-8]
      \end{tikzcd}\]

    \item[(2)] The Hasse diagrams in figure \ref{figure_1.1}
      illustrate the central proof of proposition
      \ref{proposition_1.3.3}. Indeed, these two diagrams are
      graph-isomorphic with an underlying graph isomorphism $\Phi$.
      One can take the fact that in the graph $G(P)$, the edge
      $p(a)p(b)$ correspondes to the covering $a \covers b$, and use
      that to reconstruct the poset $P$. Moreover, one can also use
      this to construct the poste $Q$, and it's corresponding Hasse
      diagram.

    \item[(3)] Concider the group $\Uc(\faktor{\Z}{12\Z})$ and the
      Kelin-$4$ group $V_4$. These two groups are group-isomorphic with the
      following graph-isomorphic Hasse diagrams:
      \[\begin{tikzcd}
  & {\langle 5,7 \rangle} &&& {V_4} \\
        {\langle 5 \rangle} & {\langle 11 \rangle} & {\langle 7 \rangle} & {\langle a \rangle} & {\langle b \rangle} & {\langle c \rangle} \\
                            & {\langle 1 \rangle} &&& {\langle e \rangle}
                            \arrow[no head, from=1-2, to=2-2]
                            \arrow[no head, from=1-2, to=2-3]
                            \arrow[no head, from=1-5, to=2-4]
                            \arrow[no head, from=2-1, to=1-2]
                            \arrow[no head, from=2-2, to=3-2]
                            \arrow[no head, from=2-3, to=3-2]
                            \arrow[no head, from=2-4, to=3-5]
                            \arrow[no head, from=2-5, to=1-5]
                            \arrow[no head, from=2-6, to=1-5]
                            \arrow[no head, from=3-2, to=2-1]
                            \arrow[no head, from=3-5, to=2-5]
                            \arrow[no head, from=3-5, to=2-6]
      \end{tikzcd}\]
      In the sense of groups, one can use these diagrams to relate the
      group structure with the poset structure of the underlying
      ground set. Then it follows that two groups are group-isomorphic
      if, and only if their Hasse diagrams are graph-isomorphic if,
      and only if their underlying posets are order-isormorphic.

    \item[(4)] Now relate the group structure of the Klein-$4$ group
      with the group structure of $S_3$, the permutation group on $3$
      elements:
      \[\begin{tikzcd}
  && {S_3} \\
  &&& {\langle (1 \ 2 \ 3) \rangle} && {V_4} \\
        {\langle (1 \ 2) \rangle} & {\langle (1 \ 3) \rangle} & {\langle (2 \ 3) \rangle} && {\langle a \rangle} & {\langle b \rangle} & {\langle c \rangle} \\
                                  && {\langle (1) \rangle} &&& {\langle e \rangle}
                                  \arrow[no head, from=1-3, to=2-4]
                                  \arrow[no head, from=1-3, to=3-1]
                                  \arrow[no head, from=2-4, to=4-3]
                                  \arrow[no head, from=2-6, to=3-5]
                                  \arrow[no head, from=3-1, to=4-3]
                                  \arrow[no head, from=3-2, to=1-3]
                                  \arrow[no head, from=3-3, to=1-3]
                                  \arrow[no head, from=3-5, to=4-6]
                                  \arrow[no head, from=3-6, to=2-6]
                                  \arrow[no head, from=3-7, to=2-6]
                                  \arrow[no head, from=4-3, to=3-2]
                                  \arrow[no head, from=4-3, to=3-3]
                                  \arrow[no head, from=4-6, to=3-6]
                                  \arrow[no head, from=4-6, to=3-7]
      \end{tikzcd}\]
      one immediately sees that $S_3 \not\simeq V_4$ as groups. In
      general Hasse diagrams for group play a fundamental role in
      their study and classifications and the subgroup structure can
      be seen clearly with them. In actuallity, Hasse diagrams for
      groups satisfy the conditions to be called another structure
      called a lattice, which will be studied in chapter 2.
  \end{enumerate}
\end{example}
