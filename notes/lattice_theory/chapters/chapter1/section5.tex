\section{Sums and Products of Posets}
\label{section_1.5}

\begin{definition}
  Let $P$ and $Q$ be disjoint posets. We define the \textbf{disjoint
  union} of $P$ and $Q$ to be the poset $P \uplus Q$ together with
  the partial order $\leq$ defined by: For all $x,y \in P \uplus Q$
  $x \leq  y $ in $P \uplus Q$ if, and only if one and only one of
  the following hold:
  \begin{enumerate}
    \item[(1)] $x,y \in P$ and $x \leq y$ in $P$.

    \item[(2)] $x,y \in Q$ and $x \leq y$ in $Q$
  \end{enumerate}
\end{definition}

\begin{definition}
  Let $P$ and $Q$ be disjoint posets. The \textbf{linear sum} of $P$
  and $Q$ is the poset $P \cup Q$ together with the relation $\leq$
  given by: For all $x,y \in P \oplus Q$, $x \leq y$ if, and only if
  one and only one of the following hold:
  \begin{enumerate}
    \item[(1)] $x,y \in P$ and $x \leq y$ in $P$.

    \item[(2)] $x,y \in Q$ and $x \leq y$ in $Q$

    \item[(3)] $x \in P$ and $y \in Q$.
  \end{enumerate}
  We denote the linear sum of $P$ and $Q$ by $P \oplus Q$.
\end{definition}

\begin{proposition}\label{proposition_1.5.1}
  Let $P$ and $Q$ be disjoint posets. Then the disjoint union $P
  \uplus Q$ and the linear sum $P \oplus Q$ are posets under their
  respective partial orders.
\end{proposition}

\begin{proposition}\label{proposition_1.5.2}
  Let $P$ be a poset. Then $P_\bot \simeq \mathbf{1} \oplus P$.
\end{proposition}
\begin{proof}
  Let $P$ be a poset not containing $0 \in \N$ (if not, make it into
  one via an order-isomorphism), then $\mathbf{1} \cap P=\emptyset$,
  and form the union $\mathbf{1} \cup P$ and make it into $\mathbf{1}
  \oplus P$ under the order relation of linear sums. Now, let $\phi:P
  \xrightarrow{} P$ be the identity order-isomorphism, and extend
  $\phi$ to $\phi_\bot:P_\bot \xrightarrow{} \mathbf{1} \oplus P$, by
  sending $\phi:\bbar{0} \xrightarrow{} 0$, and taking
  $\phi_\bot=\phi(x)$ for all $x \in P$. Suppose that $x \leq y$ in
  $P_\bot$, then either $x=\bbar{0}$ or $x \leq y$ in $P$. Then
  $\phi_\bot(x)=0$, or $\phi_\bot(x)=\phi(x) \leq
  \phi(y)=\phi_\bot(y)$ in $P$ (which also forces $x,y \in P$). By
  definition of $\leq$ over $\mathbf{1} \oplus P$, $\phi_\bot(x) \leq
  \phi_\bot(y)$, and so $\phi_\bot$ is the relevant order-isomorphism.
\end{proof}

\begin{proposition}\label{proposition_1.5.3}
  Let $P$,  $Q$, and  $R$ pairwise disjoint posets. Then the following
  hold:
  \begin{enumerate}
    \item[(1)] $P \uplus (Q \uplus R)=(P \uplus Q) \uplus R$.
    \item[(2)] $P \oplus (Q \oplus R)=(P \oplus Q) \oplus R$.
  \end{enumerate}
\end{proposition}
\begin{proof}
  We only prove assertion (1) since assertion (2) follows from an
  identical argument, taking into account the partial order of linear
  sums instead of for disjoint unions.

  Let $x,y \in P \uplus (Q \uplus R)$, and suppose that $x \leq y$ in
  $P \uplus (Q \uplus R)$. Then either $x,y \in P$ and $x \leq y$ in
  $P$, or  $x,y \in Q \uplus R$ and $x \leq y \in Q \uplus R$. That
  is, either $x,y \in P$ and $x \leq y$ in $P$, or $x,y \in Q$ and $x
  \leq y$ in $Q$, or  $x,y \in R$ and $x \leq y$ in $R$. It follows
  then that $x,y \in P \uplus Q$ and $x \leq y$ in $P \uplus Q$, or
  $x,y \in R$ and $x \leq y$ in $R$; that is, $x,y \in (P \uplus Q)
  \uplus R$  and $x \leq y$ in $(P \uplus Q) \uplus R$. Therefore $P
  \uplus (Q \uplus R) \subseteq (P \uplus Q) \uplus R$. The reverse
  inclusion holds by an identical argument.
\end{proof}

\begin{example}\label{example_1.11}
  \begin{enumerate}
    \item[(1)] The Hasse diagrams for $\mathbf{2} \uplus \mathbf{3}$,
      $\mathbf{2} \oplus \mathbf{3}$, and $\tbar{\mathbf{1}} \oplus
      \mathbf{2} \oplus \mathbf{3}$ are drawn to be:
      \[\begin{tikzcd}
        && \bullet && \bullet \\
        & \bullet & \bullet & \bullet && \bullet \\
        \bullet & \bullet & \bullet \\
        \bullet & \bullet & \bullet & \bullet && \bullet \\
        {\mathbf{2} \uplus \mathbf{3}} && \bullet && \bullet \\
        && {\mathbf{2} \oplus \mathbf{3} \simeq \mathbf{5}} && {\tbar{\mathbf{1}} \oplus \tbar{\mathbf{2}} \oplus \tbar{\mathbf{3}}}
        \arrow[from=1-5, to=4-4]
        \arrow[from=1-5, to=4-6]
        \arrow[no head, from=2-3, to=1-3]
        \arrow[no head, from=3-2, to=2-2]
        \arrow[no head, from=3-3, to=2-3]
        \arrow[no head, from=4-1, to=3-1]
        \arrow[no head, from=4-2, to=3-2]
        \arrow[no head, from=4-3, to=3-3]
        \arrow[no head, from=4-4, to=2-4]
        \arrow[no head, from=4-4, to=2-6]
        \arrow[no head, from=4-6, to=2-4]
        \arrow[no head, from=4-6, to=2-6]
        \arrow[no head, from=5-3, to=4-3]
        \arrow[no head, from=5-5, to=4-4]
        \arrow[no head, from=5-5, to=4-6]
      \end{tikzcd}\]

    \item[(2)] Define $M_n=\mathbf{1} \oplus \tbar{\mathbf{n}} \oplus
      \mathbf{1}$. The Hasse diagrams for $M_3$, $M_2 \uplus M_3$, and
      $M_2 \oplus M_3$ are drawn below:
      \[\begin{tikzcd}
        && \bullet & {M_2 \oplus M_3} && \bullet \\
        & \bullet && \bullet & \bullet & \bullet & \bullet \\
        & \bullet & \bullet && \bullet & \bullet && \bullet \\
        \bullet & \bullet & \bullet & \bullet && \bullet & \bullet & \bullet & \bullet \\
        {M_3} & \bullet &&& \bullet & {M_2 \uplus M_3} && \bullet
        \arrow[no head, from=1-3, to=2-2]
        \arrow[no head, from=1-6, to=2-7]
        \arrow[no head, from=2-2, to=2-4]
        \arrow[no head, from=2-2, to=3-3]
        \arrow[no head, from=2-4, to=1-3]
        \arrow[no head, from=2-4, to=2-5]
        \arrow[no head, from=2-5, to=1-6]
        \arrow[no head, from=2-5, to=2-6]
        \arrow[no head, from=2-6, to=2-7]
        \arrow[no head, from=2-7, to=3-6]
        \arrow[no head, from=3-2, to=4-1]
        \arrow[no head, from=3-3, to=2-4]
        \arrow[no head, from=3-5, to=4-4]
        \arrow[no head, from=3-6, to=2-5]
        \arrow[no head, from=3-8, to=4-7]
        \arrow[no head, from=4-1, to=5-2]
        \arrow[no head, from=4-2, to=3-2]
        \arrow[no head, from=4-3, to=3-2]
        \arrow[no head, from=4-4, to=5-5]
        \arrow[no head, from=4-6, to=3-5]
        \arrow[no head, from=4-7, to=4-8]
        \arrow[no head, from=4-7, to=5-8]
        \arrow[no head, from=4-8, to=4-9]
        \arrow[no head, from=4-9, to=3-8]
        \arrow[no head, from=5-2, to=4-2]
        \arrow[no head, from=5-2, to=4-3]
        \arrow[no head, from=5-5, to=4-6]
        \arrow[no head, from=5-8, to=4-9]
      \end{tikzcd}\]
  \end{enumerate}
\end{example}

\begin{proposition}\label{proposition_1.5.4}
  Let $P_1, \dots, P_n$ be pairwise disjoint posets. Define the
  relation $\leq$ on $P_1 \times \dots \times P_n$ by: for all $x,y \in
  P_1 \times \dots \times P_n$, with $x=(x_1, \dots, x_n)$ and
  $y=(y_1, \dots, y_n)$
  \begin{equation*}
    x \leq y \text{ in } P_1 \times \dots \times P_n \text{ if, and only if }
    x_i \leq y_i \text{ in } P_i \text{ for all } 1 \leq i \leq n
  \end{equation*}
  Then $\leq$ is a partial order on  $P_1 \times \dots \times P_n$.
\end{proposition}

\begin{proposition}\label{proposition_1.5.5}
  Let $P$ and $Q$ be disjoint posets, and define the relation $\leq$ o
   $P \times Q$ by: for $x,y \in P \times Q$, where $x=(x_1,x_2)$ and
   $y=(y_1,y_2)$
   \begin{equation*}
     x \leq y \text{ in } P \times Q \text{ if, and only if }
     x_1 \leq y_1 \text{ or } x_1=y_1 \text{ and } x_2 \leq y_2 \text{ in } P \times Q
   \end{equation*}
   Then $\leq$ is a partial order on $P \times Q$.
\end{proposition}

\begin{definition}
  Let $P$ and $Q$ be disjoint posets. We call the order of proposition
  \ref{proposition_1.5.5} the \mathbf{lexicographical order} on $P
  \times Q$.
\end{definition}

\begin{example}\label{example_1.12}
  The following are some examples of Hasse diagrams of products of
  posets, under the ordering of proposition \ref{proposition_1.5.4}:
\[\begin{tikzcd}[column  sep=small, row  sep=tiny]
	&&&& \bullet &&& {\mathbf{2} \times\mathbf{3}} \\
	&& \bullet & \bullet && \bullet \\
	\bullet & \times & \bullet & {=} & \bullet && \bullet \\
	\bullet && \bullet &&& \bullet \\
	{\mathbf{2}} && {\mathbf{3}}
	\arrow[no head, from=1-5, to=2-6]
	\arrow[no head, from=2-4, to=1-5]
	\arrow[no head, from=2-6, to=3-7]
	\arrow[no head, from=3-3, to=2-3]
	\arrow[no head, from=3-5, to=2-4]
	\arrow[no head, from=3-5, to=2-6]
	\arrow[no head, from=3-7, to=4-6]
	\arrow[no head, from=4-1, to=3-1]
	\arrow[no head, from=4-3, to=3-3]
	\arrow[no head, from=4-6, to=3-5]
\end{tikzcd}\]
\[\begin{tikzcd}[column sep=small,row sep=small]
	\bullet && \bullet && \bullet && \bullet \\
	& \bullet && {\veq} && \bullet \\
	\bullet && \bullet && \bullet && \bullet \\
	&& \bullet && \bullet \\
	& \bullet &&&& \bullet \\
	&&& \bullet
	\arrow[no head, from=2-2, to=1-1]
	\arrow[no head, from=2-2, to=1-3]
	\arrow[no head, from=2-6, to=1-5]
	\arrow[no head, from=2-6, to=1-7]
	\arrow[no head, from=3-1, to=5-2]
	\arrow[no head, from=4-3, to=3-1]
	\arrow[no head, from=4-3, to=3-5]
	\arrow[no head, from=4-5, to=3-3]
	\arrow[no head, from=4-5, to=3-7]
	\arrow[no head, from=5-2, to=3-3]
	\arrow[no head, from=5-6, to=3-5]
	\arrow[no head, from=5-6, to=3-7]
	\arrow[no head, from=6-4, to=4-3]
	\arrow[no head, from=6-4, to=4-5]
	\arrow[no head, from=6-4, to=5-2]
	\arrow[no head, from=6-4, to=5-6]
\end{tikzcd}\]
\end{example}

\begin{figure}[h]
  \centering
  \[\begin{tikzcd}
    &&&& \bullet & \bullet \\
    &&& \bullet & \bullet & \bullet & \bullet \\
    & \bullet & \bullet && \bullet & \bullet \\
    \bullet & \bullet & \bullet & \bullet \\
    & \bullet & \bullet
    \arrow[no head, from=1-5, to=1-6]
    \arrow[no head, from=1-5, to=2-6]
    \arrow[no head, from=1-5, to=3-2]
    \arrow[no head, from=1-6, to=2-7]
    \arrow[no head, from=1-6, to=3-3]
    \arrow[no head, from=2-4, to=1-5]
    \arrow[no head, from=2-4, to=2-5]
    \arrow[no head, from=2-4, to=4-1]
    \arrow[no head, from=2-5, to=1-6]
    \arrow[no head, from=2-5, to=4-2]
    \arrow[no head, from=2-6, to=2-7]
    \arrow[no head, from=2-6, to=3-5]
    \arrow[no head, from=2-6, to=4-3]
    \arrow[no head, from=2-7, to=3-6]
    \arrow[no head, from=2-7, to=4-4]
    \arrow[no head, from=3-2, to=4-3]
    \arrow[no head, from=3-3, to=3-2]
    \arrow[no head, from=3-3, to=4-4]
    \arrow[no head, from=3-5, to=2-4]
    \arrow[no head, from=3-5, to=3-6]
    \arrow[no head, from=3-5, to=5-2]
    \arrow[no head, from=3-6, to=2-5]
    \arrow[no head, from=3-6, to=5-3]
    \arrow[no head, from=4-1, to=3-2]
    \arrow[no head, from=4-2, to=3-3]
    \arrow[no head, from=4-2, to=4-1]
    \arrow[no head, from=4-3, to=5-2]
    \arrow[no head, from=4-4, to=4-3]
    \arrow[no head, from=4-4, to=5-3]
    \arrow[no head, from=5-2, to=4-1]
    \arrow[no head, from=5-3, to=4-2]
    \arrow[no head, from=5-3, to=5-2]
  \end{tikzcd}\]
  \caption{The Hasse diagram on the tesseract $\mathbf{2}^4 =
  \mathbf{2} \times \mathbf{2} \times \mathbf{2} \times \mathbf{2}$.}
  \label{figure_1.5}
\end{figure}

\begin{proposition}\label{proposition_1.5.6}
  Let $X=\{1, \dots, n\}$, and define $\phi:2^X \xrightarrow{}
  \mathbf{2}^n$ by:
  \begin{equation*}
    \phi(A)=(\e_1, \dots, \e_n)
  \end{equation*}
  where
  \begin{equation*}
    \e_i=
    \begin{cases}
      1, &  \text{ if } i \in A \\
      1, &  \text{ if } i \not\in A \\
    \end{cases}
  \end{equation*}
  Then $\phi$ is an order-isomorphism between $2^X$ and
  $\mathbf{2}^n$.
\end{proposition}
\begin{proof}
  Let $A, B \subseteq X$, and $\phi(A)=(\e_1, \dots, \e_n)$ and
  $\phi(B)=(\d_1, \dots, \d_n)$. If $A \subseteq B$, then for every $i
  \in A$, $i \in B$, so that $\e_i=1$ implies  $\d_i=1$, so that
  $\e_i \leq \d_i$. Hence $\phi(A) \leq \phi(B)$. The reverse
  implication holds similarly, that is, $\phi(A) \leq \phi(B)$ implies
  $A \subseteq B$.

  Now, observe that if $\phi(A)=\phi(B)$, then $(\e_1, \dots,
  \e_n)=(\d_1, \dots ,\d_n)$ so $\e_i=\d_i$ for  all $1 \leq i \leq
  n$. By definition of $\e_i$, this implies that $A=B$. Moreover,
  suppose that $x=(\e_1, \dots, \e_n)$ in $\mathbf{2}^n$. Define $A
  \subseteq X$ to be the set $A=\{i \in X : \e_i=1\}$, then by
  definition of $\phi$ and $\e_i$, $x=\phi(A)$. Therefore $\phi$ is
  the required order-isomorphism between $2^X$ and $\mathbf{2}^n$.
\end{proof}
