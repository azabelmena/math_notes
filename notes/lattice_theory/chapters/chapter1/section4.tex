\section{Construction on Posets}
\label{section_1.4}

\begin{definition}
  Let $P$ be a poset. We define the \textbf{dual} of $P$ to be the
  same underlying set $P$ together with the binary relation
  $\leq^\partial$ defined by:
  \begin{equation*}
    x \leq^\partial y \text{ in } P \text{ if, and only if }
    y \leq x \text{ in } P
  \end{equation*}
  We denote the dual of $P$ under the relation $\leq^\partial$ by
  $P^\partial$.
\end{definition}

\begin{figure}[h]
  \centering
  \[\begin{tikzcd}
    && \bullet &&&&& \bullet \\
    \bullet && \bullet && \bullet & \bullet & \bullet & \bullet & \bullet & \bullet \\
    \bullet & \bullet & \bullet & \bullet & \bullet & \bullet && \bullet && \bullet \\
    && \bullet &&&&& \bullet
    \arrow[no head, from=1-8, to=2-6]
    \arrow[no head, from=1-8, to=2-7]
    \arrow[no head, from=1-8, to=2-8]
    \arrow[no head, from=1-8, to=2-9]
    \arrow[no head, from=1-8, to=2-10]
    \arrow[no head, from=2-1, to=1-3]
    \arrow[no head, from=2-3, to=1-3]
    \arrow[no head, from=2-5, to=1-3]
    \arrow[no head, from=2-7, to=3-6]
    \arrow[no head, from=2-8, to=3-8]
    \arrow[no head, from=2-8, to=3-10]
    \arrow[no head, from=2-9, to=3-10]
    \arrow[no head, from=2-10, to=3-10]
    \arrow[no head, from=3-1, to=2-1]
    \arrow[no head, from=3-2, to=2-1]
    \arrow[no head, from=3-3, to=2-1]
    \arrow[no head, from=3-3, to=2-3]
    \arrow[no head, from=3-3, to=2-5]
    \arrow[no head, from=3-4, to=2-5]
    \arrow[no head, from=3-5, to=2-5]
    \arrow[no head, from=3-6, to=2-6]
    \arrow[no head, from=3-6, to=2-8]
    \arrow[no head, from=3-8, to=4-8]
    \arrow[no head, from=3-10, to=4-8]
    \arrow[no head, from=4-3, to=3-1]
    \arrow[no head, from=4-3, to=3-2]
    \arrow[no head, from=4-3, to=3-3]
    \arrow[no head, from=4-3, to=3-4]
    \arrow[no head, from=4-3, to=3-5]
    \arrow[no head, from=4-8, to=3-6]
  \end{tikzcd}\]
  \caption{The Hasse diagram for the diherdral group $D_8$ and the
  corresponding diagram for its dual $D_8^\partial$.}
  \label{}
\end{figure}

\begin{lemma}\label{lemma_1.4.1}
  Let $P$ be a poset. The relation $\leq^\partial$ on  $P^\partial$
  defines a partial order.
\end{lemma}

\begin{lemma}\label{lemma_1.4.2}
  Let $P$ be a poset, and $\Pb(P)$ the space of predicates on $P$. We
  define the relation $\implies^\partial$ on $\Pb(P)^\partial$ by: For
  $p,q \in \Pb(P)$,
  \begin{equation*}
    p \implies ^\partial q \text{ in } \Pb(P^\partial) \text{ if, and
    only if }
    q \implies p \text{ in } \Pb(P)
  \end{equation*}
  Then $\implies^\partial$ defines a partial order on $\Pb(P)^\partial$.
\end{lemma}

\begin{definition}
  We define the \textbf{reverse} of a logical operator to be it's
  negation. Let $X$ be any set. We define a \textbf{statement} $\Phi$
  on $X$ to be a collection of predicates in $\Pb(X)$ together with
  logical operators and the implication order $\implies$. We define
  the \textbf{dual statement} $\Phi^\partial$ on $X$ to be the
  statement whose predicates reside in $\Pb(X)^\partial$, and whose
  logical operators are the reverse of those in $\Phi$, together with
  the dual implication $\implies^\partial$.
\end{definition}

\begin{proposition}[The Duality Principle on Posets]\label{proposition_1.4.3}
  Let $P$ be a finite poset. Then a statement $\Phi$ on $P$ is true
  if, and only if its dual statement $\Phi^\partial$ on $P^\partial$
  is true.
\end{proposition}
\begin{proof}
  Let $P$ be a poset, and take $p,q \in \Pb(P)$ predicates on $P$.
  Without loss of generality, suppose we just have the statement
  $\Phi: p \implies q$ on $\Pb(P)$, then we have the dual statement
  $\Phi^\partial:q \implies p$ on $\Pb(P)^\partial$. Suppose that
  $\Phi$ is true, and for every $x \in P$ that $q(x)=T$, then $p(x)=T$
  as well. Now, suppose that statement $\Phi^\partial$ is false.
  Since $q(x)=T$ on $\Pb(p)$, and hence $\Pb(P)^\partial$ we must
  have that $p(x)=F$ for every $x \in P$. But $p(x)=T$ for every $x
  \in P$, a contradiction! Therefore $\Phi^\partial$ must be true on
  $\Pb(P)^\partial$.

  The reverse implication of this proposition holds identically.
\end{proof}

\begin{definition}
  Let $P$ be a poset. We say that $P$ has a \textbf{top element} if
  there exists a $\top \in P$ such that $x \leq \top $ for all $x \in
  P$. We say that $P$ has a \textbf{bottom element} if there exists a
  $\bot \in P$ such that $\bot \leq x$ for all $x \in P$.
\end{definition}

\begin{lemma}\label{lemma_1.4.4}
  Let $P$ be a poset for which both a bottom element and a top element
  exist. Then they are unique.
\end{lemma}
\begin{proof}
  Suppose that $P$ has a top element $\top_1$, and suppose there
  exists another top element $\top_2 \in P$. Then by definition, we
  have $\top_1 \leq \top_2$, and $\top_2 \leq \top_1$, so that
  $\top_1=\top_2$ by antisymmetry. By the duality principle on posets,
  this also proves the dual statement of $P$ having a unique bottom
  element.
\end{proof}

\begin{example}
  \begin{enumerate}
    \item[(1)] Consider the chain $\N$, then by definition of $\N$, it
      has a bottom element $0$.

    \item[(2)] The chains $\Z$, and $\R$ have no bottom, nor top
      elements.

    \item[(3)] Any anti-chain with more than one element posess
      neither a top element nor a bottom element.

    \item[(4)] Any finite chain posesses a top and a bottom element,
      as illustrated by the following diagram below.
      \[\begin{tikzcd}
        \bullet & \bullet & \bullet & \bullet
        \arrow[no head, from=1-1, to=1-2]
        \arrow[no head, from=1-2, to=1-3]
        \arrow[dashed, no head, from=1-3, to=1-4]
      \end{tikzcd}\]

      \item[(5)] Let $X$ be a set, then  $\emptyset$ and $X$ are the
        bottom and top elements of $2^X$, respectively.

      \item[(6)] Let $X$ and $Y$ be nonempty sets, and let $f:X
        \xrightarrow{} Y$ be a function. Then the top element of $X
        \partialmap Y$ is $f$ itself. None of the previous other
        examples have top elements however.

      \item[(7)] In $\Sigma^\ast$ and $\Sigma^{\ast\ast}$, the bottom
        elements are the empty binary string, denoted $\{\}$.

      \item[(8)] In $X \partialmap Y$, the bottom element is the
        partial map with empty domain.

      \item[(9)] In $[\bbar{\R},\tbar{\R}]$, the bottom element is the
        interval $[-\infty, \infty]$.
  \end{enumerate}
\end{example}

\begin{proposition}\label{proposition_1.4.5}
  Let $P$ be an ordered set, and define $P_\bot=P \cup \{\bbar{0}\}$,
  for some $\bbar{0} \in P$. Define the binary relation $\leq$ on $P$
  by: for $x,y \in P_\bot$
  \begin{equation*}
    x \leq y \text{ in } P_\bot \text{ if, and only if }
    x=\bbar{0} \text{ or } x \leq y \text{ in } P
  \end{equation*}
  Then $\leq$ defines a partial order on $P_\bot$.
\end{proposition}

\begin{definition}
  Let $P$ be a poset. We call the poset  $P_\bot$, together with the
  order relation defined in propostion \ref{proposition_1.4.5} the
  \textbf{lift} of $P$.
\end{definition}

\begin{definition}
  Let $S$ by any set. We define the \textbf{flat} on $S$ to be the
  poset $\tbar{S}_\bot$, ordered by the order relation of propostion
  \ref{proposition_1.4.5}. That is, it is $S$ made into an anti-chain
  and then lifted.
\end{definition}

\begin{figure}[h]
  \centering
  \[\begin{tikzcd}
    \bullet & \bullet & \bullet & \bullet & \dots & \bullet & \dots \\
            &&& \bullet & {\tbar{\N}_\bot}
            \arrow[no head, from=2-4, to=1-1]
            \arrow[no head, from=2-4, to=1-2]
            \arrow[no head, from=2-4, to=1-3]
            \arrow[no head, from=2-4, to=1-4]
            \arrow[dashed, no head, from=2-4, to=1-5]
            \arrow[draw=none, from=2-4, to=1-6]
            \arrow[no head, from=2-4, to=1-6]
            \arrow[dashed, no head, from=2-4, to=1-7]
  \end{tikzcd}\]
  \caption{Hasse diagram of $\tbar{\N}_\bot$}
  \label{figure_1.3}
\end{figure}

\begin{definition}
  Let $P$ be a poset, and $Q \subseteq P$ with the induced order. We
  call an element $a \in Q$  \textbf{maximal} if for every $x \in Q$,
  $a \leq x$ implies $a=x$. We call an element $b \in Q$
  \textbf{minimal} if for every $y \in Q$, $y \leq b$ implies $b=y$.
  We define the set of maximal and minimal elements of $Q$ by
  $\Max{Q}$ and $\Min{Q}$.
\end{definition}

\begin{lemma}\label{lemma_1.4.6}
  Let $P$ be a poset, and  $Q \subseteq P$. If $Q$ has a top element
  $\top_Q$, then $\Max{Q}=\{\top_Q\}$. Similarly, if $Q$ has a bottom
  element $\bot_Q$, then $\Min{Q}=\{\bot_Q\}$.
\end{lemma}

\begin{definition}
  Let $P$ be a poset, and $Q \subseteq P$ with the induced order. If
  $Q$ has a top element $\top_Q$, then we say that $\top_Q$ is the
  \textbf{greatest element} of $Q$. If $Q$ has a bottom element
  $\bot_Q$, then we say that $\bot_Q$ is the \textbf{least element} of
  $Q$. We write $\top_Q=\max{Q}$ and $\bot_Q=\min{Q}$.
\end{definition}

\begin{figure}[h]
  \centering
  \[\begin{tikzcd}
    &&&&&&& a \\
    {a_1} & {a_2} && {a_3} && \bullet \\
    &&&& \bullet \\
    &&&&& \bullet &&& \bullet \\
    & \bullet & {P_1} & \bullet &&& \bullet & {P_2} && \bullet
    \arrow[no head, from=1-8, to=2-6]
    \arrow[no head, from=1-8, to=4-6]
    \arrow[no head, from=1-8, to=4-9]
    \arrow[no head, from=2-2, to=5-2]
    \arrow[no head, from=2-4, to=5-4]
    \arrow[no head, from=2-6, to=3-5]
    \arrow[no head, from=4-9, to=5-10]
    \arrow[no head, from=5-2, to=2-1]
    \arrow[no head, from=5-2, to=2-4]
    \arrow[no head, from=5-4, to=2-2]
    \arrow[no head, from=5-7, to=4-6]
    \arrow[no head, from=5-7, to=4-9]
  \end{tikzcd}\]
  \caption{The Hasse diagrams of the posets $P_1$ and $P_2$. Here
    $P_1$ has the maximal elements $a_1, a_2, a_3$, but no greatest
  element. $P_2$ has the greatest element $\max{P_2}=a$.}
  \label{figure_1.4}
\end{figure}

\begin{lemma}\label{lemma_1.4.7}
  Any finite nonempty subset of $\N$ has a maximal element.
\end{lemma}

\begin{proposition}\label{proposition_1.4.8}
  Let $P$ be a finite poset. Every nonempty subset of $P$ has at least
  one maximal element. Moreover, for every $x \in P$, there exists a
  $y \in \Max{P}$ for which $x \leq y$.
\end{proposition}
\begin{proof}
  Let $Q \subseteq P$ be nonempty. Since $P$ is finite, then so is
  $Q$, and hence there exists a bijection  between $Q$ and a finite
  nonempty subset of $\N$. Then by lemma \ref{lemma_1.4.7}, $\N$ has a
  maximal element. Now, observe that the bijection between $Q$ and
  $\N$ is als an order-isomorphism, so by order preservation, the
  maximal element of $\N$ must correspond to a maximal element of $Q$.

  Now, by the above argument, observe that $P$ is a finite nonempty
  subset of itself, and hence has at least one maximal element $y \in
  \Max{P}$. Then since $y$ is maximal in $P$, by definition there must
  be an $x \in P$ such that $x \leq y$.
\end{proof}

\begin{example}\label{example_1.10}
  \begin{enumerate}
    \item[(1)] The set $\com{2^\N}{\N}$ has no top element, however,
      notice that for every $n \in \N$, $\com{\N}{\{n\}} \in
      \Max{\com{2^\N}{\N}}$ is a maximal element.

    \item[(2)] The collection of all finite subsets of $\N$ has no
      maximal element.

    \item[(3)] There are no maximal elements in $\Sigma^\ast$, however
      the maximal elements of  $\Sigma^{\ast\ast}$ are all infinite
      binary strings.

    \item[(4)] Let $X$ and $Y$ be sets. The maximal elements of $X
      \partialmap Y$ are precisely the total maps.

    \item[(5)] The maximal elements of $[\bbar{\R},\tbar{\R}]$ are the
      $1$-element intervals, i.e. degenerate intervals.
  \end{enumerate}
\end{example}
