%----------------------------------------------------------------------------------------
%	SECTION X.X
%----------------------------------------------------------------------------------------

\section{The Bolano-Weierstrass Theorem.}

\begin{enumerate}[label=(\arabic*)]
    \item[(1)] Notice that $\sin{n^3}$ is bounded by  $1$  ($|\sin{x}| \leq 1$ 
    for all $x \in \R$), then the sequence $\{x_n\}$ defined by:
        \begin{equation*}
            x_n=\frac{(n^2+20n+35)\sin{n^3}}{n^2+n+1}
        \end{equation*}
    is bounded as well. Thus by the Bolzano-Weierstrass theorem, $\{x_n\}$ has 
    a convergent subsequence.

    \item[(2)] Let $E \susbeteq \R$ be nonempty and bounded, and suppose that  
        $\sup{E} \notin \R$. Let $x=\sup{E}$, and let  $\{x_n\} \subseteq E$ be a realvalued 
        sequence such that  $x_n<x_{n+1}$. Then  $\{x_n\}$ is strictly inclreasing, 
        and bounded  (since $E$ is bounded), hence, by the monotone convergence 
        theorem, $\lim{x_n}$ exists and is finite. All that is left is to show 
        that  $\lim{x_n}=x$; for suppose not. First suppose that $\lim{x_n}>x$, 
        then let $x_k=x$ for  $k \geq n$, then  $x_n < x_k$, and so  $x \in \{x_n\}$, 
        which contradicts the suppositon that  $x \notin E$. Now suppose that  
        $\lim{x_n}<x$, then there is some  $k \geq n$ for which $x_n \leq x_k$, which 
        violates the supposition that  $\{x_n\}$ is strictly increasing. Thus we 
        must have $\lim{x_n}=x$ as  $n \rightarrow \infty$. Similarly, if we suppose that 
        $\inf{E} \notit E$, we get that there is a strictly decreasing sequence 
        $\{y_n\}$ converging to  $\inf{E}$, where  $y_n \in E$ for all  $n$.

    \item[(4)] By exercise  $1.1.5(a)$, we have  $x_{n+1}<x_n$ for all  $n$, so 
        $\{x_n\}$ is strictly decreasing. Then choose  $I=(0,1)$, and let  
        $\{I_n\} \subseteq I$ be a sequence of nested intervals. Notice that $|I_n| \rightarrow 0$, 
        as $n \rightarrow \infty$. So, by the nested interval theorem, choose 
            \begin{equation*}
                E=\bigcap{I_n}=(0,x_k)
            \end{equation*}
        for some $k \geq n$, we see that $x_k=0$, and since we suppose that $x_k<x_n$, 
        we see that  $x_n \rightarrow 0$.

        Now consider the sequence  $\{\frac{x_{n+1}}{x_n}\}$, rationalizing 
        $\frac{x_{n+1}}{x_n}$, we get $\frac{x_{n+1}}{x_n}=\frac{1}{2}+A$ for some 
        term $A$. Then the sequence  $A \rightarrow 0$, so $\lim{\frac{x_{n+1}}{x_n}}=
        \lim{(\frac{1}{2})+A}=\lim{\frac{1}{2}}+\lim{A}=\frac{1}{2}$.

    \item We see that of $x_{n+1}=\sqrt{2x_n+3}$, then  $x_n=\frac{x_{n+1}^2-3}{2}$, 
        so $x_n<x_{n+1}$, so  $\{x_n\}$ is strictly increasing. Again, by the nested 
        interval property for  $I=(0,3]$, we see that $\lim{x_n}=3$.

    \item By exercise  $1.1.5(b)$, we have that  $x_{n+1}<x_n$, so  $\{x_n\}$ is 
        strictly decreasing, moreover by the nested interval theorem, choosing  $I=(2, \infty)$, 
        we see that  $\lim{x_n}=2$. Now what happens when we suppose  $1 \leq x_1<2$? 
        If we let $x_1=1$, then $x_2=1$. If we let  $x_1=\frac{3}{2}$, then we see 
        that $x_2=1+\frac{\sqrt{2}}{2}$, hence $x_{n+1} \leq x_n$, so $\{x_n\}$ is 
        monotone increasing. By the same reasoning as before, we also see $\lim{x_n}=2$.

\end{enumerate}
