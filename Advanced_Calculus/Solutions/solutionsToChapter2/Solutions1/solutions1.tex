%----------------------------------------------------------------------------------------
%	SECTION X.X
%----------------------------------------------------------------------------------------

\section{Limits of Sequences}

\begin{enumerate}[label=(\arabic*)]
    \item[1]
        \begin{enumerate}

        \item Consider the sequence $\{3+\frac{1}{n}\}$ and choose $N>\frac{1}{\epsilon}$, 
        then for $n \geq N$,  $\frac{1}{n} \leq \frac{1}{N}<\epsilon$, then 
        $|\frac{1}{n}|=|(3+\frac{1}{n})-3|<\epsilon$, thus $\lim{(3+\frac{1}{n})}=3$ 
        as $n \rightarrow \infty$.
            
        \item Choose $N>\frac{2}{\epsilon}$, then $\frac{1}{n} \leq \frac{1}{N}<\frac{\epsilon}{2}$, 
        implies $|\frac{1}{n}|=2|(1-\frac{1}{n})-1|<2\frac{\epsilon}{2}=\epsilon$.

        \item Choose $N>\frac{1}{\sqrt{\epsilon}}$. 

        \item Choose $N>\frac{1}{\epsilon^2}$.
        \end{enumerate}

    \item[4] 
        \begin{enumerate}
            \item Let $\{b_n\} \rightarrow 0$ and let  $\{x_n\}$ be a realvalued 
            sequence such that  $|x_n-a| \leq b_n$ for  $n$ sufficiently large. 
            Then, for any  $\epsilon>0$, there is an  $N \in \N$ such  that whenever 
            $n \geq N$,  $|b_n|<\epsilon$, that is for the same  $N$,  $||x_n-a|| \leq |b_n|<\epsilon$. 
            Hence,  $x_n \rightarrow a$ as  $n \rightarrow \infty$.

        \item Now suppose that  $\{b_n\}$ and  $\{x_n\}$ are the same sequences as 
            before such that  $|x_n-a| \leq Cb_n$ for some  $C \in \R$. Then for any 
             $\epsilon>0$, choose  $N=\frac{\epsilon}{C}$ so that $C|b_n|<\epsilon$.  
             Then  for the same  $N$, we have  $|x_n-a| \leq C|b_n|<\epsilon$, and we 
             get the same result.
        \end{enumerate}

    \item[5]
        \begin{enumerate}
            \item If $\{x_n\}$ is a bounded realvalued sequence, then by definition, 
            there is a  $C>0$ such that  $|x_n| \leq C$. Conversely, supoose that 
            there is a $C>0$ for which  $|x_n| \leq C$, then  $-C \leq x_n \leq C$. So 
            we see that  $\{x_n\}$ is bounded below  (by $-C$) and bounded above 
            (by $C$), thus $\{x_n\}$ is bounded.

            \item If $\{x_n\}$ is a bounded realvalued sequence, then, $|x_n| \leq C$ 
            for $C>0$. Thus  $|\frac{x_n}{n^k}| \leq \frac{C}{|n^k|}$. Thus, for 
            any $\epsilon>0$, choose  $N=\frac{1}{\sqrt[k]{\epsilon}}$, thus we get that for 
            $n \geq N$,  $|\frac{x_n}{n^k}| \leq \frac{C}{|n^k|}<\epsilon$. Hence, $\frac{x_n}{n^k} 
            \rightarrow 0$ as $n \rightarrow \infty$.
        \end{enumerate}

    \item[6]
        \begin{enumerate}
            \item Let $x_n \rightarrow a$ and let  $y_n \rightarrow at$, then for any  $\epsilon>0$, there 
            is are  $N_1,N_2 \in \N$ such that $|x_n-a|<\epsilon$ and  $|y_n-a|<\epsilon$. 
            Choosing  $N=\max\{N_1,N_2\}$, we get  $n \geq N$  $|x-n-y_n|=|((x_n-y_n)
            -(a-a)| \leq |x_n-a|+|y_n-a|<\epsilon$. Therefore  $x_y-y_n \rightarrow 0$ as  $n \rightarrow \infty$.

        \item Consider $\{n\}$ for all $n \in \N$. Then  $n$ is either even or odd, 
            so choose subsequences $\{2k\}$ and  $\{2k+1\}$ for  $k \in \N$. Now 
            suppose that  $\lim{2k}=\lim{(2k+1)}=a$ as  $k \rightarrow \infty$. 
            Then $(2k+1)-(2k)=1 \rightarrow 0$, a contradiction ($\{1\}$ is a constant 
            sequence), thus  $\{n\}$ is divergent.

        \item Consider the constant sequence  $\{1\}$ and the sequence  $\{\frac{1}{n}\}$ 
            with $n \in \N$, then  $1-\frac{1}{n}=\frac{n-1}{n} \rightarrow 0$, 
            but $1 \rightarrow 1$ and $\frac{1}{n} \rightarrow 0$ as $n \rightarrow \infty$. 
            Hence the converse of  $(a)$ is not necesarily true.
        \end{enumerate}
\end{enumerate}
