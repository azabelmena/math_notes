%----------------------------------------------------------------------------------------
%	SECTION X.X
%----------------------------------------------------------------------------------------

\section{Two sided Limits}

\begin{enumerate}[label=(\arabic*)]
    \item
        \begin{enumerate}
        \item Let $\epsilon>0$, we have that  $x^2-x+1-3=(x-2)(x+2)$. If  $0 \leq \delta \leq 1$, then 
            $|x-2|<\delta$ implies  $0<x<4$. Now by the triangle inequality, $|x-2| \leq
            |x|+2<6$, so choose  $\delta=\min\{1,\frac{\epsilon}{6}\}$, then we get $0<|x-2|<\delta$ 
            implies that  $|f(x)-3|=|x-2||x+1|<6\delta \leq \epsilon$. Thus  $\lim{x^-x-1}=3$ 
            as  $x \rightarrow 2$.

        \item Notice that when  $x \neq 0$, that $ \frac{x^2-1}{x-1}=x+1$. Choose $\delta=\epsilon$, 
            then whenever  $0<|x-1|<\delta$,  $|x+1-2|=|x-1|< \delta=\epsilon$.

        \item We have that $x^3+x+1-3=x^3+x-2=(x-1)(x^2+x+2)$. If  $0 < \delta \leq 1$, 
            then  $|x-1| < \delta$ implies $0<x<2$. Thus by the triangle inequality, 
            $|x-1| \leq |x|+1 \leq 3$, so choose $\delta=\min\{1,\frac{\epsilon}{3}\}$. Then 
            $|x^3+x+1-3|=|x-1||x^2+x+2|<3\delta \leq epsilon$.
        \end{enumerate}

    \item
        \begin{enumerate}
        \item $\lim{\cos{\frac{1}{x}}}=\lim{\sin{\frac{2x\pi-1}{x}}}$, which diverges at 
            $0$, so  $\lim{\cos{\frac{1}{x}}}$ does not exist as $x \rightarrow 0$.

        \item $|\sin{\frac{1}{x}}| \leq 1$, so $|x\sin{ \frac{1}{x}}| \leq |x|$. Since 
            $|x| \rightarrow 0$ as  $x \rightarrow 0$, by the sandwich theorem, we have that 
            $|x\sin{\frac{1}{x}}| \rightarrow 0$ as $x \rightarrow 0$.

        \item Notice that  $\log{1}=0$, Then  $\lim{\frac{1}{\log{1}}}=\frac{1}{\lim{\log{1}}}$ 
            which diverges as $x \rightarrow 1$.
        \end{enumerate}

    \item
        \begin{enumerate}
            \item $\lim{\frac{x^2+\cos{x}}{2-\tan{x}}}=\frac{1}{2}$.

            \item $\lim{\frac{x^2+x-2}{x^3-x}}=\frac{3}{2}$.

            \item Let $y=\pi-x^2$, then  $x=\sqrt{\pi-y}$. at  $x=\sqrt{\pi}$, we have 
                that  $y=0$. Then take  $\lim{\frac{\sqrt[3]{y}}{\pi+\sqrt{\pi-y}}}=0$ as 
                $y \rightarrow 0$. Thus  $\lim{\frac{\sqrt[3]{\pi-x^2}}{x+\pi}}=0$ as 
                $x \rightarrow \pi$.

            \item We have  $ \frac{x^n-1}{x-1}=x^{n-1}+x^{n-2}+\dots+x+1$. Thus 
                $\lim{\frac{x^n-1}{x-1}}=n-1$ as $x \rightarrow 1$.

            \item We have $|x\sin{\frac{1}{x^2}}| \leq |x|$, letting $x=1-\cos{y}$, we 
                have  $x \rightarrow 0$ as  $y \rightarrow 0$. Thus  $cos{y}-1 \leq 
                |x\sin{\frac{1}{x^2}}| \leq 1-\cos{y}$, by the sandwich theorem, $|x\sin{\frac{1}{x^2}}| 
                \rightarrow 0$ as $y \rightarrow 0$.
        \end{enumerate}

    \item Let $\epsilon>0$, and consider  $|x^n\sin{x}-L|$.  We have that  $0<|x|<\delta$ 
        implies that  $0<|x|^n<\delta^n$, so choose  $\delta=\sqrt[3]{\epsilon}$. Then $0<|x|<\delta$ 
        implies that  $|x^n\sin{x}-L| \leq |x|^n|\sin{x}|-L<\delta^n|\sin{x}|+L<\epsilon|\sin{x}|-L$. 
        Since  $|\sin{x}| \leq 1$, for all  $x$, we have $\epsilon|\sin{x}|-L<\epsilon-L<\epsilon$. Thus 
        $\lim{x^n\sin{\frac{1}{x}}}$ exists for all $n \in \N$.
\end{enumerate}

