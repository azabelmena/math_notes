%----------------------------------------------------------------------------------------
%	SECTION 4.1
%----------------------------------------------------------------------------------------

\section{The Derivative.}

\begin{definition}
    A realvalued function $f:I \rightarrow \R$ (with $I$ an open interval) is said to be 
    \textbf{differentiable} at a point  $a \in I$, if and only if $f$ is defined 
    over all of  $I$, except possibly at  $a$, and 
        \begin{equation}
            f'(a)=\lim_{h \rightarrow 0}{\frac{f(a+h)-f(a)}{h}}=\lim_{x \rightarrow a}
            {\frac{f(x)-f(a)}{x-a}}		
        \end{equation} 
        exists. We call $f'(a)$ the \textbf{derivative} of $f$ at  $a$.
\end{definition}

This definition, illustrated geometrically, is given by the line tangent to the 
function $f$ at $a$. That is, if  $\frac{f(x)-f(a)}{x-a}$ is the slope of the 
secant cutting $f$ at  $a$, then the derivative is the limit of that slope as  $x \rightarrow a$; 
which gives us the ``instantaneous'' rate of change of $f$ at  $a$.

It need not be that  $f$ is differentiable at every point in its domain. However, 
if  $f'$ is defined on every point in the domain of  $f$, then we call  $f'$ the 
\textbf{derivative function}. Alternative notations for the derivative of $f'$ include: 
$D_x{f}$,  $ \frac{dy}{dx}$, and $f^{(1)}$.

Now if $f'$ is also differentiable on  a domain $E$, then the \textbf{excond derivative} 
of $f$ is the derivative of  $f'$, denoted by  $f''$. Similarly, we define the  $n^{th}$ 
derivative recursively to be:
    \begin{enumerate}[label=(\arabic*)]
         \item $f^{(0)}=f$, $f^{(1)}=f'$.

         \item $f^{(n+1)}=(f^{(n)})'$.
    \end{enumerate}

\begin{theorem}\label{4.1.1}
    A realvalued function $f$ is differentiable at some point  $a \in \R$ if 
    and only if there exists an open interval  $I$ and a function  $F:I \rightarrow \R$ 
    such that  $a \in \R$,  $F$ is continous at point $a$, and  $f(x)=F(x)(x-a)+f(a)$ 
    for all  $x\in I$; in which case,  $F(a)=f'(a)$.
\end{theorem}
\begin{proof}
    First, when $x \neq a$, we have  $F(x)=\frac{f(x)-f(a)}{x-a}$, which satisfies 
    $f(x)=F(x)(x-a)+f(a)$. Now suppose that  $f$ is differentiable at  $a$. Then 
    $\lim{\frac{f(x)-f(a)}{x-a}}=\lim{F}=F(a)$ as $x \rightarrow a$, and so  $F$ is 
    continous at  $a$; and so  $f'(a)=F(a)$.

    Now if  $F$ exists, and  $f(x)=F(x)(x-a)+f(a)$, then  $\lim{\frac{f(x)-f(a)}{x-a}}= 
    \lim{F}=F(a)$ as $x \rightarrow a$. So  $f$ is differentiable at $a$ with $f'(a)=F(a)$
\end{proof}

\begin{theorem}\label{4.1.2}
    Let $f:\R \rightarrow \R$ be a realvalued function. Then $f$ is differentiable at  $a$ 
    if and only if there is a function  $T$ of the form  $T(x)=mx$ such that:
        \begin{equation}
            \lim_{h \rightarrow 0}{\frac{f(a+h)-f(a)-T(h)}{h}=0}
        \end{equation} 
\end{theorem}
\begin{proof}
    Suppose that $f$ is differentiable at  $a$, with  $m=f'(a)$. Let  $T(x)=mx$, then 
    $\lim{ \frac{f(a+h)-f(a)-f'(a)h}{h}}=\lim{\frac{f(a+h)-f(a)}{h}-f'(a)}=f'(a)-f'(a)=0$, 
    as $h \rightarrow 0$. Conversely if $\lim{\frac{f(a+h)-f(a)}{h}-m}=\lim{\frac{f(a+h)-f(a)-mh}{h}}=0$ 
    as $h \rightarrow 0$. Threrefore,  $f$ is differentiable at  $a$, with  $f'(a)=m$.
\end{proof}

\begin{theorem}\label{4.1.3}
    If $f$ is differentiable at  $a$, then  $f$ is continous at  $a$.
\end{theorem}
\begin{proof}
    By theorem \ref{4.1.1}, $f(x)=F(x)(x-a)+f(a)$ is continuous.		
\end{proof}

What theroem \ref{4.1.3} says that discontinuous functions are never differentiable at 
their points of discontinuity. Likewise, the converse of this theorem is not true; we may 
have continuous functions that are not differentiable at their points of continuity.

\begin{example}
    The absolute value function $|x|$ is continous everywhere, but it is not 
    differentiable at $x=0$.
\end{example} 

\begin{definition}
    Let $I$ be a nondegenerate interval. A function  $f:I \rightarrow \R$ is said to be 
    \textbf{differentiable} on  $I$ if and only if :
         \begin{equation}
             f'_I(a)=\lim_{x \rightarrow a}{\frac{f(x)-f(a)}{x-a}}
        \end{equation} 
    for all $x \in I$, and is finite on all of $I$. $f$ is said to be \textbf{continuously differentiable} 
    on  $I$ if and only if  $f'_I$ exists and is continous for every point in  $I$.
\end{definition}

\begin{example}
    Consider $f(x)=x^{\frac{3}{2}}$, then $f'(x)=\frac{3}{2}\sqrt{x}$ is 
    continous on the interval$ [0, \infty)$. So $f$ is continously differentiable on $[0,\infty)$. 
    However,  $f'$ is not continuously differentiable on  $[0, \infty)$.
\end{example}

We denote the set of all realvalued functions whose $n^{th}$ derivative exists, and 
are contiuous on a domain  $I$ by  $C^n(I)$. We denote  $C^{\infty}(I)$ to be the 
set of all  realvalued functions that are infinitely contiuously differentiable on a domain 
 $I$.

  \begin{example}
      Let $f(x)=x^2\sin{\frac{1}{x}}$ for $x \neq 0$, and $f(x)=0$ when $x=0$. Then 
      $f$ is differentiable on $\R$, but not continuously differentiable on  $\R$ (it is 
      not contiuously differentiable on any interval containing $0$).
 \end{example} 
