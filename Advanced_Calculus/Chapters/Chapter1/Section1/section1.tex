%----------------------------------------------------------------------------------------
%	SECTION X.X
%----------------------------------------------------------------------------------------

\section{The Ordered Field Axioms}

We want to study the algebraic structure of the real numbers; the so called set $\R. $We assume standard knowlege of set 
theory. We display the field axioms below:

\begin{postulate}\label{postulate1}
  There are binary operations $+$ and $\cdot$ defined on the set $\R \times \R$, satisfying the following properties $\forall a,b,c \in \R:$

    \begin{enumerate}[label = (\arabic*)]
      \item $a+b \in \R$.

      \item $a+(b+c)=(a+b)+c$ and $a \cdot (b \cdot c)= (a \cdot b) \cdot c$.

      \item $a+b=b+a$ and $a \cdot b = b \cdot a$.

      \item $a \cdot (b+c) = (a \codt b)+(a \cdot c)$.

      \item There exists a unique element $0 \in \R$ such that $\forall a \in \R$, $a+0=a$.

      \item There exists a unique element $1 \neq 0 \in \R$ such that $\forall a \in \R$, $a \cdot 1=a$.

      \item For each $a \in \R$, there exists a unique element $-a \in \R$ such that $a+(-a)=0$.

      \item For each $a \in \R\backslash{\{0\}}$, there exists a unique element $a^{-1} \in \R$ such that $a \cdot a^{-1}=1$.
    \end{enumerate}
\end{postulate}

We call postulate \ref{postulate1} the \textbf{field axioms} for real numbers; in essence what it says is that the pairs 
$(\R,+)$ and $(\R,\cdot)$ form abelian groups, and that $\cdot$ distributes over $+$. W now get the following properties from the field axioms:

\begin{enumerate}[label=(\arabic*)]
   \item $(-1)(-1)=1$.

   \item $0a=0$.

   \item $-(a-b)=b-a$.

   \item For $a,b \in \R$ and $ab=0$, then either $a=0$ or $b=0$.
\end{enumerate}

The proofs for these are relatively easy, and can be done at anytime as an exercise. We now introduce the second postulalte 
called the \textbf{order axioms}:

\begin{postulate}
  There is a relation $<$ on $\R \times \R$ satisfying the following properties:
    \begin{enumerate}[label=(\arabic*)]
      \item Given $a,b \in \R$, one and only one of the following hold: either $a<b$, $b<a$, or $a=b$.

      \item If $a<b$ and $b<c$, then $a<c$.

      \item If $a<b$ and $c \in \R$, then $a+c<b+c$.

      \item If $a<b$ and $c>0$, then $ac<bc$; if $c<0$, then $bc<ac$.
    \end{enumerate}
\end{postulate}

We may may the following remarks: by $b>a$ we mean that $a<b$, by $a \leq b$, we mean that $a<b$ or $a=b$, and by $a<b<c$, 
we mean that $a<b$ and $b<c$.

\begin{definition}
  For $a \in \R$, $a$ is called \textbf{nonnegative} if $a \geq 0$, $a$ is called \textbf{nonpositive} if $a \leq 0$. 
  We call $a$ \textbf{positive} if $a>0$ and we call $a$ \textbf{negative} if $a<0$.
\end{definition}

\begin{example}
  If $a \in \R$, show that $a \neq 0$ implies $a^2>0$; in particular that $-1<0<1$.
\end{example}
\begin{proof}
  If $a \neq 0$, then either $a>0$ or $a<0$. If $a>0$, then clearly $a^2>0$ and we are done. If $a<0$, then $-a>0$, 
  hence $(-a)(-a)>(-a)0$, and so $a^2>0$.

  Now we have that $1 \neq 0$, and that $1^2=1>0$, subtracting we get $1-1>0-1$ so $0>-1$.s
\end{proof}

\begin{example}
  If $a \in \R$, show that $0<a<1$ implies that $0<a^2<a$, and $a>1$ implies that $a^2>a$.
\end{example}
\begin{proof}
  We have already that if $a>0$, then $a^2>0$. Now suppose that $a<1$ then multiplying by $a$ we get $(a)(a)<1(a)$, hence 
  $a^2<1$. Likewise, by the same reasoning, if $a>1$ we get $a^2>a$.
\end{proof}

\begin{example}
  Prove that:
    \begin{enumerate}[label=(\arabic*)]
      \item $0 \leq a < b$ and $0 \leq c < d$ imply that $ac<bd$.

      \item $0 \leq a < b$ imply that $0 \leq a^2 < b^2$ and $0 \leq \sqrt{a} < \sqrt{b}$.

      \item $0<a<b$ implies $0<\frac{1}{b}<\frac{1}{a}$.
    \end{enumerate}
\end{example}
\begin{proof}
  \begin{enumerate}[label=(\arabic*)]
    \item Let $0 \leq a <b$ and $0 \leq c < d$. Then $0 \leq ac <bc$, and since $c<d$, then $bc<bd$, hence we have that $0 \leq ac < bd$.

    \item Let $0 \leq a < b$. Then $0 \leq aa < ab$, and notice that $ab<bb$, so $0 \leq aa < bb$, i.e. $0 \leq a^2 < b^2$. 
        Now notice that $a=(\sqrt{a})(\sqrt{a})$ and $b=(\sqrt{b})(\sqrt{b})$, by the previous result, we have $0 \leq (\sqrt{a})(\sqrt{a}) < (\sqrt{b})(\sqrt{b})$, henve we have that $0 \leq \sqrt{a} < \sqrt{b}$.

    \item Let $0<a<b$, multiplying by $\frac{1}{b}$, we get that $0<a\frac{1}{b}<b\frac{1}{b}$ and get $0<a\frac{1}{b}<1$, 
        now multiplying again by $\frac{1}{a}$, we get $0<\frac{1}{a}a\frac{1}{b}<\frac{1}{a}$, thus we have that $0<\frac{1}{b}<\frac{1}{a}$.
  \end{enumerate}
\end{proof}

\begin{definition}
   The \textbf{absolute value} of an element $a \in \R$ is a real number $|a|$ defined such that:
   \begin{equation}
     |a|=
        \begin{cases}
          a \text{, if } a \geq 0 \\
          -a \text{, if } a<0 \\
        \end{cases}
   \end{equation}
\end{definition}

\begin{remark}
  The absolute value is multiplicative, i.e. $|ab|=|a||b|$ for all $a,b \in \R$.
\end{remark}
\begin{proof}
  We do a casewise evaluation.
    \begin{enumerate}[label=(\arabic*)]
      \item If $a=0$, or $b=0$, then $|ab|=0=|a||b|$ (since $|a|=a=0$ or $|b|=b=0$).

      \item Let $a,b >0$. then $|a|=a$ and $|b|=b$. Then $|ab|=ab$ as $ab>0$. Hence $|ab|=ab=|a||b|$.

      \item Let $a>0$ and $b<0$, then $|a|=a$ and $|b|=-b$ si $|a||b|=a(-b)=-ab=|ab|$ as $ab<0$.

      \item Let $a,b<0$, then $|a|=-a$ and $|b|=-b$ and $|a||b|=(-a)(-b)=ab=|ab|$, since $ab>0$.
    \end{enumerate}
\end{proof}

\begin{theorem}
  Let $a \in \R$ and let $M$ be nonnegative. Then $|a| \leq M$ if and only if $-M \leq a \leq M$.
\end{theorem}
\begin{proof}
  Notice that if $|a| \leq M$, then $-|a| \geq -M$. Suppose then that $a \geq 0$, then $|a|=a \leq M$, and since $-M \leq 0$, 
  we have $-M \leq a \leq M$.
  Now if $a<0$, $-|a|=a$, then $a \geq -M$, and since $a<0$, $a \geq M$.

  Conversely suppose that $-M \leq a \leq M$. Then $-M \leq a$ and $a \leq M$. For $-M \leq a$, multiplying by $-1$ we have 
  $M \geq -a$. If $a \leq 0$ then $|a|=a$ and $|a| \leq M$. If $a<0$ then $|a|=-a \leq M$. Hence we have in both cases that $|a| \leq M$.
\end{proof}

\begin{theorem}
  The absolute value satisfies the following three properties For all $a,b \in \R$:
      \begin{enumerate}[label=(\arabic*)]
        \item $|a| \geq 0$ with $|a|=0$ if and only if $a=0$.

        \item $|a-b|=|b-a|$.

        \item $|a+b| \leq |a|+|b|$.
      \end{enumerate}
\end{theorem}\
\begin{proof}
  If $a \leq 0$, then clearly $|a| \geq 0$, if $a<0$, then $|a|=-a \geq 0$. Now if $a=0$ then $|0|=0$, and if $|a|=0$, then 
  $\pm a=0$; then $a=1a=(\pm 1)^2a=(\pm 1)(\pm 1)a=(\pm 1)(\pm a)$, since $\pm a=0$, $(\pm 1)0=0$ hence $a=0$.

  Now we have $(a-b)=-(b-a)$, so $|a-b|=|-1(b-a)|=|-1||b-a|=1|b-a|=|b-a|$.

  Notice for all $x \in \R$, $|x| \leq |x|$. So $-|x| \leq x \leq |x|$, then we have for $a,b \in \R$, $-|a| \leq a \leq |a|$ 
  and $-|b| \leq b \leq |b|$, adding we get $-(|a|+|b|) \leq a+b \leq |a|+|b|$, thus $|a+b| \leq |a|+|b|$.
\end{proof}

\begin{corollary}
  For all $a,b \in \R$, $|a-b| \geq |a|-|b|$ and $||a|-|b|| \leq |a-b|$.
\end{corollary}
\begin{proof}
  Now $|a|-|b|=|a+b-b|-|b|$, by the above theorem, $|a|-|b| \leq |a-b|+|b|-|b|=|a-b|$.

  Now $||a|-|b|| \leq |a-b|$ implies $-|a-b| \leq |a|-|b| \leq |a-b|$, we need to show that $-|a-b| \leq |a|-|b|$. Then 
  $|a-b| \geq |b|-|a|$  hence $|b-a| \geq |b|-|a|$ which reduces to the second equality.
\end{proof}


\begin{example}
  Show that if $-2<x<1$ then $|x^2-x|<6$.
\end{example}
\begin{proof}
  We have that $-2<x<2$, so $|x|<2$. Then $|x^2+x| \leq |x^2|+|x|=|x||x|+|x|<(2)(2)+2=6$.
\end{proof}

\begin{theorem}
  Let $x,y,a \in \R$.
    \begin{enumerate}[label=(\arabic*)]
      \item $x<y+\epsilon$ for all $\epsilon>0$ if and only if $x \leq y$.

      \item $x>y-\epsilon$ for all $\epsilon>0$ if and only if $x \geq y$.

      \item $|a|<\epsilon$ for all $\epsilon>0$ if and only if $a=0$.
    \end{enumerate}
\end{theorem}
\begin{proof}
  \begin{enumerate}[label=(\arabic*)]
    \item Suppose that $x<y+\epsilon$ for all $\epsilon>0$, but that $x>y$. Let $\epsilon=\epsilon_0=x-y$, then $x=y+\epsilon_0$. 
        Then it is not true that $x<y+\epsilon_0$. A contradiction.

    Conversely, suppose that $x \leq y$ and that $\epsilon>0$. Then either $x<y$ or $x=y$. For $x<y$, $x+0<y+0<y+\epsilon$, 
    hence $x<y+\epsilon$. Similarly, for $x=y$, $x<y+\epsilon$.

    \item This proof is analgous to the previous. Suppose that $x>y-\epsilon$ for $\epsilon>0$. Then let $\epsilon=\epsilon_0=y-x$. 
        Then $x=y-\epsilon_0$, which contradictos our assumption that $x>y-\epsilon$.

    Conversely, let $x \geq y$ and let $\epsilon>0$. Then either $x>y$ or $x=y$. For $x>y$, we have $x-0>y-0>y-\epsilon$. 
    For $x=y$, we see clearly that $x>y-\epsilon$.

    \item Notice first that if $\epsilon>0$ then $0>-\epsilon$. Now suppose that for $a \in \R$, $|a| \leq \epsilon$. Then 
        $-\epsilon<a<\epsilon$, thus by the transitivity of $<$ we have that $0<a$ and $a<0$, which cannot happen, so it must be that $a=0$.

    Now let $a=0$, then clearly, by our assumptions, we see that $a<\epsilon$ and $-\epsilon<a$, thus $|a| \leq \epsilon$.
  \end{enumerate}
\end{proof}

\begin{definition}
  For all $a,b \in \R$, we define the \textbf{closed interval} to be the set $[a,b]=\{x : a \leq x \leq b\}$. We define the 
  \textbf{open interval} to be the set $(a,b)=\{x: a<x<b\}$. We denote a \textbf{half open interval} to be a set of the form $[a,b)=\{x: a \leq x <b\}$ or $(a,b]=\{a <x \leq b\}$.
\end{definition}

We denote $(a,\infty)=\{x: a<x\}$, $(-\infty,a)=\{x: x<a\}$ and $(-\infty,\infty)=\{x: x \in \R\}$.

\begin{definition}
  An interval $I$ is \textbf{bounded} if and only if it has the form: $[a,b],[a,b),(a,b],$ or $(a,b)$. We call $a$ and $b$ 
  the \textbf{endpoints}, or \textbf{bounds} of the interval. We call all other intervals \textbf{unbounded}.
\end{definition}

Now if $a=b$, the two bounds co incide and we call the interval \textbf{degenerate}, and if $a<b$, it is called \textbf{nondegenerate}. 
Also notice that an interval $(1,1)=\emptyset$, but $[1,1]=\{1\}$. So we notice that a degenerate open interval is the empty set, and a degenerate closed interval is a point. Now for bounded intervals, we call $|a-b|$ is called the \textbf{length} of the interval, and sometimes denoted $|I|$.

\begin{HW}
  Do exercises $1$, $5$, $7$, $9$, and $10$.
\end{HW}
