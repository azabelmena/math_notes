%----------------------------------------------------------------------------------------
%	SECTION 3.2
%----------------------------------------------------------------------------------------

\section{One Sided Limits.}

If we have a realvalued function $f(x)=\sqrt{x-1}$, we know that  $x \geq 1$, and 
we would like to talk about  $\lim{f}$ at  $x=1$. Then we can only talk about  $x \rightarrow 1$ 
from the righthand side of the function since because of our restriction on $x$.

\begin{definition}
     Let $a \in \R$. A realvalued function is said to \textbf{converge}	to $L$ as 
     $x$ \textbf{approaches } $a$ \textbf{from the right} if and only if $f$ is 
     efied on some open interval $I$ containing $a$ as a left endpoint and for 
    every $\epsilin>0$, there is a $\delta(\epsilon)>0$ such taht $a+\delta \in I$, 
    and $a<x<a+\delta$ implies $|f(x)-L|<\epsilon$. We then call $L$ the 
    \textbf{righthanded limit} of  $f$ as  $x$ approaches  $a$ and denote it 
    $\lim{f}$ as  $x \rightarrow a^+$.
\end{definition}

\begin{definition}
     Let $a \in \R$. A realvalued function is said to \textbf{converge}	to $L$ as 
     $x$ \textbf{approaches } $a$ \textbf{from the left} if and only if $f$ is 
     efied on some open interval $I$ containing $a$ as a right endpoint and for 
    every $\epsilin>0$, there is a $\delta(\epsilon)>0$ such taht $a+\delta \in I$, 
    and $a-\delta<x<a$ implies $|f(x)-L|<\epsilon$. We then call $L$ the 
    \textbf{lefthanded limit} of  $f$ as  $x$ approaches  $a$ and denote it 
    $\lim{f}$ as  $x \rightarrow a^-$.
\end{definition}

In general, we may also write $\lim_{x \rightarrow a+}{f}$ and  $\lim_{x \rightarrow a-}{f}$ 
respectively.

\begin{example}
    Let $f(x)=x+1$ for  $x \geq 0$ and  $f(x)=x-1$ for $x<0$. We see that  
    $\lim_{x \rightarrow 0^+}{f}=\lim{x+1}=1$ and  $\lim_{x \rightarrow 0^-}{f}=
    \lim{x-1}=-1$. Does the limit of  $f$ in general exist? Since the limit of $f$	
    must be unique, then it may be either of these two limits, however, since the 
    righthanded and left handed limits converge at two different values, that would 
    imply two different limits at the same $x$, which is impossible. So in general 
    we cannot say if  $\lim{f}$ exists as  $x \rightarrow 0$.
\end{example} 

\begin{theorem}\label{3.2.1}
    Let $f$ be a realvalued function. Then the limit $\lim{f}$ as  $x \rightarrow a$ 
    exists and equals  $L$ if and only if:
        \begin{equation*}
            \lim_{x \rightarrow a^+}{f}=\lim_{x \rightarrow a^-}{f}=L
        \end{equation*}
\end{theorem}
\begin{proof}
    Suppose that $f$ has limit  $L$ as  $x \rightarrow a$. Then for any  $\epsilon>0$, 
    there is a $\delta>0$ such that  $0<|x-a|<\delta$ implies  $|f(x)-L|<\epsilon$. 
    Then $a-\delta<x<a+\delta$ implies $|f(x)-L|<\epsilon$. Thus we have that 
    $x-\delta<x<a$ and  $a<x<a+\delta$ both imply  $|f(x)-L|<\epsilon$. Hence 
    we get that  $\lim_{x \rightarrow a^-}{f}=\lim_{x \rightarrow a^+}{f}=L$.

    Conversely Suppose that  $\lim_{x \rightarrow a^-}{f}=\lim_{x \rightarrow a^+}{f}=L$. 
    then by definition fo the left and righthanded limits we get that $a-\delta_1<x<a$ and 
    $a<x<a+\delta_2$ imply  $|f(x)-L|<\epsilon$, thus we have that $a-\delta_1<x<a-\delta_2$, 
    then choosing $\delta=\min\{\delta_1,\delta_2\}$, and noting that $x \neq a$, 
    we get that  $0<|x-a|<\delta$ implies  $|f(x)-L|<\epsilon$. Thus $\lim{f}=L$ 
    as  $x \rightarrow a$.
\end{proof}

Another thing we would like to study is the limit of a function $f$ (with domain $A$) 
as  $x$ gets arbitrarily large, and as  $x$ gets arbitrarily small, i.e. as 
$x \rightarrow \infty$ and  $x \rightarrow -\infty$. We say that $f \rightarrow L$ 
as  $x \rightarrow \infty$ if and only if there  exists a $C>0$ such that 
$(C,\infty) \subseteq A$ and $\epsilon>0$ such that tere is an $M \in \R$ 
such that  $x>M$ implies  $|f(x)-L|<\epsilon$ ; likewise, $f \rightarrow L$ as  $x \rightarrow -\infty$ if and only if 
there is a  $C>0$ such that  $(-\infty,C) \subseteq A$ and $\epsilon>0$ such that 
there is an $M \in \R$ such that $x<M$ implies  $|f(x)-L|<\epsilon$. 

Similarly, we 
can go on to define ``infinite limits'' of a function $f$. We say that $f \rightarrow \infty$ as 
$x \rightarrow a^+$ if and only if there is an open interval $I$ with left endpoint  $a$ 
such that for any  $M>0$, there is is a $\delta>0$ such that  $a<x<a+\delta$ implies 
$f(x)>M$. Other infinite limits can be defined similarly. We also have infinite limits 
at infinity, for example $\lim{f}=\infty$ as  $x \rightarrow \infty$ (take $f(x)=x$). 
What we mean by ``infinity'' is just to say that for  $x$ sufficiently large  (or small), 
$f$ tends to  $L$  (in the case for limits at infinity), and that $f$ grows without bound 
in either direction as  $x$ approaches some value  $a$  (for infinite limits). In 
the case of infinite limits at infinity, then we are saying that $f$ grows without bound, 
(in either direction), for $x$ sufficiently large (or small). The symbol $\infty$ 
should not be taken as a literal number, but as a concept denoting something as 
arbitrarily big, or without bound.

\begin{example}
    Show that $\lim{\frac{1}{x}}=\infty$ as $x \rightarrow 0^+$ and that $\lim{\frac{1}{x}}=
    0$ as $x \rightarrow \infty$.
\end{example} 
\begin{solution}
    For any $M>0$, we need  $ \frac{1}{x}>M$, then $x<\frac{1}{M}$, so choose 
    $\delta=\frac{1}{M}$, then for $0<x<\delta$,  $ \frac{1}{x}>M$, so by definition, 
    $lim{\frac{1}{x}}=\infty$ as $x \rightarrow 0^+$.

    Similarly, for any  $\epsilon>0$, we need  $|\frac{1}{x}|<\epsilon$, hence 
    $x>\frac{1}{\epsilon}$, so choose $M=\frac{1}{\epsilon}$. Then by definition, the 
    limit follows.
\end{solution}
