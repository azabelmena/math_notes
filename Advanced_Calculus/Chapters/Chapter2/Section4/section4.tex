%----------------------------------------------------------------------------------------
%	SECTION 2.4
%----------------------------------------------------------------------------------------

\section{Cauchy Sequences.}

Whenever we say a sequences converges, we would like to find the limit. However, 
the limit may lie outside of the sequence. It is here that building up the notion 
of a ``Cauchy sequence'' which establishes a property that is independent form 
the limit, but that guarantees that a sequence will be convergent to said limit 
(which of course we may or may not know).

\begin{definition}
    A realvalued sequence $\{x_n\}$ is said to be a \textbf{Cauchy sequence} 
    if and only if for every $\epsilon>0$, there is an  $N \in \N$ such that  
    $n, m \geq N$ implies that $|x_n-x_m|<\epsilon$.
\end{definition}

\begin{remark} 
    If $\{x_n\}$ is a convergent sequence, then  $\{x_n\}$ is cauchy.
\end{remark} 
\begin{proof}
    Suppose that $\{x_n\} \rightarrow x$ as  $n \rightarrow \infty$. For any  $\epsilon>0$  
    there is an $N \in \N$ such that forall  $n \geq N$,  $|x_n-x|<\frac{\epsilon}{2}$. 
    Now the, for any $m \geq N$ we get $|x_m-x|<\frac{\epsilon}{2}$. Then by the 
    triangle inequality: $|x_n-x_m=|(x_n-x)-(x_m-x)| \leq |x_n-x|+|x_m-x|<\epsilon$. 
    we see that $\{x_n\}$ is a Cauchy sequence.
\end{proof}

\begin{theorem}[Cauchy's Theorem]\label{2.4.1}
    Let $\{x_n\}$ be a realvalued sequence. Then $\{x_n\}$ is a Cauchy sequence 
    if and only if  $\{x_n\}$ converges to some point in $\R$.
\end{theorem}
\begin{proof}
    We need only to show that Cauchy sequences converge (the remark before hand 
    proves the converse implication). Let $\{x_n\}$ be a Cauchy sequence, then  
    $\{x_n\}$ is bounded. LEt  $\epsilon=1$, then there is some  $K \in \N$ such 
    that for all  $m,n \geq K$, $|x_n-x_n|<1$. We have that $|x_1|,|x_2|, \dots 
    |x_K|$, and $|x_{K+1}-x_K|<1$. Now $|x_{K+1}|=|x_{K+1}-x_K+x_K| \leq |x_{K+1}|
    +|x_K|<1+|x_K|$. Hence choose  $M= \max\{|x_1|,|x_2|, \dots, |x_K|, 1+|x_K|\}$. 
    Then $|x_n| \leq M$ for all  $n \in \N$. Hence every Cauchy sequence is 
    bounded, hence by the Bolzano-Weierstrass theorem,  $\{x_n\}$ has a convergent 
    subsequence  $\{x_{n_k}\}$. So there is an  $x \in \R$, such that for all 
    $\epsilon>0$, there exists  $K \in \N$ such that for all  $k \geq K$,  
    $|x_{n_k}-x|<\frac{\epsilon}{2}$.

    $\{x_n\}$ is a Cauchy sequense, hece for this same  $\epsilon>0$, there is an 
     $N_1 \in \N$ such that for $m,n \geq N_1$, $|x_n-x_m|<\frac{\epsilon}{2}$. 
     Chossing $K \geq N_1$, we have that $k \geq K$, hence  $n_k \geq k \geq K 
     \geq N_1$. For all $n \geq K$, For all  $n \geq K$,  $|x_n-x|=|x_n-x_{n_k}+
     x_{n_k}-x| \leq |x_n-x_{n_k}|+|x_{n_k}-x|<\frac{\epsilon}{2}+\frac{\epsilon}{2}=\epsilon$. 
     Thus $x_n \rightarrow x$ as  $n \rightarrow \infty$, hence all Cauchy 
     sequences are convergent.
\end{proof}

\begin{example}
    Prove that any real valued sequence $\{x_n\}$ satisfying  $|x_n-x_{n+1}|<
    \frac{1}{2^n}$ is convergent.
\end{example} 
\begin{solution}
    To prove that it is convergent, first prove that it is a Cauchy seqence. For 
    $n<m$, we have  $|x_n-x_m|=|x_n-x_{n+1}+x_{n+1}-x_{n+2}+x_{n+2}+ \dots + 
    x_{m-1}-x_m| \leq |x_n-x_{n+1}|+|x_{n+1}-x_{n+2}|+\dots+|x_{m-1}-x_m|< 
    \frac{1}{2^n}+\frac{1}{2^{n+1}}+ \dots +\frac{1}{2^{m-1}}<\frac{1}{2^{n-1}}$. 
    Then for $\epsilon>0$, there exists  $N \in \N$ with  $ \frac{1}{2^{N-1}}< \epsilon$.
    Hence, for $m,n \geq N$ we have that  $|x_n-x_m|<\frac{1}{2^{n-1}}<
    \frac{1}{2^{N-1}}<\epsilon$. Therefore, $\{x_n\}$ is a Cauchy sequence, and 
    hence converges to a point in  $\R$.
\end{solution}

\begin{remark} 
    If $|x_{n+1}-x_n| \rightarrow 0$, then it is not sufficient to deduce that 
    it is a Cauchy sequence. For example, consides  $x_n=\log{n}$. Then  $\{x_n\}$ 
    is unbounded and divergent. However,  $\log{n+1}-\got{n}=\frac{log{n+1}}
    {\log{n}}=\log{1}=0$.
\end{remark}

\begin{HW} 
    Exercises $1$,  $2$, $3$, and $6$ on page  $51$.
\end{HW}
