\documentclass[12pt, twoside]{book}

\usepackage[margin=1in]{geometry}
\usepackage[utf8]{inputenc}
\usepackage[spanish]{babel}
\usepackage{verbatim}
\usepackage{pgfplots}
    \pgfplotsset{compat=1.12,}
\usepackage{enumitem}
\usepackage{amsmath,amsfonts,amsthm,amssymb,graphicx,mathtools,hyperref}
\usepackage{wrapfig}
\usepackage[export]{adjustbox}
\usepackage{tikz}
\renewcommand\qedsymbol{$\blacksquare$}
\renewcommand{\bar}[1]{\overline{#1}}
\usetikzlibrary{positioning}
\newcommand{\A}{\mathbb{A}}
\newcommand{\B}{\mathbb{B}}
\newcommand{\C}{\mathbb{C}}
\newcommand{\D}{\mathbb{D}}
\newcommand{\E}{\mathbb{E}}
\newcommand{\F}{\mathbb{F}}
\newcommand{\G}{\mathbb{G}}
\newcommand{\Hb}{\mathbb{H}} %
\newcommand{\I}{\mathbb{I}}
\newcommand{\J}{\mathbb{J}}
\newcommand{\K}{\mathbb{K}}
\newcommand{\Lb}{\mathbb{L}} %
\newcommand{\M}{\mathbb{M}}
\newcommand{\N}{\mathbb{N}}
\newcommand{\Ob}{\mathbb{O}} %
\newcommand{\Pb}{\mathbb{P}} % 
\newcommand{\Q}{\mathbb{Q}}
\newcommand{\R}{\mathbb{R}}
\newcommand{\Sb}{\mathbb{S}} % 
\newcommand{\T}{\mathbb{T}}
\newcommand{\U}{\mathbb{U}}
\newcommand{\V}{\mathbb{V}}
\newcommand{\W}{\mathbb{W}}
\newcommand{\X}{\mathbb{X}}
\newcommand{\Y}{\mathbb{Y}}
\newcommand{\Z}{\mathbb{Z}}
\newcommand{\Ac}{\mathcal{A}}
\newcommand{\Bc}{\mathcal{B}}
\newcommand{\Cc}{\mathcal{C}}
\newcommand{\Dc}{\mathcal{D}}
\newcommand{\Ec}{\mathcal{E}}
\newcommand{\Fc}{\mathcal{F}}
\newcommand{\Gc}{\mathcal{G}}
\newcommand{\Hc}{\mathcal{H}} %
\newcommand{\Ic}{\mathcal{I}}
\newcommand{\Jc}{\mathcal{J}}
\newcommand{\Kc}{\mathcal{K}}
\newcommand{\Lc}{\mathcal{L}} %
\newcommand{\Mc}{\mathcal{M}}
\newcommand{\Nc}{\mathcal{N}}
\newcommand{\Oc}{\mathcal{O}} %
\newcommand{\Pc}{\mathcal{P}} % 
\newcommand{\Qc}{\mathcal{Q}}
\newcommand{\Rc}{\mathcal{R}}
\newcommand{\Sc}{\mathcal{S}} % 
\newcommand{\Tc}{\mathcal{T}}
\newcommand{\Uc}{\mathcal{U}}
\newcommand{\Vc}{\mathcal{V}}
\newcommand{\Wc}{\mathcal{W}}
\newcommand{\Xc}{\mathcal{X}}
\newcommand{\Yc}{\mathcal{Y}}
\newcommand{\Zc}{\mathcal{Z}}
\newcommand{\ita}[1]{\textit{#1}}
\newcommand{\com}[2]{#1 \backslash #2}
\newcommand{\oneton}{\{1,2,3,...,n\}}
\newcommand\idea[1]{\begin{gather*}#1\end{gather*}}
\newcommand\ef{\ita{f} }
\newcommand\eff{\ita{f}}
\newcommand\proofs[1]{\begin{proof}#1\end{proof}}
\newcommand\inv[1]{#1^{-1}}
\newcommand\setb[1]{\{#1\}}
\newcommand\en{\ita{n }}
\newcommand{\vbrack}[1]{\langle #1\rangle}
\renewcommand{\bbar}[1]{\underline{#1}}
\DeclareMathOperator{\ord}{ord}
\DeclareMathOperator{\Int}{Int}
\DeclareMathOperator{\diam}{diam}
\DeclareMathOperator{\re}{Re}
\DeclareMathOperator{\im}{Im}

\theoremstyle{plain}
\newtheorem{theorem}{Theorem}[section]
\newtheorem{axiom}{Axiom}[section]
\newtheorem{lemma}[theorem]{Lemma}
\newtheorem{proposition}[theorem]{Proposition}
\newtheorem{postulate}{Postulate}
\newtheorem*{corollary}{Corollary}

\theoremstyle{plain} % just in case the style had changed
\newcommand{\thistheoremname}{}
\newtheorem*{genericthm}{\thistheoremname}
\newenvironment{namedtheorem}[1]
  {\renewcommand{\thistheoremname}{#1}%
   \begin{genericthm}}
  {\end{genericthm}}

\theoremstyle{definition}
\newtheorem*{definition}{Definition}
\newtheorem{conjecture}{Conjecture}
\newtheorem{example}{Example}[chapter]
\newtheorem*{HW}{Homework}

\theoremstyle{remark}
\newtheorem*{remark}{Remark}
\newtheorem*{claim}{Claim}
\newtheorem*{note}{Note}

\renewcommand*{\proofname}{Proof}



\title{Topología}
\author{Alec Zabel-Mena.}
\date{\today}

\begin{document}

\maketitle

\chapter{Classical Cryptography.}

\label{Chapter1}

\section{Eculidian Domains.}

\begin{definition}
    Let $R$ be a commutative ring. We call a map  $N:R \xrightarrow{} \N$, with
    $N(0)=0$ a \textbf{norm}, or, \textbf{degree}. If $N(a) \geq 0$, for all $a
    \in R$, then we call $N$ \textbf{nonnegative} If $N(a)>0$ for all $a \in R$
    then we call  $N$ \textbf{positive}.
\end{definition}

\begin{definition}
    Let $R$ be a commutative ring, and  $N:R \xrightarrow{} \N$ a norm. We say
    thay $R$ is a \textbf{Euclidean domain} if for all $a,b \in R$, with $b \neq
    0$, there exist elements  $q,r \in R$ such that
    \begin{equation*}
        a=qb+r \text{ where } r=0 \text{ or } N(r)<N(b)
    \end{equation*}
    We call $q$ the  \textbf{quotient} and $r$ the  \textbf{remainder} of $a$
    when  \textbf{divided} by $b$.
\end{definition}

\begin{example}\label{2.1}
    \begin{enumerate}
        \item[(1)] Let $F$ be any field, and  $N:F \xrightarrow{} \N$ defined by
            $N(a)=0$ for all $a \in F$. Then  $N$ makes  $F$ into a Euclidean
            domain. Take  $a,b \in F$, with  $b \neq 0$, and  $q=a\inv{b}$. Then
            $a=qb+r$ where  $r=0$.

        \item[(2)] The integers $\Z$ is a Euclidean domain with norm $N(a)=|a|$,
            the absolute value of $a$. In fact, the motivation for Euclidean
            rings comes from the division theorem, or Euclid's theorem for
            integers.

        \item[(3)] Let $F$ be a field, and consider  $F[x]$. Let $N:F[x]
            \xrightarrow{} \N$ be defined by $N:f \xrightarrow{} \deg{f}$. Then
            $f$ is a Euclidean domain. If  $F$ is not a field, then it is not
            necesarily true that $F[x]$ be a Euclidean domain.

        \item[(4)] Let $D \in \Z^+$ be squarefree, and consider  $\Z[\sqrt{D}]$.
            Define $N:\Z[\sqrt{D}]=\N$ to be the absolute value of the field
            norm, that is $N(a+b\sqrt{D})=\|a+b\sqrt{D}\|^2=a^2+Db^2$. We notice
            that $\Z[\sqrt{D}]$ is an integral domain, but it is not a Euclidean
            domain. This depends on our choice of $D$. Let  $D=-1$ so tha t
            $\sqrt{D}=i$, and $i^2=-1$. Then the Gaussian integers, $\Z[i]$, is
            a Euclidean domeain. Let $x=a+ib$,  $y=c+id$ with  $y \neq 0$. In
            $\Q[i]$, the field of fractions, we have that $\frac{x}{y}=r+is$,
            where
            \begin{equation*}
                r=\frac{ac+bd}{\|y\|^2} \text{ and } s=\frac{bc-ad}{\|y\|^2}
            \end{equation*}
            Now, let $p$ and  $q$ be the integers closest to $r$ and $s$,
            respectively so that
            \begin{equation*}
                |r-p| \leq \frac{1}{2} \text{ and } |s-q| \leq \frac{1}{2}
            \end{equation*}
            Let $w=(r-p)+i(s-q)$, and take $z=wy$. Then we have  $z=x-(p+iq)y$,
            so that $x=(p+iq)y+z$, moreover, we have $N(w)=(r-p)^2+(q-s)^2 \leq
            \frac{1}{4}+\frac{1}{4}=\frac{1}{2}$. Since $\|\cdot\|$ is
            multiplicative, we have
            \begin{equation*}
                N(w)N(y) \leq \frac{1}{2}N(y)
            \end{equation*}
            which makes $\Z[i]$ into a Euclidean domain.

        \item[(5)] Let $K$ be a field. We define a  \textbf{discrete valuation}
            to be a map $\nu:K^\ast \xrightarrow{} \Z$ such that
            \begin{enumerate}
                \item[(i)] $\nu(ab)=\nu(a)+\nu(b)$.

                \item[(ii)] $\nu$ is onto.

                \item[(iii)] $\min{\{\nu(x),\nu(y)\}} \leq \nu(x+y)$, for all
                    $x,y \in K^\ast$ for which  $x+y \neq 0$.
            \end{enumerate}
            We call the set $\nu{K}=\{x \in K^\ast : \nu(x) \geq 0\}$ the
            \textbf{valuation ring} of $\nu$ and is a subring of  $K^\ast$. We
            call an integral domain  $R$ a  \textbf{discrete valuation ring} if
            there exists a discrete valuation $\nu$ on the field of fractions of
             $R$, having  $\nu{R}$ as its valuation ring.

             It can be shown that discrete valuation rings are Euclidean
             domains by the norm $N:0 \xrightarrow{} 0$ and $N=\nu$ on all
             $R^\ast$.
    \end{enumerate}
\end{example}

\begin{lemma}\label{2.1.1}
    Every ideal in a Euclidean domain $R$, is a principle ideal.
\end{lemma}
\begin{proof}
    If $I=(0)$, we are done. Now, let $N:R \xrightarrow{} \N$ be the norm of $R$,
    and consider the image $N(I)=\{N(a) : a \in I\}$. By the well ordering
    principle, $N(I)$ has a minimum element $N(d)$ for some $d \neq 0$ in  $I$
    Notice that  $(d) \subseteq I$. Now, let $a \in I$. Since $R$ is a Euclidean
    domain, there exist $q,r \in R$ for which
    \begin{equation*}
        a=qd+r \text{ where } r=0 \text{ or } N(r)<N(d)
    \end{equation*}
    Then notice that
    \begin{equation*}
        r=a-qd
    \end{equation*}
    putting $r \in I$ and  $N(r) \in N(I)$. Since $N(d)$ is the minimum element,
    we must have $r=0$ so that  $a=qd$, which puts  $a \in (d)$. Therefore
    $I=(d)$, making $I$ principle.
\end{proof}

\begin{example}\label{2.2}
    \begin{enumerate}
        \item[(1)] The polynomial ring $\Z[x]$ is not a Euclidean domain. The
            ideal $(2,x)$ is not principle.

        \item[(2)] Consider $\Z[\sqrt{-5}]$, i.e. $D=-5$. Suppose the ideal
            $(3,2+\sqrt{-5})$ is a principle ideal, that is
            $(3,2+\sqrt{5})=(a+b\sqrt{-5})$ for some $a,b \in \Z$. Then we get
            that $3=x(a+b\sqrt{-5})$ and $2+\sqrt{-5}=y(a+b\sqrt{-5})$. Then
            notice that $N(x)=a^2+5b^2=9$, and since  $a^2+5b^2 \in \Z^+$, we
            must have that $a^2+5b^2=1,3,9$.
            \begin{enumerate}
            \item[(i)] If $a^2+5b^2=9$, then $N(x)=1$ making $x=\pm 1$ and
                $a+b\sqrt{-5}=\pm3$, which cannot happen since $2+\sqrt{-5}$ is
                not divisible by $3$.

            \item[(ii)] the equation $a^2+5b^2=3$ cannot happen since it has no
                integer solutions. This makes

            \item[(iii)] $a^2+b\sqrt{5}=1$, which makes
                $(a+\sqrt{-5})=\Z[\sqrt{-5}]$, moreover, we get the equation
                $3x+y(2+\sqrt{-5})=1$ for any $x,y \in \Z[\sqrt{-5}]$.
                Multplying both sides by $2-\sqrt{-5}$, we get that
                $3|(2-\sqrt{-5})$ which is impossible.
            \end{enumerate}
            In all three cases, we were led to an impossibility, hence
            $\Z[\sqrt{-5}]$ cannot be a Euclidean domain.
    \end{enumerate}
\end{example}

\begin{definition}
    Let $R$ be a commutative ring, and  $a,b \in R$ with  $b \neq 0$. We say
    that $b$ \textbf{divides} $a$ if there is an  $x \in R$ for which  $a=bx$.
    We write  $b|a$. We also say that $a$ is a \textbf{multiple} of $b$.
\end{definition}

\begin{definition}
    Let $R$ be a commutative ring. We call a nonzero element  $d \in R$ a
    \textbf{greatest common divisor} of elements $a,b \in R$ if
    \begin{enumerate}
        \item[(1)] $d|a$ and  $d|b$.

        \item[(2)] If $c \in R$ is nonzero such that  $c|a$ and  $c|b$, then
            $c|d$.
    \end{enumerate}
    We write $d=(a,b)$.
\end{definition}

\begin{lemma}\label{2.1.2}
    Let $R$ be a commutative ring. For any $a,b \in R$ a nonzero element  $d \in
    R$ is the greatest common divisor if
    \begin{enumerate}
        \item[(1)] $(a,b) \subseteq (d)$.

        \item[(2)] If $c \in R$ is nonzero with $(a,b) \subseteq (c)$, then $(d)
            \subseteq (c)$.
    \end{enumerate}
    In particular, $d=(a,b)$.
\end{lemma}
\begin{proof}
    The first two statements follow from definition, and the last follows lemma
    \ref{2.1.1}.
\end{proof}

\begin{lemma}\label{2.1.3}
    If $R$ is a commutative ring, and $a,b \in R^\ast$, such that  $(a,b)=(d)$
    for some $d \in R^\ast$, then $d$ is the greatest common divisor of $a$ and
     $b$.
\end{lemma}

\begin{lemma}\label{2.1.4}
    Let $R$ be an inetegral domain. If $c,d \in R$ generate the same principle
    ideal, i.e. $(d)=(c)$, then $d=uc$ for some unit  $u \in R$.
\end{lemma}
\begin{proof}
    If $c=0$ or  $d=0$, we are done. Suppose then that  $c,d \neq 0$. Since
    $(d)=(c)$, we have that $d=xc$ and $c=yd$ for some  $x,y \in R$. Then
    $d=(xy)d$, which makes $d(1-xy)=0$. Since $d \neq 0$, we get $xy=1$, making
     $x$ and $y$ units of $R$.
\end{proof}
\begin{corollary}
    If $R$ is commutative, then greatest common divisors are unique.
\end{corollary}

\begin{definition}
    We call an integral domain in which every principle ideal is generated by
    two elements a \textbf{Bezout domain}.
\end{definition}

\begin{lemma}\label{2.1.5}
    Every Euclidean domain is a Bezout domain.
\end{lemma}

\begin{theorem}[The Extended Euclidean Algorithm]\label{2.1.6}
    Let $R$ be a Euclidean and $a,b \neq 0$ elements of  $R$. Let  $d=r_n$ be
    the least nonzero remainder obtained by dividing $a$ from $b$ recursively
    $n+1$ times. Then
    \begin{enumerate}
        \item[(1)] $d=(a,b)$ is the greatest common divisor of $a$ and  $b$.

        \item[(3)] There exist $x,y \in R$ for which  $ax+by=d$.
    \end{enumerate}
\end{theorem}
\begin{proof}
    By lemma \ref{2.1.1}, we get that the ideal $(a,b)$ is principle, so there
    exists a greatest common divisor of $a$ and  $b$. Now, let  $d=r_n$ be
    obtained by dividing $a$ and $b$ recursively $(n+1)$ times. Then the
    $(n+1)^{st}$ equation gives $r_{n-1}=q_{n+1}r_n$, so that $r_n|r_{n-1}$.
    Now, by induction on $n$ if  $r_n|r+{k+1}$ and $r_n|r_k$ then the
    $(k+1)^{st}$ equation gives $r_{k-1}=q_{k+1}r_k+r_{k+1}$, which implies that
    $r_n|r_{k-1}$. Therefore we get in the $1^{st}$ equation that $r_n|b$, and in
    the $0^{th}$ that $r_n|a$. That is,  $d|a$ and  $d|b$.

    Now, notice that $r_0 \in (a,b)$ and that $r_1=b-qr_0 \in (b,r_0) \subseteq
    (a,b)$. By induction on $r_n$, if  $r_{k-1},r_n \in (a,b)$ then
    \begin{equation*}
        r_{k+1}=r_{k-1}-q_{k+1}r_k \in (r_{k-1},r_n) \subseteq (a,b)
    \end{equation*}
    which puts $r_n \in (a,b)$ making $d=(a,b)$ the greatest common divisor.
\end{proof}

\begin{definition}
    Let $R$ be an integral domain, and let
    $\tilde{R}=R^\ast \cup \{0\}$ the set of units together with $0$. We call
    an element  $u \in \com{R}{\tilde{R}}$ a \textbf{universal side divisor} if
    for all $x \in R$, there is a  $z \in \tilde{R}$ such that $u|x-z$.
\end{definition}

\begin{lemma}\label{2.1.7}
    Let $R$ be an integral domain which is not a field. If $R$ is a Euclidean
    domain, then there exist universal side divisors.
\end{lemma}
\begin{proof}
    Notice that since $R$ is not a field, that $\tilde{R} \neq R$ and
    $\com{R}{\tilde{R}}$ is nonempty. Let $N$ be the norm of  $R$, and let  $u
    \in \com{R}{\tilde{R}}$ be of minimal norm. Then for all $x \in R$, take
    $x=qu+r$ with  $r=0$ or  $N(r)<N(u)$> By minimality of $N(u)$, we get $r \in
    \tilde{R}$.
\end{proof}

\begin{example}\label{2.3}
    Notice that $\pm 1$ are the only units in the ring
    $\Z[1+\frac{\sqrt{-19}}{2}]$, so that $\tilde{R}=\{0,1,-1\}$. Suppose that
    $u \in R$ is a universal side divisor, and let $N=\|\cdot\|^2$ be the field
    norm; so that $N(a+(1+\frac{\sqrt{-19}}{2})b)=a^2+ab+5b^2$. If $a,b \in \Z$
    and $b \neq 0$, then we have $a^2+ab+5b^2=(a+\frac{b}{2})^2+\frac{19}{4b^2}
    \geq 5$ so that the smallest nonzero norms are $1$ for $x=1$ and $4$ for $x=2$.
    Now, if $u$ is a universal side divisor, then  $u|2-0$ or  $u|(2 \pm 1)$
    that is $u|2$,  $u|3$ or  $u|1$ making  $u$ a nonunit divsor. If  $2=xy$
    then $4=N(x)N(y)$ and so that $N(x)=1$ or  $N(y)=1$. Hence the only
    divisors of $2$ in  $\Z[1+\frac{\sqrt{-19}}{2}]$ are $\pm 1$ or  $\pm 2$.
    Similarly the only divisors of  $3$ arew  $\pm 1$ or  $\pm 3$ hence  $u=\pm
    2$ or  $u=\pm 3$. Letting  $x=\frac{1+\sqrt{-19}}{2}$, then $x$, nor  $x
    \pm 1$ are divisible by any possible $u$. Therefore
    $\Z[1+\frac{\sqrt{-19}}{2}]$ has no universal side divisors, and cannot be a
    Euclidean domain.
    \end{example}

\section{The Lebesgue Outer Measure}

\begin{definition}
    Let $I \subseteq \R$ be an interval. We define the  \textbf{length} $l(I)$,
    of $I$ to be  $\infty$ if  $I$ is unbounded, and the difference of its
    endpoints otherwise.
\end{definition}

\begin{definition}
    Let $\{I_k\}$ be a countable collection of open bounded intervals covering a
    set $A \subseteq \R$. We define the  \textbf{Lebesgue outer measure} of $A$
    to be
    \begin{equation*}
        m^\ast(A)=\inf{\Big{\{} \sum_{k=1}^\infty{l(I_k)} :
        A \subseteq \bigcup_{k=1}^\infty{I_k} \Big{\}}}
    \end{equation*}
\end{definition}

We now go over some basic properties of the Lebesgue outer measure.

\begin{lemma}\label{8.2.1}
    For any set $A \subseteq \R$,  $m^\ast(A) \geq 0$; in particular,
    $m^\ast(\emptyset)=0$.
\end{lemma}
\begin{proof}
    By definition, since $l(I_k) \geq 0$, each $\sum{l(I_k)} \leq 0$. This makes
    $m^\ast(A) \geq 0$ for any $\{I_k\}$ a countable cover of $A$ by bounded
    open intervals.

    Notice that $\emptyset \subseteq (-\e,\e)$ for some $\e>0$, and that this
    covers $\emptyset$, so that $m^\ast(\emptyset)=l((-\e,\e))=2\e$. Then
    choosing $\e$ small enough gives us  $m^\ast(\emptyset)=0$.
\end{proof}

\begin{lemma}[Monotonicity]\label{8.2.2}
    The Lebesgue Outer Measure is monoton; that is, if $A \subseteq B$, then
    \begin{equation*}
        m^\ast(A) \leq m^\ast(B)
    \end{equation*}
\end{lemma}
\begin{proof}
    Let $\{I_k\}$ be a countable cover of $B$ by bounded open intervals. Then
    notice that  $\{I_k\}$ covers $A$ as well. Now, let
    \begin{align*}
        E   &=  \Big{\{} \sum_{k=1}^\infty{l(I_k)} : A \subseteq
                            \bigcup_{k=1}^\infty{I_k} \Big{\}}  \\
        F   &=  \Big{\{} \sum_{k=1}^\infty{l(I_k)} : B \subseteq
                            \bigcup_{k=1}^\infty{I_k} \Big{\}}  \\
    \end{align*}
    Then since $A \subseteq B$, $F \subseteq E$. Therefore, we get that
    \begin{equation*}
        \inf{E}=m^\ast(A) \leq m^\ast(B)=\inf{F}
    \end{equation*}
\end{proof}

\begin{lemma}\label{8.2.3}
    Countable sets have Lebesgue outer measure $0$.
\end{lemma}
\begin{proof}
    Let $C$ be a countable set, and  $C=\{c_k\}_{k \in \Z^+}$ an enumeration for
    $C$. Let $\e>0$, then for every  $k \in \Z^+$, define the interval
    \begin{equation*}
        I_k=(c_k-\frac{\e}{2^{k+1}}, c_k+\frac{\e}{2^{k+1}})
    \end{equation*}
    Then $\{I_k\}$ is a countable cover of $C$ by bounded open intervals. Thus
    we get
    \begin{equation*}
        0 \leq m^\ast(C) \leq \sum_{k=1}^\infty{l(I_k)}=\sum{\frac{\e}{2^k}}=\e
    \end{equation*}
    Therefore taking $\e$ small enough, we get  $0 \leq m^\ast(C) \leq 0$, and
    equality is established.
\end{proof}

\begin{lemma}\label{8.2.4}
    Let $I \subseteq \R$ be an interval. Then  $m^\ast(I)=l(I)$.
\end{lemma}
\begin{proof}
    Consider first that $I=[a,b]$, a closed bounded interval, where $a<b$. Let
    $\e>0$, then we have that  $[a,b] \subseteq (a-\e,b+\e)$, so that by
    monotonicity
    \begin{equation*}
        m^\ast([a,b]) \leq l((a-\e,b+\e))=b-a+2\e
    \end{equation*}
    Then for $\e$ small enough, this shows that
    \begin{equation*}
        m^\ast([a,b]) \leq b-a
    \end{equation*}

    Now, let $\{I_k\}$ be a countable cover of $[a,b]$ by bounded open
    intervals. Then since $[a,b]$ is compact (by Heine-Borel), there is a finite
    subcover $\{I_k\}_{k=1}^n$ of $[a,b]$. Now, since $a \in
    \bigcup_{k=1}^n{I_k}$, there is atleast one interval $I_k$ containing $a$;
    denote it $ (a_1,b_1)$. Now, $a_1<a<b_1$. If $b_1 \geq b$, then we are done
    as
    \begin{equation*}
        \sum_{k=1}^n{l(I_k)} \geq b_1-a_1>b-a
    \end{equation*}
    Otherwise, $b_1 \in [a,b)$, and since $b_1 \notin (a_1,b_1)$, there is an
    interval $(a_2,b_2)$, distinct from $(a_1,b_1)$, containing $b_1$. Now, if
    $b_2 \geq b$, we are done. Otherwise, proceeding recursively, we obtain a
    subcollection $\{(a_k,b_k)\}_{k=1}^N$ of $\{I_k\}_{k=1}^n$ for which
    \begin{equation*}
        a_1<a \text{ and } a_{k+1}<b_k \text{ for all } 1 \leq k \leq N-1
    \end{equation*}
    The process of selecting such a subcollection must terminate, which leaves
    us with $b_N>b$, so that
    \begin{equation*}
        \sum_{k=1}^n{l(I_k)} \geq \sum_{k=1}^N{l((a_k,b_k))}=
        (b_N-a_N)+\dots+(b_1-a_1) \geq b_N-a_1>b-a
    \end{equation*}
    INdeed, we get
    \begin{equation*}
        \sum_{k=1}^n{l(I_k)} \geq b-a
    \end{equation*}
    so that $m^\ast([a,b])=b-a$.

    Now, suppose that $I$ is any bounded interval. Then for  $\e>0$, there
    exist closed bounded interval $J_1$ and $J_2$ such that
    \begin{equation*}
        J_1 \subseteq I \subseteq J_2
    \end{equation*}
    and
    \begin{equation*}
        l(I)-\e<l(J_1) \leq l(I) \leq l(J_2)<l(I)+\e
    \end{equation*}
    By the monotonicity, and the above discussion of closed bounded intervals,
    we get
    \begin{equation*}
        l(I)-\e<m^\ast(J_1) \leq m^\ast(I) \leq m^\at(J_2)<l(I)+\e
    \end{equation*}
    Therefore, for $\e$ small enough, we get that $m^\ast(I)=l(I)$.

    Finally, suppose that $I$ is an unbounded interval. Then for every  $n \in
    \Z^+$, there is an interval  $J$ with  $l(J)=n$. So that
    \begin{equation*}
        n=m^\ast(J) \leq m^\ast(I)
    \end{equation*}
    This makes $m^\ast(I)=l(I)=\infty$, by definition of $l(I)$.
\end{proof}

\begin{lemma}[Translation Invariance]\label{8.2.5}
    The Lebesgue outer measure is translation invariant; that is, if $A
    \subseteq \R$, and  $y \in \R$,
    \begin{equation*}
        m^\ast(A+y)=m^\ast(A)
    \end{equation*}
\end{lemma}
\begin{proof}
    Let $\{I_k\}$ be a countable cover of $A$ by open bounded intervals. Then
    the collection  $\{I_k+y\}$ is a countable cover of the set $A+y$ by open
    bounded intervals. Moreover, notice that $l(I_k)=l(I_k+y)$. This gives us
    \begin{equation*}
        \sum{l(I_k)}=\sum{l(I_k+y)}
    \end{equation*}
    and we are done.
\end{proof}

\begin{lemma}[Countable Subadditivity]\label{8.2.6}
    The Lebesgue outer measure is countable subadditive; that is, if $\{E_k\}$
    is a countable collection of subsets of $\R$, then
    \begin{equation*}
        m^\ast\Big{(} \bigcup{E_k} \Big{)} \leq \sum{m^\ast(E_k)}
    \end{equation*}
\end{lemma}
\begin{proof}
    Let $\{E_k\}$ be a countable collection of subsets of $\R$, and let
    \begin{equation*}
        E=\bgicup{E_k}
    \end{equation*}
    If at least one of the $E_k$ has  $m^\ast(E_k)=\infty$, then we are done.
    Suppose then that $m^\ast(E_k)$ is finite for all $k \in \Z^+$. Then for
    each  $E_k$, there is a countable cover  $\{I_{k,i}\}_{i \in \Z^+}$ by
    bounded open intervals for which
    \begin{equation*}
        \bigcup{l(I_{k,i})}<m^\ast(E_k)+\frac{\e}{2^k}
    \end{equation*}
    Now, consider the countable collection
    \begin{equation*}
        \{I_{k,i}\}_{k,i \in \Z^+}=\bigcup_{k \in \Z^+}{\{I_{k,i}\}}_{i \in \Z^+}
    \end{equation*}
    Then $\{I_{k,i}\}_{k,i \in \Z^+}$ is a countable cover of $E$ by bounded
    open intervals. We get
    \begin{equation*}
        m^\ast(E) \leq
        \sum_{k \in \Z^+}{\Big{(} \sum_{i \in \Z^+}{l(I_{k,i})} \Big{)}}<
        \sum_{k \in \Z^+}{(m^\ast(E_k)+\frac{\e}{2^k})}=
        \Big{(} \sum{m^\ast(E_k)} \Big{)}+\e
    \end{equation*}
    Taking $\e>0$ small enough, gives us the required subadditivity.
\end{proof}
\begin{corollary}
    The Lebesgue outer measure is finitely subadditive.
\end{corollary}
\begin{proof}
    Recall that finite collections of sets are countable.
\end{proof}

%%----------------------------------------------------------------------------------------
%	SECTION 1.1
%----------------------------------------------------------------------------------------

\section{Cauchy Sequences}

\begin{definition}
    We call a sequence $\{x_n\}$ in a metric space $X$ a \textbf{Cauchy sequence} in $X$, or 
    more simply, \textbf{Cauchy} in $X$ if for all $\epsilon>0$, there is an  $N \in \Z^+$ such that 
    $d(x_n,x_m)<\epsilon$ whenever  $m,n \geq N$.
\end{definition}

\begin{definition}
    Let $E$ be a nonempty subset of a metrix space  $X$, and lelt $S \subseteq \R$ be the 
    all real numbers $d(x,y)$, with  $x,y \in E$. We call  $\sup{S}$ the  \textbf{diameter} 
    of $E$, and denote it  $\diam{E}$.
\end{definition}

\begin{theorem}\label{3.3.1}
    Let $\{x_n\}$ be a sequence, and let  $E_N$ be the set of all points  $p_N$ such that 
    $_N<p_{n+1}$. Then $\{x_n\}$ is Cauchy if and only if  $\lim{\diam{E_N}}=0$ as 
     $N \rightarrow \infty$.
\end{theorem}
\begin{proof}
    Let $\{x_n\}$ be Cauchy,  Let  $x_{N_1},x_{N_2} \in E$ such that $d(x_n,x_{N_1})<
    \frac{\epsilon}{2}$, and $d(x_{N_2},x_m)<\frac{\epsilon}{2}$. Then we see that $d(x_{N_1},x_{N_2}) \leq 
    d(x_{N_1},x_n)+d(x_m, x_{N_2})<\epsilon$, so $\{x_{N_k}\}$ is Cauchy and we see that 
    $\lim{\diam{E_N}}=0$. Now suppose that  $\lim{\diam{E}}=0$, then for any  $x_n,x_m \in S$, 
    $d(x_n,0)< \frac{\epsilon}{2}$ and $d(0,x_m)< \frac{\epsilon}{2}$ implies that $d(x_n,x_m) \leq 
    d(x_n,0)+d(0,x_m)<\epsilon$, whenever  $n,m>N$, for  $\epsilon>0$.
\end{proof}

\begin{theorem}\label{3.3.2}
    \begin{enumerate}[label=(\arabic*)]
        \item If $E \subseteq X$, then  $\diam{\hat{E}}=\diam{E}$.

        \item If $\{K_n\}$ is a sequence of compact sets in  $X$, such that  $K_{n+1} \subseteq K_n$, and 
            if  $\lim{\diam{K_n}}=0$ as  $n \rightarrow \infty$, then  $\bigcap_{i=1}^{infty}{K_i}$ contains 
            exactly one point.
    \end{enumerate}
\end{theorem}
\begin{proof}
    Clearly $\diam{E} \leq \diam{\hat{E}}$. Now let  $\epsilon>0$, and choose  $x,y \in \hat{E}$, then 
    there are points $x',y' \in \hat{E}$ such that  $d(x,x')<\frac{\epsilon}{2}$ and $d(y,y')<
    \frac{\epsilon}{2}$. Hence, $d(x,y) \leq d(x,x')+d(x',y')+d(y'y)<\epsilon \diam{E}$, then 
    choosing  $\epsilon$ arbitrarily small,  $\diam{\hat{E}} \leq \diam{E}$.

    Now, we also have that by the nested interval theorem that $K=\bigcap{K_i}$ is nonempty. Now 
    suppose that  $K$ contains more that one point. then $\diam{K}>0$, and since  $K \subseteq K_n$ for 
    all  $n$,  $diam{K} \leq \diam{K_n}$, a contradiction. Thus  $K$ contains exactly one element.
\end{proof}

\begin{theorem}\label{3.3.3}
    \begin{enumerate}[label=(\arabic*)]
        \item In any metric space $X$, every convergent sequence is a Cauchy sequence.

        \item If $X$ is compact, and  $\{x_n\}$ is Cauchy in  $X$, then $\{x_n\}$ 
            converges to a point in  $X$.
    \end{enumerate}
\end{theorem}
\begin{proof}
    \begin{enumerate}[label=(\arabic*)]
        \item If $x_n \rightarrow x$, and $\epsilon>0$ such that there is an  $N \in \Z^+$ such that 
            $d(x_n,x)<\frac{\epsilon}{2}$ for all $n \geq N$, then for  $m \geq N$, we have 
            $d(x_n,x_m) \leq d(x_n,x)+d(x,x_m)<\epsilon$. Thus  $\{x_n\}$ is Cauchy.

        \item Let $\{x_n\}$ be Cauchy, and let  $E_N$ be the set of all points  $x_N$ for 
            which  $x_N<x_{N+1}$. Then  $\lim{\diam{\hat{E}}}=0$, then being closed in  $X$, each 
            $\hat{E_N}$ is compact in  $X$, and  $\hat{E_{N+1}} \subseteq \hat{E_N}$, so 
            by theorem \ref{3.3.2}, there is a unique  $x \in X$ in all of $\hat{E_N}$. Now for 
            $\epsilon>0$, there is an  $N_0 \in \Z^+$ for which $\diam{\hat{E}}<\epsilon$. Then for all 
            $x_n \in \hat{E}$,  $d(x_n,x)<\epsilon$ whenever  $n \geq N_0$.
    \end{enumerate}		
\end{proof}

\begin{corollary}[The Cauchcy Criterion]
    Every Cauchy sequence in $\R^k$ converges to a point in  $\R^k$.
\end{corollary}
\begin{proof}
    Let $\{x_n\}$ be Cauchy in  $\R^k$, define $E_N$ as in  $(2)$, then for some  $N \in \Z^+$, 
$\diam{E}<1$, and so  $\{x_n\}$ us the union of all  $E_n$, and ther set of points  
$\{x_1,\dots, x_{N-1}\}$, so $\{x_n\}$ is bounded, and thus has a compact closure, it follows then 
that  $x_n \rightarrow x$ for some  $x \in \R^k$.
\end{proof}

\begin{definition}
    We call a metric space \textbf{complete} if every Cauchy sequence in the space converges.		
\end{definition}

\begin{theorem}\label{3.3.4}
    All compact metric spaces, and all Euclidean spaces are complete.
\end{theorem}

\begin{example}
    Consider $\Q$ together with the metric  $|x-y|$	. The metric space induced on $\Q$ by  $|\cdot|$ is 
    not complete.
\end{example} 

\begin{definition}
    A sequence $\{x_n\}$ in $\R$ is said to be \textbf{monotonically increasing} if 
    $x_n \leq x_{n+1}$, $\{x_n\}$ is said to be \textbf{monotonically decreasing} if $x_{n+1}<x_n$. 
    We call  $\{x_n\}$ \textbf{monotonic} if it is either monotonically increasing or monotonically 
    decreasing.
\end{definition}

\begin{theorem}\label{3.3.5}
    A monotonic sequence converges if and only if it is bounded.
\end{theorem}
\begin{proof}
    Suppose, without loss of generality, that $\{x_n\}$ is monotonically increasing. If 
    $\{x_n\}$ is bounded, then  $x_n \leq x$, then for all  $\epsilon>0$, there is an  $N \in \Z^+$ 
    for which  $x-\epsilon<x_N \leq x$. Then for  $n \geq N$,  $x_n \rightarrow x$. The converse follows 
    from theorem  \ref{3.1.2}.
\end{proof}

%%----------------------------------------------------------------------------------------
%	SECTION 1.4
%----------------------------------------------------------------------------------------

\section{The Complex Field}

\begin{definition}
    We define a \textbf{complex number} to be a pair of real numbers  $(a,b)$. We denote the 
    set of all comlex numbers by $\C$. We define the \textbf{addition} and $\textbf{multiplication}$ of 
    complex numbers to be the binary operations $+:\C \rightarrow \C$ and $\cdot: \C \rightarrow \C$ such that 
        \begin{align*}
            (a,b)+(c,d) &= (a+c,b+d) \\
            (a,b)(c,d) &= (ac-bd,ad+bc) \\
        \end{align*}
        Lastly, we define $i$ to be the complex number such that  $i=(0,1)$.
\end{definition}

\begin{theorem}\label{1.4.1}
    $\C$ forms a field together with  $+$ and  $\cdot$.
\end{theorem}

\begin{theorem}\label{1.4.2}
    For $(a,0),(b,0) \in C$,  $(a,0)+(b,0)=(a+b,0)$, and $(a,0)(b,0)=(ab,o)$.
\end{theorem}
\begin{proof}
    This is a straightforward application of the addition and multiplication of 
    complex numbers.
\end{proof}

\begin{theorem}\label{1.4.3}
    $i^2=-1$.
\end{theorem}
\begin{proof}
    $i^2=(0,1)(0,1)=(0-1,1-1)=(-1,0)=-1$.		
\end{proof}

\begin{theorem}\label{1.4.4}
    Let $(a,b) \in \C$, then  $(a+b)=a+ib$.
\end{theorem}
\begin{proof}
    $(a,b)=(a,0)+(0,b)=(a,0)+(0,1)(b,0)=a+ib$.	
\end{proof}

\begin{definition}
    Let $a,b \in \R$, and let  $z \in \C$ such that  $z=a+ib$. We define the  \textbf{complex 
    conjugate} of $z$ to be the complex number  $\bar{z}=a-ib$. Moreover, we define the 
    \textbf{real part} of  $z$ to be $a$, and the \textbf{imaginary part} of  $z$ to be $b$, 
    and we denote them $a=\re{z}$,  $b=\im{z}$
\end{definition}

\begin{theorem}\label{1.4.5}
    Let $z, w \in \C$. Then 
        \begin{enumerate}[label=(\arabic*)]
            \item $\overline{z+w}=\bar{z}+\bar{w}$.

            \item $\overline{zw}=\bar{z}\bar{w}$.

            \item $z+\bar{z}=2\re{z}$ and  $z-\bar{z}=2i\im{z}$.

            \item $z\bar{z}$ is a nonegative real number.
        \end{enumerate}
\end{theorem}
\begin{proof}
    Let $z=a+ib$, and let  $w=c+id$. Then $z+w=(a+c)+i(b+d$, so  $\overline{z+w}=(a+b)
    -i(b+d)=(a-ib)+(c-id)=\bar{z}+\bar{w}$; similarly, we get  $\overline{zw}=\bar{z}\bar{w}$. 
    Moreover, we have  $(a+ib)+(a-ib)=2a$, and  $(a+ib)-(a-ib)=2ib$, we also have that 
    $z\bar{z}=(a+ib)(a-ib)=a^2+b^2 \geq 0$, and $z\bar{z}=0$ if and only if $a=b=0$.
\end{proof}

\begin{definition}
    Let $z \in \C$. We define the  \textbf{modulus} of  $z$ to be  $|z|=\sqrt{z\bar{z}}$.
\end{definition}
\begin{remark}
    $|z|$ exists and is unique.
\end{remark}

\begin{theorem}\label{1.4.6}
    Let $z,w \in \C$, then:
        \begin{enumerate}[label=(\arabic*)]
            \item $|z| \geq 0$ and  $|z|=0$ if and only if $z=0$.

            \item  $|\bar{z}|=|z|$.

            \item $|zw|=|z||w|$.

            \item  $\re{z} \leq |z|$.

            \item  $|z+w+ \leq |z|+|w|$.
        \end{enumerate}
\end{theorem}
\begin{proof}
    Let $z=a+ib$, and  $w=c+id$. Then  $|z|=\sqrt{a^2+b^2} \geq 0$, and  $|z|=0$ if 
    and only if  $a,b=0$. Moreover,  $|\bar{z}|=|a+i(-b)|=\sqrt{a^2+(-b)^2}=\sqrt{a^2+b^2}=|z|$. 
    We also habe $|zw|^2=(a^2+b^2)(c^2+d^)=|z|^2|w|^2$, likewise,  $\|re{z}|=|a+i0|=
    \sqrt{a^2} \leq \sqrt{a^2+b^2}$. Finally we prove $(5)$.

    We have  $|z+w|^2=(x+w)(\bar{z}+\bar{w})=z\bar{z}+\bar{z}w+\bar{w}z+w\bar{w}= 
    |z|^2+w\re{z\bar{w}}+|w|^2 \leq |z|^2+2|s\bar{w}|+|w|^2=(|z|+|w|)^2$.
\end{proof}

\begin{theorem}[The Cauchy Schwarz Inequality]\label{1.4.7}
    Let $a_i,b_i \in \C$, for  $1 \leq i \leq n$. Then:
         \begin{equation}
             |\sum_{i=1}^{n}{a_i\bar{b_i}}| \leq \sum_{i = 1}^{n}{|a_i|^2}\sum_{i=1}^{n}{|b_j|^2}		
        \end{equation}
\end{theorem}
\begin{proof}
    Let $A=\sum{a_j|^2}$,  $B=\sum{|b_i|^2}$, and $C=\sum{a_i\bar{b_i}}$. If  $B=0$, then 
     $b_i=0$ for  $1 \leq i \leq n$, and we are done; so suppose that  $B>0$. Then
        \begin{align*}
            \sum{|Ba_j-Cb_j|^2} &= \sum{(Ba_j-Cb_j)(B\bar{a_j}-\bar{Cb_j})} \\
                             &= B\sum{|a_j|^2}-B\bar{C}\sum{a_j\bar{b_j}-BC\sum{\bar{a_j}b_j}}+|C^2|\sum{|b_j|^2} \\
                             &= (B^2A-B|C|^2)=B(AB-|C|^2) \geq 0 \\
        \end{align*}
    Since $B>0$, we get  $|C|^2 \leq AB$ as required.
\end{proof}

%\section{Exact Sequences of Modules}

\begin{definition}
    Let $A$ and  $C$ be modules. We call a module  $B$, containing  $A$ an
    \textbf{extension} of $C$ by  $A$ if  $\faktor{B}{A} \simeq C$.
\end{definition}

\begin{example}\label{example_4.13}
    Let $A$,  $B$, and  $C$ be modules. If  $A$ is a submodule of  $B$, then
    there exists a map 1--1 $\psi:A \xrightarrow{} B$ such that $A \simeq
    \psi(A) \subseteq B$. If $\faktor{B}{\psi(A)} \simeq C$, then there exists a
    map $\phi:B \xrightarrow{} C$ wich is onto, for which $\ker{\phi}=\psi(A)$.
    Then we get the following diagram.
    \begin{equation*}
        A \xrightarrow{\psi} B \xrightarrow{\phi} C
    \end{equation*}
\end{example}

\begin{definition}
    Let $\{A_n\}$ a collection of modules. By a \textbf{sequence}, we mean a
    diagram
    \begin{equation*}
        \dots \xrightarrow{} A_{n-1} \xrightarrow{} A_n \xrightarrow{} A_{n+1}
        \xrightarrow{} \dots
    \end{equation*}
    where each $A_i$ is mapped to $A_{i+1}$ by some module homomorphism.
\end{definition}

\begin{definition}
    Let $A$,  $B$, and  $C$ be modules. We call a pair of module homomorphisms
    $\a,\b$, defined by the diagram
    \begin{equation*}
        A \xrightarrow{\a} B \xrightarrow{\b} C
    \end{equation*}
    \textbf{exact} at $B$ if  $\a(A)=\ker{\b}$. If $\{A_n\}$ is a collection of
    modules, we call a sequence defined by the diagram
    \begin{equation*}
        \dots \xrightarrow{} A_{n-1} \xrightarrow{} A_n \xrightarrow{} A_{n+1}
        \xrightarrow{} \dots
    \end{equation*}
    an \textbf{exact sequence} if it is exact at each $A_n$.
\end{definition}

\begin{example}\label{example_4.14}
    The sequence from example \ref{example_4.13}, $A \xrightarrow{\psi} B
    \xrightarrow{\phi} C$, is exact at $B$,  since $\ker{\phi}=\psi(A)$.
\end{example}

\begin{lemma}\label{4.5.1}
    Let $R$ be a ring, and  $A$,  $B$, and  $C$  $R$-modules. The following are
    true for any $R$-module homomorphisms $\psi:A \xrightarrow{} B$ and $\psi:B
    \xrightarrow{} C$
    \begin{enumerate}
        \item[(1)] $(0) \xrightarrow{\i} A \xrightarrow{\psi} B$ is exact at $A$
            if, and only if $\psi$ is 1--1.

        \item[(2)] $B \xrightarrow{\phi} C \xrightarrow{\i'} (0)$ is exact at
            $B$ if, and only if $\phi$ is onto.
    \end{enumerate}
\end{lemma}
\begin{corollary}
    The sequence $(0) \xrightarrow{} A \xrightarrow{\psi} B \xrightarrow{\phi} C
    \xrightarrow{} (0)$ is an exact sequence if, and only if $\psi$ is 1--1,
    $\phi$ is onto, and $\ker{\phi}=\psi(A)$; that is, $B$ is an extension of
    $C$ by  $A$.
\end{corollary}

\begin{definition}
    We call an exact sequence of the form $(0) \xrightarrow{} A \xrightarrow{\psi}
    B \xrightarrow{\phi} C \xrightarrow{} (0)$ a \textbf{short exact sequence}.
\end{definition}

\begin{lemma}\label{4.5.2}
    If $A \xrightarrow{\a} B \xrightarrow{\b} C$ is exact at $Y$, then  $(0)
    \xrightarrow{} \a(A) \xrightarrow{} B \xrightarrow{} \faktor{B}{\ker{\b}}
    \xrightarrow{} (0)$ is a short exact sequence.
\end{lemma}

\begin{example}\label{example_4.15}
    \begin{enumerate}
        \item[(1)] Let $A$, and $C$ be modules. Consider the sequence
            \begin{equation*}
                (0) \xrightarrow{} A \xrightarrow{\i} A \oplus C
                \xrightarrow{\pi} C \xrightarrow{} (0)
            \end{equation*}
            where $\i$ is the inclusion map, and  $\pi$ is the projection map
            about the second coordinate; i.e. $\pi:(a,c) \xrightarrow{} c$. This
            sequence is a short exact sequence, since $\i(A)=\ker{\pi}$.

        \item[(2)] Consider the $\Z$-modules $\Z$ and $\faktor{\Z}{n\Z}$,
            the sequence
            \begin{equation*}
                (0) \xrightarrow{} \Z \xrightarrow{\i} \Z \oplus \faktor{\Z}{n\Z}
                \xrightarrow{\phi} \faktor{\Z}{n\Z} \xrightarrow{} (0)
            \end{equation*}
            is a short exact sequence giving an extension of $\faktor{\Z}{n\Z}$
            by $\Z$. Another extension is given by the short exact sequence
            \begin{equation*}
                (0) \xrightarrow{} \Z \xrightarrow{\n} \Z
                \xrightarrow{\pi} \faktor{\Z}{n\Z} \xrightarrow{} (0)
            \end{equation*}
            where $n:x \xrightarrow{} nx$, and $\pi:x \xrightarrow{} x \mod{n}$
            is the natural map. THese are ``inequivalent'' extensions of
            $\faktor{\Z}{n\Z}$ by $\Z$.

        \item [(3)] If $\phi:B \xrightarrow{} C$ is any module homomorphism,
            form the exact sequence
            \begin{equation*}
                (0) \xrightarrow{} \ker{\phi} \xrightarrow{\i} B
                \xrightarrow{\phi} \phi(B) \xrightarrow{} (0)
            \end{equation*}
            where $\i$ is the inclusion map. If  $\phi$ is onto, we may extend
            the sequence $B \xrightarrow{\phi} C$ (i.e. extend $\phi$) to a
            short exact sequence with $A=\ker{\phi}$.

        \item[(4)] Let $R$ be a ring, and  $M$ an  $R$-module homomorphism. Let
             $S$ be a set of generators for  $M$, and consider the free $R$-module
             $F(S)$ on $S$. Then
             \begin{equation*}
                 (0) \xrightarrow{} K \xrightarrow{\i} F(S) \xrightarrow{\phi}
                 \xrightarrow{} M \xrightarrow{} (0)
             \end{equation*}
             is a short exact sequence,  where $\phi$ is the unique  $R$-module
             homomorphism  which is the identity on $S$, and  $K=\ker{\phi}$.
    \end{enumerate}
\end{example}

\begin{definition}
    Let $(0) \xrightarrow{} A \xrightarrow{} B \xrightarrow{} C \xrightarrow{}
    (0)$ and $(0) \xrightarrow{} A' \xrightarrow{} B' \xrightarrow{} C'
    \xrightarrow{} (0)$ be short exact sequences. A \textbf{homomorphism} of
    sequences is a triple $(\a,\b,\y)$ of module homomorphisms, such that the
    following diagram commutes
    \begin{equation*}
        \begin{tikzcd}
            {(0)} & A & B & C & {(0)} \\
            {(0)} & {A'} & {B'} & {C'} & {(0)}
            \arrow[from=1-1, to=1-2]
            \arrow[from=1-2, to=1-3]
            \arrow[from=1-3, to=1-4]
            \arrow[from=1-4, to=1-5]
            \arrow[from=2-1, to=2-2]
            \arrow[from=2-2, to=2-3]
            \arrow[from=2-3, to=2-4]
            \arrow[from=2-4, to=2-5]
            \arrow["\gamma"', from=1-4, to=2-4]
            \arrow["\beta"', from=1-3, to=2-3]
            \arrow["\alpha"', from=1-2, to=2-2]
        \end{tikzcd}
    \end{equation*}
    If $\a$,  $\b$, and  $\y$ are  module isomorphisms, we call $(\a,\b,\y)$ an
    isomorphism of sequences.
\end{definition}


\begin{definition}
    We call two exact sequences $A \xrightarrow{} B \xrightarrow{} C$ and $A'
    \xrightarrow{} C' \xrightarrow{} B'$ \textbf{equivalent} if $A=A'$, $C=C'$,
    and there exists an isomorphism of sequences between them. We call  the
    corresponding extensions  $B$ and  $B'$  \textbf{equivalent}.
\end{definition}

\begin{lemma}\label{4.5.3}
    The composition of homomorphisms of exact sequences is a homomorphism
    of sequences.
\end{lemma}

\begin{lemma}\label{4.5.4}
    Isomorphisms of exact sequences form an equivalence relation on any set of
    exact sequences.
\end{lemma}

\begin{example}\label{example_4.16}
    \begin{enumerate}
        \item[(1)] Let $m,n \in \Z^+$ integers greater than $1$, and suppose
            that  $n|m$. Let  $k=\frac{m}{n}$, and define a map from the exact
            sequences of $\Z$-modules described by the following diagram
            \begin{equation*}
                \begin{tikzcd}
{(0)} & \Z & \Z & {\faktor{\Z}{n\Z}} & {(0)} \\
{(0)} & {\faktor{\Z}{k\Z}} & {\faktor{\Z}{k\Z}} & {\faktor{\Z}{n\Z}} & {(0)}
\arrow[from=1-1, to=1-2]
\arrow["n", from=1-2, to=1-3]
\arrow["\pi", from=1-3, to=1-4]
\arrow[from=1-4, to=1-5]
\arrow[from=2-1, to=2-2]
\arrow["\iota", from=2-2, to=2-3]
\arrow["{\pi'}", from=2-3, to=2-4]
\arrow[from=2-4, to=2-5]
\arrow["\alpha"', from=1-2, to=2-2]
\arrow["\beta"', from=1-3, to=2-3]
\arrow["\gamma"', from=1-4, to=2-4]
                \end{tikzcd}
            \end{equation*}
            where $n:x \xrightarrow{} nx$, $\pi:x \xrightarrow{} x \mod{n}$,
            $\a$,  $\b$ are the natural projections, and  $\y$ is the identity.
            For the second sequnece, we take  $\i:a \mod{k} \xrightarrow{}
            na\mod{m}$, and $\pi'$ the natural projection of $\faktor{\Z}{m\Z}$
            onto $\faktor{(\faktor{\Z}{m\Z})}{(\faktor{n\Z}{m\Z})} \simq
            \faktor{\Z}{n\Z}$. Then $(\a,\b,\y)$ describe a homorphism of short
            exact sequences.

        \item[(2)] If $(0) \xrightarrow{} \Z \xrightarrow{n} \Z \xrightarrow{\pi}
            \faktor{\Z}{n\Z} \xrightarrow{} (0)$ is the short exact sequence
            defined in example \ref{example_4.15}, mapping each module to itself
            via the map $x \xrightarrow{} -x$ gives a isomorphism of short exact
            sequences, which take this sequence onto itself. Notice however that
            this isomorphism is not an equeivalence, since it is not the
            identity on $\Z$.

        \item[(3)] Consider the diagram
            \begin{equation*}
\begin{tikzcd}
	{(0)} & {\faktor{\Z}{2\Z}} & {\faktor{\Z}{2\Z} \oplus \faktor{\Z}{2\Z}} &
                                {\faktor{\Z}{2\Z}} & {(0)} \\
	{(0)} & {\faktor{\Z}{2\Z}} & {\faktor{\Z}{2\Z} \oplus \faktor{\Z}{2\Z}} &
                                {\faktor{\Z}{2\Z}} & {(0)}
	\arrow["\psi", from=1-2, to=1-3]
	\arrow["\phi", from=1-3, to=1-4]
	\arrow[from=1-4, to=1-5]
	\arrow[from=2-1, to=2-2]
	\arrow["{\psi'}"', from=2-2, to=2-3]
	\arrow["{\phi'}"', from=2-3, to=2-4]
	\arrow[from=2-4, to=2-5]
	\arrow["i"', from=1-4, to=2-4]
	\arrow["\beta"', from=1-3, to=2-3]
	\arrow["i"', from=1-2, to=2-2]
	\arrow[from=1-1, to=1-2]
\end{tikzcd}
            \end{equation*}
            Where $i$  is the identity map, $\psi$ is a 1--1 map mapping into the
            first component of $\faktor{\Z}{2\Z} \oplus \faktor{\Z}{2\Z}$, and
            $\phi$ projects $\faktor{\Z}{2\Z} \oplus \faktor{\Z}{2\Z}$ onto its
            second component, and where $\psi'$ and  $\phi'$ behave just as  $\psi$
            and  $\phi$. If $\b$ maps $\faktor{\Z}{2\Z} \oplus \faktor{\Z}{2\Z}$
            to $\faktor{\Z}{2\Z} \oplus \faktor{\Z}{2\Z}$ by exchanging the
            factors; that is  $\b:(m,n) \xrightarrow{} (n,m)$, then this diagram
            commutes and gives an equivalence of short exact sequences which is
            not the identity.
    \end{enumerate}
\end{example}

\begin{lemma}[The Short Five Lemma]\label{4.5.5}
    Let $(\a,\b,\y)$ be a homomorphism of short exact sequences given by the
    diagram
    \begin{equation*}
        \begin{tikzcd}
            {(0)} & A & B & C & {(0)} \\
            {(0)} & {A'} & {B'} & {C'} & {(0)}
            \arrow[from=1-1, to=1-2]
            \arrow[from=1-2, to=1-3]
            \arrow[from=1-3, to=1-4]
            \arrow[from=1-4, to=1-5]
            \arrow[from=2-1, to=2-2]
            \arrow[from=2-2, to=2-3]
            \arrow[from=2-3, to=2-4]
            \arrow[from=2-4, to=2-5]
            \arrow["\gamma"', from=1-4, to=2-4]
            \arrow["\beta"', from=1-3, to=2-3]
            \arrow["\alpha"', from=1-2, to=2-2]
        \end{tikzcd}
    \end{equation*}
    then the following are true
    \begin{enumerate}
        \item[(1)] If $\a$ and  $\y$ are 1--1, then so is  $\b$.

        \item[(2)] If $\a$ and  $\y$ are onto, then so is  $\b$.

        \item[(3)] If $\a$ and  $\y$ are isomorphisms, then so is  $\b$.
    \end{enumerate}
\end{lemma}
\begin{proof}
    We chase the elements of the following diagram.
    \begin{equation*}
        \begin{tikzcd}
            {(0)} & A & B & C & {(0)} \\
            {(0)} & {A'} & {B'} & {C'} & {(0)}
            \arrow[from=1-1, to=1-2]
            \arrow["\psi", from=1-2, to=1-3]
            \arrow["\phi", from=1-3, to=1-4]
            \arrow[from=1-4, to=1-5]
            \arrow[from=2-1, to=2-2]
            \arrow["{\psi'}"', from=2-2, to=2-3]
            \arrow["{\phi'}", from=2-3, to=2-4]
            \arrow[from=2-4, to=2-5]
            \arrow["\gamma"', from=1-4, to=2-4]
            \arrow["\beta"', from=1-3, to=2-3]
            \arrow["\alpha"', from=1-2, to=2-2]
        \end{tikzcd}
    \end{equation*}
    Suppose that $\a$ and  $\y$ are 1--1, and choose an element $b \in
    \ker{\b}$. Then $\b(b)=0$, and $\phi'\b(b)=0$. Moreover, since this diagram
    commutes, we have that $\phi'\b=\y\phi$, so that $\phi'\b(b)=\y\phi(b)$,
    which implies that $\phi(b)=0$, since $\y$ is 1--1. This makes $b \in
    \ker{\phi}$. Now, since the sequences are also exact, we have
    $\ker{\phi}=\psi(A)$, so that $b \in \psi(A)$. That is, there is an $a \in
    A$ for wich $b=\psi(a)$. Now, by commutativity again, we have
    $\b\psi=\psi'\a$. So  $\b\psi(b)=\b(b)=0$, which makes $\psi'\a(a)=0$, so
    that $\a(a)=0$. Since $\a$ is 1--1, this makes $a=0$ and we get
    $b=\psi(a)=0$ which makes $\ker{\b}=(0)$. That is, $\b$ is 1--1.

    Now, suppose that $\a$ and $\y$ are onto, and let  $b' \in B'$. Then
    $\phi'(b)=\y(c)$ for some $c \in C$; since $\y$ is onto. Now, by lemma
    \ref{4.5.1} $\phi$ is onto, so there is a $b \in B$ for which $b=\y(c)$. By
    the commutativity of the diagram $\phi'\b(b)=\y(\phi(b))=\y(c)=\phi'(b')$,
    so $\phi'(b'-\b(b))=0$ which puts $b'-\b(b) \in \ker{\phi'}$. Now, by the
    exactness of the sequences, $\ker{\phi'}=\psi'(A')$, so there is an $a' \in
    A'$ for which $b'-\b(b)=\psi'(a)=a \in A$, since $\a$ is onto. Then $a \in
    \ker{\phi'}$. Now, by commutativity, observe that
    $\psi(a)=\psi(b'-\b(b))=0$, and since $\psi$ is 1--1, we get $b'-\b(b)=0$.
    That is $b'=\b(b)$ and $\b(B)=B'$, which makes $\b$ onto. Lastly, observe
    that if $\a$ and $\y$ are isomorphisms, then they are 1--1 and onto, which
    makes $\b$ 1--1 and onto, and hence an isomorphism as well.
\end{proof}


%%----------------------------------------------------------------------------------------
%	CHAPTER X
%----------------------------------------------------------------------------------------

\chapter{Lebesgue Measure} % Main chapter title

\label{Chapter2} % Change X to a consecutive number; for referencing this chapter elsewhere, use \ref{ChapterX}

%% to include section files use the \input{} command.

\section{Eculidian Domains.}

\begin{definition}
    Let $R$ be a commutative ring. We call a map  $N:R \xrightarrow{} \N$, with
    $N(0)=0$ a \textbf{norm}, or, \textbf{degree}. If $N(a) \geq 0$, for all $a
    \in R$, then we call $N$ \textbf{nonnegative} If $N(a)>0$ for all $a \in R$
    then we call  $N$ \textbf{positive}.
\end{definition}

\begin{definition}
    Let $R$ be a commutative ring, and  $N:R \xrightarrow{} \N$ a norm. We say
    thay $R$ is a \textbf{Euclidean domain} if for all $a,b \in R$, with $b \neq
    0$, there exist elements  $q,r \in R$ such that
    \begin{equation*}
        a=qb+r \text{ where } r=0 \text{ or } N(r)<N(b)
    \end{equation*}
    We call $q$ the  \textbf{quotient} and $r$ the  \textbf{remainder} of $a$
    when  \textbf{divided} by $b$.
\end{definition}

\begin{example}\label{2.1}
    \begin{enumerate}
        \item[(1)] Let $F$ be any field, and  $N:F \xrightarrow{} \N$ defined by
            $N(a)=0$ for all $a \in F$. Then  $N$ makes  $F$ into a Euclidean
            domain. Take  $a,b \in F$, with  $b \neq 0$, and  $q=a\inv{b}$. Then
            $a=qb+r$ where  $r=0$.

        \item[(2)] The integers $\Z$ is a Euclidean domain with norm $N(a)=|a|$,
            the absolute value of $a$. In fact, the motivation for Euclidean
            rings comes from the division theorem, or Euclid's theorem for
            integers.

        \item[(3)] Let $F$ be a field, and consider  $F[x]$. Let $N:F[x]
            \xrightarrow{} \N$ be defined by $N:f \xrightarrow{} \deg{f}$. Then
            $f$ is a Euclidean domain. If  $F$ is not a field, then it is not
            necesarily true that $F[x]$ be a Euclidean domain.

        \item[(4)] Let $D \in \Z^+$ be squarefree, and consider  $\Z[\sqrt{D}]$.
            Define $N:\Z[\sqrt{D}]=\N$ to be the absolute value of the field
            norm, that is $N(a+b\sqrt{D})=\|a+b\sqrt{D}\|^2=a^2+Db^2$. We notice
            that $\Z[\sqrt{D}]$ is an integral domain, but it is not a Euclidean
            domain. This depends on our choice of $D$. Let  $D=-1$ so tha t
            $\sqrt{D}=i$, and $i^2=-1$. Then the Gaussian integers, $\Z[i]$, is
            a Euclidean domeain. Let $x=a+ib$,  $y=c+id$ with  $y \neq 0$. In
            $\Q[i]$, the field of fractions, we have that $\frac{x}{y}=r+is$,
            where
            \begin{equation*}
                r=\frac{ac+bd}{\|y\|^2} \text{ and } s=\frac{bc-ad}{\|y\|^2}
            \end{equation*}
            Now, let $p$ and  $q$ be the integers closest to $r$ and $s$,
            respectively so that
            \begin{equation*}
                |r-p| \leq \frac{1}{2} \text{ and } |s-q| \leq \frac{1}{2}
            \end{equation*}
            Let $w=(r-p)+i(s-q)$, and take $z=wy$. Then we have  $z=x-(p+iq)y$,
            so that $x=(p+iq)y+z$, moreover, we have $N(w)=(r-p)^2+(q-s)^2 \leq
            \frac{1}{4}+\frac{1}{4}=\frac{1}{2}$. Since $\|\cdot\|$ is
            multiplicative, we have
            \begin{equation*}
                N(w)N(y) \leq \frac{1}{2}N(y)
            \end{equation*}
            which makes $\Z[i]$ into a Euclidean domain.

        \item[(5)] Let $K$ be a field. We define a  \textbf{discrete valuation}
            to be a map $\nu:K^\ast \xrightarrow{} \Z$ such that
            \begin{enumerate}
                \item[(i)] $\nu(ab)=\nu(a)+\nu(b)$.

                \item[(ii)] $\nu$ is onto.

                \item[(iii)] $\min{\{\nu(x),\nu(y)\}} \leq \nu(x+y)$, for all
                    $x,y \in K^\ast$ for which  $x+y \neq 0$.
            \end{enumerate}
            We call the set $\nu{K}=\{x \in K^\ast : \nu(x) \geq 0\}$ the
            \textbf{valuation ring} of $\nu$ and is a subring of  $K^\ast$. We
            call an integral domain  $R$ a  \textbf{discrete valuation ring} if
            there exists a discrete valuation $\nu$ on the field of fractions of
             $R$, having  $\nu{R}$ as its valuation ring.

             It can be shown that discrete valuation rings are Euclidean
             domains by the norm $N:0 \xrightarrow{} 0$ and $N=\nu$ on all
             $R^\ast$.
    \end{enumerate}
\end{example}

\begin{lemma}\label{2.1.1}
    Every ideal in a Euclidean domain $R$, is a principle ideal.
\end{lemma}
\begin{proof}
    If $I=(0)$, we are done. Now, let $N:R \xrightarrow{} \N$ be the norm of $R$,
    and consider the image $N(I)=\{N(a) : a \in I\}$. By the well ordering
    principle, $N(I)$ has a minimum element $N(d)$ for some $d \neq 0$ in  $I$
    Notice that  $(d) \subseteq I$. Now, let $a \in I$. Since $R$ is a Euclidean
    domain, there exist $q,r \in R$ for which
    \begin{equation*}
        a=qd+r \text{ where } r=0 \text{ or } N(r)<N(d)
    \end{equation*}
    Then notice that
    \begin{equation*}
        r=a-qd
    \end{equation*}
    putting $r \in I$ and  $N(r) \in N(I)$. Since $N(d)$ is the minimum element,
    we must have $r=0$ so that  $a=qd$, which puts  $a \in (d)$. Therefore
    $I=(d)$, making $I$ principle.
\end{proof}

\begin{example}\label{2.2}
    \begin{enumerate}
        \item[(1)] The polynomial ring $\Z[x]$ is not a Euclidean domain. The
            ideal $(2,x)$ is not principle.

        \item[(2)] Consider $\Z[\sqrt{-5}]$, i.e. $D=-5$. Suppose the ideal
            $(3,2+\sqrt{-5})$ is a principle ideal, that is
            $(3,2+\sqrt{5})=(a+b\sqrt{-5})$ for some $a,b \in \Z$. Then we get
            that $3=x(a+b\sqrt{-5})$ and $2+\sqrt{-5}=y(a+b\sqrt{-5})$. Then
            notice that $N(x)=a^2+5b^2=9$, and since  $a^2+5b^2 \in \Z^+$, we
            must have that $a^2+5b^2=1,3,9$.
            \begin{enumerate}
            \item[(i)] If $a^2+5b^2=9$, then $N(x)=1$ making $x=\pm 1$ and
                $a+b\sqrt{-5}=\pm3$, which cannot happen since $2+\sqrt{-5}$ is
                not divisible by $3$.

            \item[(ii)] the equation $a^2+5b^2=3$ cannot happen since it has no
                integer solutions. This makes

            \item[(iii)] $a^2+b\sqrt{5}=1$, which makes
                $(a+\sqrt{-5})=\Z[\sqrt{-5}]$, moreover, we get the equation
                $3x+y(2+\sqrt{-5})=1$ for any $x,y \in \Z[\sqrt{-5}]$.
                Multplying both sides by $2-\sqrt{-5}$, we get that
                $3|(2-\sqrt{-5})$ which is impossible.
            \end{enumerate}
            In all three cases, we were led to an impossibility, hence
            $\Z[\sqrt{-5}]$ cannot be a Euclidean domain.
    \end{enumerate}
\end{example}

\begin{definition}
    Let $R$ be a commutative ring, and  $a,b \in R$ with  $b \neq 0$. We say
    that $b$ \textbf{divides} $a$ if there is an  $x \in R$ for which  $a=bx$.
    We write  $b|a$. We also say that $a$ is a \textbf{multiple} of $b$.
\end{definition}

\begin{definition}
    Let $R$ be a commutative ring. We call a nonzero element  $d \in R$ a
    \textbf{greatest common divisor} of elements $a,b \in R$ if
    \begin{enumerate}
        \item[(1)] $d|a$ and  $d|b$.

        \item[(2)] If $c \in R$ is nonzero such that  $c|a$ and  $c|b$, then
            $c|d$.
    \end{enumerate}
    We write $d=(a,b)$.
\end{definition}

\begin{lemma}\label{2.1.2}
    Let $R$ be a commutative ring. For any $a,b \in R$ a nonzero element  $d \in
    R$ is the greatest common divisor if
    \begin{enumerate}
        \item[(1)] $(a,b) \subseteq (d)$.

        \item[(2)] If $c \in R$ is nonzero with $(a,b) \subseteq (c)$, then $(d)
            \subseteq (c)$.
    \end{enumerate}
    In particular, $d=(a,b)$.
\end{lemma}
\begin{proof}
    The first two statements follow from definition, and the last follows lemma
    \ref{2.1.1}.
\end{proof}

\begin{lemma}\label{2.1.3}
    If $R$ is a commutative ring, and $a,b \in R^\ast$, such that  $(a,b)=(d)$
    for some $d \in R^\ast$, then $d$ is the greatest common divisor of $a$ and
     $b$.
\end{lemma}

\begin{lemma}\label{2.1.4}
    Let $R$ be an inetegral domain. If $c,d \in R$ generate the same principle
    ideal, i.e. $(d)=(c)$, then $d=uc$ for some unit  $u \in R$.
\end{lemma}
\begin{proof}
    If $c=0$ or  $d=0$, we are done. Suppose then that  $c,d \neq 0$. Since
    $(d)=(c)$, we have that $d=xc$ and $c=yd$ for some  $x,y \in R$. Then
    $d=(xy)d$, which makes $d(1-xy)=0$. Since $d \neq 0$, we get $xy=1$, making
     $x$ and $y$ units of $R$.
\end{proof}
\begin{corollary}
    If $R$ is commutative, then greatest common divisors are unique.
\end{corollary}

\begin{definition}
    We call an integral domain in which every principle ideal is generated by
    two elements a \textbf{Bezout domain}.
\end{definition}

\begin{lemma}\label{2.1.5}
    Every Euclidean domain is a Bezout domain.
\end{lemma}

\begin{theorem}[The Extended Euclidean Algorithm]\label{2.1.6}
    Let $R$ be a Euclidean and $a,b \neq 0$ elements of  $R$. Let  $d=r_n$ be
    the least nonzero remainder obtained by dividing $a$ from $b$ recursively
    $n+1$ times. Then
    \begin{enumerate}
        \item[(1)] $d=(a,b)$ is the greatest common divisor of $a$ and  $b$.

        \item[(3)] There exist $x,y \in R$ for which  $ax+by=d$.
    \end{enumerate}
\end{theorem}
\begin{proof}
    By lemma \ref{2.1.1}, we get that the ideal $(a,b)$ is principle, so there
    exists a greatest common divisor of $a$ and  $b$. Now, let  $d=r_n$ be
    obtained by dividing $a$ and $b$ recursively $(n+1)$ times. Then the
    $(n+1)^{st}$ equation gives $r_{n-1}=q_{n+1}r_n$, so that $r_n|r_{n-1}$.
    Now, by induction on $n$ if  $r_n|r+{k+1}$ and $r_n|r_k$ then the
    $(k+1)^{st}$ equation gives $r_{k-1}=q_{k+1}r_k+r_{k+1}$, which implies that
    $r_n|r_{k-1}$. Therefore we get in the $1^{st}$ equation that $r_n|b$, and in
    the $0^{th}$ that $r_n|a$. That is,  $d|a$ and  $d|b$.

    Now, notice that $r_0 \in (a,b)$ and that $r_1=b-qr_0 \in (b,r_0) \subseteq
    (a,b)$. By induction on $r_n$, if  $r_{k-1},r_n \in (a,b)$ then
    \begin{equation*}
        r_{k+1}=r_{k-1}-q_{k+1}r_k \in (r_{k-1},r_n) \subseteq (a,b)
    \end{equation*}
    which puts $r_n \in (a,b)$ making $d=(a,b)$ the greatest common divisor.
\end{proof}

\begin{definition}
    Let $R$ be an integral domain, and let
    $\tilde{R}=R^\ast \cup \{0\}$ the set of units together with $0$. We call
    an element  $u \in \com{R}{\tilde{R}}$ a \textbf{universal side divisor} if
    for all $x \in R$, there is a  $z \in \tilde{R}$ such that $u|x-z$.
\end{definition}

\begin{lemma}\label{2.1.7}
    Let $R$ be an integral domain which is not a field. If $R$ is a Euclidean
    domain, then there exist universal side divisors.
\end{lemma}
\begin{proof}
    Notice that since $R$ is not a field, that $\tilde{R} \neq R$ and
    $\com{R}{\tilde{R}}$ is nonempty. Let $N$ be the norm of  $R$, and let  $u
    \in \com{R}{\tilde{R}}$ be of minimal norm. Then for all $x \in R$, take
    $x=qu+r$ with  $r=0$ or  $N(r)<N(u)$> By minimality of $N(u)$, we get $r \in
    \tilde{R}$.
\end{proof}

\begin{example}\label{2.3}
    Notice that $\pm 1$ are the only units in the ring
    $\Z[1+\frac{\sqrt{-19}}{2}]$, so that $\tilde{R}=\{0,1,-1\}$. Suppose that
    $u \in R$ is a universal side divisor, and let $N=\|\cdot\|^2$ be the field
    norm; so that $N(a+(1+\frac{\sqrt{-19}}{2})b)=a^2+ab+5b^2$. If $a,b \in \Z$
    and $b \neq 0$, then we have $a^2+ab+5b^2=(a+\frac{b}{2})^2+\frac{19}{4b^2}
    \geq 5$ so that the smallest nonzero norms are $1$ for $x=1$ and $4$ for $x=2$.
    Now, if $u$ is a universal side divisor, then  $u|2-0$ or  $u|(2 \pm 1)$
    that is $u|2$,  $u|3$ or  $u|1$ making  $u$ a nonunit divsor. If  $2=xy$
    then $4=N(x)N(y)$ and so that $N(x)=1$ or  $N(y)=1$. Hence the only
    divisors of $2$ in  $\Z[1+\frac{\sqrt{-19}}{2}]$ are $\pm 1$ or  $\pm 2$.
    Similarly the only divisors of  $3$ arew  $\pm 1$ or  $\pm 3$ hence  $u=\pm
    2$ or  $u=\pm 3$. Letting  $x=\frac{1+\sqrt{-19}}{2}$, then $x$, nor  $x
    \pm 1$ are divisible by any possible $u$. Therefore
    $\Z[1+\frac{\sqrt{-19}}{2}]$ has no universal side divisors, and cannot be a
    Euclidean domain.
    \end{example}

\section{The Lebesgue Outer Measure}

\begin{definition}
    Let $I \subseteq \R$ be an interval. We define the  \textbf{length} $l(I)$,
    of $I$ to be  $\infty$ if  $I$ is unbounded, and the difference of its
    endpoints otherwise.
\end{definition}

\begin{definition}
    Let $\{I_k\}$ be a countable collection of open bounded intervals covering a
    set $A \subseteq \R$. We define the  \textbf{Lebesgue outer measure} of $A$
    to be
    \begin{equation*}
        m^\ast(A)=\inf{\Big{\{} \sum_{k=1}^\infty{l(I_k)} :
        A \subseteq \bigcup_{k=1}^\infty{I_k} \Big{\}}}
    \end{equation*}
\end{definition}

We now go over some basic properties of the Lebesgue outer measure.

\begin{lemma}\label{8.2.1}
    For any set $A \subseteq \R$,  $m^\ast(A) \geq 0$; in particular,
    $m^\ast(\emptyset)=0$.
\end{lemma}
\begin{proof}
    By definition, since $l(I_k) \geq 0$, each $\sum{l(I_k)} \leq 0$. This makes
    $m^\ast(A) \geq 0$ for any $\{I_k\}$ a countable cover of $A$ by bounded
    open intervals.

    Notice that $\emptyset \subseteq (-\e,\e)$ for some $\e>0$, and that this
    covers $\emptyset$, so that $m^\ast(\emptyset)=l((-\e,\e))=2\e$. Then
    choosing $\e$ small enough gives us  $m^\ast(\emptyset)=0$.
\end{proof}

\begin{lemma}[Monotonicity]\label{8.2.2}
    The Lebesgue Outer Measure is monoton; that is, if $A \subseteq B$, then
    \begin{equation*}
        m^\ast(A) \leq m^\ast(B)
    \end{equation*}
\end{lemma}
\begin{proof}
    Let $\{I_k\}$ be a countable cover of $B$ by bounded open intervals. Then
    notice that  $\{I_k\}$ covers $A$ as well. Now, let
    \begin{align*}
        E   &=  \Big{\{} \sum_{k=1}^\infty{l(I_k)} : A \subseteq
                            \bigcup_{k=1}^\infty{I_k} \Big{\}}  \\
        F   &=  \Big{\{} \sum_{k=1}^\infty{l(I_k)} : B \subseteq
                            \bigcup_{k=1}^\infty{I_k} \Big{\}}  \\
    \end{align*}
    Then since $A \subseteq B$, $F \subseteq E$. Therefore, we get that
    \begin{equation*}
        \inf{E}=m^\ast(A) \leq m^\ast(B)=\inf{F}
    \end{equation*}
\end{proof}

\begin{lemma}\label{8.2.3}
    Countable sets have Lebesgue outer measure $0$.
\end{lemma}
\begin{proof}
    Let $C$ be a countable set, and  $C=\{c_k\}_{k \in \Z^+}$ an enumeration for
    $C$. Let $\e>0$, then for every  $k \in \Z^+$, define the interval
    \begin{equation*}
        I_k=(c_k-\frac{\e}{2^{k+1}}, c_k+\frac{\e}{2^{k+1}})
    \end{equation*}
    Then $\{I_k\}$ is a countable cover of $C$ by bounded open intervals. Thus
    we get
    \begin{equation*}
        0 \leq m^\ast(C) \leq \sum_{k=1}^\infty{l(I_k)}=\sum{\frac{\e}{2^k}}=\e
    \end{equation*}
    Therefore taking $\e$ small enough, we get  $0 \leq m^\ast(C) \leq 0$, and
    equality is established.
\end{proof}

\begin{lemma}\label{8.2.4}
    Let $I \subseteq \R$ be an interval. Then  $m^\ast(I)=l(I)$.
\end{lemma}
\begin{proof}
    Consider first that $I=[a,b]$, a closed bounded interval, where $a<b$. Let
    $\e>0$, then we have that  $[a,b] \subseteq (a-\e,b+\e)$, so that by
    monotonicity
    \begin{equation*}
        m^\ast([a,b]) \leq l((a-\e,b+\e))=b-a+2\e
    \end{equation*}
    Then for $\e$ small enough, this shows that
    \begin{equation*}
        m^\ast([a,b]) \leq b-a
    \end{equation*}

    Now, let $\{I_k\}$ be a countable cover of $[a,b]$ by bounded open
    intervals. Then since $[a,b]$ is compact (by Heine-Borel), there is a finite
    subcover $\{I_k\}_{k=1}^n$ of $[a,b]$. Now, since $a \in
    \bigcup_{k=1}^n{I_k}$, there is atleast one interval $I_k$ containing $a$;
    denote it $ (a_1,b_1)$. Now, $a_1<a<b_1$. If $b_1 \geq b$, then we are done
    as
    \begin{equation*}
        \sum_{k=1}^n{l(I_k)} \geq b_1-a_1>b-a
    \end{equation*}
    Otherwise, $b_1 \in [a,b)$, and since $b_1 \notin (a_1,b_1)$, there is an
    interval $(a_2,b_2)$, distinct from $(a_1,b_1)$, containing $b_1$. Now, if
    $b_2 \geq b$, we are done. Otherwise, proceeding recursively, we obtain a
    subcollection $\{(a_k,b_k)\}_{k=1}^N$ of $\{I_k\}_{k=1}^n$ for which
    \begin{equation*}
        a_1<a \text{ and } a_{k+1}<b_k \text{ for all } 1 \leq k \leq N-1
    \end{equation*}
    The process of selecting such a subcollection must terminate, which leaves
    us with $b_N>b$, so that
    \begin{equation*}
        \sum_{k=1}^n{l(I_k)} \geq \sum_{k=1}^N{l((a_k,b_k))}=
        (b_N-a_N)+\dots+(b_1-a_1) \geq b_N-a_1>b-a
    \end{equation*}
    INdeed, we get
    \begin{equation*}
        \sum_{k=1}^n{l(I_k)} \geq b-a
    \end{equation*}
    so that $m^\ast([a,b])=b-a$.

    Now, suppose that $I$ is any bounded interval. Then for  $\e>0$, there
    exist closed bounded interval $J_1$ and $J_2$ such that
    \begin{equation*}
        J_1 \subseteq I \subseteq J_2
    \end{equation*}
    and
    \begin{equation*}
        l(I)-\e<l(J_1) \leq l(I) \leq l(J_2)<l(I)+\e
    \end{equation*}
    By the monotonicity, and the above discussion of closed bounded intervals,
    we get
    \begin{equation*}
        l(I)-\e<m^\ast(J_1) \leq m^\ast(I) \leq m^\at(J_2)<l(I)+\e
    \end{equation*}
    Therefore, for $\e$ small enough, we get that $m^\ast(I)=l(I)$.

    Finally, suppose that $I$ is an unbounded interval. Then for every  $n \in
    \Z^+$, there is an interval  $J$ with  $l(J)=n$. So that
    \begin{equation*}
        n=m^\ast(J) \leq m^\ast(I)
    \end{equation*}
    This makes $m^\ast(I)=l(I)=\infty$, by definition of $l(I)$.
\end{proof}

\begin{lemma}[Translation Invariance]\label{8.2.5}
    The Lebesgue outer measure is translation invariant; that is, if $A
    \subseteq \R$, and  $y \in \R$,
    \begin{equation*}
        m^\ast(A+y)=m^\ast(A)
    \end{equation*}
\end{lemma}
\begin{proof}
    Let $\{I_k\}$ be a countable cover of $A$ by open bounded intervals. Then
    the collection  $\{I_k+y\}$ is a countable cover of the set $A+y$ by open
    bounded intervals. Moreover, notice that $l(I_k)=l(I_k+y)$. This gives us
    \begin{equation*}
        \sum{l(I_k)}=\sum{l(I_k+y)}
    \end{equation*}
    and we are done.
\end{proof}

\begin{lemma}[Countable Subadditivity]\label{8.2.6}
    The Lebesgue outer measure is countable subadditive; that is, if $\{E_k\}$
    is a countable collection of subsets of $\R$, then
    \begin{equation*}
        m^\ast\Big{(} \bigcup{E_k} \Big{)} \leq \sum{m^\ast(E_k)}
    \end{equation*}
\end{lemma}
\begin{proof}
    Let $\{E_k\}$ be a countable collection of subsets of $\R$, and let
    \begin{equation*}
        E=\bgicup{E_k}
    \end{equation*}
    If at least one of the $E_k$ has  $m^\ast(E_k)=\infty$, then we are done.
    Suppose then that $m^\ast(E_k)$ is finite for all $k \in \Z^+$. Then for
    each  $E_k$, there is a countable cover  $\{I_{k,i}\}_{i \in \Z^+}$ by
    bounded open intervals for which
    \begin{equation*}
        \bigcup{l(I_{k,i})}<m^\ast(E_k)+\frac{\e}{2^k}
    \end{equation*}
    Now, consider the countable collection
    \begin{equation*}
        \{I_{k,i}\}_{k,i \in \Z^+}=\bigcup_{k \in \Z^+}{\{I_{k,i}\}}_{i \in \Z^+}
    \end{equation*}
    Then $\{I_{k,i}\}_{k,i \in \Z^+}$ is a countable cover of $E$ by bounded
    open intervals. We get
    \begin{equation*}
        m^\ast(E) \leq
        \sum_{k \in \Z^+}{\Big{(} \sum_{i \in \Z^+}{l(I_{k,i})} \Big{)}}<
        \sum_{k \in \Z^+}{(m^\ast(E_k)+\frac{\e}{2^k})}=
        \Big{(} \sum{m^\ast(E_k)} \Big{)}+\e
    \end{equation*}
    Taking $\e>0$ small enough, gives us the required subadditivity.
\end{proof}
\begin{corollary}
    The Lebesgue outer measure is finitely subadditive.
\end{corollary}
\begin{proof}
    Recall that finite collections of sets are countable.
\end{proof}

%----------------------------------------------------------------------------------------
%	SECTION 1.1
%----------------------------------------------------------------------------------------

\section{Cauchy Sequences}

\begin{definition}
    We call a sequence $\{x_n\}$ in a metric space $X$ a \textbf{Cauchy sequence} in $X$, or 
    more simply, \textbf{Cauchy} in $X$ if for all $\epsilon>0$, there is an  $N \in \Z^+$ such that 
    $d(x_n,x_m)<\epsilon$ whenever  $m,n \geq N$.
\end{definition}

\begin{definition}
    Let $E$ be a nonempty subset of a metrix space  $X$, and lelt $S \subseteq \R$ be the 
    all real numbers $d(x,y)$, with  $x,y \in E$. We call  $\sup{S}$ the  \textbf{diameter} 
    of $E$, and denote it  $\diam{E}$.
\end{definition}

\begin{theorem}\label{3.3.1}
    Let $\{x_n\}$ be a sequence, and let  $E_N$ be the set of all points  $p_N$ such that 
    $_N<p_{n+1}$. Then $\{x_n\}$ is Cauchy if and only if  $\lim{\diam{E_N}}=0$ as 
     $N \rightarrow \infty$.
\end{theorem}
\begin{proof}
    Let $\{x_n\}$ be Cauchy,  Let  $x_{N_1},x_{N_2} \in E$ such that $d(x_n,x_{N_1})<
    \frac{\epsilon}{2}$, and $d(x_{N_2},x_m)<\frac{\epsilon}{2}$. Then we see that $d(x_{N_1},x_{N_2}) \leq 
    d(x_{N_1},x_n)+d(x_m, x_{N_2})<\epsilon$, so $\{x_{N_k}\}$ is Cauchy and we see that 
    $\lim{\diam{E_N}}=0$. Now suppose that  $\lim{\diam{E}}=0$, then for any  $x_n,x_m \in S$, 
    $d(x_n,0)< \frac{\epsilon}{2}$ and $d(0,x_m)< \frac{\epsilon}{2}$ implies that $d(x_n,x_m) \leq 
    d(x_n,0)+d(0,x_m)<\epsilon$, whenever  $n,m>N$, for  $\epsilon>0$.
\end{proof}

\begin{theorem}\label{3.3.2}
    \begin{enumerate}[label=(\arabic*)]
        \item If $E \subseteq X$, then  $\diam{\hat{E}}=\diam{E}$.

        \item If $\{K_n\}$ is a sequence of compact sets in  $X$, such that  $K_{n+1} \subseteq K_n$, and 
            if  $\lim{\diam{K_n}}=0$ as  $n \rightarrow \infty$, then  $\bigcap_{i=1}^{infty}{K_i}$ contains 
            exactly one point.
    \end{enumerate}
\end{theorem}
\begin{proof}
    Clearly $\diam{E} \leq \diam{\hat{E}}$. Now let  $\epsilon>0$, and choose  $x,y \in \hat{E}$, then 
    there are points $x',y' \in \hat{E}$ such that  $d(x,x')<\frac{\epsilon}{2}$ and $d(y,y')<
    \frac{\epsilon}{2}$. Hence, $d(x,y) \leq d(x,x')+d(x',y')+d(y'y)<\epsilon \diam{E}$, then 
    choosing  $\epsilon$ arbitrarily small,  $\diam{\hat{E}} \leq \diam{E}$.

    Now, we also have that by the nested interval theorem that $K=\bigcap{K_i}$ is nonempty. Now 
    suppose that  $K$ contains more that one point. then $\diam{K}>0$, and since  $K \subseteq K_n$ for 
    all  $n$,  $diam{K} \leq \diam{K_n}$, a contradiction. Thus  $K$ contains exactly one element.
\end{proof}

\begin{theorem}\label{3.3.3}
    \begin{enumerate}[label=(\arabic*)]
        \item In any metric space $X$, every convergent sequence is a Cauchy sequence.

        \item If $X$ is compact, and  $\{x_n\}$ is Cauchy in  $X$, then $\{x_n\}$ 
            converges to a point in  $X$.
    \end{enumerate}
\end{theorem}
\begin{proof}
    \begin{enumerate}[label=(\arabic*)]
        \item If $x_n \rightarrow x$, and $\epsilon>0$ such that there is an  $N \in \Z^+$ such that 
            $d(x_n,x)<\frac{\epsilon}{2}$ for all $n \geq N$, then for  $m \geq N$, we have 
            $d(x_n,x_m) \leq d(x_n,x)+d(x,x_m)<\epsilon$. Thus  $\{x_n\}$ is Cauchy.

        \item Let $\{x_n\}$ be Cauchy, and let  $E_N$ be the set of all points  $x_N$ for 
            which  $x_N<x_{N+1}$. Then  $\lim{\diam{\hat{E}}}=0$, then being closed in  $X$, each 
            $\hat{E_N}$ is compact in  $X$, and  $\hat{E_{N+1}} \subseteq \hat{E_N}$, so 
            by theorem \ref{3.3.2}, there is a unique  $x \in X$ in all of $\hat{E_N}$. Now for 
            $\epsilon>0$, there is an  $N_0 \in \Z^+$ for which $\diam{\hat{E}}<\epsilon$. Then for all 
            $x_n \in \hat{E}$,  $d(x_n,x)<\epsilon$ whenever  $n \geq N_0$.
    \end{enumerate}		
\end{proof}

\begin{corollary}[The Cauchcy Criterion]
    Every Cauchy sequence in $\R^k$ converges to a point in  $\R^k$.
\end{corollary}
\begin{proof}
    Let $\{x_n\}$ be Cauchy in  $\R^k$, define $E_N$ as in  $(2)$, then for some  $N \in \Z^+$, 
$\diam{E}<1$, and so  $\{x_n\}$ us the union of all  $E_n$, and ther set of points  
$\{x_1,\dots, x_{N-1}\}$, so $\{x_n\}$ is bounded, and thus has a compact closure, it follows then 
that  $x_n \rightarrow x$ for some  $x \in \R^k$.
\end{proof}

\begin{definition}
    We call a metric space \textbf{complete} if every Cauchy sequence in the space converges.		
\end{definition}

\begin{theorem}\label{3.3.4}
    All compact metric spaces, and all Euclidean spaces are complete.
\end{theorem}

\begin{example}
    Consider $\Q$ together with the metric  $|x-y|$	. The metric space induced on $\Q$ by  $|\cdot|$ is 
    not complete.
\end{example} 

\begin{definition}
    A sequence $\{x_n\}$ in $\R$ is said to be \textbf{monotonically increasing} if 
    $x_n \leq x_{n+1}$, $\{x_n\}$ is said to be \textbf{monotonically decreasing} if $x_{n+1}<x_n$. 
    We call  $\{x_n\}$ \textbf{monotonic} if it is either monotonically increasing or monotonically 
    decreasing.
\end{definition}

\begin{theorem}\label{3.3.5}
    A monotonic sequence converges if and only if it is bounded.
\end{theorem}
\begin{proof}
    Suppose, without loss of generality, that $\{x_n\}$ is monotonically increasing. If 
    $\{x_n\}$ is bounded, then  $x_n \leq x$, then for all  $\epsilon>0$, there is an  $N \in \Z^+$ 
    for which  $x-\epsilon<x_N \leq x$. Then for  $n \geq N$,  $x_n \rightarrow x$. The converse follows 
    from theorem  \ref{3.1.2}.
\end{proof}

%%----------------------------------------------------------------------------------------
%	SECTION 1.4
%----------------------------------------------------------------------------------------

\section{The Complex Field}

\begin{definition}
    We define a \textbf{complex number} to be a pair of real numbers  $(a,b)$. We denote the 
    set of all comlex numbers by $\C$. We define the \textbf{addition} and $\textbf{multiplication}$ of 
    complex numbers to be the binary operations $+:\C \rightarrow \C$ and $\cdot: \C \rightarrow \C$ such that 
        \begin{align*}
            (a,b)+(c,d) &= (a+c,b+d) \\
            (a,b)(c,d) &= (ac-bd,ad+bc) \\
        \end{align*}
        Lastly, we define $i$ to be the complex number such that  $i=(0,1)$.
\end{definition}

\begin{theorem}\label{1.4.1}
    $\C$ forms a field together with  $+$ and  $\cdot$.
\end{theorem}

\begin{theorem}\label{1.4.2}
    For $(a,0),(b,0) \in C$,  $(a,0)+(b,0)=(a+b,0)$, and $(a,0)(b,0)=(ab,o)$.
\end{theorem}
\begin{proof}
    This is a straightforward application of the addition and multiplication of 
    complex numbers.
\end{proof}

\begin{theorem}\label{1.4.3}
    $i^2=-1$.
\end{theorem}
\begin{proof}
    $i^2=(0,1)(0,1)=(0-1,1-1)=(-1,0)=-1$.		
\end{proof}

\begin{theorem}\label{1.4.4}
    Let $(a,b) \in \C$, then  $(a+b)=a+ib$.
\end{theorem}
\begin{proof}
    $(a,b)=(a,0)+(0,b)=(a,0)+(0,1)(b,0)=a+ib$.	
\end{proof}

\begin{definition}
    Let $a,b \in \R$, and let  $z \in \C$ such that  $z=a+ib$. We define the  \textbf{complex 
    conjugate} of $z$ to be the complex number  $\bar{z}=a-ib$. Moreover, we define the 
    \textbf{real part} of  $z$ to be $a$, and the \textbf{imaginary part} of  $z$ to be $b$, 
    and we denote them $a=\re{z}$,  $b=\im{z}$
\end{definition}

\begin{theorem}\label{1.4.5}
    Let $z, w \in \C$. Then 
        \begin{enumerate}[label=(\arabic*)]
            \item $\overline{z+w}=\bar{z}+\bar{w}$.

            \item $\overline{zw}=\bar{z}\bar{w}$.

            \item $z+\bar{z}=2\re{z}$ and  $z-\bar{z}=2i\im{z}$.

            \item $z\bar{z}$ is a nonegative real number.
        \end{enumerate}
\end{theorem}
\begin{proof}
    Let $z=a+ib$, and let  $w=c+id$. Then $z+w=(a+c)+i(b+d$, so  $\overline{z+w}=(a+b)
    -i(b+d)=(a-ib)+(c-id)=\bar{z}+\bar{w}$; similarly, we get  $\overline{zw}=\bar{z}\bar{w}$. 
    Moreover, we have  $(a+ib)+(a-ib)=2a$, and  $(a+ib)-(a-ib)=2ib$, we also have that 
    $z\bar{z}=(a+ib)(a-ib)=a^2+b^2 \geq 0$, and $z\bar{z}=0$ if and only if $a=b=0$.
\end{proof}

\begin{definition}
    Let $z \in \C$. We define the  \textbf{modulus} of  $z$ to be  $|z|=\sqrt{z\bar{z}}$.
\end{definition}
\begin{remark}
    $|z|$ exists and is unique.
\end{remark}

\begin{theorem}\label{1.4.6}
    Let $z,w \in \C$, then:
        \begin{enumerate}[label=(\arabic*)]
            \item $|z| \geq 0$ and  $|z|=0$ if and only if $z=0$.

            \item  $|\bar{z}|=|z|$.

            \item $|zw|=|z||w|$.

            \item  $\re{z} \leq |z|$.

            \item  $|z+w+ \leq |z|+|w|$.
        \end{enumerate}
\end{theorem}
\begin{proof}
    Let $z=a+ib$, and  $w=c+id$. Then  $|z|=\sqrt{a^2+b^2} \geq 0$, and  $|z|=0$ if 
    and only if  $a,b=0$. Moreover,  $|\bar{z}|=|a+i(-b)|=\sqrt{a^2+(-b)^2}=\sqrt{a^2+b^2}=|z|$. 
    We also habe $|zw|^2=(a^2+b^2)(c^2+d^)=|z|^2|w|^2$, likewise,  $\|re{z}|=|a+i0|=
    \sqrt{a^2} \leq \sqrt{a^2+b^2}$. Finally we prove $(5)$.

    We have  $|z+w|^2=(x+w)(\bar{z}+\bar{w})=z\bar{z}+\bar{z}w+\bar{w}z+w\bar{w}= 
    |z|^2+w\re{z\bar{w}}+|w|^2 \leq |z|^2+2|s\bar{w}|+|w|^2=(|z|+|w|)^2$.
\end{proof}

\begin{theorem}[The Cauchy Schwarz Inequality]\label{1.4.7}
    Let $a_i,b_i \in \C$, for  $1 \leq i \leq n$. Then:
         \begin{equation}
             |\sum_{i=1}^{n}{a_i\bar{b_i}}| \leq \sum_{i = 1}^{n}{|a_i|^2}\sum_{i=1}^{n}{|b_j|^2}		
        \end{equation}
\end{theorem}
\begin{proof}
    Let $A=\sum{a_j|^2}$,  $B=\sum{|b_i|^2}$, and $C=\sum{a_i\bar{b_i}}$. If  $B=0$, then 
     $b_i=0$ for  $1 \leq i \leq n$, and we are done; so suppose that  $B>0$. Then
        \begin{align*}
            \sum{|Ba_j-Cb_j|^2} &= \sum{(Ba_j-Cb_j)(B\bar{a_j}-\bar{Cb_j})} \\
                             &= B\sum{|a_j|^2}-B\bar{C}\sum{a_j\bar{b_j}-BC\sum{\bar{a_j}b_j}}+|C^2|\sum{|b_j|^2} \\
                             &= (B^2A-B|C|^2)=B(AB-|C|^2) \geq 0 \\
        \end{align*}
    Since $B>0$, we get  $|C|^2 \leq AB$ as required.
\end{proof}

%\section{Exact Sequences of Modules}

\begin{definition}
    Let $A$ and  $C$ be modules. We call a module  $B$, containing  $A$ an
    \textbf{extension} of $C$ by  $A$ if  $\faktor{B}{A} \simeq C$.
\end{definition}

\begin{example}\label{example_4.13}
    Let $A$,  $B$, and  $C$ be modules. If  $A$ is a submodule of  $B$, then
    there exists a map 1--1 $\psi:A \xrightarrow{} B$ such that $A \simeq
    \psi(A) \subseteq B$. If $\faktor{B}{\psi(A)} \simeq C$, then there exists a
    map $\phi:B \xrightarrow{} C$ wich is onto, for which $\ker{\phi}=\psi(A)$.
    Then we get the following diagram.
    \begin{equation*}
        A \xrightarrow{\psi} B \xrightarrow{\phi} C
    \end{equation*}
\end{example}

\begin{definition}
    Let $\{A_n\}$ a collection of modules. By a \textbf{sequence}, we mean a
    diagram
    \begin{equation*}
        \dots \xrightarrow{} A_{n-1} \xrightarrow{} A_n \xrightarrow{} A_{n+1}
        \xrightarrow{} \dots
    \end{equation*}
    where each $A_i$ is mapped to $A_{i+1}$ by some module homomorphism.
\end{definition}

\begin{definition}
    Let $A$,  $B$, and  $C$ be modules. We call a pair of module homomorphisms
    $\a,\b$, defined by the diagram
    \begin{equation*}
        A \xrightarrow{\a} B \xrightarrow{\b} C
    \end{equation*}
    \textbf{exact} at $B$ if  $\a(A)=\ker{\b}$. If $\{A_n\}$ is a collection of
    modules, we call a sequence defined by the diagram
    \begin{equation*}
        \dots \xrightarrow{} A_{n-1} \xrightarrow{} A_n \xrightarrow{} A_{n+1}
        \xrightarrow{} \dots
    \end{equation*}
    an \textbf{exact sequence} if it is exact at each $A_n$.
\end{definition}

\begin{example}\label{example_4.14}
    The sequence from example \ref{example_4.13}, $A \xrightarrow{\psi} B
    \xrightarrow{\phi} C$, is exact at $B$,  since $\ker{\phi}=\psi(A)$.
\end{example}

\begin{lemma}\label{4.5.1}
    Let $R$ be a ring, and  $A$,  $B$, and  $C$  $R$-modules. The following are
    true for any $R$-module homomorphisms $\psi:A \xrightarrow{} B$ and $\psi:B
    \xrightarrow{} C$
    \begin{enumerate}
        \item[(1)] $(0) \xrightarrow{\i} A \xrightarrow{\psi} B$ is exact at $A$
            if, and only if $\psi$ is 1--1.

        \item[(2)] $B \xrightarrow{\phi} C \xrightarrow{\i'} (0)$ is exact at
            $B$ if, and only if $\phi$ is onto.
    \end{enumerate}
\end{lemma}
\begin{corollary}
    The sequence $(0) \xrightarrow{} A \xrightarrow{\psi} B \xrightarrow{\phi} C
    \xrightarrow{} (0)$ is an exact sequence if, and only if $\psi$ is 1--1,
    $\phi$ is onto, and $\ker{\phi}=\psi(A)$; that is, $B$ is an extension of
    $C$ by  $A$.
\end{corollary}

\begin{definition}
    We call an exact sequence of the form $(0) \xrightarrow{} A \xrightarrow{\psi}
    B \xrightarrow{\phi} C \xrightarrow{} (0)$ a \textbf{short exact sequence}.
\end{definition}

\begin{lemma}\label{4.5.2}
    If $A \xrightarrow{\a} B \xrightarrow{\b} C$ is exact at $Y$, then  $(0)
    \xrightarrow{} \a(A) \xrightarrow{} B \xrightarrow{} \faktor{B}{\ker{\b}}
    \xrightarrow{} (0)$ is a short exact sequence.
\end{lemma}

\begin{example}\label{example_4.15}
    \begin{enumerate}
        \item[(1)] Let $A$, and $C$ be modules. Consider the sequence
            \begin{equation*}
                (0) \xrightarrow{} A \xrightarrow{\i} A \oplus C
                \xrightarrow{\pi} C \xrightarrow{} (0)
            \end{equation*}
            where $\i$ is the inclusion map, and  $\pi$ is the projection map
            about the second coordinate; i.e. $\pi:(a,c) \xrightarrow{} c$. This
            sequence is a short exact sequence, since $\i(A)=\ker{\pi}$.

        \item[(2)] Consider the $\Z$-modules $\Z$ and $\faktor{\Z}{n\Z}$,
            the sequence
            \begin{equation*}
                (0) \xrightarrow{} \Z \xrightarrow{\i} \Z \oplus \faktor{\Z}{n\Z}
                \xrightarrow{\phi} \faktor{\Z}{n\Z} \xrightarrow{} (0)
            \end{equation*}
            is a short exact sequence giving an extension of $\faktor{\Z}{n\Z}$
            by $\Z$. Another extension is given by the short exact sequence
            \begin{equation*}
                (0) \xrightarrow{} \Z \xrightarrow{\n} \Z
                \xrightarrow{\pi} \faktor{\Z}{n\Z} \xrightarrow{} (0)
            \end{equation*}
            where $n:x \xrightarrow{} nx$, and $\pi:x \xrightarrow{} x \mod{n}$
            is the natural map. THese are ``inequivalent'' extensions of
            $\faktor{\Z}{n\Z}$ by $\Z$.

        \item [(3)] If $\phi:B \xrightarrow{} C$ is any module homomorphism,
            form the exact sequence
            \begin{equation*}
                (0) \xrightarrow{} \ker{\phi} \xrightarrow{\i} B
                \xrightarrow{\phi} \phi(B) \xrightarrow{} (0)
            \end{equation*}
            where $\i$ is the inclusion map. If  $\phi$ is onto, we may extend
            the sequence $B \xrightarrow{\phi} C$ (i.e. extend $\phi$) to a
            short exact sequence with $A=\ker{\phi}$.

        \item[(4)] Let $R$ be a ring, and  $M$ an  $R$-module homomorphism. Let
             $S$ be a set of generators for  $M$, and consider the free $R$-module
             $F(S)$ on $S$. Then
             \begin{equation*}
                 (0) \xrightarrow{} K \xrightarrow{\i} F(S) \xrightarrow{\phi}
                 \xrightarrow{} M \xrightarrow{} (0)
             \end{equation*}
             is a short exact sequence,  where $\phi$ is the unique  $R$-module
             homomorphism  which is the identity on $S$, and  $K=\ker{\phi}$.
    \end{enumerate}
\end{example}

\begin{definition}
    Let $(0) \xrightarrow{} A \xrightarrow{} B \xrightarrow{} C \xrightarrow{}
    (0)$ and $(0) \xrightarrow{} A' \xrightarrow{} B' \xrightarrow{} C'
    \xrightarrow{} (0)$ be short exact sequences. A \textbf{homomorphism} of
    sequences is a triple $(\a,\b,\y)$ of module homomorphisms, such that the
    following diagram commutes
    \begin{equation*}
        \begin{tikzcd}
            {(0)} & A & B & C & {(0)} \\
            {(0)} & {A'} & {B'} & {C'} & {(0)}
            \arrow[from=1-1, to=1-2]
            \arrow[from=1-2, to=1-3]
            \arrow[from=1-3, to=1-4]
            \arrow[from=1-4, to=1-5]
            \arrow[from=2-1, to=2-2]
            \arrow[from=2-2, to=2-3]
            \arrow[from=2-3, to=2-4]
            \arrow[from=2-4, to=2-5]
            \arrow["\gamma"', from=1-4, to=2-4]
            \arrow["\beta"', from=1-3, to=2-3]
            \arrow["\alpha"', from=1-2, to=2-2]
        \end{tikzcd}
    \end{equation*}
    If $\a$,  $\b$, and  $\y$ are  module isomorphisms, we call $(\a,\b,\y)$ an
    isomorphism of sequences.
\end{definition}


\begin{definition}
    We call two exact sequences $A \xrightarrow{} B \xrightarrow{} C$ and $A'
    \xrightarrow{} C' \xrightarrow{} B'$ \textbf{equivalent} if $A=A'$, $C=C'$,
    and there exists an isomorphism of sequences between them. We call  the
    corresponding extensions  $B$ and  $B'$  \textbf{equivalent}.
\end{definition}

\begin{lemma}\label{4.5.3}
    The composition of homomorphisms of exact sequences is a homomorphism
    of sequences.
\end{lemma}

\begin{lemma}\label{4.5.4}
    Isomorphisms of exact sequences form an equivalence relation on any set of
    exact sequences.
\end{lemma}

\begin{example}\label{example_4.16}
    \begin{enumerate}
        \item[(1)] Let $m,n \in \Z^+$ integers greater than $1$, and suppose
            that  $n|m$. Let  $k=\frac{m}{n}$, and define a map from the exact
            sequences of $\Z$-modules described by the following diagram
            \begin{equation*}
                \begin{tikzcd}
{(0)} & \Z & \Z & {\faktor{\Z}{n\Z}} & {(0)} \\
{(0)} & {\faktor{\Z}{k\Z}} & {\faktor{\Z}{k\Z}} & {\faktor{\Z}{n\Z}} & {(0)}
\arrow[from=1-1, to=1-2]
\arrow["n", from=1-2, to=1-3]
\arrow["\pi", from=1-3, to=1-4]
\arrow[from=1-4, to=1-5]
\arrow[from=2-1, to=2-2]
\arrow["\iota", from=2-2, to=2-3]
\arrow["{\pi'}", from=2-3, to=2-4]
\arrow[from=2-4, to=2-5]
\arrow["\alpha"', from=1-2, to=2-2]
\arrow["\beta"', from=1-3, to=2-3]
\arrow["\gamma"', from=1-4, to=2-4]
                \end{tikzcd}
            \end{equation*}
            where $n:x \xrightarrow{} nx$, $\pi:x \xrightarrow{} x \mod{n}$,
            $\a$,  $\b$ are the natural projections, and  $\y$ is the identity.
            For the second sequnece, we take  $\i:a \mod{k} \xrightarrow{}
            na\mod{m}$, and $\pi'$ the natural projection of $\faktor{\Z}{m\Z}$
            onto $\faktor{(\faktor{\Z}{m\Z})}{(\faktor{n\Z}{m\Z})} \simq
            \faktor{\Z}{n\Z}$. Then $(\a,\b,\y)$ describe a homorphism of short
            exact sequences.

        \item[(2)] If $(0) \xrightarrow{} \Z \xrightarrow{n} \Z \xrightarrow{\pi}
            \faktor{\Z}{n\Z} \xrightarrow{} (0)$ is the short exact sequence
            defined in example \ref{example_4.15}, mapping each module to itself
            via the map $x \xrightarrow{} -x$ gives a isomorphism of short exact
            sequences, which take this sequence onto itself. Notice however that
            this isomorphism is not an equeivalence, since it is not the
            identity on $\Z$.

        \item[(3)] Consider the diagram
            \begin{equation*}
\begin{tikzcd}
	{(0)} & {\faktor{\Z}{2\Z}} & {\faktor{\Z}{2\Z} \oplus \faktor{\Z}{2\Z}} &
                                {\faktor{\Z}{2\Z}} & {(0)} \\
	{(0)} & {\faktor{\Z}{2\Z}} & {\faktor{\Z}{2\Z} \oplus \faktor{\Z}{2\Z}} &
                                {\faktor{\Z}{2\Z}} & {(0)}
	\arrow["\psi", from=1-2, to=1-3]
	\arrow["\phi", from=1-3, to=1-4]
	\arrow[from=1-4, to=1-5]
	\arrow[from=2-1, to=2-2]
	\arrow["{\psi'}"', from=2-2, to=2-3]
	\arrow["{\phi'}"', from=2-3, to=2-4]
	\arrow[from=2-4, to=2-5]
	\arrow["i"', from=1-4, to=2-4]
	\arrow["\beta"', from=1-3, to=2-3]
	\arrow["i"', from=1-2, to=2-2]
	\arrow[from=1-1, to=1-2]
\end{tikzcd}
            \end{equation*}
            Where $i$  is the identity map, $\psi$ is a 1--1 map mapping into the
            first component of $\faktor{\Z}{2\Z} \oplus \faktor{\Z}{2\Z}$, and
            $\phi$ projects $\faktor{\Z}{2\Z} \oplus \faktor{\Z}{2\Z}$ onto its
            second component, and where $\psi'$ and  $\phi'$ behave just as  $\psi$
            and  $\phi$. If $\b$ maps $\faktor{\Z}{2\Z} \oplus \faktor{\Z}{2\Z}$
            to $\faktor{\Z}{2\Z} \oplus \faktor{\Z}{2\Z}$ by exchanging the
            factors; that is  $\b:(m,n) \xrightarrow{} (n,m)$, then this diagram
            commutes and gives an equivalence of short exact sequences which is
            not the identity.
    \end{enumerate}
\end{example}

\begin{lemma}[The Short Five Lemma]\label{4.5.5}
    Let $(\a,\b,\y)$ be a homomorphism of short exact sequences given by the
    diagram
    \begin{equation*}
        \begin{tikzcd}
            {(0)} & A & B & C & {(0)} \\
            {(0)} & {A'} & {B'} & {C'} & {(0)}
            \arrow[from=1-1, to=1-2]
            \arrow[from=1-2, to=1-3]
            \arrow[from=1-3, to=1-4]
            \arrow[from=1-4, to=1-5]
            \arrow[from=2-1, to=2-2]
            \arrow[from=2-2, to=2-3]
            \arrow[from=2-3, to=2-4]
            \arrow[from=2-4, to=2-5]
            \arrow["\gamma"', from=1-4, to=2-4]
            \arrow["\beta"', from=1-3, to=2-3]
            \arrow["\alpha"', from=1-2, to=2-2]
        \end{tikzcd}
    \end{equation*}
    then the following are true
    \begin{enumerate}
        \item[(1)] If $\a$ and  $\y$ are 1--1, then so is  $\b$.

        \item[(2)] If $\a$ and  $\y$ are onto, then so is  $\b$.

        \item[(3)] If $\a$ and  $\y$ are isomorphisms, then so is  $\b$.
    \end{enumerate}
\end{lemma}
\begin{proof}
    We chase the elements of the following diagram.
    \begin{equation*}
        \begin{tikzcd}
            {(0)} & A & B & C & {(0)} \\
            {(0)} & {A'} & {B'} & {C'} & {(0)}
            \arrow[from=1-1, to=1-2]
            \arrow["\psi", from=1-2, to=1-3]
            \arrow["\phi", from=1-3, to=1-4]
            \arrow[from=1-4, to=1-5]
            \arrow[from=2-1, to=2-2]
            \arrow["{\psi'}"', from=2-2, to=2-3]
            \arrow["{\phi'}", from=2-3, to=2-4]
            \arrow[from=2-4, to=2-5]
            \arrow["\gamma"', from=1-4, to=2-4]
            \arrow["\beta"', from=1-3, to=2-3]
            \arrow["\alpha"', from=1-2, to=2-2]
        \end{tikzcd}
    \end{equation*}
    Suppose that $\a$ and  $\y$ are 1--1, and choose an element $b \in
    \ker{\b}$. Then $\b(b)=0$, and $\phi'\b(b)=0$. Moreover, since this diagram
    commutes, we have that $\phi'\b=\y\phi$, so that $\phi'\b(b)=\y\phi(b)$,
    which implies that $\phi(b)=0$, since $\y$ is 1--1. This makes $b \in
    \ker{\phi}$. Now, since the sequences are also exact, we have
    $\ker{\phi}=\psi(A)$, so that $b \in \psi(A)$. That is, there is an $a \in
    A$ for wich $b=\psi(a)$. Now, by commutativity again, we have
    $\b\psi=\psi'\a$. So  $\b\psi(b)=\b(b)=0$, which makes $\psi'\a(a)=0$, so
    that $\a(a)=0$. Since $\a$ is 1--1, this makes $a=0$ and we get
    $b=\psi(a)=0$ which makes $\ker{\b}=(0)$. That is, $\b$ is 1--1.

    Now, suppose that $\a$ and $\y$ are onto, and let  $b' \in B'$. Then
    $\phi'(b)=\y(c)$ for some $c \in C$; since $\y$ is onto. Now, by lemma
    \ref{4.5.1} $\phi$ is onto, so there is a $b \in B$ for which $b=\y(c)$. By
    the commutativity of the diagram $\phi'\b(b)=\y(\phi(b))=\y(c)=\phi'(b')$,
    so $\phi'(b'-\b(b))=0$ which puts $b'-\b(b) \in \ker{\phi'}$. Now, by the
    exactness of the sequences, $\ker{\phi'}=\psi'(A')$, so there is an $a' \in
    A'$ for which $b'-\b(b)=\psi'(a)=a \in A$, since $\a$ is onto. Then $a \in
    \ker{\phi'}$. Now, by commutativity, observe that
    $\psi(a)=\psi(b'-\b(b))=0$, and since $\psi$ is 1--1, we get $b'-\b(b)=0$.
    That is $b'=\b(b)$ and $\b(B)=B'$, which makes $\b$ onto. Lastly, observe
    that if $\a$ and $\y$ are isomorphisms, then they are 1--1 and onto, which
    makes $\b$ 1--1 and onto, and hence an isomorphism as well.
\end{proof}




%\bibliographystyle{abbrv}
%\bibliography{references}

\end{document}
