%----------------------------------------------------------------------------------------
%	SECTION 1.1
%----------------------------------------------------------------------------------------

\section{The Product Topology.}

We now explore more about the product topology.

\begin{definition}
    Let $J$ be an indexed set, and let  $X$ be a set. We define a  \textbf{$J$-tuple} of elements of
    $X$ to be a map $x:J \rightarrow X$, where if  $\alpha \in J$, then  $x(\alpha)=x_{\alpha}$, and
    we call it the \textbf{$\alpha$-th coordinate} of $x$. We write  $(x_{\alpha})_\alpha \in J$, or
just simply $(x_{\alpha})$
\end{definition}

\begin{definition}
Let $\{A_{\alpha}\}$ be an indexed family, and let $X=\bigcup_{\alpha \in J}{A_{\alpha}}$. We define
the \textbf{cartesian product} of $\{A_{\alpha}\}$, $\prod_{\alpha \in J}{A_\alpha}$ to be the set of
all $J$-tuples  $(x_{\alpha})$ of elements of $X$, where  $x_{\alpha} \in A_{\alpha}$
\end{definition}

\begin{theorem}\label{2.1.1}
    Let $\{X_{\alpha}\}$ be a family of topological spaces, and consider the cartesian product
    $\prod{X_{\alpha}}$. Then the collection of all cartesian products $\prod{U_{\alpha}}$, where
    $U_{\alpha}$ is open in $X_{\alpha}$, for all $\alpha$, forms a basis for the topology on
    $\prod{X_{\alpha}}$.
\end{theorem}
\begin{proof}
    Clearly $\prod{X_{\alpha}}$ itself is a basis element by the first condition. Now consider
    $\prod{U_{\alpha}}$ and $\prod{V_{\alpha}}$, then $\prod{U_{\alpha}} \cap
    \prod{V_{\alpha}}=\prod{U_{\alpha} \cap V_{\apha}}$, which is also a basis element.
\end{proof}

\begin{definition}
Let $\{X_{\alpha}$ be a family of topological spaces, and take as basis the collection of all
    products $\prod{U_{\alpha}}$ where $U_{\alpha}$, where $U_{\alpha}$ is open in $X_{\alpha}$.
    We call the topology generated by this basis the \textbf{box topology} on $\prod{X_{\alpha}}$.		
\end{definition}

\begin{definition}
    Let $\pi_{\beta}:\prod{X_{\alpha}} \rightarrow X_{\beta}$ be defined by
    $\pi_{\beta}((x_\alpha))=x_\beta$. We call this map the \textbf{projection mapping} of
    $\prod{X_{\alpha}}$ onto $X_{\beta}$		
\end{definition}

\begin{theorem}\label{2.1.2}
    Let $\Sc_{\beta}=\{\pi^{-1}_{\beta}(U_{\beta}):U_{\beta} \text{ is open in } X_{\beta}\}$, and let
    $\Sc=\bigcup{\Sc_{\beta}}$. Then $\Sc$ forms the basis for a topology on  $\prod{X_{\alpha}}$.
\end{theorem}
\begin{proof}
    Since $U_{\beta}$ is open in $X_{\beta}$, $\pi^{-1}_{\beta}(U_{\beta})) \subseteq
    \prod{X_\alpha}$. Taking $\bigcup{\Sc}$, we get that $\bigcup{\pi^{-1}_{\beta}(U_{\beta}))} =
    \prod{X}_{\beta}$ for all $\beta$. Thus  $\Sc$ is a subbasis.
\end{proof}


\begin{definition}
    Let $\pi_{\beta}$ be a projection mapping of $\prod{X_{\alpha}}$ onto $X_{\beta}$, and take as
    subbasis the collection of all $\pi^{-1}_{\beta}(U_{\beta})$ where $U_{\beta}$ is open in
    $X_{\beta}$. We call the topology generated by this subbasis the \textbf{product space
    topology}, or more generally the \textbf{product topology} on $\prod{X_{\alpha}}$.		
\end{definition}

\begin{theorem}\label{2.1.3}
    The box topology on $\prod{X_{\alpha}}$ has as basis all sets of the form $\prod{U_{\alpha}}$,
    where $U_{\alpha}$ is open in $X_{\alpha}$, and the product space topology has as basis all sets
    of the form $\prod{U_{\alpha}}$, where $U_{\alpha}$ is open in $X_{\alpha}$, and
    $U_{\alpha}=X_{\alpha}$ except only for finitely many $\alpha$.
\end{theorem}
\begin{proof}
    That the box topology has as a basis all sets of the form  $\prod{U_{\alpha}}$ is clear. Now 
    consider the basis $\Bc$ that  $\Sc$ generates, and let  $\beta_1, \dots \beta_n$ be a finite
    set of distinct indices and let $U_{\beta_i}$ be open in $X_{\beta_i}$, and
    $U_{\alpha}=X_{\alpha}$ for all other $\alpha$. Since  $B \in \Bc$ is a finite intersection of
    elements of  $\Sc$, we have that  $B=\bigcap_{i = 1}^{n}{\_{\beta_i}^{-1}(U_{\beta_i})}$.

    Now a point $x=(x_{\alpha}) \in B$ if and only if the $\beta_i$-th coordinate is in
    $U_{\beta_i}$, for $1 \leq i \leq n$, hence membership depends only on a finite number of
    $\alpha$, thus  $B=\prod{U_{\alpha}}$ where $U_{\alpha}=X_{\alpha}$ for $\alpha \neq \beta_i$
    for  $1 \leq i \leq n$.
\end{proof}

\begin{corollary}
    The box topology on $\prod{X_{\alpha}}$ is finer than the product topology on
    $\prod{X_{\alpha}}$; moreover, if $\{X_{i}\}_{i=1}^{n}$ is a finite  family of topologies, then
    the box toplogy, and the product topology on $\prod_{i=1}^{n}{X_i}$ are equal.
\end{corollary}

For convention, from now on when we consider the product $\prod{X_{\alpha}}$, we assume that it is
under the product space topology.

\begin{theorem}\label{2.1.4}
    Suppose the topology on $X_{\alpha}$ is given by a basis $\Bc_{\alpha}$. The collection of all
    sets $\prod{B_{\alpha}}$ where $B_{\alpha} \in \Bc_{\alpha}$ for each $\alpha$ is a basis for
    the box topology on  $\prod{X_{\alpha}}$.

    The same collection for a finite number of $\alpha$, and where  $B_{\alpha}=X_{\alpha}$ for all
    other $\alpha$ forms a basis for the product space topology on  $\prod{X_{\alpha}}$.
\end{theorem}
\begin{proof}
    Let $\Bc$ be the collection of all  $\prod{B_{\alpha}}$, where $B_{\alpha} \in \Bc_{\alpha}$.
    Now each $X_{\alpha}$ is already its own basis, hence so is $\prod{X_{\alpha}}$. Now let
    $\prod{U_{\alpha}}$ and $\prod{V_{\alpha}}$ be basis elements. Since $\prod{U_{\alpha}} \cap 
    \prod{V_{\alpha}}=\prod{U_{\alpha} \cap V_{\alpha}}$, for finite alpha, and since 
    $\prod{U_{\alpha}} \cap \prod{V_{\alpha}}=\prod{X_{\alpha}}$ for all other $\alpha$  (in the
    case of the product space topology), we get another basis element. Hence $\Bc$ is a basis for
    the box topology, and, provided the necessary condidition, is also a basis for the product
    topology.
\end{proof}

\begin{theorem}\label{2.1.5}
    Let $A_{\alpha}$ be a subspace of $X_{\alpha}$. Then $\prod{A_{\alpha}}$ is a subspace of
    $\prod{X_{\alpha}}$ under both the box and product space topologies.
\end{theorem}
\begin{proof}
   Since $\prod{A_{\alpha}} \cap \prod{U_{\alpha}}=\prod{A_\alpha} \cap U_{\alpha}$, and $A_{\alpha}
   \cap U_{\alpha}$ is a basis element for $X_ {\alpha}$ under the subspace topology, then it follows that
   $\prod{A_{\alpha} \cap U_{\alpha}}$ is a basis element for the same topology on
   $\prod{X_{\alpha}}$, thus $\prod{A_\alpha}$ is a subspace.
\end{proof}

\begin{theorem}\label{2.1.6}
    If $X_{\alpha}$ is a Hausdorff space, then so is $\prod{X_{\alpha}}$ under both the box and
    product space topologies.
\end{theorem}
\begin{proof}
Since $X_{\alpha}$ is a Hausdorff space, a sequence of points of $X_{\alpha}$, $\{x_{\alpha_n}\}$
converges to atmost one point. Now construct a sequence $\{x_n\}$ where $x_i=x_{\alpha_i}$ and
$x_{\alpha_i}$ is the $i$-th term of  $(x_\alpha)$, we see that $\{x_{\alpha_n}\}$ is a subsequence
of $\{x_n\}$, by deifinition, and hence $\{x_n\}$ must also converge at atmost one point.
\end{proof}

\begin{example}
    For Euclidean space $\R^n$, a basis consists of all products of the form  $(a_1,b_1) \times
    \dots \times (a_n, b_n)$ where $(a_i,b_i)$ is an open interval for all $1 \leq i  \leq n$. Since
    $\R^n$ is a finite product space, both the box and product topologies on  $\R^n$ are the same.
\end{example} 

\begin{theorem}\label{2.1.7}
    If $A_{\alpha} \subseteq X_{\alpha}$, then $\prod{\cl{A_{\alpha}}}=\cl{\prod{A_{\alpha}}}$
\end{theorem}
\begin{proof}
    Let $x=(x_{\alpha}) \in \prod{\cl{A_{\alpha}}}$ and let $U=\prod{U_{\alpha}}$ be a basis
    element (for either topology) Choosing $y_{\alpha} \in U_{\alpha} \cap A_{\alpha}$, for each
    $\alpha$, let  $y=(y_{\alpha})$. Then $y \in U$, and $y \in \prod{A_{\alpha}}$, hence $x \in
    \cl{\prod{A_{\alpha}}}$.

    Now suppose that $x \in \cr{\prod{A_{\alpha}}}$ (in either topology). Let $V_{\beta}$ be an
    open set of $X_{\beta}$ containing $x_{\beta}$. Since $\pi_{\beta}^{-1}(V_{\beta})$ is open in
    $\prod{X_{\alpha}}$ (in either topology), it containts a point $y=(y_{\alpha})$ of
    $\prod{A_{\alpha}}$. Then $y_{\beta} \in V_{\beta}$ \cap A_{\beta}, hence $x \in
    \cl{A_{\beta}}$.
\end{proof}

\begin{theorem}\label{2.1.8}
    Let $f:A \rightarrow \prod{X_{\alpha}}$ be defined by $f(a)=(f_{\alpha}(a))$, where
    $f_{\alpha}:A \rightarrow X_{\alpha}$. Letting $\prod{X_{\alpha}}$ have the product space
    topoplogy, $f$ is continous if and only if  $f_{\alpha}$ is continous for each $\alpha$.
\end{theorem}
\begin{proof}
    We know that the projection mapping $\pi_{\beta}$ is continous. Now suppose that $f$ is
    continuous, and notice that  $f_{\beta}=\pi_{\beta} \circ f$, which makes $f_{\beta}$ continuous
    for each $\beta$.

    On the other hand, suppose that  $f_{\beta}$ is continuous for each $\beta$. Notice that
    $f_{\beta}^{-1}=f^{-1} \circ \pi_{\beta}^{-1}$, since $\pi_{\beta}^{-1}(U_{\beta})$ is open in
    $\prod{X_{\alpha}}$, then so is $f^{-1} \circ
    \pi_{\beta}^{-1}(U_{\beta})=f_{\beta}^{-1}(U_\beta)$. This makes $f$ continuous.
\end{proof}

\begin{example}
    Theorem \ref{2.1.8} holds only for the product space topology and fails in general for the box
    topology. Consider $\R^{\omega}$ and define the map $f:\R \rightarrow \R^{\omega}$ by
    $f(t)=(t,t,t, \dots)$. We have that $f_n(t)=t$ is continuous, which makes $f$ continuous under
    the product topology. Now consider the box topology: let  $B=(-1,1) \times
    (-\frac{1}{2},\frac{1}{2}) \times (-\frac{1}{3}, \frac{1}{3}) \times \dots$, and supppose that
    $f^{-1}(B)$ were open. Then it contains some interval $(-\delta,\delta)$, about $0$, thus
    $\pi_{\beta} \circ f((-\delta,\delta))=f_{\beta}((-\delta,\delta))=(-\delta,\delta) \subseteq
    (-\frac{1}{n}, \frac{1}{n})$, which is absurd. Thus the only implication of the theorem that
    holds for the box topology is that $f_{\alpha}$ is continuous only when $f$ is continuou is
    continuous.
\end{example} 
