%----------------------------------------------------------------------------------------
%	SECTION 1.1
%----------------------------------------------------------------------------------------

\section{More on Metric Spaces}

We go more in depth on metric spaces here.

\begin{theorem}\label{2.3.1}
    If $A$ is a subspace of a metric space $X$, with metric  $d$, then  $d$ restricted to  $A \times
    A$ makes  $A$ into a metric space.
\end{theorem}
\begin{proof}
    Clearly $d:A \times A \rightarrow \R$ is a metric. So consider the $\epsilon$-ball about $x$,
    $B_d(x,\epsilon)$ in $X$; restricting  $d$ to  $A \times A$, consider  $A \cup B_d(x,\epsilon)$.
    For $y \in A$, there is a  $\delta$-ball about  $y$ such that  $B_d(y,\delta)
    \subseteq B_d(x,\epsilon)$; then $B_d(y,\delta) \subseteq B_d(x,\epsilon)$. This makes $A$ as a subspace,
    into a metric space.
\end{proof}

\begin{theorem}\label{2.3.2}
    The Hausdorff axiom is satisfied in every metric space.
\end{theorem}
\begin{proof}
    If $x,y \in X$ are distinct, let  $\epsilon=\frac{1}{2}d(x,y)$, by the triangle inequality, we
    have that $B_d(x,\epsilon)$ and $B_d(y,\epsilon)$ are disjoint.
\end{proof}

\begin{theorem}\label{2.3.3}
    Countable products of metrizable spaces are metrizable.
\end{theorem}
\begin{proof}
    Let $X$ be a metric space with metric  $d$. Define
    $\bar{d}(x,y)=\min\{d(x,y),1\}$ on $X$ and define  $D(x,y)=\sup\{\frac{\bar{d}(x_i,y_i)}{i}\}$
    on $X^{\omega}$. It is clear that both $\bar{d}$ and $D$ are metrics on  $X$ and  $X^{\omega}$
    respectively. We would like to show that $D$ induces the product topology on  $X^{\omega}$.

    Let $U$ be open and let  $x \in U$. Choose  $B_D(x,\epsilon) \subseteq U$ and choose $N$ large
    enough such that  $ \frac{1}{N}<\epsilon$. Now let
    $V=(x_1-\epsilon,x_1+\epsilon) \times \dots \times (x_N-\epsilon,x_N+\epsilon) \times X \times \dots$ be a basis in the product topology on
    $X^{\omega}$. Given $y \in X^{\omega}$, such that $\frac{\bar{d}(x_i,y_i)}{i} \leq
    \frac{1}{N}$, we have by definition that $D(x,y) \leq \max\{\bar{d}(x_1,y_1),
    \frac{\bar{d}(x_2,y_2)}{2}, \dots, \frac{\bar{d}(x_N,y_N)}{N}, \frac{1}{N}\}$. If $y \in V$, we
    get that  $V \subseteq B_D(x,\epsilon)$ and we are done.

    Conversely let $U=\Prod{U_i}$ be a basis of the product topology where $U_i$ is open in  $X$
    for  $i=1,\dots,n$ and  $U_i=X$ for all other indices. Now let  $x \in U$ and choose an interval
    about  $x_i$,  $(x_i-\epsilon_i,x_i+\epsilon_i)$ lying in $U_i$ with  $0<\epsilon_i \leq 1$ for
    all  $i$. Choose  $\epsilon=\min\{\epsilon_1,\frac{\epsilon_2}{2}, \dots,
    \frac{\epsilon_n}{n}\}$. Then  $x \in B_D(x,\epsilon) \subseteq U$, for if $y \in
    B_D(x,\epsilon)$, we have that $\frac{\bar{d}(x_i,y_i)}{i} \leq D(x,y)<\epsilon$, hence
    $\epsilon \leq \frac{\epsilon_i}{i}$ and $d(x_i,y_i)<\epsilon_i$, and so $y \in \Prod{U_i}$.
    Therefore $D$ induces the product space topology
\end{proof}
\begin{remark}
    This theorem generalizes theorem \ref{2.2.9} for any countable product space of a metric space
    $X$. Hence we can take theorem \ref {2.2.9} as a corollary to this theorem.
\end{remark}

We would now like to study continuous functions in metric spaces, which brings us into the realm of
analysis. We show that the ``$\epsilon$-$\delta$'' definition, and the sequence definiton of
continuity carry over.

\begin{theorem}\label{2.3.4}
    Let $f:X \rightarrow Y$ with  $X$ and  $Y$ metric spaces with metric  $d_X$ and  $d_Y$
    respectively. Then  $f$ is continuous if and only if for  $x \in X$,a nd  $\epsilon>0$, there is
    a  $\delta>0$ such that  $d_Y(f(x),f(y))<\epsilon$ whenever $d_X(x,y)<\delta$.
\end{theorem}
\begin{proof}
    Suppose that $f$ is continous and consider  $f^{-1}(B(f(x),\epsilon))$ open in $X$. Then it
    contains a $\delta$-ball $B(x,\delta)$ about $x$. If  $y \in B(x,\delta)$, then $f(y) \in
    B(f(x),\epsilon)$, as is required.

    Now suppose that for $x \in X$ and  $\epsilon>0$, that there is a  $\delta>0$ such that
    $d_Y(f(x),f(y))<\epsilon$ whenever $d_X(x,y)<\delta$, for $x\in X$. Let $V$
    be open in  $Y$ and
    let  $x \in f^{-1}(V)$. Then $f(x) \in V$, hence there is an $\epsilon$-ball  $B(f(x),\epsilon)
    \subseteq V$. By hypothesis, there is a  $\delta>0$ such that $f(B(x,\delta)) \subseteq
    B(f(x),\epsilon)$, hence $B(x,\delta) \subseteq f^{-1}(V)$ which makes $f^{-1}(V)$ open.
\end{proof}

\begin{lemma}[The Sequence Lemma]\label{2.3.5}
    Let $X$ be a topological space and let  $A \subseteq X$. If there is a sequence of points of
    $A$ converging to $x \in X$, then $x \in \cl{A}$. The converse holds if $X$ is metrizable.
\end{lemma}
\begin{proof}
    Suppose for some sequence $\{x_n\} \subseteq A$ that $x_n \rightarrow x$. By theorem \ref
    {1.6.6}, we have every neighborhood of $x$ contains points of  $A$,  hence $x \in \cl{A}$.
    Conversely, suppose that $X$ is metrizable with metric  $d$, and let  $x \in \cl{A}$. For $n
    \in \Z^+$, take  $B_d(x,\frac{1}{n})$ and take $\{x_n\}=B_d(x,\frac{1}{n}) \cap A$. Then $x_n
    \rightarrow n$, for: any open set  $U$ of  $x$ contains an  $\epsilon$-ball about  $x$,
    $B_d(x,\epsilon)$, so choose $N$ large enough so that  $ \frac{1}{N}<\epsilon$, hence $U$
    contains  $x_i$ for all  $i \geq N$.
\end{proof}

\begin{theorem}[The Sequential Criterion]\label{2.3.6}
    Let $f:X \rightarrow Y$ be continuous, then for every convergent sequence  $\{x_n\}$ converging
    to $x \in X$, the sequence  $\{f(x_n)\}$ converges to $f(x)$. the converse holds if $X$ is
    metrizable.
\end{theorem}
\begin{proof}
    Let $f$ be continuous and suppose that  $x_n \rightarrow x$. Let  $V$ be a neighborhood of
    $f(x)$, then $f^{-1}(V)$ is a neighborhood of $x$; hence there is an  $N>0$ such that  $x_n \in
    f^{-1}(V)$ whenever $n \geq N$, thus  $f(x_n) \in V$ whenever $n \geq N$.

    Conversely suppose that for every  $\{x_n\}$ converging to $x$, that  $\{f(x_n)\}$ converges to
    $f(x)$, and let $A \subseteq X$. if  $x \in \cl{A}$, by the sequence lemma, there is a sequence
    $\{x_n\} \subseteq A$ converging to $X$. Now since  $f(x_n) \rightarrow f(x)$. and $f(x_n) \in
    f(A)$, by the sequence lemma again, $f(x) \in \cl{f(A)}$.  Thus $f(\cl{A}) \subseteq
    \cl{f(A)}$ and we are done.
\end{proof}

We now consder methods for constructing continous functions on metric spaces.

\begin{lemma}\label{2.3.7}
    The additions, subtraction, and multiplication operations are continous from $\R \times \R$ to
    $\R$. The quotient operation is continous from  $\R \times \com{\R}{\{0\}}$ to $\R$.
\end{lemma}

\begin{theorem}\label{2.3.8}
    If $X$ is a topological space and if  $f,g:X \rightarrow \R$ are continous, then  $f+g$,  $f-g$,
    a nd $fg$ are continuous; moreover if  $g(x) \neq 0$ for all $x \in X$, then $\frac{f}{g}$ is
    also continuous.
\end{theorem}
\begin{proof}
    The map $h:X \rightarrow \R \times \R$ defined by $h(x)=f(x) \times g(x)$ is continuous. Then
    notice that $f+g(x)=+(f(x),g(x))=+\circ h(x)$, so by the above lemma, we get that $f+g$ is
    contincontinuous. We also have that  $f-g$ is continuous for  $f-g(x)=+(f(x),-g(x))$. The same
    argument holds for $fg$ and  $ \frac{f}{g}$.
\end{proof}

\begin{definition}
    Let $f_n:X \rightarrow Y$ be a sequence of functions from $X$ to the metric space  $Y$, with
    metric  $d$. We say that the sequence  $\{f_n\}$ \textbf{converges uniformly} to the function
    $f:X \rightarrow Y$ if for  $\epsilon>0$, there is an integer $N>0$ such that
    $d(f_n(x),f(x))<\epsilon$ whenever $n \geq N$, for all  $x \in X$.
\end{definition}

\begin{theorem}\label{2.3.9}
    Let $f_n:X \rightarrow Y$ be a sequence of continuous functions from the topological space  $X$
    to the metric space  $Y$. If  $\{f_n\}$ converges uniformly to $f$, then  $f$ is continuous.
\end{theorem}
\begin{proof}
    Let $V$ be open in  $Y$ and let  $ x_0 \in f^{-1}(V)$. Let $ y_0=f(x_0)$ and choose $\epsilon>0$
    such that  $B(y_0,\epsilon) \subseteq V$. By uniform convergence, choosing $N>0$ so that
    whenever  $n \geq N$,  $d(f_n(x),f(x))<\frac{\epsilon}{3}$ for all $x \in X$. By the continuity
    of $f_N$, choose a neighborhood  $U$ of  $ x_0$ such that $f_N(U) \subseteq
    B(f_N(x_0),\frac{\epsilon}{3})$. Then if $x \in U$, we have
    $d(f(x),f_N(x))<\frac{\epsilon}{3}$, $d(f_N(x),f_N(x_0))<\frac{\epsilon}{3}$, and
    $d(f_N(x_0),f(x_0))<\frac{\epsilon}{3}$ by the triangle inequality we get
    $d(f(x),f(x_0))<\epsilon$ which completes the proof.
\end{proof}

\begin{example}
    \begin{enumerate}
        \item[(1)] $\R^{\omega}$ is not metrizable in the box topology. Let $A=\{(x_1,x_2,\dots) \in
            \R^{\omega}:x_i>0\}$ and consider $0=(0,0,\dots) \in \R^{\omega}$. $0 \in \cl{A}$ if
            for any basis element $B=(a_1,b_1) \times (a_2,b_2) \times \dots$, $0 \in B$; then  $B
            \cap A \neq \emptyset$  (take the point $ \frac{1}{2}b \in \R^\omega$). Now let
            $\{a_n\}$ be a sequence of points of $A$ with  $a_n=(x_{1n},x_{2n}, \dots, x_ {in},
            \dots)$, since $x_{in}>0$, construct a basis element $B'=(-x_{11},x_{11}) \times
            (-x_{22},x_{22}) \times \dots$. Then $0 \in B'$, but  $\{a_n\} \not\subseteq B'$ for the
            point $x_{nn} \notin (-x_{nn},x_{nn})$. Thus $a_n \not\rightarrow 0$.

        \item[(2)] An uncountable product of  $\R$ with itself is not metrizable. Let  $J$ be
            uncountable, and let  $A=\{(x_{\alpha}) \in \R^J:x_{\alpha}=1 \text{ for all but
            finitely many } \alpha\}$. Consider $0 \in \R^J$ and let  $\Prod{U_{\apha}}$ be a basis
            for containing $0$. Now  $U_{\alpha} \neq \R$ for $\alpha_1, \dots, \alpha_n$, so let
            $(x_{\alpha}) \in A$ be defined by letting $x_{\alpha}=0$ for $\alpha_1, \dots, \alpha_n$
            and $x_{\alpha=1}$ for all other indices. Then $(x_{\alpha}) \in A \cap
            \Pord{U_{\alpha}}$.

            Nowe let $\{a_n\} \subseteq A$ and for $n \in \Z^+$ elt  $J_n=\{\alpha \in J:
            \alpha_{\alpha n} \neq 1\}$. Then we see that $\bigcup{J_n}$ is a countable union of
            finite sets, and hence countable itself. Now since $J$ is uncountable, there is a
            $\beta \in J$ for which  $\beta \notin \bigcup{J_n}$, so $a_{\beta n} \neq 1$. Lettng
            $U_{\beta}=(-1,1)$ in $\R$ let  $U=\pi_{\beta}^{-1}(U_{\beta})$ in $\R^J$. Then  $U$ is
            a neighborhood of  $0$ not containing  any points of  $\{a_n\}$, so $a_n \not\rightarrow
            0$.
    \end{enumerate}
\end{example}
