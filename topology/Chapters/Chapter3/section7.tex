\section{Limit Point Compactnes}

\begin{definition}
    We say a topological space $X$ is \textbf{limit point compact} if every
    infinite subset of $X$ has a limit point.
\end{definition}

\begin{theorem}\label{3.7.1}
    A compact topological space is limit point compact.
\end{theorem}
\begin{proof}
    Let $X$ be a compact topological space, and let  $A \subseteq X$ be an
    infinite subset of  $X$. Supposing that  $A$ has no limit point then  $A'
    \subseteq A$, making  $A$ closed. Now let  $a \in A$ and  $U_a$ a
    neighborhood of  $a$ such that $U_a \cap A=\{a\}$. Now both $\com{X}{A}$ and
    $U_a$ cover  $X$ for each $a$, and since  $X$ is compact, $X$ can be covered
    by finitely many sets of the collection  $(\com{X}{A}) \cup \{U_a\}_{a \in
    A}$. Now, $(\com{X}{A}) \cap A=\emptyset$ and only one $a$ is in each $U_a$.
    By finiteness of the covering, $A$ is finite, which contradicts the
    assumption that it is infinite. Therefore $A$ must have a limit point,
    therefore  $X$ must be limit point compact.
\end{proof}

\begin{example}\label{3.11}
    \begin{enumerate}
        \item[(1)] Let $Y$ be a topological space under the indescrete topology
             $\Tc=\{\emptyset, Y\}$, with $|Y|=2$. Let  $X=Y \times \Z^+$. Then
              $X$ is limit point compact since the infinite subsets  $y_1 \times
              \Z^+$ and  $y_2 \times \Z^+$ have limit points. However, $X$ is
              not compact, since the open cover  $\{Y \times \{n\}\}_{n \in \Z^+}$
              has no finite open subcover.

          \item[(2)] Let $S_\omega$ be the minimal uncountable well ordered set
              under the order topology.  $S_\omega$ is not compact as it has no
              greatest element. Now, let  $A \subseteq S_\omega$ be an infinite
              subset, and let  $B \subseteq A$ be countably finite. Then  $B$
              has an upperbound  $b \ion S_\omega$, so that  $B \subseteq
              [a_0,b]$, where $a_0$ is the least element of $S_\omega$. Since
              $S_\omega$ is well ordered, it has the least upperbound property,
              so  $[a_0,b]$ is compact by theorem \ref{3.6.1} Thus $B$ has a
              limit point, making  $A$ have a limit point. Therefore by theorem
               \ref{3.7.1}, $S_\omega$ is limit point compact.
    \end{enumerate}
\end{example}

\begin{definition}
    Let $X$ be a topological space, and  $\{x_n\}$ a sequence of points of $X$.
    If  $n_1<n_2<\dots$ is an increasing sequence of nonnegative integers, then
    we call the sequence $\{y_m\}$ defined by $y_i=x_{n_i}$ a
    \textbf{subsequence} of $\{x_n\}$.
\end{definition}

\begin{definition}
    We call a topological space \textbf{sequentially compact} if every sequence
    of points in the space has a convergent subsequence.
\end{definition}

\begin{theorem}\label{3.7.2}
    For any metrizable space $X$, the following are equivalent:
    \begin{enumerate}
        \item[(1)] $X$ is compact.

        \item[(2)] $X$ is limit point compact.

        \item[(3)] $X$ is sequentially compact.
    \end{enumerate}
\end{theorem}
\begin{proof}
    Let $X$ be a metrizable space. By theorem \ref{3.7.1} we have that (1)
    implies (2).

    Now, suppose that $X$ is limit point compact; that is, every infinite subset
    of  $X$ has a limit point. Consider the sequence  $\{x_n\}_{n \in \Z^+}$. If
    $\{x_n\}$ is finite, then there is a point $x \in X$ for which  $x=x_n$  for
    infinitely many $n$. Then the sequence  $\{x\}$ is a convergent subsequence
    of $\{x_n\}$ by this fact, making $X$ sequentially compact.

    Now, if  $\{x_n\}$ is an infinite sequence, then $\{x_n\}$ has a limit point
    $x \in X$ by hypothesis. Define, then, the sequence $\{y_m\}$ such that
    \begin{equation*}
        y_i=x_{n_i}=B(x,\frac{1}{i})
    \end{equation*}
    which works since every open ball interesect $\{x_n\}$ at infinitely many
    points. Then as $i \rightarrow \infty$, $\frac{1}{i} \rightarrow 0$, so that
    $\{y_m\} \rightarrow x$, making $X$ sequentially compact.

    Now, suppose that  $X$ is sequentially compact. Then every sequence
    $\{x_n\}$ of points of $X$ has a convergent subsequence. Let  $\Ac$ be an
    open cover for  $X$ and suppose that  $\Ac$ has no Lebesgue number; i.e.
    there is no  $\delta>0$ for which any set with $\diam<\delta$ has an element
    of  $\Ac$ containing it. Then for each  $n \in \Z^+$, there are sets with
    $\diam<\frac{1}{n}$, not contained in any member of $\Ac$. Denote these sets
    $C_n$ and choose $x_n \in C_n$. Then for the sequence  $\{x_n\}_{x_n \in
    C_n}$, there is a subsequence $\{y_m\}$ with $y_i=x_{n_i}$ converging to a
    point $a \in A$ for some $A \in \Ac$. Now, choose $\epsilon>0$ such that
    $B(a,\epsilon) \subseteq A$, if $i$ is sufficiently large such that
    $\frac{1}{n_i}<\frac{\epsilon}{2}$. Then take $C_{n_i} \subseteq
    B(x_{n_i},\frac{\epsilon}{2})$, if $i$ is also chosen such that
    $d(x_{n_i},a)<\frac{\epsilon}{2}$, then $C_{n_i} \subseteq B(a,\epsilon)$
    which makes $C_{n_i} \subseteq A$, a contradiction on the definition of the
    $C_n$. This forces $\Ac$ to satisfy the Lebesgue number lemma.

    Now, suppose there exists an $\epsilon>0$ such that $X$ cannot be covered by
    finitely many  $\epsilon$-balls. Define the sequence of points of $X$,
    $\{x_n\}$ by the following steps:
    \begin{enumerate}
        \item[Step 1:] Choose any $x_1 \in X$.

        \item[Step 2:] Choose $x_2 \in X$ with $x_2 \notin B(x_1,\epsilon)$.

        \item[Step 3:] Proceed inductively and choose $x_n \in X$ with  $x_n
            \notin \bigcup{B(x_i,\epsilon)}$
    \end{enumerate}

    Since the collection $\{B(x_i,\epsilon)\}_{i=1}^n$ does not cover $X$, and
    $d(x_{n+1},x_i) \geq \epsilon$, then the sequence $\{x_n\}$ has no
    convergent subsequence. This contradicts that $X$ is sequentially compact.
    So there must be a finite collection of $\epsilon$-balls covering  $X$.

    Finally, by the above, we have that  $\Ac$ has a Lebesgue number $\delta$.
    Let  $\epsilon=\frac{\delta}{2}$, then by above, we can find a finite
    collection of $\epsilon$-balls, $\{B(x_i,\epsilon)\}$, covering $X$ where
    $\diam{B(x_i,\epsilon)} \leq \frac{2}{3}\delta$. So $B(x_i,\epsilon)
    \subseteq A$ for some $A \in \Ac$, for each $x_i$, and so
    $\{B(x_i,\epsilon)\}$ forms a finite subcover of $X$. Therefore $X$ is
    compact.
\end{proof}
