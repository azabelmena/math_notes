%----------------------------------------------------------------------------------------
%	SECTION 1.1
%----------------------------------------------------------------------------------------

\section{Quotient Spaces}

\begin{definition}
    Let $X$ be a topological space, and  $X'=\{X_\alpha\}$ a partion of $X$. We
    define the \textbf{natrual map} $q:X \xrightarrow{} X'$ by taking $x
    \xrightarrow{} X_\alpha$ where $x \in X_\alpha$. We define the  \textbf{quotient
    topology} on $X'$ to be the family:
    \begin{equation*}
        \Tc = \{U' \subseteq X' : \inv{q}(U') \text{ is open in } X\}
    \end{equation*}
    We denote quotient spaces by $\faktor{X}{q}$, $\faktor{X}{X'}$, or
    $\faktor{X}{\sim}$ where $\sim$ is an equivalence relation partitioning  $X$
    into  $X'$.
\end{definition}

\begin{example}\label{2.2}
    \begin{enumerate}
        \item[(1)] Consider the space $I=[0,1]$ and let $A=\{0,1\}$. The the quotient
            space $\faktor{I}{A}$ identifies $0$ to  $1$, and hence, under the
            quotient topology, is homeomorphic to  $S^1$.

        \item[(2)] Consider the space $I \times I$ and define an equivalence
            relation $(x,0) \sim (x,1)$ for all $x \in I$. Then the quotient
            topology formed on  $\faktor{I \times I}{\sim}$ is homeomorphic
            to the cylinder  $S^1 \times I$. Defining another equivalence
            $(0,y) \sim (1,y)$ for all $y \in I$, we get the quotient space on
            $\faktor{S^1 \times I}{\sim}$ under this equivalence relations is
            homeomorphic to the torus  $S^1 \times S^1$.

        \item[(3)] Let $h:X \xrightarrow{} Y$ be a map, and define $\ker{h}$ the
            equivalence relation on $X$ such that  $x \ker{h} x'$ if, and only
            if $h(x)=h(x')$. The quotient space $\faktor{X}{\ker{h}}$ has the
            following relation to the natural map on $X$ via the commutative
            diagram
            \[\begin{tikzcd}
                X &&& Y \\
                \\
                \\
                {\faktor{X}{\ker{h}}}
                \arrow["h", from=1-1, to=1-4]
                \arrow["q"', from=1-1, to=4-1]
                \arrow["\phi"', from=4-1, to=1-4]
            \end{tikzcd}\]
            Where $\phi:\faktor{X}{\ker{h}} \xrightarrow{} Y$ is a 1--1 map
            defined by $\phi([x])=h(x)$.
    \end{enumerate}
\end{example}

\begin{definition}
    A continuous map $f:X \xrightarrow{} Y$ of a topological space $X$ onto a
    topological space  $Y$ is call an  \textbf{identification} if a subset $U$
    of  $Y$ is open if, and only if  $\inv{f}(U)$ is open in $X$. We denote the
    quotient space on  $X$ induced by $f$ by $\faktor{X}{f}$.
\end{definition}

\begin{example}\label{2.3}
    \begin{enumerate}
        \item[(1)] The natural map $q:X \xrightarrow{} \faktor{X}{\sim}$ is an
            identification, where $\sim$ is an equivalence relation on $X$
            inducing the quotient topology.

        \item[(2)] If $f:X \xrightarrow{} Y$ takes spaces $X$ onto $Y$, is open
            or closed, then  $f$ is an identification.

        \item[(3)] If $f:X \xrightarrow{} Y$ is a continuous map such that there
            exists a map $s:Y \xrightarrow{} X$ such that $f \circ s=1_Y$, then
             $f$ is an identification. We call the map  $s$ a  \textbf{section}
             of $f$.
    \end{enumerate}
\end{example}

\begin{theorem}\label{2.2.1}
    Let $f:X \xrightarrow{} Y$ be a continuous map of a topological space $X$
    onto a topological space  $Y$.  $f$ is an identification if, and only if for
    any toopological space $Z$, and all maps $g:Y \xrightarrow{} Z$, then $g$ is
    continuous if, and only if  $g \circ f$ is continuous.
    \[\begin{tikzcd}
            X &&& Z \\
            \\
            \\
            Y
            \arrow["{g \circ f}", from=1-1, to=1-4]
            \arrow["f"', from=1-1, to=4-1]
            \arrow["g"', from=4-1, to=1-4]
        \end{tikzcd}\]
\end{theorem}
\begin{proof}
    Suppos that $f$ is an identification. If  $g$ is continuous, then so is  $g
    \circ f$, by continuity of  $f$. On the other hand, if  $g \circ f$ is
    continuous, letting  $V$ be open in $Z$ we have $\inv{(g \circ
    f)}(V)=\inv{f}(\inv{g}(V))$ which is open in $X$. By hypothesis,
    $\inv{g}(V)$ is open in $Y$, which makes  $g$ continuous.

    Now, suppose that  $g$ is continuous if, and only if  $g \circ f$ is
    continuous. Let  $Z=\faktor{X}{\ker{f}}$, and $q:X \xrightarrow{}
    \faktor{X}{\ker{f}}$ the natural map. Additionally, define the 1--1 map
    $\phi:\faktor{X}{\ker{f}} \xrightarrow{} Y$ by $\phi([x])=f(x)$. Since $f$
    is onto, we get that so is  $\phi$. Consider the following commutative
    diagram:
        \[\begin{tikzcd}
            X &&& Z \\
            \\
            \\
            Y
            \arrow["q", from=1-1, to=1-4]
            \arrow["f"', from=1-1, to=4-1]
            \arrow["{\phi^{-1}}"', from=4-1, to=1-4]
        \end{tikzcd}\]
    Then $\inv{\phi} \circ f=q$ is continuous which implies that $\inv{\phi}$ is
    continuous. $\phi$ is also continuou since $q$ is an identification.
    Therefore  $\phi$ is a homeomorphism between $Y$ and $Z$. Notice now, that
    $f=\phi \circ q$. Then since $q$ and $\phi$ are continuous, this makes $f$
    continuous by composition. Moreover, $\inv{f}(U)=\inv{q}(\inv{\phi}(U))$.
    Since $q$ is an identification,  $\inv{\phi}(U)$ is open in $Z$, which makes
     $\inv{f}(U)$ open in $X$. This makes  $f$ an identification.
\end{proof}f
\begin{corollary}
    Let $f:X \xrightarrow{} Y$ be an identification, and for some space $Z$,
    define  $h:X \xrightarrow{} Z$ to be the continuous map constant on each
    fiber of $f$. Then $h \circ \inv{f}:Y \xrightarrow{} Z$ is continuous.
    \[\begin{tikzcd}
        X &&& Z \\
        \\
        \\
        Y
        \arrow["h", from=1-1, to=1-4]
        \arrow["f"', from=1-1, to=4-1]
        \arrow["{h \circ f^{-1}}"', from=4-1, to=1-4]
    \end{tikzcd}\]
    Moreover $h \circ \inv{f}$ is open or closed if, and only if $h(U)$ is open
    or closed in $Z$ whenever $U=\inv{f}(f(U))$ is open or closed in $X$.
\end{corollary}
\begin{corollary}
    If $h:X \xrightarrow{} Z$ is an identification, then the map
    $\phi:\faktor{X}{\ker{h}} \xrightarrow{} Z$ defined by $[x] \xrightarrow{}
    h(x)$ is a homeomorphism.
\end{corollary}
