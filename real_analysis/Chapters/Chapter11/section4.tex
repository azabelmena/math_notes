\section{The Fundamental Theorem of Calculus for Lebesgue Integrals}

\begin{theorem}[The Fundamental Theorem of Calculus for the Lebesgue
    Integral]\label{11.4.1}
    Let $f$ be absolutely continuous on the closed bounded interval  $[a,b]$.
    Then $f$ is differentiable almost everywhere on  $(a,b)$, and $D{f}$ is
    Lebesgue integrable on $[a,b]$, moreover
    \begin{equation*}
        \int_a^b{D{f(x)} \ dm}=f(b)-f(a)
    \end{equation*}
\end{theorem}
\begin{proof}
    Since
    \begin{equation*}
        \int_a^b{\Diff_h{f(x)} \ dm}=\Av_h{f(b)}-\Av_h{f(a)}
    \end{equation*}
    notice that
    \begin{equation*}
        \lim_{n \xrightarrow{} \infty}{\int_a^b{\Diff_{\frac{1}{n}}{f(x)} \ dm}}=
        f(b)-f(a)
    \end{equation*}
    Moreover, $f$ is the difference of increasing functions defined on  $[a,b]$,
    hence, by Lebesgue's theorem (theorem \ref{11.1.3}), $f$ is differentiable
    almost everywhere on  $(a,b)$. Therefore, $\{\Diff_{\frac{1}{n}}{f}\}
    \xrightarrow{} D{f}$ almost everywhere on $[a,b]$. Additionall,y we have
    that $\{\Diff_{\frac{1}{n}}{f}\}$ is uniformly integrable on $[a,b]$, so by
     Vital'is convergence theorem
     \begin{equation*}
        \lim_{n \xrightarrow{} \infty}{\int_a^b{\Diff_{\frac{1}{n}}{f(x)} \ dm}}=
        \int_a^b{D{f(x)} \ dm}
     \end{equation*}
     by passage of the limit under the integral.
\end{proof}

\begin{definition}
    We call a realvalued function $f$ on the closed bounded interval $[a,b]$ the
    \textbf{indefinite Lebesgue integral} of a Lebesgue integrable function $g$
    on $[a,b]$, provided
    \begin{equation*}
        f(x)=f(a)+\int_a^x{g(t) \ dm} \text{ for all } x \in [a,b]
    \end{equation*}
\end{definition}

\begin{theorem}\label{11.4.2}
    A function on a closed bounded interval $[a,b]$ is absolutely continuous if,
    and only if it is an indefinite Lebesgue integral on $[a,b]$.
\end{theorem}
\begin{proof}
    Suppose that $f$ is absolutely continuous on  $[a,b]$. Then for all $x \in
    [a,b]$, $f$ is also absolutely continuous on the closed bounded subinterval
     $[a,x]$. Hence, by the fundamental theorem of calculus for the Lebesgue
     integral,
     \begin{equation*}
         f(x)=f(a)+\int_a^x{D{f(t)} \ dm}
     \end{equation*}
     which makes $f$ an indefinite Lebesgue integral (in particular, for $D{f}$).

     Conversely, suppose that $f$ is the indefinite Lebesgue integral for some
     Lebesgue integrable function $g$ on $[a,b]$. Let $\{(a_k,b_k)\}_{k=1}^n$ be
     a finite disjoint collection of open subintervals of $(a,b)$, and take
     \begin{equation*}
         E=\bigcup_{k=1}^n{(a_k,b_k)}
     \end{equation*}
     Then by monotonicity, and additivity over domains,
     \begin{equation*}
         \sum_{k=1}^n{|f(b_k)-f(a_k)|}=
         \sum_{k=1}^n{\Big{|} \int_{a_k}^{b_k}{g(t) \ dm} \Big{|}} \leq
         \sum_{k=1}^n{\int_{a_k}^{b_k}{|g(t)| \ dm}}=\int_E{|g(t)| \ dm}
     \end{equation*}
     Now, let $\e>0$, since  $|g|$ is integrable on  $[a,b]$, there exists a
     $\d>0$ such that
     \begin{equation*}
         \int_E{|g(t)| \ dm}<\e \text{ if } E \subseteq [a,b] \text{ is
         measurable with } m(E)<\d
     \end{equation*}
     Then the above equation gives us the required criterion for $\d$ to respond
     to  $\e$.
\end{proof}
\begin{corollary}
    Let $f$ be monotone on the closed bounded interval  $[a,b]$. Then $f$ is
    absolutely continuous if, and only if
    \begin{equation*}
        \int_a^b{D{f(x)} \ dm}=f(b)-f(a)
    \end{equation*}
\end{corollary}
\begin{proof}
    Suppose that $f$ is absolutely continuous, then the fundamental theorem of
    calculus for Lebesgue integrals gives us the resulty, regardless of
    monotonicity. Conversely, suppose that $f$ is monotone increasing, and that
    \begin{equation*}
        \int_a^b{D{f(x)} \ dm}=f(b)-f(a)
    \end{equation*}
    Then let $x \in [a,b]$, we have
    \begin{equation*}
        \int_a^b{D{f(x)} \ dm}-(f(b)-f(a))=
        \Big{(} \int_a^x{D{f(t)} \ dm}-(f(x)-f(a)) \Big{)}+
        \Big{(} \int_x^b{D{f(t)} \ dm}-(f(b)-f(x)) \Big{)}=0
    \end{equation*}
    Now, both of the summands $\int_a^x{D{f(t)} \ dm}-(f(x)-f(a)) \leq 0$ and
    $\int_x^b{D{f(t)} \ dm}-(f(b)-f(x)) \leq 0$, and since they sum $0$, they
    must both be equal to  $0$; that is
    \begin{equation*}
        f(x)=f(a)+\int_a^x{D{f(t)} \ dm}
    \end{equation*}
    which makes $f$ an indefinite Lebesgue integral, and hence, absolutely
    continuous.
\end{proof}

\begin{lemma}\label{11.4.3}
    Let $f$ be Lebesgue integrable on a closed bounded interval  $[a,b]$. Then
    $f=0$ almost everywhere on $[a,b]$ if, and only if
    \begin{equation*}
        \int_{x_1}^{x_2}{f(x) \ dm}=0 \text{ for every open subinterval }
        (x_1,x_2) \text{ of } [a,b]
    \end{equation*}
\end{lemma}
\begin{proof}
    If $f=0$ almost everywhere on  $[a,b]$, then for any open subinterval
    $(x_1,x_2)$ of $[a,b]$, we get $\int_{x_1}^{x_2}{f}=0$. Conversely, suppose
    that for any open subinterval $(x_1,x_2)$ of $[a,b]$, that
    \begin{equation*}
        \int_{x_1}^{x_2}{f(x) \ dm}=0
    \end{equation*}
    Notice that $\int_E{f}=0$ for any open set $E$ contained in  $[a,b]$, hence
    for any $G_\d$ set  $G$ contained in,  $\int_G{f}=0$. Now, we also have that
    every measurable subset of $[a,b]$ is of the form $\com{G}{E_0}$, where $G$
    is a  $G_\d$ set, and  $m(E_0)=0$. Let $E^+=\{x \in [a,b] : f(x) \geq 0\}$
    and $E^-=\{x \in [a,b] : f(x) \leq 0\}$. Then  $E^+$ and  $E^-$ are
    measurable and
    \begin{equation*}
        \int_a^b{f^+(x) \ dm}=\int_{E^+}{f(x) \ dm}=0 \text{ and }
        \int_a^b{f^-(x) \ dm}=\int_{E^-}{f(x) \ dm}=0
    \end{equation*}
    so that $f^+$ and  $f^-$ vanish almost everywhere on $[a,b]$, hence $f$ must
    also vanish almost everywhere on  $[a,b]$.
\end{proof}

\begin{theorem}\label{11.3.4}
    Let $f$ be Lebesgue integrable on a closed bounded interval $[a,b]$. Then
    \begin{equation*}
        D{\int_a^x{f(t)} \ dm}=f(x) \text{ almost everywhere on } (a,b)
    \end{equation*}
\end{theorem}
\begin{proof}
    Define $F$ on  $[a,b]$ by
    \begin{equation*}
        F(x)=\int_a^x{f(x) \ dm} \text{ for all } x \in [a,b]
    \end{equation*}
    Then $F$ is an indefinite Lebesgue integral, and hence, is absolutely
    continuous. This makes  $F$ differentiable almost everywhere on  $(a,b)$. We
    now wish to show that $D{F}-f$ vanishes almost everywhere on $(a,b)$. Let
    $(x_1,x_2)$ be an open subinterval of $[a,b]$, then
    \begin{align*}
        \int_{x_1}^{x_2}{D{F(t)}-f(t) \ dm} &=
                \int_{x_1}^{x_2}{D{F(t)} \ dm}-\int_{x_1}^{x_2}{f(t) \ dm}  \\
                &= F(x_2)-F(x_1)-\int_{x_1}^{x_2}{f(t) \ dt}    \\
                &= \int_a^{x_2}{f(t) \ dm}-\int_a^{x_1}{f(t) \ dm}
                            -\int_{x_1}^{x_2}{f(t) \ dt}=0    \\
    \end{align*}
\end{proof}

\begin{definition}
    We call a function of bounded variation \textbf{singular} if its derivative
    vanishes almost everywhere on its domain.
\end{definition}

\begin{lemma}\label{11.3.5}
    An absolutely continuous function is singular if, and only if it is
    constant.
\end{lemma}
\begin{proof}
    Exercise.
\end{proof}

\begin{lemma}\label{11.3.6}
    Let $f$ be a function of bounded variation on a closed bounded interval
    $[a,b]$. Then there exists an absolutely continuous function $g$ on
    $[a,b]$, and a singular function $h$ on  $[a,b]$ for which
    \begin{equation*}
        f=g+h
    \end{equation*}
\end{lemma}
\begin{proof}
    Since $f$ is of bounded variation on  $[a,b]$, $D{f}$ is integrable on
    $[a,b]$. Define then
    \begin{equation*}
        g(x)=\int_a^x{D{f(t)} \ dm} \text{ and }
        h(x)=f(x)-\int_a^x{D{f(t)} \ dm} \text{ for all } x \in [a,b]
    \end{equation*}
    Then $f=g+h$ on $[a,b]$. Notice now that $g$ is an indefinite Lebesgue
    integral, and hence is absolutely continuous. Moreover, by theorem
    \ref{11.4.4},  $D{h}=0$ almost everywhere on $[a,b]$, which makes $h$
    singular.
\end{proof}

\begin{definition}
    Let $f$ be a function of bounded variation on a closed bounded interval
    $[a,b]$. We call the expression of $f=g+h$, where $g$ is an abosultely
    continuous function, and  $h$ is singular the  \textbf{Lebesgue
    decomposition} of $f$. We call  $h$ the  \textbf{singular part} of $f$.
\end{definition}
