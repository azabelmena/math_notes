\section{Theorems of Littlewood, Egoroff, and Lusin}

\begin{lemma}\label{9.3.1}
    Let $E$ be a measurable set having finite measure, and let  $\{f_n\}$ a
    sequence of measurable functions on $E$ converging pointwise to a realvalued
    function  $f$ on  $E$. Then for all  $\n>0$ and  $\d>0$, there exists a
    measurable subset  $A$ of $E$ and an  $N \in \Z^+$ for which
    \begin{equation*}
        |f(x)-f_n(x)|<\n \text{ on } A \text{ for all } n \geq N \text{ and }
        m(\com{E}{A})<\d
    \end{equation*}
\end{lemma}
\begin{proof}
    Since $f$ is realvulued and measurable (by lemma \ref{9.2.1}), the set
    \begin{equation*}
        \{x \in E : |f(x)-f_n(x)|<\n\}
    \end{equation*}
    is measurabe. Hence, the union
    \begin{equation*}
        E_n=\{x \in E : |f(x)-f_n(x)|<\n \text{ for all } k \geq n\}
    \end{equation*}
    is measurable. Then $\{E_n\}$ is an ascending collection of measurable sets
    with
    \begin{equation*}
        E=\bigcup{E_n}
    \end{equation*}
    Now, since  $\{f_n\}$ converges to $f$ pointwise, by the continuity of
    measure, we conclude that
    \begin{equation*}
        m(E)=\lim_{n \xrightarrow{} \infty}{m(E_n)}
    \end{equation*}
    Since $m(E)$ is finite, choose an $N \in \Z^+$ for which  $m(E_N)>m(E)$-\e,
    and define $A=E_n$. Then by excision, we get  $m(\com{E}{A})=m(E)-m(A)<\e$.
\end{proof}

\begin{theorem}[Egoroff's Theorem]\label{9.3.2}
    Let $E$ be a measurable set with finite measure, and let  $\{f_n\}$ a
    sequence of measurable functions on $E$ converging pointwise to a realvalued
    function  $f$ on $E$. Then for each $\e>0$, there exists a closed subset $F$
    of $E$ for which $\{f_n\}$ converges uniformly to $f$ on $F$, and
    $m(\com{E}{F})<\e$.
\end{theorem}
\begin{proof}
    For every $n \in \Z^+$, let $A_n$ be a measurable subset of $E$, and let
    $N(n) \in \Z^+$ such that
    \begin{equation*}
        |f(x)-f_k(x)|<\frac{1}{n} \text{ on } A_n \text{ for all } k \geq N(n)
        \text{ and where } m(\com{E}{A_n})<\frac{\e}{2^{n+1}}
    \end{equation*}
    Define
    \begin{equation*}
        A=\bigcap_{n=1}^\infty{A_n}
    \end{equation*}
    Then by DeMorgan's laws and countable subadditivity
    \begin{equation*}
        m(\com{E}{A})=\sum_{n=1}^\infty{m(\com{E}{A_n})}<
        \sum_{n=0}^\infty{\frac{\e}{2^{n+1}}}<\frac{\e}{2}
    \end{equation*}

    Now, let $\e>0$ and choose an $n_0 \in \Z^+$ such that $\frac{1}{n_0}<\e$.
    Then
    \begin{equation*}
        |f(x)-f_k(x)|<\frac{1}{n_0} \text{ on } A \text{ for all } k \geq N(n_0)
    \end{equation*}
    Now, since $A \subseteq A_{n_0}$, and our hypothesis on $n_0$, we get
    \begin{equation*}
        |f(x)-f_k(x)|<\e \text{ on } A \text{ for all } k \geq N(n_0)
    \end{equation*}
    so that $\{f_n\}$ converges uniformly to $f$ on  $A$ with
    $m(\com{E}{A})<\frac{\e}{2}$. Now, by the inner approximation theorem,
    choose an $F \susbeteq A$, closed, for which  $m(\com{A}{F})<\frac{\e}{2}$.
    Then
    $m(\com{E}{F})<m(\com{E}{A})+m(\com{A}{F})<\frac{\e}{2}+\frac{\e}{2}=\e$. So
    that $\{f_n\}$ converges uniformly to $f$ on  $F$.
\end{proof}

\begin{theorem}[Littlewood's Theorem]\label{9.3.3}
    Let $f$ be a simple function on a measurable set $E$. Then for every
    $\e>0$, there eixsts a continuous function  $g$ on  $\R$ and a closed subset
    $F$ of $E$ such that $f=g$ on  $F$ and  $m(\com{E}{F})<\e$.
\end{theorem}
\begin{proof}
    Let $\{a_1, \dots a_n\}$ the distinct values taken on by $f$, taken on the
    sets $E_1, \dots, E_n$, respectively. Then $\{E_k\}_{k=1}^n$ is a finite
    disjoint collection of measurable sets whose union is $E$, since each $a_i$
    is disjoint. Now, choose closed sets $F_1, \dots, F_n$ such that for all $1
    \leq k \leq n$, $F_k \subseteq E_k$ and $m(\com{E_k}{F_k})<\frac{\e}{n}$.
    Then
    \begin{equation*}
        F=\bigcup_{k=1^n}{F_n}
    \end{equation*}
    is closed and since $\{E_k\}_{k=1}^n$ is a disjoint collection, we get
    \begin{equation*}
        m(\com{E}{F})=\sum_{k=1}^n{m(\com{E_k}{F_k})}
    \end{equation*}

    Now, define $g:F \xrightarrow{} \R$ to take $g(x)=a_k$ for all $x \in F_k$,
    for all  $1 \leq k \leq n$. Since $\{F_k\}$ is a disjoint collection, $g$ is
    well defined. Moreover,  $g$ is continuous on $F$. Then for every  $x \in
    F_i$, there exists an interval  $I$, containing  $x$, and disjoint from the
    union $\bigcup_{k \neq i}{F_k}$. Thus $g$ is constant on  $F \cap I$. Now,
    extend  $g$ to a continuous function  $G:\R \xrightarrow{} \R$. Since
    $G(x)=a_k$ on each $F_k$, and  $m(\com{E_k}{F_k})<\frac{\e}{n}$, we get
    $f=G$ on $F$ where $m(\com{E}{F})<\e$.
\end{proof}

\begin{theorem}[Lusin's Theorem]\label{9.3.4}
    Let $f$ be a measurable function on a measurable set $E$. Then for every
    $\e>0$, there eixsts a continuous function  $g$ on  $\R$ and a closed subset
    $F$ of $E$ such that $f=g$ on  $F$ and  $m(\com{E}{F})<\e$.
\end{theorem}
\begin{proof}
    Consider first when $E$ has finite measure. By the simple approximation
    theorem (theorem \ref{9.2.5}), there exists a sequence $\{f_n\}$ of simple
    functions on $E$, converging pointwise to  $f$ on  $E$. Let  $n \in \Z^+$,
    then byt Littlewood's theorem, choose a  $g_n$ continuous on  $\R$ and a
    closed subset $F_n$ of  $E$ for which  $f=g_n$ on  $F_n$ and
    $m(\com{E}{F_n})<\frac{\e}{2^{n+1}}$. By Egoroff's theorem, there exists a
    closed set $F_0 \subseteq E$ such that $\{f_n\}$ converges uniformly to $f$
    on  $F_0$, with $m(\com{E}{F_0})<\frac{\e}{22}$. Define then
    \begin{equation*}
        F=\bigcap_{n=0}^\infty{F_n}
    \end{equation*}
    Then $F$ is closed and by DeMorgan's laws,
    \begin{equation*}
        m(\com{E}{F}) \leq \frac{\e}{2}+\sum_{n=0}^\infty{\frac{\e}{2^{n+1}}}=\e
    \end{equation*}
    Now, each $f_n$ is continuous on  $F$, and since $F \subseteq F_n$, and
    $f=g_n$ on  $f_n$, we get  $\{f_n\}$ converges uniformly to $f$ on $F$,
    since $F \subseteq F_0$. Moreover, $f|_F$ is continuous on  $F$, thus choose
    a continuous function  $g:\R \xrightarrow{} \R$ such that $f=g$ on  $F$, and
    we are done.
\end{proof}
