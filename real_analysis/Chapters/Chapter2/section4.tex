%----------------------------------------------------------------------------------------
%	SECTION 1.1
%----------------------------------------------------------------------------------------

\section{Perfect Sets}

\begin{theorem}\label{2.4.1}
    If $P \subseteq \R^k$ is a nonempty perfect set, then  $P$ is uncountable.
\end{theorem}
\begin{proof}
    Since every point of $P$ is a limit point of  $P$, we gave that  $P$ must be infinite. 
    Then suppose that  $P$ is countable. For points  $x_n \in P$, construct the sequence 
    $\{U_n\}$ of neighborhoods of  $x_n$, for  $n \in \Z^+$; now by induction, if  $U_1$ is 
    a neighborhood of $x_1$, then for $y \in \hat{U_1}$, $||x_1-y|| \leq r$ for some $r>0$. 
    Now suppos the neighborhood $U_n$ of  $x_n$ has been constructed such that  $U_n \cap P$ 
    is nonempty. Then there is a neighborhood  $U_{n+1}$ fo  $x_{n+1}$ such that  
    $\hat{U_{n+1}} \subseteq U_n$,  $x_n \notin \hat{U_{n+1}}$, and  $U_{n+1} \cap P$ is nonempty. 
    Therefore there is a  nonempty $K_n=U_n \cap P$. Since  $\hat{U_n}$ is close and 
    bounded, $\hat{U}$ is compact, and since  $x_n \notin K_{n+1}$,  $x_n \notin \bigcap_{i=1}^{\infty}{K_i}$, 
    and since  $K_n \subseteq P$,  $\bigcap{K_i}$ is empty, a contradiction.
\end{proof}

\begin{corollary}
    Let $a<b$ be real numbers. Then the closed interval  $[a,b]$ is uncountable. 
    Moreover,  $\R$ is uncountable.
\end{corollary}
\begin{proof}
    We have $[a,b]$ is closed, and perfect (since $(a,b) \susbseteq [a,b] is perfect$), 
    thus $[a,b]$ is uncountable. Moreover, take  $f:\R \rightarrow [a,b]$, by 
    $f(x)=\frac{a+b}{2}x$; then $f$ is a 1-1 mapping of  $\R$ onto  $[a,b]$, which makes 
     $\R$ uncountable.
\end{proof}

\begin{theorem}[The construction of the Cantor set]\label{2.4.2}
    There exists a perfect set in $\R$ which contains no open interval.
\end{theorem}
\begin{proof}
    Let $E_0=[0,1]$, and remove  $(\frac{1}{3},\frac{2}{3})$, and let $E_1=[0,\frac{1}{3}] \cup 
    [\frac{2}{3},1]$. Now remove the open intervals $(\frac{1}{9}, \frac{2}{9})$ $(\frac{3}{9}, \frac{6}{9})$, 
    $(\frac{7}{9}, \frac{8}{9})$,  and let $E_2=[0, \frac{2}{9}] \cup [\frac{2}{9}, \frac{3}{9}]
    \cup [\frac{6}{9}, \frac{7}{9}] \cup [\frac{7}{8}, \frac{8}{9}]$. Continuig the remove 
    the middle third of each interval, we obtain the sequence of compact sets $\{E_n\}$, 
    such that  $E_{n+1} \susbseteq E_n$, and $E_n$ is the union of  $2^n$  closed intervals 
    of length $ \frac{1}{3^n}$. Then let:
        \begin{equation}\label{eq_2.1}
            P=\bigcap_{i=1}^{\infty}{E_i}
        \end{equation}

    Then $P$ is nonempty, and compact.

    Now let  $I$ be the open interval of the form  $(\frac{3k+1}{3^m},\frac{3k+2}{3^m})$, 
    with $k,m \in \Z^+$. Then by the construction of  $P$,  $I$ has no point in  $P$, we 
    also see that every other open interval contains a subinterval of the form of  $I$; 
    them  $P$ contains no open interval.

    Now let  $x \in P$, and let  $S$ be any open interval for which  $x \in S$. LEt  $I_n$ be 
    the closed interval of  $E_n$ such that  $x \in I_n$. Choose  $n$ sufficiently large 
    such that $I_n \susbseteq S$. If  $x_n \neq x$ is an endpoint of  $I_n$, then  $x_n \in P$, 
    and so  $x$ is a limit point of  $P$. Therefore  $P$ is perfect.
\end{proof}

\begin{definition}
    The we call the set $P$ constructed in the proof of theorem \ref{2.4.2} the 
    \textbf{Cantor set}.
\end{definition}
