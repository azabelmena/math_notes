\section{Normed Vector Spaces of real-valued functions}
\label{section_12.1}

\begin{definition}
  Let $\Fc$ be the collection of all extended real-valued functions on a
  Lebesgue measurable set $E$, which are finite almost everywhere on $E$. We
  call two functions $f,g \in \Ff$ \textbf{equivalent} if $f=g$ almost
  everywhere on $E$, and write $f \simeq g$.
\end{definition}

\begin{lemma}\label{lemma_12.1.1}
  The equivalence of extended real-valued function on $\Fc$ is an equivalence
  relation.
\end{lemma}
\begin{proof}
  We have that $f(x)=f(x)$ almost everwhere on $E$, and if  $f \simeq g$, thebn
  $g \simeq f$. Now, let  $f \simeq g$, and  $g \simeq h$. Then  $f=g$ on
  $\com{E}{E_0}$, and $g=h$ on  $\com{E}{E_1}$, where $m(E_0)=0$ and $m(E_1)=0$.
  Then we have $f=h$ on  $\com{E}{(E_0 \cap E_1)}$, and since $E_0 \cap E_1
  \subseteq E_0,E_1$, $m(E_0 \cap E_1)=0$, so that $f \simeq h$.
\end{proof}

\begin{theorem}\label{theorem_12.1.2}
  The quotient space $\faktor{\Fc}{\simeq}$ is a vector space.
\end{theorem}
\begin{proof}
  Let $[f], [g] \in \faktor{\Fc}{\simeq}$, and for $\a,\b \in \R$, define
  $\a[f]+\b[g]=[\a{f}+\b{g}]$. Now, let $[f']=[f]$, and $[g']=[g]$. Then we have
  that $f \simeq f'$ and  $g \simeq g'$, so that  $f=f'$ on $\com{E}{E_0}$ an
  $g=g'$ on  $\com{E}{E_1}$, where $m(E_0)=0$ and $m(E_1)=0$. Then observe that
  $\a{f}+\b{g}=\a{f'}+\b{g'}$ on $\com{E}{(E_0 \cap E_1)}$, where $m(E_0 \cap
  E_1)=0$. This makes $\a{f}+\b{g} \simeq \a{f'}+\b{g'}$, so that
  $[\a{f}+\b{g}]=[\a{f'}+\b{g'}]$, and so linear combinations are independent of
  representatives. The rest of the vector space axioms can then be readily
  verified.
\end{proof}
\begin{remark}
  Since linear combinations are independent of representatives, we denote an
  equivalence class $[f]$ of $\faktor{\Fc}{\simeq}$ by its representative, $f$.
\end{remark}

\begin{theorem}\label{theorem_12.1.3}
  Let $1 \leq p <\infty$ and $E$ be a Lebesgue measurable set, and define
  $L^p(E)$ the collection of all $f \in \faktor{\Ff}{\simeq}$ for which
  \begin{equation*}
    \int_E{|f|^p \ dm}<\infty
  \end{equation*}
  Then $L^p(E)$ is a subspace of $\faktor{\Fc}{\simeq}$.
\end{theorem}
\begin{proof}
  Suppose that $f \simeq g$, so that  $f=g$ almost everywhere on  $E$. Then
  \begin{equation}
  \int_E{|f|^p \ dm}=\int_E{|g|^p \ dm}
  \end{equation}
  so that the elements of $L^p(E)$ are well defined.

  Now, let $a,b \in \R$, then $|a+b| \leq |a|+|b| \leq 2\max{\{|a|,|b|\}}$.
  Hence, we have
  \begin{equation*}
    \int_E{|af+bg|^p \ dm} \leq
    2\Big{(} |a|^p\int_E{|f|^p \ dm}+|b|^p\int_E{|g|^p \ dm} \Big{)}<\infty
  \end{equation*}
  so that $af+bg \in L^p(E)$.
\end{proof}
\begin{corollary}
  The set of all Lebesgue integrable functions form a vector space.
\end{corollary}
\begin{proof}
  Notice that $L^1(E)$ is the set of all Lebesgue integrable functions.
\end{proof}

\begin{definition}
  We call a real-valued function $f$ on a Lebesgue measurable set $E$
  \textbf{essentially bounded} if there exists some $M \geq 0$ for which $|f|
  \leq M$ almost everywhere on $E$; and wee call $M$ an \textbf{essential
  upperbound} of $f$ on  $E$. We denote the set of all essentially bounded
  functions on  $E$ by  $L^\infty(E)$.
\end{definition}

\begin{lemma}\label{lemma_12.1.4}
  Let $E$ be a Lebesgue measurable set. Then $L^1(E)$ is a subspace of
  $\faktor{\Fc}{\simeq}$.
\end{lemma}
\begin{proof}
  If $f \simeq g$, then  $f=g$ almost everywhere on  $E$, so that if $|f| \leq
  M$ almost everywhere on $E$, then $|g| \leq M$ almost everywhere on $E$, for
  $M \geq 0$. Therefore the elements of $L^\infty(E)$ are well defined.

  Now, let $f,g \in L^\infty(E)$. Then there are $M \geq 0$, and $N \geq 0$,
  such that $|f| \leq M$ on $\com{E}{E_0}$ and $|g| \leq N$ on $\com{E}{E_1}$,
  where $m(E_0)=0$ and $m(E_1)=0$. Let $a,b \in \R$. Then $|af+bg| \leq
  |a||f|+|b||g| \leq |a|M+|b|N$ on $\com{E}{(E_0 \cap E_1)}$, and $m(E_0 \cap
  E_1)=0$. So that $af+bg \in L^\infty(E)$.
\end{proof}

\begin{definition}
  We define a \textbf{linear functional} to be an extended real-valued function
  whose domain is a vector space of functions.
\end{definition}

\begin{defintion}
  Let $V$ be a vector space. We call a linear functional  $\| \cdot \|$ on $V$
  a \textbf{norm} if for all vectors $f,g \in V$, and any $\a \in \R$, the
  following are true:
  \begin{enumerate}
    \item[(1)] $\|f+g\| \leq \|f\|+\|g\|$.

    \item[(2)] $\|\a{f}\|=|\a|\|f\|$.

    \item[(3)] $\|f\| \geq 0$, and $\|f\|=0$ if, and only if  $f=0$.
  \end{enumerate}
  We call a vector space together with a norm a \textbf{normed vector space}. We
  call an element $f \in V$ a \textbf{unit} if $\|f\|=1$, and for $f \neq 0$, we
  define the \textbf{normilazation} of $f$ to be the unit vector
  $\frac{f}{\|f\|}$.
\end{defintion}

\begin{example}\label{example_12.1}
  \begin{enumerate}
    \item[(1)] Consider $f \in L^1(E)$, and define $\|f\|_1=\int_E{|f| \ dm}$.
      Then $\| \cdot \|_1$ is a norm on  $L^1(E)$. Indeed, we have that
      \begin{equation*}
        \|f+g\|_1=\int_E{|f+g| \ dm} \leq \int_E{|f| \ dm}+\int_E{|g| \ dm}=
        \|f\|_1+\|g\|_1
      \end{equation*}
      we also have for $\a \in \R$, that
      $\|\a{f}\|_1=\int_E{|\a{f}|}=|\a|\int_E{|f|}$, so that
      $\|\a{f}\|_1=|\a|\|f\|_1$. Lastly, observe that $\int_E{|f|} \geq 0$, and
      that $\int_E{f}=0$ if, and only if $f=0$ almost everywhere on  $E$, so
      that  $\|f\| \geq 0$, and $\|f\|=0$ if, and only if  $f=0$ almost
      everywhere on  $E$.

    \item[(2)] Let $f \in L^\infty(E)$. We define the \textbf{essential
      supremum} $\|f\|_\infty$ of $f$ to be the greatest lowerbound of all
      essential upperbounds of $f$ on $E$; that is
      \begin{equation}
        \|f\|_\infty=\inf{\{M : |f| \leq M \text{ almost everywhere on } E\}}
      \end{equation}
      We claim that $\| \cdot \|_\infty$ is a norm on  $L^\infty(E)$.

      Let $f \in \L^\infty(E)$, and $\a \in \R$. Then for some $M \geq 0$, we
      have $|\a{f}|=|\a||f| \leq |\a|M$ almost everywhere on $E$. Then by
      definition of $\|f\|_\infty$, we have $\|\a{f}\|=|\a|\|f\|_\infty$.
      Moreover, since $M \geq 0$, we have  $\|f\|_\infty \geq 0$. Now, observe
      also that if $M=0$, then  $|f| \leq 0$, so that $|f|=0$ almost everywhere
      on  $E$, so that $\|f\|=0$ if, and only if $f=0$ almost everywhere on
      $E$.

      Lastly, by definition, observe for some  $n \in \Z^+$, that
      \begin{equation*}
        |f| \leq \|f\|_\infty+\frac{1}{n} \text{ on } \com{E}{E_n}
      \end{equation*}
      where $m(E_n)=0$. Take $E_\infty=\bigcap{E_n}$, then as $ \xrightarrow{}
      \infty$, $|f| \leq \|f\|_\infty$ on $\com{E}{E_\infty}$, and
      $m(E_\infty)=0$. That is, $\|f\|_\infty$ is the smallest essential
      upperbound of $f$. Now, let $f,g \in L^\infty(E)$, then
      \begin{equation*}
        |f+g| \leq |f|+|g| \leq \|f\|_\infty+\|g\|_\infty
        \text{ almost everywhere on } E
      \end{equation*}
      therefore $\|f+g\|_\infty \leq \|f\|_\infty+\|g\|_\infty$.

    \item[(3)] For $1 \leq p < \infty$, define $l^p$ to be the collection of all
      real-valued sequences $\{a_n\}$ for which
      \begin{equation*}
        \sum_{n=1}^\infty{|a_n|^p}<\infty
      \end{equation*}
      Then $l^p$ is a vector space. Now, define $\|\{a_n\}\|_p$ by
      \begin{equation*}
        \|\{a_n\}\|=\sum_{n=1}^\infty{|a_n|^p}
      \end{equation*}
      then $\| \cdot \|_p$ defines a norm on $l^p$.

    \item[(4)] Similarly define $l^\infty$ to be the collection of all bounded
      real-valued sequences, and define  $\|\{a_n\}\|_\infty$ by
      \begin{equation*}
        \|\{a_n\}\|_\infty=\sup\{|a_n|\}
      \end{equation*}
      Then $\| \cdot \|_\infty$ is a norm on  $l^\infty$.

    \item[(5)] Let $[a,b]$ a closed bounded interval of $\R$, where $a<b$, and
      consider the vector space  $C[a,b]$ of all continuous real-valued
      functions defined on $[a,b]$. By the extreme value theorem, an $f \in
      C[a,b]$ takes a maximum value, so define $\|f\|_\max$ by
      \begin{equation*}
        \|f\|_{\max}=\max_{a \leq x \leq b}{\{|f(x)|\}}
      \end{equation*}
      Then $\|f\|_\max$ defines a norm on $C[a,b]$.
  \end{enumerate}
\end{example}
