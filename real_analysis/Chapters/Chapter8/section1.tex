\section{$\sigma$-algebras}

\begin{definition}
    Let $X$ be a nonempty set. An  \textbf{algebra} of sets on $X$ is a nonempty
    collection  $\Ac$ of subsets of $X$ which are closed under finite unions and
    complements in $X$. We call  $\Ac$ a \textbf{$\s$-algebra} if it is closed
    under countable unions.
\end{definition}

\begin{lemma}\label{8.1.1}
    Let $X$ be a set and $\Ac$ an algbra on $X$. Then $\Ac$ is closed under
    finite intersections.
\end{lemma}
\begin{proof}
    Let $\{E_\l\}$ be a collection of sets of $\Ac$. Then by finite union
    $E=\bigcup{E_\l} \in \Ac$. Then by complements,
    $\com{X}{E}=\bigcap{\com{X}{E_\l}} \in \Ac$.
\end{proof}
\begin{corollary}
    $\s$-algebras are closed under countable unions.
\end{corollary}

\begin{lemma}\label{8.1.2}
    Let $X$ be a set, and $\Ac$ an algebra on $X$. Then $\emptyset \in \Ac$ and
    $X \in \Ac$.
\end{lemma}
\begin{proof}
    By closure of finite unions, notice that if $E \in \Ac$, then
    $E \cup \com{X}{E} = X \in \Ac$ lemma \ref{8.1.1} gives us that $E \cap
    \com{X}{E}=\emptyset \in \Ac$.
\end{proof}

\begin{example}\label{example_8.1}
    \begin{enumerate}
        \item[(1)] The collections $\{\emptyset, X\}$ and $2^X$ are
            $\s$-algebras on any set $X$.

        \item[(2)] Let $X$ be an uncountable set. Then the collection
            \begin{equation*}
                \Cc=\{E \subseteq X : E \text{ is countable } \text{ or }
                \com{X}{E} \text{ is countable}\}
            \end{equation*}
            defines a $\s$-algebra of sets on $X$, since countable unions of
            countable sets are countable, and $\Cc$ is closed under complements.
            We call $\Cc$ the \textbf{$\s$-algebra of countable or cocountable
            sets}.
    \end{enumerate}
\end{example}

\begin{lemma}\label{8.1.3}
    Let $\{A_\l\}$ be a collection of $\s$-algebras on a set  $X$. Then the
    intersection
    \begin{equation*}
        \Ac=\bigcap{\Ac_\l}
    \end{equation*}
    is a $\s$-algebra on $X$. Moreover, if $F \subseteq X$, then there exists a
    unique smallest $\s$-algebra containing $F$; in particular, it is the
    intersection of all  $\s$-algebras containing  $F$.
\end{lemma}
\begin{proof}
    Notice that since each $\Ac_\l$ is a  $\s$-algebra, they are closed under
    countable unions and complements. Hence by definition,  $\Ac$ must also be
    closed under countable unions and complements.

    Now, let $F \subseteq X$ and let $\{\Ac_\l\}$ be the collection of all
    $\s$-algebras containing $F$. Then the intersection $\Ac=\bigcap{\Ac_\l}$ is
    also a $\s$-algebra containing  $F$; by above. Now, suppose that there is a
    smallest $\s$-algebra $\Bc$ containing $F$. Then we have that $\Bc \subseteq
    \Ac$. Now, by definition of  $\Ac$ as the intersection of all  $\s$-algberas
    containing  $F$, we get that  $\Ac \subseteq \Bc$; so that $\Bc=\Ac$.
\end{proof}

\begin{definition}
    Let $X$ be a nonempty set and  $F \subseteq X$. We define the $\s$-algebra
    \textbf{generated} by $F$ to be the smallest such $\s$-algebra $\Mc(F)$
    containing $F$.
\end{definition}

\begin{lemma}\label{8.1.4}
    Let $X$ be a set and let $E,F \subseteq X$. Then if $E \subseteq \Mc(F)$,
    then $\Mc(E) \subseteq \Mc(F)$.
\end{lemma}
\begin{proof}
    We have that since $E \subseteq \Mc(F)$, and $\Mc(E)$ is the intersection of
    all $\s$-algebras containing  $E$, then  $\Mc(E) \subseteq \Mc(F)$.
\end{proof}

\begin{definition}
    Let $X$ be a topological space. We define the  \textbf{Borel $\s$-algebra}
    on $X$ to be the  $\s$-algebra  $\Bc(X)$ generated by all open sets of $X$;
    that is
    \begin{equation*}
        \Bc(X)=\Mc(\Tc)
    \end{equation*}
    where $\Tc$ is the topology on $X$. We call the elements of $\Bc(X)$
    \textbf{Borel-sets}
\end{definition}

\begin{definition}
    Let $X$ be a topological space. We call a countable intersection of open
    sets of  $X$ a \textbf{$G_\d$-set} of $X$. We call a countable union of
    closed sets of $X$ an  \textbf{$F_\s$-set} of $X$.
\end{definition}

\begin{theorem}\label{8.1.5}
    The Borel $\s$-algebra on $\R$,  $\Bc(\R)$, is generated by the following.
    \begin{enumerate}
        \item[(1)] All open intervals of $\R$.

        \item[(2)] All closed intervals of $\R$.

        \item[(3)] All half-open intervals of $\R$.

        \item[(4)] All open rays of $\R$.

        \item[(5)] All closed rays of $\R$.
    \end{enumerate}
\end{theorem}

%\begin{definition}
    %Let $I$ be a nondegenerate interval on  $\R$. We define the  \textbf{length}
    %of $I$,  $l(I)$ to be the difference of its endpoints if $I$ is bounded and
     %$l(I)=\infty$ otherwise.
%\end{definition}

%\begin{definition}
    %For a set $A \susbeteq \R$ consider a countable collection  $\{I_n\}$ of
    %intervals that covers $A$. We define the \textbf{outer measure} of $A$ to be
    %the map $m^*:2^\R \xrightarrow{} [0,\infty)$ defined by: $m^*(A)
    %=\inf{\{\sum{l(I_n)}\}}$ for all such countable collections.
%\end{definition}

%\begin{lemma}\label{8.1.2}
    %Consider the outer measure $m^*:A \xrightarrow{} m^*(A)$, then the following
    %are true:
    %\begin{enumerate}
        %\item[(1)] $m^*(\emptyset)=0$.

        %\item[(2)] If $A \subseteq B$, then  $m^*(A) \subseteq m^*(B)$.
    %\end{enumerate}
%\end{lemma}
%\begin{proof}
    %Notice that $\emptyset$ is covered by the interval  $[x,x]$, whose length is
    %$0$. Then by definition  $m^*(\emptyset)=l([x,x])=0$.

    %Now let $A \subseteq B$, if  $\{I_n\}$ covers $B$, then  $\{I_n\}$ covers
    %$A$. Let  $U=\{l(I_n) : A \subseteq \bigcup{I_n}\}$ and $V=\{l(I_n) : B
    %\subseteq \bigcup{I_n}\}$. Then $\inf{U} \leq \inf{V}$, so we have that
    %$m^*(A) \leq m^*(B)$.
%\end{proof}

%\begin{example}\label{}
    %Countable sets have outer measure $0$. if  $E$ is countable, then one can
    %list its elements as $E=\{e_1, \dots, e_n, \dots\}$ such that.  $\{e_i\} \cap
    %{\e_j\}=\emptyset$ if $i \neq j$. Notice that  $\{e_i\} \subseteq E$
    %and that $E=\bigcup{\{e_i\}}$. Moreover, each $\{e_i\}$ is covered by
    %$[e_i,e_i]$, with $l([e_i,e_i])=0$. So $m^*(\{e_i\})=0$. Therefore, assuming
    %countable subadditivity of $m^*$, we get that  $m^*(E) \leq
    %\sum{m^*(\{e_i\})}=0$. Since $0 \leq m^*(E)$, we get equality.
%\end{example}

%\begin{theorem}\label{8.1.3}
    %The following is true for outer measure.
    %\begin{enumerate}
        %\item[(1)] $m^*([a,b])=m^*((a,b))=b-a$ for all $a,b \in \R$.

        %\item[(2)] $m^*$ is translation invariant. That is if $m^*(A+y)=m^*(A)$.

        %\item[(3)] $m^*$ is countably subadditive, that is if $\{E_k\}$ is a
            %countable collection of subsets of $E$, covering $E$ then
            %\begin{equation}
                %m^*(E) \leq \sum{m^*(E_k)}
            %\end{equation}
    %\end{enumerate}
%\end{theorem}
%\begin{proof}
    %Consider first the interval $[a,b]$. For $\epsilon>0$  notice
    %$(a-\epsilon,b+\epsilon)$ contains $[a,b]$. Then $m^*([a,b]) \leq
    %m^*((a-\epsilon,b+\epsilon)) \leq l((a-\epsilon,b+\epsilon))=b-a+2\epsilon$.
    %This shows that $m^*([a,b]) \leq b-a$. Now, by the Heine-Borel therem
    %$[a,b]$ is compact, so let $\{I_n\}$ be a open cover of $[a,b]$ and let
    %$\{I_k\}_{k=1}^{n}$ a finite open subcover. Then if $I_1=(a_1,b_1)$ contains
    %$a$, then we have  $a_1<a<b_1$. Now if $b_1 \geq b$ then we have
    %\begin{equation*}
        %\sum{l(I_1)} \leq b_1-a_1>b-a
    %\end{equation*}
    %If $a \leq b_1 < b$, since $b_1 \notin (a_1,b_1)$, there is an interval
    %$I_2=(a_2,b_2)$ with $b_1 \in (a_2,b_2)$, and disjoint from $I_1$. Then if
    %$b_2 \geq b_b$ then we get
    %\begin{equation*}
        %\sum{l(I_2)} \geq (b_1-a_1)+(b_2-a_2)>b_2-a_1>b-a
    %\end{equation*}
    %Proceeding inductively, we get get a subcollection $\{(a_k,b_k)\}_{k=1}^n$
    %with $a_1<a$ and $a_{k+1}b_k$ for all $1 \leq k \leq n-1$. Thus we get
    %\begin{equation*}
        %\sum{l(I_k)} \geq \sum{l((a_k,b_k))} \leq (b_n-a_n)+ \dots
        %+(b_1-a_1)=b_n-a_1>b-a
    %\end{equation*}

    %Now if $I$ is any bounded interval for  $\epsilon>0$ there exist closed
    %bounded intervals  $J_1,J_2$ with $J_1 \leq I \leq J_2$ with
    %$l(I)-\epsilon<l(J_1)<l(I)<l(J_1)<l(I)+\epsilon$. Then we have
    %$l(E)-\epsilon<m^*(J_1) \leq m^*(I) \leq m^*(J_2)<l(I)+\epsilon$. This
    %forces $l(I)=m^*(I)$.

    %Now if $I$ is unbounded there is a subinterval $J \subseteq I$ with
    %$m^*(J)=n$. then $m^*(J)=n \leq m^*(I)$. As $n$ gets alrge, so does
    %$l(I)$, so $m^*(I)=\infty$.

    %Consider now, for any set $A$, the set $A+y$. Then the collecton  $\{I_n\}$
    %covers $A$ if, and only if  $\{I_n+y\}$ covers $A+y$. Moreover, if  $I_n$ is
    %an interval, then  $l(I_n)=l(I_n+y)$. This makes $m^*(A+y)=m^*(A)$.

    %Now, let $\{E_k\}$ be a collection of open subsets of $E$, covering  $E$. If
     %$m^*(E_k)=\infty$ for some $k$, we are done. Suppose then that no  $E_k$
     %has such measure. Consider then the countable collection  $\{I_{n,k}\}_k$
     %of open and bounded initervals covering $E_n$, i.e.  $E_n \subseteq
     %\bigcup_{k}{I_{n,k}}$, and $\sum{l(I_{n,k})} \leq
     %m^*(E_n)+\frac{\epsilon}{2^n}$. Then by definition of $m^*$, we get
     %$m^*(\bigcup{E_n}) \leq
     %\sum{l(I_{n,k})}<\sum{(m^*(E_k)+\frac{\epsilon}{2^n})}<\sum{m^*(E_n)}+\epsilon$.
%\end{proof}

%\begin{definition}
    %We say that an outermeasure $m^*$ is  \textbf{finitely subadditive} if for
    %any finite collection $\{E_k\}_{k=1}^n$ of open subsets of $E$, covering
    %$E$, then
    %\begin{equation*}
        %m^*(E) \leq \sum_{k=1}^n{m^*(E_k)}
    %\end{equation*}
%\end{definition}
