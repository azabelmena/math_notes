\section{$\sigma$-algebras}

\begin{definition}
    Let $X$ be a nonempty set. An  \textbf{algebra} of sets on $X$ is a nonempty
    collection  $\Ac$ of subsets of $X$ which are closed under finite unions and
    complements in $X$. We call  $\Ac$ a \textbf{$\s$-algebra} if it is closed
    under countable unions.
\end{definition}

\begin{lemma}\label{8.1.1}
    Let $X$ be a set and $\Ac$ an algbra on $X$. Then $\Ac$ is closed under
    finite intersections.
\end{lemma}
\begin{proof}
    Let $\{E_\l\}$ be a collection of sets of $\Ac$. Then by finite union
    $E=\bigcup{E_\l} \in \Ac$. Then by complements,
    $\com{X}{E}=\bigcap{\com{X}{E_\l}} \in \Ac$.
\end{proof}
\begin{corollary}
    $\s$-algebras are closed under countable disjoint unions.
\end{corollary}
\begin{proof}
    Let $\Ac$ a $\s$-algebra, and let  $\{E_n\}$ a collection of (not
    necessarily disjoint) sets in $\Ac$. Then take
    \begin{equation}
        F_n=\com{E_n}{\Big{(} \bigcup_{k=1}^{n-1}{E_k} \Big{)}}
    \end{equation}
    Then each $F_n$ is a set in $\Ac$, and are pairwise disjoint. Moreover,
    $\bigcup{E_n}=\bigcup{F_n}$.
\end{proof}

\begin{lemma}\label{8.1.2}
    Let $X$ be a set, and $\Ac$ an algebra on $X$. Then $\emptyset \in \Ac$ and
    $X \in \Ac$.
\end{lemma}
\begin{proof}
    By closure of finite unions, notice that if $E \in \Ac$, then
    $E \cup \com{X}{E} = X \in \Ac$ lemma \ref{8.1.1} gives us that $E \cap
    \com{X}{E}=\emptyset \in \Ac$.
\end{proof}

\begin{example}\label{example_8.1}
    \begin{enumerate}
        \item[(1)] The collections $\{\emptyset, X\}$ and $2^X$ are
            $\s$-algebras on any set $X$.

        \item[(2)] Let $X$ be an uncountable set. Then the collection
            \begin{equation*}
                \Cc=\{E \subseteq X : E \text{ is countable } \text{ or }
                \com{X}{E} \text{ is countable}\}
            \end{equation*}
            defines a $\s$-algebra of sets on $X$, since countable unions of
            countable sets are countable, and $\Cc$ is closed under complements.
            We call $\Cc$ the \textbf{$\s$-algebra of countable or cocountable
            sets}.
    \end{enumerate}
\end{example}

\begin{lemma}\label{8.1.3}
    Let $\{A_\l\}$ be a collection of $\s$-algebras on a set  $X$. Then the
    intersection
    \begin{equation*}
        \Ac=\bigcap{\Ac_\l}
    \end{equation*}
    is a $\s$-algebra on $X$. Moreover, if $F \subseteq X$, then there exists a
    unique smallest $\s$-algebra containing $F$; in particular, it is the
    intersection of all  $\s$-algebras containing  $F$.
\end{lemma}
\begin{proof}
    Notice that since each $\Ac_\l$ is a  $\s$-algebra, they are closed under
    countable unions and complements. Hence by definition,  $\Ac$ must also be
    closed under countable unions and complements.

    Now, let $F \subseteq X$ and let $\{\Ac_\l\}$ be the collection of all
    $\s$-algebras containing $F$. Then the intersection $\Ac=\bigcap{\Ac_\l}$ is
    also a $\s$-algebra containing  $F$; by above. Now, suppose that there is a
    smallest $\s$-algebra $\Bc$ containing $F$. Then we have that $\Bc \subseteq
    \Ac$. Now, by definition of  $\Ac$ as the intersection of all  $\s$-algberas
    containing  $F$, we get that  $\Ac \subseteq \Bc$; so that $\Bc=\Ac$.
\end{proof}

\begin{definition}
    Let $X$ be a nonempty set and  $F \subseteq X$. We define the $\s$-algebra
    \textbf{generated} by $F$ to be the smallest such $\s$-algebra $\Mc(F)$
    containing $F$.
\end{definition}

\begin{lemma}\label{8.1.4}
    Let $X$ be a set and let $E,F \subseteq X$. Then if $E \subseteq \Mc(F)$,
    then $\Mc(E) \subseteq \Mc(F)$.
\end{lemma}
\begin{proof}
    We have that since $E \subseteq \Mc(F)$, and $\Mc(E)$ is the intersection of
    all $\s$-algebras containing  $E$, then  $\Mc(E) \subseteq \Mc(F)$.
\end{proof}

\begin{definition}
    Let $X$ be a topological space. We define the  \textbf{Borel $\s$-algebra}
    on $X$ to be the  $\s$-algebra  $\Bc(X)$ generated by all open sets of $X$;
    that is
    \begin{equation*}
        \Bc(X)=\Mc(\Tc)
    \end{equation*}
    where $\Tc$ is the topology on $X$. We call the elements of $\Bc(X)$
    \textbf{Borel-sets}
\end{definition}

\begin{definition}
    Let $X$ be a topological space. We call a countable intersection of open
    sets of  $X$ a \textbf{$G_\d$-set} of $X$. We call a countable union of
    closed sets of $X$ an  \textbf{$F_\s$-set} of $X$.
\end{definition}

\begin{theorem}\label{8.1.5}
    The Borel $\s$-algebra on $\R$,  $\Bc(\R)$, is generated by the following.
    \begin{enumerate}
        \item[(1)] All open intervals of $\R$.

        \item[(2)] All closed intervals of $\R$.

        \item[(3)] All half-open intervals of $\R$.

        \item[(4)] All open rays of $\R$.

        \item[(5)] All closed rays of $\R$.
    \end{enumerate}
\end{theorem}
