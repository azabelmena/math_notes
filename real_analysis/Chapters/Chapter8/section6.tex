\section{Nonmeasurable Sets}

\begin{definition}
    We call a set $E$ of $\R$ \textbf{nonmeasurable} if it is not a measurable
    set.
\end{definition}

\begin{lemma}\label{8.6.1}
    Let $E$ be a bnounded measurable set, and suppose there exists a bounded
    countably infinite set of reals $\Lambda$ for which the collection of
    translates  $\{\l+E\}$ is disjoint. Then $m(E)=0$.
\end{lemma}
\begin{proof}
    SInce translates of measurable sets are measurable, by countable additivity
    \begin{equation*}
        m\Big{(} \bigcup_{\l \in \Lambda}{(\l+E)} \Big{)}=
        \sum_{\l \in \Lambda}{m(\l+E)}=
        \sum_{\l \in \Lambda}{m(E)}
    \end{equation*}
    Now, since both $E$ and  $\Lambda$ are bounded, then the union
    \begin{equation*}
        \bigcup_{\l \in \Lambda}{(\l+E)}
    \end{equation*}
    is also bounded, and has finite measure. Thus, for each $l \in \Lambda$,
    $m(\l+E)=m(E)$ is finite. Additionally, by translation invariance,
    $m(\l+E)=m(E)>0$, and since $\Lambda$ is countably infinite, we must have
    $m(E)=0$.
\end{proof}

\begin{definition}
    LEt $E \subseteq \R$ be an arbitrary set. We call two points  $x,y \in E$
     \textbf{rationally equivalent} in $E$ provided  $x-y \in \Q$. We write  $x
     \sim y$.
\end{definition}

\begin{lemma}\label{8.6.2}
    For any set $E \subseteq \R$, rational equivalence defines an equivalence
    relation on $E$.
\end{lemma}
\begin{proof}
    We have that for any $x \in E$,  $x-x_=0 \in \Q$, so that $x \sim x$. Now,
    let $x,y \in E$ with $x \sim y$. Then  $x-y \in \Q$, and since $\Q$ is a
    field, $-(x-y)=y-x \in \Q$ so that $y \sim x$. Finally, let $x,y,z \in E$
    with $x \sim y$ and $y \sim z$. Then $x-y,y-z \in \Q$, and again since $\Q$
    is a field, $(x-y)+(y-z)=x-z \in \Q$ which makes $x \sim z$.
\end{proof}

\begin{definition}
    Let $E$ be a nonempty set of  $\R$. We define a  \textbf{choice set} for
    rational equivalence on $E$ to be a set  $\Cc_E$ consisting of exactly one
    member of each equivalence class in  $\faktor{E}{\sim}$; such that
    \begin{enumerate}
        \item[(1)] For all $x,y \in \Cc_E$,  $x \not\sim y$.

        \item[(2)] For every $x \in E$, there exists a $c \in \Cc_E$ such that
            $x=c+q$ for some  $q \in \Q$.
    \end{enumerate}
\end{definition}

\begin{theorem}[Vitali's Theorem]\label{8.6.3}
    Any set $E$ with positive outer measure contains a nonmeasurable set.
\end{theorem}
\begin{proof}
    By countable subadditivity, suppose that $E$ is bounded. Now, let  $\Cc_E$
    be a choice set of rational equivalence on $E$ (which exists by the axiom of
    choice). Then we claim that $\Cc_E$ is nonmeasurable.

    Indeed, suppose that $\Cc_E$ is measurable, and let $\Lambda$ be any
    countably infinite set of rationals. Then the collection  $\{\l+\Cc_E\}$ is
    disjoint and $m(\lamda+\Cc_E)=m(\Cc_E)=0$, by lemma \ref{8.6.1}. Then, by
    translatio invariance, and countable additivity, we have
    \begin{equation*}
        m\Big{(} \bigcup_{\l \in \Lambda}{\l+\Cc_E} \Big{)}=
        \sum_{\l \in \Lambda}{m(\l+\Cc_E)}=\sum_{\l \in \Lambda}{m(\Cc_E)}=0
    \end{equation*}

    Now, since $E$ is bounded,  $E \subseteq [-b,b]$ for some $b \in \R$.
    Choose, then a set  $\Lambda_0=[-2b,2b] \cap \Q$. Then $\Lambda_0$ is
    bounded and countably infinite. Now, if $x \in E$, then there is a  $c \in
    \Cc_E$ for which  $x=c+q$ for some  $q \in \Q$. But  $x,c \in [-b,b]$, so
    that $x,c \in [-2b,2b]$. Thus
    \begin{equation*}
        E \subseteq \bigcup_{\l \in \Lambda_0}{(\l+\Cc_E)}
    \end{equation*}
    but $m^\ast(E)>0$, and $m(\Cc_E)=0$, which is a contradiction! Therefore,
    $\Cc_E$ cannot be measurable.
\end{proof}

\begin{theorem}\label{8.6.4}
    There exist nonmeasurable disjoint sets $A,B \subseteq \R$ for which
    \begin{equation*}
        m^\ast(A \cup B)<m^\ast(A) \cup m^\ast(B).
    \end{equation*}
\end{theorem}
