%----------------------------------------------------------------------------------------
%	SECTION 1.1
%----------------------------------------------------------------------------------------

\section{Properties of the Integral.}

\begin{theorem}\label{7.2.1}
    Let $f$ and  $g$ be realvaulued functions integrable with respect to
    $\alpha$ over  an interval $[a,b]$. Then:
        \begin{enumerate}[label=(\arabic*)]
            \item $f+g$ and  $cf$
                    \begin{equation}
                        \int_{a}^{b}{f+g}d\alpha =
                        \int_{a}^{b}{f}d\alpha+\int_{a}^{b}{g}d\alpha
                    \end{equation}

            \item If $f \leq g$ on  $[a,b]$, then:
                    \begin{equation}
                        \int_{a}^{b}{f} d\alpha \leq \int_{a}^{b}{g d\alpha}
                    \end{equation}

            \item If $c \in (a,b)$ and  $f$  then:
                    \begin{equation}
                        \int_{a}^{c}{f} d\alpha+\int_{c}^{b}{f} d\alpha=\int_{a}^{b}{f} d\alpha
                    \end{equation}

            \item If $|f| \leq M$ on  $[a,b]$, then:
                    \begin{equation}
                        |\int_{a}^{c}{f} d\alpha| \leq M(\alpha(b)-\alpha(a))
                    \end{equation}
            \item If $f$ is intergrable with respect to  $\alpha_1$, and
                $\alpha_2$ on $[a,b]$, and $c>0$, then:
                    \begin{equation}
                        \int_{a}^{b}{f} d(\alpha_1+\alpha_2)=\int_{a}^{b}{f}
                        d\alpha_1+\int_{a}^{b}{f} d\alpha_2
                    \end{equation}
        \end{enumerate}
\end{theorem}
\begin{proof}
    \begin{enumerate}[label=(\arabic*)]
        \item Let $P$ be a partition of  $[a,b]$, then
            $L(f,P,\alpha)+L(f,P,\alpha) \leq L(f+g,P,\alpha) \leq
            U(f+g,P,\alpha) \leq U(f,P,\alpha)+U(g,P,\alpha)$. Now let
            $\epsilon>0$, there are partitions $P_1$, $P_2$, with $P=P_1 \cup
            P_2$ such that $U(f,P_1,\alpha)-L(f,P_1,\alpha)<\frac{\epsilon}{2}$,
            and $U(g,P_2,\alpha)-L(g,P_2,\alpha)<\frac{\epsilon}{2}$. Then we
            get $U(f+g,P,\alpha)-L(f+g,P,\alpha)<\epsilon$, so $f+g$ is
            intergrable. Moreover, we have that
            $U(f,P\alpha)<\int{f}+\frac{\epsilon}{2}$ which guarantees
            $\int{f+g} \leq \int{f}+\int{g}$. Doing the same with $L$, we get
            $\int{f}+\int{g} \leq \int{f+g}$, a nd we are done.

        \item If $f \leq g$, then  $0 \leq g-f$, and we have by part  $(1)$ that
            $g-f$ is intergrable, and that  $0 \leq \int{g-f}=\int{g}-\int{f}$,
            wich gives us the established inequality.

        \item Let  $P_1$ be a ppartition of $[a,c]$, and let  $P_2$ be a
            partition of $[c,b]$, and let  $P=P_1 \cup P_2$ be a partition of
            $[a,b]$. SInce  $f$ is intergrable on  $[a,b]$, there is an
            $\epsilon>0$ such that  $U(f,P,\alpha)-L(f,P,\alpha)<\epsilon$, by
            theorem \ref{7.1.3}, we have that $L(f,P_1,\alpha) \leq L(f,P,\alpha)
            \leq U(f,P,\alpha) \leq U(f,P_1,\alpha)$ and $L(f,P_2,\alpha) \leq L(f,P,\alpha)
            \leq U(f,P,\alpha) \leq U(f,P_2,\alpha)$, thus we get
            $U(f,P_1,\alpha)-L(f,P_2,\alpha)<\epsilon$ and
            $U(f,P_2,\alpha)-L(f,P,\alpha)<\epsilon$, thus $f$ is integrable on
            $[a,c]$ and on $[c,b]$.

            Now we have that  $U(f,P_1,\alpha) <
            \int_{a}^{c}{f}+\frac{\epsilon}{2}$ and
            $\int_{c}^{b}{f}+\frac{\epsilon}{2}$, so $\int_{a}^{b}{f} \leq
            \int_{a}^{c}{f}\int_{c}^{b}{f}$. With $L$ we get the reverse
            inequality, and we are done.

        \item  We have that $-M \leq f \leq M$, now let  $P$ be a partition of
            $[a,b]$, and let  $\epsilon>0$. Since  $f$ is integrable, we have
            that
                \begin{equation}
                    -M(\alpha(b)-\alpha(a)) \leq U(f,P,\alpha)-L(f,P,\alpha)<\epsilon \leq M(\alpha(b)-\alpha(a))
                \end{equation}
                which goves us for arbitrarily small $\epsilon$,  $|\int{f}|
                \leq M(\alpha(b)-\alpha(a))$.

            \item Let $P_1$ and $P_2$ be partitions of $[a,b]$, respective to
                $\alpha_1$ and $\alpha_2$, and construct $P=P_1 \cup P_2$
                respective to $\alpha_1+\alpha_2$. We have that
                $U(f,P_1,\alpha_1)-L(f,P_1,\alpha_1)<\frac{\epsilon}{2}$ and $U(f,P_2,\alpha_2)-
                L(f,P_2,\alpha_2)<\frac{\epsilon}{2}$, then we have by the
                previous inequalities that:
                    \begin{equation*}
                        U(f,P,\alpha_1+\alpha_2)-L(f,P,\alpha_1+\alpha_2)
                        \leq (U(f,P_1,\alpha_1)+U(f,P_2,\alpha_2))
                        -(L(f,P_1,\alpha_1)+L(f,P_2,\alpha_2))<\epsilon
                    \end{equation*}
                Thus, $f$ is integrable with respect to  $\alpha_1+\alpha_2$.
                Now again, by the similar reasoning used in parts $(1)$ and
                $(3)$, we get that
                $\int{f}\d(\alpha_1+\alpha_2)=\int{f} d\alpha_1+\int{f}
                d\alpha_2$.
    \end{enumerate}
\end{proof}
\begin{corollary}
    $\int{cf} d\alpha=c\int{f} d\alpha$ and $\int{f} d(c\alpha)=c\int{f} d\alpha$
\end{corollary}

\begin{theorem}\label{7.2.2}
    If $f$ and  $g$ are integrable with respect to  $\alpha$ on an interval
    $[a,b]$, then:
        \begin{enumerate}[label=(\arabic*)]
            \item $fg$ is integrable with respect to  $\alpha$ on  $[a,b]$.

            \item $|f|$ is integrable with respect to  $\alpha$ on  $[a,b]$ and:
                \begin{equation}
                    |\int_{a}^{b}{f} d\alpha| \leq \int_{a}^{b}{|f|} d\alpha
                \end{equation}
        \end{enumerate}
\end{theorem}
\begin{proof}
    \begin{enumerate}[label=(\arabic*)]
        \item Take $\phi(t)=t^2$, then by theorem \ref{7.1.10}, $\phi \circ
            f=f^2$ is integrable; then notice that $4fg=(f+g)^2-(f-g)^2$ is
            integrable, hence, so is  $fg$.

        \item Take $\phi(t)=|t|$, then again by theorem \ref{7.1.10},  $\phi
            \circ f=|f|$ is integrable. Furthermore, choose  $c=\pm1$ such that
            $0 \leq c\int{f}$, since  $cf \leq |f|$, we have that
            $|\int{f}|=c\int{f}=\int{cf} \leq \int{|f|}$.
    \end{enumerate}
\end{proof}

\begin{definition}
    Let $\psi$ be a real-valued function on an interval $[a,b]$. We call $\psi$
    a  \textbf{step function} provided there is a partition
    $P=\{a=x_0<\dots<x_n=b\}$ of $[a,b]$, and real numbers $c_1, \dots, c_n$
    such that $\psi(x)=c_i$  if $x_i<x<x_{i+1}$ for all $0 \leq i \leq n-1$. We
    define the \textbf{unit step function} $I: \R \rightarrow \{0,1\}$ to be
    $I(x)=1$ if  $x>0$ and  $I(x)=0$ if  $x \leq 0$.
\end{definition}

\begin{theorem}\label{7.2.3}
    If $s \in (a,b)$, and $f$ is bounded on  $[a,b]$ and continuous at $s$, and
    if $\alpha(x)=I(x-s)$, then  $\int{f}=f(s)$.
\end{theorem}
\begin{proof}
    Let $P=\{x_0,x_1,x_2,x_3\}$ be a partition of $[a,b]$, such that
    $a=x_0<s=x_1<x_2<x_3=b$. THen $U(f,P,\alpha)=M_2$ and  $L(f,P,\alpha)=m_2$.
    Now since $f$ is continuous, $ M_2,m_2 \rightarrow f(s)$ as $ x_2
    \rightarrow s$, and we are done.
\end{proof}

% there goes theorem 7.2.4

\begin{theorem}\label{7.2.5}
    Suppose that $\alpha$ is monotonically increasing, and differentiable on an
    interval $[a,b]$, and suppose that  $\alpha'$ is Riemann integrable on
    $[a,b]$. Let $f$ be a bounded realvalued function on  $[a,b]$, then  $f$ is
    integrable with respect to  $\alpha$ on  $[a,b]$ if and only if  $f\alpha'$
    is Riemann integrable.
\end{theorem}
\begin{proof}
    Let $\epsilon>0$, and let  $P$ be a partition of  $[a,b]$ such that
    $U(\alpha',P)-L(\alpha',P)<\epsilon$. By the mean value theorem, there is a
    $t_i \in [x_{i-1},x_i]$ for which $\Delta{\alpha_i}=\alpha'(t_i)
    \Delta{x_i}$, for $1 \leq i \leq n$. Now if $s_i \in [x_{i-1},x_i]$, then
    $\sum{|\alpha'(s_i)-\alpha'(t_i)| \Delta{x_i}}<\epsilon$, and by theorem
    \ref{7.1.6}, with  $M=\sup{|f|}$, since  $\sum{f(s_i)
    \Delta{\alpha_i}}=\sum{f(s_i)\alpha'(t_i) \Delta{x_i}}$, we get
        \begin{equation*}
            |\sum{f(s_i) \Delta{\alpha_i}}-\sum{f(s_i)\alpha'(t_i) \Delta{x_i}}|
            \leq M\epsilon
        \end{equation*}
        in particular, $\sum{f(s_i) \Delta{x_i}} \leq U(f,\alpha',P)+M\epsilon$,
        hence we get  $U(f,P,\alpha) \leq U(f\alpha',P)+M\epsilon$, hence we
        have:
            \begin{equation*}
                |\bar{\int}{f} d\alpha-\bar{\int}{f\alpha'} dx| \leq M\epsilon
            \end{equation*}
        for small enough $\epsilon$, $\bar{\int}{f} d\alpha=\bar{\int}{f\alpha'} dx$.
        We get an analogoues result for lower sums.
\end{proof}
\begin{remark}
    Taking $\alpha$ to be a step function allows us to reduce the integral to be
    a finite or infinite series; similarly, if the derivative of  $\alpha$  (if
    it exists) is integrable, then the Riemann-Stielhes integral reduces to just
    the Riemann integral.
\end{remark}

\begin{theorem}[Change of Variables]\label{7.2.6}
    Of $\phi$ is a strictly increasing continuous function, mapping the interval
    $[A,B]$ onto the interval  $[a,b]$, and if  $\alpha$ is monotonically
    increasing on  $[a,b]$, and if  $f$ is integrable with respect to  $\alpha$
    on  $[a,b]$, define  $\beta$ and  $g$ on  $[a,b]$ by $\beta=\alpha \circ
    \phi$ and  $g=f \circ \phi$, then $g$ is integrable with respect to  $\beta$
    on  $[A,B]$, and
        \begin{equation}
            \int_{A}^{B}{g} d\beta=\int_{a}^{b}{f} d\alpha
        \end{equation}
\end{theorem}
\begin{proof}
    For each partition $P$ of  $[a,b]$, construct a partitoon  $Q$ of  $[A,B]$
    such that  $x_i=\phi(y_i)$. Then the values of  $f$ on  $[x_{i-1},x_i]$ are
    the same as the values of $g$ on  $[y_{i-1},y_i]$. Then
    $U(g,Q,\beta)=U(f,P,\alpha)$ and $L(g,Q,\beta)=L(f,P,\alpha)$. It follows
    that since $f$ is integrable, then  $U(g,Q,\beta)-L(g,Q,\beta)<\epsilon$ for  $\epsilon>0$,
    thus $g$ is integrable; and the equality is established by the equality of
    their sums.
\end{proof}
\begin{corollary}
    Let $\alpha(x)=x$ and  $\beta=\phi$, and suppose that  $\phi'$ is Riemann
    integrable on  $[A,B]$. Then:
        \begin{equation}
            \int_{a}^{b}{f} dx=\int_{A}^{B}{(f\circ \phi)\phi' dy}
        \end{equation}
\end{corollary}
