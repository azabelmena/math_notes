%----------------------------------------------------------------------------------------
%	SECTION 1.1
%----------------------------------------------------------------------------------------

\section{Infinite Limits and Limits at Infinity.}

\begin{definition}
    For any $c \in \R$, the set of all real numbers $x$ such that  $x>c$ is called the
    \textbf{neighborhood of $\infty$}, and denoted $(c,\infty)$. The set of all real
    numbers $x$ such that  $x>c$ is called the \textbf{neighborhood of $-\infty$}, and
    denoted $(-\infty,c)$.
\end{definition}

\begin{definition}
    Let $f:E \rightarrow \R$ be a realvalued function. We say that  $f(t) \rightarrow A$ as $t \rightarrow x$, with
     $A$, and $x$ extended real numbers if for every neighborhood of $U$ $A$, there is a
     neighborhood $V$ of $x$ such that $V \cap E$ is nonempty, and $f(t) \in U$ for
     all $t \neq x \in V \cup E$.
\end{definition}

\begin{theorem}\label{5.7.1}
    Let $f,g:E \rightarrow \R$ be realvalued functions such that  $f \rightarrow A$, and  $g \rightarrow B$ as
    $t \rightarrow x$, for extended real nuumbers $A$, $B$, and $x$. Then the following hold
    as $t \rightarrow x$.
         \begin{enumerate}
             \item[(1)] $f \rightarrow A'$ implies $A=A'$.

             \item[(2)] $f+g \rightarrow A+B$.

             \item[(3)] $fg \rightarrow AB$.

             \item[(4)] $\frac{f}{g} \rightarrow \frac{A}{B}$.
        \end{enumer te}
        Provided that $(1)$, $(2)$, and  $(3)$ are not of the forms  $\infty-\infty$,
        0 \cdot \infty,  \frac{\infty}{\infty}, and $ \frac{A}{0}$, respectively.
\end{theorem}
\begin{proof}
    This is a direct application of the sequential criterion using the appropriate definition.
\end{proof}
