%----------------------------------------------------------------------------------------
%	SECTION 1.1
%----------------------------------------------------------------------------------------

\section{Convergent Sequences}

\begin{definition}
    A sequence $\{x_n\}$ in a metric space  $X$ is said to \textbf{converge} if there is 
    a point $x \in X$ such that for every  $\epsilon>0$, there is an  $ N \in \Z^+$ such that 
    $d(x_n,x)<\epsilon$ whenever $n \geq N$. We say $\{x_n\}$ \textbf{converges} to $x$, and we 
    call  $x$ the \textbf{limit} of  $\{x_n\}$ as  $n$ approaches  $\infty$. We write
    $x_n \rightarrow x$ as $n \rightarrow \infty$, and  $\lim_{n \rightarrow \infty}{x_n}=x$  (or $\lim{x_n}=x$).
    If $\{x_n\}$ does not converge, we say the  $\{x_n\}$ \textbf{diverges}, or \textbf{is divergent}.
\end{definition}

\begin{example}
    Consider the following sequences in $\C$.
        \begin{enumerate}[label=(\arabic*)]
            \item $\{\frac{1}{n}\}$ is bounded, and $\lim_{n \rightarrow \infty}{ \frac{1}{n}}=0$.

            \item The sequence $\{n^2\}$ us unbounded and diverges.

            \item $1+\frac{(-1)^n}{n} \rightarrow 1$ as $n \rightarrow \infty$, and  $\{1+\frac{(-1)^n}{n}\}$ is bounded.

            \item $\{i^n\}$ is bounded and divergent.

            \item  $\{1\}$ is bounded and converges to  $1$.
        \end{enumerate}
\end{example}

\begin{theorem}\label{3.1.1}
    Let $\{x_n\}$ be a sequence in a metric space, then:
        \begin{enumerate}[label=(\arabic*)]
            \item $\{x_n\}$ converges to  $x \in X$ if and only if every every neighborhood 
                of $x$ contains  $x_n$ for all but finitely many  $n$.

            \item If  $\{x_n\}$ converges to  $x$, and  $x'$, then $x=x'$.

            \item If  $\{x_n\}$ converges, then  $x_n$ is bounded.

            \item If  $E \subseteq X$, and  $x$ is a limit point of  $E$, then there 
                is a sequence  in  $E$ that converges to  $x$.
        \end{enumerate}
\end{theorem}
\begin{proof}
    Suppose $x_n \rightarrow x$, and let  $U$ be a neighborhood of  $x$. For some  $\epsilon>0$, 
    there is an $N \in \Z^+$ for which $d(x_n,x)<\epsilon$, whenever $n \geq N$, thus 
    $x_n \in U$ for finitely many  $n$. Conversely, suppose that  $x_n \in U$ for 
    some  $n \geq N$, then  letting  $\epsilon>0$, we havae  $d(x,x_n)<\epsilon$, hence 
    $x_n \rightarrow x$.

    Let  $\epsion>0$, then there are $N_1,N_2 \in Z^+$ such that $d(x_n,x)<\frac{\epsilon}{2}$, 
    and $d(x_n,x')<\frac{\epsilon}{2}$. Then choosing $N=\max\{N_1,N_2\}$, and letting 
    $\epsilon$ be arbitrarily small, we have  $d(x,x') \leq d(x,x_n)+d(x_n,x')<\frac{\epsilon}{2}+
    \frac{\epsilon}{2}=\epsilon$; and so we get that $x=x'$.

    Let  $x_n \rightarrow x$, then there is an  $N \in \Z^+$ for which  $d(x_n,x)<1$ whenever 
    $n \geq N$. Letting  $r=\max\{1,d(x_N,x)\}$, then  $d(x_n,x) \leq r$.

    Finally, let $x$ be a limit point of  $E$, then for each  $n \in Z^+$, there is an  $x_n \in E$ 
    such that  $d(x,x_n)<\frac{1}{n}$, choose $N>\frac{1}{\epsilon}$, then whenever $n\ geq N$, 
    $d(x,x_n)<\epsilon$; hence  $x_n \rightarrow x$.
\end{proof}

\begin{theorem}\label{3.1.2}
    Suppose $\{x_n\}$ and  $\{y_n\}$ are sequences in  $\C$, and that  $lim{x_n}=x$, 
    $\lim{y_n}=y$ as  $n \rightarrow \infty$. Then the following hold as  $n \rightarrow \infty$:
         \begin{enumerate}[label=(\arabic*)]
             \item $\lim{(x_n+y_n)}=\lim{x_n}+\lim{y_n}=x+y$.

             \item $lim{x_ny_n}=\lim{x_n}\lim{y_n}=xy$.

             \item $lim{\frac{x_n}{y_n}}=\frac{\lim{x_n}}{\lim{y_n}}=\frac{x}{y}$; 
                 given that $y_n, y \neq 0$.
        \end{enumerate}
\end{theorem}
\begin{proof}
    \begin{enumerate}[label=(\arabic*)]
        \item Let $\epsion>0$, then for  $N_1, N_2 \in \Z^+$, $|x_n-x|<\frac{\epsilon}{2}$ and 
            $|y_n-y|<\frac{\epsilon}{2}$. Then choose $N=\max\{N_1,N_2\}$, then whenever $n \geq N$, 
            we have  $|(x_n+y_n)-(x+y)| \leq |x_n-x|+|y_n-y|<\epsilon$.

        \item Notice that $x_ny_n-xy=(x_n-x)(y_n-y)+x(y_n-y)+y(x_x-x)$, then for  
            $N_1,N_2 \in \Z^+$, $|x_n-x|<\sqrt{\epsilon}$, and  $|y_n-y|<\sqrt{\epsilon}$. Then 
            choosing  $N=\max\{N_1,N_2\}$, then $|(x_n-x)(y_n-y)|<\epsilon$, thus 
            we have $|x_ny_n-xy|\leq |(x_n-x)(y_n-y)|+|x(y_n-y)|+|y(x_x-x)|<\epsilon$.

        \item We first show that $\frac{1}{y_n} \rightarrow \frac{1}{y}$, given that $y_n,y \neq 0$. 
            Choose  $m$ such that  $|y_n-y|<\frac{1}{2}|y|$ whenever $n \geq m$, then 
            $|y_n|>\frac{1}{2}|y|$. Then for $\epsilon>0$, there is an  $N>m$ such that whenever  $n \geq N$, 
             $|y_n-y|<\frac{1}{2}|y|^2\epsilon$. Then $|\frac{1}{y_n}-\frac{1}{y}| \leq 
             \frac{|y_n-y|}{|y_ny|}<\frac{2}{|y|^2}|y_n-y|<\epsilon$. Then choosing the sequences 
             $\{x_n\}$ and  $\{\frac{1}{y_n}\}$, the rest follows.
    \end{enumerate}
\end{proof}

\begin{corollary}
    \begin{enumerate}[label=(\arabic*)]
        \item For any $c \in \C$, and a sequene  $x_n \rightarrow x$, we have  $\lim{cx_n}=
            c\lim{x_n}=cx$ and  $\lim{(c+x_n)}=c+\lim{x_n}=c+x$ as $n \rightarrow \infty$.

        \item Provided that  $x,x_n \neq 0$, we have  $\lim{\frac{1}{x_n}}=\frac{1}{\lim{x_n}}=\frac{1}{x}$, 
            as $n \rightarrow \infty$.
    \end{enumerate}
\end{corollary}
\begin{proof}
    We choose $\{x_n\}$ and  $\{y_n\}=\{c\}$ for all  $n$, then the results follow.		
\end{proof}

\begin{theorem}\label{3.1.3}
    \begin{enumerate}[label=(\arabic*)]
        \item Let $x_n=(\alpha_{1n}, \dots \alpha_{kn}) \in \R^k$. Then $\{x_n\}$ converges to  $x$ 
            if and only if $\lim{\alpha_{jn}}=\alpha_j$ for  $1 \leq j \leq k$, as  $n \rightarrow \infty$.

        \item Let  $\{x_n\},\{y_n\}$ be sequences in  $\R^k$, and let  $\{\beta_n\}$ be a 
            sequence in  $\R$  such that $x_n \rightarrow x$,  $y_n \rightarrow y$, and  $\beta_n \rightarrow \beta$. Then 
        $\lim{(x_n+y_n)}=x+y$, $\lim{x_ny_n}=xy$, and  $lim{\beta_nx_n}=\beta x$.
    \end{enumerate}
\end{theorem}
\begin{proof}
    If $x_n \rightarrow x$, then  $|\alpha_{jn}-\alpha_j| \leq ||x_n-x||<\epsilon$, thus $\lim{\alpha_{jn}}=\alpha_j$. Conversely, 
    suppose that $\alpha_{jn} \rightarrow \alpha_j$. Then for $\epsilon>0$ there is an $N \in \Z^+$ such that 
    $n \geq N$ implies $|\alpha_{jn}-\alpha_j|<\frac{\epsilon}{\sqrt{k}}$. Then for $n \geq N$, 
        \begin{equation*}
            ||x_n-x||=\sqrt{\sum{|\alpha_{jn}-\alpha_j|^2}}<epsilon
        \end{equation*}
        To prove $(2)$, we appy part $(1)$ of this theorem together with theorem \ref{3.1.2}.
\end{proof}

\begin{theorem}[The Sandwhich Theorem]\label{3.1.4}
    Let $\{x_n\}$, $\{y_n\}$, and  $\{w_n\}$ be sequences in $\R$, and  Suppose that 
    $\lim{x_n}=\lim{y_n}=a$ and that there is an $N \in \Z^+$ such hat  $x_n \leq 
    w_n \leq y_n$ for all  $n \geq N$. Then  $\lim_{n \rightarrow \infty}{w_n}=a$. 
\end{theorem}
\begin{proof}
    Let $\epsilon>0$ and let  $\{x_n\}$ and  $\{y_n\}$ both converge to  $a$. Then by definition there are  $N_1,N_2 \in \Z^+$ 
    such that $|x_n-a|<\epsilon$ and  $|y_n-a|<\epsilon$ for  $n \geq N_1,N_2$. Now choose $N=\max\{N_0,N_1,N_2\}$, if 
    $n \geq N$, we have  $-\epsilon<x_n-a<\epsilon$, and we also have  $x_n-a<w_n-a<y_n-a$, thus we have that:
        \begin{equation*}
            -\epsilon<x_n-a<w_n-a<y_n-a<\epsilon
        \end{equation*}
    Thus we have that $|w_n-a|<\epsilon$.
\end{proof}

\begin{corollary}
    If $x_n \rightarrow \infty$ as  $n \rightarrow \infty$, and  $\{y_n\}$ is bounded, then $x_ny_n \rightarrow 0$ as 
    $n \rightarrow \infty$.
\end{corollary}
\begin{proof}
    We have that $\{y_n\}$ is bounded, hence, there is $M>0$ such that  $|y_n|<M$ for all  $n \in \Z^+$. And since $\{x_n\}$ 
    converges to $0$ we have that for any $\epsilon$ there is an  $N \in \Z^+$ such that for  $n \geq N$,  $|x_n-0|<\frac{\epsilon}{M}$.
    For $|x_ny_n-0|=|x_ny_n|<M|x_n|<M\frac{\epsilon}{M}=\epsilon$. Therefore, $x_ny_n \rightarrow 0$ as  $n \rightarrow \infty$.
\end{proof}

\begin{corollary}
    Let $\{x_n\}$, $\{y_n\}$ be sequences such that  $0 \leq x_n \leq y_n$ for  $n \geq N>0$. Then 
    if  $y_n \rightarrow 0$, then  $x_n \rightarrow 0$ as  $n \right \infty$.
\end{corollary}
\begin{proof}
    This is a special case of the sandwhich theorem.
\end{proof}
