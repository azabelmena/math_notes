\section{Contiunous Real-valued Functions}

\begin{definition}
    Let $f:E \xrightarrow{} \R$ be a real-valued function. We say that $f$ is
     \textbf{continuous} at a point $x \in E$ provided that for every  $\e>0$,
     there is a  $\d>0$ such that
     \begin{equation*}
         |f(y)-f(x)|<\e \text{ whenever } |y-x|<\d
     \end{equation*}
     We call $f$  \textbf{continuous} on all of $E$ provided it is continuous at
     every point of  $E$. We say that  $f$ is  \textbf{Lipschitz continuous} if
     there exists a $c \geq 0$ for which
     \begin{equation*}
         |f(y)-f(x)| \leq c|y-x| \text{ for all } x,y \in E
     \end{equation*}
\end{definition}

\begin{example}\label{example_1.3}
    \begin{enumerate}
        \item[(1)] Lipschitz continuous functions are continuous.

        \item[(2)] The function $f(x)=\sqrt{x}$ is continuous on $[0,1]$, but
            not Lipschitz continuous.
    \end{enumerate}
\end{example}

\begin{lemma}[The Sequential Criterion]\label{1.4.1}
    A real-valued function $f:E \xrightarrow{} \R$ is continuous at a point $x
    \in E$ if, and only if for any sequence $\{x_n\}$ of real numbers converging
    to $x$, the sequence  $\{f(x_n)\} \xrightarrow{} f(x)$.
\end{lemma}

\begin{lemma}\label{1.4.2}
    Let $f:E \xrightarrow{} \R$ a realvalued function. Then $f$ is continuous on
     $E$ if, and only if for every  $U$ open in $\R$,
     \begin{equation*}
         \inv{f}(U)=E \cap V \text{ for some } V \text{ open in } \R
     \end{equation*}
\end{lemma}
\begin{proof}
    Suppose that $\inv{f}(U)=E \cap V$ for some $V$ open in  $\R$. LEt  $x \in
    E$, and let  $\e>0$. THen the interval  $I=(f(x)-\e,f(x)+\e)$ is open,
    therefore
    \begin{equation*}
        \inv{f}(U)=\{y \in E : f(x)-\e<f(y)<f(x)+\e\}=E \cap V
    \end{equation*}
    In particular, $f(E \cap V) \subseteq I$, and $x \in E \cap V$. Now, since
    $V$ is open in  $\R$, there is a  $\d>0$ such that  $(x-\d,x+\d) \subseteq
    V$. Thus we get
    \begin{equation*}
        |f(y)-f(x)|<\e \text{ whenever } |y-x|<\d
    \end{equation*}
    by definition, $f$ is continuous on $E$.

    Conversely, suppose that $f$ is continuous and take  $U$ open in  $\R$, and
    a point  $x \in \inv{f}(U)$. Then $f(x) \in U$ so that there is an $\e>0$
    such that  $(f(x)-\e,f(x)+\e) \subseteq U$. Since $f$ is continuous at  $x$,
    we get a  $\d>0$ for which  $|f(y)-f(x)|<\e$ whenever $|y-x|<\d$. Now,
    define $I_x=(x-\d,x+\d)$ then $f(E \cap I_x) \subseteq U$. Now, define
    \begin{equation*}
        I=\bigcup_{x \in \inv{f}(U)}{I_x}
    \end{equation*}
    since $I$ is the union of open sets, $I$ is open, and we get $\inv{f}(U)=E
    \cap I$.
\end{proof}

\begin{theorem}[The Extreme Value Theorem]\label{1.4.3}
    Continuous real-valued functions on a nonempty closed, and bounded domain
    take on a minimum value and a maximum value.
\end{theorem}
\begin{proof}
    Let $f:E \xrightarrow{} \R$ be a continuous real-valued function where $E$
    is closed and bounded. We first show that $f$ is bounded. Let  $x \in E$, and
    $\d>0$ respond to  $\e=1$. Define  $I_x=(x-\d,x+\d)$. THen if $y \in E \cap
    I_x$, we get $|f(y)-f(x)|<1$, so that $|f(y)| \leq |f(x)|+1$. Now, the
    collection $\{I_x\}$ forms an open cover for $E$, and since  $E$ is closed
    and bounded, by the theorem of Heine-Borel, $E$ is compact, and has a finite
    subcover  $\{I_{x_k}\}_{k=1}^n$. Define now
    \begin{equation*}
        M=1+\max{\{|f(x_1)|, \dots, |f(x_n)|\}}
    \end{equation*}
    and let $x \in E$. THen there is a $k \Z^+$ such that $x \in I_{x_k}$, and
    hence
    \begin{equation*}
        |f(x)| \leq 1+|f(x_k)| \leq M
    \end{equation*}
    which makes $f$ bounded.

    Now to see that  $f$ takes its maximum value, let  $m=\sup{f(E)}$, which
    exists. Suppose however that $f$ failes to attain this values; i.e. there is
    no $x \in E$ for which  $f(x)=m$. Then the function $g:E \xrightarrow{} \R$
    defined by
    \begin{equation*}
        g(x)=\frac{1}{f(x)-m}
    \end{equation*}
    is continuous on $E$, but unbounded; which contradicts what was shown above.
    Therefore $f$ achieves $m$. Now, to see that  $f$ attains its minimum,
    observe the function  $-f$.
\end{proof}

\begin{theorem}[The Intermediate Value Theorem]\label{1.4.4}
    If $f:[a,b] \xrightarrow{} \R$ is a continuous real-valued function for
    which $f(a)<c<f(b)$, then there exists an $x_0 \in (a,b)$ with $f(x_0)=c$.
\end{theorem}
\begin{proof}
    Let $a_1=a$ and $b_1=b$, and take $m_1$ to be the midpoint of $[a,b]$. If
    $c<f(m_1)$, define $a_2=a_1$ and $b_2=m_1$; otherwise if $f(m_1) \geq c$,
    define $a_2=m_1$ and $b_2=b_1$. Thus we get
    \begin{equation*}
        f(a_2) \leq c \leq f(b_2)
    \end{equation*}
    and
    \begin{equation*}
        b_2-a_2=\frac{b_1-a_1}{2}
    \end{equation*}
    proceeding inductively, obtain a descending collection of closed bounded
    intervals $\{[a_n,b_n]\}$ such that
    \begin{equation*}
        f(a_n) \leq c \leq f(b_n) \text{ and } b_n-a_n=\frac{b-a}{2^{n-1}}
        \text{ for all } n \in \Z^+
    \end{equation*}
    By the nested interval theorem the intersection
    \begin{equation*}
        I=\bigcap{[a_n,b_n]}
    \end{equation*}
    is nonempty. Now, let $ x_0 \in I$, and observe that
    \begin{equation*}
        |a_n-x_0| \leq b_n-a_n=\frac{b-a}{2^{n-1}}
    \end{equation*}
    so that the sequence of endpoints $\{a_n\} \xrightarrow{} x_0$. By the
    sequential criterion, and continuity of $f$ at  $x_0$, we have $\{f(a_n)\}
    \xrightarrow{} f(x_0)$. Now, since $f(a_n) \leq c$ for all $n$,a nd
    $(-\infty,c]$ is closed, we get $f(x_0) \leq c$. A similar argument shows
    that $f(x_0) \geq c$.
\end{proof}

\begin{definition}
    A real-valued function $f:E \xrightarrow{} \R$ is said to be
    \textbf{uniformly continuous} provided for every $\e>0$, there is a  $\d>0$
    such that for all  $x,y \in E$
    \begin{equation*}
        |f(y)-f(x)|<\e \text{ whenever } |y-x|<\d
    \end{equation*}
\end{definition}

\begin{theorem}\label{1.4.5}
    A continuous real-valued function on a closed bounded set of real numbers is
    uniformly continuous.
\end{theorem}
\begin{proof}
    Let $f:E \xrightarrow{} \R$ be continuous, where $E$ is closed and bounded.
    Let  $\e>0$, then for every  $x \in E$, there is a  $\d_x>0$ such that if
    $y \in E$, and  $|y-x|<\d_x$, then  $|f(y)-f(x)|<\frac{\e}{2}$. Now, define
    $I_x=(x-\frac{\d_x}{2},x+\frac{\d_x}{2})$. Then $\{I_x\}$ is an open cover
    of $E$, and by the theorem of Heine-Borel, $E$ is compact and has a finite
    subcover $\{I_{x_k}\}_{k=1}^n$. Define then
    \begin{equation*}
        \d=\min{\{\d_{x_1}, \dots, \d_{x_n}\}}
    \end{equation*}
    and let $\e>0$. Now, let  $x,y \in E$ with  $|y-x|<\d$. Since  $\{I_{x_k}\}$
    covers $E$, there is a  $1 \leq k \leq n$ for which
    $|x-x_n|<\frac{\d_{x_k}}{2}$. Now, since
    \begin{equation*}
        |y-x|<\d \leq \frac{\d_{x_k}}{2}
    \end{equation*}
    we have
    \begin{equation*}
        |y-x_k| \leq |y-x|+|x-x_k|<\frac{\d_{x_k}}{2}+\frac{\d_{x_k}}{2}<\d_{x_k}
    \end{equation*}
    Then this gives us that
    \begin{equation*}
        |f(y)-f(x_k)|<\frac{\e}{2} \text{ and } |f(x)-f(x_k)|<\frac{\e}{2}
    \end{equation*}
    which gives us that $|f(y)-f(x)|<\e$, and so $f$ is uniformly continuous on
     $E$.
\end{proof}

\begin{definition}
    We call a real-valued function $f:E \xrightarrow{} \R$ \textbf{monotone
    increasing} if for all $x,y \in E$,  $f(x) \leq f(y)$ whenever $x \leq y$.
    We call  $f$  \textbf{monoton decreasing} if the function $-f$ is monotone
    increasing.
\end{definition}
