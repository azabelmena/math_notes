%----------------------------------------------------------------------------------------
%	SECTION 1.1
%----------------------------------------------------------------------------------------

\section{Monotonic Functions.}

\begin{definition}
    Let $f$ be a realvalued function on an interval  $(a,b)$. We say that  $f$ is 
    \textbf{monotonically increasing} on $(a,b)$ if  $a<x<y<b$ implies  $f(x) \leq f(y)$. 
    We say that  $f$ is \textbf{monotonically decreasing} on $(a,b)$ if  $a<x<y<b$ 
    implies  $f(y) \leq f(x)$. We say $f$ is \textbf{monotonic} if it is either monotonically 
    increasing or monotonically decreasing.
\end{definition}

\begin{theorem}\label{5.6.1}
    Let $f$ be monotonic on $(a,b)$ then $f(x+)$ and  $f(x-)$ exist at every point of 
    $(a,b)$ and $\sup{f}=f(x-)$ and  $\inf{f}=f(x+)$, and the following hold:
        \begin{enumerate}[label=(\arabic*)]
            \begin{enumerate}[label=(\arabic*)]
                \item If $f$ is monotonically increasing $f(x-) \leq f(x) \leq f(x+)$

                \item If $f$ is monotonically decreasing $f(x+) \leq f(x) \leq f(x-)$
            \end{enumerate}		
        \end{enumerate}
\end{theorem}
\begin{proof}
    We prove only $(1)$, since  $(2)$ is analogous. Suppose that $f$ is monotonically 
    increasing, clearly,  $f$ has an upperbound  $A$ for which  $A \leq f$. Now let  $\epsilon>0$, 
    then there is a  $\delta>0$ for which  $a<x-\delta<x$, and  $A-\epsilon<f(x-\delta) \leq A$.  Then we have 
    $f(x-\delta)<f(t) \leq A$ for all  $x-\delta<t<x$, then we get  $|f(t)-A|<\epsilon$, hence 
    $f(x-)=A$, Similarly, we get  $f(+)=-\inf{f}$. Now since  $\sup{f} \leq f \leq \inf{f}$, 
    we get the desired result.
\end{proof}

\begin{corollary}
    Monotonic functions have no infinite discontinuities.		
\end{corollary}

\begin{theorem}\label{5.6.2}
    Let $f$ be monotonic on  $(a,b)$, then the set of all  points of $(a,b)$ for which 
    $f$ is  discontinuous is atmost countable.
\end{theorem}
\begin{proof}
    Suppose, without loss of generality that $g$ is monotonically increasing, and let  $E$ 
    be the set of all points of  $(a,b)$ for which  $f$ is discontinuous. By the density of 
    $\Q$ in  $\R$, for each $x \in E$ associate $r(x) \in \Q$ such that $f(x+)<f(x)<f(x-)$. 
    Since  $x_1 < x_2$ implies $f(x_1+) \leq f(x_2-)$, then $r(x_1) \neq r(x_2)$, thus 
    $x_1 \neq x_2$, and so $r$ is a 1-1 mapping of  $E$ into  $\Q$.
\end{proof}

Now, given a countable $E$ in an interval  $(a,b)$, we can construct a monotonic function $f$ that 
is discontinuous at every point in  $E$ and continuous everywhere else. Arrange  the points of  
$E$ into a sequence  $\{x_n\}$ and let  $\{c_n\}$ be a sequence such that  $c_n>0$ for 
all  $n \in \Z^+$, such that  $\sum{c_n}$ converges. Define  $f(x)=\sum_{x_n<x}{c_n}$, for 
$x \in (a,b)$. Then we have that 
    \begin{enumerate}[label=(\arabic*)]
        \item $f$ is monotonically increasing on $(a,b)$.
            
        \item $f$ is discontinuous at every point in $E$ with $f(x_n+)-f(x_n-)=c_n$.

        \item $f$ is continuous at every point in $\com{(a,b)}{E}$.
    \end{enumerate}

\begin{definition}
    Let $f$ be a realvalued function defined on an interval $(a,b)$. We say that $f$ is 
    \textbf{continuous form the right} if $f(x+)=f(x)$, and we say $f$ is \textbf{continuous from the 
    left} if $f(x-)=f(x)$.
\end{definition}
