%----------------------------------------------------------------------------------------
%	SECTION 1.1
%----------------------------------------------------------------------------------------

\section{Affine Spaces.}

X\begin{definition}
    We call a subset $X \subseteq \R^n$  \textbf{affine} if for every $x,y \in
    X$, the line  $l(x,y)$ passing through $x$ and $y$ is contained in $X$.
\end{definition}

\begin{lemma}\label{3.1.1}
    Affine sets are convex.
\end{lemma}
\begin{proof}
    Note that the line $l(x,y)$ contains the segment $l[x,y]$ which is in $X$
    for every $x,y \in X$.
\end{proof}

\begin{theorem}\label{3.1.2}
    If $\{X_\alpha\}$ is a collection of affine (or convex) sets in $\R^n$,
    then the intersection of all $X_\alpha$ is affine (or convex) in $\R^n$.
\end{theorem}
\begin{proof}
    Let $X=\bigcap{X_\alpha}$ and let $x,y \in X$. let  $l(x,y)$ be the line
    passing through $x$ and $y$, then $l(x,y) \in X_\alpha$ for every $\alpha$,
    since  $x,y \in X_\alpha$  which is affine. This makes  $l(x,y) \in X$,
    which makes $X$ affine in  $\R^n$. The proof for convexity of  $X$ is the
    same except using the line segment  $l[x,y]$.
\end{proof}

\begin{definition}
    An \textbf{affine combination} of points $x_0, \dots,x_m \in \R^n$ is a
    point $x \in \R^n$ such that
    \begin{equation*}
        x=t_0x_1+\dots+t_mx_m
    \end{equation*}
    Where $\sum{t_i}=1$. A \textbf{convex combination} is an affine combination
    in which each $t_i \geq 0$ for $o \leq i \leq m$.
\end{definition}

\begin{example}\label{}
    The line $tx+(1-t)y$ is a convex combination in $\R^n$.
\end{example}

\begin{definition}
    We say a subset $X \subseteq \R^n$  \textbf{spans} an affine set $[X]$ if
    $[X]$ is the intersection of all affine subsets containing $X$. Similarly,
    we say  $X$  \textbf{spans} a convex set $[X]$ if $[X]$ is the intersection
    of all convex subsets containing $X$. We call these the affine and convex
    \textbf{hulls}, respectively.
\end{definition}

\begin{theorem}\label{3.1.3}
    If $x_0, \dots, x_m \in \R^n$, then the convex hull $[x_0, \dots, x_m]$ is
    the set of all convex combinations of $x_0, \dots, x_m$.
\end{theorem}
\begin{proof}
    Let $S$ be the set of all convex combinations of  $x_0, \dots, x_m$, then
    $[x_0, \dots, x_m] \subseteq S$. Now, let $t_j=1$ and $t_i=0$, then  $x_i
    \in S$ for all  $j$. Moreovoer, let  $\alpha=\sum{a_ix_i}$ and
    $\beta=\sum{b_ix_i}$ where $\sum{a_i}=\sum{b_i}=1$. Then for $t \in [0,1]$
    we have
    \begin{equation*}
        t\alpha+(1-t)\beta=t\sum{a_ix_i}+(1-t)\sum{b_ix_i}=\sum{(t(a_ix_i)+(1-t)b_ix_i)}
    \end{equation*}
    moreover, $t\sum{a_i}+(1-t)\sum{b_i}=1$ and $ta_i+(1-t)b_i \geq 0$ for all
    $0 \leq i \leq m$, so  $t\alpha+(1-t)\beta$ is a convex combination in $S$.

    Now, let  $X$ be any convex set containing  $\{x_0, \dots, x_m\}$. By
    induction on $m$, for  $m=0$,  $S=\{x_0\}$. Now let $m \geq 0$ and  $t_i
    \geq 0$ with  $\sum{t_i}=1$. Assume without loss of generality that $t_0
    \neq 1$. Then
    \begin{equation*}
        y=(\frac{t_1}{1-t_0})x_0+\dots+(\frac{t_m}{1-t_0})x_m \in X
    \end{equation*}
    which makes $x=t_0x_0+(1-t_0)y \in X$ This makes $S \subseteq [x_0, \dots,
    x_m]$.
\end{proof}

\begin{definition}
    We call points $x_0, \dots, x_m \in \R^n$ \textbf{affinely independent} if
    $\{x_1-x_0, \dots, x_m-x_0\}$ is linearly independent in $\R^n$ as a vector
    space.
\end{definition}
