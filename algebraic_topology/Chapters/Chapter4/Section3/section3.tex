%----------------------------------------------------------------------------------------
%	SECTION 1.1
%----------------------------------------------------------------------------------------

\section{Components and Local Connectedness.}

\begin{proposition}\label{3.3.1}
    Let $\sim$ be a relation defined on the topological space  $X$ by:  $x \sim y$ if there exists a
    connected subspace containing both  $x$ and  $y$. Then $\sim$ is an equivalence relation on
    $X$.
\end{proposition}
\begin{proof}
    Clearly, $x \sim x$. Now suppose that  $x \sim y$, then there is a connected subspace containing
    both  $x$ and  $y$, by definition,  $y \sim x$. Now suppose that  $x \sim y$ and  $y \sim z$.
    Then there are connected subspaces  $U$ and  $V$ with  $x,y \in U$,  $y,z \in V$. Since $y \in U
    \cap V$, by theorem \ref{3.1.4}, $U \cup V$ is a connected subspace with  $x,z \in U \cup V$.
    That is  $x \sim z$.
\end{proof}

\begin{definition}
    Let $X$ be a topological space. Define an equivalence relation  $\sim$ on  $X$ by taking  $x
    \sim y$ if there is a onnected subspace of  $X$ containing  $x$ and  $y$. We call the
    equivalence classes of  $X/\sim$  \textbf{connected components} (or \textbf{components}) of
    $X$.
\end{definition}

The following theorem gives us the equivalence classes of $X$.

\begin{theorem}\label{3.3.2}
    The components of $X$ are disjoint nonempty subspaces of  $X$, whose union is  $X$ such that
    each nonempty connected subspace of  $X$ intersects only one of them.
\end{theorem}
\begin{proof}
    Since components are equivalence classes of $X$ under $\sim$ (as defined in proposition
    \ref{3.3.1}), they partition $X$. Then they are disoint, and their union is $X$.

    Now let  $A$ be a connected subspace intersecting at least two components  $C_1$ and $C_2$ at
    elements $x_1$ and $ x_2$ respectively. Then $x_1,x_2 \in A$; by definition of $\sim$,  $x_1
    \sim x_2$. This makes $C_1=C_2$ since they are equivalence classes.

    Now let $C$ be a component with  $ x_0 \in C$ then for all $x \in C$, $ x_0 \sim x$, then there
    is a connected subspace $A_x$ of $X$ with  $x_0,x \in A_x$. Then $A \subseteq C$, moreover
        \begin{equation*}
            C=\bigcup_{x \in C}{A_x}
        \end{equation*}
    Since $A_x$ is connected for each  $x$, and  $ x_0 \in \bigcap{A_x}$, by theorem \ref{3.1.4},
    $C$ is connected.
\end{proof}
\begin{corollary}
    If $X$ is a connected topological space, then it has only one component.
\end{corollary}
\begin{proof}
    Notice that $X$ is a subspace of itself. So if $X$ intersects components  $C_1$ and $C_2$ at
    points  $x_1$ and  $x_2$ respectively, then $C_1=C_2$.
\end{proof}

\begin{lemma}\label{3.3.3}
    Components of a topological space are closed.
\end{lemma}
\begin{proof}
    Let $C$ be a component of a topological space  $X$. Since  $C$ is connected,  $\cl{C}$ is
    connected, and is a subspace of $X$ containing (any) $x,y \in C$ hence by definition $cl{C}
    \subseteq C$.
\end{proof}

\begin{example}
    Consider $\Q$ as a subspace of  $\R$ under  $|\cdot|$.  $\Q$ is Hausdorff, so consider
    a subspace  $A$ of  $\Q$ with atleast two elements  $q_1,q_2$ with $ q_1<q_2$.
    Since $\com{\R}{\Q}$ is dense in $\R$, then there exists an
    $i \in \com{\R}{\Q}$ such that $q_1<i<q_2$. Now define $U=(-\infty,i)$
    and $V=(i,\infty)$. We have $ q_1 \in U$, $ q_2 \in V$, $U \cap V=\emptyset$
    and  $A=U \cup V$, that is  $U$ and  $V$ form a seperation of $A$. Thus if
    $A$ is to be connected,  $A$ must be a singleton. That is all the
    components of  $\Q$ are singletons.

    Now if  $A$ was an open component in  $\Q$, then the subspace topology on  $\Q$ would be
    $2^{\Q}$. Now notice that the sequence $\{\frac{1}{n}\}_{\Z^+}$ has a limit point at $0$
    in the topology induced by  $|\cdot|$, but it does not have a limit point at  $0$ under
    $2^{\Q}$. So none of the components of $\Q$ are open.
\end{example}

We also define equivalence classes of a topological space using path connectedness.

\begin{proposition}\label{3.3.4}
    Let $X$ be a topological space and define $\sim$ on $X$ by $x \sim y$ if  there is an $xy$-path
    in  $X$. Then  $\sim$ is an equivalence relation on $X$.
\end{proposition}
\begin{proof}
    The function $f:[a,b] \rightarrow X$ defined by $f:t \rightarrow t$ defines a path from $x$ to
    $x$, so  $x \sim x$. Now suppose  $x \sim y$, then there is a path  $f:[0,1] \rightarrow X$ from
    $x$ to  $y$. Define  $g:[0,1] \rightarrow X$ by $g(t)=f(1-t)$. Since $f$ is a path, $g$ is
    continuous, and $g(0)=f(1)=y$ and $g(1)=f(0)=x$, so $g$ is a path from  $y$ to  $x$ and  $y \sim
    x$. Lastly, suppose  $x \sim y$ and  $y \sim z$. Then there are paths  $f:[0,1] \rightarrow X$
    and $g:[1,2] \rightarrow X$ from $x$ to  $y$ and  $y$ to  $z$ respectively.  Since
    $f(1)=g(1)=y$ for $\{1\}=[0,1] \cap [1,2]$, and $f$ and  $g$ are continuous; construct, by the
    pasting lemma, a continuous function $h:[0,2] \rightarrow X$. Then $h(0)=f(0)=x$ and
    $h(2)=g(2)=z$, hence $h$ is a path from $x$ to  $z$. That is  $x \sim z$.
\end{proof}

\begin{definition}
    Let $X$ be a topological space and  $\sim$ the equivalence relation defined by  $x \sim y$ if
    there is an  $xy$-path in  $X$. Then the equivalence classes of  $X$ are called \textbf{path
    components}.
\end{definition}

\begin{theorem}\label{3.3.5}
    The path components of a topological space $X$ are disjoint path connected subspaces of $X$
    whose union is $X$, such that each nonempty path connected subspace of $X$ intersects only one
    of them.
\end{theorem}
\begin{proof}
    That path components are disjoint and form $X$ in union is guaranteed by them being equivalence
    classes. Now let  $P_1$ and $P_2$ be path components and let $A$ be a path connected subspace
    intersecting  $ P_1$ and $ P_2$ at $ x_1$ and $x_2$ respectivley. Then $ x_1,x_2 \in A$, and since
    $A$ is path connected, $ x_1 \sim x_2$, therefore $P_1=P_2$.

    Now let $P$ be a path component and choose $ x_0 \in P$. Then for all $x \in P$,  $ x_0 \sim x$,
    that is there is a path connected subspace $A_x$ with  $ x_0,x \in X$. This make $A_x \subseteq
    P$. Since there exists a path from  $ x_0$ to $x$, both of which are in  $P$, this makes  $P$
    path connected.
\end{proof}
\begin{corollary}
    If $X$ is path connected, then it has only one path component.
\end{corollary}

\begin{example}
    The topologist's sine curve, $\cl{S}$, with $S=\{x \times \sin{\frac{1}{x}}: 0 < x \leq 1\}$ is
    connected, and hence only has one component. However, $\cl{S}$ is not path connected, notice
    that $S$ is a path component, and  $0 \times [-1,1]$ is another path component. So $\cl{S}$ has
    two path components. Here $S$ is open, in  $\cl{S}$ (for otherwise, $\cl{S}=S$ which cannot
    happen), and $0 \times [-1,1]$ is closed in $\cl{S}$.

    Now construct the space $S'$ from  $\cl{S}$ by deleting all points with rational second
    coordinates in $[-1,1]$. Then $S'$ has one component, but uncountably many path components;
    since countably many points were deleted and leaving only those points with second coordinat in
    $\com{[-1,1]}{\Q}$ (which is homeomorphic to $\com{\R}{\Q}$, which is uncountable).
\end{example}

We now define connectedness in the local.

\begin{definition}
    A topological space is \textbf{locally connected} at $x$ if for every neighborhood  $U$ of  $x$,
    there is a connected neighborhood  $V$ with $x \in V \subseteq U$. We call  $X$  \textbf{locally
    connected} if it is locally connected at every point.
\end{definition}

\begin{definition}
    A topological space is \textbf{locally path connected} at $x$ if for every neighborhood  $U$ of  $x$,
    there is a path connected neighborhood  $V$ with $x \in V \subseteq U$. We call  $X$  \textbf{locally
    path connected} if it is locally path connected at every point.
\end{definition}

\begin{theorem}\label{3.3.6}
    A topological space $X$ is locally connected if and only if for every open set $U$ of $X$, each
    component of  $U$ is open in  $X$.
\end{theorem}
\begin{proof}
    Suppose that $X$ is locally connected, and let  $U$ be open in  $X$; and let  $C$ be a component
    of  $U$. If  $x \in C$, choose a neighborhood $V$ in $X$ with $x \in V \subseteq U$, since  $V$ is
    connected,  $V \subseteq C$, thus  $C$ is open in  $X$.

    Now suppose that for every component  $C$ of  $U$, that  $C$ is open in  $X$. Let  $x \in X$,
    and a neighborhood  $U$ of  $x$, and let  $C$ be a component of  $U$ with  $x \in C$. Then we
    see that $C$ is a connected neighborhood with  $x \in C \subseteq U$. Therefore  $X$ is locally
    connected.
\end{proof}

\begin{theorem}\label{3.3.7}
    A topological space $X$ is locally path connected if and only if for every open set $U$ of $X$, each
    path component of  $U$ is open in  $X$.
\end{theorem}
\begin{proof}
    Suppose that $X$ is locally path connected, and let  $U$ be open in  $X$; and let  $P$ be a path component
    of  $U$. If  $x \in P$, choose a path connected neighborhood $V$ in $X$ with $x \in V \subseteq U$, since  $V$ is
    path connected,  $V \subseteq P$, thus  $P$ is open in  $X$.

    Now suppose that for every path component  $P$ of  $U$, that  $P$ is open in  $X$. Let  $x \in X$,
    and a neighborhood  $U$ of  $x$, and let  $P$ be a path component of  $U$ with  $x \in P$. Then we
    see that $P$ is a path connected neighborhood with  $x \in P \subseteq U$. Therefore  $X$ is locally
    path connected.
\end{proof}

\begin{theorem}\label{3.3.8}
    If $X$ is a topological space, each path component of  $X$ lies in a component of $X$. If $X$ is
    locally path connected, then path components are components.
\end{theorem}
\begin{proof}
    Since path components are path connected, they are connected, and hence must lie within
    componenpath components are path connected, they are connected, and hence must lie within
    components.

    Now let $X$ be locally path connected and  $P$ a path component, and  $C$ a component of  $X$
    sucht that  $P \subseteq C$, but  $P \neq C$. Define  $Q=\bigcup{P'}$, where each $P' \neq P$ is
    a path component intersecting  $C$, then  $P' \subseteq C$. So  $C=P \cup Q$. Since  $X$ is
    locally path connected, each path component is open in  $X$, thus  $P,Q \neq \emptyset$ are
    open, and disjoint, hence they form a seperation of $C$, which contradicts its connectedness.
    Therefore  $P=C$.
\end{proof}
