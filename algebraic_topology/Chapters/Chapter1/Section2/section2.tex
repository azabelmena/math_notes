%----------------------------------------------------------------------------------------
%	SECTION 1.2
%----------------------------------------------------------------------------------------

\section{Commutative Diagrams and Congruences.}

\begin{definition}
    A \textbf{diagram} in a category is a directed graph with its vertex set a
    subset of objects, and whose edge set are morphisms between those objects.
    We call a diagram \textbf{commutative} if for any pair of objects, every
    pair of morphisms between those objects are equal. That is if $A,A'$ and
    $B,B'$ are pairs of objects with pairs of morphisms  $f:A \xrightarrow{}
    B$, $f:A \xrightarrow{} A'$ and $f':A' \xrightarrow{} B'$, $g':B
    \xrightarrow{} B'$ we have that $g \circ f'=f \circ g'$
            \[\begin{tikzcd}
                A &&& {A'} \\
                \\
                \\
                B &&& {B'}
                \arrow["g", from=1-1, to=1-4]
                \arrow["{f'}", from=1-4, to=4-4]
                \arrow["f"', from=1-1, to=4-1]
                \arrow["g"', from=4-1, to=4-4]
            \end{tikzcd}\]
\end{definition}

\begin{definition}
    A \textbf{congruence} on a category $\Cc$ is an equivalence relation $\sim$
    on morphisms in  $\Cc$ such that:
    \begin{enumerate}
        \item[(1)] If $f \in \Hom{(A,B)}$, and $f \sim f'$, then $f' \in
            \Hom{(A,B)}$.

        \item[(2)] If $f \sim g$ and  $f' \sim g'$, then  $g \circ f \sim g'
            \circ f'$.
    \end{enumerate}
\end{definition}

\begin{figure}[h]
    \centering
    \[\begin{tikzcd}
	&&&& \textcolor{rgb,255:red,167;green,42;blue,42}{[f]} \\
	\\
	A &&&&& B \\
	\\
	& \textcolor{rgb,255:red,42;green,167;blue,42}{[g]}
	\arrow["{f_0}"{description}, shift left=5, color={rgb,255:red,167;green,42;blue,42}, curve={height=-30pt}, from=3-1, to=3-6]
	\arrow["{f_1}"{description}, shift left=4, color={rgb,255:red,167;green,42;blue,42}, curve={height=-24pt}, from=3-1, to=3-6]
	\arrow["{f_2}"{description}, shift left=3, color={rgb,255:red,167;green,42;blue,42}, curve={height=-18pt}, from=3-1, to=3-6]
	\arrow["{f_3}"{description}, shift left=2, color={rgb,255:red,167;green,42;blue,42}, curve={height=-12pt}, from=3-1, to=3-6]
	\arrow["{g_0}"{description}, shift right=5, color={rgb,255:red,42;green,167;blue,42}, curve={height=30pt}, from=3-1, to=3-6]
	\arrow["{g_1}"{description}, shift right=4, color={rgb,255:red,42;green,167;blue,42}, curve={height=24pt}, from=3-1, to=3-6]
	\arrow["{g_2}"{description}, shift right=3, color={rgb,255:red,42;green,167;blue,42}, curve={height=18pt}, from=3-1, to=3-6]
	\arrow["{g_3}"{description}, shift right=2, color={rgb,255:red,42;green,167;blue,42}, curve={height=12pt}, from=3-1, to=3-6]
	\arrow["{f_4}"{description}, shift left=1, color={rgb,255:red,200;green,50;blue,50}, curve={height=-6pt}, from=3-1, to=3-6]
	\arrow["{g_4}"{description}, shift right=1, color={rgb,255:red,42;green,167;blue,42}, curve={height=6pt}, from=3-1, to=3-6]
\end{tikzcd}\]
    \caption{An equivalence relation between morphisms.}
    \label{fig_1.1}
\end{figure}

\begin{theorem}\label{1.2.1}
    Let $\Cc$ be a category with congruence  $\sim$. Define
    $\faktor{\Cc}{\sim}$ as follows:
    \begin{enumerate}
        \item[(1)] $\obj{\faktor{\Cc}{\sim}}=\obj{\Cc}$.

        \item[(2)] $\Hom_{\faktor{\Cc}{\sim}}{(A,B)}=\{[f] : f \in
            \Hom_\Cc{(A,B)}\}$.

        \item[(3)] $[g] \circ [f]=[g \circ f]$
    \end{enumerate}
    Then $\faktor{\Cc}{\sim}$ is a category.
\end{theorem}
\begin{proof}
    We have by equivalence that $\obj{\faktor{\Cc}{\sim}}$ is a class. Moreover,
    since $\sim$ partitions $\Cc$, it partions all of the $\Hom{(A,B)}$ for each
    $A,B$. So each $\Hom{(A,B)}$ is a set, moreover, they are pariwise disjoint
    by definition of $\sim$. Now, notice that by hypothesis, composition in
    $\faktor{\Cc}{\sim}$ is well defined, so $[1_A] \circ [f]=[1_A \circ f]=[f]$
    and $[g] \circ [1_A]=[g \circ 1_A]=[g]$. This makes $\faktor{\Cc}{\sim}$ a
    category.
\end{proof}
\begin{remark}
    On can think of the cateegory $\faktor{\Cc}{\sim}$ as taking all morphisms
    with they same domain and codomain, and collapsing them into a single
    morphism.
\end{remark}

\begin{definition}
    Let $\Cc$ be a catogory and $\sim$ a congruence of $\Cc$. We call the
    category $\faktor{\Cc}{\sim}$ induced by $\sim$ the  \textbf{quotient
    category}.
\end{definition}

\begin{figure}[h]
    \centering
    \[\begin{tikzcd}
	A &&&&& B
	\arrow["{[f]}", color={rgb,255:red,167;green,42;blue,42}, curve={height=-30pt}, from=1-1, to=1-6]
	\arrow["{[g]}"', color={rgb,255:red,42;green,167;blue,42}, curve={height=30pt}, from=1-1, to=1-6]
\end{tikzcd}\]
    \caption{Morphisms in a category with the same domain and codomain get
    collapsed onto a single morphism in the correspinding quotient category.}
    \label{fig_1.2}
\end{figure}
