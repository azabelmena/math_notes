%----------------------------------------------------------------------------------------
%	SECTION 1.3
%----------------------------------------------------------------------------------------

\section{Functors.}

\begin{definition}
    Let $\Ac$ and  $\Cc$ be categories. We deine a \textbf{covariant functor} to
    be a map $F:\Ac \xrightarrow{} \Cc$ such that:
    \begin{enumerate}
        \item[(1)] $A \in \obj{\Ac}$ implies $F(A) \in \obj{\Cc}$.

        \item[(2)] If $f:A \xrightarrow{} B$ is a morphism in $\Ac$, then
            $F(f):F(A) \xrightarrow{} F(B)$ is a morphism in $\Cc$.

        \item[(3)] For all morphisms $f$ and  $g$ in  $\Ac$, for which  $g \circ
            f$ is defined, we have that  $F(g \circ f)=F(g) \circ F(f)$, and
            $F(1_A)=1_{F(A)}$.
    \end{enumerate}
\end{definition}

\begin{figure}[h]
    \centering
    \[\begin{tikzcd}
	A & A &&& B && {F(A)} & {F(A)} &&& {F(B)} \\
	\\
	\\
	& C &&&&&& {F(C)}
	\arrow["{1_A}"', from=1-2, to=1-1]
	\arrow["g", from=1-2, to=1-5]
	\arrow["f", from=4-2, to=1-2]
	\arrow[""{name=0, anchor=center, inner sep=0}, "{g \circ f}"', from=4-2, to=1-5]
	\arrow["{1_{F(A)}}"', from=1-8, to=1-7]
	\arrow["{F(g)}", from=1-8, to=1-11]
	\arrow[""{name=1, anchor=center, inner sep=0}, "{GFf)}", from=4-8, to=1-8]
	\arrow["{F(g \circ f)}"', from=4-8, to=1-11]
	\arrow["F"', curve={height=-18pt}, shorten <=31pt, shorten >=31pt, Rightarrow, from=0, to=1]
\end{tikzcd}\]
    \caption{A covariant functor taking a diagram in on category to a diagram in
    the other.}
    \label{fig_1.3}
\end{figure}

\begin{example}\label{1.3}
    \begin{enumerate}
        \item[(1)] We define the \textbf{forgetful functor} the map $F:\Cc
            \xrightarrow{} \Set$ that takes all objects in $\Cc$ to their
            underlying sets, and morphisms in $\Cc$ to themselves considered as
            functions under the usual composition. For example the forgetful
            functor  $F:\Top \xrightarrow{} \Set$ takes topological spaces  $X$
            to their underlying sets, and continuous maps to themselves,
            considered just as functions.

        \item[(2)] The \textbf{identity functor} is the functor $I:\Cc
            \xrightarrow{} \Cc$ that takes objects and morphisms in $\Cc$ to
            themselves.

        \item[(3)] Let $M$ be a topological space. Define $F_M:\Top
            \xrightarrow{} \Top$ by $F_M:X \xrightarrow{} X \times M$, and for
            each continuous map $f:X \xrightarrow{} Y$, $F(f):X \times M
            \xrightarrow{} Y \times M$ is defined by $(x,m) \xrightarrow{}
            (f(x),m)$. Then $F_M$ is a functor.

        \item[(4)] Let $A \in \obj{\Cc}$ and take the map $\Hom{(A,\ast)}:\Cc
            \xrightarrow{} \Set$ that takes $A \xrightarrow{} \Hom{(A,B)}$ and
            for each morphism $f:B \xrightarrow{} B'$, $\Hom{(A,f)}:\Hom{(A,B)}
            \xrightarrow{} \Hom{(A,B')}$ is given by $g \xrightarrow{} f \circ
            g$. With call this functor the \textbf{covariant Hom functor}, and
            denote it $f_*$.
    \end{enumerate}
\end{example}

\begin{definition}
    Let $\Ac$ and  $\Cc$ be categories. We deine a \textbf{contravariant functor}
    to be a map $G:\Ac \xrightarrow{} \Cc$ such that:
    \begin{enumerate}
        \item[(1)] $A \in \obj{\Ac}$ implies $G(A) \in \obj{\Cc}$.

        \item[(2)] If $f:A \xrightarrow{} B$ is a morphism in $\Ac$, then
            $G(f):G(B) \xrightarrow{} G(A)$ is a morphism in $\Cc$.

        \item[(3)] For all morphisms $f$ and  $g$ in  $\Ac$, for which  $g \circ
            f$ is defined, we have that  $G(g \circ f)=G(f) \circ G(g)$, and
            $G(1_A)=1_{G(A)}$.
    \end{enumerate}
\end{definition}

\begin{figure}[h]
    \centering
    \[\begin{tikzcd}
	A & A &&& B && {G(A)} & {G(A)} &&& {G(B)} \\
	\\
	\\
	& C &&&&&& {G(C)}
	\arrow["{1_A}"', from=1-2, to=1-1]
	\arrow["g", from=1-2, to=1-5]
	\arrow["f", from=4-2, to=1-2]
	\arrow[""{name=0, anchor=center, inner sep=0}, "{g \circ f}"', from=4-2, to=1-5]
	\arrow["{1_{G(A)}}", from=1-7, to=1-8]
	\arrow["{G(g)}"', from=1-11, to=1-8]
	\arrow[""{name=1, anchor=center, inner sep=0}, "{G(f)}"', from=1-8, to=4-8]
	\arrow["{G(g \circ f)}", from=1-11, to=4-8]
	\arrow["G"', curve={height=-18pt}, shorten <=30pt, shorten >=30pt, Rightarrow, from=0, to=1]
\end{tikzcd}\]
    \caption{A contravariant functor taking a diagram in on category to a diagram
        in the other.}
    \label{fig_1.4}
\end{figure}

\begin{example}\label{1.4}
    \begin{enumerate}
        \item[(1)] Let $F$ be a field, and  $\Vect$ the category of all finite
            dimensional vector spaces over $F$, whose morphisms are linear
            transformations. Define the map  $T:\Vect \xrightarrow{} \Vect$ by
            taking $T:V \xrightarrow{} V^\perp$, and $T:f \xrightarrow{} f^T$.
            That is $T$ takes vector spaces to their dual spaces, and linear
            transformation to their transpose. $T$ is a contravariant functor
            called the \textbf{dual space functor.}

        \item[(2)] Define $\Hom{(\ast,B)}:\Cc \xrightarrow{} \Cc$ by taking
            $\Hom{(\ast,B)}:A \xrightarrow{} \Hom{(A,B)}$ and for each morphism
            $g:A \xrightarrow{} A'$ in $\Cc$, $\Hom{(f,B)}:\Hom{(A',B)}
            \xrightarrow{} \Hom{(A,B)}$ is defined by taking $h \xrightarrow{} h
            \circ g$. This is analogous to the covariant $\Hom$ functor, and we
            call it the  \textbf{contravariant Hom functor.}
    \end{enumerate}
\end{example}

\begin{definition}
    We call a morphism $f:A \xrightarrow{} B$ an \textbf{equivalence} if there
    exists a morphism $g:B \xrightarrow{} A$ such taht $f \circ g=1_B$ and  $g
    \circ f=1_A$
\end{definition}

\begin{theorem}\label{1.3.1}
    Let $\Ac$ and  $\Cc$ be categories, and $F:\Ac \xrightarrow{} \Cc$ be a
    functor. If $f$ is an equivalence in  $\Ac$, then  $F(f)$ is an equivalence
    in $\Cc$.
\end{theorem}
\begin{proof}
    Suppose that $F$ is a covariant functor. Notice that if  $f:A \xrightarrow{}
    B$ is an equivalence, then there is a $g:B \xrightarrow{} A$ with $f \circ
    g=1_B$ and  $g \circ f=1_A$. Then  $F(f \circ g)=F(f) \circ
    F(g)=F(1_B)=1_{F(B)}$, and $F(g \circ f)=F(g) \circ F(f)=F(1_A)=1_{F(A)}$.

    Likewise, if $F$ is contravariant, notice that  $F(f):B \xrightarrow{} A$
    and $F(g):A \xrightarrow{} B$. Then $F(f \circ g)=F(g) \circ F(f)=1_{F(A)}$,
    and $F(g \circ f)=F(f) \circ F(g)=1_{F(B)}$. In eithe case, we find that
    $F(f)$ is an equivalence in $\Cc$.
\end{proof}
