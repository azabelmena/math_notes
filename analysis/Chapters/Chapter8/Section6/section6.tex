\section{The Cantor Set}

\begin{definition}
    We define the \textbf{cantor set} to be the set
    \begin{equation*}
        \Cc=\bigcap_{n \in \Z^+}{C_n}
    \end{equation*}
    Where $\{C_n\}$ is a monotonically decreasing countable collection such that
    $C_n$ is the disjoint union of  $2^n$ intervals each of length
    $\frac{1}{3^n}$.
\end{definition}

\begin{theorem}\label{8.6.1}
    The Cantor set is closed, uncountable, and of measure $0$.
\end{theorem}
\begin{proof}
    By the closure of intersections, $\Cc$ is closed in  $\R$ under its usual
    topology. Moreover, notice that each  $C_n$ is measurable, hence so is
    $\Cc$. Now, since  $C_n$ is measurable for each  $n$, we have that
    $m(C_n)=(\frac{2}{3})^n$ by countable additivity. Since $\Cc \subetseq C_n$,
    we get  $m(\Cc) \leq m(C_n)=(\frac{2}{3})^n$. So as $n \xrightarrow{}
    \infty$ we see that $m(\Cc)=0$.

    Now, suppose that $\Cc$ is countable and let  $\Cc=\{c_n\}_{n \in \Z^+}$ an
    enumeration. Consider $F_1$ one of the $2$ intervals of  $C_1$ not
    containing $c_1$. Let $F_2$ be one of the $2^2=4$ intervals of  $C_2$ not
    containing  $c_2$, and proceeding inductively define the collection
    $\{F_n\}$ such that:
    \begin{enumerate}
        \item[(1)] $F_n$ is closed and  $\{F_n\}$ is monotonically decreasing.

        \item[(2)] $F_n \subseteq C_n$

        \item[(3)] $c_n \notin C_n$
    \end{enumerate}
    By the nested interval theorem, we get $\bigcap{F_n}$ is nonempty. Now let
    $x \in \bigcap{F_n}$, then by definition $x \in \bigcap{C_n}=\Cc$, which
    makes $x=c_n$ for some  $n \in \Z^+$. This makes  $c_n \in \bigcap{F_n}$,
    and hence $c_n \in F_n$ which cannot happen. Therefore  $\Cc$ cannot be
    countable.
\end{proof}

\begin{definition}
    Let $\Oo=\bigcap{\Oo_n}$ where $\Oo_n$ is the union of the first  $2^n-1$
    intervals removed after $n$ Cantor deletions. Then $\Cc=\com{[0,1]}{\Oo}$.
    Define $\phi$ on $\Oo_n$ the increasing function which is constant on each
    of its $2^n-1$ open intervals, and takes the  $2^n-1$ values
    \begin{align*}
        \frac{1}{2^n}   &&  \dots   &&   \frac{2^n-1}{2^n}
    \end{align*}

    Now domain extend $\phi$ to $\Cc$ by the rule:
    \begin{equation*}
        \phi(0)=0 \text{ and } \phi(x)=\sup{\{\phi(t) : t \in \Oo \cap [0,x)\}}
        \text{ if } x \in \com{\Cc}{0}
    \end{equation*}
    We call $\phi:[0,1] \xrightarrow{} [0,1]$ the \textbf{Cantor-Lebesgue
    function}.
\end{definition}

\begin{theorem}\label{8.6.2}
    The Cantor-Lebesgue function is an increasing continuous function taking
    $[0,1]$ onto $[0,1]$. Moreover its derivative on the open set
    $\Oo=\com{\Cc}{[0,1]}$ exists with
    \begin{equation*}
        \phi'=0 \text{ on } \Oo \text{ if } m(\Oo)=1
    \end{equation*}
\end{theorem}
\begin{proof}
    By definition, $\phi$ is increasing on $\Oo$, so it must be increasing on
    its extension to  $[0,1]$. Moreover $\phi$ is also continuous on  $\Oo$, and
    so must be continuous on  $[0,1]$ by extension as well.

    Now, consider $x_0 \in \Cc$ with $x_0 \neq 0,1$. Then $x$ is not in any of
    the  $2^n-1$ intervals that were removed after  $n$ Cantor deletions. Then
    for  $n$ sufficiently large,  $x$ is inbetween two consecutave intervals in
     $\Oo_n$. Now, choose  $a_n$ in the lower bound of these intervals
     (together) and $b_n$ in the upperbound of these intervals in $\Oo_n$. Then
     by definition of  $\phi$ on  $\Oo_n$, when  $a_n<x_0<b_n$, we get
     \begin{equation*}
         \phi(b_n)-\phi(a_n)=\frac{1}{2^n}
     \end{equation*}
     For $n$ sufficiently large, we then see that  $\phi$ fails to have jump
     discontinuities at  $x_0$. This makes $\phi$ continuous at  $x_0$. Now, if
     $x_0$ is an endpoint of $[0,1]$, we get continuity by similar reasoning.

     Now, since $\phi$ is constant on each  $\Oo_n$, and  $\Oo=\bigcup{\Oo_n}$,
     we get the derivative $\phi'$ exists, and $\phi'=0$ on $\Oo$. Now, since
     $m(\Cc)=0$, we must have that $m(\Oo)=1$. Since $\phi(0)=0$ and
     $\phi(1)=1$, and $\phi$ is increasing,  $\phi$ maps  $[0,1]$ onto itself by
     the intermediate value theorem.
\end{proof}

\begin{theorem}\label{8.6.3}
    Let $\phi$ be the Cantor-Lebesgue function and let $\psi$ on $[0,1]$ be
    defined by
    \begin{equation*}
        \psi(x)=\phi(x)+x \text{ for all } x \in [0,1]
    \end{equation*}
    Then $\psi$ is strictly increasing and continuous mapping  $[0,1]$ onto
    $[0,2]$. Moreover, $\psi$ maps  $\Cc$ to a postively measured set and maps a
    measurable subset of  $\Cc$ to a nonmeasurable set.
\end{theorem}
\begin{proof}
    We have that $\phi$ and $f(x)=x$ are both strictly increasing and
    continuous, hence $\psi=\phi+f$ must also be strictly increasing and
    continuous. Now,  $\psi(0)=\phi(0)+0=0$ and $\psi(1)=\phi(1)+1=2$. Then
    since $\psi$ is alos increasing, by the intermediate value theorem,  $\psi$
    takes  $[0,1]$ onto $[0,2]$.

    Now, let $\Oo=\com{[0,1]}{\Cc}$. Then $[0,1]=\Cc \cup \Oo$ which is a
    disjoint decomposition, and so $[0,2]=\psi(\Cc) \cup \psi(\Oo)$ is also a
    disjoint decomposition. Since $\psi$ is continuous and increasing, its
    inverse is also continuous. This makes  $\psi(\Cc)$ closed and $\psi(\Oo)$
    open; which make them measurable.

    Now, let $\{I_n\}$ be an enumeration of the intervals that were removed from
    $\Cc$. Then  $\Oo=\bigcup{I_n}$ and since $\phi$ is constant on each  $I_n$,
    we get  $\psi$ mapping each  $I_n$ to a translation of itself. Since $\psi$
    is 1--1, the collection  $\{\psi(I_)\}$ is disjoint, with
    \begin{equation*}
        m(\psi(\Oo))=\sum{l(\psi(I_n))}=\sum{l(I_n)}=m(\Oo)
    \end{equation*}
    Since $m(\Cc)=0$, we must have that $m(\psi(\Oo))=1$. This makes
    $m(\psi(\Cc))=1$. Now, by Vitali's theorem, since $m(\psi(\Cc))>0$, then
    $\psi(\Cc)$ contains a nonmeasurable set $W$. Now,  $\inv{\psi}(W)$ is
    measurable with measure $0$. This completes the proof.
\end{proof}
\begin{corollary}
    There is a measurable subset of the Cantor set which is not Borel.
\end{corollary}
