%----------------------------------------------------------------------------------------
%	SECTION 1.1
%----------------------------------------------------------------------------------------

\section{The Borel-Cantelli Lemma}

\begin{definition}
    We define the \textbf{Lebesgue measure} to be the outer measure restricted
    to the $\sigma$-algebra of all measurable sets, and denote it  $m$. That is;
    \begin{equation*}
        m(E)=m^*(E)
    \end{equation*}
    Where $E$ is measurable.
\end{definition}

\begin{lemma}\label{8.4.1}
    The Lebesgue measure is countably additive.
\end{lemma}
\begin{proof}
    Let $\{E_n\}$ be a countable collection of measurable sets. By lemma
    \ref{8.2.6}, we have that $E=\bigcup{E_n}$ is measurable. Therefore, we
    have:
    \begin{equation*}
        m(E) \leq \sum{m(E_n)}
    \end{equation*}
    Consider now the subcollection $\{E_k\}_{k=1}^n$ of $E$. We get that:
    \begin{equation*}
        \sum_{k=1}^n{m(E_k)} \leq m(E) \leq \sum{E_n}
    \end{equation*}
    Now, as $n \xrightarrow{} \infty$, we see that
    \begin{equation*}
        \sum_{k=1}^n{m(E_k)}-\sum{E_n} \xrightarrow{} 0
    \end{equation*}
    So that
    \begin{equation*}
        \sum{E_n} \leq m(E)
    \end{equation*}
    This concludes the proof.
\end{proof}

\begin{theorem}\label{8.4.2}
    The Lebesgue measure is translation invariant and countably additive.
\end{theorem}
\begin{proof}
    Translation invariance follows from outer measure being translation
    invariant, and countable additivity follows from above.
\end{proof}

\begin{theorem}[The Continuity of Lebesgue Measure]\label{8.4.3}
    The following holds for Lebesgue measure:
    \begin{enumerate}
        \item[(1)] If $\{A_n\}$ is a monotonically increasing sequence of
            measurable sets, then
            \begin{equation*}
                m(\bigcup{A_n})=\lim_{n \xrightarrow{} \infty}{m(A_n)}
            \end{equation*}

        \item[(2)] If $\{B_n\}$ is a monotonically decreasing sequence of
            measurable sets, then
            \begin{equation*}
                m(\bigcap{B_n})=\lim_{n \xrightarrow{} \infty}{m(B_n)}
            \end{equation*}
    \end{enumerate}
\end{theorem}
\begin{proof}
    Let $N \in \Z^+$ be an index for which  $m(A_N)$ is infinite. By
    monotonicity, we get that $m(\bigcup{A_n})=m(A_N)=0$ for all $n \geq N$.
    Now suppose that each $m(A_n)$ is finite for all $n \in \Z^+$. Define
    $A_0=\emptyset$ and $C_n=\com{A_n}{A_{n-1}}$ for all $n \geq 1$. Then the
    collection $\{C_n\}$ is a collection of disjoint sets, and
    $\bigcup{C_n}=\bogcup{A_n}$. By the countable additivity of $m$, we get:
    \begin{equation*}
    m(\bigcup{C_n})=m(\bigcup{A_n})=\sum{m(\com{A_n}{A_{n-1}})}
    \end{equation*}
    Since each of the $A_n$ is measurable of finite measure, we have
    \begin{align*}
        \sum{m(\com{A_n}{A_{n-1}})} &=  \sum{m(A_n)-m(A_{n-1})} \\
                                &= \lim_{n \xrightarrow{}
                                \infty}{\sum_{k=1}^n{m(A_k)-m(A_{k-1})}}    \\
                                &=  \lim_{n \xrightarrow{} \infty}{m(A_n)-m(A_0)}   \\
                                &= \lim_{n \xrightarrow{} \infty}{m(A_n)}   \\
    \end{align*}

    Now let $\{B_n\}$ be a monotonically decreasing sequence of measurable sets
    of finite measure. Define $D_n=\com{B_1}{B_n}$ for all $n$. Then  $\{D_n\}$
    is a monotonically increasing sequence of measurable sets of finite measure
    (by excision), which gives us:
    \begin{equation*}
        m(\bigcup{D_n})=\lim_{n \xrightarrow{} \infty}{m(D_n)}
    \end{equation*}
    Now, by DeMorgan's laws, we have $\bigcup{D_n}=\com{B_1}{(\bigcap{B_n})}$,
    with $m(D_n)=m(B_1)-m(B_n)$ for eacgh $n$. This gives us
    \begin{equation*}
        m(\com{B_1}{(\bigcap{B_n})})=m(B_1)-m(\bigcap{B_n})
                                =m(B_1)-\lim_{n \xrightarrow{} \infty}{m(B_n)}
    \end{equation*}
    The result follows by cancellation of the terms.
\end{proof}

\begin{lemma}[The Borel Cantelli Lemma]\label{8.4.4}
    Let $\{E_n\}$ be a countable collection of measurable sets for which
    $\sum{m(E)}$ is finite. Then almost all $x \in \R$ belongs to finitely many
     $E_n$.
\end{lemma}
\begin{proof}
    For each $k \in \Z^+$,, we see thatn  $m(\bigcup_{k=n}{E_n}) \leq
    \sum{m(E_n)}$ is finite. Thus by continuity
    \begin{equation*}
        m(\bigcap_{n=1}{(\bigcup_{k=n}){E_n}})=
        \lim_{n \xrightarrow{} \infty}{m(\bigcup{E_n})}=
        \lim_{n \xrightarrow{} \infty}{m(E_n)}=0
    \end{equation*}
    Then almost all $x \in \R$ fails to be in  $\bigcap{(\bigcup{E(_n)})}$,
    which forces them to be in finitely many $E_n$.
\end{proof}
