%----------------------------------------------------------------------------------------
%	SECTION 1.1
%----------------------------------------------------------------------------------------

\section{Nonmeasurable Sets}

\begin{definition}
    We call a set $E \subseteq \R$  \textbf{nonmeasurable} if $E$ is not a
    measurable set.
\end{definition}

\begin{lemma}\label{8.5.1}
    Let $E \subseteq \R$ be a bounded measurable set, and suppose that there is
    a bounded, countably infinite set  $\Lambda$ for which
    $\{\lambda+E\}_{\lambda \in  \Lambda}$ is a collection of disjoint
    measurable sets. Then $m(E)=0$.
\end{lemma}
\begin{proof}
    By countable additivity, we see that $m(\bigcup_{\lambda \in
    \Lambda}{(\lamba+E)})=\sum_{\lambda \in \Lambda}{m(\lambda+E)}$. Since $E$
    and $\Lambda$ are bounded, so is the set  $\bigcup{(\lambda+E)}$, which
    gives it finite measure.. Now, by translation invariance, we also have that
     $m(\lambda+E)=m(E)>0$ for all $\lambda \in \Lambda$. Since  $\Lambda$ is
     countable, the sum of all $m(\lambda+E)$ is also finite. This makes
     $m(E)=0$.
\end{proof}
\begin{corollary}
   $E$ is measurable.
\end{corollary}

\begin{definition}
    Let $E \subseteq \R$, we call two points  $x,y \in E$  \textbf{rationally
    equivalent} if $|x-y| \in \Q$.
\end{definition}

\begin{lemma}\label{8.5.2}
    Rational equivalence is an equivalence relation.
\end{lemma}
\begin{proof}
    First we have for any $x \in E$,  $|x-|=0 \in \Q$. Now, suppose that  $x$ is
    rationally equivalent to $y \in E$. Then $|x-y|=|y=x| \in \Q$, which makes
    $y$ rationally equivalent to $x$. Lastly, suppose that $x$ is rationally
    equivalent to $y$, and that $y$ is rationally equivalent to $z \in E$. Then
    both $|x-y|,|y-z| \in \Q$. By the triangle inequality, we have  $|x-z| \leq
    |x-y|+|y-z| \in \Q$ which makes  $x$ rationally equivalent to $z$.
\end{proof}

\begin{theorem}[Vitali's Theorem]\label{8.5.3}
    Every set of $E \subseteq \R$ of positive outer measure contains a
    nonmeasurable subset.
\end{theorem}
\begin{proof}
    Let $\faktor{E}{\sim}$ be the set of all equivalence classes of rational
    equivalent points in $E$, Define the set $\Cc_E$ to be the collection of all
    representatives of equivalence classes in  $\faktor{E}{\sim}$ such that:
    \begin{enumerate}
        \item[(1)] No two points of $\Cc_E$ have a rational difference.

        \item[(2)] For every $x \in E$, there is a  $c \in \Cc_E$ for which
            $x=c+q$ where $q \in \Q$.
    \end{enumerate}

    Now, suppose that $E$ has measure  $m(E)>0$ and that $\Cc_E$ is measurable.
    Let  $\Lambda_0 \subseteq \Q$ be any bounded, countably infinite set. Then
    $\{\lambda+E\}_{\lambda \in \Lambda_0}$ is a collection of disjoint
    measurable sets. So we have
    \begin{equation*}
        m(\bigcup_{\lambda \in \Lambda_0}{(\lambda+\Cc)})=
        \sum_{\lambda \in \Lambda_0}{m(\lambda+\Cc_E)}
    \end{equation*}
    Now, choose $\Lambda_0=[-2b,2b] \cap \Q$. Then $\Lambda_0$ is bounded and
    countably infnite, so the above sum holds. Moreover, by the density of $\Q$
   in  $\R$,
    \begin{equation*}
        E \subseteq \bigcup_{\lambda \in \Lambda_0}{(\lamda+\Cc_E)}
    \end{equation*}
    since, if $x \i E$, then there is a  $c \in \Cc_E$ for which  $x=c+q$ and
    $q \in \Q$. Npw,  $x,c \in [-b,b]$, which makes $q \in [-2b,2b]$. This
    contradicts that $m(E)>0$. Therefore $\Cc_E$ cannot be a measurable set.
    That is,  $\Cc_E$ must be nonmeasurable.
\end{proof}
