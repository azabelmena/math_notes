\section{The Lebesgue Integral of a Bounded Measurable Function}

\begin{definition}
    For a simple function $\phi$ on a finitely measurable set $E$, the
    \textbf{integral} of $\phi$ over  $E$ is define to be
    \begin{equation*}
        \int_E{\phi}=\sum_{i=1}^n{a_im(E_i)}
    \end{equation*}
    where $\phi=\sum{a_i\chi_{E_i}}$ and $E=\bigcup{E_i}$ is a disjoint union.
\end{definition}

\begin{lemma}\label{10.1.1}
    Let $\{E_i\}_{i=1}^n$ be a finite disjoint collection of finitely measurable
    sets. Let $a_i \in \R$. Then if $\phi=\sum_{i=1}^n{a_i\chi_{E_i}}$, then
    \begin{equation*}
        \int_E{\phi}=\sum_{i=1}^n{a_im(E_i)}
    \end{equation*}
\end{lemma}
\begin{proof}
    Let $\lambda_1, \dots, \lambda_m$ the distinct values of $\phi$. For  $1
    \leq j \leq m$, take $A_j=\{x \in E : \phi(x)=\lambda_j\}$. Then we have
    \begin{equation*}
        \int_E{\phi}=\sum{\lambda_jm(E_j)}
    \end{equation*}
    Now, let $I_j$ be the set of indices fr which  $a_i=\lambda_j$. Then
    \begin{equation*}
        I=\bigcup{I_j}
    \end{equation*}
    is disjoint, and by additivity, we get
    \begin{equation*}
        m(A_j)=\sum{m(E_j)} \text{ for all } 1 \leq j \leq m
    \end{equation*}
    Therefore
    \begin{equation*}
        \sum{a_im(E_i)}=\sum_{j=1}^m{\sum{a_im(E_i)}}=\sum{\lambda_j\sum{m(E_i)}}
        =\lambda_jm(A_j)=\int_E{\phi}
    \end{equation*}
\end{proof}

\begin{lemma}\label{10.1.2}
    Let $\phi$ and  $\psi$ be simple functions defined on a finitely measurable
    set $E$. Then for any $a,b \in \R$
    \begin{equation*}
        \int_E{a\phi+b\psi}=a\int_E{\phi}+b\int_E{\psi}
    \end{equation*}
    Moreover, if $\phi \leq \psi$, then
    \begin{equation*}
        \int_E{\phi} \leq \int_E{\psi}
    \end{equation*}
\end{lemma}
\begin{proof}
    Let $\{E_i\}_{i=1}^n$ be a finite disjoint collection of finitely measurable
    subsets of $E$, with  $E=\bigcup{E_i}$. Let $\phi$ and  $\psi$ be constant
    on each  $E_i$ for all $1 \leq i \leq n$ and let $a_i$  $b_i$ the values
    taken by  $\phi$ and  $\psi$ on  $E_i$, respectively. Then we have
    \begin{equation*}
        \int_E{\phi}=\sum{a_im(E_i)} \text{ and } \int_E{\psi}=\sum{b_im(E_i)}
    \end{equation*}
    so that
    \begin{equation*}
        \int_E{a\phi_b\psi}=\sum{(aa_i+bb+i)m(E_i)}=a\sum{a_im(E_i)}+b\sum{a_im(E_i)}
        =a\int_E{\phi}+b\int_E{\psi}
    \end{equation*}
    Now, if $\phi \leq \psi$, then  $0 \leq \psi-\phi$ so that
    \begin{equation*}
        0 \leq \int_E{\psi-\phi}=\int_E{\psi}-\int_E{\phi}
    \end{equation*}
    and we are done.
\end{proof}

\begin{definition}
    Let $f$ be a bounded realvalued function on a measurable set $E$. We define
    the \textbf{upper Lebesuge integral} of $f$ to be
    \begin{equation*}
        \bar{\int_E}{f}=\sup{\{\int_E{\phi} : \phi \text{ is simple and } \phi
        \leq f \text{ on } E\}}
    \end{equation*}
    Similarly, we define the \textbf{lower Lebesgue integral} of $f$ on $E$ to
    be
    \begin{equation*}
        \bbar{\int_E}{f}=\sup{\{\int_E{\psi} : \psi \text{ is simple and } f
        \leq \psi \text{ on } E\}}
    \end{equation*}
    We say that $f$ is \textbf{Lebesgue integrable} on $E$ provided that
    \begin{equation*}
        \int_E{f}=\bar{\int_E}{f}=\bbar{\int_E}{f}
    \end{equation*}
    and we call $\int_E{f}$ the \textbf{Lebesgue integral} of $f$ on $E$.
\end{definition}

\begin{theorem}\label{10.1.3}
    If $f$is a bounded function defined on  $[a,b]$ such that $f$ is Riemann
    integrable on  $[a,b]$, then it is Lebesuge integrable on $[a,b]$. Moreover,
    the two integrals coincide.
\end{theorem}
\begin{proof}
    Let us denote $R\int{f}$ to be the Riemann-Stieltjes integral of $f$, and
    let  $I=[a,b]$. If $f$ is Riemann integrable, then we have
    \begin{equation*}
        \bar{R\int_I}{f}=\bbar{R\int_I}{f}
    \end{equation*}
    i.e.
    \begin{equation*}
        \sup{\{R\int_E{\phi} : \phi \text{ is a step function and } \phi \leq f\}}=
        \inf{\{R\int_E{\psi} : \psi \text{ is step function and } f \leq \psi\}}
    \end{equation*}
    Since $\phi$ and  $\psi$ are step functions, they are simple hence we get
    \begin{equation*}
        R\int_I{\phi}=\int_I{\phi} \text{ and } R\int_I{\psi}=\int_I{\psi}
    \end{equation*}
    This makes $R\int{f}=\int{f}$, and hence makes $f$ Lebesgue integrable on
    $[a,b]$.
\end{proof}
