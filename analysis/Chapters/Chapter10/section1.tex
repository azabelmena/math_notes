\section{The Lebesgue Integral of a Bounded Measurable Function}

\begin{definition}
    For a simple function $\phi$ on a finitely measurable set $E$, the
    \textbf{integral} of $\phi$ over  $E$ is define to be
    \begin{equation*}
        \int_E{\phi}=\sum_{i=1}^n{a_im(E_i)}
    \end{equation*}
    where $\phi=\sum{a_i\chi_{E_i}}$ and $E=\bigcup{E_i}$ is a disjoint union.
\end{definition}

\begin{lemma}\label{10.1.1}
    Let $\{E_i\}_{i=1}^n$ be a finite disjoint collection of finitely measurable
    sets. Let $a_i \in \R$. Then if $\phi=\sum_{i=1}^n{a_i\chi_{E_i}}$, then
    \begin{equation*}
        \int_E{\phi}=\sum_{i=1}^n{a_im(E_i)}
    \end{equation*}
\end{lemma}
\begin{proof}
    Let $\lambda_1, \dots, \lambda_m$ the distinct values of $\phi$. For  $1
    \leq j \leq m$, take $A_j=\{x \in E : \phi(x)=\lambda_j\}$. Then we have
    \begin{equation*}
        \int_E{\phi}=\sum{\lambda_jm(E_j)}
    \end{equation*}
    Now, let $I_j$ be the set of indices fr which  $a_i=\lambda_j$. Then
    \begin{equation*}
        I=\bigcup{I_j}
    \end{equation*}
    is disjoint, and by additivity, we get
    \begin{equation*}
        m(A_j)=\sum{m(E_j)} \text{ for all } 1 \leq j \leq m
    \end{equation*}
    Therefore
    \begin{equation*}
        \sum{a_im(E_i)}=\sum_{j=1}^m{\sum{a_im(E_i)}}=\sum{\lambda_j\sum{m(E_i)}}
        =\lambda_jm(A_j)=\int_E{\phi}
    \end{equation*}
\end{proof}

\begin{lemma}\label{10.1.2}
    Let $\phi$ and  $\psi$ be simple functions defined on a finitely measurable
    set $E$. Then for any $a,b \in \R$
    \begin{equation*}
        \int_E{a\phi+b\psi}=a\int_E{\phi}+b\int_E{\psi}
    \end{equation*}
    Moreover, if $\phi \leq \psi$, then
    \begin{equation*}
        \int_E{\phi} \leq \int_E{\psi}
    \end{equation*}
\end{lemma}
\begin{proof}
    Let $\{E_i\}_{i=1}^n$ be a finite disjoint collection of finitely measurable
    subsets of $E$, with  $E=\bigcup{E_i}$. Let $\phi$ and  $\psi$ be constant
    on each  $E_i$ for all $1 \leq i \leq n$ and let $a_i$  $b_i$ the values
    taken by  $\phi$ and  $\psi$ on  $E_i$, respectively. Then we have
    \begin{equation*}
        \int_E{\phi}=\sum{a_im(E_i)} \text{ and } \int_E{\psi}=\sum{b_im(E_i)}
    \end{equation*}
    so that
    \begin{equation*}
        \int_E{a\phi_b\psi}=\sum{(aa_i+bb+i)m(E_i)}=a\sum{a_im(E_i)}+b\sum{a_im(E_i)}
        =a\int_E{\phi}+b\int_E{\psi}
    \end{equation*}
    Now, if $\phi \leq \psi$, then  $0 \leq \psi-\phi$ so that
    \begin{equation*}
        0 \leq \int_E{\psi-\phi}=\int_E{\psi}-\int_E{\phi}
    \end{equation*}
    and we are done.
\end{proof}

\begin{definition}
    Let $f$ be a bounded realvalued function on a measurable set $E$. We define
    the \textbf{upper Lebesuge integral} of $f$ to be
    \begin{equation*}
        \bar{\int_E}{f}=\sup{\{\int_E{\phi} : \phi \text{ is simple and } \phi
        \leq f \text{ on } E\}}
    \end{equation*}
    Similarly, we define the \textbf{lower Lebesgue integral} of $f$ on $E$ to
    be
    \begin{equation*}
        \bbar{\int_E}{f}=\sup{\{\int_E{\psi} : \psi \text{ is simple and } f
        \leq \psi \text{ on } E\}}
    \end{equation*}
    We say that $f$ is \textbf{Lebesgue integrable} on $E$ provided that
    \begin{equation*}
        \int_E{f}=\bar{\int_E}{f}=\bbar{\int_E}{f}
    \end{equation*}
    and we call $\int_E{f}$ the \textbf{Lebesgue integral} of $f$ on $E$.
\end{definition}

\begin{theorem}\label{10.1.3}
    If $f$is a bounded function defined on  $[a,b]$ such that $f$ is Riemann
    integrable on  $[a,b]$, then it is Lebesuge integrable on $[a,b]$. Moreover,
    the two integrals coincide.
\end{theorem}
\begin{proof}
    Let us denote $R\int{f}$ to be the Riemann-Stieltjes integral of $f$, and
    let  $I=[a,b]$. If $f$ is Riemann integrable, then we have
    \begin{equation*}
        \bar{R\int_I}{f}=\bbar{R\int_I}{f}
    \end{equation*}
    i.e.
    \begin{equation*}
        \sup{\{R\int_E{\phi} : \phi \text{ is a step function and } \phi \leq f\}}=
        \inf{\{R\int_E{\psi} : \psi \text{ is step function and } f \leq \psi\}}
    \end{equation*}
    Since $\phi$ and  $\psi$ are step functions, they are simple hence we get
    \begin{equation*}
        R\int_I{\phi}=\int_I{\phi} \text{ and } R\int_I{\psi}=\int_I{\psi}
    \end{equation*}
    This makes $R\int{f}=\int{f}$, and hence makes $f$ Lebesgue integrable on
    $[a,b]$.
\end{proof}

\begin{example}\label{10.1}
    Let $E=\Q \cap [0,1]$, and consider the Dirichlet function $\chi_{[0,1]}$.
    Then $\chi_{[0,1]}$ is Lebesgue integrable on $[0,1]$ and
    \begin{equation*}
        \int_0^1{\chi_{[0,1]}}=\int_0^1{1 \cdot \chi_E}=m(E)=0
    \end{equation*}
\end{example}

\begin{theorem}\label{10.1.4}
    Let $f$ be a bounded measurable function on a finitely measurable set  $E$.
    Then  $f$ is integrable on  $E$.
\end{theorem}
\begin{proof}
    Choose $n \in \Z^+$, and take  $\epsilon=\frac{1}{n}$. By the simple
    approximation lemma, thre exist simple functions $\phi_n$ and $\psi_n$ on
    $E$ such that
    \begin{equation*}
        \phi_n \leq f \leq \psi_n \text{ and } 0 \leq \psi_n-\phi_n<\frac{1}{n}
    \end{equation*}
    By theorem \ref{10.1.2} we have
    \begin{equation*}
        0 \leq \int_E{\psi_n}-\int_E{\phi_n}=\int_E{\psi_n-\phi_n} \leq
        \frac{m(E)}{n}
    \end{equation*}
    Moreover, by definition, we have
    \begin{equation*}
        0 \leq \bbar{\int_E}{f} \leq \bar{\int_E}{f} \leq \int_E{\psi_n-\phi_n}
        \leq \frac{m(E)}{n}
    \end{equation*}
    This makes $f$ measurable.
\end{proof}

\begin{theorem}\label{10.1.5}
    Let $f$ and  $g$ be bounded measurable functions on a finitely measurable
    set. Then for any  $a,b \in \R$ we have
    \begin{equation*}
        \int_E{af+bg}=a\int_E{f}+b\int_E{g}
    \end{equation*}
    Moreover, if $f \leq g$, then
    \begin{equation*}
        \int_E{f} \leq \int_E{g}
    \end{equation*}
\end{theorem}
\begin{proof}
    By above, we have that $f$ and $g$ are bounded and measurable, hence so is
    $af+bg$. Now, let $b=0$. Then for  $a>0$, we have
    \begin{equation*}
        \int_E{af}=\inf_{\psi \geq af}{\int_E{\psi}}
        =a\inf_{\frac{\psi}{a} \geq f}{\int_E{\frac{\psi}{a}}}=a\int_E{f}
    \end{equation*}
    For $a<0$, since  $\int{f}=\bbar{\int}{f}=\bar{\int}{f}$ we get
    \begin{equation*}
        \int_E{af}=\inf_{\phi \geq af}{\int_E{\phi}}
        =a\sup_{\frac{\phi}{a} \geq f}{\int_E{\frac{\phi}{a}}}=a\int_E{f}
    \end{equation*}
    Now, let $a=b=1$, and let  $\psi_1,\psi_2$ be simple functions for which $f
    \leq \psi_1$ and $g \leq \psi_2$. Then $\psi_1+\psi_2$ is simple with $f+g
    \leq \psi_1+\psi_2$ on $E$. Hence  $\int{f+g}=\bar{\int}{f+g}$ and we have
    \begin{equation*}
        \int_E{f+g} \leq \int_E{\psi_1}+\int_E{\psi_2}
    \end{equation*}
    So that
    \begin{equation*}
        \int_E{f+g} \leq \int_E{f}+\int_E{g}
    \end{equation*}
    Now, let $\phi_1, \phi_2$ simple functions such that $\phi_1+\phi_2 \leq
    f+g$ on $E$. Then by similar reasoning, we get
    \begin{equation*}
        \int_E{f}+\int_E{g} \leq \int_E{f+g}
    \end{equation*}

    Lastly, let $h=g-f$, then we get
    \begin{equation*}
        0 \leq \int_E{h}=\int_E{g-f}=\int_E{g}-\int_E{f}
    \end{equation*}
\end{proof}
\begin{corollary}
    Let $f$ be a bounded measurable function on a finitely measurable set. Let
    $A,B \subseteq E$ be disjoint and measurable. Then
    \begin{equation*}
        \int_{A \cup B}{f}=\int_A{f}+\int_B{f}
    \end{equation*}
\end{corollary}
\begin{proof}
    We hace that $f\chi_A$ and  $f\chi_B$ are bounded measurable functions,
    hence so is  $f\chi_A+f\chi_B=f\chi_{A \cup B}$, since $A$ and $B$. Then we
    get
    \begin{equation*}
        \int_{A \cup B}{f}=\int_E{f\chi_{A \cup
        B}}=\int_E{f\chi_A}+\int_E{f\chi_B}=\int_A{f}+\int_B{f}
    \end{equation*}
\end{proof}
\begin{corollary}
    Let $f$ be bounded and measurable on a set of finite measure. Then
    \begin{equation*}
        |\int_E{f}| \leq \int_E{|f|}
    \end{equation*}
\end{corollary}
\begin{proof}
    Notice that $|f|$ is measurable and that  $-|f| \leq f \leq |f|$ on  $E$.
    Then we get
    \begin{equation*}
        -\int_E{|f|} \leq \int_E{f} \leq \int_E{|f|}
    \end{equation*}
\end{proof}

\begin{theorem}\label{10.1.6}
    Let $\{f_n\}$ a sequence of bounded measurable functions on a set of finite
    measure $E$. if  $\{f_n\} \xrightarrow{} f$ uniformly on $E$, then
    \begin{equation*}
        \lim_{n \xrightarrow{} \infty}{\int_E{f_n}}=\int_E{f}
    \end{equation*}
\end{theorem}
\begin{proof}
    Since $\{f_n\} \xrightarrow{} f$ unifomly, and $f_n$ is bounded and
    measurable for each $n$, then  $f$ is bounded and measurable. let
    $\epsilon>0$ and choose $N>0$ such that
    \begin{equation*}
        |f-f_n|<\frac{\epsilon}{m(E)} \text{ on } E \text{ for all } n \geq N
    \end{equation*}
    Then
    \begin{equation*}
        |\int_E{f}-\int_E{f_n}|=|\int_E{f-f_n}| \leq \int_E{|f-f_n|} \leq
        \frac{\epsilon}{m(E)} \cdot m(E)=\epsilon
    \end{equation*}
\end{proof}

\begin{example}\label{10.2}
    For all $n \in \Z^+$, define  $f_n$ on  $[0,1]$ by $0$ if  $x \geq
    \frac{1}{n}$, with $f(\frac{1}{n})=n$, $f(0)=0$ to be linear on the
    intervals $[0,\frac{1}{n}]$ and $[\frac{1}{n}, \frac{2}{n}]$. Observe that
    $\int_o^1{f_n}=1$ for al $n$. Now, let  $f=0$ almost everywhere on  $[0,1]$
    then $\{f_n\} \xrightarrow{} f$ on $[0,1]$ but $\lim{\int_0^1{f}} \neq
    \int_0^1{f}$.
\end{example}

\begin{theorem}[The Bounded Convergence Theorem]\label{10.1.7}
    Let $\{f_n\}$ be a sequence of measurable functions on a finitely
    measurable set $E$, such that there exists an $M>0$ with
    \begin{equation*}
        |f_n| \leq M \text{ on } E \text{ for all } n
    \end{equation*}
    Then ifn $\{f_n\} \xrightarrow{} f$, then
    \begin{equation*}
        \lim_{n \xrightarrow{} \infty}{\int_E{f_n}}=\int_E{f}
    \end{equation*}
\end{theorem}
\begin{proof}
    Since $\{f_n\} \xrightarrow{} f$ pointwise on $E$,  $f$ is measurable. Now,
    we also have $|f| \lweq M$. Let $A \subseteq E$ be measurable and choose  $n
    \in \Z^+$. Then we have
    \begin{equation*}
        \int_E{f_n}=\int_E{f}=\int_E{f_n-f}=
        \int_A{f_n-f}+\int_{\com{E}{A}}{f_n}-\int_{\com{E}{A}}{f}
    \end{equation*}
    Let $\epsilon>0$. Since  $m(E)$ is finite, and $f$ is realvalued, by
    Ergoroff's theorem, there exists a subset  $A \susbeteq E$ for which
    $\{f_n\} \xrightarrow{} f$ unifomrly on $A$ with
    $m(\com{E}{A})<\frac{\epsilon}{4M}$. By uniform convergence, there is an
    $N>0$ for which
    \begin{equation*}
        |f_n-f|<\frac{\epsilon}{2m(E)} \text{ on } A \text{ for all } n \geq N
    \end{equation*}
    hence we get
    \begin{equation*}
        |\int_E{f_n}-\int_E{f}| \leq
        \frac{\epsilon}{2m(E)}m(A)+2Mm(\com{E}{A})M<\epsilon \text{ for all } n
        \geq N
    \end{equation*}
    Therefore $\{\int_E{f_n}\} \xrightarrow{} \int_E{f}$ as $n \xrightarrow{}
    \infty$.
\end{proof}
