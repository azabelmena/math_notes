\section{Limits of Lebesgue Measurable Functions}

\begin{definition}
    For a sequence of functions $\{f_n\}$ with common measurable domain $E$, we
    say that $\{f_n\}$ \textbf{converges pointwise} to a function $f$ on $A
    \subseteq E$ provided that
    \begin{equation*}
        \lim{f_n}=f \text{ for every } x \in A
    \end{equation*}
    We say that $\{f_n\}$ \textbf{converges uniformly} ro $f$ on  $A$ if for
    every $\epsilon>0$, there is an $N>0$ such that
    \begin{equation*}
        |f-f_n|<\epsilon \text{ on } A \text{ whenever } n \geq N
    \end{equation*}
\end{definition}

\begin{lemma}\label{9.2.1}
    Let $\{f_n\}$ be a sequence of measurable functions on $E$, converging
    pointwise almost everywhere to a function $f$ on  $E$. Then  $f$ is
    measurable.
\end{lemma}
\begin{proof}
    Let $E_0 \suubseteq E$ with $m(E_0)=0$. Let $\{f_n\} \xrightarrow{} f$
    pointwise on $\com{E}{E_0}$. Then $f$ is measurable if, and only if
    $f|_{\com{E}{E_0}}$ is measurable. Assuming that $\{f_n\} \xrightarrow{}
    f|_{\com{E}{E_0}}$, on all of $\com{E}{E_0}$, let $c \in \R$. Then since
    $\lim{f_n}=f<c$ if, and only if there are $n,k \in \Z^+$ for whch
    $f_k<c-\frac{1}{n}$ for all $j \geq k$; since  $f_j$ is measurable, then the
    set  $\{x \in E : f_j(x)<c-\frac{1}{n}\}$ is measurable. Thus, the
    intersection of these sets is also measurable. This makes $f$ measurable.
\end{proof}

\begin{definition}
    For $A \subseteq \R$, we define the \textbf{characteristic function} of $A$
    to be the function
    \begin{equation*}
        \chi_A=\begin{cases}
                1, \text{ if } x \in A  \\
                0, \text{ if }, x \notin A  \\
            \end{cases}
    \end{equation*}
\end{definition}

\begin{definition}
    We call a realvalued function $\phi$ on a measurable set $E$ \textbf{simple}
    if it takes only finitely many values. We call the expression
    \begin{equation*}
        \phi=\sum_{i=1}^n{a_i\chi_E_i} \text{ where } E=\bigcup_{i=1}^n{E_i}
    \end{equation*}
    the \textbf{canonical representation} of $\phi$.
\end{definition}

\begin{lemma}[The Simple Approximation Lemma]\label{9.2.2}
    Let $f$ be a measurable bounded function on  $E$. Then for every
    $\epsilon>0$, there exist simple functions  $\phi_\epsilon$ and
    $\psi_\epsilon$, on $E$ such that
    \begin{equation*}
        \phi_\epsilon \leq f \leq \psi_\epsilon \text{ and } 0 \leq
        \psi_\epsilon-\phi_\epsilon<\epsilon
    \end{equation*}
\end{lemma}
\begin{proof}
    Let $(c,d)$ be an open, bounded interval with $f(E) \subseteq (c,d)$.
    Consider the partition $c=y_0<\dots<y_n=d$ of $[c,d]$ such that
    $y_k-y_{k-1}<\epsilon$ for all $1 \leq k \leq n$. Define the interval
    \begin{equation*}
        I_k=[y_{k-1}, y_k)
    \end{equation*}
    and $E_k=\inv{f}(I_k)$. Since $f$ is measurable, each  $E_k$ is measurable
    by definition. Now, define
    \begin{equation*}
        \phi_\epsilon=\sum{y_{k-1}\chi_{E_k}} \text{ and }
        \psi_\epsilon=\sum{y_k\chi_{E_k}}
    \end{equation*}
    and let $x \in E$. Since $f(E) \subseteq (c,d)$, there exists a unique $1
    \leq k \leq n$ such that  $y_{k-1} \leq f(x) \leq y_k$. Therefore we have
    that $\phi_\epsilon=y_{k-1} \leq f \leq y_k=\psi_\epsilon$. Moreover, we get
    $0 \leq y_k-y_{k-1}=\psi_\epsilon-\phi_\epsilon<\epsilon$.
\end{proof}

\begin{theorem}[The Simple Approximation Theorem]\label{9.2.3}
    An extended realfalued function $f$ on a measurable set  $E$ is measurable
    if, and only if there is a sequence $\{\phi_n\}$ of simple functions on $E$
    converging pointwise to $f$ on $E$, such that
    \begin{equation*}
        |\phi_n| \leq |f| \text{ on } E \text{ for all } n
    \end{equation*}
\end{theorem}
\begin{proof}
    Since simple functions are measurable, a function is measurable if it is the
    pointwise limit of a sequence of measurable functions.

    Conversly, suppose that $f$ is measurable, and without loss of generality,
    let  $f \geq 0$ on  $E$. Let  $n \in \Z^+$ and define
    \begin{equation*}
        E_n=\{x \in E : f \leq n\}
    \end{equation*}
    Then each $E_n$ is measurable and forms a disjoint union equal to $E$.
    Moreover,  $f|_{E_n}$ is a nonnegative, bounded measurable function. By the
    simple approximation lemma, let $\epsilon=\frac{1}{n}$, there are simple
    functions $\phi_n$ and  $\psi_n$ such that
    \begin{equation*}
        0 \leq \phi_n \leq f \leq \psi_n \text{ and } 0 \leq \psi_n-\phi_n <
        \frac{1}{n} \text{ on } E_n
    \end{equation*}
    Now, extend $\phi_n$ to $E$ by taking  $\phi_n(x)=n$ if $f(x)>n$. Then
    $\phi_n$ is a simple function with
    \begin{equation*}
        0 \leq \phi_n \leq f \text{ on } E
    \end{equation*}
    Now, if $f$ is finite, choose  $N>0$ such that  $f(x)<N$. Then we get
    \begin{equation*}
        0 \leq f-\phi_n < \frac{1}{n} \text{ for all } n \geq N
    \end{equation*}
    If $f(x)=\infty$ for some $x$, then  $\phi(x)f=n$ for all $n$ so that
    $\lim{\phi_n}=f$
\end{proof}
