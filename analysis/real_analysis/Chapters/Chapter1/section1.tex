\section{Ordered Sets}\label{section_1.1}

\begin{definition}
  Let $S$ be any set. An \textbf{order} on $S$ is a relation  $<$ such that:
  \begin{enumerate}
    \item[(1)] For $x,y \in S$, one and only one of the following hold:
      \begin{align*}
        x<y && x=y && y<x \\
      \end{align*}

    \item[(2)] $<$ is transitive over $S$; that is, for all $x,y,z \in S$ if
      $x<y$ and $y<z$, then $x<z$.
  \end{enumerate}
  We denote the relations $>$ and $\leq$ to mean $x>y$ if and only if $y<x$,
  and $x \leq y$ if and only if $x<y$, or $x=y$. We call $S$ together with  $<$
  an \textbf{ordered set}.
\end{definition}

\begin{example}
  Define $<$  on $\Q$ such that for $r,s \in \Q$, $r<s$ implies $<0s-r$.
\end{example}

\begin{definition}
  Let $S$ be an ordered set, and let  $E \subseteq S$. We say that  $E$ is
  \textbf{bounded above} there is some  $\beta \in S$ for which  $x \leq \beta$,
  for all $x \in E$. We say that $E$ is  \textbf{bounded below} if  $\beta \leq x$,
  for call  $x \in E$. We say an $\alpha \in S$ is a \textbf{least upper bound}
  of  $E$, if  $\alpha$ is an upper bound of  $E$, and for all other upper bounds,
  $\gamma$, of  $E$,  $\alpha \leq \gamma$. Likewise,  $\alpha$ is a \textbf{greatest
  lower bound}  of $E$ if  $\alpha$ is a lower bound of  $E$, and for all other
  lower bounds $\gamma$ of  $E$,  $\gamma \leq \alpha$. We denote the least upper
  bound, and greatest lower bound by  $\sup{E}$ and  $\inf{E}$, respectively.
\end{definition}

\begin{lemma}\label{theorem_1.1.1}
  Let $S$ be an ordered set, and let $E \subseteq S$. Then  $E$ has (if they
  exist) a unique least upper bound, and a unique greatest lower bound.
\end{lemma}
\begin{proof}
  Let $\alpha, \beta \in S$ be least upper bounds of $E$. Then by definition, we
  have that $\alpha \leq \beta$, and  $\beta \leq \alpha$; therefore, $\a=\b$.
\end{proof}

\begin{example}
  \begin{enumerate}
    \item[(1)] Let $A=\{p \in \Q: p^2<2\}$, and $B=\{p \in \Q: p^2>2\}$. Clearly,
      we have that every element of $B$ is an upper bound of $A$, and every
      element of $A$ is a lower bound of  $B$. Now take  $p \in \Q$ a positive
      rational, and take  $q \in \Q$ such that $q=p-\frac{p^2-2}{p+2}$. Then
      $q^2-2=\frac{2(p^2-2)}{(p+2)^2}$. Now if $p \in A$, then  $p^2-2<0$,
      which implies that  $p<q$, and $q^2<2$; thus $A$ has no largest element;
      similarly, if  $p \in B$, then  $p^2-2>0$, which implies that  $q<p$ and
      $q^2>2$, which shows that  $B$ has no least element. Thus $\sup{A}$ and  $\inf{B}$ do not exist in $\Q$.

    \item[(2)] If  $\alpha = \sup{E} \in S$, it may or may not be that  $\alpha
      \in E$. Take $E_1=\{r \in \Q: r<0\}$, and  $E_2=\{r \in \Q: r \leq 0\}$.
      Then $\sup{E_1}=\sup{E_2}=0$, but $0 \not\in E_1$, where as $0 \in E_2$

    \item[(3)] Consider the set $ \frac{1}{\Z^+}=\{\frac{1}{n}: n \in \Z^+\}$. By
      the well ordering principle, $1$ is the least element, and is also an
      upper bound of all  $ \frac{1}{n}$ for $n>1$. Now also notice that as $n$
      gets arbitrarily large, then  $ \frac{1}{n}$ gets arbitratirly small;
      that is to say $\frac{1}{n}$ ``tends'' to $0$, so  $\sup{\frac{1}{\Z^+}}=1
      \in \frac{1}{\Z^+}$, and $\inf{\frac{1}{\Z^+}}=0 \not\in \frac{1}{\Z^+}$.
  \end{enumerate}
\end{example}

\begin{definition}
  We say an ordered set $S$ has the \textbf{least upper bound property}, if
  whenever $E \subseteq S$, nonempty, and bounded above, then  $\sup{E} \in S$
  exists; likewise, $S$ has the \textbf{greatest lower bound property} if
  whenever $E$ is nonempty, bounded below then $\inf{E} \in S$ exists.
\end{definition}

\begin{example}
  \begin{enumerate}
    \item[(1)] The set of all rationals $\Q$ does not have the least upper
      bound property, nor the greatest lower bound property, take $A$,  $B$
      as in the previous example.

    \item[(2)] Let $A \subseteq \R$ be nonempty, and be bounded below. Then by
      the greatest lower bound property,  $\alpha=\inf{A} \in \R$ exists; Then
      for all  $x \in A$, $\alpha \leq x$, and for all other lower bounds  $\gamma$,
      $\gamma \leq \alpha$. Then $-x \leq -\alpha$, and  $-\alpha \leq -\gamma$,
      then we see that  $-\gamma$ and  $-\alpha$ are upperounds of  $-A$, and that
      $-\alpha$ is the least upper bound of $-A$
  \end{enumerate}
\end{example}

\begin{theorem}\label{theorem_1.1.2}
  If $S$ is an ordered set with the least upper bound property, then $S$ also
  inherits the greatest lower bound property.
\end{theorem}
\begin{proof}
  Let $B \subseteq S$, and let  $L \subseteq S$ be the set of all lower bounds
  of $B$. Then we have for any $y \in L$, $x \in B$, $y \leq x$. So every
  element of  $B$ is an upper bound of  $L$, and $L$ is nonempty, hence
  $\alpha=\sup{L} \in S$ exists. Now if  $\gamma \leq \alpha$, then $\gamma$ is not
  an upper bound of  $L$, hence  $\gamma \not\in B$; thus  $\alpha \leq x$ for
  all $x \in B$, so  $\alpha \in L$, and by definition of the greatest lower
  bound, we get $\alpha=\inf{B}$.
\end{proof}
