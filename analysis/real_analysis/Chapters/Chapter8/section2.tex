\section{The Lebesgue Outer Measure}

\begin{definition}
    Let $I \subseteq \R$ be an interval. We define the  \textbf{length} $l(I)$,
    of $I$ to be  $\infty$ if  $I$ is unbounded, and the difference of its
    endpoints otherwise.
\end{definition}

\begin{definition}
    Let $\{I_k\}$ be a countable collection of open bounded intervals covering a
    set $A \subseteq \R$. We define the  \textbf{Lebesgue outer measure} of $A$
    to be
    \begin{equation*}
        m^\ast(A)=\inf{\Big{\{} \sum_{k=1}^\infty{l(I_k)} :
        A \subseteq \bigcup_{k=1}^\infty{I_k} \Big{\}}}
    \end{equation*}
\end{definition}

We now go over some basic properties of the Lebesgue outer measure.

\begin{lemma}\label{8.2.1}
    For any set $A \subseteq \R$,  $m^\ast(A) \geq 0$; in particular,
    $m^\ast(\emptyset)=0$.
\end{lemma}
\begin{proof}
    By definition, since $l(I_k) \geq 0$, each $\sum{l(I_k)} \leq 0$. This makes
    $m^\ast(A) \geq 0$ for any $\{I_k\}$ a countable cover of $A$ by bounded
    open intervals.

    Notice that $\emptyset \subseteq (-\e,\e)$ for some $\e>0$, and that this
    covers $\emptyset$, so that $m^\ast(\emptyset)=l((-\e,\e))=2\e$. Then
    choosing $\e$ small enough gives us  $m^\ast(\emptyset)=0$.
\end{proof}

\begin{lemma}[Monotonicity]\label{8.2.2}
    The Lebesgue Outer Measure is monoton; that is, if $A \subseteq B$, then
    \begin{equation*}
        m^\ast(A) \leq m^\ast(B)
    \end{equation*}
\end{lemma}
\begin{proof}
    Let $\{I_k\}$ be a countable cover of $B$ by bounded open intervals. Then
    notice that  $\{I_k\}$ covers $A$ as well. Now, let
    \begin{align*}
        E   &=  \Big{\{} \sum_{k=1}^\infty{l(I_k)} : A \subseteq
                            \bigcup_{k=1}^\infty{I_k} \Big{\}}  \\
        F   &=  \Big{\{} \sum_{k=1}^\infty{l(I_k)} : B \subseteq
                            \bigcup_{k=1}^\infty{I_k} \Big{\}}  \\
    \end{align*}
    Then since $A \subseteq B$, $F \subseteq E$. Therefore, we get that
    \begin{equation*}
        \inf{E}=m^\ast(A) \leq m^\ast(B)=\inf{F}
    \end{equation*}
\end{proof}

\begin{lemma}\label{8.2.3}
    Countable sets have Lebesgue outer measure $0$.
\end{lemma}
\begin{proof}
    Let $C$ be a countable set, and  $C=\{c_k\}_{k \in \Z^+}$ an enumeration for
    $C$. Let $\e>0$, then for every  $k \in \Z^+$, define the interval
    \begin{equation*}
        I_k=(c_k-\frac{\e}{2^{k+1}}, c_k+\frac{\e}{2^{k+1}})
    \end{equation*}
    Then $\{I_k\}$ is a countable cover of $C$ by bounded open intervals. Thus
    we get
    \begin{equation*}
        0 \leq m^\ast(C) \leq \sum_{k=1}^\infty{l(I_k)}=\sum{\frac{\e}{2^k}}=\e
    \end{equation*}
    Therefore taking $\e$ small enough, we get  $0 \leq m^\ast(C) \leq 0$, and
    equality is established.
\end{proof}

\begin{lemma}\label{8.2.4}
    Let $I \subseteq \R$ be an interval. Then  $m^\ast(I)=l(I)$.
\end{lemma}
\begin{proof}
    Consider first that $I=[a,b]$, a closed bounded interval, where $a<b$. Let
    $\e>0$, then we have that  $[a,b] \subseteq (a-\e,b+\e)$, so that by
    monotonicity
    \begin{equation*}
        m^\ast([a,b]) \leq l((a-\e,b+\e))=b-a+2\e
    \end{equation*}
    Then for $\e$ small enough, this shows that
    \begin{equation*}
        m^\ast([a,b]) \leq b-a
    \end{equation*}

    Now, let $\{I_k\}$ be a countable cover of $[a,b]$ by bounded open
    intervals. Then since $[a,b]$ is compact (by Heine-Borel), there is a finite
    subcover $\{I_k\}_{k=1}^n$ of $[a,b]$. Now, since $a \in
    \bigcup_{k=1}^n{I_k}$, there is atleast one interval $I_k$ containing $a$;
    denote it $ (a_1,b_1)$. Now, $a_1<a<b_1$. If $b_1 \geq b$, then we are done
    as
    \begin{equation*}
        \sum_{k=1}^n{l(I_k)} \geq b_1-a_1>b-a
    \end{equation*}
    Otherwise, $b_1 \in [a,b)$, and since $b_1 \notin (a_1,b_1)$, there is an
    interval $(a_2,b_2)$, distinct from $(a_1,b_1)$, containing $b_1$. Now, if
    $b_2 \geq b$, we are done. Otherwise, proceeding recursively, we obtain a
    subcollection $\{(a_k,b_k)\}_{k=1}^N$ of $\{I_k\}_{k=1}^n$ for which
    \begin{equation*}
        a_1<a \text{ and } a_{k+1}<b_k \text{ for all } 1 \leq k \leq N-1
    \end{equation*}
    The process of selecting such a subcollection must terminate, which leaves
    us with $b_N>b$, so that
    \begin{equation*}
        \sum_{k=1}^n{l(I_k)} \geq \sum_{k=1}^N{l((a_k,b_k))}=
        (b_N-a_N)+\dots+(b_1-a_1) \geq b_N-a_1>b-a
    \end{equation*}
    INdeed, we get
    \begin{equation*}
        \sum_{k=1}^n{l(I_k)} \geq b-a
    \end{equation*}
    so that $m^\ast([a,b])=b-a$.

    Now, suppose that $I$ is any bounded interval. Then for  $\e>0$, there
    exist closed bounded interval $J_1$ and $J_2$ such that
    \begin{equation*}
        J_1 \subseteq I \subseteq J_2
    \end{equation*}
    and
    \begin{equation*}
        l(I)-\e<l(J_1) \leq l(I) \leq l(J_2)<l(I)+\e
    \end{equation*}
    By the monotonicity, and the above discussion of closed bounded intervals,
    we get
    \begin{equation*}
        l(I)-\e<m^\ast(J_1) \leq m^\ast(I) \leq m^\at(J_2)<l(I)+\e
    \end{equation*}
    Therefore, for $\e$ small enough, we get that $m^\ast(I)=l(I)$.

    Finally, suppose that $I$ is an unbounded interval. Then for every  $n \in
    \Z^+$, there is an interval  $J$ with  $l(J)=n$. So that
    \begin{equation*}
        n=m^\ast(J) \leq m^\ast(I)
    \end{equation*}
    This makes $m^\ast(I)=l(I)=\infty$, by definition of $l(I)$.
\end{proof}

\begin{lemma}[Translation Invariance]\label{8.2.5}
    The Lebesgue outer measure is translation invariant; that is, if $A
    \subseteq \R$, and  $y \in \R$,
    \begin{equation*}
        m^\ast(A+y)=m^\ast(A)
    \end{equation*}
\end{lemma}
\begin{proof}
    Let $\{I_k\}$ be a countable cover of $A$ by open bounded intervals. Then
    the collection  $\{I_k+y\}$ is a countable cover of the set $A+y$ by open
    bounded intervals. Moreover, notice that $l(I_k)=l(I_k+y)$. This gives us
    \begin{equation*}
        \sum{l(I_k)}=\sum{l(I_k+y)}
    \end{equation*}
    and we are done.
\end{proof}

\begin{lemma}[Countable Subadditivity]\label{8.2.6}
    The Lebesgue outer measure is countable subadditive; that is, if $\{E_k\}$
    is a countable collection of subsets of $\R$, then
    \begin{equation*}
        m^\ast\Big{(} \bigcup{E_k} \Big{)} \leq \sum{m^\ast(E_k)}
    \end{equation*}
\end{lemma}
\begin{proof}
    Let $\{E_k\}$ be a countable collection of subsets of $\R$, and let
    \begin{equation*}
        E=\bgicup{E_k}
    \end{equation*}
    If at least one of the $E_k$ has  $m^\ast(E_k)=\infty$, then we are done.
    Suppose then that $m^\ast(E_k)$ is finite for all $k \in \Z^+$. Then for
    each  $E_k$, there is a countable cover  $\{I_{k,i}\}_{i \in \Z^+}$ by
    bounded open intervals for which
    \begin{equation*}
        \bigcup{l(I_{k,i})}<m^\ast(E_k)+\frac{\e}{2^k}
    \end{equation*}
    Now, consider the countable collection
    \begin{equation*}
        \{I_{k,i}\}_{k,i \in \Z^+}=\bigcup_{k \in \Z^+}{\{I_{k,i}\}}_{i \in \Z^+}
    \end{equation*}
    Then $\{I_{k,i}\}_{k,i \in \Z^+}$ is a countable cover of $E$ by bounded
    open intervals. We get
    \begin{equation*}
        m^\ast(E) \leq
        \sum_{k \in \Z^+}{\Big{(} \sum_{i \in \Z^+}{l(I_{k,i})} \Big{)}}<
        \sum_{k \in \Z^+}{(m^\ast(E_k)+\frac{\e}{2^k})}=
        \Big{(} \sum{m^\ast(E_k)} \Big{)}+\e
    \end{equation*}
    Taking $\e>0$ small enough, gives us the required subadditivity.
\end{proof}
\begin{corollary}
    The Lebesgue outer measure is finitely subadditive.
\end{corollary}
\begin{proof}
    Recall that finite collections of sets are countable.
\end{proof}
