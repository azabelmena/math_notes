\section{Convergent Sequences}\label{section_3.1}

\begin{definition}
  Let $X$ be a metric space, with metric $d$. A sequence $\{x_n\}$ of points in
  $X$ is said to \textbf{converge} to a point $x \in X$ if for every $\e>0$,
  there is an  $N \in \Z+^$ for which
  \begin{equation}\label{equation_3.1}
    d(x_n,x)<\e \text{ whenever } n \geq N
  \end{equation}
  We say that $\{x_n\}$ \textbf{converges} to $x$ as $n$ approaches $\infty$. We
  call $x$ the \textbf{limit} of $\{x_n\}$ and we write
  \begin{equation*}
    \{x_n\} \xrightarrow{} x \text{ as } n \xrightarrow{} \infty \text{ or }
    \lim_{n \xrightarrow{} \infty}{\{x_n\}}=x \text{ or } \lim{\{x_n\}}=x
  \end{equation*}
  the sequence $\{x_n\}$ is said to \textbf{diverge} if it does not converge to
  any point in $X$.
\end{definition}

\begin{example}
  Consider the following sequences in $\C$.
  \begin{enumerate}

    \item[(1)] The sequence $\{\frac{1}{n}\}$ in bounded. Let $\e>0$, and choose
       $N>\frac{1}{\e}$. Then for any $n \geq N$:
       \begin{equation*}
         \Big{|} \frac{1}{n}-0 \Big{|}=\Big{|} \frac{1}{n} \Big{|} \leq
        \Big{|} \frac{1}{N} \Big{|}<\e
       \end{equation*}
       then:
       \begin{equation}
         \lim_{n \xrightarrow{} \infty}{\frac{1}{n}}=\infty
       \end{equation}

    \item[(2)] The sequence $\{n^2\}$ is unbounded and diverges.

    \item[(3)] $1+\frac{(-1)^n}{n} \rightarrow 1$ as $n \rightarrow \infty$, and
      $\{1+\frac{(-1)^n}{n}\}$ is bounded.

    \item[(4)] $\{i^n\}$ is bounded and divergent.

    \item[(5)] $\{1\}$ is bounded and converges to  $1$.
  \end{enumerate}
\end{example}

\begin{theorem}\label{theorem_3.1.1}
  Let $\{x_n\}$ be a sequence in a metric space, then:
  \begin{enumerate}
    \item[(1)] $\{x_n\}$ converges to  $x \in X$ if and only if every every
      neighborhood of $x$ contains  $x_n$ for all but finitely many  $n$.

    \item[(2)] If  $\{x_n\}$ converges to  $x$, and  $x'$, then $x=x'$.

    \item[(3)] If $\{x_n\}$ converges, then  $x_n$ is bounded.

    \item[(4)] If $E \subseteq X$, and  $x$ is a limit point of  $E$, then there
      is a sequence  in  $E$ that converges to  $x$.
  \end{enumerate}
\end{theorem}
\begin{proof}
  Suppose $x_n \rightarrow x$, and let  $U$ be a neighborhood of  $x$. For some
  $\epsilon>0$, there is an $N \in \Z^+$ for which $d(x_n,x)<\epsilon$, whenever
  $n \geq N$, thus $x_n \in U$ for finitely many  $n$. Conversely, suppose that
  $x_n \in U$ for some  $n \geq N$, then  letting  $\epsilon>0$, we have
  $d(x,x_n)<\epsilon$, hence $x_n \rightarrow x$.

  Let  $\epsion>0$, then there are $N_1,N_2 \in Z^+$ such that
  $d(x_n,x)<\frac{\epsilon}{2}$, and $d(x_n,x')<\frac{\epsilon}{2}$. Then
  choosing $N=\max\{N_1,N_2\}$, and letting $\epsilon$ be arbitrarily small, we
  have  $d(x,x') \leq d(x,x_n)+d(x_n,x')<\frac{\epsilon}{2}+\frac{\epsilon}{2}=
  \epsilon$; and so we get that $x=x'$.

  Let  $x_n \rightarrow x$, then there is an  $N \in \Z^+$ for which
  $d(x_n,x)<1$ whenever $n \geq N$. Letting  $r=\max\{1,d(x_N,x)\}$, then
  $d(x_n,x) \leq r$.

  Finally, let $x$ be a limit point of  $E$, then for each  $n \in Z^+$, there
  is an  $x_n \in E$ such that  $d(x,x_n)<\frac{1}{n}$, choose
  $N>\frac{1}{\epsilon}$, then whenever $n \geq N$, $d(x,x_n)<\epsilon$; hence
  $x_n \rightarrow x$.
\end{proof}

\begin{theorem}\label{theorem_3.1.2}
  Suppose $\{x_n\}$ and  $\{y_n\}$ are sequences in  $\C$, and that  $\lim{x_n}=x$,
  $\lim{y_n}=y$ as  $n \rightarrow \infty$. Then the following hold as
  $n \rightarrow \infty$:
  \begin{enumerate}
    \item[(1)] $\lim{(x_n+y_n)}=\lim{x_n}+\lim{y_n}=x+y$.

    \item[(2)] $\lim{x_ny_n}=\lim{x_n}\lim{y_n}=xy$.

    \item[(3)] $\lim{\frac{x_n}{y_n}}=\frac{\lim{x_n}}{\lim{y_n}}=\frac{x}{y}$;
      given that $y_n, y \neq 0$.
  \end{enumerate}
\end{theorem}
\begin{proof}
  \begin{enumerate}
    \item[(1)] Let $\epsilon>0$, then for  $N_1, N_2 \in \Z^+$,
      $|x_n-x|<\frac{\epsilon}{2}$ and
      $|y_n-y|<\frac{\epsilon}{2}$. Then choose $N=\max\{N_1,N_2\}$, then
      whenever $n \geq N$, we have  $|(x_n+y_n)-(x+y)| \leq
      |x_n-x|+|y_n-y|<\epsilon$.

    \item[(2)] Notice that $x_ny_n-xy=(x_n-x)(y_n-y)+x(y_n-y)+y(x_x-x)$, then for
      $N_1,N_2 \in \Z^+$, $|x_n-x|<\sqrt{\epsilon}$, and  $|y_n-y|<\sqrt{\epsilon}$.
      Then choosing  $N=\max\{N_1,N_2\}$, then $|(x_n-x)(y_n-y)|<\epsilon$, thus
      we have $|x_ny_n-xy|\leq |(x_n-x)(y_n-y)|+|x(y_n-y)|+|y(x_x-x)|<\epsilon$.

    \item[(3)] We first show that $\frac{1}{y_n} \rightarrow \frac{1}{y}$, given
      that $y_n,y \neq 0$. Choose  $m$ such that  $|y_n-y|<\frac{1}{2}|y|$
      whenever $n \geq m$, then $|y_n|>\frac{1}{2}|y|$. Then for $\epsilon>0$,
      there is an  $N>m$ such that whenever  $n \geq N$,
      $|y_n-y|<\frac{1}{2}|y|^2\epsilon$. Then $|\frac{1}{y_n}-\frac{1}{y}| \leq
      \frac{|y_n-y|}{|y_ny|}<\frac{2}{|y|^2}|y_n-y|<\epsilon$. Then choosing the
      sequences $\{x_n\}$ and  $\{\frac{1}{y_n}\}$, the rest follows.
  \end{enumerate}
\end{proof}
\begin{corollary}
  \begin{enumerate}
    \item[(1)] For any $c \in \C$, and a sequene  $x_n \rightarrow x$, we have
      $\lim{cx_n}=c\lim{x_n}=cx$ and  $\lim{(c+x_n)}=c+\lim{x_n}=c+x$ as
      $n \rightarrow \infty$.

    \item[(2)] Provided that  $x,x_n \neq 0$, we have
      $\lim{\frac{1}{x_n}}=\frac{1}{\lim{x_n}}=\frac{1}{x}$, as
      $n \rightarrow \infty$.
  \end{enumerate}
\end{corollary}
\begin{proof}
  We choose $\{x_n\}$ and  $\{y_n\}=\{c\}$ for all  $n$, then the results follow.
\end{proof}

\begin{theorem}\label{3.1.3}
  \begin{enumerate}
    \item[(1)] Let $x_n=(\alpha_{1n}, \dots \alpha_{kn}) \in \R^k$. Then
      $\{x_n\}$ converges to  $x$ if and only if $\lim{\alpha_{jn}}=\alpha_j$ for
      $1 \leq j \leq k$, as  $n \rightarrow \infty$.

    \item[(2)] Let $\{x_n\},\{y_n\}$ be sequences in  $\R^k$, and let
      $\{\beta_n\}$ be a sequence in  $\R$  such that $x_n \rightarrow x$,
      $y_n \rightarrow y$, and  $\beta_n \rightarrow \beta$. Then
      $\lim{(x_n+y_n)}=x+y$, $\lim{x_ny_n}=xy$, and  $lim{\beta_nx_n}=\beta x$.
  \end{enumerate}
\end{theorem}
\begin{proof}
  If $x_n \rightarrow x$, then  $|\alpha_{jn}-\alpha_j| \leq ||x_n-x||<\epsilon$,
  thus $\lim{\alpha_{jn}}=\alpha_j$. Conversely, suppose that $\alpha_{jn}
  \rightarrow \alpha_j$. Then for $\epsilon>0$ there is an $N \in \Z^+$ such that
  $n \geq N$ implies $|\alpha_{jn}-\alpha_j|<\frac{\epsilon}{\sqrt{k}}$. Then for
  $n \geq N$,
  \begin{equation*}
    ||x_n-x||=\sqrt{\sum{|\alpha_{jn}-\alpha_j|^2}}<\epsilon
  \end{equation*}
  To prove $(2)$, we appy part $(1)$ of this theorem together with theorem
  \ref{theorem_3.1.2}.
\end{proof}

\begin{theorem}[The Sandwhich Theorem]\label{theorem_3.1.4}
  Let $\{x_n\}$, $\{y_n\}$, and  $\{w_n\}$ be sequences in $\R$, and  Suppose that
  $\lim{\{x_n\}}=\lim{\{y_n\}}=a$ and that there is an $N \in \Z^+$ such that
  $x_n \leq w_n \leq y_n$ for all  $n \geq N$. Then
  $\lim_{n \rightarrow \infty}{\{w_n\}}=a$.
\end{theorem}
\begin{proof}
  Let $\epsilon>0$ and let  $\{x_n\}$ and  $\{y_n\}$ both converge to  $a$. Then
  by definition there are  $N_1,N_2 \in \Z^+$ such that $|x_n-a|<\epsilon$ and
  $|y_n-a|<\epsilon$ for  $n \geq N_1,N_2$. Now choose $N=\max\{N_0,N_1,N_2\}$, if
  $n \geq N$, we have  $-\epsilon<x_n-a<\epsilon$, and we also have
  $x_n-a<w_n-a<y_n-a$, thus we have that:
  \begin{equation*}
    -\epsilon<x_n-a<w_n-a<y_n-a<\epsilon
  \end{equation*}
  Thus we have that $|w_n-a|<\epsilon$.
\end{proof}

\begin{corollary}
  If $\{x_n\} \rightarrow 0$ as  $n \rightarrow \infty$, and $\{y_n\}$ is
  bounded, then $x_ny_n \rightarrow 0$ as $n \rightarrow \infty$.
\end{corollary}
\begin{proof}
  We have that $\{y_n\}$ is bounded, hence, there is $M>0$ such that  $|y_n|
  \leq M$ for all  $n \in \Z^+$. And since $\{x_n\}$ converges to $0$ we have
  that for any $\epsilon$ there is an  $N \in \Z^+$ such that for  $n \geq N$,
  $|x_n-0|<\frac{\epsilon}{M}$. Then
  $|x_ny_n-0|=|x_ny_n|<M|x_n|<M\frac{\epsilon}{M}=\epsilon$. Therefore,
  $x_ny_n \rightarrow 0$ as  $n \rightarrow \infty$.
\end{proof}
\begin{corollary}
  Let $\{x_n\}$, $\{y_n\}$ be sequences such that  $0 \leq x_n \leq y_n$ for
  $n \geq N>0$. Then if  $y_n \rightarrow 0$, then  $x_n \rightarrow 0$ as
  $n \xrightarrow{} \infty$.
\end{corollary}
