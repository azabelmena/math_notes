\section{Connectedness in $\C$}

\begin{definition}
    We say a metric space $X$ is connected provided there are no disjoint
    nonempty open sets  $A,B \subseteq X$ for which  $X=A \cup B$.
\end{definition}

\begin{lemma}\label{2.2.1}
    A metric space $X$ is connected if its only closed and open sets are the
    emtpyset and itself.
\end{lemma}

\begin{example}\label{example_2.5}
    Consider the space $X=\{z \in \C : |z|<1\} \cup \{z \in \C : |z-3|<1\}$.
    Let $A=\{z \in \C : |z|<1\}$ and $B=\{z \in \C : |z-3|<1\}$. Then then
    both $A$ and $B$ are open in  $X$. Moreover,  $A$ is also closed in $X$ as
    $B=\com{X}{A}$. So $X$ is not connected.
\end{example}

\begin{lemma}\label{2.2.2}
    A space $X \subseteq X$ is connected if, and only if it is an interval.
\end{lemma}
\begin{proof}
    Suppose that $X=[a,b]$, where $a,b \in \R$ and  $a<b$. Let  $A \subseteq X$
    be open, with  $a \in A$ and  $b \in B$ and where $X \neq A$. Then there is
    an $\e>0$ for which  $[a,a+\e) \subseteq A$. Let $r=\sup{\{\e : [a,a+\e)
    \subseteq A\}}$. If $a \leq x <a+r$, putting  $h=a+(r-x)>0$ there is an $\e
    > 0$ for which  $r-h<\e<r$ and $[a,a+\e) \subseteq A$. However, $a \leq
    a+(r-h)<a+\e$ putting $x \in A$. So that  $[a,a+r) \subseteq A$. Now, if
    $a+r \in A$, then by the openness of  $A$, there is a  $\d>0$ with
    $[a+r,a+r+\d] \subseteq A$, which puts $[a+r,a+r+\d) \subseteq A$. But that
    contradicrts that $r$ is a least upper boundl; so $a+r \notin A$.

    Now, if $A$ were closed, then  $a+r \in B=\com{X}{A}$, which is open, so
    that there is a $\d>0$ such that  $(a+r-\d,a+r) \subseteq B$, which
    contradicts that $[a,a+r) \subseteq A$.
\end{proof}
\begin{remark}
    Note that the first part of this proof lacks the proof for the other types
    of intervals.
\end{remark}

\begin{definition}
    Let $z,w \in \C$. We define the  \textbf{staight line segment} $[z,w]$
    from $z$ to  $w$ to be the set
    \begin{equation*}
        [z,w]=\{tw+(1-t)z : 0 \leq t \leq 1\}
    \end{equation*}
    A \textbf{polygon} from $z$ to  $w$ is defined to be the set
    \begin{equation*}
        P[z,w]=\bigcup_{k=1}^n{[z_k,w_k]}
    \end{equation*}
    where $z_1=z$,  $w_n=w$, and  $z_{k+1}=w_k$ for all $1 \leq k \leq n-1$.
    When the endpoints of the polygon are understood, we may simply just write
    $P$, or we enumerate the points of $P$ as  $P=[z,z_2, \dots, z_n,w]$.
\end{definition}

\begin{theorem}\label{2.2.3}
    An open set $U$ of  $\C$ is connected if, and only if for all  $z,w \in U$,
    there exists a polygon $P[z,w]$ from $z$ to  $w$ contained in $U$.
\end{theorem}
\begin{proof}
    Let $P[z,w] \subseteq U$ be the given polygon. Suppose that $U$ were not
    connected. Then there exist disjoint nonempty open sets $Z$ and $W$ of $U$
    (as a subspace of $\C$) for which $U=Z \cup W$. Let  $z \in Z$ and $w \in
    W$. Consider the case for when $P[z,w]=[z,w]$. Define $S=\{s \in [0,1] :
    sw+(1-s)z \in A\}$ and $T=\{s \in [0,1] : sw+(1-s)z \in B\}$. Then notice
    that $S$ and $T$ are disjoint, and that $S \cup T=[0,1]$. Moreover, they are
    open subsets of the interval $[0,1] \subseteq \R$; but $[0,1]$ is connected
    in $\R$, which is a contradiction. Therefore  $U$ must be connected.

    On the otherhand, let  $w \in Z$ and let  $P=[z,z_2, \dotsm z_n,w] \subseteq
    U$ SInce $U$ is open, there is an $\e>0$ such that  $B(w,\e) \subseteq U$.
    Now, if $u \in B(w,\e)$, then $[w,u] \subseteq B(w,\e) \subseteq U$, so the
    polygon $Q=P \cup [w, u] \subseteq U$. Hence $B(w,\e) \subseteq Z$, which
    makes $Z$ oepn. On the otherhand, consider  $u \in \com{U}{Z}$, and let
    $\e>0$ such that $B(u,\e) \subseteq U$. Then there is a $w \in Z \cap
    B(u,\e)$. Construct, then a polygon $P[z,u]$ so that $B(u,\e) \cap Z$ is
    empty. That is, $B(u,\e) \subseteq \com{U}{Z}$ making $\com{U}{Z}$ open, and
    hence $Z$ closed.
\end{proof}
\begin{corollary}
    If $U \subseteq \C$ is an open and connected set, then for all  $z,w \in U$,
    there is a polygon  $P[z,w]$ in $U$ made up of straight line segments
    parallel to either the real axis, or the imaginary axis.
\end{corollary}

\begin{definition}
    Let $X$ be a metric space. We call a subset  $C \subset X$ a
    \textbf{connected component} if it is maximally connected in $X$.
\end{definition}

\begin{example}\label{example_2.6}
    \begin{enumerate}
        \item[(1)] $A$ and  $B$ in example \ref{example_2.5} are connected
            components.

        \item[(2)] Let $X=\{\frac{1}{k} : k \in \Z^+\} \cup \{0\}$. Then every
            connected component is a point of $x$, and vise versa; with, the
            exception of $0$.
    \end{enumerate}
\end{example}

\begin{lemma}\label{2.2.4}
    Let $X$ be a metric space with  $x_0 \in X$. If $\{D_j\}$ is a collection of
    connected subsets of $X$, such that  $x_0 \in D_j$, then the union
    $D=\bigcup{D_j}$ is connected.
\end{lemma}
\begin{proof}
    Let $A \subseteq D$, which is a metric space, for which  $A$ is both open
    and closed, and nonempty. Then  $A \cap D_j$ is open and closed for all
    $j$. Now, since  $D_j$ is connected, either  $A \cap D_j=\emtyset$, or  $A
    \cap D_j=D_j$. Since  $A$ is nonempty, we must have the latter case. Then
    there exists at least one index $k$ for which $A \cap D_k=D_k$. Then if
    $x_0 \in A$, $x_0 \in A \cap D_k$ so that $x_0 \in D_k$ making $A \cap
    D_j=D_j$ for all  $j$ or  $D_j \subseteq A$. In either case, we get  $D=A$.
\end{proof}

\begin{theorem}\label{2.2.5}
    The connected components of a metric space partition the space.
\end{theorem}
\begin{proof}
    Let $\Dc$ the collection of all connected subsets of  $X$ containing a point
     $x_0 \in X$. Then $\Dc$ is nonempty by definition, and by hypothesis, we
     have that  $C=\bigcup{D_j}$ is connected, and that $x_0 \in C$.

     Now, suppose that $C \subseteq D$ for some connected st $D$. Then  $x_0 \in
     D$ so that $D \in \Dc$, and hence $D \subseteq C$. This makes $C=D$, and
     hence  $C$ is a connected component of  $X$. This then implies that
     $X=\bigcup{C_j}$ where $\{C_j\}$ is the collection of connected components
     of $X$.

     Now, consider  $\{C_j\}$, and suppose that for distinct components $C_1$
     and $C_2$, that there is an $x_0 \in C_1 \ca C_2$. Then $x_0 \in C_1$, and
     $x_0 \in C_2$ so that $C_1=C_1 \cup C_2=C_2$, which is a contradiction.
     Therefore the connected components are pairwise disjoint.
\end{proof}

\begin{lemma}\label{2.2.6}
    If $X$ is a connected metric space with  $A \subseteq X$, and  $A \subseteq
    B \subseteq \cl{A}$, then $B$ is also connected.
\end{lemma}
\begin{corollary}
    Connected components of a metric space are closed.
\end{corollary}

\begin{theorem}\label{2.2.7}
    If $U$ is open in  $\C$, then  $U$ has countably many connected components;
    each of which is open.
\end{theorem}
\begin{proof}
    Let $C \subseteq U$ a connected component, with  $x_0 \in C$. Since $U$ is
    open, there is an  $\e>0$ for which  $B(x_0,\e) \subseteq U$. Then $B(x_0,
    \e) \cup C$ is connected so that $B(x_0,\e) \cup C=C$, so that $B(x_0,\e)
    \subseteq C$. This makes each $C$ open.

    Now, let  $S=\{a+ib \in \Q(i) : a+ib \in U\}$. Then $S$ is countable by the
    density of  $\Q(i)$ in $\C$, and each connected component of $U$ contains a
    point of $S$. This implies there are countably many such components.
\end{proof}
