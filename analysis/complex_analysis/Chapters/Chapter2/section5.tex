\section{Continuity and Uniform Convergence in $\C$}

\begin{definition}
    Let $(X,d)$ and $(Y,\rho)$ be metric spaces, and $f:X \xrightarrow{} Y$ a
    function. We say that $f$ is \textbf{continuous} at a point $a \in X$ if for
    every  $\e>0$, there is a  $\d>0$ for which
    \begin{equation*}
        \rho(f(x),y)<\e \text{ whenever } 0<d(x,a)<\d
    \end{equation*}
    for some $y \in Y$ and we write $\lim_{x \xrightarrow{} a}{f(x)}=y$, or
    simply $f \xrightarrow{} y$. If $f$ is continuous at every point in  $X$, we
    say that  $f$ is  \textbf{continuous} on $X$  (or simply that $f$ is
    \textbf{continuous}).
\end{definition}

\begin{lemma}\label{2.5.1}
    Let $X$ and  $Y$ be metric spaces. If  $f:X \xrightarrow{} Y$ is a function,
    then the following statements are equivalent for any $a \in X$ with
    $y=f(a)$.
    \begin{enumerate}
        \item[(1)] $f$ is continuous at  $a$.

        \item[(2)] For any $\e>0$  $\inv{f}(B(y,\e))$ contains a ball centered
            about $a$.

        \item[(3)] If $\{x_n\}$ is a sequence of points of $X$ converging to
            $a$, then the sequence $\{f(x_n)\}$ converges to $y$.
    \end{enumerate}
\end{lemma}

\begin{lemma}\label{2.5.2}
    Let $X$ and  $Y$ be metric spaces, and  $f:X \xrightarrow{} Y$ a function.
    The following statements are equivalent.
    \begin{enumerate}
        \item[(1)] $f$ is continuous on  $X$.

        \item[(2)] For any open set $U$ of  $Y$,  $\inv{f}(U)$ is open in $X$.

        \item[(3)] For any closed set $V$ of  $Y$, $\inv{f}(V)$ is closed in
            $X$.
    \end{enumerate}
\end{lemma}

\begin{lemma}\label{2.5.3}
    Let $f:X \xrightarrow{} \C$ and $g:X \xrightarrow{} \C$ be complex-valued
    functions. If $f$ and $g$ are continuous, then for every $\a,\b \in \C$, we
    have
    \begin{enumerate}
        \item[(1)] $\a f+\b g$ is continuous.

        \item[(2)] $fg$ is continuous, and  $\frac{f}{g}$ is continuous provided
            $g(z) \neq 0$ for all $z \in X$.
    \end{enumerate}
\end{lemma}

\begin{lemma}\label{2.5.4}
    If $f:X \xrightarrow{} Y$ and $g:Y \xrightarrow{} Z$ are continuous, then $g
    \circ f:X \xrightarrow{} Z$ is continuous.
\end{lemma}

\begin{definition}
    We call a function $f:X \xrightarrow{} Y$ \textbf{uniformly continuous} if
    for every $\e>0$, there is a  $\d>0$, depending on  $\e$, such that
    \begin{equation*}
        \rho(f(x),f(y))<\e \text{ whenever } d(x,y)<\d
    \end{equation*}
    We call $f$  \textbf{Lipschitz continuous} if there exists an $M>0$ such
    that
    \begin{equation*}
        \rho(f(x),f(y))=Md(x,y) \text{ for all } x,y \in X
    \end{equation*}
\end{definition}

\begin{lemma}\label{2.5.5}
    Lipschitz continuous functions are uniformly continuous, and uniformly
    contiuous functions are continuous.
\end{lemma}

\begin{definition}
    Let $X$ be a metric space, and  $A \subseteq X$ a nonempty subset. We define
    the  \textbf{distance} from a point $x \in X$ to  $A$ to be
    \begin{equation*}
        d(x,A)=\inf{\{d(x,a) : a \in A\}}
    \end{equation*}
\end{definition}

\begin{lemma}\label{2.5.6}
    Let $X$ a metric space, and  $A \subseteq X$ nonempty. The following are
    true.
    \begin{enumerate}
        \item[(1)] $d(x,A)=d(x,\cl{A})$.

        \item[(2)] $d(x,A)=0$ if, and only if $x \in \cl{A}$.

        \item[(3)] $|d(x,A)-d(y,A)| \leq d(x,y)$ for all $x,y \in X$.
    \end{enumerate}
\end{lemma}
\begin{proof}
    Let $A \subseteq B$. Then by definition,  $d(x,B) \leq d(x,A)$, so that
    $d(x,\cl{A}) \leq d(x,A)$. Now, if $\e>0$, there is a  $y \in \cl{A}$ for
    which $d(x,y) \leq d(x,\cl{A})+\frac{\e}{2}$, and there exists an $a \in A$
    with  $d(y,a)<\frac{\e}{2}$. Then
    \begin{equation*}
        |d(x,y)-d(x,a)|<d(y,a)<\frac{\e}{2}
    \end{equation*}
    by the triangle inequality. Then $d(x,a)<d(x,y)+\frac{\e}{2}$ so that
    $d(x,A)<d(x,\cl{A})+\frac{\e}{2}$. That is $d(x,A) \leq d(x,\cl{A})$.

    Now, if $x \in \cl{A}$, then $d(x,\cl{A})=d(x,A)=0$. Conversly, if
    $d(x,A)=0$, then consider the decreasing sequence $\{a_n\}$ of $A$ such that
    $\lim{d(x,a_n)}=d(x,A)$. Then $\lim{d(x,a_n)}=0$ so that $\lim{a_n}=x$, so
    that $x \in \cl{A}$.

    Finally, we have for $a \in A$ that  $d(x,a) \leq d(x,y)+d(y,a)$, so that
    $d(x,A) \leq \inf{\{d(x,y)+d(y,a) : a \in A\}}=d(x,y)+d(y,A)$. This gives
    $d(x,A)-d(y,A) \leq d(x,y)$. Similar reasoning also gives $d(y,A)-d(x,A)
    \leq d(x,y)$ so that
    \begin{equation*}
        |d(x,A)-d(y,A)| \leq d(x,y) \text{ for all } x,y \in X
    \end{equation*}
\end{proof}
\begin{corollary}
    The function $f:X \xrightarrow{} \R$ defined by $f(x)=d(x,A)$ is Lipschitz
    continuous.
\end{corollary}

\begin{theorem}\label{2.5.7}
    Let $f:X \xrightarrow{} Y$ be continuous. Then following are true.
    \begin{enumerate}
        \item[(1)] If $X$ is compact, then so is  $f(X)$.

        \item[(2)] If $X$ is connected, so is  $f(X)$.
    \end{enumerate}
\end{theorem}
\begin{proof}
    Without loss of generality, suppose $f(X)=Y$. If $X$ is compact, et
    $\{y_n\}$ a sequence in $Y$. Then for every  $n \geq 1$, there is a sequence
    of points  $\{x_n\}$ of $X$ with  $f(x_n)=y_n$, and $\{x_{n_k}\}
    \xrightarrow{} x$. If $y=f(x)$, then by continuity, $\{y_{n_k}\}
    \xrightarrow{} y$ so that $Y$ is also compact.

    Now, if  $X$ is connected, let  $S \subseteq Y$ a nonempty set wich is both
    open and closed. Then  $\inv{f}(S) \neq \emptyset$ and $\inv{f}(S)$ is also
    open and closed, so that $X=\inv{f}(S)$ by connectivity. This makes $S=Y$,
    and so  $Y$ must also be continuous.
\end{proof}
\begin{corollary}
    If $K$ is compact or connected in  $X$, then  $f(K)$ is compact or connected
    in $Y$.
\end{corollary}
\begin{corollary}
    If $f:X \xrightarrow{} \R$ is continuous, and $X$ is connected, then  $f(X)$
    is an interval.
\end{corollary}

\begin{theorem}[The Intermediate Value Theorem]\label{2.5.8}
    If $f[a,b] \xrightarrow{} \R$ is continuous, with $f(a) \leq c \leq f(b)$,
    then there is an $x \in [a,b]$ with $f(x)=c$.
\end{theorem}
\begin{corollary}
    If $K \subseteq X$ is compact, then there exist  $x_0,y_0 \in K$ with
    $f(x_0)=\sup{\{f(x) : x \in K\}}$ and $f(y_0)=\inf{\{f(y) : y \in K\}}$.
\end{corollary}
\begin{corollary}
    If $K \subseteq X$ is nonempty, and  $x \in X$, there is a  $y \in K$ for
    which $d(x,y)=d(x,K)$.
\end{corollary}
\begin{proof}
    Define $f:X \xrightarrow{} \R$ by $f(y)=d(x,y)$. Then $f$ is continuous, and
    by above, assumes a minimum value  $y in K$. Then $f(y) \leq f(x)$ for all
    $x \in K$, so that  $d(x,y)=d(x,K)$ by definition.
\end{proof}

\begin{theorem}\label{2.5.9}
    Let $f:X \xrightarrow{} Y$ be continuous. If $X$ is compact, then  $f$ is
    uniformly continuous.
\end{theorem}
\begin{proof}
    Let $\e>0$ and suppose there is no such  $\d>0$ for which the statement
    holds. Then each  $\d=\frac{1}{n}$ in particular fais. Then there exist
    $x_n,y_n \in X$ with  $d(x_n,y_n)<\frac{1}{n}$, but where
    $\rho(f(x_n),f(y_n)) \geq \e$. Now, since $X$ is compact, there is a
    subsequence  $\{x_{n_k}\}$ of $\{x_n\}$ converging to a point $x \in X$.
    Now,  $d(x,y_{n_k}) \leq d(x,x_{n_k})+\frac{1}{n_k}$ which goes to $0$ as
    $k \xrightarrow{} \infty$. SO $\{y_{n_k}\} \xrightarrow{} x$. But if,
    $y=f(x)$, and $y=\lim{f(x_{n_k})}=\lim{f(y_{n_k})}$, then we get
    \begin{equation*}
        \e \leq \rho(f(x_{n_k}),f(y_{n_k})) \leq
        \rho(f(x_{n_k}),y)+\rho(y,f(y_{n_k}))=0
    \end{equation*}
    which is a contradiction since $\e>0$.
\end{proof}

\begin{definition}
    If $A,B \subseteq X$ are nonempty subsets of a metric space  $X$, we define
    the  \textbf{distance} between $A$ and $B$ to be
    \begin{equation*}
        d(A,B)=\inf{\{d(a,b) : a \in A, b \in B\}}
    \end{equation*}
\end{definition}

\begin{theorem}\label{2.5.10}
    let $A$ and  $B$ be disjoint subsets of a metric space  $X$; with  $B$
    closed, and  $A$ compact. Then  $d(A,B)>0$.
\end{theorem}
\begin{proof}
    Define $f:X \xrightarrow{} \R$ by $f(x)=d(x,B)$. Since $A$ and  $B$ are
    disjoint, and  $B$ is closed, $f(a)>0$ for all $a \in A$. Moreover, since
    $A$ is compact, there is an  $a \in A$ for which  $0<f(a)=\inf{\{f(x) : x
    \in A\}}=d(A,B)$.
\end{proof}

\begin{definition}
    Let $X$ be a set, and $(Y,\rho)$ a metric space; and let $\{f_n\}$ a
    sequence of functions from $X$ to  $Y$. We say that  $\{f_n\}$
    \textbf{converges uniformly} if for every $\e>0$, there is an  $N>0$,
    dependent on  $\e$ such that
    \begin{equation*}
        \rho(f(x),f_n(x))<\e \text{ whenever } n \geq N
    \end{equation*}
    for all $x \in X$. We write $\{f_n\} \xrightarrow{\text{uniformly}} f$, or
    just $\{f_n\} \xrightarrow{} f$.
\end{definition}

\begin{theorem}\label{2.5.11}
    If $f_n:X \xrightarrow{} Y$ is continuous for each $n \geq 1$, and
    $\{f_n\} \xrightarrow{\text{uniformly}} f$, then $f$ is also continuous.
\end{theorem}
\begin{proof}
    Fix $x_0 \in X$ and let $\e>0$. Since  $\{f_n\} \xrightarrow{} f$, there is
    a function $f_n$ for which  $\rho(f(x),f_n(x))<\frac{\e}{3}$ for every $x
    \in X$. Since  $f_n$ is continuous, there is a  $\d>0$ such that
    \begin{equation*}
        \rho(f_n(x_0),f_n(x))<\frac{\e}{3} \text{ whenever } d(x,x_0)<\d
    \end{equation*}
    Therefore, if $d(x_0,x)<\d$ we have
    \begin{equation*}
        \rho(f(x_0),f(x)) \leq
        \rho(f(x_0),f_n(x_0))+\rho(f_n(x_0),f_n(x))+\rho(f_n(x),f(x))<
        \frac{\e}{3}+\frac{\e}{3}+\frac{\e}{3}=\e
    \end{equation*}
    so that $f$ is continuous.
\end{proof}

\begin{theorem}[The Weierstrass $M$-test]\label{2.5.12}
    Let $u_n:X \xrightarrow{} \C$ be a function such that $|u_n(x)| \leq M_n$, for
    all $x \in X$, and suppose that the sum $\sum{M_n}$ is finite. THen $\sum{u_n}$
    is uniformly convergent.
\end{theorem}
\begin{proof}
    Let $f_n(x)=u_1(x)+\dots+u_n(x)$. Then for $n>m$,
    $|f_n(x)-f_m(x)|=|u_{m+1}(x)+\dots+u_n(x)| \leq \sum_{k=m+1}^n{M_k}$.
    Since $\sum{M_k}$ is finite, this sum converges, so that $\{f_n\}$ is
    Cauchy in $\C$. That is, there exists a  $\xi \in \C$ for which  $\{f_n(x)\}
    \xrightarrow{} \xi$. Define then $f(x)=\xi$, then $f:X \xrightarrow{} \C$ is
    a function with
    \begin{equation*}
        |f(x)-f_n(x)|=|u_{m+1}(x)+\dots+u_n(x)| \leq \sum_{k=m+1}^n{|u_k(x)|}
        \leq \sum_{k=m+1}^n{M_k}
    \end{equation*}
    Then for every $\e>0$, there is an  $N>0$ such that  $\sum{M_k}<\e$,
    whenever $n \geq N$. THus  $|f(x)-f_n(x)|<\e$ for all $x \in X$.
\end{proof}
