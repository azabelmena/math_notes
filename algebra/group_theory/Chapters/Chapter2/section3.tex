%----------------------------------------------------------------------------------------
%	SECTION 1.1
%----------------------------------------------------------------------------------------

\section{Cyclic Groups.}
\label{section1}

\begin{definition}
    Let $G$ be a group, and let  $H \leq G$ be a subgroup. We say that  $H$ is
     \textbf{cyclic} if it can be generated by a single element. That is, there
     is some $x \in H$ for which  $H=\{x^n : n \in \Z\}$. We write
     $H=\vbrack{x}$.
\end{definition}
\begin{remark}
    Notice that if $H$ is cyclic, there is a singleton set that generates  $H$,
    i.e.  $H=\vbrack{\{x\}}$.
\end{remark}

\begin{lemma}\label{2.3.1}
    If $G=\vbrack{g}$, then $G=\vbrack{\inv{g}}$.
\end{lemma}
\begin{proof}
    For any $h \in G$, if  $h=g^n$, then $\inv{h}=x^{-n}=(\inv{x})^n$.
\end{proof}

\begin{lemma}\label{2.3.2}
    If $G$ is a cyclic group, then  $G$ is Abelian.
\end{lemma}
\begin{proof}
    Let $G=\vbrack{g}$ and for any $m,n \in \Z^+$. Then
    $g^mg^n=g^{m+n}=g^{n+m}=g^mg^n$.
\end{proof}

\begin{example}
    \begin{enumerate}
        \item[(1)] In $D_{2n}$, the set $\{e,r, \dots, r^n\}$ is cyclic, and
            $\vbrack{r} \leq D_{2n}$.

        \item[(2)] $\Z=\vbrack{1}=\vbrack{-1}$.

        \item[(3)] Consider the subgroup $\C_n=\{z \in \C : z^n=1\} \leq \C$ of
            complex primitive $n$-th roots of unity in $\C$, where  $n \in
            \Z^+$. Then  $\C=\vbrack{z}=\vbrack{\frac{1}{z}}$.

        \item[(4)] $\C_4=\vbrack{i}=\vbrack{-i}$.
    \end{enumerate}
\end{example}

\begin{lemma}\label{2.3.3}
    Let $G=\vbrack{g}$ be a finite cyclic group, Then $\ord{G}=\ord{g}$.
\end{lemma}
\begin{proof}
    Let $\ord{g}=n$, by the division theorem, there exists $t,q,r \in \Z$ such
    that  $t=nq+r$, with $0 \leq r < n$ ; i.e. $ t \equiv r \mod{n}$, so
    $g^t=g^r$. Thus there are at most  $n$ such elements in  $G$. Now, suppose
    $g^i=g^j$ for  $0 \leq i < j < n$, then  $g^{j-i}=e$, since $j-i < n$, this
    contradicts the order of  $g$. Therefore, there are eaxctly $\ord{g}=n$ such
    element in $G$, since  $G=\vbrack{g}$, these are the only elements, which
    proves the result.
\end{proof}
\begin{corollary}
    If $G=\vbrack{g}$ is a finite cyclic group, then $g$ is of finite order.
\end{corollary}
\begin{proof}
    If $g$ is of infinite order, then there are infinitely many powers of $g$,
    since $g$ generates  $G$, this contradicts the finiteness of  $G$.
\end{proof}

\begin{lemma}\label{2.3.4}
    Let $G$ be a group and let  $g \in G$ and let  $m,n \in \Z^+$ distinct, such
    that  $g^n=e$ and  $g^m=e$. Then  $g^{(m,n)}=e$. Moreover, $\ord{g}|m$.
\end{lemma}
\begin{proof}
    There exist $p,q \in \Z$ such that $mp+nq=(m,n)$. Then by hypothesis,
    $g^{(m,n)}=(g^m)^p(g^n)^q=e$. Morevoer, assuming, without loss of
    generality, that $m<n$, we have by definition of the order of  $g$ that
    either $\ord{g}=m$, or $(\ord{g})|m$.
\end{proof}

\begin{theorem}\label{2.3.5}
    If $G$ and  $H$ are finite cyclic groups with  $\ord{G}=\ord{H}$, then $G
    \simeq H$.
\end{theorem}
\begin{proof}
    Let $G=\vbrack{g}$ and $H=\vbrack{h}$. Take the map $phi:\vbrack{g}
    \rightarrow \vbrack{h}$ via the rule  $g^k \rightarrow h^k$, for some $k \in
    \N$. We have that
    $\phi(g^mg^n)\phi(g^{m+n})=h^{m+n}=h^mh^n=\phi(g^m)\phi(g^n)$. So $\phi$
    defines a homomorphism. Moreover, if $\ord{G}=\ord{H}=n$, and $g^s=g^t$ for
     $s,t \in \Z^+$,  $g^{s-t}=e$, so $n|s-t$, that is  $s \equiv t \mod{n}$.
     Thus $\phi(g^s)=\phi(g^t)$. This makes $\phi$ well defined.

     Now, by definition, every element of  $\vbrack{h}$ is of the form
     $h^k=\phi(g^k)$ for some $k \in \N$, this makes  $\phi$ onto. Since  $\phi$
     is onto, and  $\ord{G}=\ord{H}$ we get that $\phi$ is  $1-1$, and so an
     isomorphism.
\end{proof}
\begin{corollary}
    If $\vbrack{g}$ is an infinite cyclic group, then $\Z \simeq \vbrack{g}$.
\end{corollary}
\begin{proof}
    Define $\phi:\Z \rightarrow \vbrack{g}$ by $m \rightarrow g^m$. Then
    $\phi(m+n)=g^{m+n}=g^mg^n=\phi(m)\phi(n)$, moreover, if $m=n$, then
    $\phi(m)=g^m=g^n=\phi(n)$; so $\phi$ is a well defined homomorphism.

    Now if  $\phi(m)=\phi(n)$, then $g^m=g^n$, that is  $g^{m-n}=e$, so $m-n=0$,
    hence  $m=n$.  $\phi$ is  $1-1$.  $\phi$ is also onto by definitionm
    therefore  $\phi$ is an isomorphism of  $\Z$ onto  $\vbrack{g}$.
\end{proof}

\begin{example}
    \begin{enumerate}
        \item[(1)] In $D_{2n}$, $\vbrack{r} \simeq \faktor{\Z}{n\Z}$, and
            $\vbrack{t}=\{e,t\} \simeq \faktor{\Z}{2\Z}$.

        \item[(2)] If $G$ is any finite cyclic group of  $\ord{G}=n$, then $G
            \simeq \faktor{\Z}{n\Z}$.
    \end{enumerate}
\end{example}

\begin{lemma}\label{2.3.6}
    If $G$ is a group, and  $g \in G$, and  $k \in \Z^*$, then the following are
    true:
    \begin{enumerate}
        \item[(1)] If $g$ is of infinite order, then so is  $g^k$.

        \item [(2)] If $\ord{g}=n$, then $\ord{g^k}=\frac{n}{(n,k)}$.
    \end{enumerate}
\end{lemma}
\begin{proof}
    Suppose that $g$ is of infinite order, but that  $\ord{g^k}=m$ for some $m
    \in \Z^+$. Then  $(g^k)^m=g^{km}=e$. Now, either $km>0$, or  $-km>0$, thus,
    we get  $\ord{g} \leq km$ or $\ord{g} \leq -km$. Both contradict the
    infinite order of $g$.

    Now let  $\ord{g}=n$, and let $h=g^k$ and define  $d=(n,k)$. Then $n=dm$ and
     $k=dl$ for  $m,l \in \Z$,  $m>0$. Then we have $na+kb=d$, for $a,b \in \Z$;
     this implies that  $ma+lb=1$, so  $(m,l)=1$. Now, let $\ord{h}=p$, then
     $h^m=g^{km}=g^{dlm}=(g^{dm})^l=(g^n)^l=e$, so $p|m$. On the other hand,
     since  $(m,l)=1$, we get $m|p$, thus  $p=\ord{h}=m$. Since $m=\frac{n}{d}$,
     we get the result.
\end{proof}
\begin{corollary}
    Of $k|n$, then  $\ord{g^k}=\frac{n}{k}$.
\end{corollary}
\begin{proof}
    If $k|n$, then  $(n,k)=k$.
\end{proof}

\begin{lemma}\label{2.3.7}
    Let $\vbrack{g}$ be a cyclic group, then:
    \begin{enumerate}
        \item[(1)] If $g$ is of infinite order, then  $\vbrack{g}=\vbrack{g^k}$
            if, and only if $k=\pm 1$.

        \item [(2)] If $\ord{g}=n$, then $\vbrack{g}=\vbrack{g^k}$ if, and only
            if $(n,k)=1$.
    \end{enumerate}
\end{lemma}
\begin{proof}
    First, if $k=\pm 1$, tjem  $g^k=g$ or  $g^k=\inv{g}$. By lemma \ref{2.3.1},
    we get the result. Now suppose $\vbrack{g}=\vbrack{g^k}$ for $k>1$. Then
    $g=g^k$ for some  $k>1$. If  $k$ is odd, then  $k=2l+1$ and if  $k$ is even,
     $k=2l$ for  $l \in \Z^+$. Then  $g=g^{2l+1}=g^{2l}g$, making $g^{2l}=e$. On
     the otherhand, if $g=g^k=g^{2l}$, then $g^{2l}\inv{g}=g^{2l-1}=e$. Both
     these cases contradict the infinite order of $g$. So  $k \leq 1$. Now,
     since  $\vbrack{g}=\vbrack{e}$ cannot happen, $k \neq 0$. Now if $k<-1$,
     then we get the same result using  $-k$. Thus either  $k=1$ or  $k=-1$.

     Now suppose that  $\ord{g}=n$. Then $g^k$ generates a subgroup of
     $\ord{g^k}=\frac{n}{(n,k)}$. Now $\vbrack{g}=\vbrack{g^k}$ if, and only if
     $\ord{g}=\ord{g^k}$. That is, if and only if  $n(n,k)=n$, i.e. if, and only
     if $(n,k)=1$.
\end{proof}
\begin{corollary}
    The number of generators of $\vbrack{g}$ is $\phi(n)$, the Euler-$\phi$
    function.
\end{corollary}

\begin{example}
    \begin{enumerate}
        \item[(1)] Any $k \in \faktor{\Z}{n\Z}$, coprime with $n$ generates
            $\faktor{\Z}{n\Z}$. So the generators of $\faktor{\Z}{n\Z}$ are the
            elements of $U(\faktor{\Z}{n\Z})$.

        \item[(2)] $\faktor{\Z}{6\Z}=\vbrack{1}=\vbrack{5}$.

        \item[(3)]
            $\faktor{\Z}{12\Z}=\vbrack{1}=\vbrack{5}=\vbrack{7}=\vbrack{11}$.
    \end{enumerate}
\end{example}

\begin{theorem}\label{2.3.8}
    Let $G=\vbrack{g}$ be a cyclic group. The following are true:
    \begin{enumerate}
        \item[(1)] Every subgroup of $\vbrack{g}$ is cyclic, and has the form
            $\vbrack{g^d}$, with $0 \leq d < \ord{g}$.

        \item[(2)] If $\vbrack{g}$ is infinite, then for any $m,n \in \Z^+$
            distinct,  $\vbrack{g^m} \neq \vbrack{g^n}$. Moreover,
            $\vbrack{g^m}=\vbrack{g^{|m|}}$, and there is a $1-1$ map of
            subgroups of $\vbrack{g}$ onto $\N$.

        \item[(3)] If $\ord{\vbrack{g}}=n$, then for each divisor $k$ of  $n$,
            there is a unique subgroup of order  $k$, which is  $\vbrack{g^d}$
            where $d=\frac{n}{k}$.
    \end{enumerate}
\end{theorem}
\begin{proof}
    First, let $H \leq \vbrack{g}$, if $K=\vbrack{e}$, we are done. Otherwise,
    there is some $k \neq 0$ with  $g^k \in K$. If  $k<0$, then  $g^{-k} \in K$.
    Npw, define $P=\{l \in \Z^+ : g^l \in K\}$. We have by above that $P$ is
    nonempty, thus by the Well Ordering Principle,  $P$ has a least element,
    $d$. Now,  $g^d \in K$ and  $K \leq \vbrack{g}$, so $\vbrack{g^d} \leq K$.
    Now, for any $g^k \in K$, by the division theorem, we have  $k=qd+r$,
    $0 \leq r < d$, with  $q, r \in \Z$. Then  $g^r=g^{k-qd}=g^k(g^d)^{-q}$.
    Since $g^k,g^d \in K$, by the minimality of  $d$, we must have  $r=0$. So
    $k=qd$, this make $g^k=(g^d)^q \in \vbrack{g^d}$ and hence $K \leq
    \vbrack{g^d}$. Thus $K$ is cyclic. Moreover, if $\ord{g}=n$, and if $d>n$,
    then by the division theorem  $d \equiv r \mod{n}$, hence there are  $n$
    subgroups of $\vbrack{g}$.

    Now, if $\vbrack{g}$ is infinite, and if $\vbrack{g^m}=\vbrack{g^n}$, then
    $g^m=g^n$, so $g^{m-n}=e$, implying that $g$ has finite order; hence
    $\vbrack{g}$ has finite order. This cannot happen, so  $\vbrack{g^m} \neq
    \vbrack{g^n}$. Morevoer, we get $\vbrack{g^m}=\vbrack{g^{-m}}$, so
    $\vbrack{g^m}=\vbrack{g^{|m|}}$.

    Now, define $\phi:m \rightarrow \vbrack{g^m}$. By above, we get $\phi$ is
    $1-1$ and onto, so we have established  a  $1-1$ correspondance between the
    subgroups of  $\vbrack{g}$ onto $\N$.

    Finally, let $\ord{\vbrack{g}}=n$ and let $k|n$. Then letting
    $d=\frac{n}{(n,k)}=\frac{n}{k}$ we have $\vbrack{g^d}$ is a subgroup of
    order $\ord{\vbrack{g^d}}=k$. Now, suppose $K$ is any other subgroup of
    $\ord{K}=k$. Then $K=\vbrack{g^l}$, where $l \in \Z^+$ is the smallest
    integer of the set $P$ in the above arguments. Then
    $\ord{K}=\frac{n}{(n,l)}=\frac{n}{d}$, so $\frac{n}{d}=\frac{n}{(n,l)}$ so
    $d=(n,l)$. In particular, $d|l$, so  $g^l \in \vbrack{g^d}$ and since
    $\ord{K}=k$, this makes $K=\vbrack{g^d}$.
\end{proof}

\begin{example}
    \begin{enumerate}
        \item[(1)] The subgroups of $\faktor{\Z}{12\Z}$ are:
            \begin{align*}
                \faktor{\Z}{12\Z} &= \vbrack{1}=\vbrack{5}=\vbrack{7}=\vbrack{11} \\
                \vbrack{2}        &= \vbrack{10}    \\
                \vbrack{3}        &= \vbrack{9}     \\
                \vbrack{4}        &= \vbrack{8}     \\
                \vbrack{6}                          \\
                \vbrack{0}                          \\
            \end{align*}

        \item[(2)] Let $G$ be any group, and let  $g \in G$. We have that
            $C(g)=C(\vbrack{g})$ and $\vbrack{g} \leq N(\vbrack{g})$.
    \end{enumerate}
\end{example}
