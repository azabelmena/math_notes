\section{Lattices of Groups.}
\label{section1}

\begin{definition}
    Let $G$ be a group. We define the  \textbf{lattice} of subgroups of $G$ to
    be a directed graph whose set of vertices are subgroups of  $G$, and whose
    edges are given by the relation $(H,K)$ if $H \leq K$, for  $H,K$ subgroups
    of  $G$.
\end{definition}

We detail an algorithm for generating the lattice of subgroups of a given group
$G$.

\begin{algorithm}
    For any group $G$:
    \begin{enumerate}
        \item[\textbf{step 1:}] Plot $G$ and  $\vbrack{e}$ opposite each other.

        \item[\textbf{step 2:}] Plot a gubgroup $H$ of  $G$. If there are no
            subgroups $A$ such that $H \leq A \leq G$, connect  $H$ to  $G$. If
            there are no subgroups $A$ such that  $\vbrack{e} \leq A \leq H$,
            connect $H$ to  $\vbrack{e}$. If there are no subgroups left to
            plot, go to \textbf{step 5}.

        \item[\textbf{step 3:}] For any other subgroup $K$ plotted on the graph,
            if there is no subgroup $A$ such that  $H \leq A \leq K$, or $K \leq
            A \leq H$, then connect  $K$ to  $H$.

        \item[\textbf{step 4:}] return to \textbf{step 2}.

        \item[\textbf{step 5}] return.
    \end{enumerate}
\end{algorithm}

The lattice of subgroups of a group, can also be viewed as a lattice of
partially ordered sets, instead of a graph. We focus however on the graph
theoretic apprach. We now ist numerous examples.

\begin{example}\lable{2.11}
    \begin{enumerate}
        \item[(1)] For $\faktor{\Z}{m\Z}=\vbrack{1}$, we plot the lattice of
            subgroups for $m=2,4,12, p^n$ where $p$ prime and $n \in \Z^+$:
            \begin{figure}[h]
\[\begin{tikzcd}
	&&&&& {\langle 1 \rangle} \\
	&&&&& {\langle p \rangle} \\
	& {\langle 1 \rangle} && {\langle 1 \rangle} && {\langle p^2 \rangle} \\
	& {\langle 2 \rangle} & {\langle 2 \rangle} && {\langle 3 \rangle} & \vdots \\
	{\langle 1 \rangle} & {\langle 4 \rangle} & {\langle 4 \rangle} && {\langle 6 \rangle} & {\langle p^n \rangle} \\
	{\langle 0 \rangle} & {\langle 0 \rangle} && {\langle 0 \rangle} && {\langle 0 \rangle}
	\arrow[no head, from=5-1, to=6-1]
	\arrow[no head, from=5-2, to=4-2]
	\arrow[no head, from=4-2, to=3-2]
	\arrow[no head, from=4-3, to=5-3]
	\arrow[no head, from=5-3, to=6-4]
	\arrow[no head, from=6-4, to=5-5]
	\arrow[no head, from=5-5, to=4-5]
	\arrow[no head, from=4-5, to=3-4]
	\arrow[no head, from=3-4, to=4-3]
	\arrow[no head, from=4-3, to=5-5]
	\arrow[no head, from=6-2, to=5-2]
	\arrow[no head, from=2-6, to=3-6]
	\arrow[no head, from=3-6, to=4-6]
	\arrow[no head, from=4-6, to=5-6]
	\arrow[no head, from=5-6, to=6-6]
	\arrow[no head, from=1-6, to=2-6]
\end{tikzcd}\]
                \caption{The lattices of $\faktor{\Z}{2\Z}$, $\faktor{\Z}{8\Z}$,
                $\faktor{\Z}{12\Z}$, and  $\faktor{\Z}{p^n\Z}$ respectively.}
                \label{fig_2.1}
            \end{figure}
            Notice that only the lattice of $\faktor{\Z}{12\Z}$ conatains any
            cycles.

        \item[(2)] The lattice of subgroups of $D_8=\vbrack{r,t}$ is that of
            figure \ref{fig_2.2}
            \begin{figure}[h]
\[\begin{tikzcd}
	&& {\langle t,r \rangle} \\
	{\langle t,r^2 \rangle} && {\langle r \rangle} && {\langle r^2,rt \rangle} \\
	&& {\langle r^2 \rangle} \\
	{\langle t \rangle} & {\langle r^2t \rangle} && {\langle r^3t \rangle} & {\langle rt \rangle} \\
	&& {\langle e \rangle}
	\arrow[no head, from=1-3, to=2-1]
	\arrow[no head, from=2-1, to=3-3]
	\arrow[no head, from=3-3, to=2-5]
	\arrow[no head, from=2-5, to=1-3]
	\arrow[no head, from=1-3, to=2-3]
	\arrow[no head, from=2-3, to=3-3]
	\arrow[no head, from=3-3, to=5-3]
	\arrow[no head, from=2-1, to=4-1]
	\arrow[no head, from=4-1, to=5-3]
	\arrow[no head, from=5-3, to=4-5]
	\arrow[no head, from=4-5, to=2-5]
	\arrow[no head, from=2-5, to=4-4]
	\arrow[no head, from=4-4, to=5-3]
	\arrow[no head, from=5-3, to=4-2]
	\arrow[no head, from=4-2, to=2-1]
\end{tikzcd}\]
                \caption{The lattices of $D_8$.}
                \label{fig_2.2}
            \end{figure}
            This lattice can be drawn as a planar graph, however it is not true
            that all lattices of $D_8$ are planar. The lattice of $D_{16}$
            for example, cannot be drawn as a planar graph.

        \item[3] The lattice of $S_3$ is the attice in figure \ref{fig_2.3}
            \begin{figure}[h]
\[\begin{tikzcd}
	&& {S_3} \\
	&&&& {\langle (1 \ 2 \ 3)\rangle} \\
	{\langle (1 \ 2)\rangle} & {\langle (1 \ 3)\rangle} & {\langle (2 \ 3)\rangle} \\
	\\
	&& {\langle (1)\rangle}
	\arrow[no head, from=1-3, to=3-3]
	\arrow[no head, from=1-3, to=2-5]
	\arrow[no head, from=2-5, to=5-3]
	\arrow[no head, from=3-3, to=5-3]
	\arrow[no head, from=1-3, to=3-2]
	\arrow[no head, from=3-2, to=5-3]
	\arrow[no head, from=1-3, to=3-1]
	\arrow[no head, from=3-1, to=5-3]
\end{tikzcd}\]
                \caption{The lattices of $S_3$.}
                \label{fig_2.3}
            \end{figure}

        \item[(4)] The lattice of $U(\faktor{\Z}{12\Z})$ is that of figure
            \ref{fig_2.4}
            \begin{figure}[h]
\[\begin{tikzcd}
	& {\langle 5,7 \rangle} \\
	{\langle 5 \rangle} & {\langle 7 \rangle} & {\langle 11 \rangle} \\
	& {\langle 1 \rangle}
	\arrow[no head, from=1-2, to=2-2]
	\arrow[no head, from=1-2, to=2-1]
	\arrow[no head, from=2-1, to=3-2]
	\arrow[no head, from=3-2, to=2-3]
	\arrow[no head, from=2-3, to=1-2]
	\arrow[no head, from=2-2, to=3-2]
\end{tikzcd}\]
                \caption{The lattices of $U(\faktor{\Z}{12\Z})$.}
                \label{fig_2.4}
            \end{figure}

        \item[(5)] The \textbf{Klein-$4$ group} is the Abelian group $V_4$
            defined by the following Cayley table:
            \begin{equation*}
                \begin{matrix}
                    1 & a & b & c \\
                    a & 1 & c & b \\
                    b & c & 1 & a \\
                    c & b & a & 1 \\
                \end{matrix}
            \end{equation*}
            and has the lattice of figure \ref{fig_2.5}.
            \begin{figure}[h]
\[\begin{tikzcd}
	& {V_4} \\
	{\langle a \rangle} & {\langle b \rangle} & {\langle c \rangle} \\
	& {\langle e \rangle}
	\arrow[no head, from=1-2, to=2-2]
	\arrow[no head, from=1-2, to=2-1]
	\arrow[no head, from=2-1, to=3-2]
	\arrow[no head, from=3-2, to=2-3]
	\arrow[no head, from=2-3, to=1-2]
	\arrow[no head, from=2-2, to=3-2]
\end{tikzcd}\]
                \caption{The lattices of $V_4$.}
                \label{fig_2.5}
            \end{figure}
            Notice that as a graph, the lattice of subgroups of $V_4$ is
            isomorphic to the lattice of subgorups of $U(\faktor{\Z}{12\Z})$.
    \end{enumerate}
\end{example}

These examples provide intereseting notions. The first is the notion of the
``join'' of two subgroups.

\begin{definition}
    For any two distinct subgroups $H$ and $K$ of a group, the \textbf{join} of
    $H$ and  $K$ is the smallest subgroup, $vbrack{H,K}$ containing both  $H$
    and  $K$.
\end{definition}

\begin{lemma}\label{2.5.1}
    Let $H$ and  $K$ be distinct subgroups of a group. Then the join
    $\vbrack{H,K}$ is unique.
\end{lemma}
\begin{proof}
    Supose there exists another smallest subgroup containing $H$ and $K$, and
    call it $A$. Then  $A \subseteq \vbrack{H,K}$. But by definition,
    $\vbrack{H,K}$ is minimal, thus it must be contained in $A$ as well, so
    $\vbrack{H,K} \subseteq A$. This establishes equality, and hence uniqueness.
\end{proof}

\begin{lemma}\label{2.5.2}
    If $G$ and  $H$ are isomorphic groups, then they have the same lattice of
    subgroups.
\end{lemma}
