%----------------------------------------------------------------------------------------
%	SECTION 1.1
%----------------------------------------------------------------------------------------

\section{Special Subgroups.}
\label{section1}

We introduce now, some very important examples of subgroups.

\begin{definition}
    Let $G$ be a group. We define the \textbf{centralizer} of an element $a \in
    G$ to be the set $C(a)=\{g \in G: ga\inv{g}=a\}$. We define the
    \textbf{centralizer} of a nonemppty subset $A$ of  $G$ to be the set
    $C(A)=\{g \in G : ga\inv{g}=a, \text{ for all } a \in A\}$.
\end{definition}

\begin{lemma}\label{2.2.1}
    Let $G$ be a group. Then for  $a \in G$,  $C(a) \leq G$. Likewise, for $A
    \subseteq G$ nonempty, $C(A) \leq G$.
\end{lemma}
\begin{proof}
    Notice that given $a \in A$,  $C(a) \subseteq C(A)$. Then we have that $C(a)
    \nonempty$, for $ea\inv{e}=eae=a$, so $e \in C(a)$, this also implies that
    $e \in C(A)$.

    Now let $x,y \in C(a)$ then $xa\inv{x}=a$, and $ya\inv{y}=a$. Notice then
    that  $a=\inv{y}ay=\inv{y}a\inv{(\inv{y})}$, so we have $\inv{y} \in C(a)$.
    Then
    $(xy)a\inv{(xy)}=xya\inv{y}\inv{x}=x(ya\inv{y})\inv{x}=x(y\inv{y}a\inv{y}y)
    \inv{x}=xa\inv{x}=a$, so $xy \in C(a)$, thus $C(a) \leq G$, if we take $a
    \in A$ arbitrary, this makes  $C(A) \leq G$ as well.
\end{proof}
\begin{corollary}
    $C(a) \leq C(A)$ for any $a \in A$.
\end{corollary}
\begin{corollary}
    $C(A)=\bigcap_{a \in A}{C(a)}$.
\end{corollary}
\begin{remark}
    In the abve corollary, notice that $\bigcup{C(a)}$ is not necessarily a
    disjoint union.
\end{remark}

\begin{example}
    Let $G$ be abelian, then  $C(G)=G$.
\end{example}

\begin{definition}
    Let $G$ be a group. We define the  \textbf{center} of $G$ to be the set
    $Z(G)=\{g \in G : gx=xg \text{ for all } x \in G\}$.
\end{definition}

\begin{lemma}\label{2.2.2}
    Let  $G$ be a group. Then  $Z(G) \leq G$.
\end{lemma}
\begin{proof}
    Notice that if $g \in Z(G)$, then for any $x \in G$, $gx=xg$ implies that
    $gx\inv{g}=g$, making $g \in C(G)$; likewise, $g \in C(G)$ implies
    $gx\inv{g}=x$ which implies that $xg=gx$, for any  $x \in G$; so  $g \in
    Z(G)$. That is, $Z(G)=C(G)$ which makes $Z(G)$ a subgroup.
\end{proof}

\begin{definition}
    Let $G$ be a group and let  $A \subseteq A$. Define $gA\inv{g}=\{ga\inv{g} :
    a \in A\}$, where $g \in G$. We define the \textbf{normalizer} of $A$ to be
    the set  $N(A)=\{g \in G : gA\inv{g}=A\}$.
\end{definition}

\begin{lemma}\label{2.2.3}
    Let $G$ be a group, and let  $A \subseteq G$. Then  $N(A) \leq G$.
\end{lemma}
\begin{proof}
    Let $x,y \in N(A)$, then $xA\inv{x}=A$ and $yA\inv{y}=A$, then for any  $a
    \in A$, then for some $a,b \in A$, $xa\inv{a}=b$ and $ya\inv{a}=b$. Then
    $(xy)a\inv{(xy)}=x(ya\inv{y})\inv{x}=xb\inv{x}=b$, thus $xy \in N(A)$.
    Similarly, $xa\inv{x}=b$ implies $a=\inv{x}bx$, thus $\inv{x} \in N(A)$.
    This makes $N(A) \leq G$.
\end{proof}
\begin{corollary}
    $C(A) \leq N(A)$.
\end{corollary}

\begin{example}
    \begin{enumerate}
        \item[(1)] If $G$ is abelian, then  $ab=ba$ for all  $a,b \in G$, thus
            $G=Z(G)$. Similarly, we get $ga\inv{g}=g\inv{g}a=a$ for all $a \in A
            \subseteq G$ and  $g \in G$, thus $C(A)=N(A)=G$.

        \item[(2)] Consider the dihedral group $D_{2n}$. Let $A=\{e,r, \dots,
            r^{n-1}\} \leq D_{2n}$. Then $C(A)=A$, We have that $A \subseteq
            C(A)$, since $r^jr^ir^{-j}=r^{j+i-j}=r^i$. On the other hand we have
            $(r^jt)r^i\inv{(r^jt)}=r^{j+i-j}t^2=r^i$, which makes $C(A)
            \subseteq A$. Moreover, $N(A)=D_{2n}$, since by the above
            computations we also get $D_{2n} \subseteq N(A)$.

        \item[(3)] In $D_{2n}$, $Z(D_{2n})=\{r^i : r^-=r^{-i}\}$, where $i \in
            \faktor{\Z}{n\Z}$. So in $D_8$, $Z(D_8)=\{e, r^2\}$. Essentially, to
            find the center of $D_{2n}$, find all those powers $i$ of $r$ for
            which  $i \equiv -i \mod{n}$.

        \item[(4)] Let $A=\{(1), (1 \ 2)\} \leq S_3$. Then $C(A)=N(A)=A$.
            Moreover, $Z(S_3)=\vbrack{(1)}$. Notice that since $S_3 \simeq D_6$, then
            to preserve the group structure,  $Z(S_3) \simeq Z(D_6)$. Notice
            then that $Z(D_6)=\vbrack{e}$.
    \end{enumerate}
\end{example}

We can treat the fact that, for a subset $A$ of a group  $G$, the normalizer and
centralizer of  $G$, and the center of  $G$ are all special cases of group
actions.

\begin{definition}
    If $G$ is a group acting on a set  $A$, then we define the
    \textbf{stabalizer} of $a \in A$ in $G$ to be the set  $\stab{a}=\{g \in G :
    ga=a\}$. We define the \textbf{stabalizer} of $A$ to be  $\stab{A}=\{g \in G
: ga=a \text{ for all } a \in a\}.$
\end{definition}

\begin{lemma}\label{2.2.4}
    Let $G$ be group acting on a set  $S$. Then for any  $s \in S$,  $\stab{s}
    \leq G$.
\end{lemma}
\begin{proof}
    For $a,b \in \stab{s}$, we have $(ab)s=a(bs)=as=s$, so $ab \in \stab{a}$.
    Likewise, $\inv{a} \in \stab{s}$.
\end{proof}
\begin{corollary}
    $\ker{S} \leq G$.
\end{corollary}
\begin{proof}
    The proof is identical to that of the above lemma.
\end{proof}
\begin{corollary}
    $\stab{s} \leq \ker{s}$.
\end{corollary}
\begin{proof}
    We have that both $\stab{s}$ and $\ker{s}$ are both groups. Then notice that
    since $s \in S$, then  $\stab{s} \subseteq \ker{s}$.
\end{proof}
\begin{corollary}
    $\ker{s}=\bigcap_{s \in S}{\stab{s}}$.
\end{corollary}

\begin{example}
    \begin{enumerate}
        \item[(1)] Consider the dihedral group $D_8$ of symmetries of a square
            acting on the labeling $\faktor{\Z}{4\Z}$ of vertices. Then
            $\stab{i}=\{e,r^2t\}$, where  $r^2t$ is the reflection of the square
            about the line crossing the vertex  $i$ and the center of the
            square.

        \item [(2)] In general, consider the dihedral group $D_{2n}$ of
            symmetries of an $n$-gon acting on the labeling  $\faktor{\Z}{n\Z}$
            of vertices. Notice that if  $r^jt \in \stab{i}$, for any $i \in
            \faktor{\Z}{n\Z}$, then $(r^jt)i=i$. Then $n+j-i=i$, i.e. $j \equiv
            2i \mod{n}$, and so $\stab{i}=\{r^jt : j \equiv 2i \mod{n}\}=\{e,
            r^2t, r^4t, \dots, r^{2n-2}t\}$.

        \item[(3)]
    \end{enumerate}
\end{example}

\begin{definition}
    Let $G$ be a group acting on a set $A$ We define the \textbf{conjugation}
    action of $G$ on  $A$ to be the map $G \times A \rightarrow A$
    defined by $(g,a) \rightarrow ga\inv{g}$. We define the \textbf{conjugation}
    of $A$, $gA\inv{g}$ to be the image of this action.
\end{definition}

\begin{lemma}\label{2.2.5}
    Conjugation is a group action.
\end{lemma}
\begin{proof}
    Let $G$ be  and $A$ a nonempty set. Take the map  $G \times A \rightarrow A$
    via $(g,a) \rightarrow ga\inv{g}$. Then for $e \in G$ the identity,
    $(e,a) \rightarrow ea\inv{a}=a$, now for $x,y \in G$,  $(xy,a) \rightarrow
    (xy)a\inv{(xy)}=x(ya\inv{a})\inv{x}=(x,(y,a))$. Thus this map is a group
    action, and $G$ acts on $A$ via conjugation.
\end{proof}

\begin{lemma}\label{2.2.6}
    Let  $G$ be a group acting on a set $A \subseteq G$ via conjugation. Then
    $\stab{A}=N(A)$.
\end{lemma}
