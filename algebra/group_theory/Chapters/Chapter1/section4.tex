\section{The General and Special Linear Groups of $n \times n$ Matrices.}
\label{section1}

One special class of groups are those that can be defined on matrices. We first
need to define, in an elementary sense what a ``field'' is; though we will not
go into their study here. We assume familiarity with matrix algebra such as
matrix multiplication and determinants. This makes this section, in a sense,
optional.

\begin{definition}
    Let $F$ be a set together with binary operations  $+$, called
    \textbf{addition} and $\cdot$, called  \textbf{multiplication}. We call
    $(F,+,\cdot)$ a \textbf{field} if:
    \begin{enumerate}
        \item[(1)] $(F,+)$ forms an abelian group.

        \item[(2)] $(F^*,\cdot)$ forms an abelian group; where
            $F^*=\com{F}{\{e\}}$, $e$ is the identity of  $F$ under  $+$.

        \item[(3)] $\cdot$  \textbf{distributes} over $+$; that is, for  $a,b,c
            \in F$,  $a(b+c)=ab+ac$.
    \end{enumerate}
\end{definition}

\begin{example}
    \begin{enumerate}
        \item[(1)] The sets $\Q$ and  $\R$ are fields under the usual addition
            and multiplication.

        \item[(2)]  $\C$ is a field under complex addition and complex
            multiplication. So is  $\R$ if we take all $a \in \R$ to have the
            form  $a+i0$.

        \item[(3)] $\faktor{\Z}{p\Z}$, with $p \in \Z^+$ prime forms a field
            under addition and multiplication  $\mod{p}$.
    \end{enumerate}
\end{example}

\begin{definition}
    Let $F$ be a field. We define  $F^{n \times n}$ to be the field of all $n
    \times n$ matrices with entries in $F$. We define the \textbf{general linear
    group} to be $GL(n,F)=\{A \in F^{n \times n} : \det{A} \neq 0\}$. We define
    the \textbf{special linear group} to be $SL(n,F)=\{A \in F^{n \times n} :
    \det{A}=1\}$. If $F=\faktor{\Z}{p\Z}$, we write
    $GL(n,\faktor{\Z}{p\Z})=GL(n,p)$ and $SL(n,\faktor{\Z}{p\Z})=SL(n,p)$.
\end{definition}

\begin{theorem}\label{1.5.1}
    For any field $F$, and  $n \in \Z^+$,  $GL(F,n)$ forms a group under matrix
    multiplication.
\end{theorem}
\begin{proof}
    Let $A,B \in GL(F,n)$ be $n \times n$ matrices. Then  $\det{A} \neq 0$ and
    $\det{B} \neq 0$, so $det{AB}=\det{A}\det{B} \neq 0$, by a well known
    property of determinants. So $GL(F,n)$ is closed. Now since matric
    multiplication is associative, then  $GL(F,n)$ satisfies the associative
    law.

    Now consider the $n \times n$ identity matrix $I$, we have for any $A \in
    GL(F,n)$, $AI=IA=A$, moreover,  $\det{I}=1 \neq 0$ making $I \in GL(F,n)$.
    Likewise, since for $A \in GL(F,n)$, $\det{A} \neq 0$, $A$ is invertible, by
    well known properties of matrices, so  $\inv{A}$ exists, and
    $\det{\inv{A}}=\det{A} \neq 0$. Thus $\inv{A} \in GL(F,n)$ and since
    $A\inv{A}=\inv{A}A=I$, this makes $\inv{A}$ the inverse of $A$.
\end{proof}
\begin{corollary}
    $SL(F,n)$ forms a group under matrix multiplication.
\end{corollary}
\begin{proof}
    Notice that $SL(F,n) \subseteq GL(F,n)$, so $SL(F,n)$ inherits closure (and
    associativity). Now, for $A \in SL(F,n)$, $A \in GL(F,n)$, so $\inv{A}$
    exists. Moreover, $\det{\inv{A}}=\det{A}=1$, making $\inv{A} \in SL(F,n)$.
    This also implies that $I \in SL(F,n)$.
\end{proof}

\begin{example}
    \begin{equation*}
        GL(2,2)=\{
            \begin{pmatrix}
                1   &   0   \\
                0   &   1   \\
            \end{pmatrix},
            \begin{pmatrix}
                0   &   1   \\
                1   &   0   \\
            \end{pmatrix},
            \begin{pmatrix}
                1   &   1   \\
                0   &   1   \\
            \end{pmatrix},
            \begin{pmatrix}
                1   &   1   \\
                1   &   0   \\
            \end{pmatrix},
            \begin{pmatrix}
                0   &   1   \\
                1   &   1   \\
            \end{pmatrix},
            \begin{pmatrix}
                1   &   0   \\
                1   &   1   \\
            \end{pmatrix}
        \}
    \end{equation*}
    Labeling these elements as $I$,  $A$,  $B$,  $C$,  $D$, and  $E$,
    consecutively we find the orders to be:  $\ord{I}=1$, $\ord{A}=2$,
    $\ord{B}=2$, $\ord{C}=3$, $\ord{D}=3$, and $\ord{E}=2$.
\end{example}

\begin{example}
    Consider $GL(2,2)$, and considering the labeling of the above example, we
    compute the Cayeley table to be:
    \begin{equation*}
        \begin{pmatrix}
            I   &   A   &   B   &   C   &   D   &   E   \\
            A   &   I   &   D   &   E   &   B   &   C   \\
            B   &   C   &   I   &   A   &   E   &   D   \\
            C   &   B   &   E   &   D   &   I   &   A   \\
            D   &   E   &   I   &   I   &   C   &   I   \\
            E   &   D   &   C   &   B   &   A   &   I   \\
        \end{pmatrix}
    \end{equation*}
    which is not symmetirc, hence $GL(2,2)$ is not Abelian. In general, for $n,
    p \in \Z^+$ and $p$ prime, for  $A,B \in GL(n,p)$ we have
    \begin{equation*}
        \begin{pmatrix}
            A   &   0   \\
            0   &   1   \\
        \end{pmatrix}
        \begin{pmatrix}
            B   &   0   \\
            0   &   1   \\
        \end{pmatrix}=
        \begin{pmatrix}
            AB  &   0   \\
            0   &   1   \\
        \end{pmatrix}
    \end{equation*}
    while
    \begin{equation*}
        \begin{pmatrix}
            B   &   0   \\
            0   &   1   \\
        \end{pmatrix}
        \begin{pmatrix}
            A   &   0   \\
            0   &   1   \\
        \end{pmatrix}=
        \begin{pmatrix}
            BA  &   0   \\
            0   &   1   \\
        \end{pmatrix}
    \end{equation*} then
    \begin{equation*}
        \begin{pmatrix}
            AB  &   0   \\
            0   &   1   \\
        \end{pmatrix}=
        \begin{pmatrix}
            BA  &   0   \\
            0   &   1   \\
        \end{pmatrix}
    \end{equation*}
    if, and only if $AB=BA$, which is in general, not true for matrices. So
    $GL(n,p)$ is not necesarrily Abelian.
\end{example}

Now, we would like to observe the order of the group $GL(n,p)$, the order of
$SL(n,p)$ will be derived later.

\begin{example}
    \begin{enumerate}
        \item[(1)] We have that if $F$ is a field with  $\ord{F}=p$, then
            $\ord{GL(n,F)}<p^{n^2}$, for, notice for any $A \in F^{n \times n}$,
            there are $n^2$ entries, and  $p$ choices for each entry, thus
            $\ord {F^{n \times n}}=n^2$, now, by definition,  $GL(n,F)$ excludes
            those with $\det=0$, thus we get the result.

        \item[(2)] Let $A \in GL(2,2)$ where:
            \begin{equation*}
                A=\begin{pmatrix}
                    a   &   b   \\
                    c   &   d   \\
                  \end{pmatrix}
            \end{equation*}
            where $a,b,c,d \in \faktor{\Z}{2\Z}$. Then we have that if $ad-bc
            \neq 0$, then  $ad \neq bc$, thus $a$ is a multiple of  $c$ and  $d$
            is a multiple of  $b$, let us consider the columns. We have that if
             $a=c=0$, then  $\det{A}=0$, thus $a$ and  $c$ cannot be both  $0$,
             also notice that there are $2^2$ possible choices for  $a$
             and  $c$, so the fist column, $\begin{pmatrix}a \\ c
             \\\end{pmatrix}$, hase $2^2-1$ possible choices. Now, obvserving
             column $\begin{pmatrix}b \\ d \\\end{pmatrix}$, we have the $2^2$
             choices for both entries, however, since  $b$ and  $d$ are
             multiples of eachother, we must exclude the $2$ choices for the
             multiples $ad$ and  $bc$. Thus the column $\begin{pmatrix}b \\ d
             \\\end{pmatrix}$ has $2^2-2$ choices. That is,
             $\ord{GL(2,2)}=(2^2-1)(2^2-2)=2 \cdot 3=6$.
    \end{enumerate}
\end{example}

Observing further, we can see that $\ord{GL(n,3)}=(3^n-1) \dots (3^n-3)$, and so
on. Thus we have:

\begin{theorem}\label{1.5.2}
    For $n, p \in \Z^+$ and  $p$ prime :
    \begin{equation}
        \ord{GL(n,p)}=\prod_{j=1}^{n-1}{(p^n-p^{n-j})}
    \end{equation}
\end{theorem}
\begin{proof}
    Consider the $n \times n$ matrix  $A=(a_{ij}) \in GL(n,p)$, observe that
    there are $p^n-1$ choices for the first column, $\begin{pmatrix}a_{11} \\
    \vdots \\ a_{n1}\end{pmatrix}$ since $a_{11}, \dots, a_{n1}$ cannot all be
    $0$. Now, we have $\det{A}=\sum_{j=1}^n{(-1)^{i+j}a_{ij}\det{A_{ij}}}$, with
    $A_{ij}$ the cofactor of $A$ about the entry $a_ij$. So, for the $j$^{th}
    column $\begin{pmatrix}a_{1j} \\ \vdots \\ a_{nj}\end{pmatrix}$, then
    $a_{ij}\det{A_{ij}}=\sum_{l=1}^{j-1}{(-1)^{i+j}a_{il}\det{A_{il}}}+
    \sum_{l=1}^{j+1}{(-1)^{i+j}a_{il}\det{A_{il}}}$, for which there are
    $p^n-p^j$ choices, given that each of the  $a_{ij}$ entries are multiples of
    the previous entries, for $1 \leq i \leq n$. Taking $2 \leq j \leq n$
    (since we already evaluated the first column), we get there are:
    \begin{equation*}
        \prod_{j=1}^{n-1}{(p^n-p^{n-j})}
    \end{equation*}
    choices for the matrix $A$. Since  $A$ is abritrary, we get the order of
    $GL(n,p)$.
\end{proof}

We also need to comment on the order of $GL(n,F)$ when the field $F$ is
infinite.

\begin{theorem}\label{1.5.3}
    For any field $F$,  $GL(n,F)$ is of infinite order if, and only if $F$ is
    of infinite order.
\end{theorem}
\begin{proof}
    We show by contrapositives. Suppose that $F$ is finite with  $\ord{F}=k$.
    Then by the same argument of theorem \ref{1.5.2}, we find there are
    $\prod{(k^n-k^j)}$ matrices $A \in GL(n,F)$. Any additional elements
    contradict this result, and so $\orf{GL(n,F)}=\prod{(k^n-k^j)}$.

    On the otherhand, if $\ord{GL(n,F)}=k$ then there are $k$ $n \times n$
    matrices over  $F$ with $\det \neq 0$. Now, if  $F$ were not finite, then
    there exists a distrinct matrix  $A \in GL(n,F)$, making $\ord{GL(n,F)}=k+1$
    a contradiciton. Thus, $F$ must be finite.
\end{proof}

We now introduce a seperate group from the general and special linear groups.

\begin{definition}
    Let $F$ be a field. We define the \textbf{Heisenberg} group over $F$ to be
    the set:
    \begin{equation}
        H(F)=\{
            \begin{pmatrix}
                1   &   a   &   b   \\
                0   &   1   &   c   \\
                0   &   0   &   1   \\
            \end{pmatrix} :
            a,b,c \in F
        \}
    \end{equation}
    That is, $H(F)$ is the set of all upper triangular matrices over $F$ with
    diagonal entries equal to $1$  (the identity element of $F$).
\end{definition}

\begin{lemma}\label{1.5.4}
    For any field $F$,  $H(F)$ is a group under matrix multiplication.
\end{lemma}
\begin{proof}
    Let $X=\begin{pmatrix}
                1   &   a   &   b   \\
                0   &   1   &   c   \\
                0   &   0   &   1   \\
           \end{pmatrix}$, and
         $Y=\begin{pmatrix}
                1   &   d   &   e   \\
                0   &   1   &   f   \\
                0   &   0   &   1   \\
           \end{pmatrix}$. Then
          $XY=\begin{pmatrix}
                1   &  a+d  &   e+af+b  \\
                0   &   1   &   f+c     \\
                0   &   0   &   1   \\
              \end{pmatrix}$.
    So $H(F)$ is closed. Additionally, $H(F)$ inherits the associativity of
    matrix multiplication.

    Now, we get that $I=\begin{pmatrix}
                            1    &   0   &   0   \\
                            0    &   1   &   0   \\
                            0    &   0   &   1   \\
                        \end{pmatrix}$
    serves as the identity, and the matrix $Y=\begin{pmatrix}
                                                1   &  -a   &   ac-b    \\
                                                0   &   1   &   -c      \\
                                                0   &   0   &   1       \\
                                              \end{pmatrix}$
    serves as an inverse to $X$. This makes  $H(F)$ into a group.
\end{proof}
\begin{corollary}
    $H(F)$ is non-Abelian.
\end{corollary}
\begin{corollary}
    $\ord{H(F)}=(\ord{F})^3$.
\end{corollary}
\begin{proof}
    Let $\ord{F}=k$, then we have $n$ choices for  $a$,  $b$, and  $c$, hence
    $n^3$ choices for an arbitrary matrix in  $H(F)$.
\end{proof}
\begin{corollary}
    $H(F)$ is finite if, and only if $F$ is finite.
\end{corollary}
