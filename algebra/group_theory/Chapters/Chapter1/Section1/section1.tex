%----------------------------------------------------------------------------------------
%	SECTION 1.1
%----------------------------------------------------------------------------------------

\section{Motiviations of Groups.}
\label{section1}

The notion of a ``group'' is perhaps the most fundamental notion in all of
abstract algebra (without going into more granular structures which tend to be
reserved for more advanced settings). It forms the building blocks for the study
of the ``ring'' and ``field'' structures prevalent in algebra, and it even forms
its (rightfully) interesting field of study.

THere are a number of motivations and examples for group theory and its study.
Historically, the definition of a group finds itself motivated by the study of
algebraic equations; specifically polnyomial equations of the form
$ax^2+bx+c=0$, $ax^3+bx^2+cx+d=0$, and more generally,
$a_nx^n+\dots+a_1x+a_0=0$. That is the so called quadratic, cubic, and other
polynomial equations. Specifically, the solutions of such equations were of
interest, and finding a general form for them proved invaluble. It is know that
the quadratic, $ax^2+bx+c=0$ has the general solution:
\begin{equation}
    x = -\frac{b}{2a} \pm \frac{\sqrt{b^2-4ac}}{2a}
\end{equation}
 dubbed the ``quadratic'' formula. The general solution for the cubic equation
 is far more convoluted. What about the solution for polynomial equations of
 higher degree? It was shown that the quintic equation (of degree 5) and higher
 has no general form for their solutions.

 The study of these polynomial equations and their geenral solutions led
 mathematicians such as Neils Abel and Evariste Galois to develop the
 fondations for algebra. In particular, Galois, on the eve of his duel composed
 a manuscript concerning these equations which would form the foundations for
 group theory in which others would build upon.

 There are other applications and motivations for group theory asides from
 polynomials. IN number theory, one is concerned with the properites of positive
 integers, which form a ``group'' structure. For example, if the integer
 $a^{40}$ has its last two digits $01$ if it is not congruent  $0 \mod{2}$, nor
 $0 \mod{5}$; i.e. it is not divisible by either $2$ nor  $5$. Leonhard Euler
 proved this using group theoretic techniques developed by Joseph Louis
 Lagrange.

 Group theory also plays an essential role in finding rational solutions  (i.e.
 solutions in $\Q$), to certain algebraic equations called Diophantine
 equations; $y^2=x^3-x$ is an example. Solutuions to these equations can be
 found by intersecting straight lines with these curves. The result is that if
 an arithmetic is defined on the points of thes curves using line intersections,
 the one can obtain a group sturcture. The most famous example of this is are
 elliptic curves, which have the form  $y^2=f(x)$ where $f$ is a polynomial.
 These curves form a group structure and have an intimite connection to number
 theory. Pierre de Fermat was interested in Diophantine equations, especially
 the equation $x^n+y^n=z^n$; which was claimed not to have solutions for  $n>2$,
 and to which Fermat scribbled in a copy of Diophantus' book: ``I have a
 marvelous proof to this theorem, but the margin is to small to contain it ''.
 This problem became known as ``Fermat's Last Theorem'' and went unproved until
 Andrew Wiles proved it using Elliptic curves (its much more complicated than
 that, but the story of Fermat's last theorem rightfully deserves its own book).

 The study of the arithmetic of Elliptic curve is an area of active research and
 has wide applications, specifally in the field of cryptography.
