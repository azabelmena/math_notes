%----------------------------------------------------------------------------------------
%	SECTION 1.1
%----------------------------------------------------------------------------------------

\section{Alternative Definitions.}
\label{section1}

There is another way for driving the theory for quotient groups that does not
involve the use of homomorphisms at all. Instead, it relies on the notion of
equivalence classes.

\begin{definition}
    Let $G$ be a group and  $H \leq G$ a subgroup. For $a,b \in G$, we say that
     $a$ is  \textbf{congruent} to $b$  $\mod{H}$ if $\inv{b}a \in H$. We write
     $a \equiv b \mod{H}$.
\end{definition}

\begin{lemma}\label{3.2.1}
    For any subgroup $H$ of any group, the relation  $\equiv$ of congruence
    $\mod{H}$ is an equivalence relation.
\end{lemma}
\begin{proof}
    Let $G$ be group and and  $H \leq G$. We have that  $\inv{a}a = e' \in H$ so
    that $a \equiv a \mod{H}$.

    Now, suppose that $a \equiv b \mod{H}$. Then $\inv{b}a \in H$, so
    $\inv{b}a=h$. Then $\inv{h}=\inv{a}b \in H$; so $b \equiv a \mod{H}$.
    Lastly, suppose that $a \equiv b \mod{H}$ and that $b \equiv c \mod{H}$.
    Then  $\inv{b}a \in H$ and $\inv{c}b \in H$. By closure then,
    $(\inv{c}b)(\inv{b}a)=\inv{c}(b\inv{b})a=\inv{c}a \in H$. So $\inv{c}a \in
    H$ making $a \equiv c \mod{H}$.
\end{proof}
\begin{corollary}
    The equivalence classes of $\faktor{G}{\equiv}$ are precisely the left
    cosets $aH$, of $H$ in  $G$.
\end{corollary}
\begin{proof}
    We have that the equivalence classes of $\equiv$ in  $G$ are of the from
    $[a]=\{b \in G : b \equiv a \mod{H}\}$. Then for $b \in [a]$, we have
    $\inv{a}b=h$ for some $h \in H$, so  $b=ah$. Likewise, for  $b \in aH$,
    $b=ah'$ for some $h' \in H$; hence  $\inv{a}b =h' \in H$. This makes
    $[a]=aH$.
\end{proof}

\begin{example}\label{3.3}
    Consider the subgroup $n\Z \leq Z$ and define  $\sim$ to be $a \sim b
    \mod{n\Z}$ if $a-b \in n\Z$. Then if  $a \sim b \mod{n\Z}$ we have
    $a-b=nm$, so that  $n|(a-b)$, making $a \equiv b \mod{n}$. Conversely, if $a
    \equiv b \mod{n}$, then $n|(a-b)$, making $a-b=nk$, hence  $a \sim b
    \mod{n\Z}$. Therefore, the equivalence relation $\sim$ defined is exaclty
    that of $\equiv_n$ of congruence mod $n$. Thus
    $\faktor{Z}{\sim}=\faktor{Z}{\equiv_n}=\faktor{Z}{n\Z}$.
\end{example}

Since $aH$ describes the equivalence classes of the relation  $a \equiv b
\mod{H}$, they partion the group $G$ into disjoint subsets. That is:

 \begin{theorem}\label{3.2.2}
     Let $G$ be a group, and  $H \leq G$. Then the left cosets of $H$ in $G$
     partition $G$ into disjoint subsets. That is:
     \begin{equation}
         G = \bigcup_{g \in G}{gH}
     \end{equation}
     Where $aH \cap bH=\emptyset$ for  $a,b \in G$ distinct.
\end{theorem}
\begin{proof}
    We have that $g \in gH$ for all $g \in G$, so that  $G$ is the union of left
    cosets of  $H$ in  $G$. Now suppose that $ \in aH \cap bH$. Then  $g=ah$ and
     $g=bh'$. So $ah=bh'$, so  $\inv{b}a = h'\inv{h}=h''$. So $b \in aH$. This
     makes $aH=bH$. Therefore if $aH$ and $bH$ are distinct, they are disjoint.
\end{proof}

Now to define the quotient group, it is not enough to define left cosets as
equivalence classes. In general, coset multiplcation is not well defined for
arbitrary cosets. So we wish to make an assertion.

\begin{theorem}\label{3.2.3}
    Let $G$ be a group and  $H \leq G$, and define the operation of coset
    multiplication by  $aHbH=abH$. Then this opertion is well defined if, and
    only if for any  $h \in H$,  $gh\inv{g} \in H$ for all $g \in G$.
\end{theorem}
\begin{proof}
    Suppose that $aHbH=abH$ is well defined. That is, if  $a' \in aH$ and $b'
    \in bH$ then  $a'Hb'H=a'b'H=abH$. Now, let  $g \in G$ and  $h \in H$. Then
    take $\inv{g}H=h\inv{g}H$. Then, since coset multiplication is well defined,
    we get $g\inv{h}\inv{g}H=H$, so that $g\inv{h}\inv{g} \in H$ for $\inv{h} \in
    H$ arbitrary.

    Conversly, suppose that $gh\inv{g} \in H$ then $gh\inv{g}=h'$, hence
    $gh=h'g$. That is  $gh \in Hg$ and  $h'g \in gH$ making  $gH=Hg$. Thus if
    $a' \in aH$ and $b' \in bH$ then  $a'=ah_1$, $b'=bh_2$ so that
    $a'b'=(ah_1)(bh_2)=a(h_1b)h_2=ab(bh')h_2=ab(h'h_2)$, making $a'b' \in abH$.
    Therefore, $a'Hb'H=a'b'H=abH$ making coset multiplication well defined.
\end{proof}
\begin{proof}
    If $H$ is such that $gh\inv{g} \in H$ for all $g \in G$, then  $gH=Hg$. That
    is, every left coset is also a right coset.
\end{proof}
\begin{corollary}
    If $gh\inv{g} \in H$ for all $g \in G$, then the set of all cosets of  $H$
    in  $G$ form a group under coset mutliplication.
\end{corollary}
\begin{proof}
    This proof is idential to that of theorem \ref{3.1.6}.
\end{proof}

\begin{definition}
    Let $G$ be a group and $N$ a subgroup of  $H$. For  $n \in N$ we
    $gn\inv{g}$ the \textbf{conjugate} of $n$ by  $g$. We call the set
    $gN\inv{g}=\{gn\inv{g} : n \in N\}$ the \textbf{conjugate} of $N$ by $g$. We
    say that the subgroup $N$ is \textbf{normal} in $G$ if $gN\inv{g}=N$. That
    is, every every element of $G$ normalizes  $N$. If  $N$ is a normal subgroup
    of  $G$, we write  $N \unlhd G$.
\end{definition}

\begin{definition}
    Leet $G$ be a group, and let $N \unlhd G$ be a normal subgroup of $G$. Then
    the  \textbf{quotient group} $\faktor{G}{N}$ is the group of all cosets of
    $N$ in  $G$.
\end{definition}

\begin{lemma}\label{3.2.4}
    Let $G$ be a group, and  $N \leq G$. Then the following are equivalent:
    \begin{enumerate}
        \item[(1)] $N \unlhd G$.

        \item [(2)] The normalizer of $N$ is precisely  $G$.

        \item[(3)] $gN=Ng$ for all  $g \in G$.

        \item [(4)] $gN\inv{g} \subseteq N$.
    \end{enumerate}
\end{lemma}

\begin{theorem}\label{3.2.5}
    Let $G$ be a group and  $N \leq G$. Then  $N$ is normal if, and only if it
    is the kernel of some homomorphism defined on $G$.
\end{theorem}
\begin{proof}
    We have shown that if $\phi$ is a homomorphism on  $G$, then  $\ker{\phi}
    \unlhd G$; see lemmas \ref{3.1.4} and \ref{3.2.4}.

    Now, suppose that $N \unlhd G$. Now, let $\phi:G \rightarrow \faktor{G}{N}$
    be defined by $\phi:g \rightarrow gN$. Since $N$ is normal, coset
    multiplicaton is well defined and, we have that $abN=aNbN$, so that $\phi$
    is a homomorphism. Now, notice then that
    $\ker{\phi}=\{g \in G : \phi(g)=N\}$. So if $g \in \ker{\phi}$, then $gN=N$,
    so that $g \in N$. Moreover, if  $g \in N$, then  $\phi(g)=gN=N$, so that $g
    \in \ker{\phi}$. This makes $\phi=N$.
\end{proof}
\begin{corollary}
    $\phi$ is onto.
\end{corollary}
\begin{proof}
    Notice that the cosets of $N$ in  $G$ partition  $G$, so that each distinct
    $g$ gets mapped to a distinct $gN$.
\end{proof}
\begin{remark}
    With this theorem, we have tied together the equivalence of the quotient
    group as defined with fibers, to the quotient group as defined with
    equivalence classes of cosets of normal subgroups.
\end{remark}

The homomorphism in theorem \ref{3.2.5} warrants its own definition.

\begin{definition}
    Let $G$ be a group and  $N \unlhd G$. The homomorphism from $G$ onto
    $\faktor{G}{N}$ defined by $g \rightarrow gN$ is called the \textbf{natrual
    projection} of $G$ onto  $\faktor{G}{N}$. If $H \leq \faktor{G}{N}$, then
    the \textbf{complete fiber} of $H$ in  $G$ is the preimage of $H$ under the
    natrual projection.
\end{definition}

We conclude with some examples.

\begin{example}\label{3.4}
    \begin{enumerate}
        \item[(1)] let $G$ be any group. Then $G$ and $\vbrack{e}$ are normal
            in $G$, with quotient groups $\faktor{G}{\vbrack{e}} \simeq G$ and
            $\faktor{G}{G} \simeq \vbrack{e}$.

        \item[(2)] If $G$ is Abelian, then any subgroup  $H \leq G$ is normal.
            We have  $gh=hg$ for all  $g \in G$,  $h \in h$, so that  $gH=Hg$
            for all  $g \in G$. Thus quotient groups can be described for every
            subgroup.

        \item [(3)] Every subgroup of $\Z$ is normal, and cyclic, so
            $N=\vbrack{n}=N=\vbrack{-n}=n\Z$. The quotient group
            $\faktor{\Z}{n\Z}=\vbrack{1}$ is also cyclic.

        \item[(4)] If $N \leq Z(G)$ for a group $G$, then  $N \unlhd G$.
    \end{enumerate}
\end{example}
