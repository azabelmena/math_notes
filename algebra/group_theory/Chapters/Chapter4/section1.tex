\section{Group Actions and Permutation Representations.}

We now present the notion of a group ``acting'' on a given set. The study of
these ``actions'' will allow us to prove results for groups and also for finding
underlying structures of specific sets.

\begin{definition}
    A \textbf{left} group \textbf{action} of a group $G$ on a set  $A$ is a map
    $\cdot: G \times A
    \rightarrow A$ such that for all $g \in G$ and  $a \in A$:
    \begin{enumerate}
        \item[(1)] $g_1 \cdot (g_2 \cdot a)=(g_1g_2) \cdot a$.

        \item[(2)] $e \cdot a=a$, where  $e$ is the identity of  $G$.
    \end{enumerate}
\end{definition}
\begin{remark}
    As before, we will drop explicit mention of the action $\cdot$ and mearly
    werite $ga$. It should be taken into account that the group action $\cdot$
    is not a binary operation.
\end{remark}
\begin{remark}
    Similarly, one can define \textbf{right} group actions. Our study will
    consist of left group actions, so we drop the indication.
\end{remark}

\begin{lemma}\label{4.1.1}
    Let $G$ be a group acting on a set $S$. For each $g \in G$, define the map
    $s_g:S \rightarrow S$ by $s_g:a \rightarrow ga$. Then:
    \begin{enumerate}
        \item[(1)] For each $g \in G$,  $s_g$ is a permutation of  $S$.

        \item [(2)] The map $G \rightarrow A(S)$ defined by $g \rightarrow
            s_g$ is a homomorphism.
    \end{enumerate}
\end{lemma}
\begin{proof}
    Let $a,b \in S$ and suppose that $s_g(a)=s_g(b)$. Then $ga=gb$,
    and by the cancellation laws,  $a=b$. This makes $s_g$  $1-1$. On the
    otherhand, since $ga \in A$ for all $a \in A$, we get that $s_g$ is
    onto. This makes  $s_g$ a permutation. Now, consider  $\inv{g} \in G$,
    and notice that
    $s_{\inv{g}}s_g(a)=s_{\inv{g}}(ga)=\inv{g}(ga)=(\inv{g}g)a=ea=a$.
    Thus we get $\inv{s_g}=s_{\inv{g}}$.

    Now consider the map $\phi:g \rightarrow s_g$. Then
    $\phi(gg')(a)=s_{gg'}(a)=(gg')a=g(g'a)=s_gs_{g'}(a)= \phi(g)\phi(g')$.
\end{proof}
\begin{remark}
    The main takeaway of this lemma is that group actions on a set $S$ are
    merely permutations of the elements of  $S$.
\end{remark}

\begin{definition}
    Let $G$ be a group acting on a set  $S$, and define $s_g:a \rightarrow
    ga$, and define $\phi:G \rightarrow A(S)$ by $\phi:g \rightarrow s_a$.
    We call $\phi$ the  \textbf{permutation representation} of $S$ associated
    with  $g$.
\end{definition}

\begin{example}
    Let $G$ be a group, and  $A \neq \emptyset$. Then:

    \begin{enumerate}
        \item[(1)] Define the action $ga=a$ for all  $g \in G$. Then
            $g_1(g_2)a=g_1a=a$ and $(g_1g_2)a=a$, so $g_1(g_2)a=(g_1g_2)a$, and
            $ea=a$, so we indeed have an action. We call this the
            \textbf{trivial} action, and we say that $G$ acts
            \textbf{trivially} on $A$. Define then  $s_g:a \rightarrow
            ga=a$, then $s_g$ is the identity map. So the permutation
            representation associated with  $g$ is the identity map.

        \item[(2)] In the vector space axioms, scalar multiplication $\cdot:F^*
            \times V \rightarrow V$ is an action of $F^*$ on  $V$. We have for any
             $\alpha,\beta \in F$, and  $v \in V$,
             $\alpha(\beta)v=(\alpha\beta)v$, and $1v=v$. Here $F$ is a field,
             and so  $F^*$ forms a multiplicative group under the relavent
             multiplication.

         \item [(3)] For any $S \neq \emptyset$, the symmetric group  $A(S)$
             acts on $S$ via the action  $sa=s(a)$.

         \item[(4)] Consider again the regular $n$-gon. Label its veritces to be
             the set $\faktor{\Z}{n\Z}$, then we can see that the symmetries of
             the $n$-gon act on vertices of the $n$-gon.

             Consider the map $D_{2n} \times \faktor{\Z}{n\Z}\rightarrow
             \faktor{\Z}{n\Z}$ of $D_{2n} \times \faktor{\Z}{n\Z}$ onto
             $\faktor{\Z}{n\Z}$ via the map $(r^jt,i) \rightarrow r^jt(i)$,
             where $j \in \faktor{\Z}{n\Z}$. This map forms a group action of
             $D_{2n}$ on $\faktor{\Z}{n\Z}$. Also notice that distinct
             symmetries induce distinct permutations of the vertices.

         \item[(5)] Let $G$ be any group, and let  $A=G$. Then the binary
             operation on $G$ is a group action of  $G$ onto itself. We have
             $a(bc)=(ab)c$, and so the first prppertiy is satisfied by
             associativity, and $ea=a$ and the second property is satisfied by
             the identity law. We call the binary operation a \textbf{left
             regular} action. Also notice that distinct elements of $G$ induce
             distinct permutations of  $G$.
    \end{enumerate}
\end{example}

We end the section, and the chapter with two more definitions.

\begin{definition}
    Let $G$ be a group acting on a set $A$. We call the action of $G$ on  $A$
     \textbf{faithful} if distinct elements of $G$ induce distinct permutations
     on  $G$. That is if $\phi$ is the permutation representation associated
     with  $G$, then  $\phi$ is  $1-1$.
\end{definition}

\begin{definition}
    Let $G$ be a group acting on a set $A$. We define the \textbf{kernel} of the
    group action on $A$ to be  $\ker{A}=\{g \in G : ga=a \text{ for all } a \in
    A\}$
\end{definition}

\begin{lemma}\label{4.1.2}
    The kernel of a group action is precisely the kernel of the associated
    permutation representation.
\end{lemma}
\begin{proof}
    Let $G$ be a group acting on a set  $A$. Now, for any  $g \in \ker{A}$, we
    have $ga=a$ for all  $a \in A$. Now let $\sigma:a \rightarrow ga$, and
    consider the permutation representation of $A$ with respect to  $g$,
    $\phi:g \rightarrow s_g$. We can then see that $\phi(g)=s_g$,
    for which $s_g(a)=a$ for all $a \in A$, so that  $\phi(g)=(1)$, hence,
    $g \in \ker{\phi}$. Conversely, we also get that if $g \in \ker{\phi}$, then
    $\phi(g)=s_g=(1)$, so that for all $a \in A$, $a=ga$, which makes  $g
    \in \ker{A}$.
\end{proof}
\begin{corollary}
    Two group elements induce the same permutation on $A$ if, and only if they
    are in the same coset of the kernel; i.e. if, and only if they are in the
    same fiber of the permutation representation $\phi$.
\end{corollary}

\begin{example}\label{4.2}
    \begin{enumerate}
        \item[(1)] Let $n \in \Z^+$, then the symmetric group $S_n$, acts on the
            set $A=\{1, \dots, n\}$ by the action $s \cdot i=s(i)$. for all $i
            \in A$. The permutation representation of $A$ is then the map
            $\phi:S_n \rightarrow S_n$ taking $s \rightarrow s$; i.e. the
            identity map. Snce $\phi$ is  $1$--$1$, we get that $\phi$ is
            faithful, and that for each $i \in A$, the \stab{i} \simeq
            $S_{n-1}$.

        \item[(2)] Consider $D_8$ acting on the set $A$ of  $4$ vertices of a
        square with  $A=\{1,2,3,4\}$. Letting $r$ be the rotation of the square
        by  $\frac{\pi}{2}$, and $t$ the reflecton of the square about the
        line cutting vertices $1$ and $3$. Then the permutations of $A$ via
        $r$ and  $t$ are  $s_r=(1 \ 2 \ 3 \ 4)$ and $s_t = (2 \ 4)$. Notice then
        that $s_{rt}=s_rs_t=(1 \2 \ 3 \4)(2 \ 4)=(1 \ 4)(2 \ 3)$. The action of
        $D_8$ on $A$ is also faithful, since  $\ker{A}=\vbrack{e}$. We can also
        see that $\stab{a} \leq D_8$ and that $\ord{(\stab{a})}=2$.

    \item[(3)] Consider the square labeled by the vertices in the
        previous example and let $A$ be the set of unordered pairs of opposite
        vertices  $A=\{\{1,3\}, \{2,4\}\}$. Then $D_8$ acts on $A$ with the
        permutations $s_r$ and  $s_t$. Letting  $1'=\{1,3\}$ and $2'=\{2,4\}$,
        we get that $s_r=(1' \ 2')$ and $s_t=(1')$. This action is not faithful,
        as $\ker{A}=\vbrack{t,r^2}$.
    \end{enumerate}
\end{example}

\begin{theorem}\label{4.1.3}
    For any group $G$ acting on a nonempty set $S$, there is a 1--1 correspondence
    of teh actions of  $G$ onto homomorphisms of  $G$ into $A(S)$.
\end{theorem}
\begin{proof}
    Let $\phi:G \rightarrow A(S)$ be a homomorphism, and define the action
    $ga=\phi(g)(a)$. Then the kernel of the action is $\ker{A}=\{g \in G :
    \phi(g)(a)=a\}=\{g \in G : \phi(g)=e\}=\ker{G}$. Moreover, the permutation
    representation of $ga$ is precisely  $\phi$.
\end{proof}

\begin{definition}
    For any group $G$, a  \textbf{permutation representation} of $G$ is a
    homomorphism of  $G$ into $A(S)$.
\end{definition}

\begin{lemma}\label{4.1.4}
    Let $G$ be a group acting on a nonempty set $A$. The  $\sim$ on $A$ defined
    by  $a \sime b$ if, and only if  $a=gb$ for some  $g \in G$ is an
    equivalence relation.
\end{lemma}
\begin{proof}
    We have $a=ea$, so  $a \sim a$. Now, if $a \sim b$, then  $a=gb$ for  $g \in
    G$. Then  $b=\inv{g}a$ so that $b \sim a$. Now let  $a \sim b$ and  $b \sim
    c$. Then  $a=gb$ and  $b=hc$ for  $g,h \in G$. Then  $a=g(hc)=(gh)c$, and by
    closure of $G$, this makes  $a \sim c$.
\end{proof}
\begin{corollary}
    For each $a \in A$, the number of elements in the equivalence class of  $a$
    is the index of  $\stab{a}$ in $G$; i.e.  $[G:\stab{a}]$.
\end{corollary}
\begin{proof}
    Let $\Oc_a=\{ga : g \in G\}$ be the equivalence class of $a \in A$ under
    $\sim$. Notice that  $g\stab{a}$ is a left coset of $G$, and so consider the
    map  $\phi:ga \rightarrow g\stab{a}$ from $\Oc_a \rightarrow
    \faktor{G}{\stab{a}}$. We get that $\phi$ is onto since $ga \in \Oc_a$ for
    every $g \in G$. Moreover,  $\phi$ is 1--1 since  $ga=hb$ if, and only if
    $(\inv{h}g)a=b$ so that $(\inv{h}g)\stab{a}=\stab{b}$, hence
    $g\stab{a}=h\stab{b}$. This make $\Oc_a \simeq \faktor{G}{\stab{a}}$.
\end{proof}

\begin{definition}
    Let $G$ be a group acting on a nonempty set  $A$. We call the equivalence
    class $\orb{a}=\{ga : g \in G\}$ of $a$ the  \textbf{orbit} of $a$. We call
    the action of  $G$ on  $A$  \textbf{transitive} if there is only one orbit.
\end{definition}

\begin{example}\label{}
    \begin{enumerate}
        \item[(1)] For any group $G$ and set  $A$, if  $G$ acts trivially on
            $A$, then  $\stab{a}=G$ for all $a \in A$, and  $\orb{A}=A$. Then
        the trivial action is transitive if, and only if $|A|=1$.

    \item[(2)] The symmetric group $S_n$ acts on  $A=\{1, \dots n\}$
        transitively in its usual action. Notice that for any $i \in A$,
        $[S_n:\stab{i}]=|A|=n$.

    \item[(3)] If $G$ is any group acting on $A$, and  $H \leq G$, then  $H$
        also acts on $A$. Now suppose that  $G$ acts on  $A$ transitively, in
        particular let $G=S_n$. Let  $H=\vbrack{(1 \ 2),(3 \ 4)} \leq S_n$. Then
        the orbits of  $H$ on  $A=\{1, \dots, n\}$ are the sets $\{1,2\}$ and
        $\{3,4\}$, and there is no element taking $2 \rightarrow 3$. Thus $H$
        does not act transitively on $A$. For any group $G$ acting on  $A$
        transitively,  $H$ need not act on  $A$ transitively.

    \item[(4)] $D_8$ acts transitively on the vertices of a square. The
        stabalizer of any vertex is the subgroup of order $2$ generated by the
        reflection about the line passing through that vertex.
    \end{enumerate}
\end{example}

\begin{theorem}\label{4.1.5}
    Every element of $S_n$ has a unique cycle decomposition.
\end{theorem}
\begin{proof}
    Let $A=\{1, \dots, n\}$ and let $s \in S_n$, and consider  $\vbrack{s}$.
    Then $\vbrack{s}$ acts on $A$, and hence partitions it into orbits. Now,
    consider an orbit  $\Oc$ under  $s$, and let  $x \in \Oc$. Then we have that
     $\Oc=A$, and there is a 1--1 correspondence of cosets of $\stab{x}$ in $G$
     onto elements of  $\Oc$, defined by the map  $s^ix \rightarrow
     s^i\stab{x}$. Now, since $\vbrack{s}$ is cyclic, we get $\stab{x} \unlhd
     \vbrack{s}$, and $\faktor{\vbrack{s}}{\stab{x}}$ must be cyclic of order
     $d$, where  $d$ is the least integer for which  $s^d \in \stab{x}$. Then we
     get that $d=[\vbrack{s},\stab{x}]=|\Oc|$ and the cosets can be listed in
     order as:
     \begin{align*}
         \stab{x}   &&   s\stab{x}   &&   \dots   &&   s^{d-1}\stab{x} \\
     \end{align*}
     So $\Oc$ has elements
     \begin{align*}
         x      &&      s(x)    &&      \dots       &&      s^{d-1}(x)      \\
     \end{align*}
     So $s$ cycles through  $\Oc$; that is  $s$ is a  $d$-cycle on $\Oc$ which
     gives  $s$ a cycle decomposition.

     Moreover, teh orbits of  $\vbrack{s}$ are uniquely determined by $s$, so
     choosing  $s^i(x)$ we can reorder the elements of $\Oc$ as
     \begin{align*}
         s^i(x) && s^{i+1}(x) && \dots && s^{d-1}(x) && x && s(x) && \dots &&
         s^{i-1}(x) \\
     \end{align*}
     which is a cycle permutation.
\end{proof}

\begin{definition}
    We call the subgroups of the symmetric group \textbf{permutation groups}.
\end{definition}
