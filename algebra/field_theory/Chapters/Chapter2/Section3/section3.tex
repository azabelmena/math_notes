\section{Finite Fields}

We reiterate some previous results about finite fields.

\begin{lemma}\label{2.3.1}
    Let $E$ be a finite field over  $\F_p$. Then $E$ is an extension of finite
    degree $[E:\F_p]=n$. Moreover, if $|E|=p^n$, and $E$ is the splitting field
    of the polynomial $x^{p^n}-x$ over $\F_p$.
\end{lemma}

\begin{proof}
    Suppose that $E$ is a finite field, but that the extension
    $\faktor{E}{\F_p}$ is infinite. The $E$, as a vector space over  $\F_p$, has
    an infinite basis  $\{\a_1, \a_2, \dots\}$. Moreover, since every element of
    $E$ is a linear comination of this basis, we obtain a contradiction as there
    are infinite such combinations, but $E$ is finite. Therefore $[E:\F_p]=n$,
    for some $n \in \Z^+$.

    Let $\a,\b$ be roots. Then  $\a^{p^n}=\a$, and $\b^{p^n}=\b$. Then
    $(\a\b)^{p^n}=\a\b$ and $(\inv{\a})^{p^n}=\inv{\a}$. Moreover,
    $(\a+\b)^{p^n}=\a+\b$. So the set of $p^n$ disctinct roots of  $x^{p^n}-x$
    is closed under addition, multiplication, and inverses in its splitting
    field. Let $F$ be that splitting field. Notice that  $F \subseteq E$,
    moreover, $[F:\F_p]=n$ so that $|F|=p^n$. We also have that
    $\Uc(F)$ is a cyclic group of order $p^n-1$, so that $E \subseteq F$,
    since $\a^{p^n-1}=1$. Therefore $E$ is the splitting field of
    $x^{p^n}-x$ over $\F_p$, and so contains all the roots of $x^{p^n}-x$.
    Notice that since $E$ is a splitting field, it is unique up to isomorphism.
\end{proof}
\begin{remark}
    Since the splitting fields of $x^{p^n}-x$ over $\F_p$ are unique up to
    isomorphism, we denote them by  $\F_p$ from now on.
\end{remark}
\begin{corollary}
    $\F_{p^n}$ is Galois over $\F_p$ with Galois group isomorphic to
    $\faktor{\Z}{n\Z}$.
\end{corollary}
\begin{proof}
    Notice that $\F_{p^n}$ is normal and seperable over $\F_p$. Morever, that
    the Frobenius automorphism generates the Galois group of order  $n$.
\end{proof}
\begin{corollary}
    All subfields of $\F_{p^n}$ are Galois over $\F_p$, and in 1--1
    with the divisors of $n$. Moreover, they are of the form $\F_{p^d}$ for all
    $d|n$.
\end{corollary}
\begin{proof}
    We have that
    \begin{equation*}
        \Gal{\faktor{\F_{p^n}}{\F_p}}=\langle \s \rangle \simeq
        \faktor{\Z}{n\Z}
    \end{equation*}
    where $\s:\a \xrightarrow{} \a^p$ is the Frobenius automorphism. By the
    fundamental theorem of Galois theory, each subfield of $\F_{p^n}$
    corresponds to a subgroup of $\faktor{\Z}{n\Z}$, which are defined by the
    divisors of $n$. Hence, there is precisely one field $\F_{p^d}$ for each
    $d|n$, with  $[\F_{p^d}:\F_p]=d$. Now, since $\faktor{\Z}{n\Z}$ is Abelian,
    every subgroup is normal, and so each $\F_{p^d}$ is normal over $\F_p$.
    Since they are also seperable, the are Galois over $\F_p$.
\end{proof}
\begin{corollary}
    The fields $\F_{p^d}$ are precisely those fixed by $\s^d$; that is,
    $\Fc(\langle \s^d \rangle)=\F_{p^d}$ for all $d|n$.
\end{corollary}

\begin{example}\label{}
    The irreducible polynomial $x^4+1$ over  $\Z$ is reducible $\mod{p}$, by $p$
    a prime integer. Consider  $x^4+1 \in \F_p[x]$, if $p=2$, then
    $x^4+1=(x+1)^4$ and we are done, since $\char{\F_2}=2$. Now, if $p$ is an
    odd prime, notice that  $p \equiv 1,3,5,7 \mod{8}$ so that $p^2 \equiv 1
    \mod{8}$. That is, $8|p^2-1$. Then the polynomial  $x^8-1|x^{p^2-1}-1$. We
    have then that $x^4+1|x^8-1|x^{p^2-1}-1|x^{p^2}-x$; so that the roots of
    $x^4+1 \mod{p}$ are also the roots of $x^{p^2}-x \mod{p}$, and equivalently,
    are fixed by the automorphism $\s^2$, $\s$ being the Frobenius automorphism.
    Since $\F_{p^2}$ is the splitting field of $x^{p^2}-x$, and hence consists
    of exactly the roots of $\F_{p^2}$, we get that if $\a$ is a root of
    $x^{p^2}-x$, then $[\F_p(\a) : \F_p] \leq 2$ so that $x^4+1$ is not
    irreducible over  $\F_p$.
\end{example}

\begin{lemma}\label{2.3.2}
    The finite field $\F_{p^n}$ is simple.
\end{lemma}
\begin{proof}
    We have that $\Uc(\F_p)$ is a multiplicative group of finite order;
    moreover, it is the finite subgroup of the multiplicative group of a field.
    Therefore $\Uc(\F_p)$ is cyclic. Then if $\a$ is a generator of
    $\Uc(\F_p)$, then $\F_{p^n}=\F_p(\a)$ and there exists an irreducible
    polynomial of degree $n$ over  $\F_p$.
\end{proof}

\begin{lemma}\label{2.3.3}
    $x^{p^n}-x$ is precisely the product of all irreducible polynomials of
    degree $\deg=d$, $d|n$ over  $\F_p$.
\end{lemma}
\begin{proof}
    Consider $\F_{p^r}$ as the quotient ring $\faktor{\F_p[x]}{(m)}$, where
    $m(x)$ os the minimal polynomial of a root $\a$ of  $x^{p^n}-x$. Then it
    follows that $m(x)|x^{p^n}-x$ and $\deg{m}=n$.

    COnversely, if $p(x)$ is an irreducible polynomial of degree $\deg{p}=d$,
    with $p|x^{p^n}-x$, if $\a$ is a root of  $p$, then  $\F_p(\a) \subseteq
    \F_{p^n}$; and $[\F_p(\a) : \F_p]=d$. That makes $d|n$, and $\F_p(\a)$
    is Galois over $\F_p$. Now, since $\F_{p^n}$ consists preecisely of the
    roots of $x^{p^n}-x$, grouping together the factores $x-\a$ according to the
    degre $d$, we obtain the result.
\end{proof}

\begin{definition}
    We define the \textbf{M\"obius function} of an integer $n \in \Z^+$ to be
    \begin{equation*}
        \m(n)=\begin{cases}
            1,  &   \text{ if } n=1 \\
            0,  &   \text{ if } n \text{ has a square factor} \\
            (-1)^r,  &   \text{ if } n \text{ has } r \text{ distinct prime
            factors} \\
        \end{cases}
    \end{equation*}
\end{definition}

\begin{theorem}\label{2.3.4}
    The number of irreducible polynomials of degree $n$ over  $\F_p$ is
    \begin{equation*}
        \psi(n)=\frac{1}{n}\sum_{d|n}{\mu(d)p^{\frac{n}{d}}}
    \end{equation*}
\end{theorem}
\begin{proof}
    Let $\psi(n)$ the number of irreducible polynomials of degree $\deg=n$ over
    $\F_p$. By the above lemma, we get that  $p^n=\sum_{d|n}{d\psi(d)}$. Using
    the M\"obius inversion formula for $n\psi(n)$, we get that
    $n\psi(n)=\sum_{d|n}{\mu(d)p^{\frac{n}{d}}}$.
\end{proof}
