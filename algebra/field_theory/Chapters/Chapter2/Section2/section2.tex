\section{The Fundamental Theorem of Galois Theory.}

\begin{definition}
    A \textbf{linear character} of a group $G$ with values in a field  $L$ is a
    homomorphism  $\x:G \xrightarrow{} \Uc(L)$. We say that distinct linear
    characters $\x_1, \dots, \x_n$ of $G$ are  \textbf{linearly independent}
    over $L$ if they are linearly independent, as functions, over  $G$.
\end{definition}

\begin{theorem}\label{2.2.1}
    If $\x_1, \dots, \x_n$ are distinct linear characters of a group $G$ with
    values in a field  $L$, then they are linearly independent over  $L$.
\end{theorem}
\begin{proof}
    Suppose that $\x_1, \dots, \x_n$ are linearly dependent, and choose a
    dependence relation with minimum of $m$ nonzero coefficients $a_1, \dots,
    a_m \in L$, so that
    \begin{equation*}
        a_1\x_1+\dots+a_n\x_m=0
    \end{equation*}
    Then for any $g \in G$, we have
    \begin{equation*}
        a_1\x_1(g)+\dots+a_n\x_m(g)=0
    \end{equation*}
    Now, let $g_0 \in G$, with $\x_1(g_0) \neq \x_m(g_0)$. Then
    \begin{equation*}
        a_1\x_1(g_0g)+\dots+a_n\x_n(g_0g)=a_1\x(g_0)\x(g)+\dots+a_m\x_m(g_0)\x_m(g)=0
    \end{equation*}
    multiplying the preceeding equation with the above by $\x_m(g_0)$ and
    subtracting from the above equation, we get
    \begin{equation*}
        a_1(\x_1(g_0)-\x_m(g_0))\x_1(g)+\dots+a_m(\x_1(g_0)-\x_m(g_0))\x_m(g)=0
    \end{equation*}
    which gives a linear dependence relation with fewer than $m$ nonzero
    coefficients; which contradicts our choice of  $m$. Therefore  $\x_1, \dots,
    \x_n$ must be linearly independent.
\end{proof}
\begin{corollary}
    If $\s_1, \dots, \s_n$ are distinct embeddings of a field $K$ into a field
    $L$, then they are linearlyy independent as functions.
\end{corollary}

\begin{theorem}\label{2.2.2}
    Let $G=\{\s_1, \dots, s_n\}$ where $\s_1=\i$ a subgroup of automorphisms of
    a field $K$, and let  $F$ be the corresponding fixed field. THen
    \begin{equation*}
        [K:F]=\ord{G}=n
    \end{equation*}
\end{theorem}
\begin{proof}
    Suppose that $n>[K:F]$, and consider the basis $\{\w_1, \dots, \w_m\}$ of
    $\faktor{K}{F}$ as a vector space so that $[K:F]=m$. Then the matrix
    equation
    \begin{equation}
        \begin{pmatrix}
            \s_1\w_1    &   \dots   &   \s_n\w_m    \\
            \vdots      &   \ddots  &   \vdots  \\
            \s_n\w_1    &   \dots   &   \s_n\w_m    \\
        \end{pmatrix}
        \begin{pmatrix}
            x_1 \\  \vdots  \\  x_m \\
        \end{pmatrix}
        =0
    \end{equation}
    has nontrivial solution $\begin{pmatrix} \b_1 \\ \vdots \\ \b_n
    \end{pmatrix}$. Let $a_1, \dots, a_m \in F$, so that $\s_i\a_j=\a_j$ for
    each  $1 \leq i \leq n$ and $1 \leq j \leq m$. Multyplying by
    $\begin{pmatrix} \s_1a_1 \\ \vdots \\ \s_ma_1 \end{pmatrix}$, we obtain
    \begin{equation*}
        \begin{pmatrix}
            a_1\s_1\w_1    &   \dots   &   a_1\s_1\w_1    \\
            \vdots      &   \ddots  &   \vdots  \\
            a_m\s_m\w_1    &   \dots   &   a_m\s_n\w_m    \\
        \end{pmatrix}
        \begin{pmatrix}
            \b_1    \\  \vdots  \\  b_n \\
        \end{pmatrix}
        =0
    \end{equation*}
    so that we can obtain the equation
    \begin{equation*}
        \s_1(a_1\w_1+\dots+a_m\w_m)\b_1+\dots+\s_n(a_1\w_1+\dots+a_m\w_m)\b_n=0
    \end{equation*}
    Where $\b_1, \dots, \b_n$ are not all $0$. Now, since  $\{\w_1, \dots,
    \w_m\}$ is an $F$-basis for  $K$, for all  $\a \in K$, we get that
    $\a=a_1\w_1+\dots+a_m\w_m$. So we have from the above equation
    \begin{equation*}
        (\s_1\a)\b_1+\dots(\s_n\a)\b_n=0
    \end{equation*}
    so that $\{\s_1, \dots, \s_n\}$ are linearly dependent over $K$; which
    contradicts the above corollary. No $n \leq [K:F]$.

    Now, suppose that $n<[K:F]$, and thte tere are more than $n$  $F$-linearly
    independent elements  $\a_1, \dots, \a_{n+1} \in K$. Then
    \begin{equation*}
        \begin{pmatrix}
            \s_1\a_1    &   \dots   &   \s_1\a_{n+1}    \\
            \vdots  &   \ddots  &   \vdots  \\
            \s_n\a_1    &   \dots   &   \s_n\a_{n+1}
        \end{pmatrix}
        \begin{pmatrix}
            x_1 \\  \dots   \\  x_{n+1}
        \end{pmatrix}
        =0
    \end{equation*}
    has nontrivial solution with entries $\b_1, \dots, \b_{n+1} \in K$. Now, if
    $\b_i \in F$ for all $1 \leq i \leq n+1$, we get an immediate contradiction
    of the linear independence of  $\{\a_1, \dots, \a_{n+1}\}$ over $F$. So at
    least one  $\b_i \not\in F$.

    Now, choose a nontrivial solution with minimum of  $r$ nonzero entries
    $\b_i$. Suppose also that  $\b_r=1$, then at least one  $\b_i \not\in F$,
    for  $1 \leq i \leq r-1$, and so  $r>1$. SUppose then that  $\b_1 \not\in
    F$. Then the matrix equation
    \begin{equation*}
        \begin{pmatrix}
            \s_1\a_1    &   \dots   &   \s_1\a_{r-1}    &   \s_1\a_r    \\
            \vdots  &   \ddots  &   \vdots  \\
            \s_n\a_n    &   \dots   &   \s_n\a_{r-1}    &   \s_n\a_r    \\
        \end{pmatrix}
        \begin{pmatrix}
            \b_1    \\ \vdots   \\  \b_{r-1}    \\  1   \\
        \end{pmatrix}
        =0
    \end{equation*}
    Now, since $\b_1 \not\in F$, there exists an automorphsim $\s_{k_0}$ of $K$
    with  $\s_{k_0}\b_1 \neq \b_1$ for $1 \leq k_0 \leq n$. Applying $\s_{k_0}$
    to each row of the above equation yields a row of the form
    \begin{equation*}
        \s_{k_0}(\s_j\a_1)(\s_{k_0}\b_1)+\dots+\s_{k_0}(\s_j\a_{r-1})
            (\s_{k_0}\b_{r-1})+\s_j\a_r=0
    \end{equation*}
    However, since $G$ is a group,  $\s_{k_0}\s_j=\s_i$ for $1 \leq i,j \leq n$,
    so we get
    \begin{equation*}
        (\s_i\a_1)(\s_{k_0}\b_1)+\dots+(\s_i\a_{r-1})(\s_{k_0}\b_{r-1})+\s_i\a_r=0
    \end{equation*}
    Subtracting this equation from the one preceeding it, we obtain
    \begin{equation*}
        (\s_i\a_1)(\b_1-\s_{k_0}\b_1)+\dots+(\s_i\a_{r-1})(\b_{r-1}-\s_{k_0}\b_{r-1})
            =0
    \end{equation*}
    with $x_1=\b_1-\s_{k_0}\b_1 \neq 0$. This choice of $k_0$ gives fewer than
    $r$ nonzero coefficients of a nontrivial solutions, which contradicts the
    choice of  $r$. Therefore  $n=[K:F]$.
\end{proof}
\begin{corollary}
    If $K$ is a finite extension over a field  $F$, then
    \begin{equation*}
        \ord{\Aut{\faktor{K}{F}}} \leq [K:F]
    \end{equation*}
    with equality holding if, and only if $F$ is the fixed field of
    $\Aut{\faktor{K}{F}}$.
\end{corollary}
\begin{proof}
    Let $F_1$ be the fixed field of $\Aut{\faktor{K}{F}}$, so that $F \subseteq
    F_1 \subseteq K$. TYhen $[K:F_1]=\ord{\Aut{\faktor{K}{F}}}$, hence
    \begin{equation*}
        [K:F]=(\ord{\Aut{\faktor{K}{F}}})[F_1:F]
    \end{equation*}
\end{proof}
\begin{corollary}
    If $G$ is a finite subgroup of automorphisms of a field  $K$, and $F$ is its
    fixed field, then  $\Aut{\faktor{K}{F}}=G$ so that $K$ is Galois over $F$
    with Galois group  $G$.
\end{corollary}
\begin{proof}
    By definition, we have that since $G$ fixes the elements of $F$, then $G \leq
    \Aut{\faktor{K}{F}}$. THen $\ord{G}=[K:F]$ and by the above corollary, we
    get
    \begin{equation*}
        \ord{\Aut{\faktor{K}{F}}} \leq [K:F]
    \end{equation*}
    so that
    \begin{equation*}
        [K:F]=\ord{G} \leq \ord{\Aut{\faktor{K}{F}}} \leq [K:F]
    \end{equation*}
    and equality holds.
\end{proof}
\begin{corollary}
    If $G$ and  $H$ are distinct finite subgroups of  $\Aut{K}$, then their
    fixed fields are also distinct.
\end{corollary}
\begin{proof}
    Let $F_G$ the fixed field of  $G$, and  $F_H$ the fixed field of  $H$. I f
    $F_G=F_H$, then we have that $H$ fixes  $F_G$, and since any automorphism
    fixing  $F_G$ is in  $G$, we have  $H \leq G$. By similar reasoning, we get
     $G \leq H$ so that  $G=H$.
\end{proof}

\begin{theorem}\label{2.2.3}
    The extension $K$ over a field  $F$ is Galois if, and only if $K$ is the
    splitting field of some seperable polynomial in $F$. Moreover, every
    irreducible polynomial over $F$ having at least one root in  $K$ splits over
     $K$.
\end{theorem}
\begin{proof}
    By lemma \ref{2.1.5}, the splitting field of a seperable polynomial over a
    field is Galois.

    Now, suppose that $K$ is Galois over  $F$, and let  $p(x) \in F[x]$ an
    irreducible polynomial with a root $\a \in K$. Consider, for each  $\s_i \in
    \Gal{\faktor{K}{F}}$ the elements
    \begin{align*}
        \a  &&   \s_2\a  &&   \dots   &&   \s_n\a  \\
    \end{align*}
    where $\s_1=\i$, and let
    \begin{align*}
        \a  &&   \a_2  &&   \dots   &&   \a_n  \\
    \end{align*}
    be the distinct elements taken on by these permutations (in no particular
    order). If $\t \in \Gal{\faktor{K}{F}}$, by the group law, we get
    $\t\s_i=\s_j$ for all  $1 \leq i,j \leq n$. APplying  $\t$ to  $\a_i$ we
    het permutations of the elements  $\a, \a_2, \dots, \a_n$. Then the
    polynomial $f(x)=(x-\a)(x-\a_2) \dots (x-\a_n)$ has coefficients fixed by
    the elements of $\Gal{\faktor{K}{F}}$. That is, the coefficients lie in the
    fixed field $F$. Hence  $f \in F[x]$.

    Now, since $p$ is irreducible with root $\a$, it is the minimal polynomial
    for $\a$ over $F$, and hence  $p|f$. Moreover, we can pbserve that $f|p$, so
    that  $p(x)=f(x)$, which makes  $p(x)$ seperable with all its roots in  $K$.

    Now, let $\{\w_1, \dots, \w_n\}$ be a basis for $\faktor{K}{F}$ as a vector
    space, and let $p_i(x)$ the minimal polynomial for $\w_i$ over $F$ for all
    $1 \leq i \leq n$. Then  $p_i$ is seperable, with roots in  $K$. Let
    $g(x)=p_1(x) \dots p_n(x)$ (where this product is squarefree). Then if $E$
    is the splitting field of  $g$ over  $F$, then $\w_i \in E$ for all  $1 \leq
    i \leq n$, so that  $K \subseteq E$. On the otherhand, since  $g$ splits
    over  $K$, we get  $E \subseteq K$, and so  $K=E$ is the splitting field of
     $g$ over $F$.
\end{proof}

\begin{definition}
    LEt $K$ be an extension of a field  $F$. If  $\a \in K$, and  $\s \in
    \Gal{\faktor{K}{F}}$, we call the permutations $\s\a$  \textbf{Galois
    conjugates} (or simply \textbf{conjugates}) of $\a$ over $F$. If $E$ is a
    sbufield of  $K$ containing  $F$, then we call  $\s{E}$ the
    \textbf{conjugate field} of $E$over  $F$.
\end{definition}

\begin{theorem}[The Fundamental Theorem of Galois Theory]\label{2.2.4}
    Let $K$ be Galois over a field  $F$. Then there is a 1--1 correspondence of
    the subfields  $E$ of  $K$, each containing $F$, onto the subgroups of
    $\Gal{\faktor{F}{K}}$. This correspondence is given by taking each $E$ to
    elements of $\Gal{\faktor{K}{F}}$ fixing $E$, and by taking subgroups of
    $\Gal{\faktor{K}{F}}$ to thier fixed fields. Moreover, this correspondence
    gives the following
    \begin{enumerate}
        \item[(1)] If $E_1$ and $E_2$ correspond to $H_1$ and $H_2$, then $E_1
            \subseteq E_2$ if, and only if $H_2 \leq H_1$.

        \item[(2)] $[K:E]=\ord{H}$ and $[E:F]=[\Gal{\faktor{K}{F}}:H]$.

        \item[(3)] $K$ is Galois over  $E$ with Galois group $H=\Gal{\faktor{K}{E}}$.

        \item[(4)] $E$ is Galois over  $F$ if, and only if  $H \unlhd
            \Gal{\faktor{K}{F}}$, and $\Gal{\faktor{E}{F}}$ is isomorphic to the
            quotient group of $\Gal{\faktor{K}{F}}$ by $H$.

        \item[(5)] If $E_1$ and $E_2$ correspond to $H_1$ and $H_2$, then $E_1
            \cap E_2$ corresponds to $\langle H_1, H_2 \rangle$, and $E_1E_2$
            corresponds to $H_1 \cap H_2$. Moreover, the lattice of subfields of
            $K$ containing  $F$ is dual to the lattice of subgroups of
            $\Gal{\faktor{K}{F}}$
    \end{enumerate}
\end{theorem}
\begin{proof}
    Take $G=\Gal{\faktor{K}{F}}$, and $H \leq G$. Obtain a unique fixed field of
     $E$, of $H$. Since fixed fields of distinct subgroups of $G$ are distinct,
     we get that the correspondence is 1--1.a Now, let  $K$ be the splitting
     field of a seperable polynomial  $f(x)$ over $F$. Consider $f(x)$ as a
     polynomial over $E$, for any subfield $E$ of  $K$ containing  $F$. Then
     $K$ is the splitting field of  $f$ over  $E$, and by theorem \ref{2.2.3},
     $K$ is Galois over  $E$. Then  $E$ is the fixed field of the automorphism
     group  $\Aut{\faktor{K}{E}}$ which is a subgroup of $G$. Hence every
     subfield of  $K$ containing  $F$ is the fixed field of a subgroup of  $G$,
     and we have proved that the correspondence is onto and we get the required
     bijection. We also have by lemma \ref{2.1.4}, that this correspondence is
     inclusion reversing.

     Now, notice that for any automorphism $\s \in G$, $\s|_E$ is an embedding
     of  $E$ onto $\s{E}$ and $\s{E} \subseteq K$ is a subfield. Conversely, let
     $\t:E \xrightarrow{} \t{E} \subseteq \cl{F}$ be an embedding of $E$. If $\a
     \in E$, and  $m_\a$ is the minimum polynomial of  $\a$ over  $F$, then
     $\t{\a}$ is alos a root of $m_\a$, and  $K$ contains all the roots of
     $m_\a$. Now,  $K$ is the splitting field of  $f$, and hence, also the
     splitting field of  $\t{f}(x)$. Then if we extend $\t$ to an isomorphism
     $\s$, we get the diagram
     \[\begin{tikzcd}
            {\sigma:} & K & K \\
            {\tau:} & E & {\tau{E}}
            \arrow[from=1-2, to=1-3]
            \arrow[from=2-2, to=2-3]
            \arrow[no head, from=1-3, to=2-3]
            \arrow[no head, from=1-2, to=2-2]
      \end{tikzcd}\]
      and since $\s$ fixes  $F$, then every embedding $\t$ of $E$ fixing  $F$ is
      the restriction of some automorphism  $\s$ of  $K$.

      Now, two autmorphisms  $\s,\s' \in G$ restrict to the same embedding if,
      and only if  $\inv{\s}\s'=\i$, the identity map on $E$. This makes
      $\inv{\s}\s' \in H$; i.e. $\s' \in \s{H}$. Since $\Gal{\faktor{K}{E}}=H$,
      and letting $\Emb{\faktor{E}{F}}$ be the set of all embeddings of $E$ into
       $\cl{F}$, we get
       \begin{equation*}
           |\Emb{\faktor{E}{F}}|=[G:H]=[E:F].
       \end{equation*}
       Also notice that since automorphisms are also embeddings,
       $\Aut{\faktor{E}{F}} \subseteq \Emb{\faktor{E}{F}}$. Now, $E$ is Galois
       over  $F$ if, and only if  $\ord{\Aut{\faktor{E}{F}}}=[E:F]$. Hence $E$
       is Galois over  $F$ if, and only if
       $\Aut{\faktor{E}{F}}=\Emb{\faktor{E}{F}}$; that is, each embedding of $E$
       is an automorphism of $E$ which makes $\s{E}=E$.

       Notice, now, that if $\s{\a} \in \s{E}$, then
       $(\s{h}\inv{s})(\s{\a})=\s{(h\a)}=\s{\a}$, since $h\a=\a$ for all $H$ in
        $H$. So  $\s{H}\inv{\s}$ fixes $\s{E}$. Moreover, we have
        $\ord{\s{H}\inv{\s}}=[K:\s{E}]=[K:E]$, so that
        $\ord{\s{H}\inv{\s}}=\ord{H}$. This makes $\s{H}\inv{\s}$ the group
        fixing $\s{E}$.

        Now, $\s{E}=E$ for all $\s \in G$ if, and only if  $\s{H}\inv{s}=H$. That
        is, $E$ is Galois over $F$ if, and only if  $H \unlhd
        \Gal{\faktor{K}{F}}$. Since the embedding of $E$ over  $F$ are cosets of
         $H$ in  $G$, and  $H \unlhd G$, the correspondence gives that
         \begin{equation*}
             \faktor{G}{H} \simeq \Gal{\faktor{E}{F}}
         \end{equation*}

         Lastly, let $E_1,E_2 \subseteq K$ subfields contain ing $F$, and let
         $H_1,H_2 \leq G$ subgroups fixing $E_1$ and $E_2$. Then $H_1 \cap H_2$
         fixes both $E_1$ and $E_2$, and so fixes $E_1E_2$. Conversely, if $h
         \in H_1$ such that $h$ fixes $E_2$, then $h \in H_1 \cap H_2$. So that
         $H_1 \cap H_2$ corresponds to $E_1E_2$. By similar reasoning, we get
         that $\langle H_1, H_2 \rangle$ corresponds to $E_1 \cap E_2$.
\end{proof}

\begin{example}\label{example_2.4}
    \begin{enumerate}
        \item[(1)] The lattices of $\Q(\sqrt{2},\sqrt{3})$ and $\Q(\sqrt[3]{2},
            \xi)$ (where $\xi^3=1$) indicate all of the subfields of these
            fields. We have that the lattice of $\Q(\sqrt{2},\sqrt{3})$ is
            isomorphic to the lattice of the Klein $4$-group  $V_4$, which has
            all its subgroups normal. Thus we get every subfield of
            $\Q(\sqrt{2},\sqrt{3})$ is Galoix over $\Q$.

            \begin{figure}[h]
                \centering
                \[\begin{tikzcd}
                        & {\Q(\sqrt{2},\sqrt{3})} &&& {\langle \s,
                        \t, \n \rangle} \\ {\Q(\sqrt{2})} & {\Q(\sqrt{6})} &
                        {\Q(\sqrt{3})} & {\langle \s \rangle} &
                        {\langle \t \rangle} & {\langle \n \rangle} \\
                        & {\Q} &&& {\langle \i \rangle}
                        \arrow[no head, from=1-2, to=2-2]
                        \arrow[no head, from=2-2, to=3-2]
                        \arrow[no head, from=3-2, to=2-3]
                        \arrow[no head, from=2-3,to=1-2]
                        \arrow[no head, from=1-2,to=2-1]
                        \arrow[no head,from=2-1, to=3-2]
                        \arrow[no head,from=1-5, to=2-5]
                        \arrow[no head,from=2-5,to=3-5]
                        \arrow[no head,from=3-5,to=2-6]
                        \arrow[no head,from=2-6,to=1-5]
                        \arrow[no head,from=1-5,to=2-4]
                        \arrow[no head,from=2-4,to=3-5]
                \end{tikzcd}\]
                \caption{The lattice of subfields of $\Q(\sqrt{{2},\sqrt{{3}}})$
                and the lattice of subgroups of
            $\Gal{\faktor{\Q(\sqrt{{2},\sqrt{{3}}})}{\Q}}$.}
                \label{figure_2.1}
            \end{figure}

            On the other hand, the lattice for $\Q(\sqrt[3]{2}, \xi)$ is
            isomorphic to the lattice of $S^3$ where the only normal subgroup is
            the nontrivial subgroup of order $3$; moreover, only  $\Q(\xi)$ is
            Galois over $\Q$ with  $\Gal{\faktor{\Q(\xi)}{\Q}} \simeq
            \faktor{S_3}{\langle \s \rangle}$, where $\langle \s \rangle$ is the
            cyclic subgroup of order $2$.

        \item[(2)] Consider $\Q(\sqrt{2}+\sqrt{3}) \subseteq \Q(\sqrt{2},\sqrt{3})$.
            If $m(x)$ is the minimal polynomial of $\sqrt{2}+\sqrt{3}$, then
            observe that is has as roots the distinct conjugates
            \begin{equation*}
                \pm \sqrt{2} \pm \sqrt{{3}}
            \end{equation*}
            so that
            \begin{equation*}
                m(x)=(x+(\sqrt{2}+\sqrt{3}))(x-(\sqrt{2}+\sqrt{3}))
                        (x+(\sqrt{2}-\sqrt{3}))(x-(\sqrt{2}-\sqrt{3}))
                    =x^4+10x+1
            \end{equation*}
            Moreover, $x^4+10x+1$ is irreducible. Then only the automorphism
            $\i$ of  $\{\i, \s, \t, \s\t\}$ fixes $\sqrt{2}+\sqrt{{3}}$ so that
            the fixing group of $\Q(\sqrt{2}+\sqrt{3})$ is preciesly that of
            $\Q(\sqrt{2},\sqrt{3})$. So
            $\Q(\sqrt{}+\sqrt{3})=\Q(\sqrt{2},\sqrt{3})$.

        \item[(3)] Consider the splitting field of $x^8+1$ over  $\Q$, generated
            by the elements  $\sqrt[8]{2}$ and $\xi$, where  $\xi^8=1$  (i.e. a
            primitive $8$-th root of unity). Let $\z=\sqrt[8]{2}$, and notice
            that $\z^4=\sqrt{2}$, and the splitting field of $x^8-2$ over  $\Q$
            is  $\Q(\sqrt[8]{2},i)$ of degree $[\Q(\sqrt[8]{2},i):\Q]=16=4^2$.
            Consider then all possible maps on $\z$ and  $i$ given by $\z
            \xrightarrow{} \xi^a\z, i \xrightarrow{} \pm i$. Define then the
            automorphisms
            \begin{equation*}
                \s:\z \xrightarrow{} \xi\z,i \xrightarrow{} i
                \text{ and }
                \t:\z \xrightarrow{} \z, i \xrightarrow{} -i
            \end{equation*}
            Since
            $\xi=\frac{\sqrt{2}}{2}+i\frac{\sqrt{2}}{2}=\frac{1+i}{2}\z^4$, we
            compute that $\s:\xi \xrightarrow{} \xi^5$, and $\t:\xi
            \xrightarrow{} \xi^7$. We can then compute the Galois group by
            noting that $\s^8=\t^2=\i$, and  $\s\t=\t\s^3$, so that
            \begin{equation*}
                \Gal{\faktor{\Q(\sqrt[8]{2},i)}{\Q}}=
                \langle \s,\t : \s^8=\t^2=\i \text{ and } \s\t=\t\s^3 \rangle
            \end{equation*}
            which describes the quasidihedral group of order $16$.
    \end{enumerate}
\end{example}
