\section{Field Extensions.}

\begin{definition}
    We define the \textbf{characteristic} of a field $F$ to be the smallest
    positive integer  $p$, such that  $p \cdot 1=0$, where  $1$ is the identity
    of  $F$. We write  $\Char{F}=p$, and if no such $p$ exists, then we write
    $\Char{F}=0$.
\end{definition}

\begin{lemma}\label{1.1.1}
    Let $F$ be a field, then  $\Char{F}$ is either $0$, or a prime integer.
\end{lemma}
\begin{proof}
    Let $\char{F}=p$. If $p=0$, then we are done. Now suppose that  $p=mn$, with
     $m,n \in \Z^+$. Then $p \cdot 1=(mn)1=(n \cdot 1)(m \cdot 1)=mn=0$, which
     makes $m$ and  $n$  $0$ divisors. Since  $F$ is a field, and hence an
     integral domain, this is impossible, and hence  $p$ must be prime.
\end{proof}
\begin{corollary}
    If $\Char{F}=p$, then for all $a \in F$,  $pa=\underbrace{a+\dots+a}_{p
    \text{ times}}$.
\end{corollary}
\begin{proof}
    We have $pa=p(a \cdot 1)=(p \cdot 1)a$.
\end{proof}

\begin{example}\label{}
    \begin{enumerate}
        \item[(1)] Both $\Q$ and  $\R$ have  $\Char=0$. Similarly,
            $\Char{\Z}=0$, even though $\Z$ is just an integral domain.

        \item[(2)] $\Char{\faktor{\Z}{p\Z}}=p$ and
            $\Char{\faktor{\Z}{p\Z}[x]}=p$ for any prime $p$.
    \end{enumerate}
\end{example}

\begin{definition}
    We define the \textbf{prime subfield} of a field $F$ to be the subfield of
    $F$ generated by  $1$.
\end{definition}

\begin{example}\label{}
    \begin{enumerate}
        \item[(1)] The prime subfields of $\Q$ and  $\R$ is  $\Q$.

        \item[(2)] Let $\faktor{\Z}{p\Z}(x)$ the field of rational functions
            over $\faktor{\Z}{p\Z}$. Then the prime subfield of
            $\faktor{\Z}{p\Z(x)}$ is $\faktor{\Z}{p\Z}$. Similarly, the prime
            subfield for $\faktor{\Z}{p\Z}[x]$ is also $\faktor{\Z}{p\Z}$.
    \end{enumerate}
\end{example}

\begin{definition}
    If $K$ is a field containing a field $F$, then we call  $K$  \textbf{field
    extension} over $F$, and write  $\faktor{K}{F}$ (not the quotient field!) or
    denote it by the diagram
    \[\begin{tikzcd}
        K \\
        F
        \arrow[no head, from=1-1, to=2-1]
    \end{tikzcd}\]
\end{definition}

\begin{lemma}\label{1.1.2}
    Every field is a field extension of its prime subfield.
\end{lemma}

\begin{lemma}\label{1.1.3}
    Let $K$ an extension over a field  $F$. Then  $K$ is a vector space over
    $F$.
\end{lemma}

\begin{definition}
    Let $\faktor{K}{F}$ a field extension. We define the \textbf{degree} of $K$
    over $F$, $[K:F]$ to be the dimension of  $\faktor{K}{F}$ as a vector space.
\end{definition}

\begin{definition}
    Let $F$ be a field, and  $f \in F[x]$ a polynomial. We call am element $\a
    \in R$ a  \textbf{root} (or \textbf{zero}) of $f$ if  $f(\a)=0$.
\end{definition}

\begin{lemma}\label{1.1.4}
    Let $\phi:F \xrightarrow{} L$ a field homomorphism. Then either $\phi=0$, or
     $\phi$ is 1--1.
\end{lemma}

\begin{lemma}\label{1.1.5}
    Let $F$ be a field, and  $p \in F[x]$ an irreducible polynomial. Then there
    exists a field $K$ containing an embedding of  $F$, such that  $p$ has a
    root in  $K$.
\end{lemma}
\begin{proof}
\end{proof}
\begin{corollary}
    There exists a field extension of $F$ containing a root of  $p$.
\end{corollary}
