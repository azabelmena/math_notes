\section{Definitions and Examples}
\label{section1}

\begin{definition}
    A \textbf{commutative ring} $A$ is a set together with two binary operations
    $+:(a,b) \xrightarrow{} a+b$ and $\cdot:(a,b) \xrightarrow{} ab$ called
    \textbf{additon} and \textbf{multiplication} such that:
    \begin{enumerate}
        \item[(1)] $A$ is an Abelian group over $+$, where we denote the
            identity element as $0$ and the inverse of each $a \in A$ as $-a$.

        \item[(2)] For any $a,b \in A$,  $ab \in A$ and $a(bc)=(ab)c$. That is,
            $A$ is closed under multiplication, and  multiplication is
            associative.

        \item[(3)] $a(b+c)=ab+ac$ and $(a+b)c=ac+bc$.

        \item[(4)] $ab=ba$ for all  $a,b \in A$.
    \end{enumerate}
     If there exists an element $1 \in A$ such that  $a1=1a=A$, then we call
     $A$ a ring with \textbf{identity}. If $1=0$, we call  $A$ the  \textbf{zero
     ring} and write $A=0$.
\end{definition}

\begin{definition}
    A commutative ring $k$ with identity $1 \neq 0$ is called a
    \textbf{field} if for all $a \in k$, where  $a \neq 0$, there exists
    a $b \in A$ such that $ab=1$.
\end{definition}

\begin{lemma}\label{1.1.1}
    Let $A$ be a commutative ring with identity. Then the following are true for
    all $a,b \in A$.
    \begin{enumerate}
        \item[(1)] $0a=a0=0$.

        \item[(2)] $(-a)b=a(-b)=-(ab)$.

        \item[(3)] $(-a)(-b)=ab$

        \item[(4)] $1 \neq 0$, then $1$ is unique and $-a=(-1)a$.
    \end{enumerate}
\end{lemma}
\begin{proof}
    \begin{enumerate}
        \item[(1)] Notice $0a=(0+0)a=0a+0a$, so that $0a=0$. Likewise, $a0=0$ by
            the same reasoning.

        \item[(2)] Notice that $b-b=0$, so $a(b-b)=ab+a(-b)=0$, so that
            $a(-b)=-(ab)$. The same argument with $(a-a)b$ gives $(-a)b=-(ab)$.

        \item[(3)] By the inverse laws of addition in $A$, we have
            $-(a(-b))=-(-(ab))$, so that $(-a)(-b)=ab$.

    \item[(4)] Suppose $A$ has identity $1 \neq 0$, and suppose there is an
        element $2 \in A$ for which  $2a=a2=a$ for all $a \in A$. Then we have
        that $1 \cdot 2=1$ and  $1 \cdot 2=2$, making  $1=2$; so  $1$ is unique.
        Now, we have that $a+(-a)=0$, so that $1(a+(-a))=1a+1(-a)=1a+(-a)=0$ So
        $(-a)=-(1a)=(-1)a$ by (2).
    \end{enumerate}
\end{proof}

\begin{definition}
    Let $A$ be a ring. We call an element  $a \in A$ a  \textbf{zero divisor} if
    $a \neq 0$ and there exists an element  $b \neq 0$ such that  $ab=0$.
    Similarly, we call $a \in A$ a \textbf{unit} if there is a $b \in A$ for
    which  $ab=ba=1$. We call an element $a$  \textbf{nilpotent} if there exists
    some $n \in \Z^+$ for which $x^n=0$.
\end{definition}

\begin{definition}
    Let $A$ be a ring. We call the set of all units in  $A$ the \textbf{group
    of units} and denote it $\Uc(A)$, or $A^\ast$.
\end{definition}

\begin{lemma}\label{1.1.3}
    Let $A$ be a commutative ring with identity $1 \neq 0$. Then the group of units
    $\Uc(A)$ forms an Abelian group under multiplication.
\end{lemma}
\begin{proof}
    Let $a,b \in A$ be units in $A$. Then there are $c,d \in A$ for which
    $ac=ca=1$ and  $bd=db=1$. Consider then $ab$. Then  $ab(dc)=a(bd)c=ac=1$ and
    $(dc)ab=d(ca)b=db=1$ so that $ab$ is also a unit in $A$. Moreover $\Uc(A)$
    inherits the associativity of  $\cdot$ and $1$ serves as the identity
    element of $A^\ast$. Lastly, if $a \in A^\ast$ is a unit there is a $b
    \in A$ for which $ab=ba=1$. This also makes $b$ a unit in $A$, and the
    inverse of $a$. Now, since $A$ is a commutative ring, the multiplication in
     $\Uc(A)$ is commutative, making $\Uc(A)$ Abelian.
\end{proof}
\begin{corollary}
    $a$ is a zero divisor if, and only if it is not a unit.
\end{corollary}
\begin{proof}
    Suppose that $a \neq 0$ is a zero divisor. Then there is a  $b \in A$ such
    that $b \neq 0$ and $ab=0$. Then for any $v \in A$,  $v(ab)=(va)b=0$ so that
    $a$ cannot be a unit. On the other hand let  $a$ be a unit, and  $ab=0$ for
    some  $b \neq 0$. Then there is a  $v \in A$ for which
    $v(ab)=(va)b=1b=b=0$. Then $b=0$ which is a contradiction.
\end{proof}
\begin{corollary}
    If $k$ is a field, then it has no zero divisors.
\end{corollary}
\begin{proof}
    Notice by definition of a field, every element is a unit, except for $0$.
\end{proof}

\begin{definition}
    A commutative ring with identity $1 \neq 0$ is called an \textbf{integral
    domain} if it has no zero divisors.
\end{definition}

\begin{lemma}\label{1.1.3}
    Any finite integral domain is a field.
\end{lemma}
\begin{proof}
    Let $A$ be a finite integral domain and consider the map on $A$, by $x
    \xrightarrow{} ax$. By above, this map is 1--1, moreover since $A$ is
    finite, it is also onto. So there is a $b \in A$ for which $ab=1$, making
    $a$ a unit. Since $a$ is abitrarily chosen, this makes $A$ a field.
\end{proof}
\begin{corollary}
    If $k$ is a field it is a (not necessarily finite) integral domain.
\end{corollary}

\begin{definition}
    A \textbf{subring} of a ring $A$ is a subgroup of $A$ closed under
    multiplication.
\end{definition}
