\section{Extensions and Contractions of Ideals}

For this section, $A$ and  $B$ denote commutative rings with identity.

\begin{definition}
    Let $\phi:A \xrightarrow{} B$ be a ring homomorphism. We define the
    \textbf{extension} of the ideal $\af$ of  $A$ to be the ideal $\af^e$
    generated by $B\phi(\af)$. That is, $\af=B\phi(\af)$.
\end{definition}

\begin{lemma}\label{1.10.1}
    Let $f:A \xrightarrow{} B$ a ring homomorphism and $\af$ an ideal of $A$.
    Then
    \begin{equation*}
        \af^e=\{\sum{y_if(x_i) : y_i \in B \text{ and } x_i \in \af}\}
    \end{equation*}
\end{lemma}
\begin{proof}
    This follows directly from the definition of $\af^e$.
\end{proof}

\begin{definition}
    Let $\phi:A \xrightarrow{} B$ a ring homomorphism. We define the
    \textbf{contraction} of the ideal $\bf$ of  $\bf$ to be the preimage
    $\inv{\phi}(\bf)$, and denote it $\bf^c$; that is,  $\bf^c=\inv{\phi}(\bf)$.
\end{definition}

\begin{lemma}\label{1.10.2}
    Let $\phi:A \xrightarrow{} B$ a ring homomorphism, and $\bf$ and ideal of
    $\bf$. Then $\bf$ is an ideal of $A$. Moreover, if $\bf$ is prime in $B$,
    then  $\bf^c$ is prime in $A$.
\end{lemma}
\begin{proof}
    Let $x \in \bf^c$. Then  $\phi(x) \in \bf$, so that $-\phi(x)=\pji(-x) \in
    \bf$, which puts $-x \in \bf^c$; similarly, we get $x+y \in \bf^c$ whenever
    $x,y \in \bf^c$. Lastly, notice that if  $a \in A$, and  $x \in \bf^c$, then
     $\phi(a)\phi(x)=\phi(ax) \in \bf$, so that $\bf^c$ is an ideal.

     Now, suppose that  $\bf$ is prime. Then since  $\bf \neq (1_B)$, $\bf^c
     \neq (1_A)$. Now, let $ab \in \bf^c$. Then  $\phi(ab)=\phi(a)\phi(b) \in
     \bf$. Since $\bf$ is prime, this puts  $\phi(a) \in \bf$ or $\phi(b) \in
     \bf$; that is, $a \in \bf^c$ or  $b \in \bf^c$. Therefore, $\bf^c$ must
     also be prime.
\end{proof}

\begin{example}\label{example_1.16}
    Let $\phi:A \xrightarrow{} B$ a ring homomorphism. We have that for any prime
    ideal $\bf$ of $B$, $\bf^c$ is prime. The same is not true for extensions.
    If $\af$ is prime in $A$,  $\af^e=B\phi(\af)$ need not be prime in $B$.
\end{example}

\begin{lemma}\label{1.10.3}
    Let $\phi:A \xrightarrow{} B$ be a ring homomorphism, with $f=\i \circ \pi$,
    where  $\pi$ is onto and  $\i$ is 1--1. Then there exists a 1--1
    correspondece between the ideals $\phi(A)$ and the ideals of $A$ containing
     $\ker{\phi}$. Moreover, prime ideals correspond to prime ideals.
     \[\begin{tikzcd}
        A \\
        {\phi(A)} & B
        \arrow["\pi"', from=1-1, to=2-1]
        \arrow["\iota"', from=2-1, to=2-2]
        \arrow["{\phi=\iota \circ \pi}", from=1-1, to=2-2]
      \end{tikzcd}\]
\end{lemma}

\begin{example}\label{example_1.17}
    Consider the map $\Z \xrightarrow{} \Z[i]$ where $i^2=-1$. A prime ideal
    $(p)=p\Z$ may or may not be prime when extended to $\Z[i]$. Now, $\Z[i]$ is
        a PID, so that we have the following.
        \begin{enumerate}
            \item[(1)] $(2)^e=((1+i)^2)$ in $\Z[i]$; that is, it is the square
                of a prime ideal in $\Z[i]$.

            \item[(2)] If $p \equiv 1 \mod{4}$, then $(p^e)$ is the product of
                two prime ideals in $\Z[i]$, and if $p \equiv 3 \mod{4}$,
                $(p)^e$ is a prime ideal in $\Z[i]$.
        \end{enumerate}
\end{example}

\begin{lemma}\label{1.10.4}
    Let $\phi:A \xrightarrow{} B$ be a ring homomorphism. Then the following are
    true for ideals $\af$ and  $\bf$ of  $A$ and  $B$, respectively.
    \begin{enumerate}
        \item[(1)] $\af \subseteq \af^{ec}$ and $\bf^{ce} \subseteq \bf$.

        \item[(2)] $\af^e=\af^{ece}$, and $\bf^c=\bf^{cec}$.

        \item[(3)] If $C$ is the set of all contracted ideals in $A$, and  $E$
            is the set of all extended ideals in  $B$, then
            \begin{equation*}
                C=\{\af \subseteq A : \af=\af^{ec}\} \text{ and }
                E=\{\bf \subseteq B : \bf=\bf^{ce}\}
            \end{equation*}
            Moreover, there exists a 1--1 correspondence of $C$ onto $E$  given
            by the map $\af \xrightarrow{} \af^e$.
    \end{enumerate}
\end{lemma}
\begin{proof}
    First, consider $\af$ in  $A$. Then  $\af^e=B\phi(\af)$, so that if $x \in
    \af$, then  $\phi(x) \in f(\af)$, that is $x \in \af^{ec}$. Similarly, we
    get $\bf^{ce} \subseteq \bf$.

    Now, for the second assertion, we have
    \begin{equation*}
        (\bf^{ce})^c \subseteq \bf^c \subseteq (\bf^c)^{ec}
    \end{equation*}
    so that $\bf^c=\bf^{cec}$. Similarly, we get $\af^e=\af^{ece}$.

    Finally, let $\af \in C$. Then there is a $\bf$ in $B$ for which
    $\af=\bf^c$. Then  $\af^e=\bf^{ce}=\bf^{cec}=\af^{ec}$. Conversely, if
    $\af=\af^{ec}$, then $\af$ is the contraction of $\af^e$. We use a similar
    argument to prove the result for $E$.
\end{proof}

\begin{lemma}\label{1.10.5}
    If $\af_1,\af_2$ are ideals of $A$ and  $\bf_1,\bf_2$ are ideals of $B$,
    then the following are true.
    \begin{enumerate}
        \item[(1)] $(\af_1+\af_2)^e=\af_1^e+\af_2^e$ and $\bf_1^c+\bf_2^c
            \subseteq (\bf_1+\bf_2)^c$.

        \item[(2)] $(\af_1 \cap \af_2)^e \subseteq \af_1^e \cap \af_2^e$ and
            $\bf_1^c \cap \bf_2^c=(\bf_1 \cap \bf_2)^c$.

        \item[(3)] $(\af_1\af_2)^e=\af_1^e\af_2^e$ and $\bf_1^c\bf_2^c
            \subseteq (\bf_1\bf_2)^c$.

        \item[(4)] $(\af_1:\af_2)^e \subseteq (\af_1^e:\af_2^e)$ and
            $(\bf_1:\bf_2)^c=(\bf_1^c:\bf_2^c)$.

        \item[(5)] $(\rad{\af})^e \subseteq \rad{\af^e}$ and
            $(\rad{\bf})^c=\rad{\bf^c}.
    \end{enumerate}
\end{lemma}
