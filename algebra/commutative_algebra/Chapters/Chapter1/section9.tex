\section{Operations on Ideals}

We observe some additional properties, of ideals, namely, concerining operations
on ideals. For this section, assume we are working over a commutative ring $A$
with identity, unless otherwise specified.

\begin{lemma}\label{1.6.1}
    Let $A$ be a commutative ring with identity, and let $\af$, and $\bf$ ideals
    of  $A$. Then the following are true
    \begin{enumerate}
        \item[(1)] $\af+\bf$ is the smallest ideal of $A$ containing both $\af$
            and  $\bf$.

        \item[(2)] $\af\bf=\{\sum{x_iy_i} : x_i \in \af \text{ and } y_i \in \bf\}$.

        \item[(3)] $a\f\bf \subseteq \af \cap \bf$, and $\af \cap \bf$ is the
            largest ideal contained in both $\af$ and $\bf$.
    \end{enumerate}
\end{lemma}

\begin{lemma}\label{1.6.2}
    Sums, intersections, and products of ideals in a commutative ring with
    identity are commutative, and associative. Moreover, the product of ideals
    distributes over the sum of ideals. That is, if $\af$, $\bf$, and $\cf$ are
    ideals, then
    \begin{equation*}
        \af(\bf+\cf)=\af\bf+\af\cf
    \end{equation*}
\end{lemma}

\begin{lemma}\label{1.6.3}
    For any ideals $\af$, $\bf$, and $\cf$
    \begin{equation*}
        \af \cap (\bf+\cf)=\af \cap \bf+\af \cap \cf \text{ if }
        \bf \subseteq \af \text{ or } \cf \subseteq \af
    \end{equation*}
\end{lemma}

\begin{definition}
    We call two ideals $\af$ and  $\bf$ \textbf{coprime}, or \textbf{comaximal}
    if $\af+\bf=(1)$.
\end{definition}

\begin{lemma}\label{1.6.4}
    The following are true for anyu ideals $\af$ and  $\bf$.
    \begin{enumerate}
        \item[(1)] if $\af$ and $\bf$ are coprime, then $\af \cap \bf=\af\bf$.

        \item[(2)] $\af$ and  $\bf$ are coprime if, and only if there exists an
            $x \in \af$ and a $y \in \bf$ for which $x+y=1$.
    \end{enumerate}
\end{lemma}
\begin{corollary}
    If $m,n \in \Z^+$ are coprime then their ideals $n\Z$ and $m\Z$ are coprime.
\end{corollary}

\begin{definition}
    Let $\{A_\a\}$ be a (not necessarily countable) collection of commutative
    rings with identity. We define the \textbf{direct product} of $\{A_\a\}$ to
    be the set
    \begin{equation*}
        A=\prod_{\a}{A_\a}
    \end{equation*}
\end{definition}

\begin{lemma}\label{1.6.5}
    Let $\{A_\a\}$ be a collection of commutative rings with identity. Then the
    direct product of $\{A_\a\}$ forms a commutative ring with identity under
    componentwise addition and componentwise multiplication.
\end{lemma}

\begin{lemma}\label{1.6.6}
    Let $A$ be a commutative ring, and $\af_1, \dots, \af_n$ ideals of $A$.
    Define the ring homomorphism  $\phi:A \xrightarrow{}
    \prod{\faktor{A}{\af_i}}$ by
    \begin{equation*}
        \phi:x \xrightarrow{} (x+\af_1, \dots, x+\af_n)
    \end{equation*}
    Then the following are true.
    \begin{enumerate}
        \item[(1)] If $\af_i$ and  $\af_j$ are coprime whenever  $i \neq j$,
            then
            \begin{equation*}
                \prod{\af_i}=\bigcap{\af_i}
            \end{equation*}

        \item[(2)] $\phi$ is onto if, and only if  $\af_i$ and  $\af_j$ are
            coprime whenever $i \neq j$.

        \item[(3)] $\phi$ is 1--1 if, and only if
            \begin{equation*}
                \bigcap{\af_i}=(0)
            \end{equation*}
    \end{enumerate}
\end{lemma}
\begin{proof}
    By induction on $n$ it was shown that for $n=2$ that if $\af_1,\af_2$ are
    coprime, then $\af_1\af_2=\af_1 \cap \af_2$. Now, suppose that
    \begin{equation*}
        \bf=\prod_{i=1}^n{\af_i}=\bigcap_{i=1}^n{\af_i}
    \end{equation*}
    for all $n \geq 2$ and consider the case for $n+1$. Since
    $\af_i+\af_{n+1}=(1)$ (they are coprime by hypothesis), we have $x_i+y_i=1$
    where  $x_i \in \af_i$ and  $y_i \in \af_{n+1}$. Hence notice that
    \begin{equation*}
        \prod_{i=1}^n{x_i}=\prod_{i=1}^n{1-y_i} \equiv 1 \mod{\af_{n+1}}
    \end{equation*}
    so that $\bf+\af_{n+1}=(1)$. Hence $\bf\af_{n+1}=\bf \cap \af_{n+1}$ which
    completes the proof.

    Suppose now, that $\phi$ is onto. Then there exists an  $x \in A$ such that
     $\phi(x)=(1,0, \dots, 0)$ so that $x \equiv 1 \mod{\af_1}$ and $x \equiv 0
     \mod{a_i}$ for $i>1$. Hence  $1=(1-x)+x \in \af_1+\af_i$ for all $i>1$.
     This makes  $\af_1$ and $\af_i$ coprime. We can repeat this argument for
     any inex  $j \neq i$. Conversely suppose that $\af_1$ and $\af_i$ are
     coprime. Then $\af_1+\af_i=(1)$ for all $i>1$ and we have  $u_i+v_i=1$ for
     some  $u_i \in \af_1$ and $v_i \in \af_i$. Take then  $x=\prod{v_i}$. Then
     \begin{equation*}
         x=\prod{1-u_i} \equiv 1 \mod{\af_1} \text{ and } x \equiv 0 \mod{\af_i}
     \end{equation*}
     thus $\phi(x)=(1,0, \dots, 0)$. repeating for each index $j \neq i$, we get
     that  $\phi$ is onto. Finally, observe that
     \begin{equation*}
        \ker{\phi}=\{x \in A : (x+\af_1, \dots,x+\af_n)=(\af_1, \dots, \af_n)\}=
      \bigcap_{i=1}^n{\af_n}
     \end{equation*}
     Which gives us the equivalent condition for $\phi$ to be 1--1.
\end{proof}

\begin{lemma}\label{1.6.7}
    The following are true for any commutative ring with identity.
    \begin{enumerate}
        \item[(1)] If $\pf_1, \dots, \pf_n$ are prime ideals, containing an
            ideal $\af$, then $\af \subseteq \pf_i$ for some  $1 \leq i \leq n$.

        \item[(2)] If $\af_1, \dots, \af_n$ are ideals, and $\pf$ is a prime
            ideal containing  $\bigcap{\af_i}$, then $\af_i \subseteq \pf$ for
            some  $1 \leq i \leq n$.
    \end{enumerate}
\end{lemma}
\begin{proof}
    For the first assertion, the result is vacaciously true for $n=1$. Now
    suppose the result is true for all  $n \geq 1$. Then for every  $1 \leq i
    \leq n$, there is an  $x_i \in \af$ such that  $x_i \in \pf_j$ whenever  $i
    \neq j$. Now, if  $x_i \notin \pf_i$, we are done. Otherwise, $x_i \in \pf$,
    and consider
    \begin{equation*}
        y=\sum_{i=1}^n{x_1x_2 \dots x_{i-1}x_{i+1} \dots x_n}
    \end{equation*}
    Then $y \in \af$ but $y \notin \pf_i$, which puts  $\af \not\subseteq
    \pf_i$, for all $1 \leq i \leq n$, hence $\af \subseteq \pf_{n+1}$ and we
    are done.

    For the second assertion, suppose that $\af_i \not\subseteq \pf$ for all
    $1 \leq i \leq n$. Then let  $x_i \in \af$ such that  $x_i \notin\pf$. Then
    we have
    \begin{equation*}
        \prod{\x_i} \in \prd{\af_i}
    \end{equation*}
    but $\prod{x_i} \notin \pf$, hence $\bigacap{\af_i} \not\subseteq \pf$.
\end{proof}
\begin{corollary}
    If $\pf=\bigcap{\af_i}$, then $\pf=\af_i$ for some  $1 \leq i \leq n$.
\end{corollary}

\begin{definition}
    Let $\af$ and $\bf$ be ideals. We define the \textbf{ideal quotient} of
    $\af$ and  $\bf$ to be the set
    \begin{equation*}
        (\af:\bf)=\{x \in A : x\bf=\af\}
    \end{equation*}
    We define the \textbf{annihilator} of $\bf$ to be $(0:\bf)$ and denote it
    $\Ann{\bf}$.
\end{definition}

\begin{lemma}\label{1.6.8}
    Ideal quotients of ideals are ideals.
\end{lemma}
\begin{proof}
    Let $\af$ and $\bf$ be ideals. Then if $x \in (\af;\b:)$, we have $x\bf
    \subseteq \af$. Now, let $a \in A$. Then $a(x\bf)=(ax)\bf \subseteq \af$ so
    that $ax \in (\af:\bf)$. Notice also that since $x\bf \subseteq \af$, then
    $-x\bf \subseteq \af$. Now, let $x,y \in (\af:\bf)$. Then $x\bf \subseteq
    \af$ and $y\bf \subseteq \af$, thus $x\bf+y\bf=(x+y)\bf \subseteq \af$ so
    that $x+y \in (\af:\bf)$.
\end{proof}
\begin{corollary}
    $\Ann{\bf}$ is an ideal. Moreover, the set of zero divisors in the
    underlying ring is given by
    \begin{equation*}
        D=\bigcup_{x \neq 0}{\Ann{(x)}}
    \end{equation*}
\end{corollary}

\begin{example}\label{example_1.7}
    Let $m,n \in \Z$, where  $m=\prod{p^{\m_p}}$ and $n=\prod{p^{\v_p}}$. Then
    $(m\Z:n\Z)=q\Z$ where
    \begin{equation*}
        q=\prod{p^{\y_p}} \text{ and }
        \y_p=\max{\{\m_p-\v_p,0\}}=\m_p-\min{\{\m_p,\v_p\}}
    \end{equation*}
\end{example}

\begin{lemma}\label{1.6.9}
    The following are true for any ideals $\af$,  $\bf$, and  $\cd$.
    \begin{enumerate}
        \item[(1)] $\af \subseteq (\af:\bf)$.

        \item[(2)] $(\af:\bf)\bf \subseteq \af$.

        \item[(3)] $((\af:\bf):\cf)=(a:\bf\cf)=((\af:\cf):\bf)$.

        \item[(4)] If $\{\af_i\}$ is a collection of ideals, then
            \begin{equation*}
                \Big{(} \bigcap{\af_i}:\bf \Big{)}=\bigcap{(\af_i:\bf)}
            \end{equation*}

        \item[(5)] If $\{b_i\}$ is a collection of ideals, then
            \begin{equation*}
                \Big{(} \af:\sum{\bf} \Big{)}=\bigcap{(\af:\bf_i)}
            \end{equation*}
    \end{enumerate}
\end{lemma}
\begin{proof}
    Left as an excercise.
\end{proof}

\begin{definition}
    For every ideal $\af$ of a commutative ring $A$, with identity, we define
    the \textbf{radical} of $\af$ to be the set
    \begin{equation*}
        \rad{\af}=\{x \in A : x^n \in \af \text{ for some } n \in \Z^+\}
    \end{equation*}
\end{definition}

\begin{lemma}\label{1.6.10}
    For any ideal $af$, $\rad{\af}$ is an ideal of $A$.
\end{lemma}
\begin{proof}
    Consider the natural map $\phi:A \xrightarrow{} \faktor{A}{\af}$ given by $x
    \xrightarrow{} x+\af$. Then notice that
    \begin{equation*}
        \rad{\af}=\inv{\phi}(\Nil{\faktor{A}{\af}})
    \end{equation*}
\end{proof}

\begin{lemma}\label{1.6.11}
    The following are true for any ideals $\af$ and  $\bf$.
    \begin{enumerate}
        \item[(1)] $\af \subseteq \rad{\af}$.

        \item[(2)] $\rad{(\rad{\af})}=\rad{\af}$.

        \item[(3)] $\rad{\af\bf}=\rad{(\af \cap \bf)}=\rad{\af} \cap \rad{\bf}$.

        \item[(4)] $\rad{\af}=(1)$ if, and only if $\af=(1)$.

        \item[(5)] $\rad{\af+\bf}=\rad{(\rad{\af}+\rad{\bf})}$.

        \item[(6)] If $\pf$ is a prime ideal, then  $\rad{\pf^n}=\pf$ for any $n
            \in \Z^+$.
    \end{enumerate}
\end{lemma}

\begin{lemma}\label{1.6.12}
    The radical of an ideal $\af$ is the intersection of all prime ideals
    containing  $\af$.
\end{lemma}

\begin{lemma}\label{1.6.13}
    The set of zerodivisors of a commutative ring with identity is
    \begin{equation*}
        D=\bigcup_{x \neq 0}{\rad{(\Ann{(x)})}}
    \end{equation*}
\end{lemma}

\begin{example}\label{example_1.8}
    Let $m \in \Z$ and  $p_i \in \Z^+$ for $1 \leq i \leq r$ distinct prime
    divisros of $m$. Then
    \begin{equation*}
        \rad{m\Z}=(p_1 \dots p_r)\Z=\bigcap_{i=1}^r{p_i\Z}
    \end{equation*}
\end{example}

\begin{lemma}\label{1.6.14}
    Let $\af$ and  $\bf$ be ideals such that  $\rad{\af}$ and $\rad{\bf}$ are
    coprime. Then $\af$ and  $\bf$ are coprime.
\end{lemma}
\begin{proof}
    We have
    \begin{equation*}
        \rad{(\af+\bf)}=\rad{(\rad{\af}+\rad{\bf})}=\rad{(1)}=(1)
    \end{equation*}
    this makes $\af+\bf=(1)$.
\end{proof}
