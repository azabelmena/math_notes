\section{Noetherian Rings}

\begin{definition}
    Let $A$ be a commutative ring with identity. We say a sequence of ideals
    $\{\af_n\}$ is an \textbf{ascending chain} of ideals if $\af_n \subseteq
    \af_{n+1}$ for all $n \in \Z^+$. We say that the chain  $\{\af_n\}$
    \textbf{stabalizes} if there exists some $k \geq n$, $\af_k=\af_n$.
\end{definition}

\begin{definition}
    Let $A$ be a commutative ring with identity. We call $A$ \textbf{Noetherian}
    if every ascending chain of ideals of $A$ stabalizes. We say that $A$
    satisfies the \textbf{ascending chain condition} on ideals.
\end{definition}

\begin{lemma}\label{1.12.1}
    If $\af$ is an ideal of a Noetherian ring  $A$, then the factor ring
    $\faktor{A}{\af}$ is also Noethrian. In particular, the image of a Noetherian
    ring under any ring homomorphism is Noetherian.
\end{lemma}
\begin{proof}
    This follows by the isomorphism theorems for ring homomorphisms.
\end{proof}

\begin{theorem}\label{1.12.2}
    The following are equivalent for any ring $A$.
    \begin{enumerate}
        \item[(1)] $A$ is Noetherian.

        \item[(2)] Every nonempty collection of ideals of $A$ contains a maximal
            element under inclusion.

        \item[(3)] Every ideal of $A$ is finitelt generated.
    \end{enumerate}
\end{theorem}
\begin{proof}
    Let $A$ be Noetherian, and let  $\Ac$ an nonempty collection of ideals of
    $A$. Choose an ideal  $\af_1 \in \Ac$. If $\af_1$ is maximal, we are done. If
    not, then there is an ideal $\af_2 \in \Ac$ for which  $\af_1 \subseteq \af_2$.
    Now, if $\af_2$ is maximal, we are done. Otherwise, proceeding inductively, if
    there are no maximal ideals of $A$ in $\Ac$, then by the axiom of choice,
    construct the infinite strictly increasing chain
    \begin{equation*}
         \af_1 \subseteq \af_2 \subseteq \dots
    \end{equation*}
    of ideal of $A$. This contradicts that  $A$ is Noetherian, so  $\Ac$ must
    contain a maximal element.

    Now, suppose that any nonempty collection of ideals of  $A$ contains
    a maximal element. Let  $\Ac$ the collection of all finitely generated
    ideals of $A$, and let $\af$ be any ideal of  $A$. By hypothesis, $\Ac$ has a
    maximal element  $\af'$. Now suppose that $\af \neq \af'$, and choose an
    $x \in \com{\af}{\af'}$, then the ideal generated by $\af'$ and  $x$ is finitely
    generated, and so is in  $\Ac$; but that contradicts the maximality of  $\af'$.
    Therefore we must have  $\af=\af'$.

    Finally, suppose every ideal of $A$ is finitely genrated, and let
    $\af=(a_1, \dots, a_n)$. Let
    \begin{equation*}
        \af_1 \subseteq \af_2 \subseteq \dots
    \end{equation*}
    an ascending chain of ideals of $A$ for which
    \begin{equation*}
        \af=\bigcup_{n \in \Z^+}{\af_n}
    \end{equation*}
    Since $a_i \in \af$ for each  $1 \leq j \leq n$, we have that  $a_i \in
    \af_{i_j}$ and $i \in \Z^+$. Now, let  $m=\max{\{j_1, \dots, j_n\}}$ and
    coinsider the ideal $\af_m$. Then  $a_i \in \af_m$ for each $i$, which makes
    $\af \subseteq \af_m$. That is, $\af_n=\af_m$ for some  $n \geq m$; which
    makes $A$ Noetherian.
\end{proof}

\begin{example}\label{example_1.1}
    \begin{enumerate}
        \item[(1)] Every principle ideal domain (PID) is Noetherian, since any
            collection of ideals has a maximal element. Moreover, lemma
            \ref{1.7.5} states that PIDs satisfy the ascending chain condition.

        \item[(2)] The rings $\Z$, $\Z[i]$, and $k[x]$ (where $k$ is a field)
            are Noetherian.

        \item[(3)] The multivariate polynomial ring $\Z[x_1, x_2, \dots]$ is not
            Noetheria, since the ideal $(x_1, x_2, \dots)$ is not finitely
            generated.
    \end{enumerate}
\end{example}

\begin{theorem}[Hilbert's Basis Theorem]\label{1.12.3}
    If $A$ is a Noetherian ring, then so is the polynomial ring $A[x]$.
\end{theorem}
\begin{proof}
    Let $\af$ be an ideal of  $A[x]$, and let $L$ be the set of all leading
    coefficients of polyonimials in $\af$. Notice that since  $0 \in \af$, then  $0
    \in L$, so that $L$ is nonempty. Moreover, let $f(x)=ax^d+\dots$ and
    $g(x)=bx^e+\dots$ polynomials in $\af$ of degree  $\deg{f}=d$ and $\deg{g}=e$,
    with leading coefficients $a, b \in A$. Then for any  $r \in A$, we have the
    coefficient  $ra-b=0$, or  $ra-b$ is the leading coefficient of the
    polynomial  $rx^ef-x^dg \in \af$. In either case, we get  $ra-b \in L$. This
    makes  $L$ an ideal of  $A$. Now, since  $A$ is Noetherian  $L$ is finitely
    generated ; let $L=(a_1, \dots, a_n)$. Then for every $1 \leq  i \leq n$,
    let  $f_i \in \af$ the polynomial of degree  $\deg{f_i}=e_i$ whose leading
    coefficient is $a_i$. Take, then  $N=\max{\{e_1, \dots, e_n\}}$. Then for
    any $d \in \faktor{\Z}{N\Z}$, let $L_d$ be the set of all leading
    coefficients of polynomials in  $\af$, of degree $d$, together with $0$. Let
    $f_{di} \in \af$ a polynomial of degree $\deg{f_{di}}=d$ with leading
    coefficient $b_{di}$. We wish to show that
    \begin{equation*}
        \af=(f_1, \dots, f_n) \cup (f_{d1}, \dots f_{nd})
    \end{equation*}

    Let $\af'=(f_1, \dots, f_n) \cup (f_{d1}, \dots f_{nd})$. By construction,
    since the generators were chosen from $\af$,  $\af' \subseteq \af$. Now, if
    $\af \neq \af'$. Then there is a nonzero polynomial $f \in \af$ of minimum degree
     not contained in $\af'$  (i.e $f \notin \af'$). Let $\deg{f}=d$, and let $a$ be
     the leading coefficient of  $f$. Suppose that  $d \geq N$. Since  $a \in
     L$, $a$ is an  $A$-linear combination of the generators of  $L$; i.e.
     \begin{equation*}
         a=r_1a_1+\dots+r_na_n
     \end{equation*}
     where $r_1, \dots, r_n \in A$. Let
     \begin{equation*}
        g=r_1x^{d-e_1}f_1+\dots+r_nx^{d-e_n}f_n
     \end{equation*}
     then $g \in \af'$ and has degree $\deg{g}=d$ and leading coefficient $a$.
     Hence  $f-g \in \af'$ is of smaller degree, and by the minimality of  $f$,
     $f-g=0$, which makes  $f=g \in \af'$; a contradiction. THerefore $\af=\af'$

     Now, if $d<N$, then  $a \in L_d$, and so is an  $A$-linear combniation of
     generators of  $L_d$; that is
     \begin{equation*}
         a=r_1b_{d1}+\dots+r_nb_{dn}
     \end{equation*}
     where $r_1, \dots, r_n \in A$. Then let
     \begin{equation*}
         g=r_1f_{d1}+\dots+r_nf_{dn}
     \end{equation*}
     then $g \in \af'$ is a polynomial of degree  $\deg{g}=d$ and leading
     coefficient $a$; which gives us the above contraditction.

     Therefore, $\af=\af'$, and since $\af'$ is finitely generated,  $A[x]$ is
     Noetherian.
 \end{proof}
 \begin{corollary}
     Let $k$ be a field. Then the polynomial ring in  $n$ variables  $k[x_1,
     \dots, x_n]$ is Noetherian.
 \end{corollary}

 \begin{definition}
     Let $k$ be a field. We call a ring  $A$ a  \textbf{$k$-algebra} if $k$ is
     contained in the center of  $A$  (i.e. $k \subseteq Z(A)$), and $1_k=1_A$.
     We call  $A$ a  \textbf{finitely generated} $k$-algbera if $A$ is generated
     by  $k$ together with a finite set $\{r_1, \dots, r_n\}$ of elements of
     $A$.
 \end{definition}

 \begin{definition}
     Let $k$ be a field and $A$ and  $S$  $k$-algebras. We call a map  $\phi:A
     \xrightarrow{} S$ a \textbf{$k$-algebra homomorphism} if $\phi$ is a ring
     homomorphism, and  $\phi$ is the identity on $k$.
 \end{definition}

 \begin{lemma}\label{1.12.4}
     Let $k$ be a field. Then a ring $A$ is a finitely generated $k$-algebra if,
     and only if there exists a $k$-algebra homomorphism $\phi:k[x_1, \dots,
     x_n] \xrightarrow{} A$ taking $k[x_1, \dots, x_n]$ onto $A$.
 \end{lemma}
 \begin{proof}
     If $A$ is generated by elements  $r_1, \dots, r_n$ as a $k$-algebra, then
     define the map $\phi:k[x_1, \dots, x_n] \xrightarrow{} A$ by taking $x_i
     \xrightarrow{} r_i$, for all $1 \leq i \leq n$, and  $\phi(a)=a$ for all $a
     \in k$. Then  $\phi$ extends to a ring homomorphism of  $k[x_1, \dots,
     x_n]$ onto $A$.

     Conversly, let  $\phi$ be a $k$-algebra homomorphism of $k[x_1, \dots,
     x_n]$ onto $A$,  such that the images $\phi(x_1), \dots \phi(x_r)$ generate
     $A$ as a  $k$-algebra. Then $A$ is finitely generated, and since  $k[x_1,
     \dots, x_n]$ is Notherian by the corollory to Hilbert's basis theorem, $A$
     is a quotient of a Noetherian ring, and hence  $A$ is Noetherian. This
     makes  $A$ a finitely generated  $k$-algebra.
 \end{proof}

 \begin{example}\label{example_1.2}
     Let $A$ be a  $k$-algebra, for some field  $k$, viewed as a finite
     dimensional vector space over  $k$. In particular, let
     $A=\faktor{k[x]}{(f(x))}$, where $f(x)$ is a nonzero polynomial over $k$.
     Then  $A$ is a finitely generated  $k$-algebra, since it has a finite
     basis, and that basis serves as a generator for $A$ as a  $k$-algebra.
     Thus, we have the ideals of $A$ are $k$-subspaces. Moreover, any ascending
     chain of ideals of $A$ has at most  $\dim_k{A}-1$ distinct terms, which
     means that $A$ satisfies the ascending chain condition.
 \end{example}
