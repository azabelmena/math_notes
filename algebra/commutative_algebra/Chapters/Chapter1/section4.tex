\section{Properties of Ideals}

\begin{definition}
    Let $R$ be a commutative ring with identity. We call the smallest ideal
    containing a nonempty subset $A$ in $R$ the  \textbf{ideal generated} by
    $A$, and we write $(A)$. We call an ideal \textbf{principle} if it is
    generated by a single element of $R$, i.e. $\af=(a)$ for some $a \in \af$. We
    say that the ideal $(A)$ is \textbf{finitely generated} if $|A|$ is finite,
    and if  $A=\{a_1, \dots, a_n\}$, then we denote $(A)=(a_1, \dots, a_n)$.
\end{definition}

\begin{example}\label{1.13}
    \begin{enumerate}
        \item[(1)] In any commutative ring with identity, the trivial ideal and
            $R$ are the ideals generated by $0$ and $1$, respectively, so we
            write them as $(0)$ and $R=(1)$.

        \item[(2)] In $\Z$, we can write the ideals  $n\Z=(n)=(-n)$. Notice that
            every ideal in $\Z$ is a principle ideal. Moreover, for $m,n \in
            \Z$, $n|m$ if, and only if  $n\Z \subseteq n\Z$. Notice that
            $m\Z+n\Z=d\Z$ is the ideal generated by $n$ and $n$, where $d=(m,n)$
            is the greatest common divisor of $m$ and  $n$. Indeed, by
            definition, $d|m,n$ so that $d\Z \subseteq m\Z+n\Z$, and if
            $c|m,n$, then  $c|d$, making  $m\Z+n\Z \subseteq d\Z$. Then
            $m\Z+n\Z=d\Z$ is the ideal generated by the greatest common divisor
             $(m,n)$ and consists of all diophantine equations of the form
             \begin{equation*}
                 mx+ny=(m,n)
             \end{equation*}
             In general, we can define the \textbf{greatest common divisor} for
             integers $n_1, n_2, \dots, n_m$ to be the smallest such integer $d$
             generating the ideal $n_1\Z+\dots+n_m\Z=d\Z$. We then write
             $d=(n_1, \dots, n_m)$.

         \item[(3)] Consider the ideal $(2,x)$ of $\Z[x]$. $(2,x)$ is not a
             principle ideal. We have that $(2,x)=\{2p_xq : p,q \in \Z[x]\}$,
             and that $(2,x) \neq \Z[x]$. Suppose that $(2,x)=(a)$ for some
             polynomial $a \in \Z[x]$, then $2 \in (a)$, so that $2=p(x)a(x)$,
             of degree $\deg{p}+\deg{a}$. This makes $p$ and $a$ constant
             polynomials in $\Z[x]$. Now, since $2$ is prime in $\Z$, then only
             values for $p$ and  $q$ are  $p=\pm1$ and $a=\pm2$. If $a(x)=\pm1$,
             then every polynomial in $\Z[x]$ can be written as a polynomial in
             $(a)$, so that $(a)=\Z[x]$, impossible. If $a(x)=\pm2$, then since
             $x \in (a)$, we get $x=2q(x)$ where $q \in \Z[x]$. This cannot
             happen, so that $(a) \neq (2,x)$.
    \end{enumerate}
\end{example}

\begin{lemma}\label{1.4.1}
    Let $\af$ an ideal in ring $R$ with identity. Then
    \begin{enumerate}
        \item[(1)] $\af=(1)$ if, and only if $\af$ contains a unit.

        \item[(2)] If $R$ is commutative, then $R$ is a field if, and only if
            its only ideals are $(0)$ and $(1)$.
    \end{enumerate}
\end{lemma}
\begin{proof}
    Recall that $R=(1)$. Now, if $\af=(1)$, then $1 \in \af$, and  $1$ is a unit.
    Conversly, suppose that $u \in \af$ with $u$ a unit. By definition, we have
    that  $r=r \cdot 1=r(uv)=r(vu)=(rv)u$, so that $1 \in \af$. This makes
    $\af=(1)$.

    Now, if $R$ is a field, then it is a commutative ring, moreover every  $r
    \neq 0$ is a unit in $R$, which makes $r \in \af$ for some ideal $\af \neq
    (0)$. This makes every $\af \neq (0)$ equal to $(1)$. Conversly, if $(0)$ and
    $(1)$ are the only ideals of the commutative ring $R$, then every $r \neq 0
    \in (1)$, which makes themn units. Hence all nonzero $r$ is a unit in $R$.
    This makes $R$ into a field.
\end{proof}
\begin{corollary}
    If $k$ is a field, then any nonzero ring homomorphism $\phi$ defined on $k$
    is 1--1.
\end{corollary}
\begin{proof}
    If $k$ is a field, then either $\ker{\phi}=(0)$ or $\ker{\phi}=(1)$. Now,
    since $\ker{\phi} \neq R$, we must have $\ker{\phi}=(0)$.
\end{proof}

\begin{definition}
    For any ideal $\mf$ in a ring $R$, we call $\mf$  \textbf{maximal}
    if $\mf \neq R$, and if  $\nf$ is an ideal with  $\mf \subseteq N \subseteq
    R$, then either  $\mf=\nf$ or $\nf=R$.
\end{definition}

\begin{lemma}\label{1.4.2}
    If $R$ is a commutative ring with identity, every proper ideal is contained
    in a maximal ideal.
\end{lemma}
\begin{proof}
    Let $\af$ a proper ideal of $R$. Let $\Sc=\{N : N \neq (1) \text{ is a proper
    ideal, and } \af \subseteq N\}$. Then $\Sc \neq \emptyset$, as  $\af \in \Sc$,
    and the relation $\subseteq$ partially orders $\Sc$. Let  $\Cc$ be a chain
    in $\Sc$ and define
    \begin{equation*}
        J=\bigcup_{A \in \Cc}{A}
    \end{equation*}
    We have that $J \neq \emptyset$ since  $(0) \in J$. Now, let $a,b \in J$ ,
    then we have that either  $(a) \subseteq (b)$ or $(b) \subseteq (a)$, but
    not both. In either case, we have $a-b \in J$ so that  $J$ is closed under
    additive inverse. Moreover, since  $A \in \Cc$ is an ideal, by definition,
    $J$ is closed with respect to absorbption. This makes  $J$ an ideal.

    Now, if  $1 \in J$, then $J=(1)$ and $J$ is not proper, and $A=(1)$ by
    definition of $J$. This is a contradiction as $A$ must be proper. Thereofre
    $J$ must also be a proper ideal. Therefore,  $\Cc$ has an upperbound in
    $\Sc$, therefore, by Zorn's lemma,  $\Sc$ has a maximal element $\mf$, i.e.
    it has a maximal ideal $\mf$ with  $\af \subseteq \mf$.
\end{proof}

\begin{lemma}\label{1.4.3}
    Let $R$ be a commutative ring with identity. An ideal $\mf$ is maximal if,
    and only if  $\faktor{R}{\mf}$ is a field.
\end{lemma}
\begin{proof}
    If $\mf$ is maximal, then ther is no ideal $I \neq (1)$ for which $\mf
    \subseteq \af \subseteq R$ By the fourth isomorphism theorem, the ideals of
    $R$ containing  $\af$ are in 1--1 correspondence with the those of
    $\faktor{R}{\mf}$. Therefore $\mf$ is maximal if, and only if the only
    ideals of $\faktor{R}{\mf}$ are $(\mf)$ and $(1+\mf)$.
\end{proof}

\begin{example}\label{1.15}
    \begin{enumerate}
        \item[(1)] Let $n \geq 0$ an integer. The ideal  $n\Z$ is maximal in
            $\Z$ if and only if  $\faktor{\Z}{n\Z}$ is a field. Therefore $n\Z$
            is maximal if, and only if  $n=p$ a prime in $\Z$. So the maximal
            ideals of $\Z$ are those  $p\Z$ where  $p$ is prime.

        \item[(2)] $(2,x)$ is not principle in $\Z[x]$, but it is maximal in
            $\Z[x]$, as $\faktor{\Z[x]}{(2,x)} \simeq \faktor{\Z}{2\Z}$ which
            is a field.

        \item[(3)] The ideal $(x)$ is not maximal in $\faktor{\Z}{n\Z}$, since
            $\faktor{\Z}{(x)} \simeq \Z$, which is not a field. Moreover, $(x)
            \subseteq (2,x) \subseteq \Z[x]$. We construct this isomorphism by
            identifying $x=0$, then all polynomials of $\faktor{\Z[x]}{(x)}$ only
            have constant term in $\Z$.
    \end{enumerate}
\end{example}

\begin{definition}
    We call an ideal $\pf$ in a commutative ring  $R$ with identity
    \textbf{prime} if $\pf \neq (1)$ and if $ab \in \pf$ then either $a \in \pf$
    or $b \in \pf$. Alternatively, if $(ab) \subseteq \pf$ then $(a) \subseteq
    \pf$ or $(b) \subseteq \pf$.
\end{definition}

\begin{example}\label{1.16}
    The prime ideals of $\Z$ are $p\Z$ with  $p$ prime together with $(0)$.
\end{example}

\begin{lemma}\label{1.4.4}
    An ideal $\pf$ in a commutative ring with identity, $R$, is prime if, and
    only if $\faktor{R}{pf}$ is an integral domain.
\end{lemma}
\begin{proof}
    Suppose that $\pf$ is prime, and let  $(a+\pf)(b+\pf)=ab+\pf=\pf$. This
    gives us that $ab \in \pf$ and hence  $a \in \pf$ or  $b \in \pf$. Then
    either $a+\pf=\pf$ or $b+\pf=\pf$ in $\faktor{R}{\pf}$. Conversly, if
    $\faktor{R}{\pf}$ is an integral domain, then for any $a+\pf,b+\pf$
    $ab+\pf=\pf$ implies that either $a+\pf=\pf$ or  $b+\pf=\pf$. Then
     $a \in \pf$ or  $b \in \pf$, only when  $ab \in \pf$. This makes $\pf$ prime.
\end{proof}
\begin{corollary}
    Every maximal ideal is a prime ideal.
\end{corollary}

\begin{example}\label{1.17}
    \begin{enumerate}
        \item[(1)] The prime ideals of $\Z$ are $p\Z$, where $p$ is prime, which
            are the maximal ideals of $\Z$.

        \item[(2)] The ideal $(x)$ in $\Z[x]$ is a prime ideal, but it is not
            maximal as $(x) \subseteq (2,x) \subseteq \Z[x]$.
    \end{enumerate}
\end{example}

\begin{definition}
    We define a commutative ring with identity to be a \textbf{local ring} if it
    has only one maximal ideal. We define the \textbf{residue field} of $R$ to
    be the field $k=\faktor{R}{\af}$.
\end{definition}
