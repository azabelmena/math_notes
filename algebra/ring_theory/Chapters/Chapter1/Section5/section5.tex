\section{Rings of Fractions.}

\begin{lemma}\label{1.5.1}
    Let $R$ a commutative ring, and  $D \subseteq R$ be nonempty with $0 \notin
    D$ such that $D$ contains no zero divisors of $R$ and that it is closed
    under multiplication. Define the relation $\sim$ on  $R \times D$ by
    \begin{equation*}
        (a,b) \sim (c,d) \text{ if, and only if } ad-bc=0
    \end{equation*}
    Then $\sim$ is an equivalence relation on  $R \times D$.
\end{lemma}
\begin{proof}
    We have $ab-ab=0$ so that  $(a,b) \sim (a,b)$. Moreover, if $ad-bc=0$, then
     $bc-ad=0$ so that  $(a,b) \sim (c,d)$ implies $(c,d) \sim (a,b)$. Lastly,
     let $(a,b) \sim (c,d)$ and $(c,d) \sim (e,f)$. Then $ad-bc=0$ and
     $cf-ed=0$, so $af-eb=(ad-bc)f+d(cf-de)=0$ so that $(a,b) \sim (e,f)$.
\end{proof}

\begin{theorem}\label{1.5.2}
    Let $R$ a commutative ring, and  $D \subseteq R$ be nonempty with $0 \notin
    D$ such that $D$ contains no zero divisors of $R$ and that it is closed
    under multiplication. Then there exists a commutative ring $Q$ with identity
    such that  every element of  $D$ is a unit of $Q$.
\end{theorem}
\begin{proof}
    Define the equivalence relation $\sim$ on  $R \times D$ by
    \begin{equation*}
        (a,b) \sim (c,d) \text{ if, and only if } ad-bc=0
    \end{equation*}
    Label the equivalence classes of $\sim$ over $R \times D$ as
    $\frac{a}{b}=\{(c,d) \in R \times D : ad-bc=0\}$. Let
    \begin{equation*}
        Q=\faktor{R}{\sim}
    \end{equation*}
    The factor set of $\sim$ over $R \times D$ and define binary operations $+$
    and  $\cdot$ by
    \begin{align*}
        \frac{a}{b}+\frac{c}{d}     &=  \frac{ad+bc}{bd}    \\
        \frac{a}{b}\frac{c}{d}     &=  \frac{ac}{bd}    \\
    \end{align*}

    Suppose that $\frac{a}{b}=\frac{a'}{b'}$ and $\frac{c}{d}=\frac{c'}{d'}$.
    Then $ab'-a'b=0$ and  $cd'-c'd=0$. Then
    \begin{align*}
        (ad+bc)(b'd')       &=  adb'd'+bcb'd'  \\
                            &=  ab'dd'+cd'bb'   \\
                            &=  a'bdd'+c'dbb'   \\
                            &=  (a'd'c'd')bd    \\
    \end{align*}
    So that $+$ is well define. By similar reasoning,  $\cdot$ is also well
    defined.

    Now, let  $\frac{a}{b},\frac{c}{d} \in Q$. Then
    $\frac{a}{b}+\frac{c}{d}=\frac{ad+bc}{bd} \in Q$, as $ad+bc \in R$, and
    since  $b,d \in D$,  $bd \in D$. Moreover
    \begin{equation*}
        \frac{a}{b}+(\frac{c}{d}+\frac{e}{f})=\frac{a}{b}+\frac{cf+de}{df}
        =\frac{adf+bcf+bde}{bdf}
    \end{equation*}
    and
    \begin{equation*}
        (\frac{a}{b}+\frac{c}{d})+\frac{e}{f}=\frac{ad-bc}{bd}+\frac{e}{f}
        =\frac{adf+bcf+bde}{bdf}
    \end{equation*}
    so that $+$ is associative. Now, take  $c=0$ and  $d \in D$, and we have
    \begin{equation*}
        \frac{a}{b}+\frac{0}{d}=\frac{ad}{bd}=\frac{a}{b}
    \end{equation*}
    Since, $abd-abd=0$ making $\frac{ad}{bd}=\frac{a}{b}$. Similarly, take
    $c=-a$ and  $d \in D$ and we get
    \begin{equation*}
        \frac{a}{b}+\frac{-a}{d}=\frac{0}{b}
    \end{equation*}
    So $\frac{0}{d}$ is the identity, and $\frac{-a}{d}$ is the inverse of
    $\frac{a}{b}$. Lastly, since $R$ is commutative, this makes
    \begin{equation*}
        \frac{a}{b}+\frac{c}{d}=\frac{c}{d}+\frac{a}{b}
    \end{equation*}

    Now, notce that since $D$ is closed under multiplication, we have
    \begin{equation*}
        \frac{a}{b}\frac{c}{d}=\frac{ac}{bd} \in Q
    \end{equation*}
    Moreover,
    \begin{equation*}
        \frac{a}{b}(\frac{c}{d}\frac{e}{f})=\frac{a}{b}\frac{ce}{df}=\frac{ace}{bdf}
        =\frac{ac}{bd}\frac{e}{f}=(\frac{a}{b}\frac{c}{d})\frac{e}{f}
    \end{equation*}
    Additionally, since $R$ is commutative, we get
    \begin{equation*}
        \frac{a}{b}\frac{c}{d}=\frac{c}{d}\frac{a}{b}
    \end{equation*}
    Lastly, take $c=d$ and  $d \in D$. Then
    \begin{equation*}
        \frac{a}{b}\frac{d}{d}=\frac{ad}{bd}=\frac{a}{b}
    \end{equation*}
    This makes $Q$ a commutative righ with identity. Moreover, every element of
     $D$ is a unit in $Q$ Moreover $R \subseteq Q$ by taking $r \xrightarrow{}
     \frac{r}{d}$ for some $d \in D$.
\end{proof}
\begin{corollary}
    The ring $Q$ satisfies the following
    \begin{enumerate}
        \item[(1)] $Q$ contains a copy of $R$ as a subring, and every element of
            $Q$ is of the form $r\inv{d}$. Moreover, if $D=\com{R}{\{0\}}$, then
            $Q$ is a field.

        \item[(2)] $Q$ is unique and the smallest ring containing a copy of  $R$
            for which every element of  $D$ is a unit.
    \end{enumerate}
\end{corollary}
\begin{proof}
    Imbed $R$ into $Q$ first. Define the map  $\iota:R \xrightarrow{} Q$ by
    taking $r \xrightarrow{} \frac{rd}{d}$ where $d \in D$. Notice that
    $\frac{rd}{d}=\frac{re}{e}$, so that $\iota$ is well defined. Now, since
    $\frac{d}{d}$ is the identity of $Q$, we have
    \begin{equation*}
        \frac{rsd}{d}=\frac{arsd}{dd}=\frac{rd}{d}\frac{sd}{d}
    \end{equation*}
    So that $\iota$ is a ring homomorphism. Now, subce no  $d \in D$ is a zero
    divisor, we have that $\ker{\iota}=(0)$ This makes  $\iota$ 1--1. Therefore,
    by the first isomorphism theorem, we get
    \begin{equation*}
        \iota(R) \simeq R
    \end{equation*}
    Since $\iota(R) \subseteq Q$, $\iota$ is the required imbedding.

    Now, if $D=\com{R}{\{0\}}$, this makes every $r \in R$, nonzero into a unit
    of  $Q$. Then $Q$ has no zero divisors making it an integral domain, thus
    $Q$ is a field.

    Lastly, let $\phi:R \xrightarrow{} S$ be a 1--1 ring homomorphsim such that
    every $\phi(d)$ is a unit in $S$, where $S$ is a commutative ring with unit
    and $d \in D$. Define the map  $\Phi:Q \xrightarrow{} S$ by taking $r\inv{d}
    \xrightarrow{} \phi(r\inv{d})$. Then $\Phi$ is a 1--1 ring homomorphism, so
    that by the first isomorphism theorem,
     \begin{equation*}
         \Phi(Q) \simeq S
     \end{equation*}
     This makes $Q$ unique.
\end{proof}

\begin{definition}
    Let $R$ a commutative ring, and  $D \subseteq R$ be nonempty with $0 \notin
    D$ such that $D$ contains no zero divisors of $R$ and that it is closed
    under multiplication. Define the equivalence relation $\sim$ on  $R \times D$
    by
    \begin{equation*}
        (a,b) \sim (c,d) \text{ if, and only if } ad-bc=0
    \end{equation*}
    and let
    \begin{equation*}
        Q=\faktor{R \times D}{\sim}
    \end{equation*}
    Then we call the commutative ring $Q$, with identity  $1=\frac{d}{d}$, the
    \textbf{ring of fractions} of $R$. If $D=\com{R}{\{0\}}$ and $R$ is an
    integral domain, we call  $Q$ the \textbf{field of fractions}.
\end{definition}

\begin{lemma}\label{1.5.3}
    If $R$ is an integral domain, and  $Q$ its field of fractions, and  $F$ is a
    field containing  $R' \simeq R$, then the subfield of $F$ generated by $R$
    is isomorphic to $Q$.
\end{lemma}
\begin{proof}
    Let $\phi:R \xrightarrow{} R'$ the ring isomorphism between $R$ and  $R'$.
    Then the  $\phi:R \xrightarrow{} F$ is 1--1. Define then the map $\Phi:Q
    \xrightarrow{} F$ by $r\inv{d} \xrightarrow{} \phi(r\inv{d})$. Then $\Phi$
    is 1--1 and by the first isomorphism theorem, $\Phi(Q) \simeq Q$. Moreover,
    $\Phi(R)=\phi(R)=R' \subseteq \Phi(Q)$. Now, for all $r,s \in R$, we have
    $\phi(r\inv{s}) \in \Phi(Q)$ and since every element of $Q$ is of the form
    $r\inv{s}$, any subfield containiing $R'$ contains  $\Phi(Q)$.
\end{proof}

\begin{example}\label{1.18}
    \begin{enumerate}
        \item[(1)] The field of fractions of $\Z$ is $\Q$. Indeed, the
            construction of the ring of fractions of a commutative ring with
            identity is inspired by constructing $\Q$ from $\Z$.

        \item[(2)] The field of fractions of $\Q$ is $\Q$ itself. In general if
            $F$ is a field, it is its own field of fractions.

        \item [(3)] The field of fractions of $\Z[\sqrt{D}]$ is $\Q[\sqrt{D}]$.

        \item[(4)] $2\Z$ as a subring has no zero divisors, so the field of
            fractions of $2\Z$ is also $\Q$.
    \end{enumerate}
\end{example}
