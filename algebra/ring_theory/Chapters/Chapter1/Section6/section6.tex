\section{Sun Tzu's Theorem.}

\begin{definition}
    Let $\{R_\alpha\}$ a collection of commutative rings with identity. We
    define the \textbf{direct product} of $\{R_\alpha\}$ to be the direct product
    of $\{R_\alpha\}$ as a group, nmade into a ring by the operation $(r_\alpha),
    (s_\alpha) \xrightarrow{} (r_\alpha s_\alpha)$. We write $R=R_1 \times R_2
    \times \dots$ when $\{R_\alpha\}$ is a countable collection.
\end{definition}

\begin{definition}
    We call the ideals $A,B \subseteq R$ of a ring $R$ \textbf{comaximal} if
    $A+B=R$.
\end{definition}

\begin{example}\label{1.19}
    If $(m,n)=1$, then the ideals $n\Z$  $m\Z$ with  $m\Z+n\Z=\Z$ are comaximal.
     $m\Z+n\Z=\Z$ is the set of all diophantine equations of the form
     \begin{equation*}
         mx+ny=1
     \end{equation*}
\end{example}

\begin{theorem}[Sun-Tzu's Theorem]\label{1.6.1}
    LEt $A_1, \dots, A_k$ be ideals in a commutative ring $R$ with identity.
    Then the map
    \begin{align*}
        R   &   \xrightarrow{} \faktor{R}{A_1} \times \dots \times
        \faktor{R}{A_k} \\
        r   &   \xrightarrow{} (r+A_1, \dots, r+A_k)    \\
    \end{align*}
    is a ring homomorphism with kernel
    \begin{equation*}
        K=\bigcap_{i=1}^k{A_i}
    \end{equation*}
    Moreover if for all $1 \leq i,j \leq k$ with  $i \neq j$,  $A_i$ and  $A_j$
    are comaximal, then this map is onto with  $\bigcap{A_i}=\prod{A_i}$ so that
    \begin{equation*}
        \faktor{R}{A_1 \dots A_k} \simeq \faktor{R}{A_1} \times \dots \times
        \faktor{R}{A_k}
    \end{equation*}
\end{theorem}
\begin{proof}
    Let $k=2$, and  $A_1=A$ and $A_2=B$. Consider the map
    \begin{align*}
        \phi:R  &   \xrightarrow{} R  \\
        r   &   \xrightarrow{}  (r+A,r+B)   \\
    \end{align*}
    Then $rs \xrightarrow{} (rs+A,rs+B)=(r+A,r+B)(s+A,s+B)$ so that $\phi$ is a
    ring homomorphism. Now, let  $r \in \ker{\phi}$, then $(r+A,r+B)=(A,B)$ so
    that $r \in A \cap B$, conversly if  $r \in A \cap B$ then  $r+A=A$ and
    $r+B=B$ so that  $r \in \ker{\phi}$. Thereofore
    \begin{equation*}
        \ker{\phi}=A \cap B
    \end{equation*}

    Now, suppose that $A$ and  $B$ are comaximal, that is, $A+B=(1)$. Then there
    is an  $x \in A$, and a  $y \in B$ such that  $x+y=1$. Then $\phi(x)=(0,1)$
    and $\phi(y)=(1,0)$ and $x=1-y \in 1+B$. Now, take  $r+A$,  $s+B$, then
    \begin{equation*}
        \phi(rx+sy)=\phi(r)\phi(x)+\phi(s)+\phi(y)=(r+A,r+B)(0,1)+(s+A,s+B)(1,0)
        =(r+A,s+B)
    \end{equation*}
     this makes $\phi$ onto, moreover notice that  $AB \subseteq A \cap B$, and
     if  $A+B=(1)=R$, then for every $x \in AB$,  $c=c \cdot 1=cx+cy \in AB$ so
     that  $A \cap B=AB$.

     Now, by induction on $k \geq 2$, takine  $A=A_1$ and $B=A_2 \dots A_k$ by
     repeating the above argument, we get the result.
\end{proof}
\begin{corollary}
    Let $n=p_1^{a_1} \dots p_k^{a_k} \in \Z^+$ be the prime factorization of
    $n$, where  $p_1 \neq \dots \neq p_k$. Then
    \begin{equation*}
        \faktor{\Z}{n\Z} \simeq U(\faktor{\Z}{p_1^{a_1}\Z}) \times \dots \times
        U(\faktor{\Z}{p_1^{a_1}\Z})
    \end{equation*}
\end{corollary}

\begin{remark}
    Sun-Tzu's theorem is most commonly know as the Chines Remainder theorem,
    however it is the belief of the author that the name of the theorem should
    credit the author whenever possible. Also note that the Sun-Tzu of this
    theorem \textit{is not} the same Sun-Tzu who penned \textit{The Art of War}.
\end{remark}
