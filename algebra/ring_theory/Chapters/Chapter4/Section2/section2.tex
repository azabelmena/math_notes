\section{Module Homomorphisms and Factor Modules}

We make the conventions that when we say ``ring'', we mean a ring with identity,
and when we say ``module'', we mean unital modules.

\begin{definition}
    Let $R$ be a ring, and $M$ and  $N$ modules. We call a map $\phi:M
    \xrightarrow{} N$ a  \textbf{module homomorphism} over $R$ (or an
    \textbf{$R$-module homomorphism}) if for all $x,y \in M$, and  $r \in R$
    \begin{enumerate}
        \item[(1)] $\phi(x+y)=\phi(x)+\phi(y)$.

        \item[(2)] $\phi(rx)=r\phi(x)$.
    \end{enumerate}
    We call $\phi$ a \textbf{module isomorphism} (or an \textbf{$R$-module
    isomorphism}) if $\phi$ is 1--1 and takes $M$ onto $N$; and we call  $M$ and
     $N$  \textbf{isomorphic} and write $M \simeq N$.
\end{definition}

\begin{definition}
    Let $R$ be a ring,  $M$ and  $N$  $R$-modules, and let $\phi:M
    \xrightarrow{} N$ be a module homomorphism over $R$. We define the
    \textbf{kernel} of $\phi$ to be the set
    \begin{equation*}
        \ker{\phi}=\{m \in M : \phi(m)=0\}
    \end{equation*}
    We denote the set $\Hom_R{(M,N)}$ to be the set of all module homomorphisms
    over $R$ from $M$ into $N$.
\end{definition}

\begin{lemma}\label{4.2.1}
    Let $R$ be a ring, and  $M$, and $N$ module s. let $\phi:M \xrightarrow{} N$
    be a module homomorphism of $R$. Then the following are true
    \begin{enumerate}
        \item[(1)] $\ker{\phi}$ is a submodule of $M$.

        \item[(2)] $\phi(M)$ is a submodule of $N$.
    \end{enumerate}
\end{lemma}
\begin{proof}
    Notice that if $\phi:M \xrightarrow{} N$ is a module homomorphism, then it
    is also a group homomorphism, since $M$ and $N$ are both groups under their
    respective additions. This makes $\ker{\phi}$ and $\phi(M)$ into subgroups
    of $M$ and  $N$ respectively (this also implies that both $\ker{\phi}$ and
    $\phi(M)$ are nonempty).

    Now, let $r,s \in R$, and  $x,y \in \ker{\phi}$. Then we get
    $\phi(rx+sy)=r\phi(x)+s\phi(y)=r0+s0=0$, so that $rx+sy \in \ker{\phi}$.
    Similarly, if $\phi(x), \phi(y) \in \phi(M)$, then
    $r\phi(x)+s\phi(y)=\phi(rx)+\phi(sy)=\phi(rx+sy) \in \phi(M)$. By the
    submodule criterion for unital modules, we are done.
\end{proof}

\begin{example}\label{example_4.4}
    \begin{enumerate}
        \item[(1)] Module homomorphisms need not be ring homomorphisms. Consider
            the ring $\Z$ as a module over itself. Then the map  $x
            \xrightarrow{} 2x$ is a $\Z$-module homomorphism, but not a ring
            homomorphism, since this map takes  $1 \xrightarrow{} 2$, and $2
            \neq 1$ in  $\Z$. Likewise, ring homomorphisms need not be module
            homomorphisms. Let $F$ be a field, and consider the map $f(x)
            \xrightarrow{} f(x^2)$, where $f(x) \in F[x]$. This map is a ring
            homomorphism, but it is not an  $F[x]$-module homomorphism when
            considering $F[x]$ as a module over itself.

        \item[(2)] Let $R$ be a ring, and  $n \in \Z^+$ Consider the \textbf{$i$-th
            projection map} $\pi_i:R^n \xrightarrow{} R$ defined by
            \begin{equation*}
                \pi_i:(x_1, \dots, x_n) \xrightarrow{} x_i
            \end{equation*}
            Then $\pi_i$ is an $R$-module homomorphism. Moreover, we have
            $\pi_i(R^n)=R$ (so that $\pi_i$ is onto) and $\ker{\p_i}=\{(x_1,
            \dots, x_n) : x_i=0\}$.

            \item[(3)] For any field $F$, and any vector spaces $V$ and $W$ over
                 $F$, the usual linear transofrmations  $T:V \xrightarrow{} W$
                 are $F$-module homomorphisms by definition.

             \item[(4)] $\Z$-module homomorphisms are Abelian group
                 homomorphisms. Recall that  module homomorphisms are group
                 homomorphisms, and that $\Z$ is an Abelian group.

             \item[(5)] Let $R$ be a ring, and  $I$ an ideal of  $R$. Suppose
                 that  $M$ and  $N$ are  $R$-modules which are annihilated by
                 $I$. Then any $R$-module homomorphism of $M \xrightarrow{} N$
                 is also an $(\faktor{R}{I})$-module homomorphism.
    \end{enumerate}
\end{example}

\begin{lemma}\label{4.2.2}
    Let $R$ be a ring, and let $M$, $N$, and $L$ be modules over $R$. Then the
    following are true
    \begin{enumerate}
        \item[(1)] A map $\phi:M \xrightarrow{} N$ is a module homomorphism over
            $R$ if, and only if $\phi(rx+sy)=r\phi(x)+s\phi(y)$ for any $x,y
            \in M$ and  $r,s \in R$.

        \item[(2)] If $\phi, \psi \in \Hom_R{(M,N)}$, then $\phi+\psi \in
            \Hom_R{(M,N)}$; where $+$ is the addition of functions. Moreover, if
             $R$ is commutative, defining the action of  $R$ on  $\Hom_R{(M,N)}$
             by $(r\phi)(x)=r(\phi(x))$ makes $r\phi \in \Hom_R{(M,N)}$, and
             makes $\Hom_R{(M,N)}$ into a module over $R$.

         \item[(3)] If $\phi \in \Hom_R{(M,N)}$ and $\psi \in \Hom_R{(N,L)}$,
             then $\psi \circ \phi \in \Hom_R{(M,L)}$ where $\circ$ is the usual
             function composition.

         \item[(4)] $\Hom_R{(M,M)}$ is a ring with identity under the operations
             of function addition and function composition. Moreover, if $R$ is
             commutative, then  $\Hom_R{(M,M)}$ is an $R$-algebra.
    \end{enumerate}
\end{lemma}
\begin{proof}
\end{proof}
