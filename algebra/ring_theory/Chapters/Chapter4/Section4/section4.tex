\section{Tensor Products of Modules}

\begin{definition}
    Let $S$ be a ring, and $R$ a subring of $S$. Let $f:S
    \xrightarrow{} R$ a ring homomorphism with $f(1_S)=1_R$ and let $N$ be an
    $R$-module, with the scalar multiplication defined by $rn=f(r)n$. We call
    $N$ a \textbf{restriction of scalars} from $S$ to $R$. We call $S$ an
    \textbf{extension} of $R$.
\end{definition}

\begin{example}\label{example_4.8}
    \begin{enumerate}
        \item[(1)] Let $S$ a  ring and $R$ a subring of $S$. We can
            restrict an $S$-module $N$ to be an $R$-module, but the reverse is
            not in general true. Consider $\Z$ as a $\Z$-module, then $\Z$ cannot
            be a $\Q$-module, for if it is, then the element $\frac{1}{2}1=z$ is
            an element of $\Z$ for which $z+z=1$, which cannot happen. Notice
            however that $\Z$ can be imbedded into a $\Q$-module, namely $\Q$ itself.

        \item[(2)] Consider al possible $\Z$-module homomorphisms of
            $\faktor{\Z}{2\Z}$ into a $\Q$-module $V$. Since $\Q$ is a field $V$
            is a vector space over $\Q$, and every element has infinite order.
            Since the elements of $\faktor{\Z}{2\Z}$ are of finite order, then
            $x \xrightarrow{} 0$ for all $x \in \faktor{\Z}{2\Z}$. This shows
            that $\faktor{\Z}{2\Z}$ cannot be imbedded into a $\Q$-module.
    \end{enumerate}
\end{example}

\begin{definition}
    Let $R$ and $S$ be rings, and let $N$ be an $R$-module. Consider the map $S
    \times N \xrightarrow{} N$ defined by $(s,n) \xrightarrow{} sn$. Consider
    $H$ the subgroup generated by all elements of the form
    $(s_1+s_1,n)-(s_1,n)-(s_2,n)$, $(s,n_1+n_2)-(s,n_1)-(s,n_2)$, and
    $(sr,n)-(s,rn)$ for all $s,s_1,s_2 \in S$, $r \in R$, and $n,n_1,n_2 \in N$.
    We define the \textbf{tensor product} of $S$ together with  $N$ over $R$ to
    be the quotient group $\faktor{N}{H}$. We denote this tensor product by $S
    \otimes_R N$. We denote the elements of $S \otimes_R N$ as $s \otimes n$ and
    call them \textbf{simple tensors}. We call sums of simple tensors
    \textbf{tensors}.
\end{definition}

\begin{lemma}\label{4.4.1}
    Let $R$ and $S$ be rings, and let $N$ be an $R$-module. Then the following
    relations hold on the tensor product $S \otimes_R N$
    \begin{enumerate}
        \item[(1)] $(s_1+s_2) \otimes n=s_1 \otimes n+s_2 \otimes n$.

        \item[(2)] $s \otimes (n_1+n_2)=s \otimes n_1+s \otimes n_2$.

        \item[(3)] $sr \otimes n=s \otimes rn$.
    \end{enumerate}
\end{lemma}
\begin{proof}
    Since $S \otimes_R N$ is the quotient group of $N$ by $H$, where $H$ the
    subgroup generated by all elements of the form $(s_1+s_1,n)-(s_1,n)-(s_2,n)$,
    $(s,n_1+n_2)-(s,n_1)-(s,n_2)$, and $(sr,n)-(s,rn)$; that is: $H=\langle
    (s_1+s_1,n)-(s_1,n)-(s_2,n), (s,n_1+n_2)-(s,n_1)-(s,n_2), (sr,n)-(s,rn)
    \rangle$. Then the tensors $s \otimes n$ are just cosets of $H$ in $N$, and
    the relations defining $H$ give us the relations on tensors.
\end{proof}

\begin{theorem}\label{4.4.2}
    Let $S$ be a ring, and $R$ a subring of $S$. Let $N$ be an
    $R$-module, then the tensor product $S \otimes_R N$ is in $S$-module under
    the action
    \begin{equation*}
        s\sum{s_i \otimes n_i}=\sum{(ss_i) \otimes n_i}
    \end{equation*}
    where the sums are finite.
\end{theorem}
\begin{proof}
    Let $s' \in S$, then
    \begin{align*}
        (s'(s_1+s_2),n)-(s's_1,n)-(s's_2,n) &= (s's_1+s's_2,n)-(s's_1,n)-(s's_2,n) \\
        (s's,n_1n_2)-(s's,n_1)-(s's,n_2)    \\
        (s'(sr),n)-(s's,rn) &=  ((s's)r,n)-(s's,rn) \\
    \end{align*}
    each belong to the generating set
    of $H=\langle (s_1+s_1,n)-(s_1,n)-(s_2,n), (s,n_1+n_2)-(s,n_1)-(s,n_2),
    (sr,n)-(s,rn) \rangle$. Since elements of $H$ are sums of elements in $H$,
    for every $\sum{(s_i,n_i)} \in H$, we get $\sum{(s's_i,n_i)} \in H$

    Now, suppose that $\sum{s_i \otimes n_i}=\sum{s'_i \otimes n'_i}$ are
    representatives of the same tensor in $S \otimes_R N$. Then
    $\sum{(s_i,n_i)}-\sum{(s'_i,n'_i)} \in H$ so that $\sum{ss_i \otimes
    n}=\sum{ss'_i \otimes n} \in S \otimes_R N$. This makes the action well
    defined.

    Finally, observe that for every tensor $s_i \otimes n_i \in S \otimes_R N$,
    that the following relation holds, by lemma \ref{4.4.1}
    \begin{equation*}
        (s+s')(s_i \otimes n_i)=s(s_i \otimes n_i)+s'(s_i \otimes n_i)
    \end{equation*}
    The other relations for modules also hold by lemma \ref{4.4.1}.
\end{proof}

\begin{definition}
    Let $S$ be a rings and $R$ a subring of $S$, and let $N$ be an
    $R$-module. We call $S \otimes_R N$, as a module, the \textbf{left
    $S$-module obtained by an extension of scalars} from $N$.
\end{definition}

\begin{theorem}\label{4.4.3}
    Let $R$ a subring of a ring $S$, and let $N$ be a left $R$-module. Let $\i:N
    \xrightarrow{} S \otimes_R N$ be the $R$-module homomorphism defined by $\i:
    \xrightarrow{} 1 \otimes n$. If $L$ is any left $S$-module, and $\phi:N
    \xrightarrow{} L$ is an $R$-module homomorphism, then there exists a unique
    $S$-module homomorphism $\Phi:S \otimes_R N \xrightarrow{} L$ for which
    $\phi=\Phi \circ \i$. That is, the following diagram commutes.
    \[\begin{tikzcd}
        N & {S \otimes_R N} \\
        & L
        \arrow["\iota", from=1-1, to=1-2]
        \arrow["\Phi", from=1-2, to=2-2]
        \arrow["\phi"', from=1-1, to=2-2]
    \end{tikzcd}\]
    Conversely, if $\Phi:S \otimes_R N \xrightarrow{} L$ is an $S$-module
    homomorphism, then the map $\phi=\Phi \circ \i$ is an $R$-module
    homomorphism.
\end{theorem}
\begin{proof}
    Let $\phi:N \xrightarrow{} L$ an $R$-module homomorphism. By the uiversal
    property of free modules, there exists a $\Z$-module homomorphism from the
    free $\Z$-module $F(S \times N)$ to  $L$ defined by $(s,n) \xrightarrow{}
    s\phi(n)$. Since $\phi$ is an $R$-module homomorphism, the generators of $S
    \otimes_R N$, as a quotient group, map to $0$ in $L$. Therefore, there
    exists a well define $\Z$-module homomorphism $\Phi:S \otimes_R N
    \xrightarrow{} L$ taking $s \otimes n \xrightarrow{} s\phi(n)$. Then for
    simple tensors, we get $s'\Phi(s \otimes
    n)=s'(s\phi(n))=(s's)\phi(n)=\Phi(s'(s \otimes n))$ for all $s' \in S$.
    Since $\Phi$ is additive, we get an $S$-module homomorphism.

    Now, $S \otimes_R N$, as an $S$-module, is generated by all $1 \otimes n$,
    and any $S$-module homomorphism is uniquely determined by its value on these
    elements. Now, observe that $\Phi(1 \otimes n)=1\phi(n)=\phi(n)$, so
    $\phi=\Phi \circ \i$ and $\Phi$ is uniquely determined by $\phi$.

    Now, if we have the map $\phi=\Phi \circ \i$, where $\Phi$ is defined as
    above, and $\i:n \xrightarrow{} 1 \otimes n$, then since $\Phi$ is an
    $S$-module homomorphsm, and $\i$ is an $R$-module homomorphism, then $\phi$
    must be an $R$-module homomorphism.
\end{proof}
\begin{corollary}
    $\faktor{N}{\ker{\i}}$ is the largest unique quotient of $N$ that can be
    imbedded into any $S$-module. In particular, $N$ can be imbedded as an
    $R$-module into some left $S$-module if, and only if $\i$ is 1--1.
\end{corollary}
\begin{proof}
    We have that $\faktor{N}{\ker{\i}}$ is mapped into $S \otimes_R N$
    injectively. Now, let $\phi$ be an 1--1 $R$-module taking
    $\faktor{N}{\ker{\i}} \xrightarrow{} L$. Then $\ker{\i} \xrightarrow{} (0)$
    by $\phi$ so that $\ker{\i} \leq \ker{\phi}$. This makes
    $\faktor{N}{\ker{\phi}}$ a quotient of $\faktor{N}{\ker{\i}}$.
\end{proof}

\begin{example}\label{example_4.9}
    \begin{enumerate}
        \item[(1)] Let $N$ be an $R$-module, and take $\phi:N \xrightarrow{} N,
            n \xrightarrow{} n$. Then $R \otimes_R N \simeq R$ via this map. In
            particular, if $A$ is any Abelian group, then $\Z \otimes_\Z A
            \simeq A$.

        \item[(2)] Let $A$ be a finite Abelian group of order $n$. THen $\Q
            \otimes_\Z A=(0)$. Observe that $1 \otimes 0=1 \otimes (0+0)=1
            \otimes 0+1 \otimes 0$, so that $1 \otimes 0=0$. Then for any $q
            \otimes a \in \Q \otimes_\Z A$, we have $q \otimes a=(\frac{q}{n}n)
            \otimes a=\frac{q}{n} \otimes na=\frac{q}{n} \otimes 0=\frac{q}{n}(1
            \otimes 0)=\frac{q}{n}0=0$; since $na=0$. Moreover, the map $\i:A
            \xrightarrow{} \Q \otimes_\Z A$ is the zero map, so that if $A \neq
            \langle 0 \rangle$, any map $\phi:\Q \xrightarrow{} A$ is also the
            zeromap.

        \item[(3)] Let $R$ be a ring, and consider the free module $R^n$, of
            rank $n$. Then $S \otimes_R R^n \simeq S^n$, the free module over
            $S$ of rank $n$. Then $\Q \otimes_\Z \Z^n \simeq \Q^n$. Moreover,
            observe that $\Z^n$ is a subgroup of $\Q^n$.

        \item[(4)] Let $K$ be a field, and $F$ a subfield of $K$. Let $V$ an
            $n$-dimensional vector space over $F$. Then $K \otimes_F V \simeq
            K^n$. We recall that $V \simeq F^n$.

        \item[(5)] Let $R$ be a commutative ring with identity, and $G$ a finite
            group. Let $H \leq G$. For any $RH$-module $N$, we define the
            \textbf{induced module} $RG \otimes_{RH} N$ obtained as an extension
            of scalars from $N$ as an $RG$-module.
    \end{enumerate}
\end{example}

\begin{definition}
    Let $R$ be a ring, and $N$ a left $R$-module, and $M$ a right $R$-module.
    THe quotient of the free $\Z$-module on $M \times N$ by the subgroup
    \begin{equation*}
        H=\langle (m_1+m_1,n)-(m_1,n)-(m_2,n), (m,n_1+n_2)-(m,n_1)-(m,n_2), (mr,n)-(m,rn) \rangle
    \end{equation*}
    is called the \textbf{tensor product} of $M \times N$ over $R$, and denoted
    $M \otimes_R N$. We call the elements $m \times n$ \textbf{simple tensors},
    and we call sums of simple tensors \textbf{tensors}.
\end{definition}

\begin{lemma}\label{4.4.4}
    Let $R$ be a ring, and $N$ and $M$ left and right $R$-modules, respectively.
    Then the following relations hold.
    \begin{enumerate}
        \item[(1)] $(m_1+m_2) \otimes n=m_1 \otimes n+m_2 \otimes n$.

        \item[(2)] $m \otimes (n_1+n_2)=m \otimes n_1+m \otimes n_2$.

        \item[(3)] $mr \otimes n=m \otimes rn$.
    \end{enumerate}
\end{lemma}
\begin{proof}
    The proof is identical to that of lemma \ref{4.4.1}.
\end{proof}

\begin{definition}
    Let $R$ be a ring, and let $M$ and $N$ be right and left $R$-modules,
    respectively, and let $L$ be an Abelian group. We call the map $\phi:M
    \times N \xrightarrow{} L$ \textbf{$R$-balanced} or \textbf{middle linear}
    with resepct to $R$ if for all $m_1,m_2 \in M$, and $n_1,n_2 \in N$, and $r
    \in R$
    \begin{enumerate}
        \item[(1)] $\phi(m_1+m_2,n)=\phi(m_1,n)+\phi(m_2,n)$.

        \item[(2)] $\phi(m,n+1+n_2)=\phi(m,n_1)+\phi(m,n_2)$

        \item[(3)] $\phi(mr,n)=\phi(m,rn)$.
    \end{enumerate}
\end{definition}

\begin{lemma}\label{4.4.5}
    Let $R$ be a ring, and $M$ and $N$ right, and left $R$-modules respectively.
    The map $\i:M \times N \xrightarrow{} M \otimes_R N$ defined by $\i:(m,n)
    \xrightarrow{} m \otimes n$ is $R$-balanced.
\end{lemma}
\begin{proof}
    This is by definition of the tensor product.
\end{proof}

\begin{theorem}\label{4.4.6}
    Let $R$ be a ring with identity, and $M$ and $N$ right and left modules,
    respectively. Let $\i:M \times N \xrightarrow{} M \otimes_R N$ the
    $R$-balanced map defined by $\i:(m,n) \xrightarrow{} m \otimes n$. Then the
    following are true.
    \begin{enumerate}
        \item[(1)] If $\Phi:M \otimes_R N \xrightarrow{} L$ is a group
            homomorphsm and $L$ an Abelian group, then the map $\phi=\Phi \circ
            \i$ is $R$-balanced from $M \times N \xrightarrow{} L$.

        \item[(2)] If $L$ is an Abelian group, and $\phi:M \times N
            \xrightarrow{} L$ is an $R$-balanced map, then there exists a unique
            group homomorphism $\Phi:M \otimes_R N \xrightarrow{} L$ for wich
            $\phi=\Phi \circ \i$.
    \end{enumerate}
\end{theorem}
\begin{proof}
    We first have that if $\phi(m,n)=\Phi(m \otimes n)$, then
    $\phi(m_1+m_2,n)=\Phi((m_1+m_2) \otimes n)=\Phi(m_1 \otimes n)+\Phi(m_2
    \otimes n)=\phi(m_1,n)+\phi(m_2,n)$. Similarly, the rest of the conditions
    for $R$-balanced maps hold for $\phi$, making it $R$ balanced.

    Conversely by the universal propery, $\phi$ defines a unique $\Z$-module
    homomorphism $\tilde{\phi}:F(M \times N) \xrightarrow{} L$ for wich
    $\tilde{\phi}(m,n)=\phi(m,n) \in L$. Now, if $\phi$ is $R$-balanced, then
    $\tilde{\phi}$ takes $(m,n) \xrightarrow{} 0$, that is, we get
    $\tilde{\phi}((mr,n)-(m,rn))=0$. Then the subgroup $H$ defined by
    \begin{equation*}
        H=\langle (m_1+m_1,n)-(m_1,n)-(m_2,n), (m,n_1+n_2)-(m,n_1)-(m,n_2), (mr,n)-(m,rn) \rangle
    \end{equation*}
    is a subgroup of $\tilde{\phi}(F(M \times N))$. Then $\tilde{\phi}$ indues a
    homomorphism $\Phi:M \otimes_R N \xrightarrow{} L$. By definition we have
    $\Phi(m \otimes n)=\tilde{\phi}(m,n)=\phi(m,n)$.
\end{proof}
\begin{corollary}
    Let $D$ be an Abelian group, and $\i':M \times N \xrightarrow{} D$ such that
    \begin{enumerate}
        \item[(1)] $\i'(M \times N)$ generates $D$ as a group.

        \item[(2)] Every $R$-balanced map on $M \times N$ factors through $\i'$.
    \end{enumerate}
    Then there exists an isomorphism $f:M \otimes_R N \xrightarrow{} D$ with
    $\i'=f \circ \i$.
\end{corollary}
\begin{proof}
    Since $\i'$ is $R$-balanced, there is a unique group homomorphism $f:M
    \otimes_R N \xrightarrow{} D$ such that $\i'=f \circ \i$. Then $\i'=(f(m
    \otimes n))$. Since we also have that $D=\langle \i'(M \times N) \rangle$,
    we get that $f$ is onto. Since $\i'$ is $R$-balanced, there exists a unique
    group homomorphism $g:D \xrightarrow{} M \otimes_R N$ with $\i=g \circ \i'$.
    Then $m \otimes n=(g \circ f)(m \otimes n)$, and since every $m \otimes n$
    generates $M \otimes_R N$, we get $g \circ f=i_D$ (the identity on $D$), so
    that $f$ is 1--1.
\end{proof}

\begin{definition}
    Let $R$ and $S$ be rings with identity. An Abelian group $M$ is caled an
    \textbf{$(S,R)$-bimodule}  if $M$ is a left $S$-module, a right $R$-module,
    and $s(mr)=(sm)r$ for all $s \in S$, $r \in R$, and $m \in M$.
\end{definition}

\begin{example}\label{example_4.10}
    \begin{enumerate}
        \item[(1)] Let $S$ be a ring with identity and $R$ a subring of $S$ with
            $1_R=1_S$. Then $S$ is an $(S,R)$-bimodule by associativity in $S$.
            Now, if $f:R \xrightarrow{} S$ is a ring homomorphism with
            $f(1_R)=1_S$, then $S$ is a right $R$-module under the action
            $sr=sf(r)$.

        \item[(2)] Let $I$ be an ideal of a ring $R$. The quotient ring
            $\faktor{R}{I}$ is an $(\faktor{R}{I},R)$-bimodule.

        \item[(3)] Let $R$ be a commutative ring (not necessarily with
            identity), then any $R$-module $M$ is an $(R,R)$-bimodule, since
            $rm=mr$ for all $m \in M$ and $r \in R$. We call $M$ the
            \textbf{standard $R$-module structure} over $R$.

        \item[(4)] Let $R$ and $S$ be rings where $R \subseteq Z(S)$. Let $M$ be
            a left $S$-module, then $R$ is a commutative ring, and we have for
            all $s \in S$, $r \in R$, and $m \in M$ that
            $(sm)r=r(sm)=(sr)m=s(rm)=s(mr)$. So $M$ can be considered as a right
            $R$-module. This makes $M$ an $(S,R)$-bimodule under the right
            action of $R$.
    \end{enumerate}
\end{example}

\begin{definition}
    Let $R$ be a commutative ring with identity, and let $M$ and $N$, and $L$ be
    left $R$-modules. We call a map $\phi:M \times N \xrightarrow{} L$
    \textbf{$R$-bilinear} if it is $R$-linear in each factor. That is,
    \begin{align*}
        \phi(r_1m_1+r_2m_2,n)   &=  r_1\phi(m_1,n)+r_2\phi(m_2,n)   \\
        \phi(m,r_1n_1+r_2n_2)   &=  r_1\phi(m,n_1)+r_2\phi(m,n_2)   \\
    \end{align*}
    for all $r_1,r_2 \in R$, $m_1,m_2,m \in M$, and $n_1,n_2,n \in N$.
\end{definition}

\begin{lemma}\label{4.4.7}
    Let $R$ be a commutative ring, and let $M$ and $N$ be left $R$-modules. Let
    $M$ have the standard $R$-module structure. Then $M \otimes_R N$ is a left
    $R$-module with
    \begin{equation*}
        r(m \otimes n)=rm \otimes n=m \otimes (rn)
    \end{equation*}
    Moreover, the map $\i:(m,n) \xrightarrow{} m \otimes n$ is $R$-bilinear.
\end{lemma}
\begin{proof}
    Since $M$ and $N$ are left $R$-modules, and since $M$ has an
    $(R,R)$-bimodule structure, the tensor product $M \otimes_R N$ is a left
    $R$-module. Moreover, notice that $\i:(m,n) \xrightarrow{} m \otimes n$ is
    additive in each factor since $r(m \otimes n)=rm \otimes n=mr \otimes n=m
    \otimes rn$. This makes $\i$ a bilinear map.
\end{proof}
\begin{corollary}
    There exists a 1--1 correspondence of $R$-bilinear maps from $M \times N
    \xrightarrow{} L$ onto $R$-module homomorphisms from $M \otimes_R N
    \xrightarrow{} L$ given by the following diagram
    \[\begin{tikzcd}
        {M \times N} & {M \otimes_R N} \\
        & L
        \arrow["\iota", from=1-1, to=1-2]
        \arrow["\Phi", from=1-2, to=2-2]
        \arrow["\phi"', from=1-1, to=2-2]
    \end{tikzcd}\]
\end{corollary}
\begin{proof}
    If $\phi:M \times N \xrightarrow{} L$ is bilinear, then it is $R$-balanced.
    So the corresponding map $\Phi:M \otimes_R N \xrightarrow{} L$ is a group
    homomorphism. Notice now that
    \begin{equation*}
        \Phi((rm) \otimes n)=\phi(rm,n)=r\phi(m,n)=r\Phi(m \otimes n)
    \end{equation*}
    Since $\Phi$ is additive, this makes $\Phi$ a ring homomorphism. Now, if
    $\Phi$ is an $R$-module homomorphism with $\phi=\Phi \circ \i$, since $\i$
    is a bilenar map, it is easy to check that $\phi$ must also be bilinear.
\end{proof}

\begin{example}\label{example_4.11}
    \begin{enumerate}
        \item[(1)] In $M \otimes_R N$, $m \otimes 0=m \otimes (0+0)=m \otimes
            0+m \otimes 0$, so $m \otimes 0=0$.

        \item[(2)] Notice for any $a \in \faktor{\Z}{2\Z}$, $3a=a$, so for every
            $b \in \faktor{\Z}{3\Z}$, $a \otimes b=3a \otimes b=a \otimes 3b=a
            \otimes 0=0$. This shows that every simple tensor in
            $\faktor{\Z}{2\Z} \otimes_\Z \faktor{\Z}{3\Z}$ reduces to $0$. In
            particular, $1 \otimes 1=0$. Thus there are no nonzero bilinear maps
            mapping $\faktor{\Z}{2\Z} \otimes_\Z \faktor{\Z}{3\Z}$ onto any
            Abelian group.

        \item[(3)] Consider $\faktor{\Z}{2\Z} \otimes_\Z \faktor{\Z}{2\Z}$. This
            tensor product is generated by $0 \otimes 0=1 \otimes 0=0 \otimes
            1=0$, and $1 \otimes 1$. The map
            \begin{eqnarray*}
                \faktor{\Z}{2\Z} \otimes_\Z \faktor{\Z}{2\Z} & \xrightarrow{} &
                                             \faktor{\Z}{2\Z}   \\
                a \otimes b     & \xrightarrow{} & ab \mod{2}   \\
            \end{eqnarray*}
            is a nonzero and bilenear map. Moreover, notice that $2(1 \otimes
            1)=0$, so that $\ord{(1 \otimes 1)}=2$. That is
            \begin{equation*}
                \faktor{\Z}{2\Z} \otimes_\Z \faktor{\Z}{2\Z} \simeq \faktor{\Z}{2\Z}
            \end{equation*}

        \item[(4)] In general, whenever $m,n \in \Z^+$ have greatest common
            divisor $d$, then
            \begin{equation*}
                \faktor{\Z}{m\Z} \otimes_\Z \faktor{\Z}{n\Z} \simeq \faktor{\Z}{d\Z}
            \end{equation*}
            Moreover, observe that $a \otimes b=ab(1 \otimes 1)$, so that
            \begin{equation*}
                \faktor{\Z}{m\Z} \otimes_\Z \faktor{\Z}{n\Z}=\langle 1 \otimes 1 \rangle
            \end{equation*}
            making it cyclic. Moreover, $\ord{\langle 1 \otimes 1\rangle}|d$.

            Now, the map
            \begin{eqnarray*}
                \faktor{\Z}{m\Z} \times \faktor{\Z}{n\Z} & \xrightarrow{} &
                                             \faktor{\Z}{d\Z}   \\
                a \times b     & \xrightarrow{} & ab \mod{d}   \\
            \end{eqnarray*}
            is well defined since $d=(m,n)$. Moreover, $\phi$ is a $\Z$-bilinear
            map. Now, the induced map $\Phi:\faktor{\Z}{m\Z} \otimes_\Z
            \faktor{\Z}{m\Z} \xrightarrow{} \faktor{\Z}{d\Z}$ maps $1 \otimes 1
            \xrightarrow{} 1 \mod{d}$, so that $\faktor{\Z}{m\Z} \otimes_\Z
            \faktor{\Z}{m\Z}$ has order atleast $d$. Since $\faktor{\Z}{m\Z}
            \otimes_\Z \faktor{\Z}{n\Z}$ is cyclic and has order dividing $d$,
            by Lagrange's theorem its order is preciesly $d$. This proves the
            isomorphims.

        \item[(5)] In $\faktor{\Q}{\Z} \otimes_\Z \faktor{\Q}{\Z}$, the simple
            tensors have the form
            \begin{equation*}
                \frac{a}{b} \mod{\Z} \otimes \frac{c}{d} \mod{\Z}
            \end{equation*}
            Then
            \begin{equation*}
                \frac{a}{b} \otimes \frac{c}{d}=
                d(\frac{a}{bd}) \otimes \frac{c}{d}=
                \frac{a}{bd} \otimes d(\frac{c}{d})=
                \frac{a}{bd} \otimes c=
                \frac{a}{bd} \otimes 0=0
            \end{equation*}
            so that $\faktor{\Q}{\Z} \otimes \faktor{\Q}{\Z}=(0)$. Similarly, $A
            \otimes_\Z B=(0)$ for any divisible Abelian group $A$, and any
            torsion Abelian group $B$.

        \item[(6)] We have $\Q \otimes_\Q \Q \simeq \Q \otimes_\Z \Q$ as left
            $\Q$-modules, and $\C \otimes_\C \C \simeq \C \otimes_\R \C$ as left
            $\C$-modules.

        \item[(7)] Let $f:R \xrightarrow{} S$ be a ring homomorphism with
            $f(1_R)=1_S$. Then the action $(s,r) \xrightarrow{} sf(r)$ gives $S$
            the right $R$-module structure so that $S$ is an  $(S,R)$-bimodule.
            Then for any left $R$-module  $N$, the tensor product $S \otimes_R
            N$ is a left $R$-module obtained by \textbf{changing the base} from
            $R$ to  $S$.

        \item[(8)] Let $f:R \xrightarrow{} S$ a ring homomorphism with
            $f(1_R)=1_S$, then $S \otimes_R R \simeq S$ as a left  $S$-module.
            Consider now  $\phi:S \times R \xrightarrow{} S$ by $(s,r)
            \xrightarrow{} sr$. Then $\phi$ is  $R$-balanced and
            \begin{align*}
                \phi(s_1+s_2,r) &=  \phi(s_1,r)+\phi(s_2,r) \\
                \phi(sr,r') &=  \phi(s,rr') \\
            \end{align*}
            Now, there exists a group homomorphism $\Phi:S \otimes_R R
            \xrightarrow{} S$ with $\Phi(s \otimes r)=sr$, since
            \begin{equation*}
                \Phi(s'(s \otimes r))=\Phi(ss' \otimes
                r)=(ss')r=(s's)r=s'(sr)=s'\Phi(s \otimes r)
            \end{equation*}
            then $\Phi$ is also an $S$-module homomorphism. Now, take the map
            $\Phi':S \xrightarrow{} S \otimes_R R$ by $s \xrightarrow{} s
            \otimes 1$. THis map gives an inverse to $\Phi$, since  $\Phi \circ
            \Phi'(s)=\Phi(s \otimes 1)=s$ and $\Phi' \circ \Phi(s \otimes
            r)=\Phi'(sr)=sr \otimes 1=sr$. So that $\Phi' \circ \Phi=\Phi \circ
            \Phi'=i$, the identity map.

        \item[(9)] Let $R$ be a ring and  $I$ an ideal in  $R$. Let  $N$ be a
            left  $R$ module, since  $\faktor{R}{I}$ is an $(\faktor{R}{I},
            R)$-bimodule, $\faktor{R}{I} \otimes_R N$ is a left $\faktor{R}{I}$
            module. Define now
            \begin{equation*}
                IN=\Big{\{} \sum{a_in_i} : a_i \in I, n_i \in N \Big{\}}
            \end{equation*}
            where the sums taken are finite. Then
            \begin{equation*}
                \faktor{R}{I} \otimes_R N \simeq \faktor{N}{IN}
            \end{equation*}
            Moreover, it is generated, as an Abelian group by the simple tensors
            of the form
            \begin{equation*}
                (r \mod{I}) \otimes n=r(1 \otimes n)
            \end{equation*}
            That is, $1 \otimes n$ generates $\faktor{R}{I} \otimes_R N$ as an
            $\faktor{R}{I}$-module for every $n \in N$.

            Now, the map $n \xrightarrow{} 1 \otimes n$, takes $a_in_i
            \xrightarrow{} 1 \otimes a_in_i=a_i \otimes n_i=0$, so that $IN$ is
            in the kernel of this map. Thus, we have an  $R$-module homomorphism
             $f$ from  $\faktor{N}{IN}$ onto $\faktor{R}{I} \otimes_R N$ with
             $f(n)=1 \otimes n$. Then the map
             \begin{eqnarray*}
                 \faktor{R}{I} \times N & \xrightarrow{} & \faktor{R}{I}
                                             \otimes_R N    \\
                 (r,n)  & \xrightarrow{} &  rn \mod{IN} \\
             \end{eqnarray*}
             is a well defined $R$-balanced map. There there exists a group
             homomorphism $g:\faktor{R}{I} \times_R N \xrightarrow{}
             \faktor{N}{IN}$ with $g(r \otimes n)=rn \mod{IN}$. Then $fg=gf=i$,
             wich makes  $f$ 1--1. Therefore
             \begin{equation*}
                 \faktor{N}{IN} \simeq \faktor{R}{I} \otimes_R N
             \end{equation*}
    \end{enumerate}
\end{example}

\begin{theorem}\label{4.4.8}
    Let $R$ be a ring, and let $M,M'$ be right  $R$-modules, and let  $N,N'$ be
    left  $R$ modules, and suppose that $\phi:M \xrightarrow{} M'$ and $\psi:N
    \xrightarrow{} N'$ are $R$-module homomorphisms. Then the following are
    true.
    \begin{enumerate}
        \item[(1)] There exists a unique group homomorphism $\phi \otimes \psi:
            M \otimes_R N \xrightarrow{} M' \otimes_R N'$ such that for all $m
            \in M$ and $n \in N$
            \begin{equation*}
                \phi \otimes \psi(m \otimes n)=\phi(m) \otimes \psi(n)
            \end{equation*}

        \item[(2)] If $M$ and $M'$ are  $(S,R)$-bimodules, for a given ring $S$,
            and  $\phi$ is also an  $S$-module homomorphism, then $\phi \otimes
            \psi$ is a homomorphism of left $S$-modules. In particular, if  $R$
            is commutative, then  $\phi \otimes \psi$ is an  $R$-module
            homomorphism for  $(R,R)$-bimodules.

        \item[(3)] If $\l:M' \xrightarrow{} M''$ and $\m:N' \xrightarrow{} N''$
            are $R$-module homomorphisms, then
            \begin{equation*}
                (\l \otimes \m) \circ (\phi \otimes \psi)=
                (\l \circ \phi) \otimes (\m \circ \psi)
            \end{equation*}
    \end{enumerate}
\end{theorem}
\begin{proof}
    Consider the map $(m,n) \xrightarrow{} \phi(m) \otimes \psi(n)$ from $M
    \times N \xrightarrow{} M' \otimes_R N'$. Since $\phi$ and $\psi$ are
    $R$-module homomorphisms, this map is  $R$-balanced, and so by theorem
    \ref{4.4.6}, assertion (1) follows.

    Now, for assertion (2), notice that the left action of $S$ on $M$ and
    $\phi$ being an $S$-module homomorphism gives  $\phi \otimes \psi(s(m
    \otimes n))=s\phi(n) \otimes \psi()$. Since $\phi \otimes \psi$ is additive,
    we get that  $\phi \otimes \psi$ is an  $S$-module homomorphism. Finally,
    for assetion (3), the uniqueness condition of theorem \ref{4.4.6} gives us
    the result.
\end{proof}

\begin{theorem}\label{4.4.9}
    Let $R$, and  $T$ be rings, and let $M$ a right  $R$-module, let $N$ be
    an  $(R,T)$-bimodule, and let $L$ be a left  $T$ module. Then there exists a
    unique isomorphsim of Abelian groups, taking $(m \otimes n) \otimes l
    \xrightarrow{} m \otimes (n \otimes l)$, for which
    \begin{equation*}
        (M \otimes_R N) \otimes_T L \simeq M \otimes_R (N \otimes_T L)
    \end{equation*}
    Moreover, if $M$ is an  $(S,R)$-bimodule for some ring $S$, then we have an
    isomorphism of $S$-modules.
\end{theorem}
\begin{proof}
    The $(R,T)$-bimodule structure of $N$ makes  $M \otimes_R N$ into a right
    $T$-module, and  $N \otimes_T L$ into a left  $R$-module. Thus the
    isomorphism is well defined. Now, for every  $l \in L$, fixed, the map
    $(m,n) \xrightarrow{} m \otimes (n \otimes l)$ is $R$-balanced, and hence,
    there is a homomorphism
    \begin{eqnarray*}
        M \otimes_R N & \xrightarrow{} & M \otimes_R (N \otimes_T L)    \\
        (m,n)   & \xrightarrow{} & m \otimes (n \otimes l)  \\
    \end{eqnarray*}
    That is, we have a map
    \begin{eqnarray*}
        (M \otimes_R N) \times L & \xrightarrow{} & M \otimes_R (N \otimes_T L)    \\
        (m \otimes n, l)   & \xrightarrow{} & m \otimes (n \otimes l)  \\
    \end{eqnarray*}
    which is also well defined. Moreover it is $T$-balanced, and so induces a
    homomorphism
    \begin{eqnarray*}
        (M \otimes_R N) \otimes_T L & \xrightarrow{}    &   M \otimes_R (N
                                                \otimes_T L)    \\
        (m \otimes n) \otimes l & \xrightarrow{}    &   m \otimes (n \otimes l) \\
    \end{eqnarray*}
    By similar reasoning, we can construct the inverse of this map and establish
    the isomorphism.

    Now, suppose that $M$ is an  $(S,R)$-bimodule. THen for $s \in S$, and $t
    \in T$, we have
    \begin{equation*}
        s((m \otimes n)t)=(s(m \otimes n))t
    \end{equation*}
    so $M \otimes_R N$ is an  $(S,T)$-bimodule. This makes $(M \otimes_R N)
    \otimes_T L$ into a left $S$-module. Since  $N \otime_T L$ is a left
    $R$-module, then  $M \otimes_R (N \otimes_T L)$ is a left $S$-module, and
    the group isomorphism follows to be a homomorphism of  $S$-modules.
\end{proof}
\begin{corollary}
    If $R$ is commutative, and $M$, $N$, and $L$ are  $R$-modules, then  $(M
    \otimes N) \otimes L=M \otimes (N \otimes L)$ as $(R,R)$-bimodules.
\end{corollary}

\begin{definition}
    Let $R$ be a commutative ring with identity, and let $M_1, \dots, M_n$, and
    $L$ be $R$-modules with the standard $R$-module structure. A map  $\phi:M_1
    \times \dots \times M_n \xrightarrow{} L$ is called \textbf{$n$-multilinear}
    over $R$ if it is an  $R$-module homomorphism in each factor.
\end{definition}

\begin{lemma}\label{4.4.10}
    Let $R$ be a commutative ring, and let  $M_1, \dots, M_n$, $L$ be
    $R$-modules. Let $\i:M_1 \times \dots \times M_n \xrightarrow{} M_1 \otimes
    \dots \otimes M_n$ be defined by $(m_1,\dots,m_n) \xrightarrow{} m_1 \otimes
    \dots \otimes m_n$. The following are true.
    \begin{enumerate}
        \item[(1)] For every $R$-moduel homomorphism  $\Phi:M_1 \otimes \dots
            \otimes M_n \xrightarrow{} L$, the map $\phi=\Phi \circ \i$ is
            $n$-multilinear.

        \item[(2)] If $\phi:M_1 \times \dots \times M_1 \xrightarrow{} L$ is
            $n$-multilinear, then there exists a unique $R$-module homomorphism
            $\Phi:M_1 \otimes \dots \otimes M_n \xrightarrow{} L$ for wich
            $\phi=\Phi \circ \i$.
    \end{enumerate}
\end{lemma}

\begin{theorem}\label{4.4.11}
    Let $R$ be a ring, and  $M$ and  $M'$ be right  $R$-modules, and let  $N$
    and  $N'$ be left  $R$-modules. Then there exists unique group isomorphisms
    for which
    \begin{align*}
        (M \oplus M') \otimes_R N &\simeq (M \otimes_R N) \oplus (M' \otimes_R N)   \\
        M \otimes_R (N \oplus N') &\simeq (M \otimes_R N) \oplus (M \otimes_R N')   \\
    \end{align*}
    such that $(m,m') \otimes n \xrightarrow{} (m \otimes n, m' \otimes n)$ and
    $m \otimes (n,n') \xrightarrow{} (m \otimes n, m \otimes n')$. If $M$ and
    $M'$ are  $(S,R)$-bimodules, then the isomorphisms are isomorphism of
    $S$-modules. If $R$ is commutative, then we have an isomorphism of
    $R$-modules.
\end{theorem}
\begin{proof}
    The map $(M \oplus M') \times N \xrightarrow{} (M \otimes_R N) \oplus (M'
    \otimes_R N)$ defined by $((m,m'), n) \xrightarrow{} (m \otimes n, m'
    \otimes n)$ is well define. Moreover, it is $R$-balanced, and hence induces
    a homomorphism  $f:(M \oplus M') \otimes_R N \xrightarrow{} (M \otimes_R N)
    \oplus (M' \otimes N)$ for which $f((m,m'),n)=(m \otimes n, m' \otimes n)$.
    Consider now the $R$-balanced maps
    \begin{eqnarray*}
        M \times N & \xrightarrow{} &   (M \oplus M') \otimes_R N   \\
        M' \times N & \xrightarrow{} &   (M \oplus M') \otimes_R N   \\
    \end{eqnarray*}
    given by
    \begin{eqnarray*}
        (m,n)   &   \xrightarrow{}  &   (m,0) \otimes n \\
        (m',n)   &   \xrightarrow{}  &   (0,m') \otimes n \\
    \end{eqnarray*}
    Define now homomorphisms $M \otimes_R N \xrightarrow{} (M \oplus M') \otimes_R
    N$ and $M' \otimes_R N \xrightarrow{} (M \oplus M') \otimes_R N$. Then we
    get a homomorphism $g:(M \otimes_R N) \oplus (M' \otimes_R N) \xrightarrow{}
    (M \oplus M') \otimes_R N$ with
    \begin{equation*}
        g((m \otimes n_1, m' \otimes n_2))=(m,0) \otimes n_1+(0,m') \otimes n_2
    \end{equation*}
    Then $fg=gf$, the identity map, and are  $S$-module homomorphisms when  $M$
    and  $M'$ are $(S,R)$-bimodules. By similar reasoning, we get the same
    results for the isomorphism $M \otimes_R (N \oplus N') &\simeq (M \otimes_R N)
    \oplus (M \otimes_R N')$
\end{proof}
\begin{corollary}
    $M \otimes_R \bigoplus{N_i} \simeq \bigoplus{(M \otimes_R N_i)}$.
\end{corollary}
\begin{corollary}
    The module obtained from the free $R$-module $R^n$ by extension of scalars
    from $R$ to $S$ is the free  $S$-module  $S^n$.
\end{corollary}
\begin{corollary}
    If $R$ is commutative, and  $M \simeq R^s$, and  $N \simeq R^t$, the free
    $R$-modules with bases $\{m_1, \dots, m_s\}$ and $\{n_1, \dots, n_t\}$,
    respectively, then $M \otimes_R N$ is a free  $R$-module of rank  $st$ with
    basis  $\{m_i \otimes n_j\}$ for all $1 \leq i \leq s$ and  $1 \leq j \leq
    t$.
\end{corollary}

\begin{lemma}\label{4.4.12}
    Let $R$ be a commutative ring, and  $M$ and  $N$ left  $R$-modules,
    considered as $(R,R)$-bimodules. Then there exists a unique $R$-module
    isomorphism, taking  $m \otimes n \xrightarrow{} n \otimes m$ for which
    \begin{equation*}
        M \otimes_R N \simeq N \otimes_R M
    \end{equation*}
\end{lemma}
\begin{proof}
    The map $M \times N \xrightarrow{} N \otimes M$ defined by $(m,n)
    \xrightarrow{} n \otimes m$ is $R$-balanced, and so induces a unique
    homomorphism $f:M \otimes N \xrightarrow{} N \otimes M$ by $m \otimes n
    \xrightarrow{} n \otimes m$. Likewise, there exists a unique homomorphism
    $f:N \otimes M \xrightarrow{} M \otimes N$ by $n \otimes m \xrightarrow{} m
    \otimes n$, and $fg=gf=i$, the identity map.
\end{proof}

\begin{lemma}\label{4.4.13}
    Let $R$ be a commutative ring, and  $A$ and  $B$  $R$-algebras. Then teh
    multiplication
    \begin{equation*}
        (a \otimes b)(a' \otimes b)=aa' \otimes bb'
    \end{equation*}
    is well defined, and makes $A \otimes_R B$ into an  $R$-algebra.
\end{lemma}
\begin{proof}
    Notice that by definition, $r(a \otimes b)=ra \otimes b=a \otimes br=(a
    \otimes b)r$ for every $r \in R$, $a \in A$, and $b \in B$. Now, the map
    $\phi:(A \times B) \times (A \times B) \xrightarrow{} A \otimes_R B$ defined
    by $(a,b,a',b') \xrightarrow{} aa' \otimes bb'$ is multilinear over $R$.
    Then there exists an  $R$-module homomorphism  $\Phi:(A \otimes_R B)
    \otimes_R (A \otimes_R B) \xrightarrow{} A \otimes_R B$ taking $(a \otimes
    b) \otimes (a' \otimes b') \xrightarrow{} aa' \otimes bb'$. This gives us a
    bilinear map $\Phi:(A \otimes_R B) \otimes_R (A \otimes_R B) \xrightarrow{} A
    \otimes_R B$ defined by $(a \otimes b, a' \otimes b') \xrightarrow{} aa'
    \otimes bb'$. This makes multiplication well defined. Moreover, this
    multiplication satisfies the distributive laws, so that $A \otimes_R B$ is
    an  $R$-algebra.
\end{proof}

\begin{example}\label{example_4.12}
    The tensor product $\C \otimes_\R \C \simeq \C^4$ is a free $\R$-module of
    rank $4$, with the basis  $\{1 \otimes 1, 1 \otimes i, i \otimes 1, i
    \otimes i\}$, where $i^2+1=0$. Moreover, this tensor product is a
    commutative ring. Notice then that  $(i \otimes i)(i \otimes i)=i^2 \otimes
    i^2=-1 \otimes -1=(-1)(-1) \otimes 1=1$ Then we have $(i \otimes i)(i
    \otimes i+1)=0$ so that $\C \otimes_\R \C$ is not an integral domain. Now,
    notce also that  $\C \otimes_\R \C$ is an  $\R$-algebra whose left and right
    actions are the same. Then  $\C \otimes_\R \C$ is a left  $\C$-module, since
     $\C$ is a  $(\C,\R)$-bimodule. Likewise, $\C \otimes_\R \C$ is aright  $\C$
     module.
\end{example}
