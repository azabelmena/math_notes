\section{Definitions and Examples}

\begin{definition}
    Let $R$ be a ring. A  \textbf{left module} over $R$, or a  \textbf{left
    $R$-module}, is a set $M$ together with a binary operation  $+:M \times M
    \xrightarrow{} M$, called \textbf{addition}, and a left action $R \times M
    \xrightarrow{} M$ of $R$ on $M$, defined by $(r,m) \xrightarrow{} rm$,
    called  \textbf{left scalar multiplication} such that $M$ is an Abelian group
    under $+$, and for all $r,s \in R$ and $m,n \in M$
    \begin{enumerate}
        \item[(1)] $(r+s)m=rm+sm$

        \item[(2)] $(rs)m=r(sm)$

        \item[(3)] $r(m+n)=rm+rn$
    \end{enumerate}
    We define \textbf{right modules} over $R$ similarly, where the action is a
    right action $M \times R \xrightarrow{} M$ defined by $(m,r) \xrightarrow{}
    mr$.
\end{definition}

\begin{definition}
    Let $R$ be a ring with identity  $1$, and  $M$ a left  (or right module)
    over $R$. We call  $M$  \textbf{unital} if $1m=m$ (or $m1=m$) for all
    $m \in M$.
\end{definition}

\begin{lemma}\label{4.1.1}
    If $R$ is a commutative ring, then every left module of $R$ is also a right
    module over $R$.
\end{lemma}

\begin{definition}
    Let $R$ be a ring, and $M$ be a module over $R$. A \textbf{submodule} of $M$
    over  $R$ is a subgroup $N$ of $M$ which is closed under the action of  $R$
    on $M$; i.e. that is $rn \in N$ for all  $r \in R$ and  $n \in N$.
\end{definition}

\begin{example}\label{}
    \begin{enumerate}
        \item[(1)] Every module $M$, has as itself, and the subgroup $(0)$ as
            submodules. We call $(0)$ the \textbf{trivial module}.

        \item[(2)] If $F$ is a field, then every  $F$-module is a vector space
            over $F$. Moreover, every submodule of an  $F$-module is a subspace
            of the overlying vector space. It can be shown that if  $F$ is a
            field, then the definition for modules and vector spaces coincide.

        \item[(3)] If $R$ is a ring, then  $R$ is a left module over itself with
            scalar multiplication being the the multipliaction of $R$. Moreover,
            $R$ is a right module over itself (these structures need not
            coincide).

        \item[(4)] Let $R$ be a ring with identity  $1$, and  $n \in \Z^+$. Make
             $R^n$ into an left $R$-module by defining $+$ to be component wise
             addition, and left scalar multiplication $R \times R^n
             \xrightarrow{} R^n$ defined by
             \begin{equation*}
                 r(a_1, \dots, a_n)=(ra_1, \dots, ra_n)
             \end{equation*}
             We call $R^n$ the  \textbf{free module} of \textbf{rank} $n$ over
             $R$. Notice that if  $R$ is a field, then $R^n$ become the usual
             $n$-space over $R$.

         \item[(5)] The field $\R$ is an  $\R$-module, a  $\Q$-module, and a
             $\Z$-module. More generally, if $R$ is a ring with identity $1_R$,
             and $S$ is a subring of $R$ with identity $1_S$, such that $1_R=1_S$,
             then every  $R$-module is also an  $S$-module.

         \item[(6)] Let $R$ be a ring, and  $M$ an  $R$-module, and let  $I$ be
             an ideal of  $R$, such that  $am=0$ for all  $a \in I$ and  $m \in
             M$. Then we say that  $I$  \textbf{annihilates} $M$. Make $M$ into
             an left $(\faktor{R}{I})$-module by defining the addition to be the
             addition of $M$, and left scalar multiplication to be the action
             $\faktor{R}{I} \times M \xrightarrow{} M$ defined by
             \begin{equation*}
                 (r+I)m=rm
             \end{equation*}

         \item[(7)] Let $A$ be any Abelian group. We can make $A$ into a left
             $\Z$-module by defining $+$ to be group operation on $A$, and
             defining left scalar multiplication $(n,a) \xrightarrow{} na$ by
             \begin{equation*}
                 na=\begin{cases}
                     a+\dots+a \text{ ($n$-times)}, \text{ if } n>0  \\
                     0, \text{ if } n=0 \\
                     (-a)+\dots+(-a) \text{ ($n$-times)}, \text{ if } n<0  \\
                    \end{cases}
             \end{equation*}
             Then it can be shown that this action makes $A$ into a unital
             $\Z$-module (notice that since $\Z$ is commutative, $A$ is also a
             right $\Z$-module).

             Conversely, if we have $M$ as a  $\Z$-module, then by definition
             $M$ is an Abelian group. That is, we have shown that every Abelian
             group is a $\Z$-module, having as  $\Z$-submodules all subgroups.

             Notice now, that if  $A$ is an Abelian group, and  $x \in A$ is of
             order  $n$, the $nx=0$. Therefore, there exist scalars  $n \in \Z$
             and elements  $x \in A$ for which  $nx=0$. Notice, then if
             $\ord{A}=m$, then by Lagrange's theorem, $mx=0$ for all  $x \in A$;
             this shows that  $A$ is also a $(\faktor{\Z}{m\Z})$-module.

         \item[(8)] Consider the example above. If $A$ is a
             $(\faktor{\Z}{p\Z})$-module, since $\faktor{\Z}{p\Z}$ is a field,
             then $A$ can be considered as a vector space over the finite field
              $\F_p=\faktor{\Z}{p\Z}$. We call these group  \textbf{elementary
                  Abelian $p$-groups}. For example, the Klein $4$-group, $V_4$
                  is an elementary Abelian $2$ -group (it is a vector space over
                  $\F_2$ of $\dim{V_4}=2$).
    \end{enumerate}
\end{example}

\begin{example}[Polynomials]\label{}
    \begin{enumerate}
        \item[(1)] Let $F$ be a field, and consider the polynomial riong
            $F[x]$. Let $V$ be a vector space over  $F$, and let  $T$ be a
            linear transformation. Define  $+$ on  $V$ to be the usual vector
            addition, and define the left action  $F[x] \times V \xrightarrow{}
            V$ by
            \begin{equation*}
                p(x)v=(a_nT^n+\dots+a_1T+a_0)v=a_nT^(v)+\dots+a_1T(v)+a_0v
            \end{equation*}
            Where $p(x)=a_nx^n+\dots+a_1x+a_0$, and $a_0, \dots, a_n \in F$.
            Then $p(x)$ acts on $V$ as the linear transofrmation $T$. Then the
            action $F[x] \times V \xrightarrow{} V$ makes $V$ into a left
            $F[x]$-module.

        \item[(2)] Consider the vector space $V$ over $F$ as the affine $n$-space
            $F^n$, and consider the \textbf{left shift operator} $T(x_1, \dots,
            x_n)=(x_2, \dots, x_n,0)$. Letting $\{e_1, \dots, e_n\}$ the
            standard basis of $F^n$ over  $F$, notice that
            \begin{equation*}
                T^k(e_i)=\begin{cases}
                            e_{i-k}, \text{ if } i>k    \\
                            0, \text{ if } i \leq k \\
                         \end{cases}
            \end{equation*}
            Then $T$ determines the action of any polynomial $p \in F[x]$ on a
            vector $v \in V$. This shows that the action  $F[x] \times V
            \xrightarrow{} V$ depends on our choice of $T$. This gives a  1--1
            correspondence between all $V$ as $F[x]$-modules, and all pairs
            $(V,T)$, where $V$ is a vector space and  $T$ is a gien linear
            transformation.

        \item [(3)] Consider the vector space $V$ as an $F[x]$-module, with $T$
            the linear transformation givenby the action of  $x$ on  $V$. Let
            $W \subseteq V$ be an  $F[x]$-submodule. Then $W$ must also be an
            $F$-module, and hence must be a subspace of  $V$. Moreover, notice
            that $T$ takes $W$ onto itself; i.e.  $T(w) \in W$ for all $w \in
            W$, so that  $W$ is  \textbf{$T$-invariant}.

            Conversely, if $T$ is a  $T$-invariant subspace of  $V$ as a vector
            space over  $F$, then  $T^n(W) \subseteq W$ for all $n \in \Z^+$.
            Moreover, any linear combination of  $T^n$ takes  $W$ onto itself,
            so that  $W$ is also invariant by the action of any polynomial in
            $T$. This makes  $W$ into an  $F[x]$-submodule. That is, there is a
            1--1 correspondence between $F[x]$-submodules, and $T$-invariant
            subspaces of  $V$.
    \end{enumerate}
\end{example}

\begin{remark}
    From here on, we will mean ``module'' to mean left module, and ``scalar
    multiplaction'' to mean left scalar multiplication, unless otherwise specified.
\end{remark}

\begin{lemma}[The Submodule Criterion for Unital Modules]\label{lemma_4.1.2}
    Let $R$ be a ring with identity, and $M$ be a unital module over $R$. A
    nonempty subset $N$ of $M$ is a submodule over  $R$ if, and only if for every
    $r,s \in R$, and  $x,y \in N$,  $rx+sy \in N$.
\end{lemma}
\begin{proof}
    Suppose that $N$ is a submodule for $M$. Then $0 \in N$, so that  $N$ is
    nonempty. Moreover, by the axioms for modules, observe that if we restrict
    the scalar multiplication of $M$ to $N$  (i.e. $R \times N \xrightarrow{}
    N$), then $rx+sy \in N$ for all  $x,y \in N$ and  $r,s \in R$; that is the
    action of $R$ on $N$ takes  $N$ onto itself.

    Conversely, suppose that $N$ is nonempty, and that for any $x,y \in N$, and
    $r,s \in R$, that  $rx+sy \in N$. Take  $s=0$, then we have that $rx \in N$,
    which makes $N$ closed under scalar multiplication. Moreover, let  $r=1$ amd
     $s=1$, then we get $x+y \in N$; and for  $r=0$ and  $s=-1$, we have $-y \in
     N$. By the subgroup criterion, this makes  $N \leq M$. Therefore  $N$ is a
     submodule of $M$.
\end{proof}

\begin{definition}
    Let $R$ be a commutative ring with identity. An \textbf{$R$-algebra} is a
    ring $A$ with identity, together with a ring homomorphism $f:R
    \xrightarrow{} A$, taking $1_R \xrightarrow{} 1_A$ such that the subring
    $f(R)$ of $A$ is contained in the center of  $A$; i.e.  $f(R) \subseteq
    Z(A)$.
\end{definition}

\begin{lemma}\label{lemma_4.1.3}
    Let $R$ be a commutative with with identity. If  $A$ is an $R$-algebra,
    then $A$ is a left module of $R$, and a right module of $R$.
\end{lemma}
\begin{proof}
    Let $+$ be the usual addition for $A$, and define the actions $R \times A
    \xrightarrow{} A$ and $A \times R \xrightarrow{} A$ by
    \begin{equation*}
        ra=f(r)a \text{ and } ar=af(r)
    \end{equation*}
    respectively. Notice by definition that $f(r)a=af(r)$, so that $ra=ar$.
\end{proof}

\begin{definition}
    Let $R$ be a commutative wrng with identity, and let $A$ and  $B$
    $R$-algebras. A  \textbf{$R$-algebra homomorphism} is a ring homomorphism
    $\phi:A \xrightarrow{} B$, taking $1_A \xrightarrow{} 1_B$, such that
    $\phi(ra)=r\phi(a)$ for all $r \in R$ and  $a \in A$; where  $ra=f(r)a$ is
    the underlying scalar multiplication of $A$. We call $\phi$ a
    \textbf{$R$-algebra isomorphism} if $\phi$ is a ring isomorphism.
\end{definition}

\begin{example}\label{}
    Let $R$ be a commutative ring with identity  $1_R$.
    \begin{enumerate}
        \item[(1)] Every commutative ring with identity is a $\Z$-algebra. Take
            $f:\Z \xrightarrow{} R$ via the map $f:n \xrightarrow{} n \cdot
            1_R$.

        \item[(2)] For any ring $A$ with identity  $1_A$, if  $R$ is a subring
            of the center  $Z(A)$, containing $1_A$, then $A$ is an
            $R$-algebra. In particular, a commutative ring  $A$ with identity is
            an  $R$-algebra for any subring  $R$ of  $Z(A)$ containing $1_A$.

        \item[(3)] The polynomial ring $R[x]$, the multivariate polynomial ring
            $R[x_1, \dots, x_n]$, and the ring $RG$, for any finite group  $G$
            are all  $R$-algebras.

        \item[(4)] If $A$ is an  $R$-algebra, then the  $R$-module structure of
             $A$ depends on the image  $f(R) \subseteq Z(A)$. In fact, every
             algebra  $A$ arises from a subring of  $Z(A)$, containing $1_A$, up
             to homomorphism.

         \item[(5)] Let $F$ be a field, and $A$ an  $F$-algebra. Then $F$ itself
             is a subring of $Z(A)$ containing $1_A$, in fact,  $1_A=1_F$.
    \end{enumerate}
\end{example}

\begin{lemma}\label{lema_4.1.4}
    Let $R$ be a commutative ring with identity. Then a ring  $A$ with identity
     $1_A$ is an  $R$-algebra if, and only if it is a unital module satisfying
     \begin{equation*}
         r(ab)=(ra)b=a(rb) \text{ for all } r \in R \text{ and } a,b \in A
     \end{equation*}
     where $ra=f(r)a$ is the underlying action of $R$ on $A$.
\end{lemma}
\begin{proof}
    Certainly, if $A$ is an  $R$-algebra, then it is a unital module. Moreover,
    notice that if $ra=f(r)a$, then $r(ab)=f(r)(ab)=(f(r)a)b=a(f(r)b)$, so that
    $r(ab)=(ra)b=a(rb)$.

    Conversly, suppose that $A$ is a unital module satisfying
     \begin{equation*}
         r(ab)=(ra)b=a(rb) \text{ for all } r \in R \text{ and } a,b \in A
     \end{equation*}
     And where $ra=f(r)a$, where $f:R \xrightarrow{} A$ is a ring homomorphism.
     Then $1_Ra=f(1_R)a=a$, so that $f(1_R)=1_A$; that is, $f:1_R \xrightarrow{}
     1_A$. Moreover, since $r(ab)=(ra)b=a(rb)$, we have that
     $(f(r)a)b=a(f(r)b)$, i.e. that $f(r)a=af(r)$ for all $a,b \in A$ and  $r
     \in R$. This makes $f(r) \in Z(A)$, so that $f(R) \subseteq Z(A)$. This
     makes $A$ into an  $R$-algebra.
\end{proof}
