\section{Exact Sequences of Modules}

\begin{definition}
    Let $A$ and  $C$ be modules. We call a module  $B$, containing  $A$ an
    \textbf{extension} of $C$ by  $A$ if  $\faktor{B}{A} \simeq C$.
\end{definition}

\begin{example}\label{example_4.13}
    Let $A$,  $B$, and  $C$ be modules. If  $A$ is a submodule of  $B$, then
    there exists a map 1--1 $\psi:A \xrightarrow{} B$ such that $A \simeq
    \psi(A) \subseteq B$. If $\faktor{B}{\psi(A)} \simeq C$, then there exists a
    map $\phi:B \xrightarrow{} C$ wich is onto, for which $\ker{\phi}=\psi(A)$.
    Then we get the following diagram.
    \begin{equation*}
        A \xrightarrow{\psi} B \xrightarrow{\phi} C
    \end{equation*}
\end{example}

\begin{definition}
    Let $\{A_n\}$ a collection of modules. By a \textbf{sequence}, we mean a
    diagram
    \begin{equation*}
        \dots \xrightarrow{} A_{n-1} \xrightarrow{} A_n \xrightarrow{} A_{n+1}
        \xrightarrow{} \dots
    \end{equation*}
    where each $A_i$ is mapped to $A_{i+1}$ by some module homomorphism.
\end{definition}

\begin{definition}
    Let $A$,  $B$, and  $C$ be modules. We call a pair of module homomorphisms
    $\a,\b$, defined by the diagram
    \begin{equation*}
        A \xrightarrow{\a} B \xrightarrow{\b} C
    \end{equation*}
    \textbf{exact} at $B$ if  $\a(A)=\ker{\b}$. If $\{A_n\}$ is a collection of
    modules, we call a sequence defined by the diagram
    \begin{equation*}
        \dots \xrightarrow{} A_{n-1} \xrightarrow{} A_n \xrightarrow{} A_{n+1}
        \xrightarrow{} \dots
    \end{equation*}
    an \textbf{exact sequence} if it is exact at each $A_n$.
\end{definition}

\begin{example}\label{example_4.14}
    The sequence from example \ref{example_4.13}, $A \xrightarrow{\psi} B
    \xrightarrow{\phi} C$, is exact at $B$,  since $\ker{\phi}=\psi(A)$.
\end{example}

\begin{lemma}\label{4.5.1}
    Let $R$ be a ring, and  $A$,  $B$, and  $C$  $R$-modules. The following are
    true for any $R$-module homomorphisms $\psi:A \xrightarrow{} B$ and $\psi:B
    \xrightarrow{} C$
    \begin{enumerate}
        \item[(1)] $(0) \xrightarrow{\i} A \xrightarrow{\psi} B$ is exact at $A$
            if, and only if $\psi$ is 1--1.

        \item[(2)] $B \xrightarrow{\phi} C \xrightarrow{\i'} (0)$ is exact at
            $B$ if, and only if $\phi$ is onto.
    \end{enumerate}
\end{lemma}
\begin{corollary}
    The sequence $(0) \xrightarrow{} A \xrightarrow{\psi} B \xrightarrow{\phi} C
    \xrightarrow{} (0)$ is an exact sequence if, and only if $\psi$ is 1--1,
    $\phi$ is onto, and $\ker{\phi}=\psi(A)$; that is, $B$ is an extension of
    $C$ by  $A$.
\end{corollary}

\begin{definition}
    We call an exact sequence of the form $(0) \xrightarrow{} A \xrightarrow{\psi}
    B \xrightarrow{\phi} C \xrightarrow{} (0)$ a \textbf{short exact sequence}.
\end{definition}

\begin{lemma}\label{4.5.2}
    If $A \xrightarrow{\a} B \xrightarrow{\b} C$ is exact at $Y$, then  $(0)
    \xrightarrow{} \a(A) \xrightarrow{} B \xrightarrow{} \faktor{B}{\ker{\b}}
    \xrightarrow{} (0)$ is a short exact sequence.
\end{lemma}

\begin{example}\label{example_4.15}
    \begin{enumerate}
        \item[(1)] Let $A$, and $C$ be modules. Consider the sequence
            \begin{equation*}
                (0) \xrightarrow{} A \xrightarrow{\i} A \oplus C
                \xrightarrow{\pi} C \xrightarrow{} (0)
            \end{equation*}
            where $\i$ is the inclusion map, and  $\pi$ is the projection map
            about the second coordinate; i.e. $\pi:(a,c) \xrightarrow{} c$. This
            sequence is a short exact sequence, since $\i(A)=\ker{\pi}$.

        \item[(2)] Consider the $\Z$-modules $\Z$ and $\faktor{\Z}{n\Z}$,
            the sequence
            \begin{equation*}
                (0) \xrightarrow{} \Z \xrightarrow{\i} \Z \oplus \faktor{\Z}{n\Z}
                \xrightarrow{\phi} \faktor{\Z}{n\Z} \xrightarrow{} (0)
            \end{equation*}
            is a short exact sequence giving an extension of $\faktor{\Z}{n\Z}$
            by $\Z$. Another extension is given by the short exact sequence
            \begin{equation*}
                (0) \xrightarrow{} \Z \xrightarrow{\n} \Z
                \xrightarrow{\pi} \faktor{\Z}{n\Z} \xrightarrow{} (0)
            \end{equation*}
            where $n:x \xrightarrow{} nx$, and $\pi:x \xrightarrow{} x \mod{n}$
            is the natural map. THese are ``inequivalent'' extensions of
            $\faktor{\Z}{n\Z}$ by $\Z$.

        \item [(3)] If $\phi:B \xrightarrow{} C$ is any module homomorphism,
            form the exact sequence
            \begin{equation*}
                (0) \xrightarrow{} \ker{\phi} \xrightarrow{\i} B
                \xrightarrow{\phi} \phi(B) \xrightarrow{} (0)
            \end{equation*}
            where $\i$ is the inclusion map. If  $\phi$ is onto, we may extend
            the sequence $B \xrightarrow{\phi} C$ (i.e. extend $\phi$) to a
            short exact sequence with $A=\ker{\phi}$.

        \item[(4)] Let $R$ be a ring, and  $M$ an  $R$-module homomorphism. Let
             $S$ be a set of generators for  $M$, and consider the free $R$-module
             $F(S)$ on $S$. Then
             \begin{equation*}
                 (0) \xrightarrow{} K \xrightarrow{\i} F(S) \xrightarrow{\phi}
                 \xrightarrow{} M \xrightarrow{} (0)
             \end{equation*}
             is a short exact sequence,  where $\phi$ is the unique  $R$-module
             homomorphism  which is the identity on $S$, and  $K=\ker{\phi}$.
    \end{enumerate}
\end{example}

\begin{definition}
    Let $(0) \xrightarrow{} A \xrightarrow{} B \xrightarrow{} C \xrightarrow{}
    (0)$ and $(0) \xrightarrow{} A' \xrightarrow{} B' \xrightarrow{} C'
    \xrightarrow{} (0)$ be short exact sequences. A \textbf{homomorphism} of
    sequences is a triple $(\a,\b,\y)$ of module homomorphisms, such that the
    following diagram commutes
    \begin{equation*}
        \begin{tikzcd}
            {(0)} & A & B & C & {(0)} \\
            {(0)} & {A'} & {B'} & {C'} & {(0)}
            \arrow[from=1-1, to=1-2]
            \arrow[from=1-2, to=1-3]
            \arrow[from=1-3, to=1-4]
            \arrow[from=1-4, to=1-5]
            \arrow[from=2-1, to=2-2]
            \arrow[from=2-2, to=2-3]
            \arrow[from=2-3, to=2-4]
            \arrow[from=2-4, to=2-5]
            \arrow["\gamma"', from=1-4, to=2-4]
            \arrow["\beta"', from=1-3, to=2-3]
            \arrow["\alpha"', from=1-2, to=2-2]
        \end{tikzcd}
    \end{equation*}
    If $\a$,  $\b$, and  $\y$ are  module isomorphisms, we call $(\a,\b,\y)$ an
    isomorphism of sequences.
\end{definition}


\begin{definition}
    We call two exact sequences $A \xrightarrow{} B \xrightarrow{} C$ and $A'
    \xrightarrow{} C' \xrightarrow{} B'$ \textbf{equivalent} if $A=A'$, $C=C'$,
    and there exists an isomorphism of sequences between them. We call  the
    corresponding extensions  $B$ and  $B'$  \textbf{equivalent}.
\end{definition}

\begin{lemma}\label{4.5.3}
    The composition of homomorphisms of exact sequences is a homomorphism
    of sequences.
\end{lemma}

\begin{lemma}\label{4.5.4}
    Isomorphisms of exact sequences form an equivalence relation on any set of
    exact sequences.
\end{lemma}

\begin{example}\label{example_4.16}
    \begin{enumerate}
        \item[(1)] Let $m,n \in \Z^+$ integers greater than $1$, and suppose
            that  $n|m$. Let  $k=\frac{m}{n}$, and define a map from the exact
            sequences of $\Z$-modules described by the following diagram
            \begin{equation*}
                \begin{tikzcd}
{(0)} & \Z & \Z & {\faktor{\Z}{n\Z}} & {(0)} \\
{(0)} & {\faktor{\Z}{k\Z}} & {\faktor{\Z}{k\Z}} & {\faktor{\Z}{n\Z}} & {(0)}
\arrow[from=1-1, to=1-2]
\arrow["n", from=1-2, to=1-3]
\arrow["\pi", from=1-3, to=1-4]
\arrow[from=1-4, to=1-5]
\arrow[from=2-1, to=2-2]
\arrow["\iota", from=2-2, to=2-3]
\arrow["{\pi'}", from=2-3, to=2-4]
\arrow[from=2-4, to=2-5]
\arrow["\alpha"', from=1-2, to=2-2]
\arrow["\beta"', from=1-3, to=2-3]
\arrow["\gamma"', from=1-4, to=2-4]
                \end{tikzcd}
            \end{equation*}
            where $n:x \xrightarrow{} nx$, $\pi:x \xrightarrow{} x \mod{n}$,
            $\a$,  $\b$ are the natural projections, and  $\y$ is the identity.
            For the second sequnece, we take  $\i:a \mod{k} \xrightarrow{}
            na\mod{m}$, and $\pi'$ the natural projection of $\faktor{\Z}{m\Z}$
            onto $\faktor{(\faktor{\Z}{m\Z})}{(\faktor{n\Z}{m\Z})} \simq
            \faktor{\Z}{n\Z}$. Then $(\a,\b,\y)$ describe a homorphism of short
            exact sequences.

        \item[(2)] If $(0) \xrightarrow{} \Z \xrightarrow{n} \Z \xrightarrow{\pi}
            \faktor{\Z}{n\Z} \xrightarrow{} (0)$ is the short exact sequence
            defined in example \ref{example_4.15}, mapping each module to itself
            via the map $x \xrightarrow{} -x$ gives a isomorphism of short exact
            sequences, which take this sequence onto itself. Notice however that
            this isomorphism is not an equeivalence, since it is not the
            identity on $\Z$.

        \item[(3)] Consider the diagram
            \begin{equation*}
\begin{tikzcd}
	{(0)} & {\faktor{\Z}{2\Z}} & {\faktor{\Z}{2\Z} \oplus \faktor{\Z}{2\Z}} &
                                {\faktor{\Z}{2\Z}} & {(0)} \\
	{(0)} & {\faktor{\Z}{2\Z}} & {\faktor{\Z}{2\Z} \oplus \faktor{\Z}{2\Z}} &
                                {\faktor{\Z}{2\Z}} & {(0)}
	\arrow["\psi", from=1-2, to=1-3]
	\arrow["\phi", from=1-3, to=1-4]
	\arrow[from=1-4, to=1-5]
	\arrow[from=2-1, to=2-2]
	\arrow["{\psi'}"', from=2-2, to=2-3]
	\arrow["{\phi'}"', from=2-3, to=2-4]
	\arrow[from=2-4, to=2-5]
	\arrow["i"', from=1-4, to=2-4]
	\arrow["\beta"', from=1-3, to=2-3]
	\arrow["i"', from=1-2, to=2-2]
	\arrow[from=1-1, to=1-2]
\end{tikzcd}
            \end{equation*}
            Where $i$  is the identity map, $\psi$ is a 1--1 map mapping into the
            first component of $\faktor{\Z}{2\Z} \oplus \faktor{\Z}{2\Z}$, and
            $\phi$ projects $\faktor{\Z}{2\Z} \oplus \faktor{\Z}{2\Z}$ onto its
            second component, and where $\psi'$ and  $\phi'$ behave just as  $\psi$
            and  $\phi$. If $\b$ maps $\faktor{\Z}{2\Z} \oplus \faktor{\Z}{2\Z}$
            to $\faktor{\Z}{2\Z} \oplus \faktor{\Z}{2\Z}$ by exchanging the
            factors; that is  $\b:(m,n) \xrightarrow{} (n,m)$, then this diagram
            commutes and gives an equivalence of short exact sequences which is
            not the identity.
    \end{enumerate}
\end{example}

\begin{lemma}[The Short Five Lemma]\label{4.5.5}
    Let $(\a,\b,\y)$ be a homomorphism of short exact sequences given by the
    diagram
    \begin{equation*}
        \begin{tikzcd}
            {(0)} & A & B & C & {(0)} \\
            {(0)} & {A'} & {B'} & {C'} & {(0)}
            \arrow[from=1-1, to=1-2]
            \arrow[from=1-2, to=1-3]
            \arrow[from=1-3, to=1-4]
            \arrow[from=1-4, to=1-5]
            \arrow[from=2-1, to=2-2]
            \arrow[from=2-2, to=2-3]
            \arrow[from=2-3, to=2-4]
            \arrow[from=2-4, to=2-5]
            \arrow["\gamma"', from=1-4, to=2-4]
            \arrow["\beta"', from=1-3, to=2-3]
            \arrow["\alpha"', from=1-2, to=2-2]
        \end{tikzcd}
    \end{equation*}
    then the following are true
    \begin{enumerate}
        \item[(1)] If $\a$ and  $\y$ are 1--1, then so is  $\b$.

        \item[(2)] If $\a$ and  $\y$ are onto, then so is  $\b$.

        \item[(3)] If $\a$ and  $\y$ are isomorphisms, then so is  $\b$.
    \end{enumerate}
\end{lemma}
\begin{proof}
    We chase the elements of the following diagram.
    \begin{equation*}
        \begin{tikzcd}
            {(0)} & A & B & C & {(0)} \\
            {(0)} & {A'} & {B'} & {C'} & {(0)}
            \arrow[from=1-1, to=1-2]
            \arrow["\psi", from=1-2, to=1-3]
            \arrow["\phi", from=1-3, to=1-4]
            \arrow[from=1-4, to=1-5]
            \arrow[from=2-1, to=2-2]
            \arrow["{\psi'}"', from=2-2, to=2-3]
            \arrow["{\phi'}", from=2-3, to=2-4]
            \arrow[from=2-4, to=2-5]
            \arrow["\gamma"', from=1-4, to=2-4]
            \arrow["\beta"', from=1-3, to=2-3]
            \arrow["\alpha"', from=1-2, to=2-2]
        \end{tikzcd}
    \end{equation*}
    Suppose that $\a$ and  $\y$ are 1--1, and choose an element $b \in
    \ker{\b}$. Then $\b(b)=0$, and $\phi'\b(b)=0$. Moreover, since this diagram
    commutes, we have that $\phi'\b=\y\phi$, so that $\phi'\b(b)=\y\phi(b)$,
    which implies that $\phi(b)=0$, since $\y$ is 1--1. This makes $b \in
    \ker{\phi}$. Now, since the sequences are also exact, we have
    $\ker{\phi}=\psi(A)$, so that $b \in \psi(A)$. That is, there is an $a \in
    A$ for wich $b=\psi(a)$. Now, by commutativity again, we have
    $\b\psi=\psi'\a$. So  $\b\psi(b)=\b(b)=0$, which makes $\psi'\a(a)=0$, so
    that $\a(a)=0$. Since $\a$ is 1--1, this makes $a=0$ and we get
    $b=\psi(a)=0$ which makes $\ker{\b}=(0)$. That is, $\b$ is 1--1.

    Now, suppose that $\a$ and $\y$ are onto, and let  $b' \in B'$. Then
    $\phi'(b)=\y(c)$ for some $c \in C$; since $\y$ is onto. Now, by lemma
    \ref{4.5.1} $\phi$ is onto, so there is a $b \in B$ for which $b=\y(c)$. By
    the commutativity of the diagram $\phi'\b(b)=\y(\phi(b))=\y(c)=\phi'(b')$,
    so $\phi'(b'-\b(b))=0$ which puts $b'-\b(b) \in \ker{\phi'}$. Now, by the
    exactness of the sequences, $\ker{\phi'}=\psi'(A')$, so there is an $a' \in
    A'$ for which $b'-\b(b)=\psi'(a)=a \in A$, since $\a$ is onto. Then $a \in
    \ker{\phi'}$. Now, by commutativity, observe that
    $\psi(a)=\psi(b'-\b(b))=0$, and since $\psi$ is 1--1, we get $b'-\b(b)=0$.
    That is $b'=\b(b)$ and $\b(B)=B'$, which makes $\b$ onto. Lastly, observe
    that if $\a$ and $\y$ are isomorphisms, then they are 1--1 and onto, which
    makes $\b$ 1--1 and onto, and hence an isomorphism as well.
\end{proof}
