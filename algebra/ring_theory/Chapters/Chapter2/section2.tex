\section{Principle Ideal Domains.}

\begin{definition}
    An integral domain $R$ is called a \textbf{principle ideal domain (PID)} if
    every ideal in $R$ is principle.
\end{definition}

\begin{example}\label{2.4}
    \begin{enumerate}
        \item[(1)] Every Euclidean domain is a PID, as dictated by lemma
            \ref{2.1.1}. Hence the rings $\Z$ and  $\Z[i]$ are PIDs, however,
            the polynomial ring $\Z[x]$ is not principle, recall the ideal
            $(2,x)$.

        \item[(2)] The ring $\Z[\sqrt{-5}]$ is not a PID, consider the ideal
            $(3,2+\sqrt{-5})$. However, notice that
            $(3,1+\sqrt{-5})(3,1-\sqrt{-5})=(3)$ is principle, despite
            $(3,1+\sqrt{-5})$ and $(3,1-\sqrt{-5})$ are not principle.

        \item[(3)] The ring $\Z[1+\frac{\sqrt{-19}}{2}]$ is a PID, but not a
            Euclidean domain.
    \end{enumerate}
\end{example}

\begin{lemma}\label{2.2.1}
    Let $R$ be a principle ideal domain and let $d$ be a generator for the ideal
    $(a,b)$, for $a,b \in R$. Then the following are true.
    \begin{enumerate}
        \item[(1)] $d=(a,b)$; i.e. $d$ is the greatest common divisor of $a$ and
            $b$.

        \item[(2)] There exist $x,y \in R$ for which  $ax+by=d$.

        \item[(3)] $d$ is unique up to unit.
    \end{enumerate}
\end{lemma}

\begin{lemma}\label{2.2.2}
    Every nonzero prime ideal in a principle ideal domain $R$ is maximal.
\end{lemma}
\begin{proof}
    Let $(p) \neq (0)$ be a prime ideal of $R$,. Let $(m)$ be an ideal of $R$
    containing $(p)$. Then $p \in (m)$ so that $p=rm$ for some  $r \in R$. Now,
    since $p$ is prime, and $rm \in (p)$, then either $r \in (p)$ or $m \in
    (p)$. If $m \in (p)$, then $(p)=(m)$. Otherwise, if $r \in (p)$, then $r=ps$
    for some  $s \in R$. Then  $p=rm=pms=p(ms)$ which makes $ms=1$, hence $m$ is
    a unit, which makes $(m)=(0)$.
\end{proof}
\begin{corollary}
    If $R$ is any commutative ring, such that the polynomial ring $R[x]$ is a
    principle ideal domain, then $R$ is necessarily a field.
\end{corollary}
\begin{proof}
    If $R[x]$ is a PID, then $R \subseteq R[x]$, as a subring, must be an
    integral domain. Consider now, the ideal $(x)$, then $\faktor{R[x]}{(x)}
    \simeq R$ which makes $(x)$ prime by lemma \ref{1.4.4}. Therefore $(x)$ is
    maximal, which then makes $R$ a field by lemma \ref{1.4.3}.
\end{proof}

\begin{definition}
    Let $R$ be a commutative ring, and  $N:R \xrightarrow{} \N$ a norm. We call
    $N$ a \textbf{Dedekin-Hasse norm} if $N$ is a positive norm suc that for
    all $a,b \in N$, either $a \in (b)$, or there exists an element $c \in
    (a,b)$ such that $N(c)<N(b)$.
\end{definition}

\begin{lemma}[The Dedekin-Hasse Criterion]\label{2.2.3}
    An integral domain $R$ is a PID if, and only if it has a Dedekin-Hasse norm.
\end{lemma}
\begin{proof}
    Let $I \neq (0)$ an ideal of $R$. Let $a \in I$ a nonzero element, so that
    $(a,b) \subseteq I$. Since $N$ is Dedekin-Hasse, and by minimality of $b$,
    we get that $a \in (b)$ so that $I=(b)$ is principle.
\end{proof}

\begin{example}\label{2.5}
    Consider the ring $\Z[1+\frac{\sqrt{-19}}{2}]$. With norm  $N=\|\cdot\|^2$
    the field norm. Let $x,y \in \Z[1+\frac{\sqrt{-19}}{2}]$ be nonzero elements
    and that $\frac{x}{y} \notin \Z[1+\frac{\sqrt{-19}}{2}]$. Write
    \begin{equation*}
        \frac{x}{y}=\frac{a+b\sqrt{-19}}{c} \in \Q[1+\frac{\sqrt{-19}}{2}]
    \end{equation*}
    where $a,b,c$ are all coprime, with  $c>1$. Then there are integers $u,v,w$
    with $av+bu+cw=1$, then  $au-19bv=cq+r$ for some quotient $q$ and remainder
    $r$ with  $N(r) \leq \frac{c}{2}$ and let $s=u+v\sqrt{-19}$ and
    $t=q-w\sqrt{-19}$. Then we find that
    \begin{equation*}
        0<N(\frac{x}{y}s-t) \leq \frac{1}{4}+\frac{19}{c^2}
    \end{equation*}
    Then $s=1, t=\frac{(a-1)+b\sqrt{-19}}{2} \in R$ satisfy $0<N(\frac{x}{y}s-t)$

    Now, suppose that $c=3$, then $3 \nmid (a^2+19b^2)$. Then $a^2+19b^2=3q+r$
    with  $r=1$ or $r=2$. Then $s=a-b\sqrt{-19},t=q$ statisfy
    $0<N(\frac{x}{y}s-t)$. Finally, for $c=4$, with $a,b$ not both even, so
    that $a^2+19b^2$ is odd. Then $a^2+19b^2=4q+r$ so for $q,r \in \Z$ with
    $0<r<4$, then $s=a-b\sqrt{-19}, t=q$ satisfy $0<N(\frac{x}{y}s-t)$. Now, if
    both $a$ and  $b$ are odd, then  $a^2+19b^2 \equiv 1+3 \mod{8}$ so taht
    $a^2+19b^2=8q+4$ for some  $q \in \Z$, then
    \begin{equation*}
        s=\frac{a-b\sqrt{-19}}{2} \text{ and } t=q
    \end{equation*}
    satisfy $0<N(\frac{x}{y}s-t)$.
\end{example}
