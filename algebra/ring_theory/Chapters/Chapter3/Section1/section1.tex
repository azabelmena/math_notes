%----------------------------------------------------------------------------------------
%	SECTION 1.1
%----------------------------------------------------------------------------------------

\section{More on Polynomial Rings.}

\begin{theorem}\label{3.1.1}
    Let $I$ be an ideal of  $R$ and  $I[x]$ the ideal of $R[x]$ generated by
    $I$. Then
    \begin{equation*}
        \faktor{R[x]}{I[x]} \simeq \faktor{R}{I}[x]
    \end{equation*}
    Moreover, if $I$ is a prime ideal in $R$, then $I[x]$ is a prime ideal in
    $R[x]$.
\end{theorem}
\begin{proof}
    Consider the map $\pi:R[x] \xrightarrow{} \faktor{R}{I}[x]$ given by $f
    \xrightarrow{} f \mod{I}$. That is, reduce $f$ modulo  $I$.  Then $\pi$ is a
    ring homomorphism with kernel $\ker{\pi}=I[x]$. By the first isomorphism
    theorem, we get
    \begin{equation*}
        \faktor{R[x]}{I[x]} \simeq \faktor{R}{I}[x]
    \end{equation*}
    Now, let $I$ be a prime ideal in $R$, Then we have that $\faktor{R}{I}$ is
    an integral domain, hence, so is $\faktor{R}{I}[x]$, which makes $I[x]$ a
    prime ideal of $R[x]$.
\end{proof}

\begin{example}\label{3.1}
    Consider the ideal $n\Z$ in  $\Z$. By above, we have
    \begin{equation*}
        \faktor{\Z[x]}{n\Z[x]} \simeq \faktor{\Z}{n\Z}[x]
    \end{equation*}
    with natural map reduction of polynomials modulo $n$. If $n$ is composite,
    then the ring $\faktor{\Z}{n\Z}[x]$ is not an integral domain. If $n=p$ a
    prime, then  $\faktor{\Z}{n\Z}[x]$ is an integral domain.
\end{example}

\begin{definition}
    We define the \textbf{polynomial ring} in $n$  \textbf{variables} $x_1,
    \dots, x_n$ with \textbf{coefficients} in $R$ inductively to be
    \begin{equation*}
        R[x_1, \dots, x_n]=R[x_1, \dots, x_{n-1}][x_n]
    \end{equation*}
    and is the set of all \textbf{multivariate polynomials} of the form $f(x_1,
    \dots,x_n)=\sum{ax_1^{d_1} \dots x_n^{d_n}}$. We call the monic term
    $x_1^{d_1} \dots x_n^{d_n}$ of $f$ a  \textbf{monomial}. We define the
    \textbf{degree} of a monomial to be $\deg{x_1^{d_1} \dots
    x_n^{d_n}}=d_1+\dots+d_n$ and we define the \textbf{degree} of $f$ to be
    $\deg{f}=\max{\{\deg{x_1^{d_1} \dots x_n^{d_n}}\}}$ (i.e. the maximum degree
    of all monomials of $f$). If all the monomials of $f$ have the same degree,
    we call  $f$  \textbf{homogeneous}, or, a \textbf{form}.
\end{definition}

\begin{lemma}\label{3.1.2}
    Let $R$ be a ring. Then  $R[x_1, \dots, x_n]$ is a ring.
\end{lemma}

\begin{example}\label{3.2}
    Consider the polynomial ring $\Z[x,y]$ in two variables $x$ and  $y$ with
    integer coefficients. Then $p(x,y)=2x^3+xy-y^2$ and has $\deg{p}=3$. The
    polynomial $q(x,y)=-3xy+2y^2+x^2y^3$ has $\deg{q}=5$. The sum
    \begin{equation*}
        p+q(x,y)=2x^3-2xy+y^2+x^2y^3 \text{ has degree } \deg{p+q}=5
    \end{equation*}
    and the product
    \begin{equation*}
        pq(x,y)=-6x^4y+4x^3y^2+2x^5y^3-3x^2y^2+5xy^3+x^3y^4-2y^4-x^2y^5
    \end{equation*}
    had degree $\deg{pq}=8$.
\end{example}
