\section{Polynomial Rings over Fields.}

\begin{theorem}\label{3.2.1}
    Let $F$ be a field. Then the polnomial ring $F[x]$ is a Euclidean domain.
    That is, if $a(x),b(x) \in F[x]$, with $b(x) \neq 0$, then there exist
    unique polynomials $q(x), r(x) \in F[x]$ such that
    \begin{equation*}
        a(x)=q(x)b(x)+r(x) \text{ where } r(x)=0 \text{ or } \deg{r}<\deg{b}
    \end{equation*}
\end{theorem}
\begin{proof}
    If $a(x)=0$, then take $q(x)=r(x)=0$ and we are done. Now, suppose that
    $a(x) \neq 0$ and let $\deg{a}=n$. Then by induction on $n$, let
    $\deg{b}=m$. If $n<m$, then take  $q(x)=0$ and $r(x)=a(x)$. Otherwise, we
    have $n \geq m$. Now, write
    \begin{align*}
        a(x)    &=  a_0+a_1x+\dots+a_nx^n   \\
        b(x)    &=  b_0+b_1x+\dots+b_mx^m   \\
    \end{align*}
    Let $a'(x)=a(x)-\frac{a_n}{b_m}x^{n-m}b(x)$. Then $\deg{a} \leq n$, and
    since $a_n,b_m \in F$ and  $b_m \neq 0$,  $a'$is well defined. By induction,
    let  $q'(x),r(x) \in F[x]$ such that
    \begin{equation*}
        a'(x)=q'(x)b(x)+r(x) \text{ where } r(x)=0 \text{ or } \deg{r}<\deg{b}
    \end{equation*}
    Then take $q(x)=q'(x)-\frac{a_n}{b_m}x^{n-m}$ Then we have
    \begin{equation*}
        a(x)=q(x)b(x)+r(x) \text{ where } r(x)=0 \text{ or } \deg{r}<\deg{b}
    \end{equation*}

    Now, for uniqueness, suppose that $q_1,r_1 \in F[x]$ are such that
    \begin{equation*}
        a(x)=q_1(x)b(x)+r_1(x) \text{ where } r_1(x)=0 \text{ or } \deg{r_1}<\deg{b}
    \end{equation*}
    Then $r(x)=a(x)-q(x)b(x)$ and $r_1(x)=a(x)-q_1(x)b(x)$ both have degree
    $\deg<m$. Then the difference  $r(x)-r_1(x)=b(x)(q(x)-q_1(x))$ also has
    degree less than $m$. Moreover, we have that
    $\deg{b(q-q_1)}=\deg{b}+\deg{(q-q_1)}=m+\deg{(q-q_1)}$. This makes
    $q(x)-q_1(x)=0$, so that $q(x)=q_1(x)$. It follows then that $r(x)=r_1(x)$.
\end{proof}
\begin{corollary}
    If $F$ is a field, then $F[x]$ is a principle ideal domain. Moreover, it is
    a Unique Factorization Domain.
\end{corollary}

\begin{example}\label{example_3.3}
    \begin{enumerate}
        \item[(1)] We have yet another example of $\Z[x]$ not being a PID, and
            that is because $\Z$ is not a field.

        \item[(2)] The ring $\Q[x]$ is a PID, since $\Q$ is a field. Moreover,
            notice that the ideal  $(2,x)$ is not principle in $\Z[x]$, but is
            principle in $\Q[x]$, since $2$ is a unit in  $\Q[x]$. Moreover,
            $(2,x)=(1)$, making it the entire ring.

        \item[(3)] If $p$ is prime, then the ring  $\faktor{\Z}{p\Z}[x]$ is a
            PID, since $\faktor{\Z}{p\Z}$ is a field. If $p=2$, then the ideal
            $(2,x)=(x)$ and is prinicple in $\faktor{\Z}{2\Z}[x]$. If $p \neq 2$,
            then  $2$ is a unit, making  $(2,x)=(1)$; the entire ring.

        \item[(4)] The multivariate polynomial ring $\Q[x,y]=\Q[x][y]$ is not a
            PID, since $\Q[x]$ is not a field. Notice also that $(x,y)$ is not
            prinicple in $\Q[x,y]$.
    \end{enumerate}
\end{example}
