\section{Noetherian Rings}

\begin{definition}
    Let $R$ be a ring. We call a nondecreasing sequence $\{I_n\}_{n \in \Z^+}$
    of ideals of $R$ an  \textbf{ascending chain of ideals}. We call $R$
    \textbf{Noetherian} if it satisfies the \textbf{ascending chain
    considition}; that is, if $\{I_n\}$ is an ascending chain of ideals of $R$,
    then there exists an $m \in \Z^+$ for which  $I_n=I_m$ for all  $n \geq n$.
\end{definition}

\begin{lemma}\label{1.1.1}
    If $I$ is an ideal of a Noetherian ring  $R$, then the factor ring
    $\faktor{R}{I}$ is also Noethrian. In particular, the image of a Noetherian
    ring under any ring homomorphism is Noetherian.
\end{lemma}
\begin{proof}
    This follows by the isomorphism theorems for ring homomorphisms.
\end{proof}

\begin{theorem}\label{1.1.2}
    The following are equivalent for any ring $R$.
    \begin{enumerate}
        \item[(1)] $R$ is Noetherian.

        \item[(2)] Every nonempty collection of ideals of $R$ contains a maximal
            element under inclusion.

        \item[(3)] Every ideal of $R$ is finitelt generated.
    \end{enumerate}
\end{theorem}
\begin{proof}
    Let $R$ be Noetherian, and let  $\Ic$ an nonempty collection of ideals of
    $R$. Choose an ideal  $I_1 \in \Ic$. If $I_1$ is maximal, we are done. If
    not, then there is an ideal $I_2 \in \Ic$ for which  $I_1 \subseteq I_2$.
    Now, if $I_2$ is maximal, we are done. Otherwise, proceeding inductively, if
    there are no maximal ideals of $R$ in $\Ic$, then by the axiom of choice,
    construct the infinite strictly increasing chain
    \begin{equation*}
         \dots \subseteq I_1 \subseteq I_2 \subseteq \dots
    \end{equation*}
    of ideal of $R$. This contradicts that  $R$ is Noetherian, so  $\Ic$ must
    contain a maximal element.

    Now, suppose that any nonempty collection of ideals of  $R$ contains
    a maximal element. Let  $\Ic$ the collection of all finitely generated
    ideals of $R$, and let $I$ be any ideal of  $R$. By hypothesis, $\Ic$ has a
    maximal element  $I'$. Now suppose that $I \neq I'$, and choose an
    $x \in \com{I}{I'}$, then the ideal generated by $I'$ and  $x$ is finitely
    generated, and so is in  $\Ic$; but that contradicts the maximality of  $I'$.
    Therefore we must have  $I=I'$.

    Finally, suppose every ideal of $R$ is finitely genrated, and let
    $I=(a_1, \dots, a_n)$. Let
    \begin{equation*}
        I_1 \subseteq I_2 \subseteq \dots
    \end{equation*}
    an ascending chain of ideals of $R$ for which
    \begin{equation*}
        I=\bigcup_{n \in \Z^+}{I_n}
    \end{equation*}
    Since $a_i \in I$ for each  $1 \leq j \leq n$, we have that  $a_i \in
    I_{i_j}$ and $i \in \Z^+$. Now, let  $m=\max{\{j_1, \dots, j_n\}}$ and
    coinsider the ideal $I_m$. Then  $a_i \in I_m$ for each $i$, which makes $I
    \subseteq I_m$. That is,  $I_n=I_m$ for some  $n \geq m$; which makes  $R$
    Noetherian.
\end{proof}

\begin{example}\label{example_1.1}
    \begin{enumerate}
        \item[(1)] Every principle ideal domain (PID) is Noetherian, since any
            collection of ideals has a maximal element.

        \item[(2)] The rings $\Z$, $\Z[i]$, and $k[x]$ (where $k$ is a field)
            are Noetherian.

        \item[(3)] The multivariate polynomial ring $\Z[x_1, x_2, \dots]$ is not
            Noetheria, since the ideal $(x_1, x_2, \dots)$ is not finitely
            generated.
    \end{enumerate}
\end{example}

\begin{theorem}[Hilbert's Basis Theorem]\label{1.1.3}
    If $R$ is a Noetherian ring, then so is the polynomial ring $R[x]$.
\end{theorem}
\begin{proof}
    Let $I$ be an ideal of  $R[x]$, and let $L$ be the set of all leading
    coefficients of polyonimials in $I$. Notice that since  $0 \in I$, then  $0
    \in L$, so that $L$ is nonempty. Moreover, let $f(x)=ax^d+\dots$ and
    $g(x)=bx^e+\dots$ polynomials in $I$ of degree  $\deg{f}=d$ and $\deg{g}=e$,
    with leading coefficients $a, b \in R$. Then for any  $r \in R$, we have the
    coefficient  $ra-b=0$, or  $ra-b$ is the leading coefficient of the
    polynomial  $rx^ef-x^dg \in I$. In either case, we get  $ra-b \in L$. This
    makes  $L$ an ideal of  $R$. Now, since  $R$ is Noetherian  $L$ is finitely
    generated ; let $L=(a_1, \dots, a_n)$. Then for every $1 \leq  i \leq n$,
    let  $f_i \in I$ the polynomial of degree  $\deg{f_i}=e_i$ whose leading
    coefficient is $a_i$. Take, then  $N=\max{\{e_1, \dots, e_n\}}$. Then for
    any $d \in \faktor{\Z}{N\Z}$, let $L_d$ be the set of all leading
    coefficients of polynomials in  $I$, of degree $d$, together with $0$. Let
    $f_{di} \in I$ a polynomial of degree $\deg{f_{di}}=d$ with leading
    coefficient $b_{di}$. We wish to show that
    \begin{equation*}
        I=(f_1, \dots, f_n) \cup (f_{d1}, \dots f_{nd})
    \end{equation*}

    Let $I'=(f_1, \dots, f_n) \cup (f_{d1}, \dots f_{nd})$. By construction,
    since the generators were chosen from $I$,  $I' \subseteq I$. Now, if
    $I \neq I'$. Then there is a nonzero polynomial $f \in I$ of minimum degree
     not contained in $I'$  (i.e $f \notin I'$). Let $\deg{f}=d$, and let $a$ be
     the leading coefficient of  $f$. Suppose that  $d \geq N$. Since  $a \in
     L$, $a$ is an  $R$-linear combination of the generators of  $L$; i.e.
     \begin{equation*}
         a=r_1a_1+\dots+r_na_n
     \end{equation*}
     where $r_1, \dots, r_n \in R$. Let
     \begin{equation*}
        g=r_1x^{d-e_1}f_1+\dots+r_nx^{d-e_n}f_n
     \end{equation*}
     then $g \in I'$ and has degree $\deg{g}=d$ and leading coefficient $a$.
     Hence  $f-g \in I'$ is of smaller degree, and by the minimality of  $f$,
     $f-g=0$, which makes  $f=g \in I'$; a contradiction. THerefore $I=I'$

     Now, if $d<N$, then  $a \in L_d$, and so is an  $R$-linear combniation of
     generators of  $L_d$; that is
     \begin{equation*}
         a=r_1b_{d1}+\dots+r_nb_{dn}
     \end{equation*}
     where $r_1, \dots, r_n \in R$. Then let
     \begin{equation*}
         g=r_1f_{d1}+\dots+r_nf_{dn}
     \end{equation*}
     then $g \in I'$ is a polynomial of degree  $\deg{g}=d$ and leading
     coefficient $a$; which gives us the above contraditction.

     Therefore, $I=I'$, and since $I'$ is finitely generated,  $R[x]$ is
     Noetherian.
 \end{proof}
 \begin{corollary}
     Let $k$ be a field. Then the polynomial ring in  $n$ variables  $k[x_1,
     \dots, x_n]$ is Noetherian.
 \end{corollary}

 \begin{definition}
     Let $k$ be a field. We call a ring  $R$ a  \textbf{$k$-algebra} if $k$ is
     contained in the center of  $R$  (i.e. $k \subseteq Z(R)$), and $1_k=1_R$.
     We call  $R$ a  \textbf{finitely generated} $k$-algbera if $R$ is generated
     by  $k$ together with a finite set $\{r_1, \dots, r_n\}$ of elements of
     $R$.
 \end{definition}

 \begin{definition}
     Let $k$ be a field and $R$ and  $S$  $k$-algebras. We call a map  $\phi:R
     \xrightarrow{} S$ a \textbf{$k$-algebra homomorphism} if $\phi$ is a ring
     homomorphism, and  $\phi$ is the identity on $k$.
 \end{definition}

 \begin{lemma}\label{1.1.4}
     Let $k$ be a field. Then a ring $R$ is a finitely generated $k$-algebra if,
     and only if there exists a $k$-algebra homomorphism $\phi:k[x_1, \dots,
     x_n] \xrightarrow{} R$ taking $k[x_1, \dots, x_n]$ onto $R$.
 \end{lemma}
 \begin{proof}
     If $R$ is generated by elements  $r_1, \dots, r_n$ as a $k$-algebra, then
     define the map $\phi:k[x_1, \dots, x_n] \xrightarrow{} R$ by taking $x_i
     \xrightarrow{} r_i$, for all $1 \leq i \leq n$, and  $\phi(a)=a$ for all $a
     \in k$. Then  $\phi$ extends to a ring homomorphism of  $k[x_1, \dots,
     x_n]$ onto $R$.

     Conversly, let  $\phi$ be a $k$-algebra homomorphism of $k[x_1, \dots,
     x_n]$ onto $R$,  such that the images $\phi(x_1), \dots \phi(x_r)$ generate
     $R$ as a  $k$-algebra. Then $R$ is finitely generated, and since  $k[x_1,
     \dots, x_n]$ is Notherian by the corollory to Hilbert's basis theorem, $R$
     is a quotient of a Noetherian ring, and hence  $R$ is Noetherian. This
     makes  $R$ a finitely generated  $k$-algebra.
 \end{proof}

 \begin{example}\label{example_1.2}
     Let $R$ be a  $k$-algebra, for some field  $k$, viewed as a finite
     dimensional vector space over  $k$. In particular, let
     $R=\faktor{k[x]}{(f(x))}$, where $f(x)$ is a nonzero polynomial over $k$.
     Then  $R$ is a finitely generated  $k$-algebra, since it has a finite
     basis, and that basis serves as a generator for $R$ as a  $k$-algebra.
     Thus, we have the ideals of $R$ are $k$-subspaces. Moreover, any ascending
     chain of ideals of $R$ has at most  $\dim_k{R}-1$ distinct terms, which
     means that $R$ satisfies the ascending chain condition.
 \end{example}
