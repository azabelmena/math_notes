\section{Affine Varieties.}

\begin{definition}
    Let $K$ be an algebraically closed set. We define \textbf{affine $n$-space},
    $\A^n(K)$ over $K$ to be the set of all $n$-tuples of elements of $K$. We
    call elements  $P=(a_1, \dots, a_n) \in \A^n(K)$ \textbf{points} and we call
    each of the $a_i$ \textbf{coordinates} of $P$. When context is clear, we
    simply write $\A^n(K)$ as $\A^n$.
\end{definition}

\begin{definition}
    Let $\A^n(K)$ be the affine space over an algebraically closed field $K$.
    Let  $f \in K[x_1, \dots, x_n]$ and define $f(P)=f(a_1, \dots, a_n)$. We
    define the set of \textbf{zeros} of $f$ to be:
    \begin{equation*}
        Z(f)=\{P \in \A^n(K) : f(P)=0\}
    \end{equation*}
    For any $T \subseteq K[x_1, \dots, x_n]$, the \textbf{zero} set of $T$ is
    defined to be
    \begin{equation*}
        Z(T)=\{P \in \A^n(K) : f(P)=0, \text{ for all } f \in T\}
    \end{equation*}
    If $T=(a)$ the ideal of $K[x_1, \dots, x_n]$ generated by $T$, then we
    simply write  $Z(T)=Z(a)$.
\end{definition}

\begin{definition}
    We call a subset $Y \subseteq \A^n(K)$, of the affine space over $K$
    \textbf{algebraic} (or an \textbf{algebraic set}), if $Y=Z(T)$ for some $T
    \subseteq K[x_1, \dots, x_n]$.
\end{definition}

\begin{lemma}\label{1.1.1}
    The collection of all algebraic sets of an affine space $\A^n$ forms a
    topology under closed sets.
\end{lemma}
\begin{proof}
    Let $\A^n=Z(0)$ and $\emptyset=Z(1)$. Then $\A^n$ and $\emptyset$ are both
    algebraic. Now, let  $X$ and  $Y$ be algebraic, then there are  $S,T$ such
    that  $X=Z(S)$ and $Y=Z(T)$. Now, let $P \in X \cup Y$, then  $P$ is a zero
    of any polynomial $f \in ST$, conversly, suppose that $P \in Z(ST)$ where $P
    \notin Y$. There exists a polynomial  $f \in S$ with  $f(P) \neq 0$. Now,
    for any $g \in T$, we have that if $fg(P)=0$, then $g(P)=0$, so that $P \in
    S$. Therefore we have  $X \cup Y=Z(ST)$, making $X \cup Y$ algebraic. So
    that the collection of algebraic sets is closed under finite intersection.

    Lastly, consider a collection $\{Y_\alpha\}$ of algebraic sets, where
    $Y_\alpha=Z(T_\alpha)$ for some $T_\alpha$. Let
    \begin{equation*}
        Y=\bigcap{Y_\alpha} \text{ and } T=\bigcup{T_\alpha}
    \end{equation*}
    and let $P \in Y$. Then $P$ is in every $Y_\alpha$ making it a root of some
    $f_\alpha \in T_\alpha$, thus  $P \in Z(T)$. Similarly, if $P \in Z(T)$,
    then $P \in Y$, making $Y=Z(T)$, and making the collection of algebraic sets
    closed under arbitrary intersections.
\end{proof}

\begin{definition}
    We define the \textbf{Zariski topology} on $\A^n$ to be the topology taking
    as open sets complements of algebraic sets.
\end{definition}
