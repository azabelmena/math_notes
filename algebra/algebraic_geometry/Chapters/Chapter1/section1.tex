\section{Affine Varieties}

\begin{definition}
    Let $k$ be an algebraically closed field. We define  \textbf{affine
    $n$-space} over $k$ to be the set $\A^n(k)$ of $n$-tuples of elements of
    $k$. We write simply $\A^n$ when  $k$ is understood. We call the elements of
    $\A^n$  \textbf{points} and if $P=(a_1, \dots, a_n)$ is a point of $\A^n$,
    we call each $a_i$ the \textbf{coordinates} of $P$.
\end{definition}

\begin{example}\label{example_1.1}
    Let $k$ be any algebraically closed field, and consider the multivariate
    polynomial ring $k[x_1, \dots, x_n]$. We can interperate the elements of
    $k[x_1, \dots, x_n]$ as functions from affine space $\A^n(k)$ to $k$ by
    taking  $f(P)=f(a_1, \dots, a_n)$, where $f \in k[x_1, \dots, x_n]$ and $P
    \in \A^n(k)$. This leads us to be able to talk about the set of zeros of a
    polynomial over $k$.
\end{example}

\begin{definition}
    Let $k$ be an algebraically closed field, and $f \in k[x_1, \dots, x_n]$ a
    multivariate polynomial over $k$. We define the \textbf{set of zeros} of $f$
    to be the set
    \begin{equation*}
        Z(f)=\{P \in \A^n(k) : f(P)=0\}
    \end{equation*}
    Let $T$ be a subset of  $k[x_1, \dots, x_n]$. Then we define the \textbf{set
    of zeros} of $T$ to be
    \begin{equation*}
        Z(T)=\bigcap_{f \in T}{Z(f)}
    \end{equation*}
    Now, if $I=(f_1, \dots, f_r)$ is an ideal of  $k[x_1, \dots, x_n]$ generated
    by $T$, then we write  $Z(T)=Z(I)=Z(f_1, \dots, f_r)$.
\end{definition}

\begin{definition}
    Let $k$ be an algebraically closed field. We call a subset $Y$ of $\A^n$ an
    \textbf{algebraic set} if there exists some $T \subseteq k[x_1, \dots, x_n]$
    for which $Y$ is the set of zeros of $T$; i.e. $Y=Z(T)$.
\end{definition}

\begin{lemma}\label{lemma_1.1.1}
    Let $k$ be an algebraically closed field. Then algebraic sets of $\A^n$ make
     $\A^n$ into a topology under closed sets.
\end{lemma}
\begin{proof}
    Let $\A^n=Z(0)$ and $\emptyset=Z(1)$. Then $\A^n$ and $\emptyset$ are both
    algebraic. Now, let  $X$ and  $Y$ be algebraic, then there are $S,T$ such
    that  $X=Z(S)$ and $Y=Z(T)$. Now, let $P \in X \cup Y$, then $P$ is a zero
    of any polynomial $f \in ST$, conversly, suppose that $P \in Z(ST)$ where $P
    \notin Y$. There exists a polynomial  $f \in S$ with $f(P) \neq 0$. Now, for
    any $g \in T$, we have that if $fg(P)=0$, then $g(P)=0$, so that $P \in S$.
    Therefore we have  $X \cup Y=Z(ST)$, making $X \cup Y$ algebraic. So that
    the collection of algebraic sets is closed under finite intersection.

    Lastly, consider a collection $\{Y_\alpha\}$ of algebraic sets, where
    $Y_\alpha=Z(T_\alpha)$ for some $T_\alpha$. Let
    \begin{equation*}
        Y=\bigcap{Y_\alpha} \text{ and } T=\bigcup{T_\alpha}
    \end{equation*}
    and let $P \in Y$. Then $P$ is in every $Y_\alpha$ making it a zero of some
    $f_\alpha \in T_\alpha$, thus  $P \in Z(T)$. Similarly, if $P \in Z(T)$, then
    $P \in Y$, making $Y=Z(T)$, and making the collection of algebraic sets
    closed under arbitrary intersections.
\end{proof}

\begin{definition}
    We define the \textbf{Zariski topology} on affine $n$-space  $\A^n$ to be
    the topology on $\A^n$ whose open sets are complements of algebraic sets.
\end{definition}

\begin{example}\label{example_1.2}
    Consider the Zariski topology on affine $1$-space  $\A^1$. Now, since $k[x]$
    is a PID,  every algebraic set of $\A^1$ is the set of zeros of preciesly
    one polynomial. Moreover, by the algebraic closure of  $k$, for any nonzero
    polynomial  $f$ over  $k$, we have
    \begin{equation*}
        f(x)=c(x-a_1) \dots (x-a_n)
    \end{equation*}
    where $c,a_1, \dots, a_n \in k$. Then $Z(f)=\{a_1, \dots, a_n\}$, so that
    the algebraic sets of $\A^1$ are the emptyset, itself, and finite subsets.
    Thus the Zariski topology on  $\A^1$ consists of complements of finite sets,
     the emptyset, and $\A^1$ itself. Notice that this topology is not
     Hausdorff.
\end{example}

\begin{definition}
    Let $X$ be a topological space, and  $Y$ a subspace of  $X$. We call  $Y$
     \textbf{irreducible} if it cannot be written as the union $Y=Y_1 \cup Y_2$
     of two sets $Y_1$ and $Y_2$ closed in $Y$. We make the convention that the
     emptyset is not irreducible.
\end{definition}

\begin{example}\label{example_1.3}
    \begin{enumerate}
        \item[(1)] Notice that the affine space $\A^1$ is irreducible. We have
            the only closed sets are finite sets, and since  $k$ is
            algebraically closed, and hence infinite, then  $\A^1$ must be
            infinite.

        \item[(2)] Subspaces of irreducible spaces are irreducible and dense.

        \item[(3)] If $Y$ is an irreducible space of a topological space  $X$,
            then the closure  $\cl{Y}$ is also irreducible in $X$.
    \end{enumerate}
\end{example}

\begin{definition}
    We define an \textbf{algebraic affine variety} to be an irreducible closed
    subset of $\A^1$ under the Zariski topology. We define an open set of an
    affine variety to be a  \textbf{quasi-affine variety}.
\end{definition}
