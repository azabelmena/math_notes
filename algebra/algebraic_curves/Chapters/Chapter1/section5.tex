\section{Irreducible Components}

\begin{definition}
    Let $k$ be a field. We call an algebraic set $X \subseteq \A^n(k)$
    \textbf{reducible} if it can be written as the union of two algebraic sets;
    that is, there exist $X_1,X_2 \subseteq \A^n(k)$ such that $X=X_1 \cup X_2$.
    We call an algebraic set \textbf{irreducible} if it is not reducible.
\end{definition}

\begin{example}\label{example_1.8}
    \begin{enumerate}
        \item[(1)] The algebraic sets defined by the equations $y^2=x^3-x$,
            $y^2=x^3+x^2$, and $z^2=x^2+y^2$ in $\A^2(\R)$ and $\A^3(\R)$,
            respecively, are irreducible.

        \item[(2)] The algebraic set described by the equation
            $y^2-xy-x^2y+x^2=0$ is reducible in $\A^2(\R)$.
    \end{enumerate}
\end{example}

\begin{lemma}\label{1.5.1}
    An algebraic set is irreducible if, and only if its ideal is prime.
\end{lemma}
\begin{proof}
    Let $k$ be a field, and  $X \subseteq \A^n(k)$. Suppose that the ideal
    $I(X)$ is not prime. Let $f_1f_2 \in I(X)$, but $f_1,f_2 \not\in I(X)$.
    Then
    \begin{equation*}
        X=(X \cap V(f_1)) \cup (X \cap V(f_2))
    \end{equation*}
    and $X \cap V(f_1) \subseteq X$ and $X \cap V(f_2) \subseteq X$. This makes
    $X$ reducible, by definition.

    Conversely, suppose that $X$ is reducible, and that $X=X_1 \cup X_2$ for
    $X_1, X_2 \subseteq \A^n(k)$. Then $I(X) \subseteq I(X_1)$ and $I(X)
    \subseteq I(X_2)$. Let $f_1 \in I(X_1)$ and $f_2 \in I(X_2)$, but $f_1,f_2
    \not\in I(X)$. Then $f_1f_2 \in I(X)$, but $f_1,f_2 \not\in I(X)$, so that
    $I(X)$ is not prime.
\end{proof}

\begin{example}\label{example_1.9}
    \begin{enumerate}
        \item[(1)] Consider the polynomial $f(z,w)=w-z^2$ in $\C[z,w]$, and let
            $g \in I(V(f))$. Then for every point $P \in V(f)$, $g(P)=0$.
            Moreover, by the division theorem for polynomials, there
            are polynomials $q,r \in
            (\C[z])[w]$ for which
            \begin{equation*}
                g(z,w)=f(z,w)q(z,w)+r(z,w) \text{ where } r=0 \text{ or }
                \deg_w{r} < \deg_w{f}
            \end{equation*}
            Notice that the degree of $r$ in  $w$,  $\deg_w{r}=0$ so that $r \in
            \C[z]$. Now, since $g(P)=f(P)q(P)+r(P)=r(P)=0$, and $r$ is a
            polynomial only in $z$, then $r(z,w)=r(z)=0$. That is
            $g(z,w)=f(z,w)q(z,w)$ so that $g \in (f)$. This makes $I(V(f))=(f)$.
            Lastly, notice that $\faktor{\C[z,w]}{(f)} \simeq \C[z]$ which is an
            integral domain. This makes $(f)$ a prime ideal, and hence the
            algebraic set $V(w-z^2)$ is irreducible.

        \item[(2)] Let $f(z,w)=w^4-z^2$ and $g(z,w)=w^4-z^2w^2+zw^2-z^2$. Then
            observe that
            \begin{align*}
                V(f,g)  &= V(f) \cap V(g) \\
                        &=  (V(w^2-z) \cup V(w^2+z)) \cap
                                (V(w+z) \cup V(w-z) \cup V(w^2+z)) \\
                        &= V(w^2+z) \\
            \end{align*}

        \item[(3)] Consider the polynomial $f(x,y)=y^2+x^2(x-1)^2$. $f$ factors
            into
            \begin{equation*}
                f(x,y)=(x^2-x-iy)(x^2-x+iy)
            \end{equation*}
            in $\C[x,y]$, where $i^2=-1$. This makes $f$ irreducible in $\R[x,y]$.
            However, notice that $f(x,y)=0$ only at the points $(1,0)$ and
            $(0,0)$, so that $|V(f)|=2$. This makes $V(f)$ reducible as an
            algebraic set in $\A^2(\R)$.
    \end{enumerate}
\end{example}

\begin{example}\label{example_1.10}
    \begin{enumerate}
        \item[(1)] Let $k$ be a finite field. Then very subset of $\A^n(k)$ is
            an algebraic set, and so we can partition $\A^n(k)=X \cup Y$, where
            $X$ and $Y$ are algebraic, and  $X \neq \A^n(k)$ and $Y \neq
            \A^n(k)$. This makes $\A^n(k)$ reducible. One can also observe that
            the ideal $(0)$ is not prime in $k[x_1, \dots, x_n]$ for finite
            fields.

        \item[(2)] Now, suppose that $k$ is an infinite field. Then the ideal
            $(0)$ is prime in $k[x_1, \dots, x_n]$, so that $I(\A^n(k))=(0)$ is
            prime. This makes $\A^n(k)$ irreducible.
    \end{enumerate}
\end{example}

\begin{lemma}\label{1.5.2}
    Any collection of algebraic sets has a minimal member.
\end{lemma}
\begin{proof}
    If $\{X_\a\}$ is a collection of algebraic sets in $\A^n$, then by theorem
    \ref{1.4.2} the collection of ideals $\{I(X_\a)\}$ has a maximal member.
    Choose such a maximal member $I(X_{\a_0})$, then the corresponding algebraic
    set $X_{\a_0}$ is a minimal member of the collection $\{X_\a\}$.
\end{proof}

\begin{theorem}\label{1.5.3}
    Any algebraic set can be uniquely expressed as the disjoint union of
    irreducible algebraic sets. That is; for any algebraic set $X \subseteq
    \A^n$, there exist unique pairwise disjoint $X_1, \dots, X_m \subseteq \A^n$
    for which
    \begin{equation*}
        X=X_1 \cup \dots \cup X_m
    \end{equation*}
\end{theorem}
\begin{proof}
    We first show that such a decomposition exists for every algebraic set in
    $\A^n$. Let  $\Sc$ be the collection of all algebraic sets which cannot be
    expressed as a (not necessarily disjoint) union of (not necessarily unique)
    irreducible algebraic sets. Let $X$ be a minimal element of $\Sc$. Then
    $X$ is not irreducible. Hence there exist $X_1, X_2 \subseteq \A^n$ for
    which $X=X_1 \cup X_2$; suppose further that $X_1,X_2 \subseteq X$. By the
    minimality of $X$,  $X_1, X_2 \not\in \Sc$, so that
    \begin{equation*}
        X_i=\bigcup_{j=1}^{m_i}{X_{ij}}
    \end{equation*}
    where each $X_{ij}$ is irreducible. This makes
    \begin{equation*}
        X=\bigcup_{i=1,j=1}^{m,m_i}{X_{ij}}
    \end{equation*}
    which contradicts that $X \in \Sc$. Therefore $\Sc$ must be empty, and every
    algebraic set can be expressed as the union of irreducible algebraic sets.

    Now, take  $X=X_1 \cup \dots X_m$, where each $X_i$ is irreducible, and
    discard all those $X_i$ for which  $X_i \subseteq X_j$ for all $i \neq j$.
    This makes $X$ a disjoint union. Now, suppose that  $X=Y_1 \cup \dots Y_r$.
    Then
    \begin{equation*}
        X_i=\bigcup_{j=1}^r{(Y_j \cap X_i)}
    \end{equation*}
    so that $X_i \subseteq Y_j$ for some $j$. Similarly, we get that  $Y_j
    \subseteq X_k$ for some $k$. Thus $X_i \subseteq X_k$, but since $X$ is
    already a disjoint union, this makes  $i=k$ so that $X_i=Y_j$ and $m=r$.
    Thus the decomposition of $X$ into mutually disjoint irreducible algebraic
    sets is unique.
\end{proof}
\begin{corollary}
    If $X,Y \in \A^n$ are algebraic such that $X \subseteq Y$, then each
    irreducible component of $X$ is contained in an irreducible component of
    $Y$.
\end{corollary}
\begin{proof}
    Let
    \begin{align*}
        X   &= \bigcup_{i=1}^m{X_i} \\
        Y   &= \bigcup_{j=1}^n{Y_j} \\
    \end{align*}
    where $X_i$ and $Y_j$ are the irreducible components of  $X$ and  $Y$,
    respectively for all  $1 \leq i \leq m$ and $1 \leq j \leq n$. Now, consider
    the collection $\Sc$ of all irreducible components of $X$ not contained in
    an irreducible component of $Y$ Then $\Sc$ has a minimal member  $X'$, such
    that for every $1 \leq j \leq n$, $X' \not\subseteq Y_j$, then $X'
    \not\subseteq Y$. Now, since $X'$ is an irreducible component, we get  $X'
    \subseteq X \subseteq Y$, which is a contradicion. This makes $\Sc$ the
    emptyset.
\end{proof}

\begin{definition}
    Let $k$ be a field, and $X \subseteq \A^n(k)$ an algebraic set. Let $X=X_1
    \cup \dots X_m$ the decomposition of $X$ into the union of pairwise disjoint
    irreducible algebraic sets. We call each $X_i$ an \textbf{irreducible
    component} of $X$.
\end{definition}
