\section{Hilbert's Nullstellensatz}

\begin{theorem}[The Weak Nullstellensatz]\label{theorem_1.7.1}
    Let $k$ be an algebraically closed field, then if $\af$ is a proper ideal of
     $k[x_1, \dots, x_n]$, $V(\af)$ is nonempty.
\end{theorem}
\begin{proof}
    Since any proper ideal is contained in a maximal ideal, and algebraic sets
    of maximal ideals are minimal, suppose, without loss of generality, that
    $\af$ is a maximal ideal in $k[x_1, \dots, x_n]$. Take $l=\faktor{k[x_1,
    \dots x_n]}{\af}$. Since $\af$ is maximal, $l$ is a field, and $k \subseteq
    l$ is a subfield. Now, suppose that $k=l$, then for every  $i$, there is an
    element  $a_i \in k$ such that $x_i-a_i \in \af$. But $(x_1-a_1, \dots,
    x_n-a_n)$ is a maximal ideal, so that $\af=(x_1-a_1, \dots, x_n-a_n)$ and
    $V(\af)=V(x_1-a_1, \dots, x_n-a_n)$ which is nonempty.
\end{proof}

\begin{lemma}[Zariski]\label{lemma_1.7.2}
  Let $l$ be a field, ring-finite over a subfield $k$. Then  $l$ is
  module-finite over $k$; in particular,  $l$ is algebraic over  $k$.
\end{lemma}
\begin{proof}
  Let $v \in l$, and take $l=k(v)$. Consider the map $\phi:k[x] \xrightarrow{}
  l$ taking $x \xrightarrow{} v$, and let $\ker{\phi}=(f)$, $f$ irreducible over
   $k$. Then $\faktor{k[x]}{(f)} \simeq k[v]$, and $(f)$ is a prime ideal. Now, if
   $f=0$, then $k[v] \simeq k(v)$, which makes $l=k(x)$, so that $l$ is not
   ring-finite over  $k$; this is impossible by hypothesis. Then  $f \neq 0$,
   (and  supposing, without loss of generality, that $f$ is monic), since $(f)$
   is prime and $f$ is irreducible, then $(f)$ is maximal in $k[x]$. This makes
   $k[v]$ a field, and $k[v]=k(v)$. Moreover, $f(v)=0$ so $v$ is algebraic over
    $k$, and  $l=k[v]$, which makes $l$ module-finite over  $k$.

    Now, let  $l=k[v_1, \dots, v_{n+1}]$, and suppose the result holds for $n$.
    Let $k_1=k[v_1]$, then by induction on $n$  $l=k_1[v_2, \dots, v_{n+1}]$, is
    module-finite over $k_1$. Suppose that $v_1$ is not algebraic over $k_1$.
    Each $v_i$ satisfies  $v_i^{n_i}+a_{i1}v_i^{n_i-1}+\dots=0$, where $a_{i1}
    \in k_1$. Let $a \in k_1$, then $av_i^{n_i}+aa_{i1}v_i^{n_i-1}+\dots=0$.
    Then for any $z \in l$, there is an $m \in \Z^+$ for which $a^mz$ is an
    integral element over  $k_1$. In particular, we have $z \in k(v_1)$. Now,
    since $k(v_1) \simeq k(x)$, this is impossible, so that $l$ must be
    module-finite over  $k$.
\end{proof}

\begin{theorem}[Hilbert's Nullstellensatz]\label{theorem_1.7.3}
    Let $k$ be an algebraically closed field, and  $\af$ an ideal of  $k[x_1,
    \dots, x_n]$. Then $I(V(\af))=\sqrt{\af}$.
\end{theorem}
\begin{proof}
    We have that $\sqrt{\af} \subseteq I(V(\af))$. Now, let $\af=(f_1, \dots,
    f_r)$, since $k[x_1, \dots, x_n]$ is Noetherian, and let $g \in I(V(\af))$.
    Let $\bf=(f_1, \dots, f_r,x_{n+1}g-1)$ an ideal of $k[x_1, \dots,
    x_n,x_{n+1}]$. Then $\bf \subseteq \A^{n+1}(k)$ is empty, since $g$ vanishes
    whenever all $f_i=0$. By the weak nullstellensats, this puts  $1 \in \bf$ so
    that
    \begin{equation*}
        1=\sum{a_i(x_1, \dots, x_{n+1})f_i(x_1, \dots, x_{n+1})}+b(x_1, \dots,
        x_{n+1})(x_{n+1}g(x_1, \dots, x_n)-1)
    \end{equation*}
    Letting $y=\frac{1}{x_{n+1}}$, we get
    \begin{equation*}
        y^n=\sum{c_i(x_1, \dots, x_{n+1})f_i(x_1, \dots, x_{n+1})}+d(x_1, \dots,
        x_{n+1})(g(x_1, \dots, x_n)-y)
    \end{equation*}
    Substituting $g$ for $y$ gives us
    \begin{equation*}
        g^n=\sum{c_i(x_1, \dots, x_{n+1})f_i(x_1, \dots, x_{n+1})} \in \sqrt{\af}
    \end{equation*}
    so that $I(V(\af)) \subseteq \sqrt{\af}$, and we are done.
\end{proof}
\begin{corollary}
    If $\af$ is a radical ideal in  $k[x_1, \dots, x_n]$, then $I(V(\af))=\af$.
\end{corollary}
\begin{corollary}
    If $\af$ is a prime ideal in $k[x_1, \dots, x_n]$, the $V(\af)$ is
    irreducible. Moreover, there is a 1--1 correspondence between prime ideals
    of $k[x_1, \dots, x_n]$ and irreducible sets of $\A^n(k)$; where maximal
    ideals correspond to points.
\end{corollary}
\begin{corollary}
    For any $f \in k[x_1, \dots, x_n]$, if $f=f_1^{r_1} \dots f_m^{m_r}$ is the
    decomposition of $f$ into irreducible factors, then  $V(f)=V(f_1) \cup \dots
    V(f_m)$ is the decomposition of $f$ into irreducible components, and
    $I(V(f))=(f_1 \dots f_m)$ Moreover, there is a 1--1 correspondence between irreducible
    polynomials of  $k[x_1, \dots, x_n]$, and irreducible hypersurfaces of
    $\A^n(k)$.
\end{corollary}
\begin{corollary}
    If $\af$ is an ideal of  $k[x_1, \dots, x_n]$, then $V(\af)$ is finite, if,
    and only if $\faktor{k[x_1,\dots,x_n]}{\af}$ is a finite dimensional vector
    space over $k$, with
    \begin{equation*}
        |V(\af)| \leq \dim{\faktor{k[x_1,\dots,x_n]}{\af}}
    \end{equation*}
\end{corollary}
\begin{proof}
    Suppose that $l=\faktor{k[x_1,\dots,x_n]}{\af}$ is a finite dimensional
    vector space over $k$, and let  $P_1, \dots, P_n \in V(\af)$. Choose
    polynomials $f_1, \dots, f_m \in k[x_1, \dots, x_n]$ such that $f_i(P_j)=0$
    for any $i \neq j$, and $f_i(P_i)=1$. Consider now the residue classes of
    $f_i$ in $\af$, if
    \begin{equation*}
        \sum{\l_i(f_i+\af)}=0 \text{ where } \l \in k
    \end{equation*}
    We have $\sum{\l_if_i} \in \af$, so that
    \begin{equation*}
        \l_i=(\sum{\l_if_i})(P_i)=0
    \end{equation*}
    which makes the collection of vectors $\{f_i+\af\}$ linearly independent in
    $l$. This makes $V(\af)$ finite with $|V(\af)|=m \leq \dim{l}$.

    Conversely, suppose that $V(\af)$ is finite, and that $V(\af)=\{P_1, \dots,
    P_m\}$, where $P_i=(a_{i1}, \dots, a_{im})$, and
    \begin{equation*}
        f_j(x_1, \dots, x_n)=\prod_{j=1}^m{(x_j-a_{ij})} \text{ in } k[x_1,
        \dots, x_n]
    \end{equation*}
    Then $f_j \in I(V(\af))$, so that $f^N \in \af$ for some $N \in \Z^+$. Taking
    $N$ large enough and $(f_j+\af)^N=0$, we get $(x_j+\af)^N$ is a $k$-linear
    combination of the vectors $\{1, x_1+\af, \dots, (x_n+\af)^{rN-1}\}$. By
    induction, we see that for any $s \geq N$, $(x_j+\af)^s$ is a $k$-linear
    combination of the vectors $\{1+\af, \dots, (x_j+\af)^{rN-1}\}$. Then the
    collection of vectors $\{(x_1+\af)^m_1, \dots, (x_n+\af)^{m_n}\}$, with
    $m_i<rN$ generates $\faktor{k[x_1,\dots,x_n]}{\af}$ as a finite dimensional
    vector space over $k$.
\end{proof}

\begin{example}\label{example_1.6}
  \begin{enumerate}
    \item[(1)] Consider when $k$ is not an algebraically closed field; in
      particular,  $k=\R$, and let  $f(x,y)=x^2-y+1$. We have then that $(f)$ is
      a proper ideal of $\R[x,y]$, however, $V(f)=\emptyset$ since the only
      solutions to the equation $y=x^2+1$ are $(i,0)$ and $(-i,0)$, where
      $i^2=-1$. That is, $V(f)=V(x-i) \cup V(x+i) \not\subseteq \A^2(\R)$. This
      shows that  the weak nullstellensatz does not hold in general for
      non-algebraically closed fields. In consequences, niether will Hilbert's
      nullstellensatz, and any of its subsequent corollaries.

    \item[(2)] Let $f(x,y,z)=x^2+y^2-1$ and $g(x,y,z)=x^2-z^2-1$ in $V(f,g)$.
      Suppose that $P=(x,y,z) \in V(f,g)$, then we get $x^2+y^2-1=x^2-z^2-1$, so
      that  $y^2+z^2=(y-iz)(y+iz)=0$ and $i^2=-1$. That is  $P \in (V(y-iz) \cup
      V(y+iz)) \cap V(f)=V(f,y-iz) \cup V(f,y+iz)$. Conversely, if $P=(x,y,z)$
      is such that $x^2+y^2-1=0$ and either  $y-iz=0$ or $y+iz=0$, then we get
      that  $x^2+y^2-1=y^2+z^2$, so that $x^2-z^2-1=0$ and  $P \in V(f,g)$. That
      is,
      \begin{equation*}
        V(x^2+y^2-1,x^2-z^2-1)=V(x^2+y^2-1,y-iz) \cup V(x^2+y^2-1,y+iz)
      \end{equation*}
      Lastly, observe that $\faktor{\C[x,y,z]}{(f,y\pm{iz})} \simeq
      \faktor{\C[x,y]}{(f)}$, and since $f$ is prime in $\C[x,y,z]$ we get
      that $V(f,y\pm{iz})$ is irreducible.

    \item[(3)] Consider $V=V(x^2-y,x^3-z)=\{(t,t^2,t^3) \in \A^3(\C) : t \in
      \C\}$. Observe that $\faktor{\C[x,y,z]}{(x^2-y,x^3-z)} \simeq \C[t]$ via
      the map $x \xrightarrow{} t$, $y \xrightarrow{} t^2$, $z \xrightarrow{}
      z^3$. This makes $(x^2-y,x^3-z)$ prime in $\C[x,y,z]$, and hence
      $V(x^2-y,x^3-z)$ is irreducible. Moreover, by Hilbert's Nullstellensatz,
      and that fact that prime ideals are radical, we have
      $I(V)=\sqrt{(x^2-y,x^3-z)}=(x^2-y,x^3-z)$.

    \item[(4)] For any $\l \in k$, $k$ an algebraically closed field, the
      algebraic set $V(y^2-x(x-1)(x-\l))$ is irreducible. Observe that
      $y^2-x(x-1)(x-\l)$ is a monic polynomial of degree $2$ over $k[x]$, and
      moreover since $x(x-1)(x-\l)$ has degree $3$ over $k$, $y^2-x(x-1)(x-\l)$
      is irreducible over $k$.

    \item[(5)] Let $\af=(y^2-x^2,y^2+x^2)$ in $\C[x,y]$ and let $P=(X,Y)$ such
      that $Y^2-X^2=0$ and  $Y^2+X^2=0$. Then  $Y=\pm{X}, \pm{iX}$. Then
      $V(\af)=\{(X,X), (X,-X), (X,iX), (X,-iX)\}$, and $|V(\af)|=4$. This makes
      $\faktor{\C[x,y]}{\af}$ a finite dimensional vector space of dimension at
      most $\dim{\faktor{\C[x,y]}{\af}}=4$. Moreover, since
      $\deg{(y^2-x^2)}=\deg{(y^2+x^2)}=2$, we have that $4 \leq
      \dim{\faktor{\C[x,y]}{\af}}$. This makes $\faktor{\C[x,y]}{\af}$ a vector
      space of dimension exactly $4$ over $\C$.
  \end{enumerate}
\end{example}
