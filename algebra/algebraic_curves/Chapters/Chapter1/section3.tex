\section{Ideals of Algebraic Sets}\label{section_1.3}

\begin{lemma}\label{1.3.1}
    Let $k$ be a field, and  $X \times \A^n(k)$. Consider the set $I(X)=\{f \in
    k[x_1, \dots, x_n] : f(P)=0 \text{ for all } P \in X\}$. Then $I(X)$ forms
    an ideal of $k[x_1, \dots, x_n]$.
\end{lemma}
\begin{proof}
    Let $f,g \in I(X)$. Then for all $P \in X$,  $f(P)=0$, and $g(P)=0$, so that
    $f+g(P)=f(P)+g(P)=0$. Moreover, $-f(P)=0$ as well. So $I$ is a subgroup of
    $k[x_1, \dots, x_n]$ under addition. Now, take $f \in I(X)$ and $g \in
    k[x_1, \dots, x_n]$. Then $fg(P)=0$ for all $P \in X$ which makes  $I(X)$
    into an ideal.
\end{proof}

\begin{definition}
    Let $k$ be a field and  $X \subseteq \A^n(k)$. We define the \textbf{ideal}
    of $X$ to be the ideal $I(X)=\{f \in k[x_1, \dots, x_n] : f(P)=0 \text{ for
    all } P \in X\}$ of $k[x_1, \dots, x_n]$.
\end{definition}

\begin{lemma}\label{1.3.2}
    Let $k$ be a field. The following are true for all $X,Y \subseteq \A^n(k)$
    and for all $S \subseteq k[x_1, \dots, x_n]$.
    \begin{enumerate}
        \item[(1)] If $X \subseteq Y$, then $I(Y) \subseteq I(X)$.

        \item[(2)] $I(\emptyset)=k[x_1, \dots, x_n]$ and $I(\A^n(k))=(0)$.

        \item[(3)] $S \subseteq I(V(S))$ and $X \subseteq V(I(X))$.

        \item[(4)] $V(I(V(S)))=V(S)$ and $I(V(I(X)))=I(X)$.
    \end{enumerate}
\end{lemma}
\begin{proof}
    Let $X,Y \subseteq \A^n(k)$, wiht $X \subseteq Y$. Let $f \in I(Y)$, then
    for all $P \in Y$, $f(P)=0$. Now, since $P \in X$, we get for all $P \in X$
    $f(P)=0$ so that $f \in I(X)$.

    Observe now that the polynomial $1(x_1, \dots, x_n)=1$ has no points in
    $\A^n(k)$ as roots, so that $I(\emptyset)=k[x_1, \dots, x_n]$. Likewise, for
    the polynomial $0(x_1, \dots, x_n)=0$, every point in $\A^n(k)$ is a root,
    so that $I(\A^n(k))=(0)$.

    For the third assertion, let $S \subseteq k[x_1, \dots, x_n]$. If $f \in V(S)$,
    then for every  $P \in V(S)$, $f(P)=0$, by definition. This makes $S
    \subseteq I(V(S))$. Likewise, if $X \subseteq \A^n(k)$ and $P \in X$, then
    for all $f \in I(X)$, $f(P)=0$, so that $P \in V(I(X))$.

    Lastly, let $P \in V(S)$, and $f \in I(V(S))$. By definition, $f(P)=0$ so
    that $V(S) \subseteq V(I(V(S)))$. Conversely, let $P \in V(I(V(S)))$ then
    for every $f \in I(V(S))$, $f(P)=0$, which puts $P \in V(S)$ so that
    $V(I(V(S))) \subseteq V(S)$. Likewise, by similar reasoning we conclude that
    $I(V(I(X)))=I(X)$.
\end{proof}
\begin{corollary}
    If $k$ is an infinite field, then for any  $a_1, \dots, a_n \in k$, $I(a_1,
    \dots, a_n)=(x_1-a_1, \dots, x_n-a_n)$.
\end{corollary}
\begin{proof}
    Let $f \in I(a_1, \dots, a_n)$. Since $k$ is infinite, and  $f(a_1, \dots,
    a_n)=0$,
    \begin{equation*}
        f(x_1, \dots, x_n)=\sum{g_i(x_1, \dots, x_n)(x_i-a_i)}
    \end{equation*}
    so $f \in (x_1-a_1, \dots, x_n-a_n)$. Conversely, if $f \in (x_1-a_1, \dots,
    x_n-a_n)$, we observe that $f \in I(a_1, \dots, a_n)$.
\end{proof}

\begin{definition}
    Let $\af$ be an ideal of a ring  $R$. We define the  \textbf{radical} of
    $\af$
    to be the set
    \begin{equation*}
        \sqrt{\af}=\{a \in R : a^n \in \af, \text{ for some } n \in \Z^+\}
    \end{equation*}
    We call $I$ a \textbf{radical ideal} if $I=\sqrt{I}$.
\end{definition}

\begin{lemma}\label{1.3.3}
    Let $R$ be a ring, and $\af$ an ideal of $R$. Then $\sqrt{\af}$ is also an
    ideal of $R$.
\end{lemma}
\begin{proof}
    Let $a,b \in \sqrt{\af}$, then $a^m \in \af$ and $b^n \in \af$ for some $m,n
    \in \Z^+$. Now, observe that
    \begin{equation*}
        (a+b)^{m+n}=a^{m+n}+\sum_{i=1}^{m+n-2}{{m+n \choose
        i}a^{i}b^{m+n-i}}+b^{m+n}
    \end{equation*}
    Now, $a^{m+n}=a^ma^n \in \af$ and $b^{m+n}=b^nb^m \in \af$ by the ideal
    properties of $\af$. Moreover, notice if $i \geq n$, then $a^ib^{m+n-i} \in
    \af$; on the otherhand, if $m \leq m+n-i$, then $a^ib^{m-n-i} \in \af$. This
    makes each $a^ib^{m-n-i} \in \af$, and that $(a+b)^{m+n} \in \af$. Also
    observe that if $a^n \in \af$, then $(-a)^n=-(a^n) \in \af$. So that
    $\sqrt{\af}$ is an additive subgroup of $R$.

    Lastly, suppose that if $a \in \sqrt{R}$, and $r \in R$, then we have
    $(ra)^n=r^na^n \in \af$ for some $n \in \Z^+$. Thus $ra \in \sqrt{\af}$. This
    makes $\sqrt{\af}$ an ideal of $R$.
\end{proof}
\begin{corollary}
    $\sqrt{\af}$ is a radical ideal of $R$.
\end{corollary}
\begin{proof}
    Observe that $\sqrt{\af} \subseteq \sqrt{\sqrt{\af}}$. Now, let $a \in
    \sqrt{\sqrt{\af}}$, then $a^n \in \sqrt{\af}$ for some $n \in \Z^+$, so that
    $(a^n)^m=a^{mn} \in \af$ for some $m \in \Z^+$. This makes $a \in
    \sqrt{\af}$. So $\sqrt{\sqrt{\af}} \subseteq \sqrt{\af}$. This makes
    $\sqrt{\af}$ radical.
\end{proof}

\begin{lemma}\label{1.3.4}
    Any prime ideal in a ring $R$ is radical.
\end{lemma}
\begin{proof}
    Let $\pf$ be a prime ideal. We have that $\pf \subseteq \sqrt{\pf}$. Now, let
    $a \in \sqrt{\pf}$. Then for some $n \in \Z^+$, we have that $a^n \in \pf$.
    Since $\pf$ is prime, either $a \in \pf$ or $a^{n-1} \in \pf$. If $a \in
    \pf$, we are done; otherwise we have $a^{n-1}=aa^{n-2} \in \pf$. Repeating
    this process recursively, we obtain that $a \in \pf$, so that
    $\pf=\sqrt{\pf}$.
\end{proof}

\begin{lemma}\label{1.3.5}
    Let $k$ be a field, then for any $X \subseteq \A^n(k)$, $I(X)$ is a radical
    ideal.
\end{lemma}
\begin{proof}
    For any $f \in I(X)$, notice that $f^n(P)=f(f^{n-1}(P))=\dots=
    \underbrace{f(f(\dotsf(P)))}_{n\text{ times}}$
\end{proof}

\begin{example}\label{example_1.8}
    Observe that $\faktor{\R[x]}{(x^2+1)} \simeq \C$ is a field, so that
    $(x^2+1)$ is a maximal ideal, hence a prime ideal, and hence, a radical
    ideal. Observe also that $V(x^2+1)=\emptyset$, so that $I(V(x^2+1))=\R[x]$.
    Therefore, $(x^2+1)$ is not the ideal of any nonempty set of $\A^1(\R)$.
\end{example}

\begin{lemma}\label{1.3.6}
    If $X$ and $Y$ are algebraic sets in $\A^n(k)$, then $I(X)=I(Y)$ if, and
    only if $X=Y$.
\end{lemma}
\begin{proof}
    If $X=Y$, then we can observe that $I(X)=I(Y)$. Conversely, suppose that
    $I(X)=I(Y)$, and let $f \in I(X)$. Then for all $P \in X$, we have $f(P)=0$.
    Since $I(X)=I(Y)$, we must have that $P \in Y$ so that $X \subseteq Y$. In
    similar fashion, we get that $Y \subseteq X$.
\end{proof}

\begin{theorem}\label{1.3.7}
    Let $k$ be a field. The ideal $(x_1-a_1, \dots, x_n-a_n)$ of
    $k[x_1, \dots, x_n]$ is a maximal ideal of $k[x_1, \dots, x_n]$ and the
    natural map
    \begin{equation*}
        k \xrightarrow{} \faktor{k[x_1, \dots, x_n]}{(x_1-a_1, \dots, x_n-a_n)}
    \end{equation*}
    defines an isomorphism.
\end{theorem}
\begin{proof}
    Define the map $\phi:k[x_1, \dots, k_n] \xrightarrow{} k$ defined by the
    rule $f(x_1, \dots, x_n) \xrightarrow{} f(a_1, \dots, a_n)$ where $a_1,
    \dots, a_n \in k$. Then notice that $\ker{\phi}=(x_1-a_1, \dots ,x_n-a_n)$.
    Now, consider $f(x_1, \dots, x_n)=1+0x_1+\dots+0x_n \in k[x_1, \dots, x_n]$.
    Then $f(a_1,\dots,a_n)=1+0a_1+\dots+0a_n=1 \in \phi(k[x_1, \dots, x_n])$. So
    that $\phi$ is onto. By the first isomorphism theorem for ring
    homomorphisms, we get
    \begin{equation*}
        \faktor{k[x_1, \dots, x_n]}{(x_1-a_1, \dots, x_n-a_n)} \simeq k
    \end{equation*}
    So that $(x_1-a_1, \dots, x_n-a_n)$ is a maximal ideal. Notice also that
    $\Phi=\pi \circ \phi$ where $\pi:k \xrightarrow{}
    \faktor{k[x_1, \dots, x_n]}{(x_1-a_1, \dots, x_n-a_n)}$ is the natural map.
    So $\pi$ defines the isomorphism.
\end{proof}
