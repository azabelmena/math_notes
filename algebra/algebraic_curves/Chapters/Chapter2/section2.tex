\section{Affine Coordinate Changes}

Let $\phi:\A^n(k) \xrightarrow{} \A^m(k)$ a polynomial map defined by
$\phi=(\phi_1, \dots, \phi_m)$, and let $F \in k[x_1, \dots, x_n]$. We denote
$F^\phi=\Phi(F)$, where $\Phi$ is the induced homomorphism of $\phi$. Let $V$ be
an affine variety, we denote $I^\phi=I^\phi(V)=I(V)^\phi$ to be the ideal
generated by all $F^\phi$, where  $F \in I(V)$. We denote
$V^\phi=V(I^\phi)=\inv{\Phi}(V)$.

\begin{example}\label{example_2.1}
  If $V$ is a ahypersurface of  $F$, then  $V^\phi$ is the hypersurface of
  $f^\phi$.
\end{example}

\begin{definition}
  An \textbf{affine change of coordinates} is a polynomial map
  $\phi:\A^n(k) \xrightarrow{} \A^n(k)$, $\phi=(\phi_1, \dots, \phi_n)$, for
  which $\phi$ is 1--1 and onto, and each component polynomial $\phi_i$ has
  total degree  $1$.
\end{definition}

\begin{lemma}\label{lemma_2.2.1}
  Let $\phi:\A^n(k) \xrightarrow{} \A^n(k)$ be an affine change of coordinates.
  Then each component polynomial of $\phi$ has the form:
  \begin{equation}\label{equation_2.1}
    \phi_i(x_1, \dots, x_n)=\sum_{j}{a_{ij}x_j}+a_{i0}
  \end{equation}
  Moreover, $\phi=\phi'' \circ \phi'$ where $\phi'$ is a linear map, and
  $\phi''$ is a translation.
\end{lemma}
\begin{proof}
  That each $\phi_i$ has the form of equation \ref{equation_2.1} follows from
  definition. Since $\deg{\phi_i}=1$, it can only be the sum of monomial terms
  of degree $1$, plus a possible constant. Moreover, define
  \begin{align*}
    \phi_i'(x_1, \dots, x_n)  &=  \sum_{j}{a_{ij}x_j} \\
    \phi_i''(x_1, \dots, x_n) &=  x_j+a_{i0} \\
  \end{align*}
  Then the maps $\phi'=(\phi_1', \dots, \phi_n')$ and $\phi''=(\phi_1'', \dots,
  \phi_n'')$ are polynomial maps, and $\phi=\phi'' \circ \phi'$.
\end{proof}

\begin{lemma}\label{lemma_2.2.2}
  Let $\phi:\A^n(k) \xrightarrow{} \A^n(k)$ be a polynomial map with the form
  $\phi=\phi'' \circ \phi'$, where $\phi'$ is a linear map, and  $\phi''$ is a
  translation. Then $\phi$ is an affine change of coordinates if, and only if
  $\phi'$ is invertible.
\end{lemma}
\begin{proof}
  Observe first that $\phi$ is the composition of linear polynomial maps, and
  hence must also be a linear polynomial map, that is, each  $\phi_i$ has total
  degree $1$. Moreover, since $\phi''$ is a translation, $\phi''$ is 1--1 and
  onto.

  Now, suppose that $\phi$ is 1--1 and onto, then  $\phi'' \circ \phi'$ is 1--1
  and onto, and since translations are invertible, $\phi'$ must be 1--1 and onto
  as well, that is, $\phi'$ is invertible. Conversely, if  $\phi'$ is
  invertible, we get that $\phi=\phi'' \circ \phi'$ is invertible, and hence
  1--1 and onto.
\end{proof}

\begin{theorem}\label{theorem_2.2.3}
  Let $\phi$ and $\psi$ be affine changes of coordinates. Then $\phi \circ \psi$
  and $\inv{\phi}$ are also affine changes of coordinates. That is, the affine
  changes of coordinates on $\A^n(k)$ form a group under composition.
\end{theorem}
\begin{proof}
  Let $\phi=(\phi_1, \dots, \phi_n)$ and $\psi=(\psi_1, \dots, \psi_n)$. Then
  for each $1 \leq i \leq m$, $\deg{\phi_i}=1$ and $\deg{\psi_i}=1$, hence
  $\deg{\phi_i \circ \psi_i}=1$, likewise, $\deg{\inv{\phi}}=1$. Moreover, each
  $\phi \circ \psi$ is 1--1 and onto since $\phi$ (and hence $\inv{\phi}$) and
  $\psi$ are 1--1 and onto.

  Now, we also have $\phi=\phi'' \circ \phi'$ and $\psi=\psi'' \circ \psi'$,
  where $\phi'$ and $\psi'$ are linear maps, and $\phi''$ and  $\psi''$ are
  translations maps. We show that $\phi \circ \psi=\xi'' \circ \xi'$. Wherer
  $\xi'$ is a linear map and  $\xi''$ is a translation. Let $T_i'$,  $T_i''$,
  $S_i'$,  $S_i''$ correspond to $\phi_i$ and  $\psi_i$ respectively. Then we
  have $S_i'' \circ S_i'=\sum_{j}{b_{ij}x_j}+b_{i0}$ so that:
  \begin{equation*}
    (T_i'' \circ T_i) \circ (S_i'' \circ S_i)=
    \sum_{j}{a_{ij}(\sum_{j}{b_{ij}x_j}+b_{i0})}+a_{i0}=
    \sum_j{a_{ij}b_{ij}x_j}+(\sum_{j}{a_{ij}b_{i0}}+a_{i0})
  \end{equation*}
  so that $(T_i'' \circ T_i) \circ (S_i'' \circ S_i)=\sum_{j}{c_{ij}x_j}+c_{i0}$
  so that $\phi \circ \psi$ has that required form. Lastly if
  $\phi_i=\sum_{j}{a_{ij}x_j}+a_{i0}$, then
  $\inv{\phi}_i=\sum_{j}{\inv{a_{ij}}y_j}-\sum_j{\inv{a_{ij}}a_{i0}}$. Lastly,
  observe that the set of affine coordinate changes on $\A^n(k)$ is a non-empty
  subset of bijections of the permutation group on $\A^n(k)$ so the group
  structure holds (i.e. the affine coordinate changes form a subgroup).
\end{proof}
\begin{corollary}
  If $\phi:\A^n(k) \xrightarrow{} \A^n(k)$ is an affine change of coordinate,
  then $\phi$ is an isomorphism between $\A^n(k)$ and itself.
\end{corollary}
\begin{proof}
  By the group structure, we can compute $\phi \circ \inv{\phi}=\id_{\A^n(k)}$
  and $\inv{\phi} \circ \phi=\id_{\A^n(k)}$.
\end{proof}

Now, let $\phi=(\phi_1, \dots, \phi_n)$ be an affine coordinate change on
$\A^n(k)$. Then each $\phi_i$ has the form of equation \ref{equation_2.1}, and
so $\phi$ describes a system of linear equations in  $\A^n(k)$. Therefore, we
can represent $\phi$ as
\begin{equation}\label{equation_2.2}
  \phi(x_1, \dots, x_n)=
  \begin{pmatrix}
    a_{11}  & \dots   & a_{1n} \\
    a_{21}  & \dots   & a_{2n} \\
    \vdots  & \ddots  & \vdots  \\
    a_{n1}  & \dots   & a_{nn} \\
  \end{pmatrix}
  \begin{pmatrix}
    x_1 \\
    x_2 \\
    \vdots  \\
    x_n
  \end{pmatrix}+
  \begin{pmatrix}
    b_{10}  \\
    b_{20}  \\
    \vdots  \\
    b_{n0}
  \end{pmatrix}
\end{equation}

so that

\begin{equation*}
  \phi(x)=Ax+B
\end{equation*}

where $x=(x_1, \dots, x_n)$ and $A$ and $B$ are the $n \times n$ and $n \times
1$ matrices in the matrix spaces $k^{n \times n}$ and $k^{n \times 1}$.

\begin{lemma}\label{lemma_2.2.4}
  Let $\phi:\A^n(k) \xrightarrow{} \A^n(k)$ be a polynomial map of the form
  $\phi(x)=Ax+B$, where $A$ and $B$ are matrices in $k^{n \times n}$ and $k^{n
  \times 1}$ respectively. Then $\phi$ is an affine change of coordinates if,
  and only if $A$ is nonsingular.
\end{lemma}
\begin{proof}
  This follows from lemma \ref{lemma_2.2.2} and putting $\phi'$ and $\phi''$
  into the forms $\phi'(x)=Ax$ and  $\phi''(x)=x+B$.
\end{proof}

\begin{definition}
  We call an affine variety $V \subseteq \A^n(k)$ a \textbf{linear subvariety}
  if $V=V(F_1, \dots, F_r)$ and $\deg{F_i}=1$ for all $1 \leq i \leq r$.
\end{definition}

\begin{lemma}\label{lemma_2.2.4}
  Let $V \subseteq \A^n(k)$ be a linear subvariety, and $\phi$ an affine change
  of coordinates on $\A^n(k)$. Then $V^\phi$ is also a linear subvariety of
  $\A^n(k)$.
\end{lemma}
\begin{proof}
  Let $V=V(F_1, \dots, F_r)$ where $\deg{F_i}=1$ for all $1 \leq i \leq r$. Let
  $\phi(x)=Ax+B$ an affine coordinate change, with $A \in k^{n \times n}$ and
  $B \in k^{n \times 1}$. Then by theorem \ref{theorem_2.2.3}, $\inv{\phi}$ is
  also an affine coordinate change of the form $\inv{\phi}=\inv{A}y-\inv{A}B$.
  Define then the polynomials $G_1, \dots, G_r$ by
  \begin{equation*}
    G_i=F_i^{\inv{\phi}}
  \end{equation*}
  then $\deg{G_i}=\deg{F_i}=1$, for all $1 \leq i \leq r$, and
  \begin{equation*}
    G_i^{\phi}(P)=(F^{\inv{\phi}})^{\phi}(P)=F_i^{\inv{\phi}\phi}(P)=F_i(P)=0
    \text{ for all } P \in V
  \end{equation*}
  So that $V^\phi=V(G_1, \dots, G_r)=V(F_1^{\inv{\phi}}, \dots,
  F_r^{\inv{\phi}})$.
\end{proof}
\begin{corollary}
  If $V$ is non-empty, then there exists an affine change of coordinates
  $\phi:\A^n(k) \xrightarrow{} \A^n(k)$ such that $V^\phi=V(x_{m+1}, \dots,
  x_n)$ for some $m<n$. Moreover, such an $m$ is independent of the choice of
  $\phi$.
\end{corollary}
%\begin{proof}
  %Consider $V$ as a subspace of the vector space $\A^n(k)$, and let $\dim{V}=m$.
  %Now, observe that $V=V(F_1, \dots, F_r)$ where $\deg{F_i}=1$ for all $1 \leq i
  %\leq r$, by definition. Then $\{F_1, \dots, F_r\}$ is a basis for $V$, so that
  %$r=m$. Let $\{x_1, \dots, x_m\}$ be linearly independent in $V$, then
  %$F_i(x)=0$ for any linear combination $x$ of $\{x_1, \dots, x_m\}$. Now, let
  %$\phi:\A^n(k) \xrightarrow{} \A^n(k)$ be a polynomial map of the form
  %$\phi(x)=Ax+B$, where $A$ is the  $n \times n$ matrix constructed as:
  %\begin{equation*}
    %A=
    %\begin{pmatrix}
      %a_{11}        & \dots   & a_{1n}  \\
      %\vdots        & \ddots  & \vdots  \\
      %a_{m1}        & \dots   & a_{mn}  \\
      %a_{(m+1)1}    & \dots   & a_{(m+1)n}  \\
      %\vdots        & \ddots  & \vdots  \\
      %a_{n1}        & \dots   & a_{nn}  \\
    %\end{pmatrix}
  %\end{equation*}
  %Where $a_{ij}$ are the components of $x_i$ when  $1 \leq i \leq m$, and
  %$a_{ij}$ are the coefficients of $F_{n-i+1}$ when $m+1 \leq i \leq n$, for all
  %$1 \leq j \leq n$. Then $A$ consists of linearly independent row vectors, and
  %hence $A$ is nonsingular. This makes $\phi$ an affine coordinate change.
  %Moreover, for all $1 \leq i \leq m$, we get $F_{n-i+1}(x_i)=0$, so that for
  %any $(a_1, \dots, a_n) \in \A^n(k)$
  %\begin{equation*}
    %\phi:(a_1, \dots, a_m,a_{m+1}, \dots, a_n) \xrightarrow{}
    %(0, \dots, 0,a_{m+1}, \dots, a_n)
  %\end{equation*}
  %so that $V^\phi=V(x_{m+1}, \dots, x_n)$.

  %Now, suppose that $m<s$, and that there is an affine coordinate change
  %$\phi(x)=Ax+B$ for which $V(x_{m+1}, \dots, x_n)=V(x_{s+1}, \dots, x_n)$. Then
  %for every $(0, \dots, 0,a_{m+1}, \dots, a_n)$:
  %\begin{equation*}
    %\phi:(0, \dots, 0,a_{m+1}, \dots, a_n) \xrightarrow{}
    %(0, \dots, 0,a_{s+1}, \dots, a_n)
  %\end{equation*}
  %so that $\phi:(0, \dots, 0, a_{m+1}-a_{s+1}, 0 \dots, 0) \xrightarrow{} 0$, so
  %that $\ker{\phi}=\ker{A} \neq (0)$, so that $A$ is non-singular. This makes
  %the set $\{\phi_{m+1}, \dots, \phi_n\}$ linearly dependent. However, since
  %$V(x_{s+1}, \dots, x_n)=V(x_{m+1}^\phi, \dots, x_n^\phi)$ generates $\A^n(k)$
  %as a vector space, $\{\phi_1, \dots, \phi_n\}$ must be a basis, which is a
  %contradiction. This forces $m \leq s$, and hence $m<s \leq s$ so that $m=s$.
%\end{proof}
\begin{corollary}
  $V$ is isomorphic as a variety to  $\A^m(k)$.
\end{corollary}

\begin{definition}
  Let $V \subseteq \A^n(k)$ a non-empty affine variety. We define the
  \textbf{dimension} of $V$ to be the integer $m<n$ such that there is an affine
  change of coordinates $\phi$ on $\A^n(k)$ for which $V^\phi=V(x_{m+1}, \dots,
  x_n)$.
\end{definition}

\begin{example}\label{example_2.4}
  Let $P=(a_1, \dots, a_n)$ and $Q=(b_1, \dots, b_n)$ distinct points of
  $\A^n(k)$. We define the \textbf{line} throug $P$ and $Q$ to be the set
  \begin{equation*}
    \bar{PQ}=\{(a_1+t(b_1-a_1), \dots, a_n+t(b_1-a_n)) : t \in k\}
  \end{equation*}
  The following are true:
  \begin{enumerate}
    \item[(1)] If $\phi$ is an affine coordinate change, then
      $\phi(\bar{PQ})=\bar{\phi(P)\phi(Q)}$ is the line through $\phi(P)$ and
      $\phi(Q)$.

    %\item[(2)] Lines are linear subvarieties of dimension $1$, and any linear
      %subvariety is isomorphic to the line through any two of its points.

    %\item[(3)] In $\A^2(k)$, a line is isomorphic to a hyperplane. And if $P,P'
      %\in \A^2(k)$ and $L_1$ and $L_2$ are two distinct lines through $P$, and
      %$L_1'$ and $L_2'$ are two distinct lines through $P'$, then there is an
      %affine coordinate change  $\phi$ on  $\A^2(k)$ for which $\phi(P)=P'$, and
      %$\p(L_i)=L_i'$ for $i=1,2$.
  \end{enumerate}
\end{example}

%\begin{example}\label{example_2.5}
  %Consider $\A^n(\C)=\C^n$ with the usual topology obtained by identifying
  %$\C^n$ with  $\R^{2n}$. We call a set $S \subseteq \C^n$
  %\textbf{path-connected} if for any two points $P,Q \in S$, there exists a
  %continuous mapping $\y:[0,1] \xrightarrow{} S$ for which $\y(0)=P$ and
  %$\y(1)=Q$. Then the following are true:
  %\begin{enumerate}
    %\item[(1)] $\com{\C}{S}$ is path connected for any finite set $S$.

    %\item[(2)] If $V \subseteq \C^n$ is an algebraic set, then $\com{\C^n}{V}$
      %is path-connected. Indeed, let $\bar{PQ}$ be the line through $P$ and  $Q$
      %in $\C^n$. Then observe that $\bar{PQ} \cap V$ is finite, and $\bar{PQ}
      %\simeq \C$.
  %\end{enumerate}
%\end{example}
