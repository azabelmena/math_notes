\section{Affine Coordinate Changes}

Let $\phi:\A^n(k) \xrightarrow{} \A^m(k)$ a polynomial map defined by
$\phi=(\phi_1, \dots, \phi_m)$, and let $F \in k[x_1, \dots, x_n]$. We denote
$F^\phi=\Phi(F)$, where $\Phi$ is the induced homomorphism of $\phi$. Let $V$ be
an affine variety, we denote $I^\phi=I^\phi(V)=I(V)^\phi$ to be the ideal
generated by all $F^\phi$, where  $F \in I(V)$. We denote
$V^\phi=V(I^\phi)=\inv{\Phi}(V)$.

\begin{example}\label{example_2.1}
  If $V$ is a ahypersurface of  $F$, then  $V^\phi$ is the hypersurface of
  $f^\phi$.
\end{example}

\begin{definition}
  An \textbf{affine change of coordinates} is a polynomial map
  $\phi:\A^n(k) \xrightarrow{} \A^n(k)$, $\phi=(\phi_1, \dots, \phi_n)$, for
  which $\phi$ is 1--1 and onto, and each component polynomial $\phi_i$ has
  total degree  $1$.
\end{definition}

\begin{lemma}\label{lemma_2.2.1}
  Let $\phi:\A^n(k) \xrightarrow{} \A^n(k)$ be an affine change of coordinates.
  Then each component polynomial of $\phi$ has the form:
  \begin{equation}\label{equation_2.1}
    \phi_i(P)=\sum_{j}{a_{ij}x_j}+a_{i0} \text{ where } P \in \A^n(k)
  \end{equation}
  Moreover, $\phi=\phi'' \circ \phi'$ where $\phi'$ is a linear map, and
  $\phi''$ is a translation.
\end{lemma}
\begin{proof}
  That each $\phi_i$ has the form of equation \ref{equation_2.1} follows from
  definition. Since $\deg{\phi_i}=1$, it can only be the sum of monomial terms
  of degree $1$, plus a possible constant. Moreover, define
  \begin{align*}
    \phi_i'(P)  &=  \sum_{j}{a_{ij}x_j} \\
    \phi_i''(P) &=  x_j+a_{i0} \\
  \end{align*}
  Then the maps $\phi'=(\phi_1', \dots, \phi_n')$ and $\phi''=(\phi_1'', \dots,
  \phi_n'')$ are polynomial maps, and $\phi=\phi'' \circ \phi'$.
\end{proof}

\begin{lemma}\label{lemma_2.2.2}
  Let $\phi:\A^n(k) \xrightarrow{} \A^n(k)$ be a polynomial map with the form
  $\phi=\phi'' \circ \phi'$, where $\phi'$ is a linear map, and  $\phi''$ is a
  translation. Then $\phi$ is an affine change of coordinates if, and only if
  $\phi'$ is invertible.
\end{lemma}
\begin{proof}
  Observe first that $\phi$ is the composition of linear polynomial maps, and
  hence must also be a linear polynomial map, that is, each  $\phi_i$ has total
  degree $1$. Moreover, since $\phi''$ is a translation, $\phi''$ is 1--1 and
  onto.

  Now, suppose that $\phi$ is 1--1 and onto, then  $\phi'' \circ \phi'$ is 1--1
  and onto, and since translations are invertible, $\phi'$ must be 1--1 and onto
  as well, that is, $\phi'$ is invertible. Conversely, if  $\phi'$ is
  invertible, we get that $\phi=\phi'' \circ \phi'$ is invertible, and hence
  1--1 and onto.
\end{proof}

\begin{theorem}\label{theorem_2.2.3}
  Let $\phi$ and $\psi$ be affine changes of coordinates. Then $\phi \circ \psi$
  and $\inv{\phi}$ are also affine changes of coordinates. That is, the affine
  changes of coordinates on $\A^n(k)$ form a group under composition.
\end{theorem}
\begin{proof}
  Let $\phi=(\phi_1, \dots, \phi_n)$ and $\psi=(\psi_1, \dots, \psi_n)$. Then
  for each $1 \leq i \leq m$, $\deg{\phi_i}=1$ and $\deg{\psi_i}=1$, hence
  $\deg{\phi_i \circ \psi_i}=1$, likewise, $\deg{\inv{\phi}}=1$. Moreover, each
  $\phi \circ \psi$ is 1--1 and onto since $\phi$ (and hence $\inv{\phi}$) and
  $\psi$ are 1--1 and onto.

  Now, we also have $\phi=\phi'' \circ \phi'$ and $\psi=\psi'' \circ \psi'$,
  where $\phi'$ and $\psi'$ are linear maps, and $\phi''$ and  $\psi''$ are
  translations maps. We show that $\phi \circ \psi=\xi'' \circ \xi'$. Wherer
  $\xi'$ is a linear map and  $\xi''$ is a translation. Let $T_i'$,  $T_i''$,
  $S_i'$,  $S_i''$ correspond to $\phi_i$ and  $\psi_i$ respectively. Then we
  have $S_i'' \circ S_i'=\sum_{j}{b_{ij}x_j}+b_{i0}$ so that:
  \begin{equation*}
    (T_i'' \circ T_i) \circ (S_i'' \circ S_i)=
    \sum_{j}{a_{ij}(\sum_{j}{b_{ij}x_j}+b_{i0})}+a_{i0}=
    \sum_j{a_{ij}b_{ij}x_j}+(\sum_{j}{a_{ij}b_{i0}}+a_{i0})
  \end{equation*}
  so that $(T_i'' \circ T_i) \circ (S_i'' \circ S_i)=\sum_{j}{c_{ij}x_j}+c_{i0}$
  so that $\phi \circ \psi$ has that required form. Lastly if
  $\phi_i=\sum_{j}{a_{ij}x_j}+a_{i0}$, then
  $\inv{\phi}_i=\sum_{j}{\inv{a_{ij}}y_j}-\sum_j{\inv{a_{ij}}a_{i0}}$. Lastly,
  observe that the set of affine coordinate changes on $\A^n(k)$ is a non-empty
  subset of bijections of the permutation group on $\A^n(k)$ so the group
  structure holds (i.e. the affine coordinate changes form a subgroup).
\end{proof}
\begin{corollary}
  If $\phi:\A^n(k) \xrightarrow{} \A^n(k)$ is an affine change of coordinate,
  then $\phi$ is an isomorphism between $\A^n(k)$ and itself.
\end{corollary}
\begin{proof}
  By the group structure, we can compute $\phi \circ \inv{\phi}=\id_{\A^n(k)}$
  and $\inv{\phi} \circ \phi=\id_{\A^n(k)}$.
\corolloryend{proof}

\begin{definition}
  We call an affine variety $V \subseteq \A^n(k)$ a \textbf{linear subvariety}
  if $V=V(F_1, \dots, F_r)$ and $\deg{F_i}=1$ for all $1 \leq i \leq r$.
\end{definition}

\begin{lemma}\label{lemma_2.2.4}
  Let $V \subseteq \A^n(k)$ be a linear subvariety, and $\phi$ an affine change
  of coordinates on $\A^n(k)$. Then $V^\phi$ is also a linear subvariety of
  $\A^n(k)$.
\end{lemma}
\begin{proof}
  Let $F_1, \dots, F_r \in k[x_1, \dots, x_n]$ all have total degree $1$, such
  that $V=V(F_1, \dots, F_r)$. Then each $F_i^\phi=F_i(\phi_1, \dots, \phi_n)$.
  Then for each $1 \leq i \leq r$, $\deg{F_i^\phi}=1$ and $V^\phi=V(I^\phi)=
  V(F_1^\phi, \dots, F_r^\phi)$.
\end{proof}
\begin{corollary}
  if $V$ is non-empty, then there exists an affine change of coordinates
  $\phi:\A^n(k) \xrightarrow{} \A^n(k)$ such that $V^\phi=V(x_{m+1}, \dots,
  x_n)$ for some $m<n$. Moreover, such an $m$ is independent of the choice of
  $\phi$.
\end{corollary}
\begin{corollary}
  $V$ is isomorphic as a variety to  $\A^m(k)$.
\end{corollary}

\begin{definition}
  Let $V \subseteq \A^n(k)$ a non-empty affine variety. We define the
  \textbf{dimension} of $V$ to be the integer $m<n$ such that there is an affine
  change of coordinates $\phi$ on $\A^n(k)$ for which $V^\phi=V(x_{m+1}, \dots,
  x_n)$.
\end{definition}

\begin{example}\label{example_2.2}
  Let $P=(a_1, \dots, a_n)$ and $Q=(b_1, \dots, b_n)$ distinct points of
  $\A^n(k)$. We define the \textbf{line} throug $P$ and $Q$ to be the set
  \begin{equation*}
    \bar{PQ}=\{(a_1+t(b_1-a_1), \dots, a_n+t(b_1-a_n)) : t \in k\}
  \end{equation*}
\end{example}
