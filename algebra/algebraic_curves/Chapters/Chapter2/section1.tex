\section{Coordinate Rings and Polynomial Maps}\label{section_2.1}

\begin{definition}
  We call an irreducible affine algebraic set over an algebraically closed field
  an \textbf{affine variety}.
\end{definition}

\begin{definition}
  Let $V \subseteq \A^n(k)$ an affine variety. We define the \textbf{coordinate
  ring} to be the quotient:
  \begin{equation*}
    \Oc(V)=\faktor{k[x_1, \dots, x_n]}{I(V)}
  \end{equation*}
\end{definition}

\begin{definition}
  Let $V \subseteq \A^n(k)$ be an affine variety, and let $\Fc(V,k)$ be the ring
  of all functions taking $V \xrightarrow{} k$. We call a function $f \in
  \Fc(V,k)$ a \textbf{polynomial function} if there is a polynomial $F \in
  k[x_1, \dots, x_n]$ for which $f(P)=F(P)$ for all $P \in V$. We denote the set
  of all polynomial functions by $\Pc(V,k)$.
\end{definition}

\begin{lemma}\label{lemma_2.1.1}
  Let $V$ be an affine variety. Then $\Pc(V,k)$ is a ring, in particular, it is
  a subring of $\Fc(V,k)$. Moreover, two polynomials $F,G \in k[x_1, \dots,
  x_n]$ determine the same polynomial function if, and only if $F \equiv G
  \mod{I(V)}$ in $\Oc(V)$.
\end{lemma}
\begin{proof}
  Let $f$ and  $g$ polynomial functions, the there exist polynomials $F,G$ over
  $k$ such that for any  $P \in V$, $f(P)=F(P)$ and $g(P)=G(P)$. This makes
  $f+g(P)=F+G(P)$ and $fg(P)=FG(P)$ under the usual addition and multiplication
  of polynomials. This makes $\Pc(V,k)$ a subring of $\Fc(V,k)$.

  Now, suppose that two polynomials $F(x_1, \dots, x_n)$ and $G(x_1, \dots,
  x_n)$ determine the same polynomial function. That is, there is an $f \in
  \Pc(V,k)$ for which $F(P)=f(P)$ and $G(P)=f(P)$ for all $P \in V$. Then we
  have $F(P)-G(P)=f(P)-f(P)=0$ for all $P \in V$, so that $F-G \in I(V)$.
  Conversely, suppose that $F \equiv G \mod{I(V)}$; that is $F-G \in I(V)$. Then
  for every $P \in V$, $F(P)-G(P)=$ so $F(P)=G(P)$, and hence they must
  determine the same polynomial function.
\end{proof}
\begin{corollary}
  $\Pc(V,k) \simeq \Oc(V)$.
\end{corollary}
\begin{proof}
  Take the map $\Psi: \Pc(V,k) \xrightarrow{} k[x_1, \dots, x_n]$ by $f
  \xrightarrow{} F$ where $f(P)=F(P)$ for all $P \in V$. Observe that $\Psi$ is
  a ring homomorphism. Moreover, by definition of polynomial functions,
  $\Psi(\Pc(V,k))=k[x_1, \dots, x_n]$, and $\ker{\Psi}=I(V)$.
\end{proof}

\begin{definition}
  Let $V \subseteq \A^n(k)$ and $W \subseteq \A^n(k)$ be affine varieties. A map
  $\phi:V \xrightarrow{} W$ is called a \textbf{polynomial map} if there exist
  polynomials $T_1, \dots, T_m \in k[x_1, \dots, x_n]$ such that
  \begin{equation*}
    \phi(P)=(T_1(P), \dots, T_m(P)) \text{ for all } P \in V
  \end{equation*}
  We call each $T_i$ a \textbf{component polynomial} of $\phi$.
\end{definition}

\begin{lemma}\label{lemma_2.1.2}
  Let $\phi:V \xrightarrow{} W$ be any polynomial map. Then there exists a
  homomorphism $\Phi:\Fc(W,k) \xrightarrow{} \Fc(V,k)$ defined by $f
  \xrightarrow{} f \circ \phi$. Moreover, if $\phi$ is a polynomial map, then
  $\Phi$ restricts to a homomorphism from  $\Oc(W) \xrightarrow{} \Oc(V)$.
\end{lemma}
\begin{proof}
  Define $\Phi:\Fc(W,k) \xrightarrow{} \Fc(W,k)$ by $f \xrightarrow{} f \circ
  \phi$. Let $f,g \in \Fc(V,k)$, then $(f+g) \circ \phi=(f \circ \phi)+(g \circ
  \phi)$ and $(fg) \circ \phi=(f \circ \phi)(g \circ \phi)$. Moreover, since
  $\phi:V \xrightarrow{} W$, and $f:W \xrightarrow{} k$, $f \circ \phi: V
  \xrightarrow{} k$, so $f \in \Fc(V,k)$. So $\Phi$ is a well defined
  homomorphism induced by $\phi$.

  Now, take $\Phi$ restricted to $\Oc(W)$, and $g \in \Phi(\Oc(W))$. Then $g=f
  \circ \phi$ where $f:W \xrightarrow{} k$. Now, since $\phi$ is a polynomial
  map, we get $f \circ \phi(P)=f(T_1(P), \dots, T_m(P))=f \circ (T_1, \dots,
  T_m)(P) \in k[x_1, \dots, x_n]$. Then $g-(f \circ (T_1, \dots, T_m))(P)=0$ for
  all $P \in V$. By lemma \ref{lemma_2.1.1}, this makes $g \equiv f \cirt (T_1,
  \dots, T_m) \mod{I(V)}$, so $g \in \Oc(V)$. Thus $\Phi(\Oc(W)) \subseteq
  \Oc(V)$, and $\Phi$ can be restricted.
\end{proof}

\begin{lemma}\label{lemma_2.1.3}
  Any polynomial map $T:\A^n(k) \xrightarrow{} \A^m(k)$ uniquely determines its
  coordinate polynomials.
\end{lemma}
\begin{proof}
  Let $P \in V$, and suppose there are polynomials $T_1, \dots, T_m$ and $S_1,
  \dots, S_m$ such that $T(P)=(T_1(P), \dots, T_m(P))$ and
  $T(P)=(S_1(P), \dots, S_m(P))$ for some given polynomial map
  $T:V \xrightarrow{} W$. Then
  \begin{align*}
    T(P)-T(P) &= T(P)=(T_1(P), \dots, T_m(P))-(S_1(P), \dots, S_m(P)) \\
        &=  (T_1-S_1(P), \dots, T_m-S_m(P))  \\
        &=  0
  \end{align*}
  so for each $1 \leq i \leq m$, $T_i-S_i(P)=0$, by lemma \ref{lemma_1.1.4},
  $T_i-S_i=0$ and we are done.
\end{proof}

\begin{theorem}\label{theorem_2.1.4}
  Let $V \subseteq \A^n(k)$ and $W \subseteq \A^m(k)$ be affine varieties. THen
  there exists a natural 1--1 correspondence between polynomial maps
  $\phi:V \xrightarrow{} W$ and homomorphisms $\Phi:\Oc(W) \xrightarrow{} \Oc(V)$.
  Moreover any such $\phi$ is the restriction of a polynomial map from
  $\A^n(k) \xrightarrow{} \A^m(k)$.
\end{theorem}
\begin{proof}
  Let $\a:\Oc(W) \xrightarrow{} \Oc(V)$ a homomorphism and take $T_i \in
  k[x_1, \dots, x_n]$, $1 \leq i \leq m$ such that $\a(x_i \mod{I(W)}) \equiv
  T_i \mod{I(V)}$. Define $T=(T_1, \dots, T_m)$. Then $T:\A^n(k) \xrightarrow{}
  \A^m(k)$ is a polynomial map by definition. Define $\tilde{T}:\Fc(\A^m,k)
  \xrightarrow{} \Fc(\A^n,k)$ as in \ref{lemma_2.1.2}, by that same lemma,
  $\tilde{T}$ restricts to $\Oc(W) \xrightarrow{} \Oc(V)$. Now, take $g \in
  \tilde{T}(\Oc(W))$. Then $g=f \circ T$, where $f \equiv 0 \mod{I(W)}$. Then $f
  \circ \tilde{T} \equiv 0 \mod{I(V)}$, and so we have $g \equiv  f \circ \tilde{T}
  \equiv 0 \mod{I(V)}$. Therefore $g \in I(V)$ and $\tilde{T}(I(W)) \subseteq
  I(V)$. In particular, $T(V) \subseteq W$. Then $T$ restricts to a polynomial
  map  $\phi:V \xrightarrow{} W$, now since $\a(x_i \mod{I(V)}) \equiv
  T_i \mod{I(V)}$, we get precisely that $\a=\Phi$, where $\Phi:f \xrightarrow{}
  f \circ \phi$.
\end{proof}

\begin{definition}
  Let $\phi:V \xrightarrow{} W$ a polynomial map. We call the homomorphsim
  $\Phi:\Oc(W) \xrightarrow{} \Oc(V)$ defined by $f \xrightarrow{} f \circ \phi$
  the homomorphism \textbf{induced} by $\phi$.
\end{definition}

\begin{definition}
  We call a polynomial map $\phi:V \xrightarrow{} W$ an \textbf{isomorphism} if
  there is a polynomial map $\psi:W \xrightarrow{} V$ such that $\psi \circ
  \phi=\id_V$ and  $\phi \circ \psi=\id_W$, and we call the varieties $V$ and
  $W$  \textbf{isomorphic} if such a $\phi$ exists, and we write  $V \simeq W$.
\end{definition}

\begin{lemma}\label{lemma_2.1.5}
  Two affine varieties are isomorphic if, and only if their coordinate rings are
  isomorphic.
\end{lemma}
\begin{proof}
  Suppose $V$ and $W$ are isomorphic, and let $\phi:V \xrightarrow{} W$ the
  underlying isomorphism. Define then $\Phi:\Oc(W) \xrightarrow{} \Oc(V)$ the
  homomorphism induced by $\phi$. Let $\psi:W \xrightarrow{} V$ the polynomial
  map making $\phi$ an isomorphsm, and let $\Psi:\Oc(V) \xrightarrow{} \Oc(W)$
  its induced homomorphism. Then for $f \in \Oc(W)$, $\Psi \circ \Phi(f)=f \circ
  (\psi \circ \phi)=f$, so $\Psi \circ \Phi=\id_{\Oc(W)}$. Likewise, $\Phi \circ
  \Psi=\id_{\Oc(V)}$, which makes $\Phi$ a ring isomorphism, and $\Oc(V) \simeq
  \Oc(W)$. The converse holds by similar reasoning.
\end{proof}
