%----------------------------------------------------------------------------------------
%	SECTION 1.1
%----------------------------------------------------------------------------------------

\section{Lagrange's Theorem.}
\label{section1}

\begin{theorem}[Lagrange's Theorem.]\label{3.3.1}
    Let $G$ and  $H$ be groups, then the order of $H$ divides the order of $G$,
    i.e.  $\ord{H}|\ord{G}$.
\end{theorem}
\begin{proof}
    Let $H \leq G$ and consider the coset $gH$. We claim that  $\ord{H}=|gH|$.
    If $g' \in gH$, then  $g'=gh$, for some  $h \in H$. Now, define the map
    $\phi:H \rightarrow gH$ by $h \rightarrow gh$. First, we have that  if
    $gh=gh'$ for  $h,h' \in G$, then by cancellation, $h=h'$. Moreover, since
    $gH$ is defined for all  $h \in H$,  $\phi$ is onto. Therefore  $\phi$ is a
    1-1 mapping of  $H$ onto  $gH$, thus  $|gH|=\ord{H}$.

    Now, the group $G$ is partitioned into disjoint subsets by the cosets of
    $H$ in  $g$. Assume that there are $k$ such cosets. Then:
    \begin{equation*}
        \ord{G}=\sum_{g \in G}{|gH|}=\sum_{i=1}^k{\ord{H}}=k\ord{H}.
    \end{equation*}
    Therefore, $\ord{H}|\ord{G}$.
\end{proof}
\begin{corollary}
    There are
    \begin{equation*}
        \frac{\ord{G}}{\ord{H}}
    \end{equation*}
    cosets of $H$ in  $G$.
\end{corollary}
\begin{proof}
    We have that $\ord{G}=k\ord{H}$, where $k$ is the number of cosets of $H$ in
    $G$.
\end{proof}
\begin{corollary}
    If $N \unlhd G$, then
    \begin{equation}
    \ord{\faktor{G}{N}}=\frac{\ord{G}}{\ord{N}}
    \end{equation}
\end{corollary}

\begin{definition}
    Let $G$ be a (not necessarily finite) group, and  $H \leq H$. We define the
    \textbf{index} of $H$ in $G$ to be the number of left (or right) cosets of
    $H$ in $G$, and write $[G:H]$.
\end{definition}
\begin{remark}
    By the above corollaries, we have that if $G$ is finite, then
    $[G:H]=\frac{\ord{G}}{\ord{H}}$.
\end{remark}

\begin{lemma}\label{3.3.2}
    If $G$ is a finite group, and  $g \in G$, then  $\ord{g}|\ord{G}$.
\end{lemma}
\begin{proof}
    Consider the cyclic subgroup generated by $g$, $\vbrack{g} \leq G$. By
    Lagrange's theorem, we get $\ord{\vbrack{g}}=\ord{g}|\ord{G}$.
\end{proof}
\begin{corollary}
    For any $g \in G$,  $g^{\ord{G}}=e$.
\end{corollary}
\begin{proof}
    $g^{\ord{G}}=g^{k\ord{g}}=(g^{\ord{g}})^k=e^k=e$, for some $k \in \Z^+$.
\end{proof}

\begin{lemma}\label{3.3.3}
    IF $G$ is a finite group of prime order $p$, then $G$ is cyclic.
\end{lemma}
\begin{proof}
    Let $\ord{G}=p$ a prime, and let $g \in G$ be such that $g \neq e$. Then
    $\ord{g} > 1$; moreover, we have $\ord{g}|p$, by lemma \ref{3.3.2}. Then
    either $\ord{g}=1$ or $\ord{g}=p$. By our choice of $g$, its order cannot be
     $1$, thus  $\ord{g}=p=\ord{G}$. Then $\ord{\vbrack{g}}=\ord{G}$, and since
     $\vbrack{g} \leq G$, we can conclude that $G=\vbrack{g}$.
\end{proof}
\begin{corollary}
    If $G$ is of prime order, then  $G \simeq \faktor{\Z}{p\Z}$.
\end{corollary}
\begin{proof}
    Take the homomorphism $g^i \rightarrow i$, for some generator $g \in G$, and
     $1 \leq i \leq \ord{g}$
\end{proof}

\begin{example}\label{3.5}
    \begin{enumerate}
        \item[(1)] Let $H=\vbrack{(1 \ 2 \ 3)} \leq S_3$. Now, we have that $H
            \leq N(H) \leq S_3$, so by Lagrange's theorem,
            $\ord{H}|\ord{N(H)}|\ord{S_3}$. Now, $\ord{H}=3$ and $\ord{S_3}=6$.
            This makes $N(H)=H$ or $N(H)=S_3$. Notice, then that $(1 \ 2)(1 \ 2
            \ 3)(1 \ 2)=(1 \ 2 \ 3)$, and $(1 \ 2)=\inv{(1 \ 2)}$, so $(1 \ 2)$
            conjugates $(1 \ 2 \ 3)$, so $(1 \ 2) \in N(H)$. This makes $N(H)
            \neq H$, so $N(H)=S_3$; by lemma \ref{3.2.4}, $H \unlhd S_3$.

        \item[(2)] Let $G$ be any group, and let  $H \leq G$, such that
            $[G:H]=2$. Thent he total number of left cosets in $G$ is  $H$ and
            $gH$. Now, suppose that  $Hg=H$, since $g \in Hg$, we get $g \in H$,
            this makes  $Hg=H$, but this also makes $gH=H$, implying  $[G:H]=1$,
            which cannot happen. Thus it must be that $Hg=gH$. By lemma \ref
            {3.2.4}, this makes $H \unlhd G$.

        \item[(3)] The relation $\unlhd$ defined by $H \unldh G$ if  $H$ is a
            normal subgroup of  $G$, is not in general transitive. Take $D_8$.
            We have
            \begin{equation*}
                \vbrack{t} \unlhd \vbrack{t,r^2} \unlhd D_8
            \end{equation*}
            but $\vbrack{t}$ is not normal in $ D_8$, for $(rt)t(rt)=r^2t \neq
            t$, so $N(\vbrack{t}) \neq D_8$.
    \end{enumerate}
\end{example}

For some nonexamples, we have:

\begin{example}\label{3.6}
    \begin{enumerate}
        \item[(1)] Let $H=\vbrack{(1 \ 2)} \leq S_3$. Notice that $(1 \ 3)(1 \
            2)(1 \ 3)=(2 \ 3) \neq (1 \ 2)$, since $(1 \ 3)=\inv{(1 \ 3)}$, we
            get that $(1 \ 3) \notin H$. Thus $N(H) \neq S_3$ meaning that $H$
            is not normal in $S_3$. Moreover, notice that $[S_3:H]=3$, then
            there are $3$ left cosets of $H$ in $S_3$, namely, $H$,  $(1 \ 3)H$,
            and $H(1 \ 3)$.

        \item[(2)] For general $S_n$,  $n \in \Z^+$ let  $G_i=\{s \in S_3:
            s(i)=i\}$ for $1 \leq i \leq n$. If $t \in S_3$ such that $t(i)=j$
            then $ts(i)=j=t(i)$, so that $ts\inv{t}=e$. Moreover, if $m(i)=j$,
            then $\inv{t}m(i)=i$, so that $m \in tG_i$. Thus
            \begin{equation*}
                tG_i=\{m \in S_n : m(i)=j\}
            \end{equation*}
            Similarly, taking $k=\inv{t}(j)$, we get
            \begin{equation*}
                G_it=\{l \in S_n : l(k)=i\}
            \end{equation*}

            Now, if $tG_i$ and  $mG_i$ are distinct cosets, we have  $tG_i \cap
            mG_i = \emptyset$ by theorem \ref {3.2.2}. Moreover, notice that
            there are $n$ such cosets, for each distinct image of $i$.
            Therefore, $[S_n:G_i]=n$. Notice that, in general $tG_i \neq G_it$,
            and that  $N(G_i)=G_i$.

        \item[(3)] $\vbrack{r}$ is the only normal subgroup of order $2$ in
            $D_8$.
    \end{enumerate}
\end{example}

\begin{theorem}[Cauchy]\label{3.3.4}
    If $G$ is a finite group, and  $p|\ord{G}$ for some prime $p$, then  $G$
    contains an element of order  $p$.
\end{theorem}

\begin{theorem}[Sylow]\label{3.3.5}
    If $G$ is a finite group of order  $p^am$,  $p$ prime and  $p \not| m$, then
    $G$ has an element of order  $p^a$.
\end{theorem}

\begin{theorem}\label{3.3.6}
    Let $H$ and  $K$ be subgroups of a group, and let  $HK=\{hk : h \in H, k \in
    K\}$. If $H$ and $K$ are finite, then
    \begin{equation}
        \ord{HK}=\frac{\ord{H}\ord{K}}{\ord{H \cap K}}
    \end{equation}
\end{theorem}
\begin{proof}
    Notice that $HK=\bigcup_{h \in H}{hK}$ is a union of left cosets of $K$ in
    $H$. Now, for  $h,h' \in H$,  $hK=h'K$ if, and only if  $\inv{h}h' \in K$,
    i.e. if, and only if $\inv{h}h' \in H \cap K$. So that $hK=h'K$ if and only
    if  $h(H \cap K)=h'(H \cap K)$. Thus the number of left cosets of $H \cap K$
    in $H$ is the same as the number of left cosets of  $K$ in  $H$. By
    Lagrange's thereorem, we then have:
    \begin{equation*}
        [H:K]=[H: H \cap K]=\frac{\ord{H}}{\ord{H \cap K}}
    \end{equation*}
    Then, noticing that there are $\ord{K}$ elements in each coset $hK$, gives
    the result.
\end{proof}

\begin{lemma}\label{3.3.7}
    If $H$ and $K$ are subgroups of a group, then $HK$ is a subgroup if, and
    only if  $HK=KH$.
\end{lemma}
\begin{proof}
    Suppose first, that $HK=KH$. Let $a,b \in HK$, then $a=hk$ and  $b=h'k'$ so
     $ab=(hk)(h'k')=h(kh')k'=h(h''k'')k=(hh'')(k''k) \in HK$. Aditionaly,
     $\inv{a}=\inv{k}\inv{h} \in KH=HK$. Thus $HK$ is a subgroup.

     Conversely, suppose that  $HK$ is a subgroup. Then we have  $K \leq HK$ and
      $H \leq HK$, then by closure, we have that  $KH \subseteq HK$. On the
      other hand, since  $HK$ is a subgroup, then for every  $a \in HK$,
      $\inv{a} \in HK$. Then if $a=hk$,  $\inv{a}=\inv{k}\inv{h}$. That is, $HK
      \subseteq KH$. Teh equality is established.
\end{proof}
\begin{corollary}
    If $H \leq N(K)$ then $HK$ is a subgroup.
\end{corollary}
\begin{proof}
    If $H \leq N(K)$, then for any $h \in H$, and  $k \in K$,  $hk\inv{h} \in
    K$, so $hk=k'h$ for some  $k' \in K$. So  $hk \in KH$. Conversely, we get
    $k'h \in HK$ so that  $HK=KH$.
\end{proof}
\begin{corollary}
    if $K$ is normal, then $HK$ is a subgroup for any $H$.
\end{corollary}

\begin{definition}
    Let $H$ be a subgroup of any group. We say that a subset $A$ of the group
    \textbf{normalizes} $H$ if  $A \subseteq N(H)$. Similarly, we say $A$
    \textbf{centralizes} $H$ if  $A \subseteq C(H)$.
\end{definition}
