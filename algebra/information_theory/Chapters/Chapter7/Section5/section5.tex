%----------------------------------------------------------------------------------------
%	SECTION 1.1
%----------------------------------------------------------------------------------------

\section{Modules.}
\label{section1}

\begin{definition}
    Let $R$ be a ring. We say a nonempty set $M$ is a \textbf{left module} over $R$  (or a
    \textbf{left $R$-module}) if there are operations $+:M \times M \rightarrow M$ and $\cdot:R
    \times M \rightarrow M$ such that $(M,+)$ is an abelian group, and for any $r,s \in R$ and $a,b
    \in M$:
        \begin{enumerate}
            \item[(1)] $r(a+b)=ra+rb$.

            \item[(2)] $r(sa)=(rs)a$.

            \item[(3)] $(r+s)a=ra+sa$.
        \end{enumerate}
        Similarly, we call $M$ a \textbf{right module} (or \textbf{right $R$-module}) over $R$ if
        $(a+b)r=ar+br$, $(as)r=a(sr)$, and $a(r+s)=ar+as$.
    We call $M$  \textbf{unital} if $R$ ha a unit element, and  $1m=m$ for all  $m \in M$.
\end{definition}

We focus on left modules.

\begin{example}
    \begin{enumerate}
        \item[(1)] All vector spaces are unital left modules over any field $F$.		

        \item[(2)] Let $G$ be a group together with an arbitrary operation  $+$ and define an action
            $\cdot:\Z \times G \rightarrow G$ by $(n,a) \rightarrow na \in G$. Then the properties
            of exponents in groups gives $r(a+b)=ra+rb$, $r(sa)=(rs)a$, and $(r+s)a=ra+sa$. This
            makes every group a left $\Z$-module.

        \item[(3)] Let $R$ be a ring, and let  $M$ be a left ideal of  $R$. Take  $r,m \rightarrow
            rm$. Since $M$ is an ideal,  $rm \in M$, and by the multiplicative associative, and
            distributive laws, $M$ is a left $R$-module.

        \item[(4)] Any ring $R$ is a left  (and right) module over itself.

        \item[(5)] Let $R$ be a ring, and  $(\lambda)$ a left ideal of $R$. Consider the quotient
            ring  $R/(\lambda)$. define $+$ by  $(a+\lambda)+(b+\lambda)=(a+b)+\lambda$ and
            $r(a+\lambda)=ra+\lambda$. Clearly these operations are well defined, and
            $(R/(\lambda),+)$ forms a group; moreover,
            $(a+\lambda)+(b+\lambda)=(a+b)+\lambda=(b+\lambda)+(a+\lambda)$, so $R/(\lambda)$ is
            abelian under $+$. Now notice that
            $r(a+b+\lambda)=r(a+b)+\lambda=ra+rb+\lambda=(ra+\lamda)+(rb+\almbda)=r(a+\lambda)+r(b+\lambda)$,
            $r(sa+\lambda)=rsa+\lambda=rs(a+\lambda)$, and
            $(r+s)(a+\lambda)=(r+s)a+\lambda=ra+rs+\lambda=r(a+\lambda)+s(a+\lambda)$. This makes
            $R/(\lambda)$ a left $R$-module. We call this module the  \textbf{left quotient module}
            of $R$ by  $(\lambda)$.
    \end{enumerate}
\end{example} 

\begin{definition}
    Let $M$ be an  $R$-module  (left or right) and $A \subseteq M$, we call  $A$ a
    \textbf{submodule} of $M$ is  $A \leq M$ and whenever  $r \in R$ and  $a \in A$,  $ra \in A$, or
    $ar \in A$.
\end{definition}

\begin{definition}
    If $M$ is an  $R$-module with a collection of submodules $\{M_i\}_{i=1}^s$. We call $M$ the
    \textbf{direct sum} of $\{M_i\}$ if for every $m \in M$, there are uniquely determined  $m_i \in
    M_i$ for  $1 \leq i \leq s$, such that  $m=m_1+\dots+m_s$. We write $M=M_1 \oplus \dots \oplus
    M_s$, or $M=\bigoplus_{i=1}^s{M_i}$.
\end{definition}

\begin{definition}
    An $R$-module is \textbf{cyclic} if there exists $m_0 \in M$ such that $m=rm_0$ (or $m=m_0r$)
    for all $m \in M$ and some  $r \in R$.
\end{definition}

\begin{definition}
    We say an $R$-module is  \textbf{finitely generated} if there exists $a_1,\dots ,a_n \in M$ such
    that for every $m \in M$,  $m=r_1a_1+\dotsr_na_n$ (or $m=a_1r_1+\dots+a_nr_n$) for $r_1, \dots,
    r_n \in R$. We call $\{a_i\}_{i=1}^n$ the \textbf{generating set}; and we call it a
    \textbf{minimal generating set} if $\com{\{a_i\}}{a_j}$ does not generate $M$, for  $1 \leq i,j
    \leq n$. We call the size of a minimal generatng set the  \textbf{rank} of $M$ and denote it
    $\rank{M}$.
\end{definition}

Most of the definitions are stated for both left and right $R$-modules. However, we consider the
following theorems only for left $R$-modules.

\begin{theorem}[The Fundamental Theorem on Finite Modules]\abel{2.5.1}
    Let $R$ be a Euclidean domain; then any finitely generated module  $M$ is the direct sum of a
    finite number of cyclic submodules.
\end{theorem}
\begin{proof}
    By definition, if $M$ is finitely generated, then there are  $a_1, \dots, a_n \in M$ for which
    every element of $M$ is of the form  $r_1a_1+\dots+r_na_n$, for $r_1, \dots, r_n \in R.$ If $M$
    is indeed a direct sum of a finite collection of cyclic submodules, then each  $r_ia_i$ is
    uniquely determined.

    By induction in the rank of $M$; if  $\rank{M}=1$, then $M$ is generated by a single element
    $m_0$. That is, for some $r \in R$, every element of  $M$ has the form  $rm_0$; this makes $M$
    cyclic by definition, and hence the direct sum of itself.

    Now suppose for  $\rank{M}=q$, that $M=\bigoplus_{i=1}^q{M_q}$, where $M_i$ is a cyclic
    submodule. Suppose now that  $\rank{M}=q+1$ and let $\{a_i\}_{i=1}^{q+1}$ be a minimal
    generating set for $M$. Then there are  $r_1, \dots, r_{q+1} \in R$ for which
    $r_1a_1+\dots+r_{q+1}a_{q+1}=0$ (the identity of $(M,+)$). If $r_1a_1=\dots=r_{q+1a_{q+1}}=0$,
    then $M=\bigoplus_{i=1}^{q+1}{M_i}$ and we are done.

    Now suppose that not all the $r_ia_i$ are  $0$. Since  $R$ is a Euclidean domain, with a degree
    function $\deg$, there is an element  $s_1$ of minimum degree occuring as a coefficient in a
    relation of  $\{a_i\}_{i=1}^{q+1}$. Thenn $s_1a_1+\dots+s_{q+1}a_{q+1}=0$, where $\deg{s_1} \leq 
    \deg{s_i}$ for all $1<i\leq q+1$. Now if  $r_1a_1+\dots+r_{q+1}a_{q+1}=0$, then $s_1|r_1$, for
    if $r_1=ms_1+t$ with $t=0$ or  $\deg{t}<\deg{s_1}$, then taking
    $(ms_1)a_1+\dots(ms_{q+1})a_{q+1}=0$ and subtracting $r_1a_1+\dots+r_{q+1}a_{q+1}$, we get
    $ta_1+(r_2-ms_2)a_2+\dots+(r_{q+1}-ms_{q+1})a_{q+1}=0$, since $\deg{t} < \deg{s_1}$, and $s_1$
    has minimum such degree, this makes $t=0$. 

    We also have $s_1|s_i$ for all $1 \leq i \leq q+1$  (obviously $s_1|s_1$). For, suppose that
    $s_1 \not|s_i$ for all $1<i\leq q+1$, then  $s_2=m_2s_1+t$ with $\deg{t}<\deg{s_1}$. Now
    $a_1'=a_1+m_2a_2+\dots+m_{q+1}a_{q+1}, m_2a_2, \dots, m_{q+1}a_{q+1}$ also generate $M$;
    however, $s_1a_1'+ta_2+s_3a_3+\dots+s_{q+1}a_{q+1}=0$, so $t$ is a coefficient occuring in some
    relation of $\{a_i\}$. But $\deg{t}<\deg{s_1}$, which contradicts that $s_1$ has minumum such
    degree, so $t=0$ and hence  $s_1|s_2$. Similarly we get $s_1|s_i$.

    Now consider $a_1^*=a_1+m_2a_2+\dots+m_{q+1}a_{q+1},a_2, \dots, a_{q+1}$. They generate $M$;
    moreover  $s_1a_1^*=s_1a_1+(s_1m_2)a_2+\dots+(s_1m_{q+1})a_{q+1}=s_1a_1+\dots+s_{q+1}a_q+1=0$.
    If $ra_1^*=ra_1+(rm_2)a_2+\dots+(rm_{q+1})a_{q+1}=0$, then there is some relation on $\{a_i\}$
    for which $a_1$ has coefficent $r$, i.e.  $s_1|r$, so $r_1a_1^*=0$. Letting $M_1$ the cyclic
    submodule generated by $a_1^*$, and $M_2$ the submodule finitely generated by
    $\{a_i\}_{i=2}^{q+1}$, we have $M_1 \cap M_2=0$ and $M=M_1+M_2$; hence $M=M_1 \oplus M_2$. Now
    by hyothesis, we get $M_2=M_2' \oplus M_3 \oplus \dots \oplus M_{q+1}$, each of which is a
    cyclic submodule of $M$; which completes the proof.
\end{proof}
\begin{corollary}
    Any finite abelian group is the direct product of cyclic groups.
\end{corollary}
\begin{proof}
    Consider the finite abelian group $G$ as a  $\Z$-module.
\end{proof}

\begin{theorem}\abel{2.5.2} 
    The number of non-isomorphic finite abelian groups of order $p^n$ is  $p(n)$; where $p(n)$ is the
    number of partitions of $n$.
\end{theorem}
\begin{proof}
    Let $G$ be a finite abelian group of order $\ord{G}=p^n$, for $n,p \in \Z^+$ and  $p$ prime. By
    the corollary to the fundamental theorem,  $G=G_1 \times \dots \times G_k$, where $G_i$ is a
    cyclic group of order  $\ord{G_i}=p^{n_i}$, where $n_k \leq \dots \leq n_1 \leq n_1$. Then
        \begin{equation*}
            \odr{G_1 \times G_2}=\frac{\ord{G_1}\ord{G_2}}{\ord{(G_1 \cap G_2)}}.
        \end{equation*}
        Since $G_1 \times G_2$ is a direct product, $\ord{(G_1 \cap G_1)}=(e)$, so $\ord{G_1 \times
        G_2}=\ord{G_1}\ord{G_2}=p^{n_1}p^{n_2}=p^{n_1+n_2}$. Continuing this way we get
        $p^n=\ord{G}=\ord{(G_1 \times \dots \times G_k)}=p^{n_1+\dots+n_k}$, hence $n=n_1+\dots+n_k$
        making $\{n_i\}_{i=1}^k$ a partition of $n$.

        On the other hand, if $\{n_i\}_{i=1}^k$ is a partition of $n$, then we construct  $G$ of
        $\ord=p^n$ as follows: for  $1 \leq i \leq k$, let  $G_i$ be a cyclic group of order
        $\ord{G_i}=p^{n_i}$ and let $G$ be the external direct product of  $\{G_i\}_{i=1}^k$. $G$ is
        an abelian group of order  $p^n$. Hence for each partition of  $n$, there isa abelian group
        of order  $p^n$, if we take  $p^{n_i}$ for $1 \leq i \leq k$, characterizing  $G$ up to
        isomorphism, we get a  $1-1$ correspondence ofo non-isomorphic finite abelian groups of
        order  $p^n$ and partitions if  $n$.
\end{proof}
\begin{corollary}
    The number of non-isomorphic finite abelian groups of order $p_1^{n_1} \dots p_k^{n_k}$ for
    $p_i$ distinct primes is  $p(n_1) \dots p(n_k)$.
\end{corollary}

We now observe $R$-modules in the context of homomorphisms.

\begin{definition}
    Let $R$ be a ring, and let  $M$ and  $N$ be  left $R$-modules. We define a map  $T:M \rightarrow
    N$to be a \textbf{left $R$-homomorphism} if
        \begin{enumerate}
            \item[(1)] $(m_1+m_2)T=m_1T+m_2T$.

            \item[(2)] $(rm_1)T=r(m_1)T$.
        \end{enumerate}
    We define the \textbf{kernel} of $T$ to be  $\ker{T}=\{x \in M: xT=0\}$. We define the
    \textbf{image} of $T$ to be  $\im{T}=\{xT: x \in M\}$.
\end{definition}

Here we mean $xT$ to be  $T(x)$ to reduce notational encumberance. In the case of composition of
$R$-homomorphisms, we mean  $TS$  $S \circ T$.

\begin{example}
    Let $T:M \rightarrow N$ and $S:N \rightarrow Q$ be left $R$-homomorphisms. Define  $TS:M
    \rightarrow Q$ by $xTS=(xT)S$. Then for $r, s \in R$ and  $m_1, m_2 \in M$, we have that
    $(rm_1+sm_2)TS=r((m_1T)S)+s((m_2T)S)$. Which makes $TS$ into an $R$-homomorphism. It is easy to
    see then that  $\ker{TS}=\{xT:xTS=0\}$.
\end{example} 

\begin{lemma}\abel{2.5.3}
    Let $M$ and  $N$ be left  $R$-modules, and let  $T:M \rightarrow N$ be a left $R$-homomorphism.
    Then  $\ker{T}$ and $\im{T}$ are submodules of $M$ and  $N$ respectively.
\end{lemma}
\begin{proof}
    Since $T$ is a  $R$-homomorphism, it is a group homomorphism; hence  $\ker{T} \leq M$/ Now
    letting $r \in R$ and  $x \in \ker{T}$, $(rx)T=r(xT)=r0=0$, putting $rx \in \ker{T}$. 
    Similarly, by the bilinearity of $T$,  $\im{T} \leq N$ and $xT \in \im{T}$. since $rx \in M$,  
    and $r(xT)=(rx)T$, we get that $r(xT) \in \im{T}$.
\end{proof}

\begin{lemma}\abel{2.5.4}
    Let $T$ be an $R$-homomorphism. Then  $T$ is  $1-1$ if and only if  $\ker{T}=0$.
\end{lemma}
\begin{proof}
    Suppose that $T$ is  $1-1$. Then  $xT=yT$ implies  $x=y$, this makes  $\ord{(\ker{T})}=1$, hence
    $\ker{T}=0$. Now suppose that $ker{T}=0$, and let $xT=yT$. Then  $xT-yT=(x-y)T=0$, so $x-y \in
    \ker{T}$. This makes $x-y=0$, hence  $x=y$ which makes  $T$  $1-1$.
\end{proof}

\begin{definition}
    Let $M$ and  $N$ be  $R$-modules. We say that an  $R$-homomorphism  $T:M \rightarrow N$ is an
    \textbf{$R$-isomorphism} if $T$ is  $1-1$ from  $M$ onto  $N$. In this case, we say that  $M$
    and  $N$ are  \textbf{$R$-isomorphic}, and write $M \simeq_R N$.
\end{definition}

We would also like to define what a ``left quotient module'' much in the same manner we described
the left quotient module'' of a ring $R$ by a left ideal  $(\lambda)$. Our motivation is the fact
that if $M$ is a left $R$-module, and  $A \subseteq M$ is a submodule, then since  $(r,a) \in R
\times A$ implies $ra \in A$, this makes  $A$ into a left ideal of  $R$. So already we have that
$R/A$ is a left quotient module of  $R$ by  $A$.

We would like to take this same quotient, restricting $R$ to  $M$. Define the operations  $+:M/A
\times M/A \rightarrow M/A$ by $(a+A)+(b+A)=(a+b)+A$ and $\cdot:R \times M/A \rightarrow M/A$ by
$r(a+A)=ra+A$. Like in the case of quotient modules by ideals, these operations are well defined,
and make $(M/A,+)$ into a group; moreover they satisfy the rest of the axioms for modules. Thus we
then have the following definition.

\begin{definition}
    Let $M$ be a left $R$-module and  $A \subseteq M$ a submodule. Define the operations  $+$ and
    $\cdot$ by  $(a+A)+(b+A)=(a+b)+A$ and $r(a+A)=ra+A$, respectively. We call the module $M/A$ the
     \textbf{left quotient module} of $M$ by  $A$.
\end{definition}

\begin{lemma}\abel{2.5.5.}
    Let $M$ be an a left $R$-module, and let $A \subseteq M$ be a submodule. Then there exists a
    left  $R$-homomorphism from  $M$ onto  $M/A$.
\end{lemma}
\begin{proof}
    Take the map $m \rightarrow m+A$ which defines a left $R$- homomorphism for
    $(rm+sn)+A=r(m+A)+s(n+A)$; this map is also onto by definition.
\end{proof}

\begin{theorem}\abel{2.5.6}
    Let $M$ and  $N$ be  $R$-modules. If $T:M \rightarrow N$ is an $R$-homomorphism from  
    $M$ onto  $T$, then $N \simeq_R M$.
\end{theorem}
\begin{proof}
    By the fundamental theorem for group homomorphisms, we have that as groups, $N \simeq
    M/\ker{T}$. By the aximos of modules, this makes $N \simeq_R M/\ker{T}$.
\end{proof}

\begin{definition}
    We call an $R$-module  $M$  \textbf{irreducible} if its only submodules are $0$ and  $M$.
\end{definition}
