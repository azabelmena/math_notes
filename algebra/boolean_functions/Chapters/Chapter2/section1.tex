\section{Coordinate Rings and Polynomial Maps}

\begin{definition}
  An \textbf{affine variety} in $\A^n(k)$ is an irreducible algebraic set in
  $\A^n(k)$. If $V$ is an affine variety in  $\A^n(k)$, we define the
  \textbf{coordinate ring} of $V$ to be the quotient ring:
  \begin{equation*}
    \Oc(V)=\faktor{k[x_1, \dots, x_]}{I(V)}
  \end{equation*}
\end{definition}

\begin{definition}
  Let $V$ be an affine variety in  $\A^n(k)$, and let $\Fc(V,k)$ the set of all
  functions $V \xrightarrow{} k$. We call a function $f \in \Fc(V,k)$ a
  \textbf{polynomial function} if there exists a polynomial $F \in k[x_1, \dots,
  x_n]$ such that $f(P)=F(P)$ for any point $P \in V$. We denote the collection
  of all polynomial functions from $V \xrightarrow{} k$ by $\Pc(V,k)$.
\end{definition}

\begin{theorem}\label{theorem_2.1.1}
  Let $V$ be an affine variety of  $\A^n(k)$. Then $\Pc(V,k)$ is a subring of
  $\Fc(V,k)$; moreover two polynomials $F$ and $G$ over $k$ determine the same
  polynomial function if, and only if $F-G \in I(V)$.
\end{theorem}
\begin{proof}
  Let $f$ and  $g$ polynomial functions, the there exist polynomials $F,G$ over $k$
  such that for any  $P \in V$, $f(P)=F(P)$ and $g(P)=G(P)$. This makes
  $f+g(P)=F+G(P)$ and $fg(P)=FG(P)$ under the usual addition and multiplication
  of polynomials. This makes $\Pc(V,k)$ a subring of $\Fc(V,k)$.

  Now, let $F,G \in k[x_1, \dots, x_n]$ such that $F$ and  $G$ determine the
  same polynomial function  $f$; i.e.. for every $P \in V$, $f(p)=F(P)$ and
  $f(P)=G(P)$. Then $F-G(P)=0$ which makes $F-G \in I(V)$. Conversely, if $F-G
  \in I(V)$, then for any $P \in V$, $F-G(P)=0$ so that $F(P)=G(P)$. By
  definition this means that both $F$ and  $G$ determine the same polynomial
  function.
\end{proof}
\begin{corollary}
  $\Oc(V) \simeq \Pc(V,k)$.
\end{corollary}

\begin{definition}
  Let $V \subseteq \A^n(k)$ and $W \subseteq \A^m(k)$ be affine varieties. We
  call a map $\phi:V \xrightarrow{} W$ a \textbf{polynomial map} if there exist
  polynomials $T_1, \dots, T_m \in k[x_1, \dots, x_n]$ such that
  $\phi(P)=(T_1(P), \dots, T_m(P))$ for any point $P \in V$.
\end{definition}

\begin{lemma}\label{lemma_2.1.2}
  Let $V \subseteq \A^n(k)$ and $W \subseteq \A^m(k)$ affine varieties. Let
  $\phi:V \xrightarrow{} W$ be any function. Then there exists a map
  $\Phi:\Fc(W,k) \xrightarrow{} \Fc(V,w)$ such that $\Phi=f \circ \phi$.
  Moreover if $\phi$ is a polynomial map, then $\Phi(\Oc(W)) \subseteq
  \Oc(V)$, and $\Phi$ restricts to a homomorphism  $\Oc(W) \xrightarrow{}
  \Oc(V)$.
\end{lemma}

\begin{theorem}\label{theorem_2.1.3}
  Let $V \subseteq \A^n(k)$ and $W \subseteq \A^m(k)$ be affine varieties. Then
  there is a natural 1--1 correspondence of functions $V
  \xrightarrow{} W$ onto homomorphisms $\Oc(W) \xrightarrow{} \Oc(V)$.
  Moreover any such function $V \xrightarrow{} W$ is the restriction of a
  polynomial map $\A^n(k) \xrightarrow{} \A^m(k)$.
\end{theorem}
\begin{proof}
  Let $\a:\Oc(W) \xrightarrow{} \Oc(V)$ a homomorphism and take $T_i \in
  k[x_1, \dots, x_n]$, $1 \leq i \leq m$ such that $\a(x_i \mod{I(W)})=T_i
  \mod{I(V)}$. Define $T=(T_1, \dots, T_m)$. Then $T:\A^n(k) \xrightarrow{}
  \A^m(k)$ is a polynomial map by definition. Define $\tilde{T}:\Oc(\A^m)
  \xrightarrow{} \Oc(\A^n)$; by lemma \ref{lemma_2.1.2}, we have
  $\tilde{T}(\Oc(\A^m(k))) \subseteq \Oc(\A^n(k))$. That is, for any $V
  \subseteq \A^n(k)$ and $W \subseteq \A^m(k)$, we get $\tilde{T}(I(W)) \subseteq
  I(V)$. That is $T(V) \subseteq W$. Therefore $T$ restricts to a polynomial map from
  $V \xrightarrow{} W$. In fact, it restricts preciesly to $\a$.
\end{proof}

\begin{definition}
  We call a polynomial map $\phi:V \xrightarrow{} W$ an \textbf{isomorphism} if
  there is a polynomial map $\psi:W \xrightarrow{} V$ such that $\psi \circ
  \phi=\id_V$ and  $\phi \circ \psi=\id_W$.
\end{definition}

\begin{lemma}\label{lemma_2.1.4}
  Two affine varieties are isomorphic if, and only if their coordinate rings are
  isomorphic.
\end{lemma}
