\section{Bent Functions on Fields of Even Characteristic}

\begin{definition}
  We definte the \textbf{support} of a Boolean function $f$ over $\F_2^n$ to be
  \begin{equation*}
    \supp{f}=\{x \in \F_2^n : f(x) \neq 0\}
  \end{equation*}
  we define the \textbf{Hamming weight} of $f$ to be $w(f)=|\supp{f}|$. We
  define the \textbf{Hamming distance} be between two Boolean functions $f$ and
   $g$ over  $\F_2^n$ to be
   \begin{equation*}
     d(f,g)=w(f-g)=|\{x \in \F_2^n : f(x)-g(x) \neq 0\\}|
   \end{equation*}
\end{definition}

\begin{definition}
  The \textbf{nonlinearity} of a Boolean function $f$ over  $\F_2^n$ is defined
  to be the minimum Hamming distance $NL$ between $f$ and all affine Boolean
  functions over $\F_2^n$; i.e.
  \begin{equation}\label{equation_1.9}
    NL(f)=\min{\{d(f,g) : g \text{  is affine Boolean over } \F_2^n\}}
  \end{equation}
  We define the \textbf{nonlinearity} of an $(n,m)$-Boolean function $f$ to be
  the minimum distance between all nonzero linear combinations of the coordinate
  functions of  $f$, and all affine Boolean functions over $\F_2^n$.
\end{definition}

\begin{lemma}\label{lemma_1.2.1}
  Let $f$ be an $(n,m)$-Boolean function. Then
  \begin{equation}\label{equation_1.10}
    NL(f)=\min{\{NL(\langle b,f \rangle) : b \in \F_2^n \text{ and } b \neq 0\}}
  \end{equation}
\end{lemma}

\begin{definition}
  Considering a Boolean function as a $\Z$-valued function, the  \textbf{Walsh
  transform} $\l{f}$ of an $(n,m)$-Boolean function $f$ is defined to be the
  function  $\l{f}:\F_2^n \times \F_2^m \xrightarrow{} \Z$ defined by:
  \begin{equation}\label{equation_1.11}
    \l{f}(a,b)=\sum_{x \in \F_2^m}{(-1)^{\langle b,f(x) \rangle+\langle a,x \rangle}}
    \text{ where } a \in \F_2^n \text{ and } b \in \F_2^m
  \end{equation}
  We call the value $\l{f}(a,b)$ a \textbf{Walsh coefficient} of $f$. We define
  the \textbf{Walsh spectrum} \Lambda{f}, and the \textbf{extended Walsh
  spectrum} $\Lambda'{f}$ of $f$ by:
  \begin{align*}
    \Lambda{f}    &=  \{\l{f}(a,b) : a \in \F_2^n, b \in \F_2^m, \text{ and } b
                        \neq 0\}  \\
    \Lambda{f}    &=  \{|\l{f}(a,b)| : a \in \F_2^n, b \in \F_2^m, \text{ and } b
                        \neq 0\}  \\
  \end{align*}
\end{definition}

\begin{lemma}\label{lemma_1.2.2}
  The Walsh transform of a Boolean function $f$ over  $\F_2^n$ is given by:
  \begin{equation}\label{equation_1.12}
    \l{f}(a)=\sum_{x \in \F_2^n}{(-1)^{f(x)+\langle a,x \rangle}} \text{ where }
    a \in \F_2^n
  \end{equation}
\end{lemma}

\begin{lemma}\label{lemma_1.2.3}
  The Walsh transform of an $(n,m)$-Boolean function is gven by:
  \begin{equation}\label{equation_1.12}
    \l{f}(a,b)=\sum_{x \in \F_{2^m}}{(-1)^{\tr_n(bf(x))+\tr_n(ax)}}
    \text{ where } a \in \F_{2^n} \text{ and } b \in \F_{2^m}
  \end{equation}
\end{lemma}
\begin{proof}
  Identify $\F_2^n$ with  $\F_{2^n}$ and $\F_2^m$ with  $\F_{2^m}$ Observe that
  $\langle x,y \rangle=\tr_n(xy)$.
\end{proof}

\begin{proposition}\label{proposition_1.1.4}
  For any $(n,m)$-Boolean function $f$, we have
  \begin{equation}\label{equation_1.14}
    \l{f}(a,b)=2^n-2w(\langle b,f(x) \rangle+\langle a,x \rangle)=
    2^n-2d(\langle b,f(x) \rangle, \langle a,x \rangle)
  \end{equation}
  moreover, we get
  \begin{align}
    d(\langle b,f(x) \rangle, \langle a,x \rangle) &=  2^{n-1}-\frac{1}{2}\l{f}(a,b) \\
    d(\langle b,f(x) \rangle, \langle a,x \rangle+1) &=  2^{n-1}+\frac{1}{2}\l{f}(a,b)
  \end{align}
  lastly we get
  \begin{equation}\label{equation_1.17}
    NL(f)=2^{n-1}-\frac{1}{2}\l{f}
  \end{equation}
\end{proposition}
\begin{corollary}
  The nonlinearity and the extended Walsh spectrum of an $(n,m)$-Boolean
  functiom are preserved under CC-equivalence.
\end{corollary}

\begin{proposition}[Parseval's Relation]\label{proposition_1.1.5}
  For any Boolean function $f$ over  $\F_2^n$, we have:
  \begin{equation}\label{equation_1.18}
    \sum_{a \in \F_2^n}{\l{f}(a)}=2^{2n}
  \end{equation}
\end{proposition}
\begin{corollary}[The Universal Bound on Nonlinearity]
  \begin{equation}\label{equation_1.19}
    NL(f) \leq 2^{n-1}-2^{\frac{n}{2}-1}
  \end{equation}
\end{corollary}

\begin{definition}
  We call an $(n,m)$-Boolean function $f$ \textbf{bent} if it achieves the
  Universal bound on nonlinearity of equation \ref{equation_1.19}; that is:
  \begin{equation*}
    NL(f)=2^{n-1}-2^{\frac{n}{2}-1}
  \end{equation*}
\end{definition}

\begin{theorem}\label{theorem_1.2.6}
  An $(n,m)$-Boolean function is bent only if $n$ is even and  $m \leq
  \frac{n}{2}$.
\end{theorem}

\begin{thoerem}\label{theorem_1.2.7}
  An $(n,m)$-function is bent if, and only if one of the following hold:
  \begin{enumerate}
    \item[(1)] For every non-zero $c \in \F_2^m$, the Boolean function $\langle
      c,f(x) \rangle$ is bent.

    \item[(2)] $\l{f}(a,b)=\pm{2^{\frac{n}{2}}}$ for any $a \in \F_2^n$ and for
      any non-zero $b \in \F_2^m$.

    \item[(3)] $f$ is perfect nonlinear.
  \end{enumerate}
\end{thoerem}

\begin{lemma}\label{lemma_1.2.8}
  If $f$ is a Boolean bent function over $\F_2^n$, for $n>2$, then:
  \begin{equation*}
    d^\circ(f) \leq \frac{n}{2}
  \end{equation*}
\end{lemma}

\begin{definition}
  We define the \textbf{dual} of a Boolean bent function $f$ over  $\F_2^n$ to
  be the Boolean function $f^\perp$ defined by:
  \begin{equation}\label{equation_1.20}
    \l{f}(a)=2^{\frac{n}{2}}(-1)^{f^\perp(a)}
  \end{equation}
\end{definition}

\begin{lemma}\label{lemma_1.2.8}
  For any Boolean bent function $f$,  $\f^\perp$ is also bent, and
  $(f^\perp)^\perp=f$.
\end{lemma}

\begin{definition}
  We define the following for any Boolean bent function $f$ over $\F_2^n$:
  \begin{enumerate}
    \item[(1)] We say $f$ belongs to the  \textbf{Maiorana-McFarlan class}, or
      is \textbf{MM-class} if for every $x,y \in \F_2^{\frac{n}{2}}$:
      \begin{equation}\label{equation_1.21}
        f(x,y)=\pi(y)+g(x)
      \end{equation}
      where $\pi$ is a permutation of $\F_2^{\frac{n}{2}}$, and $g$ is a
      balanced Boolean function over $\F_2^{\frac{n}{2}}$.

    \item[(2)] We say $f$ belongs to the \textbf{complete Maiorana-Mcfarlan
      class}, or is \textbf{completed MM-class} if $f$ is  EA-equivalent to an
      MM-class function.
  \end{enumerate}
\end{definition}

\begin{theorem}\label{1.2.9}
  The completed MM-class contains all quadratic Boolean functions.
\end{theorem}

\begin{definition}
  We say a Boolean bent function belongs to the $PS_{ab}$-class if for any $x,y
  \in \F_2^{\frac{n}{2}}$
  \begin{equation}\label{equation_1.22}
    f(x,y)=g(\frac{x}{y})
  \end{equation}
  where $g$ is a balanced Boolean function over  $\F_2^{\frac{n}{2}}$, and where
  we make the convention that $\frac{1}{0}=0$.
\end{definition}

\begin{lemma}\label{lemma_1.2.10}
  Any $PS_{ab}$-class function has optimal algebraic degree. That is:
  \begin{equation*}
    d^\circ=\frac{n}{2}
  \end{equation*}
\end{lemma}
