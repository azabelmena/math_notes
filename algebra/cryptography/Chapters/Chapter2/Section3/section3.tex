%----------------------------------------------------------------------------------------
%	SECTION 1.1
%----------------------------------------------------------------------------------------

\section{Spurious Keys.}
\label{section1}

The goal of cryptanalysis is to recover the key from a sufficiently large enough
body of ciphertext. Supposing that an adversary launches a cyrptanalytic attack,
we make the following definition.

\begin{definition}
    In a cryptanalytic attex, we call incorrectly determined, but possible keys
     \textbf{spurious keys}.
\end{definition}

\begin{example}
    Suppose an adversary obtaines the ciphertext \lstinline{WNAJW} and
    determines a shift cipher has been used. Then there are two possible keys,
    $F=5$ giving the plaintext \lstinline {river} and $W=22$ giving the
    plaintext \lstinline{arena}. However, only one of these keys is the correct
    key, and the other is spurious.
\end{example}

\begin{definition}
    Let $(\Pc,\Cc,\Kc)$ be a cryptosystem. We define $H(K,C)$ to be the
    \textbf{key equivocation} wich measures the uncertainty of the key $K$ given
    a ciphertext $C$.
\end{definition}

\begin{theorem}\label{2.3.1}
    Let $(\Pc,\Cc,\Kc)$ be a cryptosystem with cipher $(e,d)$, then
    $H(K|C)=H(K)+H(P)-H(C)$.
\end{theorem}
\begin{proof}
    Notice that $H(K,P,C)=H(C|K,P)+H(K,P)$. Now $K$ and $P$ uniquely determine
    $C$, given $y=e(x)$; so $H(C|K,P)=0$. So $H(K,P,C)=H(K,P)=H(K)+H(P)$ (since
    $K$ and  $P$ are independent).

    Similarly, $C$ and $K$ uniquely determine  $P$, since  $x=d(y)$, so
    $H(P|K,C)=0$, and $H(K,P,C)=H(K,C)=H(K)+H(K|C)$. Now, rearranging terms and
    substituting, we get $H(K|C)=H(K)+H(P)-H(C)$.
\end{proof}

\begin{example}
    Again, considering example $(2.1)$, $H(K|C)$ is about $1.5+0.81-1.85=0.46$
    bits of uncertainty. Computing with conditional entropy, we compute the
    probability matrix  $(P(K=K_i|y=j))$ for $1 \leq i \leq 3$ and  $1 \leq j
    \leq 4$ to obtain:
    \begin{equation*}
        \begin{pmatrix}
            1           &       0           &           0           \\
            \frac{6}{7} & \frac{1}{7}       &           0           \\
            0           & \frac{3}{4}       &       \frac{1}{4}     \\
            0           &           0       &           1           \\
        \end{pmatrix}
    \end{equation*}
    then $H(K)=0.46$
\end{example}

\begin{definition}
    Let $L$ be a natural language and  $P^n$ the random variable with
    probability distribution all  $n$-grams of plaintext. We define the
    \textbf{entropy} of $L$ to be:
    \begin{equation}
        H_L=\lim_{n \rightarrow \infty}{\frac{H(P^n)}{n}}
    \end{equation}
    and the \textbf{redundancy} of $L$ to be:
    \begin{equation}
        R_L=1-\frac{H_L}{\log{|\Pc|}}
    \end{equation}
    Where $\Pc$ is the plaintex space.
\end{definition}

\begin{theorem}\label{2.3.2}
    Let $(\Pc,\Cc,\Kc)$ be a cryptosystem with $|\Cc|=|\Pc|$ and keys chose
    equiprobably. Let  $L$ be the underlying natural language, then given a
    string of ciphertext of length  $n$, with  $n$ sufficiently large, the
    expected number of spurious keys s:
    \begin{equation}
        \bar{s}_n \geq \frac{|\Kc|}{|\Pc|^{nR_L}}
    \end{equation}
\end{theorem}
\begin{remark}
    $\bar{s}_n \rightarrow 0$ exponentially quickly as $n \rightarrow \infty$.
\end{remark}

\begin{definition}
    The \textbf{unicity distance}, $n_0$, of a cryptosystem is define to be the
    value of $n$ for which  $\bar{s}_n \rightarrow 0$. I.e. it is the average
    ammount of ciphertext an adversary needs to uniquely determine the correct
    key, given enough time and resources.
\end{definition}

\begin{lemma}\label{2.3.3}
    As $n \rightarrow \infty$,
    \begin{equation}
        n_0=\frac{\log{|\Kc|}}{R_L\log{|\Pc|}}
    \end{equation}
\end{lemma}
