%----------------------------------------------------------------------------------------
%	SECTION 1.1
%----------------------------------------------------------------------------------------

\section{Hamming Codes.}
\label{section1}

\begin{definition}
    Let $H$ be an  $m \times q^m-1$ matrix with entries in a finite field $\F_q$
    of  $q$ elements, such that the columns of  $H$ are the  $q^m-1$ vectors of
     $\F_q^m$. We define the \textbf{$q$-ary Hamming code} to be the
     $(q^m-1,q^m-1-m)$-linear code over $\F_q$ whose check matrix is $H$.
\end{definition}
\begin{remark}
    When $q=2$, we call  $H$ the parity check matrix of the binary
    $(2^m-1,2^m-1-m)$ Hamming code.
\end{remark}

\begin{definition}
    \begin{enumerate}
        \item[(1)] Consider the $(2^m-1,2^m-1-m)$-binary Hamming code. If we
            have $y=x+z$, for a codeword  $x$ and error pattern  $z$, and
            $z=0$, then  $s=0$. On the otherhand, if  $w_H(z)=1$, then $s$ is
            the  $i$-th column of the parity check matrix $H$.  This makes
            syndrome decoding for the Hamming code easy.

        \item [(2)] The binary $(2^m-1,2^m-1-m)$ Hamming code can $\Cc$ correct
            $1$ error, if and only if all Hamming spheres of radius  $1$ are
            disjoint. Notice a  given Hamming sphere $B_1(x)$ has $n+1$
            codewords, so $\Cc$  can have atmost  $\frac{2^n}{n+1}$ codewords.
            If $n=2^m-1$, we then get  $2^{2^m-1-m}$ possiible codewords.
    \end{enumerate}
\end{definition}

\begin{definition}
    We call a code $\Cc$ over a field $F$, with minimum distance $d=2e+1$,
    \textbf{perfect} if every $x \in F^n$ has distance at most  $e$ to any other
    codeword of  $\Cc$.
\end{definition}
