%----------------------------------------------------------------------------------------
%	SECTION 1.1
%----------------------------------------------------------------------------------------

\section{Syndrome Decoding.}
\label{section1}

\begin{definition}
    Let $\Cc$ be an $(n,k)$-code over a field $F$. For $x \in \Cc$, and $y \in
    F^n$ we call  $z=y-x$ an \textbf{error pattern}. If $z_i \neq 0$, we say
    that the  $i$-th component of  $x$ has \textbf{error}. If $H$ is a check
    matrix, we call  $s=Hy^T$ the  \textbf{syndrome} of the vector $y=x+z$.
\end{definition}

\begin{lemma}\label{2.2.1}
    If $H$ is a check matrix for an  $(n,k)$-code $\Cc$ over a field $F$, and
    $x \in \Cc$, with error pattern  $y=x+z$ for $z \in F^n$, then $Hy^T=Hz^T$.
\end{lemma}
\begin{proof}
    We have $Hy^T=H(x+z)^T=H(x^T+z^T)=Hx^T+Hz^T=Hz^T$, since $x \in \Cc$.
\end{proof}
\begin{corollary}
    $x$ has error if, and only if  $s \neq 0$.
\end{corollary}

\begin{definition}
    Let $\Cc$ be an  $(n,k)$-code over a field $F$, with check matrix  $H$. Let
     $z \in F^n$ such that  $y=x+z$. We define the set of all solutions to the
     syndrome  $Hz^T=s$ to be a  \textbf{coset} of the code $\Cc$.
\end{definition}

\begin{lemma}\label{2.2.2}
    Every coset of an $(n,k)$-code $\Cc$ has the form  $\Cc_0=\Cc+z_0$.
\end{lemma}
\begin{proof}
    Let $\Cc_0$ be a coset for the code $\Cc$, given by  $z_0 \in F^n$. We have
    that for any $x \in \Cc$,  $x+z_0 \in \Cc_0$, since $H(x+z_0)^T=Hz_0^T=s$;
    that is, $x+z_0$ is also a solution to the syndrome. Now, if $y$ is a
    solution to the syndrome $Hz_0^T=s$, then $y=x+z_0$ for some $x \in \Cc$, so
     $y \in \Cc+z_0$.
\end{proof}

\begin{definition}
    We define a \textbf{$q$-ary symmetric channel} ($q$SC) to be a DMC with
    $A_X=A_Y=\F_q$, taking  $X \rightarrow Y=X+Z$ with $Z$ a random vector with
    independently distributed components and  $P(Z=0)=1-(q-1)\epsilon$ and
    $P(Z=z)=\epsilon$ whenever $z \neq 0$.
\end{definition}

\begin{definition}
    Let $\Cc$ be an  $(n,k)$-code. Then \textbf{Hamming weight} of a codeword $x
    \in \Cc$ is the number of nonzero components of  $x$. That is:
    \begin{equation}
        w_H(x)=|\{x_i : x_ \neq 0\}|
    \end{equation}
\end{definition}

\begin{example}
    Let $X \rightarrow Y=X+Z$ over a $q$SC. Then for  $z \in n\F_q^n$,
    $P(Z=z)=(1-(q-1)\epsilon)^{n-w_H(z)}\epsilon^{w_H(z)}$ If $\epsilon \leq
    \frac{1}{q}$, then $P(Z=z)$ is decreasing with respect to $w_H(z)$.
\end{example}

\begin{definition}
    Let $\Cc$ be an $(n,k)$-code with check matrix $H$. We define the
    \textbf{standard matrix} for $\Cc$ to be the  $(n-1) \times (k+n)$ matrix
    whoes rows are the cosets of $\Cc$ The vector of least weight in each row is
    called the  \textbf{coset leader}.
\end{definition}

\begin{example}
    Consider the $(5,3)$-code of example $2.1$ with check matrix
    \begin{equation*}
        H=\begin{pmatrix}
                1 & 1 & 0 & 0 & 0 \\
                1 & 0 & 1 & 1 & 1 \\
          \end{pmatrix}
    \end{equation*}
    $H$ has the syndromes  $00$,  $01$,  $10$, and  $11$; so we calculate the
    standard matrix for  $H$ to be:
    \begin{equation*}
        \begin{pmatrix}
            00000 & 00011 & 00101 & 00110 & 11001 & 11010 & 11100 & 11111 \\
            00100 & 00111 & 00001 & 00010 & 11101 & 11110 & 11000 & 11011 \\
            01000 & 01011 & 01101 & 01110 & 10001 & 10010 & 10100 & 10111 \\
            10000 & 10011 & 10101 & 10110 & 01001 & 01010 & 01100 & 01111 \\
        \end{pmatrix}
    \end{equation*}
\end{example}

This brings us to an algorithm for syndrom decoding with the cosets of linear
codes.

\begin{algorithm}[Syndrome Decoding for a $q$SC.]
    Given an $(n,k)$-linear code over a field $F$ and check matrix $H$.
    Assume that $x$ is transmitted over a  $q$SC, and is recieved as  $y$.
    \begin{enumerate}
        \item[(1)] Compute the syndrome $s=Hy^T$

        \item[(2)] Find a minimum weight vector in the coset of $s$, label it
            $z_0$.

        \item[(3)] Return $\hat{x}=y-z_0$.
    \end{enumerate}
\end{algorithm}
\begin{remark}
    If $n$ and $k$ are both small, we can go about step  $(2)$ using a standard
    matrix, implemented as a lookup table.
\end{remark}
