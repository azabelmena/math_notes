%----------------------------------------------------------------------------------------
%	SECTION 1.1
%----------------------------------------------------------------------------------------

\section{Definitions, Generator, and Check Matrices.}
\label{section1}

\begin{definition}
    Wed define an  \textbf{$(n,k)$-linear code)} over a field $F$ to be a
    $k$-dimensional subspace $\Cc$ of the  $n$-dimensional vector space  $F^n$
    over  $F$.
\end{definition}
\begin{remark}
    We shall be focusing exclusively on the finit fields $\F_p$ where  $p=2,3$.
    Then in this case, we can consider the vector spaces to be extension fields
    of  $\F_p$. We shall prove theorems and lemmas however, for general fields,
    unless specified.
\end{remark}

\begin{definition}
    Let $\Cc$ be an  $(n,k)$-linear codeover a field $F$. We we call a $k \times
    n$matrix $G$ a  \textbf{generator matrix} for $\Cc$ if its row space is $\Cc$.
\end{definition}

\begin{example}\cite{mceliece}
    \begin{enumerate}
        \item[(1)] A $(5,1)$-linear code, $\Cc_1$, over $\F_2$ with generator
            matrix:
            \begin{equation*}
                G_1=\begin{pmatrix}
                        1   &   1    &   1   &   1   &   1   \\
                    \end{pmatrix}
            \end{equation*}
            It contains the codewords $00000$ and $(11111)$; and has rate
            $\frac{1}{5}$. We call $\Cc_1$ the \textbf{binary repitition code}.

        \item[(2)] The $(5,3)$-code $\Cc_2$ with generator matrix:
            \begin{equation*}
                G_2=\begin{pmatrix}
                        1   &   1   &   1   &   0   &   0   \\
                        0   &   0   &   1   &   1   &   0   \\
                        1   &   1   &   1   &   1   &   1   \\
                    \end{pmatrix}
            \end{equation*}
            $\Cc_2$ has rate $\frac{3}{5}$.

        \item[(3)] The \textbf{$(7,4)$-Hamming Code}, $\Cc_3$ over $\F_2$ with
            generator matrix:
            \begin{equation*}
                G_3=\begin{pmatrix}
                        1   &   0   &   0   &   0   &   0   &   1   &   1   \\
                        0   &   1   &   0   &   0   &   1   &   0   &   1   \\
                        0   &   0   &   1   &   0   &   1   &   1   &   0   \\
                        0   &   0   &   0   &   1   &   1   &   1   &   1   \\
                    \end{pmatrix}
            \end{equation*}
            The $(7,4)$-Hamming code has rate $\frac{4}{7}$.
    \end{enumerate}
\end{example}

\begin{lemma}\label{2.1.1}
    If $\Cc$ is an  $(n,k)$-code over a field $F$, and if  $G$ is a generator
    matrix for  $\Cc$, then so is any matrix row-equivalent to  $G$.
\end{lemma}
\begin{proof}
    Let $A$ be an  $k \times n$ matrix row-equivalent to $G$. Then, take  $A
    \rightarrow G$ via the sequence of elementary matrices $\{E_i\}_{i=1}^m$.
    That is, $G=E_m \dots E_2E_2A$. Then for any $v \in F^n$. we can take $Av
    \rightarrow Gv$ via this same sequence; that is  $Gv=E_m \dots E_2E_1Av$.
    Thus, $A$ generates the same set of vectors as  $G$, and hence has the same
    row space.
\end{proof}
\begin{remark}
    Thus, using this lemma, one would ideally like to find a generator matrix in
    Row-Reduced-Echelon form, for ease of computation.
\end{remark}

\begin{definition}
    If $\Cc$ is an  $(n,k)$-code over a field $F$, we define a  \textbf{check}
    for $\Cc$ to be the equation:
    \begin{equation}
        a_1x_1+\dots+a_nx_n=0
    \end{equation}
    satisfied for all $x \in \Cc$. We define the \textbf{dual code} of $\Cc$ to
    be the orthogonal complement
    \begin{equation}
        \Cc^{\perp}=\{a \in F^n : \vbrack{a,x}=0\}
    \end{equation}
    Where $\vbrack{a,x}$ is the inner product of $a$ and  $x$.
\end{definition}

\begin{proof}
    If $\Cc$ is an  $(n,k)-code$, then $\Cc^\perp$ is an  $(n,n-k)$-linear code.
\end{proof}
\begin{proof}
    We have by a result from \cite{herstein} (theorem $4.I$), that $F^n=\Cc
    \oplus \Cc^\perp$, ($\oplus$ the direct sum). Then
    $\dim{F^n}=\dim{\Cc}+\dim{\Cc^\perp}$. Therefore, $\dim{\Cc^\perp}=n-k$.
\end{proof}

\begin{definition}
    Let $\Cc$ be an $(n,k)$-linear code over a field $F$. We define a
    \textbf{check} matrix for $\Cc$ the be an $n \times (n-k)$ matrix $H$ such
    that  $Hx^T=0$.
\end{definition}

\begin{lemma}\label{2.1.2}
    If $H$ is a check matrix for the $(n,k)$-code $\Cc$, then $H$ is a generator
    matrix for the dual code  $\Cc^\perp$.
\end{lemma}
\begin{proof}
    For any $x=(x_1, \dots, x_n) \in \Cc$, we have that $Hx^T=0$, by definition.
    Thus, for any row $a=(a_1, \dots, a_n)$ of $H$. That is,
    $a_1x_1+\dots+a_nx_n=\vbrack{a,x}=0$, making $a \in \Cc^\perp$. Since $a$
    is an arbitrary row of $H$, this holds for every row of  $H$. Thus the row
    space of $H$ is equal to $\Cc^\perp$.
\end{proof}

\begin{lemma}\label{2.1.3}
    Let $\Cc$ be an  $(n,k)$-code over a field $F$, and let  $G$ be a generator
    matrix for the code. If  $G$ has the form  $G=(I_{k \times k}|A)$, then the
    check matrix for $\Cc$, corresponding to $G$ has the form
    \begin{equation}
        H=(-A^T|I_{(n-k) \times (n-k)})
    \end{equation}
\end{lemma}

\begin{example}\cite{mceliece}
    Consider the generator matrices for the codes in example $1.1$, then:
    \begin{enumerate}
        \item[(1)] $H_1=\begin{pmatrix}
                    1 & 1 & 0 & 0 & 0 \\
                    1 & 0 & 1 & 0 & 0 \\
                    1 & 0 & 0 & 1 & 0 \\
                    1 & 0 & 0 & 0 & 1 \\
                  \end{pmatrix}$

        \item[(2)] $H_2=\begin{pmatrix}
                    1 & 1 & 0 & 0 & 0 \\
                    1 & 0 & 1 & 1 & 1 \\
                  \end{pmatrix}$

        \item[(3)] $H_3=\begin{pmatrix}
                    0 & 1 & 1 & 1 & 1 & 0 & 0 \\
                    1 & 0 & 1 & 1 & 0 & 1 & 0 \\
                    1 & 1 & 0 & 1 & 0 & 0 & 1 \\
                  \end{pmatrix}$
    \end{enumerate}
\end{example}

\begin{theorem}\label{2.1.4}
    Let $\Cc$ be an  $(n,k)$-code over a field $F$. Then there is a unique
    $k \times n$ Row-Reduced-Echelon matrix $G$ such that  $x \in \Cc$ if, and
    only if $x$ is in the row space of $G$. Likewise, there exists an $(n-k)
    \times n$ matrix $H$ such that  $x \in \Cc$ if, and only if $Hx^T=0$.
\end{theorem}
\begin{corollary}
    If $\Cc$ is used on a memoryless channel, then  $G=(I_{k \times k}|A)$ and
    $H=(-A^T|I_{(n-k) \times (n-k)})$.
\end{corollary}
