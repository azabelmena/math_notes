%----------------------------------------------------------------------------------------
%	SECTION 1.1
%----------------------------------------------------------------------------------------

\section{Row Equivalence.}

We can use the properties of vector spaces and bases to 
summarise the row equivalence of matrices.

\begin{definition}
    Let $A$ be an  $m \times n$ matrix over a field  $F$. We 
    define the $i$\textbf{-th row vector} of $A$ to be the 
    vector  $\alpha_i=(A_{i1}, \dots A_{in})$ in the vector 
    space $F^n$. We call the subspace of $F^n$ spanned by the 
    row vectors of  $A$ the  \textbf{row space} of $A$ and 
    denote it $\row{A}$. We call the dimension of $\row{A}$ 
    the \textbf{rank} of $A$ and denote it  $\rank{A}$.
\end{definition}

\begin{theorem}\label{2.7.1}
    Row equivalent matrices have the same row space.
\end{theorem}
\begin{proof}
    If $A$ is an  $m \times n $ matrix and $P$ is a  
    $k \times m$ matrix whose row vectors are defined by 
    $\beta_i=P_{i1}\alpha_1+\dots+P_{im}\alpha_m$, where 
    $\alpha_i$ are the row vectors of  $A$ for  $1 \leq i 
    \leq m$. That is  $\row{B}$ is a subspace of $\row{A}$. 
    Now if $P$ is  $m \times m$ and invertible, then  $B$ is 
    row equivalent to  $A$, so  $A=P^{-1}B$ makes $\row{A}$ 
    a subspace of $\row{B}$. That is $\row{A}=\row{B}$.
\end{proof}

\begin{theorem}\label{1.7.2}
    Let $R$ be a nonzero row reduced echelon matrix. Then the 
    nonzero rows of  $R$ form a basis for  $\row{R}$.
\end{theorem}
\begin{proof}
    Let $\rho_1, \dots, \rho_r \in \row{R}$ be the nonzero
    row vectors of $R$. Clearly  $\span{\row{R}}=\{\rho_1, 
    \dots, \rho_r\}$. All that is left is linear independence.
    Since $R$ is a row reduced echelon matrix, we have by 
    definition that there are $k_1 ,\dots k_r \in \Z^+$ such 
    that:
        \begin{enumerate}
            \item[(1)] $R_{ij}=0$ if $j<k$.

            \item[(2)] $R_{ik_j}=\delta_{ij}$ (the Kronecker 
                delta).

            \item[(3)] $k_1< \dots <k_r$.
        \end{enumerate}
        Now suppose that $\beta=(b_1, \dots, b_r) \in\row{R}$.
        Then we have $\beta=c_1\rho_1+\dots_r\rho_r$ for 
        $c_i \in F$ for  $1 \leq i \leq r$. Expanding  
        $\rho_i$ and by above, we see that  $c_j=b_{k_j}$. In 
        particular, $\beta=0$ implies that 
        $\beta=c_1\rho_1+\dots_r\rho_r=1$, which implies 
        that $c_j$ is the  $k_j$-th coordinate of the  $0$ 
        vector. That is  $c_j=0$ for all  $j$.
\end{proof}

\begin{theorem}\label{2.7.3}
    Let $m,n \in \Z^+$ and let  $F$ be a field. Suppose that 
    $W$ is a subspace of  $F^n$ and that  $\dim{W} \leq m$. 
    Then there is precisely one $m \times n$ row reduced 
    echelon matrix over  $F$ with $W$ as its row space.
\end{theorem}
\begin{proof}
    There is at least one such matrix as described above. Now 
    select $m$ vectors  $\alpha_1, \dots, \alpha_m \in W$ for 
    which $\Span{W}=\{\alpha_i\}_{i=1}^m$. Let $A$ be the $m 
    \times n$ matrix with  $\row{A}=\{\alpha_i\}$, and let $R$
    be a row reduced echelon matrix row equivalent to  $A$. 
    then  $\row{R}=\row{A}=W$ by theorem \ref{2.7.1}. 

    Let $R$ be any row reduced echelon matrix with  $\row{R} 
    =W$, and let $\rho_1, \dots, \rho_r$ be the nonzero 
    vectors of $R$; and suppose theat the leading nonzero 
    entry of $\rho_i$ occurs in column  $k_i$ with  $1 \leq i \leq r.$. By
    theorem \ref {2.7.2}, these vectors form a basis for $W$, and by the proof
    of the same theorem, we had  $\beta=\sum_{i=1}^{r}{b_{k_i}\rho_i}$. Thus any
    vector $\beta$ is determined by the coordinates  $b_{k_i}$. Now suppose that
    $\beta \neq 0 \in W$. By above, we get that  $R_{ij}=0$ for $i>s$ adn  $j
    \leq k_s$, thus  $\beta=(0, \dots, 0, b_{k_s}, b_n)$ where $b_{k_s} \neq 0$.
    Also note that there exists a vector in $W$ with nonzero  $k_s$-th
    coordinate; namely  $\rho_s$.

    Now that  $R$ is uniquely determined by  $W$, let us describe it in terms of
    $W$.  Consider  $\beta \in W$. If  $\beta \neq 0$, then  $\beta=(0, \dots,
    0,b_t, \dots, b_n)$ where $b_t \neq 0$. Let  $k_1, \k_r \in \Z^+$ be those
    $t$ for which there is some  $\beta \neq 0$. Take  $k_1< \dots<k_r$. For
    each $k_s$, there is one and only one  $\rho_s$ for which the  $k_s$-th
    coordinate is $1$ and every ther coordinate is $0$. Then  $R$ is the  $m
    \times n$ matrix whose row space is  $\row{R}=\{\rho_1, \dots, \rho_r,0,
    \dots, 0\}$.
\end{proof}
\begin{corollary}
    Each $m \times n$ matrix $A$ over $F$ is equivalent to precisely one row
    reduced echelon matrix.
\end{corollary}
\begin{corollary}
    Let $A$ and  $B$ be $m \times n$ matrices over  $F$. Then  $B$ is row
    equivalent to  $A$ if and only if  $\row{A}=\row{B}$. 
\end{corollary}

\begin{theorem}\label{2.7.4}
    Let $A$ and  $B$ be  $m \times n$ matrices over a field  $F$. The following
    are equivalent:
        \begin{enumerate}
            \item[(1)] $B$ is row equivalent to  $A$.		

            \item[(2)] $\row{B}=\row{A}$.
                
            \item[(3)] $B=PA$ where  $P$ is an  $m \times m$ invertible matrix.

            \item [(4)] The homogeneous systems $AX=0$ and  $BX=0$
        \end{enumerate}
\end{theorem}
